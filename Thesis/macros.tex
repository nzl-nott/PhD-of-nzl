
%%% symbols and macros, mostly from the HoTT book



\newcommand{\prd}[1]{\Pi_{#1}}
\newcommand{\sm}[1]{\Sigma_{#1}}

% \newcommand{\smmultiline}[1]{\Sigma \; \left #1 \right}

\newcommand{\lam}[1]{\lambda #1 .}

%own
\newcommand{\fa}[2]{\forall (#1 : #2) . \,}
% \newcommand{\fa}[1]{\forall (#1). \,}
\newcommand{\fasimple}[1]{\forall #1 . \,}
\newcommand{\fapairs}[3]{\fa {#1 \, #2} #3}
\newcommand{\fatypes}[1]{\fa #1 \type} % forall types

%%% Definitional equality (used infix) %%%
\newcommand{\jdeq}{\equiv}      % An equality judgment
\newcommand{\defeq}{\vcentcolon\equiv}  % A judgmental equality currently being defined

%%% Homotopies (written infix) %%%
\newcommand{\eqvsym}{\simeq}
\newcommand{\eqv}[2]{\ensuremath{#1 \eqvsym #2}\xspace}
\newcommand{\eqvauto}[1]{\eqv{#1}{#1}}


%%% own notation: equivalences
\newcommand{\iseqown}[1]{\ensuremath{\mathsf{e}_{#1}}\xspace}
\newcommand{\eqvsymspace}{\enspace \eqvsym \enspace}
\newcommand{\eqvspace}[2]{\ensuremath{#1 \eqvsymspace #2}\xspace}
\newcommand{\eqvautospace}[1]{\eqvspace{#1}{#1}}


%%% Identity functions %%%
\newcommand{\idfunc}[1][]{\ensuremath{\mathsf{id}_{#1}}\xspace}

%%% Identity types %%%
\newcommand{\idsym}{{=}}
\newcommand{\id}[3][]{\ensuremath{#2 =_{#1} #3}\xspace}
\newcommand{\idspace}[3][]{\ensuremath{#2 \, =_{#1} \, #3}\xspace}
\newcommand{\idtype}[3][]{\ensuremath{\mathsf{Id}_{#1}(#2,#3)}\xspace}
\newcommand{\idtypevar}[1]{\ensuremath{\mathsf{Id}_{#1}}\xspace}
\newcommand{\idauto}[2][]{\id[#1]{#2}{#2}}


%%% Reflexivity terms %%%
\newcommand{\refl}[1]{\ensuremath{\mathsf{refl}_{#1}}\xspace}

%%% Path concatenation (used infix, in diagrammatic order) %%%
\newcommand{\ct}{%
  \mathchoice{\mathbin{\raisebox{0.5ex}{$\displaystyle\centerdot$}}}%
             {\mathbin{\raisebox{0.5ex}{$\centerdot$}}}%
             {\mathbin{\raisebox{0.25ex}{$\scriptstyle\,\centerdot\,$}}}%
             {\mathbin{\raisebox{0.1ex}{$\scriptscriptstyle\,\centerdot\,$}}}
}

%%% Path reversal %%%
\newcommand{\opp}[1]{\mathord{{#1}^{-1}}}
\let\rev\opp
\let\sym\opp

%%% Transport (covariant) %%%
\newcommand{\trans}[2]{\ensuremath{{#1}_{*}\mathopen{}\left({#2}\right)\mathclose{}}\xspace}

\newcommand{\isequiv}{\ensuremath{\mathsf{isequiv}}}
\newcommand{\idequiv}{\ensuremath{\textsf{id-equiv}}}


%%% Universe types %%%
\newcommand{\UU}{\ensuremath{\mathcal{U}}\xspace}
\let\bbU\UU
\let\type\UU
% own
\newcommand{\UUp}{{\UU_\bullet}}
% \newcommand{\UUtt}[2]{\ensuremath{ \UU_{#1}^{\mathsmaller{\leq} #2}}}
% \newcommand{\UUt}[1]{\ensuremath{ \UU^{\mathsmaller{\leq} #1}}}
%%% Natural numbers
\newcommand{\N}{\ensuremath{\mathbb{N}}\xspace}
%\newcommand{\N}{\textbf{N}}
\let\nat\N
\newcommand{\suc}{\mathsf{succ}}
\newcommand{\add}{\mathsf{add}}
% \newcommand{\ack}{\mathsf{ack}}
\newcommand{\ass}{\mathsf{ass}}
\newcommand{\ite}{\mathsf{iter}}

%own
\newcommand{\Q}{\ensuremath{\mathbb{Q}}\xspace}
% \newcommand{\R}{\ensuremath{\mathbb{R}}\xspace}



\newcommand{\dbl}{\ensuremath{\mathsf{double}}}


%%% The empty type
\newcommand{\emptyt}{\ensuremath{\mathbf{0}}\xspace}

%%% The unit type
\newcommand{\unit}{\ensuremath{\mathbf{1}}\xspace}
\newcommand{\ttt}{\ensuremath{\star}\xspace}

%%% The two-element type
\newcommand{\bool}{\ensuremath{\mathbf{2}}\xspace}
\newcommand{\btrue}{{1_{\bool}}}
\newcommand{\bfalse}{{0_{\bool}}}
\let\two\bool
\let\trueb\btrue
\let\falseb\bfalse

\newcommand{\swap}{\ensuremath{\textsf{swap}}}
\newcommand{\swapequiv}{\ensuremath{\textsf{swap-equiv}}}


%%% Truncation levels
\newcommand{\iscontr}{\ensuremath{\mathsf{isContr}}}
\newcommand{\isprop}{\ensuremath{\mathsf{isProp}}}
\newcommand{\isset}{\ensuremath{\mathsf{isSet}}}


\def\compare#1#2#3#4{\if#1#3\if#2#41\else0\fi\else0\fi}

\newcommand{\istype}[1]{
  \edef\a{\compare-2#1\empty\empty}
  \if\a1 \iscontr \else
  \edef\b{\compare-1#1\empty\empty}
  \if\b1 \isprop \else
  \edef\c{#1}
  \if0\c \isset \else
  \mathsf{is}\mbox{-}{#1}\mbox{-}\mathsf{type} \fi\fi\fi
}



%%% Univalence
\newcommand{\ua}{\ensuremath{\mathsf{ua}}\xspace} % the inverse of idtoeqv
\newcommand{\idtoeqv}{\ensuremath{\mathsf{idtoeqv}}\xspace}
\newcommand{\univalence}{\ensuremath{\mathsf{univalence}}\xspace} % the full axiom


%%% Circles and spheres
\newcommand{\Sn}{\mathbb{S}}
\newcommand{\base}{\ensuremath{\mathsf{base}}\xspace}
\newcommand{\lloop}{\ensuremath{\mathsf{loop}}\xspace}
\newcommand{\surf}{\ensuremath{\mathsf{surf}}\xspace}

%%% Map on paths %%%
\newcommand{\mapfunc}[1]{\ensuremath{\mathsf{ap}_{#1}}\xspace} % Without argument
\newcommand{\map}[2]{\ensuremath{{#1}\mathopen{}\left({#2}\right)\mathclose{}}\xspace}
\let\Ap\map
%\newcommand{\Ap}[2]{\ensuremath{{#1}\left({#2}\right)}\xspace}
\newcommand{\mapdepfunc}[1]{\ensuremath{\mathsf{apd}_{#1}}\xspace} % Without argument
% \newcommand{\mapdep}[2]{\ensuremath{{#1}\llparenthesis{#2}\rrparenthesis}\xspace}
\newcommand{\mapdep}[2]{\ensuremath{\mapdepfunc{#1}\mathopen{}\left(#2\right)\mathclose{}}\xspace}
\let\apfunc\mapfunc
\let\ap\map
\let\apdfunc\mapdepfunc
\let\apd\mapdep

%own
\newcommand{\happly}{\ensuremath{\mathsf{happly}}}

%% OWN: UIP
\newcommand{\uipplain}{\mathsf{UIP}}
\newcommand{\uipext}[1]{\uipplain_{#1}}

\newcommand{\dec}{\mathsf{decidable}}
\newcommand{\deceq}{\mathsf{discrete}}


% % \newcommand{\trunc}[2]{\left\lVert #1 \right\rVert_{#2}} % note: \| is not symmetric, is it?
% % \newcommand{\truncmap}[2]{\left\lvert #1 \right\rvert_{#2}} % note: \| is not symmetric, is it?
% \newcommand{\trunc}[1]{\left\lVert #1 \right\rVert} % note: \| is not symmetric, is it?
% \newcommand{\truncmap}[1]{\left\lvert #1 \right\rvert} % note: \| is not symmetric, is it?

%%% Bracket/squash/truncation types %%%
\newcommand{\trunc}[2]{\mathopen{}\left\Vert #2\right\Vert_{#1}\mathclose{}}
\newcommand{\ttrunc}[2]{\bigl\Vert #2\bigr\Vert_{#1}}
\newcommand{\Trunc}[2]{\Bigl\Vert #2\Bigr\Vert_{#1}}
\newcommand{\truncf}[1]{\Vert \blank \Vert_{#1}}
\newcommand{\tproj}[3][]{\mathopen{}\left|#3\right|_{#2}^{#1}\mathclose{}}
\newcommand{\tprojf}[2][]{|\blank|_{#2}^{#1}}
\def\pizero{\trunc0}
%\newcommand{\brck}[1]{\trunc{-1}{#1}}
%\newcommand{\Brck}[1]{\Trunc{-1}{#1}}
%\newcommand{\bproj}[1]{\tproj{-1}{#1}}
%\newcommand{\bprojf}{\tprojf{-1}}



\newcommand{\bbbrck}[1]{\Bigl\Vert #1 \Bigr\Vert}
\newcommand{\brck}[1]{\trunc{}{#1}}
\newcommand{\bbrck}[1]{\ttrunc{}{#1}}
\let\bracket\bbrck
\newcommand{\Brck}[1]{\Trunc{}{#1}}
\let\Bracket\Brck
\newcommand{\bproj}[1]{\tproj{}{#1}}
\newcommand{\bprojf}{\tprojf{}}

% OWN
\newcommand{\populated}[1]{\langle \! \langle #1 \rangle \! \rangle}


%OWN
\newcommand{\fst}{\pi_1 \!}
\newcommand{\snd}{\pi_2 \!}
\newcommand{\inl}{\mathsf{inl}\xspace}
\newcommand{\inr}{\mathsf{inr}\xspace}
\newcommand{\inone}{\mathsf{in_1}\xspace}
\newcommand{\intwo}{\mathsf{in_2}\xspace}
\newcommand{\inthree}{\mathsf{in_3}\xspace}


%\newcommand{\hproptype}{\textbf{hProp}}


\newcommand{\LEM}[1]{\ensuremath{\mathsf{LEM}_{#1}}\xspace}
\newcommand{\choice}[1]{\ensuremath{\mathsf{AC}_{#1}}\xspace}
