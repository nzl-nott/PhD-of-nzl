\documentclass[11pt, mathserif,handout]{beamer}
\usepackage{etex}


\usepackage{agda}
\usepackage{ucs}
\usepackage[utf8x]{inputenc}

\usepackage{relsize}
\usepackage{autofe}
\usepackage{textgreek}

\setbeamertemplate{navigation symbols}{}


\usetheme{AnnArbor}
\setbeamercovered{highly dynamic} 
\usepackage{xspace}
\usepackage{ifthen}
\usepackage{amsmath}
\usepackage{amssymb}
\usepackage{stmaryrd}

\usepackage[english]{babel}
\usepackage{graphicx}
%\usepackage{proof}
%\usepackage{multimedia}
\usepackage{animate}
\usepackage[all]{xy}
\usepackage{booktabs}
\usepackage[nobibnewpage,notocbib]{apacite}
\usepackage[absolute,overlay,quiet]{textpos}

%\definecolor{blueish}{rgb}{0.2, 0.2, 0.7}

\usepackage[T1]{fontenc}
\usepackage{cmbright}
\usepackage{fancybox}

\usepackage[inference]{semantic}


\usepackage{mypack}

% Background Color

\definecolor{rice}{RGB}{250,245,193}
\definecolor{navy}{RGB}{193,235,250}
\definecolor{lpink}{RGB}{242,198,222}
\definecolor{lgreen}{RGB}{218,250,193}

\setbeamercolor{background canvas}{bg=white}

\setbeamercolor{yellow}{fg=black,bg=yellow}
\setbeamercolor{lightyellow}{fg=black,bg=yellow!40}
\setbeamercolor{orange}{fg=black,bg=orange}
\setbeamercolor{lightorange}{fg=black,bg=orange!40}
\setbeamercolor{green}{fg=black,bg=green}
\setbeamercolor{lightgreen}{fg=black,bg=green!40}
\setbeamercolor{blue}{fg=black,bg=blue}
\setbeamercolor{lightblue}{fg=black,bg=blue!40}
\setbeamercolor{red}{fg=black,bg=red}
\setbeamercolor{lightred}{fg=black,bg=red!40}
\setbeamercolor{lightpink}{fg=black,bg=pink!40}


\setbeamercolor{frametitle}{fg=black,bg=lgreen}
\setbeamercolor{title}{fg=black,bg=lgreen}

\setbeamercolor{boxtitle}{fg=black,bg=navy}
\setbeamercolor{boxbg}{fg=black,bg=rice}
\setbeamercolor{structure}{bg=black, fg=lgreen}



\setbeamercolor{palette primary}{bg=rice, fg=lgreen} % changed this
\setbeamercolor{palette secondary}{bg=navy, fg=lgreen} % changed this
\setbeamercolor{palette tertiary}{bg=navy, fg=lgreen} % changed this

\setbeamercovered{invisible}

\newenvironment{colorblock}
{\setbeamercolor{item}{fg=lpink,bg=lpink}\begin{beamerboxesrounded}[upper=boxtitle,lower=boxbg,shadow=true]}
{\end{beamerboxesrounded}}


\newenvironment{colorblock2}[2]
{\setbeamercolor{item}{fg=#1,bg=#1}\begin{beamerboxesrounded}[upper=#1,lower=#2,shadow=true]}
{\end{beamerboxesrounded}}


\author{
Nuo Li
}

\institute{
School of Computer Science \\  University of Nottingham, UK
}






\date[04/07/14]{FP Away day, July 2014}
\title{Definable quotients in \itt}
\subtitle[FP Away day 2014]{}


\addtobeamertemplate{sidebar right}{}{\vspace{-2cm}}
\addtobeamertemplate{footline}{}{\vspace{-1cm}\hskip2pt(\insertframenumber/\inserttotalframenumber) \insertshortsubtitle{} -- \insertshortdate\vskip2.2pt}
\setbeamercolor{footline}{fg=purple}


\begin{document}

% 1
\frame{\titlepage}


% 2


\begin{frame}
\frametitle{What is a quotient?}

\begin{itemize}
\item In arithmatics, 4 is the quotient of $8 \div 2$ (or $8/2$)
\pause
\item In set theory, given a set $A$ and an equivalence relation
  $\sim$ on it, a quotient set $\qset{A}$ is the collection of equivalence
  classes.

\pause
\item In type theory, a quotient type is a type $A$ whose equality
  is redefined by an equivalence relation $\sim$ on it.
\end{itemize}


\end{frame}

% 3

\begin{frame}{Quotients in type theory}
%\frametitle{Quotient types}

\begin{itemize}
\item \emph{Quotient type} is an extensional notion which is not available
  in \itt.

\pause

\item \emph{Setoids} are sometimes used to represent quotients.
A setoid consists of
\begin{enumerate}
\item a set $A$,
\item a relation $\sim : A \to A \to \Prop$, and
\item it is an equivalence, i.e. it is reflexive, symmetric and transitive.
\end{enumerate}

\pause

\item Some quotients are definable inductively without using quotient type former,
  e.g.\ the set of rational numbers $\Q$

\pause

\item Algebraic structures for these quotients in \itt:

\begin{itemize}
\item Prequotient
\item Quotient (dependent eliminator)
\item Quotent (Hofmann's version)
\item Exact quotient
\item Definable quotient 
\end{itemize}

\end{itemize}


\end{frame}


% 4
\begin{frame}
\frametitle{Prequotient}


\begin{colorblock}{\center{Prequotient}}
Given a setoid $(A,\sim)$,  a \emph{prequotient} $(Q,[\_],\sound)$ over that setoid consists of
\begin{enumerate}
\item \label{enum:Q} a set $Q$,
\item \label{enum:box}a function $[\_]: A \rightarrow Q$,
\item \label{enum:sound} a proof \emph{sound} that  the function $[\_]$ is compatible with the relation $\sim$,
that is \[\sound\colon (a,b : A) \rightarrow a\sim b \rightarrow [a] = [b],\]
\end{enumerate}

\end{colorblock}

\pause

In category theory, prequotient corresponds to a commute diagram of
morphisms -- \emph{(co)fork}

\[\xymatrix{
R\ar@<0.5ex>[r]^{\pi_0}\ar@<-0.5ex>[r]_{\pi_1}& A\ar[r]^{[\_]}
& Q
}\]

\end{frame}

\begin{frame}
\frametitle{Quotient (Hofmann's version)}


\begin{colorblock}{\center{Quotient (Hofmann's)}}
A prequotient $(Q,[\_],\sound)$ is a quotient (Hofmann's) if we also have

\[\lift : (f : A \to B) \to (\forall a,b \to a\sim b \to f~a
\equiv f~b) \to (Q \to B)\]
such that $\text{lift}-\beta : \lift ~f ~p \class a = f~a$,

together with an induction principle.
Suppose $B$ is a predicate, 
\[\qind : ((a: A)\to B ~ [ a ] \to ((q : Q) \to B~q)\]
\end{colorblock}



In category theory, a quotient is just a coequalizer (a special fork)

\begin{displaymath}
    \xymatrix{R \ar@<0.5ex>[r]^{\pi_1} \ar@<-0.5ex>[r]_{\pi_2} & A \ar[r]^{ [\_]}
      \ar[dr]_{f} & Q
      \ar@{.>}[d]^{\hat{f}} \\
& & Q' }
\end{displaymath}

\end{frame}


\begin{frame}


\frametitle{Quotient (dependent eliminator)}

\begin{colorblock}{\center{Quotient(dependent eliminator)}}
A prequotient $(Q,[\_],\sound)$ is a quotient if we also have
\begin{enumerate}
\setcounter{enumi}{3}
\item \label{enum:elim}
for any $B: Q\rightarrow\Set$, an eliminator
 \begin{align*}
 \qelim_B\,\,:\,\,\,&(f\colon (a:A) \rightarrow B\,\class a) \\
        {\rightarrow}\, &((p:a\sim b) \rightarrow f\,a \simeq_{\sound\,p}f\,b)\\
        {\rightarrow}\, &((q:Q) \rightarrow B\,q)
 \end{align*}
such that $\text{qelim}-\beta : \qelim_B\,f \,p\,\class a\equiv f\,a$.
\end{enumerate}
\end{colorblock}

They are equivalent.

\end{frame}





\begin{frame}
\frametitle{Exact quotient}

\begin{itemize}
\item A quotient is \emph{exact} (or effective) if

$$exact :(\forall a,b : A) \rightarrow  \class a \equiv \class b \rightarrow a \sim b$$

i.e.\ a coequalizer is exact if following diagram commutes (i.e.\ a pullback)

\[\xymatrix{
R\pullbackcorner\ar[r]^{\pi_1}\ar[d]_{\pi_2} & A\ar[d]^{[\_]} \\
A\ar[r]_{[\_]} & Q
}\]

\item Propositional extensionality  (univalence
  for propositions) $\implies$ every quotient is exact.
\end{itemize}

\end{frame}




\begin{frame}{Definable quotient}


\begin{colorblock}{\center{Definable quotient}}
Given a setoid $(A,\sim)$, a \emph{definable quotient} is a
prequotient $(Q, [\_], \sound)$ with 
\begin{align*}
\emb &: Q \rightarrow A\\
\complete &: (a : A) \rightarrow \emb {\class a} \sim a\\
\stable &: (q:Q) \rightarrow \class{\emb\,q} \equiv q.
\end{align*}
\end{colorblock}

It corresponds to a split coequalizer:


\[\xymatrix{
R\ar@<0.5ex>[r]^{\pi_0}\ar@<-0.5ex>[r]_{\pi_1}& A\ar[r]^{[\_]}
& Q
}\]

a fork with morphisms $\emb : Q \to A$ and $t : A \to R$ such that 
$[\_] \circ emb = 1_C$ , $\emb \circ [\_]  = \pi_2 \circ t$ and $\pi_1
\circ t = 1_B$. ($t ~ a = (\emb [ a ] , a) $)

\end{frame}

\begin{frame}

\frametitle{Relations of these structures}

\begin{itemize}

\item Definable quotients give rise to exact quotients.

\begin{enumerate}

\item Quotient (dependent version) $\iff$  Quotient (Hofmann's version)

\item Definable quotients $\to$ quotient

\item Definable quotient is exact

Proof:

$[a] = [b]$ \\
$\implies \emb ~[a] = emb~[b] $  (by congruence) \\
$\implies a \sim \emb ~[a] =
emb~[b] \sim b$  (by completeness) \\
$\implies a \sim b$

\end{enumerate}

\end{itemize}


\end{frame}

\begin{frame}[allowframebreaks,t]{Example : Integers $\Z$}

\begin{itemize}
\item setoid integer : 

$$\Z_0 :=\N \times \N$$

$$ (n_1, n_2) \sim (n_3 , n_4)  :=  n_1 + n_4 = n_3 + n_2$$

An example of the operation defined on setoid integers: addition

\begin{code}
\\
\>\AgdaFunction{\_+\_} \AgdaSymbol{:} \AgdaFunction{ℤ₀} \AgdaSymbol{→} \AgdaFunction{ℤ₀} \AgdaSymbol{→} \AgdaFunction{ℤ₀}\<%
\\
\>\AgdaSymbol{(}\AgdaBound{x+} \AgdaInductiveConstructor{,} \AgdaBound{x-}\AgdaSymbol{)} \AgdaFunction{+} \AgdaSymbol{(}\AgdaBound{y+} \AgdaInductiveConstructor{,} \AgdaBound{y-}\AgdaSymbol{)} \AgdaSymbol{=} \AgdaSymbol{(}\AgdaBound{x+} \AgdaFunction{ℕ+} \AgdaBound{y+}\AgdaSymbol{)} \AgdaInductiveConstructor{,} \AgdaSymbol{(}\AgdaBound{x-} \AgdaFunction{ℕ+} \AgdaBound{y-}\AgdaSymbol{)}\<%
\\
\end{code}

\item set integer:

\begin{code}
\\
\>\AgdaKeyword{data} \AgdaDatatype{ℤ} \AgdaSymbol{:} \AgdaPrimitiveType{Set} \AgdaKeyword{where}\<%
\\
\>[0]\AgdaIndent{2}{}\<[2]%
\>[2]\AgdaInductiveConstructor{+\_} \<[8]%
\>[8]\AgdaSymbol{:} \AgdaDatatype{ℕ} \AgdaSymbol{→} \AgdaDatatype{ℤ}\<%
\\
\>[0]\AgdaIndent{2}{}\<[2]%
\>[2]\AgdaInductiveConstructor{-suc\_} \AgdaSymbol{:} \AgdaDatatype{ℕ} \AgdaSymbol{→} \AgdaDatatype{ℤ}\<%
\\
\>\<\end{code}

\item Normalisaton function:

\begin{code}
\>\AgdaFunction{[\_]} \<[18]%
\>[18]\AgdaSymbol{:} \AgdaFunction{ℤ₀} \AgdaSymbol{→} \AgdaDatatype{ℤ}\<%
\\
\>\AgdaFunction{[} \AgdaBound{m} \AgdaInductiveConstructor{,} \AgdaInductiveConstructor{0} \AgdaFunction{]} \<[18]%
\>[18]\AgdaSymbol{=} \AgdaInductiveConstructor{+} \AgdaBound{m}\<%
\\
\>\AgdaFunction{[} \AgdaInductiveConstructor{0} \AgdaInductiveConstructor{,} \AgdaInductiveConstructor{suc} \AgdaBound{n} \AgdaFunction{]} \<[18]%
\>[18]\AgdaSymbol{=} \AgdaInductiveConstructor{-suc} \AgdaBound{n}\<%
\\
\>\AgdaFunction{[} \AgdaInductiveConstructor{suc} \AgdaBound{m} \AgdaInductiveConstructor{,} \AgdaInductiveConstructor{suc} \AgdaBound{n} \AgdaFunction{]} \AgdaSymbol{=} \AgdaFunction{[} \AgdaBound{m} \AgdaInductiveConstructor{,} \AgdaBound{n} \AgdaFunction{]}\<%
\end{code}

\item To find a canonical choice for each equivalence
  class, embedding function:

\begin{code}
\\
\>\AgdaFunction{⌜\_⌝} \<[11]%
\>[11]\AgdaSymbol{:} \AgdaDatatype{ℤ} \AgdaSymbol{→} \AgdaFunction{ℤ₀}\<%
\\
\>\AgdaFunction{⌜} \AgdaInductiveConstructor{+} \AgdaBound{n} \AgdaFunction{⌝} \<[11]%
\>[11]\AgdaSymbol{=} \AgdaBound{n} \AgdaInductiveConstructor{,} \AgdaNumber{0}\<%
\\
\>\AgdaFunction{⌜} \AgdaInductiveConstructor{-suc} \AgdaBound{n} \AgdaFunction{⌝} \AgdaSymbol{=} \AgdaNumber{0} \AgdaInductiveConstructor{,} \AgdaInductiveConstructor{suc} \AgdaBound{n}\<%
\\
\end{code}


\emph{sound} \emph{complete} \emph{stable} and \emph{exact}
can be verified.

\end{itemize}

\end{frame}





\begin{frame}[allowframebreaks,t]{Example : Rational numbers $\Q$}

\begin{itemize}

\item Setoid encoding : unreduced fractions $$\Q_0 = \Z \times \N$$

Note: $1/2$ represents $\frac{1}{3}$

Equivalence relation : $$ \frac{a}{b} = \frac{c}{d} := a \times d = c \times b $$


\item Set encoding: reduced fractions 

$$\Q = \Sigma ((n , d) : \Z \times \N)). \mathsf{coprime} ~n ~(d +1)$$

\item Normalisation is essentially just fraction reducing

\begin{code}
\\
\>\AgdaFunction{[\_]} \AgdaSymbol{:} \AgdaDatatype{ℚ₀} \AgdaSymbol{→} \AgdaRecord{ℚ}\<%
\\
\>\AgdaFunction{[} \AgdaSymbol{(}\AgdaInductiveConstructor{+} \AgdaInductiveConstructor{0}\AgdaSymbol{)} \AgdaInductiveConstructor{/suc} \AgdaBound{d} \AgdaFunction{]} \AgdaSymbol{=} \AgdaInductiveConstructor{ℤ.+\_} \AgdaNumber{0} \AgdaFunction{÷} \AgdaNumber{1}\<%
\\
\>\AgdaFunction{[} \AgdaSymbol{(}\AgdaInductiveConstructor{+} \AgdaSymbol{(}\AgdaInductiveConstructor{suc} \AgdaBound{n}\AgdaSymbol{))} \AgdaInductiveConstructor{/suc} \AgdaBound{d} \AgdaFunction{]} \AgdaKeyword{with} \AgdaFunction{gcd} \AgdaSymbol{(}\AgdaInductiveConstructor{suc} \AgdaBound{n}\AgdaSymbol{)} \AgdaSymbol{(}\AgdaInductiveConstructor{suc} \AgdaBound{d}\AgdaSymbol{)}\<%
\\
\>\AgdaFunction{[} \AgdaSymbol{(}\AgdaInductiveConstructor{+} \AgdaInductiveConstructor{suc} \AgdaBound{n}\AgdaSymbol{)} \AgdaInductiveConstructor{/suc} \AgdaBound{d} \AgdaFunction{]} \AgdaSymbol{|} \AgdaBound{di} \AgdaInductiveConstructor{,} \AgdaBound{g} \AgdaSymbol{=} \AgdaFunction{GCD′→ℚ} \AgdaSymbol{(}\AgdaInductiveConstructor{suc} \AgdaBound{n}\AgdaSymbol{)} \AgdaSymbol{(}\AgdaInductiveConstructor{suc} \AgdaBound{d}\AgdaSymbol{)} \<[55]%
\>[55]\<%
\\
\>[5]\AgdaIndent{30}{}\<[30]%
\>[30]\AgdaBound{di} \AgdaSymbol{(λ} \AgdaSymbol{())} \AgdaSymbol{(}\AgdaFunction{C.gcd-gcd′} \AgdaBound{g}\AgdaSymbol{)}\<%
\\
\>\AgdaFunction{[} \AgdaSymbol{(}\AgdaInductiveConstructor{-suc} \AgdaBound{n}\AgdaSymbol{)} \AgdaInductiveConstructor{/suc} \AgdaBound{d} \AgdaFunction{]} \AgdaKeyword{with} \AgdaFunction{gcd} \AgdaSymbol{(}\AgdaInductiveConstructor{suc} \AgdaBound{n}\AgdaSymbol{)} \AgdaSymbol{(}\AgdaInductiveConstructor{suc} \AgdaBound{d}\AgdaSymbol{)}\<%
\\
\>\AgdaSymbol{...} \AgdaSymbol{|} \AgdaBound{di} \AgdaInductiveConstructor{,} \AgdaBound{g} \AgdaSymbol{=} \AgdaFunction{-} \AgdaFunction{GCD′→ℚ} \AgdaSymbol{(}\AgdaInductiveConstructor{suc} \AgdaBound{n}\AgdaSymbol{)} \AgdaSymbol{(}\AgdaInductiveConstructor{suc} \AgdaBound{d}\AgdaSymbol{)} \AgdaBound{di} \AgdaSymbol{(λ} \AgdaSymbol{())} \<[50]%
\>[50]\<%
\\
\>[12]\AgdaIndent{13}{}\<[13]%
\>[13]\AgdaSymbol{(}\AgdaFunction{C.gcd-gcd′} \AgdaBound{g}\AgdaSymbol{)}\<%
\\
\end{code}

\item Embedding is trivial -- forgetting proof of coprimarity

\begin{code}
\\
\>\AgdaFunction{⌜\_⌝} \AgdaSymbol{:} \AgdaRecord{ℚ} \AgdaSymbol{→} \AgdaDatatype{ℚ₀}\<%
\\
\>\AgdaFunction{⌜} \AgdaBound{x} \AgdaFunction{⌝} \AgdaSymbol{=} \AgdaSymbol{(}\AgdaFunction{ℤcon} \AgdaSymbol{(}\AgdaFunction{ℚ.numerator} \AgdaBound{x}\AgdaSymbol{))} \AgdaInductiveConstructor{/suc} \AgdaSymbol{(}\AgdaFunction{ℚ.denominator-1} \AgdaBound{x}\AgdaSymbol{)}\<%
\\
\end{code}

\end{itemize}

\end{frame}


\begin{frame}[allowframebreaks,t]{The application of definable quotients}

\begin{itemize}
\item As long as we have defined operations or have proved theorems
  for the setoid representations, we can lift them by just mixing normalisation and embedding functions.  e.g.\

In the case of Integers:

\begin{code}
\\
\>\AgdaFunction{lift₁} \AgdaSymbol{:} \AgdaSymbol{(}\AgdaBound{op} \AgdaSymbol{:} \AgdaFunction{ℤ₀} \AgdaSymbol{→} \AgdaFunction{ℤ₀}\AgdaSymbol{)} \AgdaSymbol{→} \AgdaDatatype{ℤ} \AgdaSymbol{→} \AgdaDatatype{ℤ}\<%
\\
\>\AgdaFunction{lift₁} \AgdaBound{op} \AgdaSymbol{=} \AgdaFunction{[\_]} \AgdaFunction{∘} \AgdaBound{op} \AgdaFunction{∘} \AgdaFunction{⌜\_⌝}\<%
\\
\end{code}


\end{itemize}

Advantages of definable quotients:

\begin{itemize}

\item Setoid has simpler encoding, better manipulation

\item More auxiliary functions

\item Flexible in conversions of two representations

\item Benefit from both

\end{itemize}

\end{frame}



\begin{frame}[allowframebreaks,t]{Case : The distributivity of
    integers}

\begin{itemize}

\item For set encoding:

\begin{code}
\\
\>\AgdaFunction{distˡ} \<[7]%
\>[7]\AgdaSymbol{:} \AgdaFunction{\_ℤ*\_} \AgdaFunction{DistributesOverˡ} \AgdaFunction{\_ℤ+\_}\<%
\\
\>\AgdaFunction{distˡ} \AgdaBound{x} \AgdaBound{y} \AgdaBound{z} \AgdaKeyword{with} \AgdaFunction{sign} \AgdaBound{y} \AgdaFunction{S≟} \AgdaFunction{sign} \AgdaBound{z}\<%
\\
\>\AgdaFunction{distˡ} \AgdaBound{x} \AgdaBound{y} \AgdaBound{z} \<[15]%
\>[15]\AgdaSymbol{|} \AgdaInductiveConstructor{yes} \AgdaBound{p} \<[23]%
\>[23]\<%
\\
\>[4]\AgdaIndent{6}{}\<[6]%
\>[6]\AgdaKeyword{rewrite} \AgdaBound{p} \<[16]%
\>[16]\<%
\\
\>[6]\AgdaIndent{12}{}\<[12]%
\>[12]\AgdaSymbol{|} \AgdaFunction{lem1} \AgdaBound{y} \AgdaBound{z} \AgdaBound{p} \<[25]%
\>[25]\<%
\\
\>[6]\AgdaIndent{12}{}\<[12]%
\>[12]\AgdaSymbol{|} \AgdaFunction{lem2} \AgdaBound{y} \AgdaBound{z} \AgdaBound{p} \AgdaSymbol{=} \<[27]%
\>[27]\<%
\\
\>[0]\AgdaIndent{6}{}\<[6]%
\>[6]\AgdaFunction{trans} \AgdaSymbol{(}\AgdaFunction{cong} \AgdaSymbol{(λ} \AgdaBound{n} \AgdaSymbol{→} \AgdaFunction{sign} \AgdaBound{x} \AgdaFunction{S*} \AgdaFunction{sign} \AgdaBound{z} \AgdaFunction{◃} \AgdaBound{n}\AgdaSymbol{)} \<[47]%
\>[47]\<%
\\
\>[0]\AgdaIndent{6}{}\<[6]%
\>[6]\AgdaSymbol{(}\AgdaFunction{ℕdistˡ} \AgdaFunction{∣} \AgdaBound{x} \AgdaFunction{∣} \AgdaFunction{∣} \AgdaBound{y} \AgdaFunction{∣} \AgdaSymbol{(}\AgdaFunction{∣} \AgdaBound{z} \AgdaFunction{∣}\AgdaSymbol{)))} \<[36]%
\>[36]\<%
\\
\>[0]\AgdaIndent{6}{}\<[6]%
\>[6]\AgdaSymbol{(}\AgdaFunction{lem3} \AgdaSymbol{(}\AgdaFunction{∣} \AgdaBound{x} \AgdaFunction{∣} \AgdaPrimitive{ℕ*} \AgdaFunction{∣} \AgdaBound{y} \AgdaFunction{∣}\AgdaSymbol{)} \AgdaSymbol{(}\AgdaFunction{∣} \AgdaBound{x} \AgdaFunction{∣} \AgdaPrimitive{ℕ*} \AgdaFunction{∣} \AgdaBound{z} \AgdaFunction{∣}\AgdaSymbol{)} \AgdaSymbol{\_)}\<%
\\
\>\AgdaFunction{distˡ} \AgdaBound{x} \AgdaBound{y} \AgdaBound{z} \AgdaSymbol{|} \AgdaInductiveConstructor{no} \AgdaBound{¬p} \AgdaSymbol{=} \AgdaBound{...}\<%
\\
\end{code}

Three lemmas for first case, second case has to be proved from scratch.

\item For setoid encoding:

It can be done automatically


\begin{code}
\\
\>\AgdaFunction{distˡ} \AgdaSymbol{:} \<[9]%
\>[9]\AgdaFunction{\_*\_} \AgdaFunction{DistributesOverˡ} \AgdaFunction{\_+\_}\<%
\\
\>\AgdaFunction{distˡ} \AgdaSymbol{(}\AgdaBound{a} \AgdaInductiveConstructor{,} \AgdaBound{b}\AgdaSymbol{)} \AgdaSymbol{(}\AgdaBound{c} \AgdaInductiveConstructor{,} \AgdaBound{d}\AgdaSymbol{)} \AgdaSymbol{(}\AgdaBound{e} \AgdaInductiveConstructor{,} \AgdaBound{f}\AgdaSymbol{)} \AgdaSymbol{=} \AgdaFunction{solve} \AgdaNumber{6} \<[40]%
\>[40]\<%
\\
\>[0]\AgdaIndent{2}{}\<[2]%
\>[2]\AgdaSymbol{(λ} \AgdaBound{a} \AgdaBound{b} \AgdaBound{c} \AgdaBound{d} \AgdaBound{e} \AgdaBound{f} \AgdaSymbol{→} \AgdaBound{a} \AgdaFunction{:*} \AgdaSymbol{(}\AgdaBound{c} \AgdaFunction{:+} \AgdaBound{e}\AgdaSymbol{)} \AgdaFunction{:+} \AgdaBound{b} \AgdaFunction{:*} \AgdaSymbol{(}\AgdaBound{d} \AgdaFunction{:+} \AgdaBound{f}\AgdaSymbol{)} \AgdaFunction{:+}\<%
\\
\>[2]\AgdaIndent{6}{}\<[6]%
\>[6]\AgdaSymbol{(}\AgdaBound{a} \AgdaFunction{:*} \AgdaBound{d} \AgdaFunction{:+} \AgdaBound{b} \AgdaFunction{:*} \AgdaBound{c} \AgdaFunction{:+} \AgdaSymbol{(}\AgdaBound{a} \AgdaFunction{:*} \AgdaBound{f} \AgdaFunction{:+} \AgdaBound{b} \AgdaFunction{:*} \AgdaBound{e}\AgdaSymbol{))}\<%
\\
\>[2]\AgdaIndent{6}{}\<[6]%
\>[6]\AgdaFunction{:=}\<%
\\
\>[2]\AgdaIndent{6}{}\<[6]%
\>[6]\AgdaBound{a} \AgdaFunction{:*} \AgdaBound{c} \AgdaFunction{:+} \AgdaBound{b} \AgdaFunction{:*} \AgdaBound{d} \AgdaFunction{:+} \AgdaSymbol{(}\AgdaBound{a} \AgdaFunction{:*} \AgdaBound{e} \AgdaFunction{:+} \AgdaBound{b} \AgdaFunction{:*} \AgdaBound{f}\AgdaSymbol{)} \AgdaFunction{:+}\<%
\\
\>[2]\AgdaIndent{6}{}\<[6]%
\>[6]\AgdaSymbol{(}\AgdaBound{a} \AgdaFunction{:*} \AgdaSymbol{(}\AgdaBound{d} \AgdaFunction{:+} \AgdaBound{f}\AgdaSymbol{)} \AgdaFunction{:+} \AgdaBound{b} \AgdaFunction{:*} \AgdaSymbol{(}\AgdaBound{c} \AgdaFunction{:+} \AgdaBound{e}\AgdaSymbol{)))} \AgdaInductiveConstructor{refl} \AgdaBound{a} \AgdaBound{b} \AgdaBound{c} \AgdaBound{d} \AgdaBound{e} \AgdaBound{f}\<%
\\
\end{code}

\item Combined with "Reflection'' technique, it can be done completely
  automatic, like the \textbf{ring} tactic in Coq.

\item It can be generalised to any properties which can be converted
into equations of natural numbers.

\item In practice, we manually construct the proof for efficiency,
  which is still simpler.

\begin{code}
\\
\>\AgdaFunction{distˡ} \AgdaSymbol{:} \AgdaFunction{\_ℤ₀*\_} \AgdaFunction{DistributesOverˡ} \AgdaFunction{\_ℤ₀+\_}\<%
\\
\>\AgdaFunction{distˡ} \AgdaSymbol{(}\AgdaBound{a} \AgdaInductiveConstructor{,} \AgdaBound{b}\AgdaSymbol{)} \AgdaSymbol{(}\AgdaBound{c} \AgdaInductiveConstructor{,} \AgdaBound{d}\AgdaSymbol{)} \AgdaSymbol{(}\AgdaBound{e} \AgdaInductiveConstructor{,} \AgdaBound{f}\AgdaSymbol{)} \AgdaSymbol{=} \<[32]%
\>[32]\<%
\\
\>[0]\AgdaIndent{2}{}\<[2]%
\>[2]\AgdaFunction{cong₂} \AgdaPrimitive{\_ℕ+\_} \AgdaSymbol{(}\AgdaFunction{dist-lemˡ} \AgdaBound{a} \AgdaBound{b} \AgdaBound{c} \AgdaBound{d} \AgdaBound{e} \AgdaBound{f}\AgdaSymbol{)}\<%
\\
\>[0]\AgdaIndent{2}{}\<[2]%
\>[2]\AgdaSymbol{(}\AgdaFunction{sym} \AgdaSymbol{(}\AgdaFunction{dist-lemˡ} \AgdaBound{a} \AgdaBound{b} \AgdaBound{d} \AgdaBound{c} \AgdaBound{f} \AgdaBound{e}\AgdaSymbol{))}\<%
\\
%
\\
\>\AgdaFunction{dist-lemˡ} \AgdaSymbol{:} \AgdaSymbol{∀} \AgdaBound{a} \AgdaBound{b} \AgdaBound{c} \AgdaBound{d} \AgdaBound{e} \AgdaBound{f} \AgdaSymbol{→} \<[28]%
\>[28]\<%
\\
\>[0]\AgdaIndent{2}{}\<[2]%
\>[2]\AgdaBound{a} \AgdaPrimitive{ℕ*} \AgdaSymbol{(}\AgdaBound{c} \AgdaPrimitive{ℕ+} \AgdaBound{e}\AgdaSymbol{)} \AgdaPrimitive{ℕ+} \AgdaBound{b} \AgdaPrimitive{ℕ*} \AgdaSymbol{(}\AgdaBound{d} \AgdaPrimitive{ℕ+} \AgdaBound{f}\AgdaSymbol{)} \AgdaDatatype{≡}\<%
\\
\>[0]\AgdaIndent{2}{}\<[2]%
\>[2]\AgdaSymbol{(}\AgdaBound{a} \AgdaPrimitive{ℕ*} \AgdaBound{c} \AgdaPrimitive{ℕ+} \AgdaBound{b} \AgdaPrimitive{ℕ*} \AgdaBound{d}\AgdaSymbol{)} \AgdaPrimitive{ℕ+} \AgdaSymbol{(}\AgdaBound{a} \AgdaPrimitive{ℕ*} \AgdaBound{e} \AgdaPrimitive{ℕ+} \AgdaBound{b} \AgdaPrimitive{ℕ*} \AgdaBound{f}\AgdaSymbol{)} \<[43]%
\>[43]\<%
\\
\>\AgdaFunction{dist-lemˡ} \AgdaBound{a} \AgdaBound{b} \AgdaBound{c} \AgdaBound{d} \AgdaBound{e} \AgdaBound{f} \AgdaSymbol{=} \AgdaFunction{trans}\<%
\\
\>[0]\AgdaIndent{2}{}\<[2]%
\>[2]\AgdaSymbol{(}\AgdaFunction{cong₂} \AgdaPrimitive{\_ℕ+\_} \AgdaSymbol{(}\AgdaFunction{ℕdistˡ} \AgdaBound{a} \AgdaBound{c} \AgdaBound{e}\AgdaSymbol{)} \AgdaSymbol{(}\AgdaFunction{ℕdistˡ} \AgdaBound{b} \AgdaBound{d} \AgdaBound{f}\AgdaSymbol{))}\<%
\\
\>[0]\AgdaIndent{2}{}\<[2]%
\>[2]\AgdaSymbol{(}\AgdaPostulate{swap23} \AgdaSymbol{(}\AgdaBound{a} \AgdaPrimitive{ℕ*} \AgdaBound{c}\AgdaSymbol{)} \AgdaSymbol{(}\AgdaBound{a} \AgdaPrimitive{ℕ*} \AgdaBound{e}\AgdaSymbol{)} \AgdaSymbol{(}\AgdaBound{b} \AgdaPrimitive{ℕ*} \AgdaBound{d}\AgdaSymbol{)} \AgdaSymbol{(}\AgdaBound{b} \AgdaPrimitive{ℕ*} \AgdaBound{f}\AgdaSymbol{))}\<%
\\
\end{code}


\end{itemize}

\end{frame}

\begin{frame}
\frametitle{Summary}

\begin{itemize}

\item Quotients types are not available in \itt.

\item Setoids are not good enough.

\item Algebraic structures to encode quotient sets which are definable.

\item The advantages of using definable quotient structures.

\item \Large{Any Questions?}

\end{itemize}

\end{frame}


\end{document}