\chapter{Introduction}


\todo{what we are going to write}


\todo{overviews of each part write as much as you can}


mention the definable quotient is very useful 
compact overview.




\section{Overview}

\todo{short introductions to each section}


%compact, comprehensive Overview:


In Chapter 2, we will discuss the backgroud of this research. Type theory is a popular topic in
theoretical computer science. It is quite powerful not only a a theory
but also as a programming language. We use a dependent functional
programming language called Agda which is design based on \mltt. The
related work will also be discussed in this chapter.


In Chapter 3, we will discuss quotient types which is the topic of
this thesis in detail. Quotient types
can be understood as a interpretation of quotient set in set
theory. It is an extensional concept which is also related to other extensional concepts. It can be encoded in different ways. Categorically speaking it
is a coequalizer, and a split quotient is a just a split coequalizer.


In Chapter 4, we start introducing one of our achievements, the
definable quotients. It is usually very unreadable, unorganised and
complicated to write some programs without abstracting. It is also
applied to quotient types. If we have some types that can be abstract
as a quotient type of some common types, then it will be easily
encoded and manipulated. As a example, integers can be encoded as the
quotients types of paired natural numbers over the equivalence
relation that two pairs are equal if they represent the same
subtraction.


In Chapter 5, we will talk about the setoid model approach to encode
extensional concepts. The work is mainly extending the setoid model
done by Altenkirch in \cite{alti:lics99} to
quotient types.


In Chapter 6, we will discuss the new area between mathematics and
computer science -- \hott. We will talk about the higher inductive
types and also the \wog-model which is used to interpret
homotopy types in \itt. Quotient types can be encoded \hott simply.