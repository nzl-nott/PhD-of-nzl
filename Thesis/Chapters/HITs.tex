\chapter{Homotopy Type Theory and higher inductive types}\label{HITs}


Homotopy Type Theory is a new branch between theoretical computer science
and mathematics and it is a variant of \mltt{}. We have 
Vladimir Voevodsky's univalence axiom which identifies isomorphic structures.
I will not explain this topic in detail here, but a well-written text
book on Homotopy Type Theory which is written by many brilliant mathematicians and computer
scientists is available now \cite{hott-online}. 

With higher inductive types, it is possible to define the quotient
types in Homotopy Type Theory. However the implementation of
Homotopy Type Theory in \itt{} is still n open problem. We
work on defining semi-simplicial types and \wog{} to solve it. There
is also the very new model using cubical sets proposed by Bezem,
Coquand and Huber in \cite{bezem2013model}.

In the next chapter we will present an syntactic implementation of \wog{} following
Brunerie's approach. 