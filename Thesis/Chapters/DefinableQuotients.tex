\chapter{Definable Quotients}

\todo{Before 18th-Dec-2013}

Some types can be defined without quotient, however it does not
possess good properties from being quotient types. Examples like
integers, which can be defined as either negative or non-negative,


Sometimes, quotient types are more difficult to reason about than
their base types. 
We can achieve more convenience by manipulating base types and then lifting the operators and propositions according to the relation between quotient types and base types.
Therefore it is worthwhile for us to conduct a research project on the
implementation of quotients in \itt.


The work of this project will be divided into several phases. This
report introduces the basic notions in my project on implementing
quotients in type theory, such as type theory setoids, and quotient
types, reviews some work related to this topic and concludes with some
results of the first phase. 
The results done by Altenkirch, Anberr\'{e}e and I in \cite{aan} will be explained with a
few instances of quotients.


\subsection{Integers}


\paragraph{Operations}

\paragraph{Properties}


\paragraph{Comparison}

\todo{The advantage of use Quotient algebraic structure: the proving of distributivity}


\subsection{Rational numbers}

\todo{Complete the proving part}


\subsection{Real numbers and more}

\todo{axiomitised}

\todo{Why real numbers are not definable in \itt{}?}

\todo{Complete the proving part, talk about the definition in HoTT?}

\subsection{Multisets(bags)}

\begin{definition}

A multiset (or bag) is a set without the constraint that there is no repetitive elements.

\end{definition}

\todo{axiomitised?}
\todo{Complete the definition and some examples or using quotients?}