\chapter{Definable Quotients}

\todo{Before 18th-Dec-2013}

Some types can be defined without quotient, however it does not
possess good properties from being quotient types. Examples like
integers, which can be defined as either negative or non-negative,


Sometimes, quotient types are more difficult to reason about than
their base types. 
We can achieve more convenience by manipulating base types and then lifting the operators and propositions according to the relation between quotient types and base types.
Therefore it is worthwhile for us to conduct a research project on the
implementation of quotients in \itt.


The work of this project will be divided into several phases. This
report introduces the basic notions in my project on implementing
quotients in type theory, such as type theory setoids, and quotient
types, reviews some work related to this topic and concludes with some
results of the first phase. 
The results done by Altenkirch, Anberr\'{e}e and I in \cite{aan} will be explained with a
few instances of quotients.


\subsection{Integers}

Here I will explain these structures by using the example of integers
in Agda. All integers are the result of subtraction between two
natural numbers. Therefore we can use a pair of natural numbers in a subtraction
expression to represent the resulting integer.
For example, $1 - 4 = - 3$ says that the pair of natural numbers $(1,4)$
represents the integer $- 3$. Assuming we have the necessary definitions
of natural numbers, the base type of this quotient is:

$$\Z_0=\N \times \N$$

Mathematically we know that for any two pairs of natural numbers $(n_1, n_2)$ and $(n_3, n_4)$, 
$$ n_1 - n_2 = n_3 - n_4\iff n_1 + n_4 = n_3 + n_2$$

Because the results of subtraction are the same, we can infer that the
two pairs represent the same integer, so the equivalence relation
$\sim$ for $\Z_0$ could be defined as

% \begin{code}

% _∼_ : Rel ℤ₀
% (n1 , n2) ∼ (n3 , n4) = (n1 + n4) ≡ (n3 + n2)

% \end{code}

Here |≡| is propositional equality. Of course we must prove $\sim$ is an
equivalence relation then we can define the setoid $(\Z_0,\sim)$ in
Agda as\footnote{the proof $\sim isEquivalence$ is omitted here}

% \begin{code}

% ℤ-Setoid : Setoid
% ℤ-Setoid = record
%   { Carrier           = ℤ₀
%   ; _≈_                = _∼_
%   ; isEquivalence = _∼_isEquivalence
%   }

% \end{code}

In set theory, we can immediately derive the quotient set from this
setoid which is the set of integers $\Z$, but in current setting of \itt,
we need to define $\Z$ as follows

% \begin{code}

% data ℤ : Set where
%   +_    : (n : ℕ) → ℤ
%   -suc_ : (n : ℕ) → ℤ

% \end{code}

This is called normal form or canonical form of integers.

The next step is to prove that it is the quotient type of the setoid $(\Z_0,\sim)$.
To relate the setoid and the potential quotient type, we need to
provide a mapping function from the base type $\Z_0$ to the target
type $\Z$ which should be the normalisation function

% \begin{code}

% [_]                     : ℤ₀ → ℤ
% [ m , 0 ]             = + m
% [ 0 , suc n ]        = -suc n
% [ suc m , suc n ] = [ m , n ]

% \end{code}

The first property to prove is the \emph{sound} property,

% \begin{code}

% sound                      : ∀ {x y} → x ∼ y → [ x ] ≡ [ y ]

% \end{code}


\paragraph{Operations}

\paragraph{Properties}


\paragraph{Comparison}

\todo{The advantage of use Quotient algebraic structure: the proving of distributivity}


\subsection{Rational numbers}

\todo{Complete the proving part}


\subsection{Real numbers and more}

\todo{axiomitised}

\todo{Why real numbers are not definable in \itt{}?}

\todo{Complete the proving part, talk about the definition in HoTT?}

\subsection{Multisets(bags)}

\begin{definition}

A multiset (or bag) is a set without the constraint that there is no repetitive elements.

\end{definition}

\todo{axiomitised?}
\todo{Complete the definition and some examples or using quotients?}