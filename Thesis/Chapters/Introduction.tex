\chapter{Introduction}

\mltt (or just Type Theory) is a type theory which serves as a
foundation of constructive mathematics and is also a dependently typed
programming language. Different from other foundations like set
theory, it is not based on predicate logic but internalises the BHK
interpretation of intuitionistic logic through the Curry-Howard
isomorphism. It identifies propositions with types such that proofs of
a proposition become terms of the corresponding type. Viewed as a
programming language, this means that we can express a specification
as a type, and a program of that type will satisfy the
specification. Moreover, one can write programs and reason about them
in the same language resulting in certified programs. Implementations
of Type Theory include NuPRL, LEGO, Coq, Agda, Epigram, Pi-Sigma.


Viewed as a foundation for mathematics, Type Theory is a powerful tool
for constructively proving theorems with computerised verification. An
example is the formal proof of the four-colour theorem by Georges
Gonthier \cite{gonthier08ams} \footnote{More formalised mathematics
  can be found in \cite{sbfm}.}.

There are two versions of \mltt, the \emph{intensional} version (\itt
or ITT) and the \emph{extensional} version (\ett or ETT).  They differ
in the treatment of two notions of equality, \emph{propositional
  equality} and \emph{definitional equality}.  In ITT, if two
expressions can be computed to the same object then we make the
judgement that they are definitionally equal. On the other hand, we
have the \emph{identity type} or propositional equality which is a
type expressing the equality of two terms. Definitional equality
implies propositional equality, but not the other way around which is
usually called equality reflection. In ETT, they are identified, which
makes definitional equality and thereby type checking undecidable.


In \itt, propositional equality is intensional. Some extensional
equality types such as the equality of two point-wise equal functions,
the equality of two logically equivalent propositions, and equality of
two ``equivalence classes'' of a quotient $[a]$, $[b]$ where $a \sim
b$, are not inhabited. There are several extensional concepts (see
\Cref{extensionality}) which are useful and justifiable but not
available in ITT.  Nevertheless ITT is still preferable to ETT as the
basis for programming languages, because of its good computational
properties.  Therefore, we would like to extend ITT with these
extensional concepts, and the notion of quotient types is one of them.
%For example, $\lambda n \to n$ and $\lambda n \to n + 0$ are intensionally different \footnote{Considering the definition of $+$ where $n+0$ can not be reduced to $n$.} so they are not propositionally equal.
%Even if we encode the method to compute
%%the definitional equality of the outputs for each given input, it can
%only decide the equality for for finite inputs.
%Therefore the equality of two functions are undecidable in general.

%A type checker can easily decide whether two values of a inductive type are the intensionally equal, but not whether two functions are extensionally equal. 


%and is essential for construction of \maths and programs.


\section{Quotient types}


\emph{Quotient} is a primitive notion in mathematics. In arithmetic,
quotient refers to the result of a division:

$$8 \div 4 = 2 ~~ \text{or}~~ 8/4 = 2.$$

The notion is generalised in more abstract branches of mathematics,
such as set theory, group theory, topology etc. For example in set
theory, given a set $A$ and an equivalence relation $\sim$, the set of
all equivalence classes of $\sim$ is called the \emph{quotient set} of
$A$ by $\sim$.

An \textbf{equivalence relation} is a binary relation which is 

\begin{itemize}
\item reflexive: $\forall a \in A, a \sim a$,
\item symmetric: $\forall a~ b \in A, a \sim b \to b \sim a$
\item transitive: $\forall a ~ b~ c \in A, a \sim b \to b \sim c \to a \sim c$.
\end{itemize}

The \textbf{equivalence class} of an element $a$ is a subset of $A$
which contains all elements equivalent to $a$:

\begin{equation*}
[a] = \{x \in A \;| \; a \sim x \}
\end{equation*}

The \textbf{quotient set} of $A$ by $\sim$ is just the set of equivalence classes:

\begin{equation*}
\qset{A} = \{ [ a ] \;|\; a : A \}
\end{equation*}

Similarly, we can also ``divide'' a group, space, category or another
algebraic structure by a given structure-preserving equivalence
relation on it.

Naturally one would also expect \textbf{quotient types} in Type
Theory. Intuitively speaking, a \emph{quotient type} $\qset{A}$ is a
type $A$ whose equality is redefined by an equivalence relation on
it. In \ett, it is possible to redefine the equalities of types.  For
example, in NuPRL which is an implementation of \ett, there is a
quotient operator which builds a new type from a given type and an
equivalence relation on it \cite{DBLP:books/daglib/0068834}. However
it is not possible to recover the witness of the equality between two
equal elements in quotient types \cite{nog:02}.


Because of the good computational properties, we would like to have
quotient types in \itt as well.  However in the traditional
formulation of \itt, such a type former does not exist because there
is no attached equivalence on each type except definitional equality
which can not be changed. Instead \textbf{setoids} are usually used to
represent quotients:

\begin{definition}
\textbf{Setoid}.
\noindent A setoid $(A,\sim,\text{eqv}_{\sim})$ (usually written as just $(A,\sim)$) consists of
\begin{enumerate}
\item a set (type) $A : \Set$,
\item a binary relation $\sim : A \to A \to \Prop$, and
\item a proof that it is an equivalence, i.e.\ reflexive, symmetric and transitive.
\end{enumerate}
\end{definition}

Notice that this notion is also called a \emph{total setoid}. If the
relation of the setoid is not required to be reflexive it is called a
\emph{partial setoid}. In this thesis, the word "setoid" refers to a
total setoid.


A function $f : A \to B$ is well-defined on a setoid $(A,\sim)$ only if it respects $\sim$:

\begin{definition}\label{compatible}
We say a function $f : A \to B$ \textbf{respects} $\sim$ if

$$\forall (x, y : A) \to x \sim y \to f(x) =_{B} f(y)$$
\end{definition}

However using setoids to represent quotients is not an ideal
solution. Since it is an alternative representation of sets,
everything defined on $\Set$ has to be redefined on $\Setoid$ again.
Examples are functions between setoids, equalities on setoids,
products on setoids, etc.  In fact, in other branches of mathematics,
the quotient object is essentially the same kind of object as the base
one.  Therefore, it is better to have a representation of the quotient
$\qset{A}$ which is in the same sort as $A$ is.

In fact not all quotients have to be defined using a quotient type
former. For example integers are usually represented as pairs of
natural numbers $\N \times \N$ which are equivalent if subtracting
first number from the second gives the same result. This gives rise to
a quotient. However the set of integers can also be defined
inductively from the observation that $\Z \simeq \N + \N$.  For such
quotients, the set definition can be seen as a normal form of the
equivalence classes. There is a mapping from the setoid representation
to the set representation called the \textbf{normalisation
  function}. In this thesis we say that such quotients are
\emph{definable via a normalisation (function)} (see \Cref{dq}).


Some quotients are not definable via normalisation, for example the
set of real numbers represented by Cauchy sequences of rational
numbers, the finite multisets represented as lists quotiented by
permutation equivalence (or bag equivalence
\cite{DBLP:conf/itp/Danielsson12}), the non-terminating programs
represented by the partiality monad quotiented by weak bisimilarity
and so on. In these cases, a general schema to define quotient types
is essential.

If we simply introduce quotient types as axioms in \itt, we lose the
\emph{canonicity} property, in other words, we can construct
non-canonical terms of $\N$ which can not be reduced to numerals (see
\Cref{quotientcanonicity}). In fact, similar issues arise when adding
other extensional concepts as axioms e.g.\ functional
extensionality. Therefore it is essential to find a computational
interpretation of these extensional concepts including quotient types.

To achieve these goals, we have to "refine" our interpretation of
types. Usually a type is treated as a set without attached
equality. If a type is interpreted as a \emph{setoid}, in other words
internalising propositional equality, quotient types can be defined
simply by replacing ``internal'' equality. This is called \emph{setoid
  interpretation} which is inspired by Bishop's \cite{bishop}
definition of sets and has been studied by Martin Hofmann
\cite{hof:phd,hof:95:sm} and Thorsten Altenkirch
\cite{alti:lics99,alti:ott-conf}. Based on this interpretation, we can
build a setoid model in \itt which gives us the computational
interpretation of quotient types.

For a long time, the nature of identity types was mysterious in
\itt. Intuitively, the uniqueness of identity proofs (UIP), stating
that two terms of the same identity type are always propositionally
equal, is valid because there is at most one canonical element
expressing the equality between two objects. However UIP is not
derivable from the eliminator for identity type $\J$ (see
\Cref{typerule}) but needs an extra eliminator $\K$ suggested by
Thomas Streicher \cite{streicherinvestigations}.  Furthermore, Hofmann
and Streicher \cite{MR1686862} propose a groupoid interpretation of
\itt where $\K$ is refuted, hence UIP fails. The groupoid
interpretation can be seen as a generalisation of the setoid one,
where the identity type is not a proposition but a set. It means that
there can be several proofs of the same identity which are not equal.

In fact, the groupoid interpretation of types can be extended to
$\omega$-groupoids which are generalisations of groupoids. Roughly
speaking, an $\omega$-groupoid consists of objects, morphisms between
objects, morphisms between morphisms and so on, having infinite levels
of morphisms. All of these morphisms are isomorphisms which hold up to
higher isomorphisms. These isomorphisms are called equivalences. An
introduction to $\omega$-groupoids is given in \Cref{wogintro}.  Since
Grothendieck's homotopy hypothesis states that $\omega$-groupoids are
spaces \cite{homotopyhyp}, we can interpret types as spaces indeed. In
recent years, such an interpretation has been developed into a new
field called \hott. In \hott, types are interpreted as spaces
(abstractly) or as \emph{weak} $\omega$-groupoids. However, it is very
difficult to describe all levels of coherence conditions of
\emph{weak} $\omega$-groupoid such as groupoid laws. A more commonly
used approach is therefore to define them in terms of Kan simplicial
sets or cubical sets (See \Cref{cihott}).  Nevertheless, it is
possible to build a syntactic type theory to describe \wog in \itt
(see \Cref{wog}).

%Usually the coherence conditions of \wog are too involved to state as a globular set, it is more popular to describe them using some simpler incarnations  like simplicial sets or cubical sets. However it is still possible to build a syntactic type theory to describe \wog in \itt.

In \hott, the most important axiom is \emph{univalence} which was
suggested by Voevodsky \cite{VV}. Roughly speaking, univalence states
that identity of types corresponds to equivalence. Many extensional
concepts are derivable from this axiom, including functional
extensionality, propositional extensionality, quotient types. For
example, Voevodsky has proposed an impredicative encoding of quotient
types (see \Cref{impredicative}). The computational interpretation of
univalence remains an open problem, but it is likely to be solved by a
recently proposed model called \emph{cubical sets model} (Bezem,
Coquand and Huber \cite{bezem2013model}).


Quotient types can be applied in the formalisation of mathematics and
in program verification. As we mentioned before, one of the
fundamental mathematical notions, \emph{real numbers} can be defined
as a quotient where the base set is the set of Cauchy sequences of
rational numbers. From a programming perspective, they provide more
algebraic datatypes and enables us to reason about infinite types and
semantics-based verification of concurrent programs as suggested by
Hofmann \cite{hof:phd}.


\section{Structure of the thesis}



In \Cref{bg}, we introduce \mltt as the basis of our study. We briefly
describe its history and present its basic rules.  We also introduce
our main technical tool -- Agda, a dependently typed functional
programming language based on the intensional version of \mltt. Then
we discuss the missing extensional concepts in \itt excluding quotient
types. We also describe \hott which is an extension of \mltt by the
univalence axiom and higher inductive types which allow constructors
for internal equalities. We discuss how this theory gives rise to
extensional concepts.


In \Cref{qt}, we provide the syntactic rules of quotient types
together with a discussion of effectiveness. Categorically speaking, a
quotient type is a \emph{coequalizer}. We also explain the rules of
quotient types given by an adjunction. In \hott, our quotient types
become quotient \emph{sets}. We first introduce Voevodsky's
impredicative encoding of quotient sets together with proofs that all
essential rules are derivable. We also introduce quotient inductive
types (QITs) i.e.\ quotient sets defined using higher inductive types.
 

In \Cref{dq}, we introduce one of our original developments, the
definable quotient structure. We observe that there are some quotients
which are definable inductively in \mltt without adding a new quotient
type formation rule. A definable quotient consists of a setoid
representation $(A, \sim)$, a set representation $Q$ and a
normalisation function $[\_] : A \to Q$ which gives the normal form
for each "equivalence class".  As an example, integers can be encoded
as the quotient types of paired natural numbers over the equivalence
relation that two pairs are equal if they share the same result of
subtraction. Integers can also be defined inductively as a set. The
definable quotients structure is an abstraction of the relation
between the two representations and provides a flexible way of
conversing between them. In fact, it can be seen as a manual
construction of quotient types.


In \Cref{rl}, we discuss quotients that are not definable as an
inductive type with a normalisation function, such as the real
numbers, finite multisets and the partiality monad. We present a proof
of the undefinability of real numbers as Cauchy sequences
($\qset{R_0}$) with a normalisation function. The proof was mainly
conducted by Nicolai Kraus. The proof is based on Brouwer's continuity
principle -- all definable functions are continuous, which is
inconsistent if we have it within \mltt as shown by Escardo and Xu
\cite{incon} but holds meta-theoretically. We prove that $\qset{R_0}$
is \emph{connected}, and it implies that all functions $R_0 \to
\qset{R_0}$ that respect the equivalence relation of Cauchy sequences
are constant. Therefore there is no definable normalisation
endofunction for Cauchy sequences. Similarly we also prove that
non-terminating programs encoded using the partiality monad quotiented
by weak bisimilarity, are also undefinable with a normalisation
function. For unordered tuples such as unordered pairs and finite
multisets represented by lists quotiented by permutation, it is also
impossible for to find a canonical normalisation function unless the
underlying set has a decidable total order.


In \Cref{models}, we discuss several models of Type Theory where
quotient types are available. We present an implementation of the
setoid model encoding extensional concepts. The work is an extension
of the setoid model by Altenkirch \cite{alti:lics99} with quotient
types. Some other models including models of \hott are also discussed.


In \Cref{wog}, we present a new formalisation of the syntax of weak
$\omega$-groupoids in Agda using heterogeneous equality. We show how
to recover basic constructions on $\omega$-groupoids using suspension
and replacement. In particular we show that any type forms a groupoid
and we outline how to derive higher dimensional composition. We
present a possible semantics using globular sets and discuss the
issues which arise when using globular types instead. The work in the
chapter has been published in \cite{LFMTP14} together with Thorsten
Altenkirch and Ond\v{r}ej Ryp\'{a}\v{c}ek.


In the Appendices, we show our Agda code corresponding to the work in
\Cref{dq}, \Cref{models} and \Cref{wog}.
