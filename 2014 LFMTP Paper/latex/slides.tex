\documentclass[12pt, mathserif,handout]{beamer}

%\documentclass[12pt, mathserif]{beamer}

\usepackage{etex}


\usepackage{agda}

\usepackage{mypack}

\usepackage{ucs}
\usepackage[utf8x]{inputenc}

\usepackage{relsize}
\usepackage{autofe}
\usepackage{textgreek}

\setbeamertemplate{navigation symbols}{}


\usetheme{Madrid}
\setbeamercovered{highly dynamic} 
\usepackage{xspace}
\usepackage{ifthen}
\usepackage{amsmath}
\usepackage{amssymb}
\usepackage{stmaryrd}
\usepackage{proof}

\usepackage[english]{babel}
\usepackage{graphicx}
\usepackage{multimedia}
\usepackage{animate}
\usepackage[all]{xy}
\usepackage{booktabs}
\usepackage[nobibnewpage,notocbib]{apacite}
\usepackage[absolute,overlay,quiet]{textpos}

\definecolor{blueish}{rgb}{0.2, 0.2, 0.7}

\usepackage[T1]{fontenc}
\usepackage{cmbright}
\usepackage{fancybox}

\usepackage[inference]{semantic}

% Background Color

\definecolor{rice}{RGB}{245,245,220}
\setbeamercolor{background canvas}{bg=rice}

\setbeamercolor{yellow}{fg=black,bg=yellow}
\setbeamercolor{lightyellow}{fg=black,bg=yellow!40}
\setbeamercolor{orange}{fg=black,bg=orange}
\setbeamercolor{lightorange}{fg=black,bg=orange!40}
\setbeamercolor{green}{fg=black,bg=green}
\setbeamercolor{lightgreen}{fg=black,bg=green!40}
\setbeamercolor{blue}{fg=black,bg=blue}
\setbeamercolor{lightblue}{fg=black,bg=blue!40}
\setbeamercolor{red}{fg=black,bg=red}
\setbeamercolor{lightred}{fg=black,bg=red!40}
\setbeamercolor{lightpink}{fg=black,bg=pink!40}


%\setbeamercolor{frametitle}{fg=white,bg=orange}
%\setbeamercolor{title}{fg=white,bg=orange}

\setbeamercovered{invisible}

%\begin{colorblock}{orange}{lightyellow}{\center{Heterogeneous equality for Tm}}


\newenvironment{colorblock}[2]
{\setbeamercolor{item}{fg=#1,bg=#1}\begin{beamerboxesrounded}[upper=#1,lower=#2,shadow=true]}
{\end{beamerboxesrounded}}



% shorthands


\author{Thorsten Altenkirch \& Nuo Li \& Ond\v{r}ej Ryp\'a\v{c}ek}

\author[Thorsten Altenkirch \& Nuo Li \& Ond\v{r}ej
Ryp\'a\v{c}ek]{Thorsten Altenkirch \inst{1} \and Nuo Li \inst{1} \and
  Ond\v{r}ej Ryp\'a\v{c}ek \inst{2}}
\institute[shortinst]{\inst{1} 
School of Computer Science \\  University of Nottingham, UK \and %
                      \inst{2} University of Oxford, UK}






\date[17/07/14]{17/07/14}
\title{Some constructions on $\omega$-groupoids}
\subtitle[LFMTP 2014]{}


\addtobeamertemplate{sidebar right}{}{\vspace{-2cm}}
\addtobeamertemplate{footline}{}{\vspace{-1cm}\hskip2pt(\insertframenumber/\inserttotalframenumber) \insertshortsubtitle{} -- \insertshortdate\vskip2.2pt}
\setbeamercolor{footline}{fg=purple}


\begin{document}

% 1
\frame{\titlepage}


% 2

\begin{frame}{Outline}
\begin{itemize}
\item Introduction to \wog
\item Basic syntax of \tig (describing the structure of \wog)
\item Heterogeneous equality for syntactic terms
\item Coherences constructions
\item Semantic interpretation
\end{itemize}
\end{frame}




\begin{frame}[allowframebreaks,c]{Introduction to \wog}

\textbf{What are \wog?}


\begin{columns}[onlytextwidth]
 
    \begin{column}{0.7\textwidth}
\begin{itemize}
\item A higher dimensional category ($\omega$-category)

\item Infinite levels of morphisms 

\item Generalization of setoids, groupoids
\item Every morphism is an equivalence (generalization of isomorphism)
\item Equalities are weak e.g. $(f \circ g) \circ h \to f \circ (g \circ h)$

\end{itemize}


    \end{column}
   \begin{column}{0.5\textwidth}
      \xymatrix{ 
 a
 \ar@/_4pc/[ddd]^p="p" 
 \\
 \\
 \\
 b
 \ar@/_4pc/[uuu]^{p^{-1}}="p^{-1}" 
 \only<2->{\ar@2{->} @/^2pc/ "p";"p^{-1}" _{H}="H"}
 \only<3->{\ar@2{<-} @/_2pc/ "p";"p^{-1}" ^{H ^{-1}}="H^{-1}"}
 \only<4->{\ar@3{->} @/^1pc/ "H";"H^{-1}"_{...}="A"}
 \only<5->{\ar@3{<-}@/_1pc/  "H";"H^{-1}"^{...}="A"}
}
    \end{column}
​  \end{columns}

\framebreak


\textbf{Why are we interested in \wog?}
\begin{itemize}
\item Interpretation of types in \hott
\item Isomorphic types are equal
\item Abstract datatype, abstract reansoning
\item Extensional concepts
\item Weak $\omega$-groupoid model
\end{itemize}

\framebreak

\textbf{Formalizations of \wog in type theory}


\begin{itemize}
\item Warren's \emph{strict} $\omega$-groupoid model
\item Altenkirch and Rypacek's syntactic approach
\item Brunerie's syntactic approach: \tig (TIG)

\item This paper:
\begin{itemize}
\item implement \tig in Agda
\item develop constructions
\end{itemize}
\end{itemize}

\textbf{Agda}
\begin{itemize}
\item Dependently typed programming languages, theorem prover
\item An implementation of intensional \mltt
\end{itemize}

% \item Altenkirch and Rypacek's syntactic approach
% \item Brunerie's syntactic approach to describe the internal structure
%   of \wog using a type theory called \tig

% \item In 1983, Grothendieck's definition of \wog
% \item Maltsiniotis and Ara's simplified versions

% \item Related work: ($\omega$-)groupoid model of type theory
% \begin{itemize}
% \item Hofmann and Streicher's groupoid model
% \item Warren's \emph{strict} $\omega$-groupoid model
% \item Altenkirch and Rypacek's syntactic approach
% \item Brunerie's syntactic approach to describe the internal structure
%   of \wog using a type theory called \tig
%\end{itemize}



\end{frame}


\begin{frame}[c]{Basic syntax of \tig}


\begin{itemize}

\item Fundamental elements (in Agda code)

\begin{code}\>\<
\\
\>\AgdaKeyword{data} \AgdaDatatype{Con} \<[19]%
\>[19]\AgdaSymbol{:} \AgdaPrimitiveType{Set}\<%
\\
\>\AgdaKeyword{data} \AgdaDatatype{Ty} \AgdaSymbol{(}\AgdaBound{Γ} \AgdaSymbol{:} \AgdaDatatype{Con}\AgdaSymbol{)} \<[19]%
\>[19]\AgdaSymbol{:} \AgdaPrimitiveType{Set}\<%
\\
\>\AgdaKeyword{data} \AgdaDatatype{Tm} \<[19]%
\>[19]\AgdaSymbol{:} \AgdaSymbol{\{}\AgdaBound{Γ} \AgdaSymbol{:} \AgdaDatatype{Con}\AgdaSymbol{\}(}\AgdaBound{A} \AgdaSymbol{:} \AgdaDatatype{Ty} \AgdaBound{Γ}\AgdaSymbol{)} \AgdaSymbol{→} \AgdaPrimitiveType{Set}\<%
\>\<\end{code}


\item Types: basic objects, equality of objects, equality of equality...

\begin{columns}[onlytextwidth]
 
    \begin{column}{0.4\textwidth}


\begin{equation*}
\infer{\Gamma \vdash * ~ \text{type}}
{} 
\end{equation*}

    \end{column}
   \begin{column}{0.4\textwidth}

\begin{equation*}
\infer{\Gamma \vdash a =_{A} b ~\text{type}}
{\Gamma \vdash a, b : A}
\end{equation*}

    \end{column}
​  \end{columns}



\end{itemize}
\end{frame}


\begin{frame}[allowframebreaks,c]{Operations and equalities}

\begin{itemize}

\item Operations and equality in

\begin{itemize}

\item \textbf{Setoid}: 
\begin{equation*}
\begin{array}[b]{l} 
\text{id} : x = x \\
\_^{-1} : x= y \to y  = x \\
\_ \circ\_ : y = z \to x = y \to x = z
 \end{array}
\end{equation*}

\item \textbf{Groupoid}: 

\begin{equation*}
\begin{array}[b]{l}
\lambda : id \circ p = p \\
\rho : p \circ id = p \\
\alpha : p \circ (q \circ r) = (p \circ q) \circ r \\
\kappa : p^{-1}\circ p = id \\
\kappa' : p \circ p^{-1} = id
 \end{array}
\end{equation*}

\item \textbf{\wogs} : we have much more operations e.g. vertical/horizontal composition, and provable equalities on higher
  dimensions e.g. interchange law, coherence laws

%\xy 
%(-11.5,0)*+{\bullet}="1"; 
%(0,0)*+{\bullet}="2"; 
%{\ar@/^1.25pc/ "1";"2"}; 
%{\ar@/_1.25pc/ "1";"2"}; 
%{\ar "1";"2"}; 
%{\ar@{=>} (-5.75,-1)*{};(-5.75,-4)*{}} ; 
%{\ar@{=>} (-5.75,4)*{};(-5.75,1)*{}} ; 
%(0,0)*+{\bullet}="1"; 
%(11.5,0)*+{\bullet}="2"; 
%{\ar "1";"2"}; 
%{\ar@/^1.25pc/ "1";"2"}; 
%{\ar@/_1.25pc/ "1";"2"}; 
%{\ar@{=>} (5.75,-1)*{};(5.75,-4)*{}} ; 
%{\ar@{=>} (5.75,4)*{};(5.75,1)*{}} ; 
%\endxy

Example: There are
   two ways to show $(f \circ id) \circ g = f \circ g$

\hspace{1cm}
\xymatrix{
(f \circ id) \circ g \ar[rr]^{\alpha} \ar[drr]_{\rho \cdot id}="0"  && f \circ
(id \circ g)  \ar[d]^{id \cdot \lambda} \\
&& f \circ g
}
%\ar@{=>}"0"; {}

\end{itemize}

\item In general we call them \textbf{coherence constants} (or \textbf{coherences})

\item Infinitely many coherence constants, How can we encode them?
\end{itemize}

\end{frame}




\begin{frame}[allowframebreaks,t,c]{Contractible Contexts and Coherences}

\begin{itemize}

\item Fact: All coherences arising automatically from induction principle for identity type (or J eliminator)

\item \textbf{contractible contexts}
\begin{itemize}
\item $\epsilon, * $
\item $\epsilon,x : *, y : *, \alpha : x = y$
\item ... $\Gamma, y: A , \alpha : x = y$ (Given $\Gamma \vdash A$ and
  $\Gamma \vdash x : A$)

\end{itemize}


\item Assume $\epsilon,x : * \vdash x = x$ (weakening)
$\Rightarrow \epsilon,x : *, x : *, \alpha : x = x \vdash x =
x$ (J-eliminator)
$\Rightarrow \epsilon,x : *, y : *, \alpha : x = y \vdash y = x$

%\item All constructions in contractible contexts are coherences

\item How about coherences in non-cont contexts?

\item In general

\begin{equation*}
\infer{\Gamma \vdash \text{coh$_B$} : B [\delta]  }
{\vdash \Delta~ \text{contractible} & \Delta \vdash B & \delta : \Gamma \to \Delta} 
\end{equation*}

\end{itemize}

\end{frame}


\begin{frame}

\frametitle{Reasoning about syntactic terms}
\begin{itemize}
\item Using homogeneous equality to reason about syntactic terms, we
  have to eliminate \textbf{subst} in equalities like $\text{subst}~ p~ x = y$, $\text{subst} ~p
  (\text{subst} ~p^{-1}~ x) = x$
\item \textbf{Heterogeneous equality} (JM equality) for Tm

\begin{code}\>\<%
\\
\>\AgdaKeyword{data} \AgdaDatatype{\_≅\_} \AgdaSymbol{\{}\AgdaBound{Γ} \AgdaSymbol{:} \AgdaDatatype{Con}\AgdaSymbol{\}\{}\AgdaBound{A} \AgdaSymbol{:} \AgdaDatatype{Ty} \AgdaBound{Γ}\AgdaSymbol{\}} \<[29]%
\>[29]\<%
\\
\>[2]\AgdaIndent{9}{}\<[9]%
\>[9]\AgdaSymbol{:} \AgdaSymbol{\{}\AgdaBound{B} \AgdaSymbol{:} \AgdaDatatype{Ty} \AgdaBound{Γ}\AgdaSymbol{\}} \AgdaSymbol{→} \AgdaDatatype{Tm} \AgdaBound{A} \AgdaSymbol{→} \AgdaDatatype{Tm} \AgdaBound{B} \AgdaSymbol{→} \AgdaPrimitiveType{Set} \AgdaKeyword{where}\<%
\\
\>[0]\AgdaIndent{2}{}\<[2]%
\>[2]\AgdaInductiveConstructor{refl} \AgdaSymbol{:} \AgdaSymbol{(}\AgdaBound{b} \AgdaSymbol{:} \AgdaDatatype{Tm} \AgdaBound{A}\AgdaSymbol{)} \AgdaSymbol{→} \AgdaBound{b} \AgdaDatatype{≅} \AgdaBound{b}\<%
\\
\>\<\end{code}

\item Justification: The equality of inductively defined types are
  decidable, From Hedberg's Theorem, it is safe to assert $\AgdaDatatype{Ty} ~\AgdaBound{Γ}$ are sets (in the sense of UIP)
\end{itemize}
\end{frame}


\begin{frame}[allowframebreaks,c]{Construction of Coherences}

\begin{itemize}

\item Now we can construct all these infinite number of coherences

\item For each coherence, two versions:
\begin{enumerate}
\item minimum version e.g.\
  $\epsilon,x : *, y : *, \alpha : x = y, z: A,
  \beta : y = z \vdash x = z$ ($\_\circ^{*}\_$)
\item general version e.g.\
  $\Gamma,x : A, y : A, \alpha : x = y, z: A,
  \beta : y = z \vdash x = z$ ($\_\circ\_$)
\end{enumerate}

\item A minimum version is always in a contractible context and can be obtained by identity substitution 

\item General version is more complicated


\item \textbf{Replacement}: to obtain the general version from minimum

\begin{equation*}
\infer{\Gamma, \Delta^{A} \vdash \text{coh}_B^{A} : B^{A}}
{\Gamma \vdash A & \vdash \Delta~ \text{contractible} & \Delta \vdash \text{coh}_B^{*} : B} 
\end{equation*}


\item There is no $\Gamma, \Delta^{A} \Rightarrow \Delta$

\item Solution: Filter out variables in $\Gamma, \Delta^{A}$ which are unnessary to build $A$
\item Think reversely: build a "filtered" context using

\item \textbf{Suspension}: Assume $A$ is of level $n$. suspend $\Delta$ $n$ times, i.e. add a \emph{stalk} in front of $\Delta$

\begin{equation*}
\infer{\Sigma_A ~ \Delta \vdash \Sigma_A ~\text{coh}_B : \Sigma_A ~ B}
{\Gamma \vdash A &  \vdash \Delta ~\text{contractible} & \Delta \vdash B} 
\end{equation*}

\framebreak


\item and naturally we have a
substitution called \textbf{filter}

$$\text{filter}_A : \Gamma, \Delta^{A} \Rightarrow \Sigma_A ~ \Delta$$



\item Case: Assume $\Delta = (x : *)$ , $B = (x = x)$
and in $\Gamma = (a : *, b : *, c : *)$, $A = (a = b)$ (level $1$)

$\Sigma_{A} ~ \Delta = (x_0 : *, x_1 : *, x : x_0 = x_1)$

$\Gamma, \Delta^{A} = (a : *, b : *, c : *, x : a = b)$

$x_0 \mapsto a , x_1 \mapsto b , x \mapsto x$

\item Finally because suspension preserves contractibility, $\Sigma_{A} ~ \Delta$ is contractible

\item In general, $\text{coh}_{B}^{A} := (\Sigma_A ~(\text{coh}_{B})) [ \text{filter}_A ]$ 

\framebreak

\item Application: \textbf{Reflexivity}

\begin{itemize}

\item 1st step: reflexivity (id) in a minimum contractible context

$$x : * \vdash \text{coh}_{x=x} : x = x$$

\item 2nd step: reflexivity for arbitrary type $A$ in arbitrary
  context $\Gamma$

By suspension: $$\Sigma_A ~(x:*) \vdash \Sigma_A ~(\text{coh}_{x=x}) : x = x$$

Replacement defined using filter $$ \text{coh}_{x=x}^{A} := (\Sigma_A ~(\text{coh}_{x=x})) [ \text{filter}_A ]$$ 
\end{itemize}

\end{itemize}
\end{frame}



\begin{frame}{Semantics}

\begin{itemize}

\item A syntactic Grothendieck \wog is a globular set with an
  interpretation of syntactic coherence terms (coh)

\item A globular set $A$ consists coinductively of:
\begin{itemize}
\item A set obj$_{A}$
\item For every $x,y : \text{obj}_{A}$, a globular set Hom$_{A}(x, y)$
\end{itemize}

\item  Example: the identity globular set $Id^{\omega} A$
\begin{itemize}
\item obj$_{Id^{\omega} A} = A$
\item Hom$_{Id^{\omega} A}(a, b) = Id^{\omega} A (a = b)$
\end{itemize}

\item The interpretation of contexts, types and terms

\end{itemize}

\end{frame}











\begin{frame}{Conclusion}


\begin{itemize}

\item Types bear the structure of weak $\omega$-groupoids: the tower
  of iterated identity types

\item An implementation of syntactic \wog in Agda
  \begin{itemize}
  \item {Basic syntax of the type theory \tig}
  \item Heterogeneous equality for terms
  \item Constructions of coherences
  \item Semantic interpretation with globular sets
  \end{itemize}

\item To complete a \wogs model of type theory
\item A computational interpretation of univalent axiom in \itt


\item Question?
\end{itemize}


\end{frame}



\end{document}