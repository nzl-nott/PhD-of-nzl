%%%%%%%%%%%%%%%%%%%%%%%%%%%%%%%%%%%%%%%%%
% Nuo Li's Doctoral Thesis



\documentclass[11pt,a4paper,UKenglish,twoside,openright]{Thesis}


\usepackage{etex}

\PassOptionsToPackage{utf8x}{inputenc}

%\DeclareUnicodeCharacter{00A0}{ }

\graphicspath{{Pictures/}} % Specifies the directory where pictures are stored

\usepackage[all]{xy}
%agda literal file
\usepackage{agda}

\usepackage{multicol}
\usepackage{dsfont} % math symbols
\usepackage{amsthm}
\usepackage{relsize}
\usepackage{color}
\usepackage{amsmath}
\usepackage{wasysym}
\usepackage{amsfonts}
\usepackage{amssymb}
\usepackage{bbm}
\usepackage[greek,english]{babel}
\usepackage{stmaryrd}
\usepackage{textgreek}


%\usepackage{unicode-math}

%\usepackage[T1]{fontenc}
\usepackage[utf8x]{inputenx}
\usepackage{autofe} % font for code environment


\usepackage{booktabs}% horizontal lines

\usepackage{bookmark}

\usepackage{proof} % For type derivation
\setlength{\inferLineSkip}{4pt}

\usepackage{diagxy} % commutative diagram 

\usepackage{bcprules}

\usepackage[square, numbers, comma, sort&compress]{natbib} 
% Use the natbib reference package - read up on this to edit the reference style; if you want text (e.g. Smith et al., 2012) for the in-text references (instead of numbers), remove 'numbers' 

%\usepackage[toc,page]{appendix}


\usepackage{xkvltxp}
\usepackage[]{fixme}
\fxsetup{
    status=draft,
    author=,
    layout=inline,
    theme=color
}

\definecolor{fxnote}{rgb}{0.8000,0.0000,0.0000}
% define the background colour:
\colorlet{fxnotebg}{yellow}

% refedine the layout macro:
\makeatletter
\renewcommand*\FXLayoutInline[3]{%
  \@fxdocolon {#3}{%
    \@fxuseface {inline}%
    \colorbox{fx#1bg}{\color {fx#1}\ignorespaces #3\@fxcolon #2}}}
\makeatother

%\usepackage{glossaries}


%\usepackage[textwidth=3cm]{todonotes}
%\reversemarginpar
%\setlength{\marginparwidth}{3cm}


%%% symbols and macros, mostly from the HoTT book



\newcommand{\prd}[1]{\Pi_{#1}}
\newcommand{\sm}[1]{\Sigma_{#1}}

% \newcommand{\smmultiline}[1]{\Sigma \; \left #1 \right}

\newcommand{\lam}[1]{\lambda #1 .}

%own
\newcommand{\fa}[2]{\forall (#1 : #2) . \,}
% \newcommand{\fa}[1]{\forall (#1). \,}
\newcommand{\fasimple}[1]{\forall #1 . \,}
\newcommand{\fapairs}[3]{\fa {#1 \, #2} #3}
\newcommand{\fatypes}[1]{\fa #1 \type} % forall types

%%% Definitional equality (used infix) %%%
\newcommand{\jdeq}{\equiv}      % An equality judgment
\newcommand{\defeq}{\vcentcolon\equiv}  % A judgmental equality currently being defined

%%% Homotopies (written infix) %%%
\newcommand{\eqvsym}{\simeq}
\newcommand{\eqv}[2]{\ensuremath{#1 \eqvsym #2}\xspace}
\newcommand{\eqvauto}[1]{\eqv{#1}{#1}}


%%% own notation: equivalences
\newcommand{\iseqown}[1]{\ensuremath{\mathsf{e}_{#1}}\xspace}
\newcommand{\eqvsymspace}{\enspace \eqvsym \enspace}
\newcommand{\eqvspace}[2]{\ensuremath{#1 \eqvsymspace #2}\xspace}
\newcommand{\eqvautospace}[1]{\eqvspace{#1}{#1}}


%%% Identity functions %%%
\newcommand{\idfunc}[1][]{\ensuremath{\mathsf{id}_{#1}}\xspace}

%%% Identity types %%%
\newcommand{\idsym}{{=}}
\newcommand{\id}[3][]{\ensuremath{#2 =_{#1} #3}\xspace}
\newcommand{\idspace}[3][]{\ensuremath{#2 \, =_{#1} \, #3}\xspace}
\newcommand{\idtype}[3][]{\ensuremath{\mathsf{Id}_{#1}(#2,#3)}\xspace}
\newcommand{\idtypevar}[1]{\ensuremath{\mathsf{Id}_{#1}}\xspace}
\newcommand{\idauto}[2][]{\id[#1]{#2}{#2}}


%%% Reflexivity terms %%%
\newcommand{\refl}[1]{\ensuremath{\mathsf{refl}_{#1}}\xspace}

%%% Path concatenation (used infix, in diagrammatic order) %%%
\newcommand{\ct}{%
  \mathchoice{\mathbin{\raisebox{0.5ex}{$\displaystyle\centerdot$}}}%
             {\mathbin{\raisebox{0.5ex}{$\centerdot$}}}%
             {\mathbin{\raisebox{0.25ex}{$\scriptstyle\,\centerdot\,$}}}%
             {\mathbin{\raisebox{0.1ex}{$\scriptscriptstyle\,\centerdot\,$}}}
}

%%% Path reversal %%%
\newcommand{\opp}[1]{\mathord{{#1}^{-1}}}
\let\rev\opp
\let\sym\opp

%%% Transport (covariant) %%%
\newcommand{\trans}[2]{\ensuremath{{#1}_{*}\mathopen{}\left({#2}\right)\mathclose{}}\xspace}

\newcommand{\isequiv}{\ensuremath{\mathsf{isequiv}}}
\newcommand{\idequiv}{\ensuremath{\textsf{id-equiv}}}


%%% Universe types %%%
\newcommand{\UU}{\ensuremath{\mathcal{U}}\xspace}
\let\bbU\UU
\let\type\UU
% own
\newcommand{\UUp}{{\UU_\bullet}}
% \newcommand{\UUtt}[2]{\ensuremath{ \UU_{#1}^{\mathsmaller{\leq} #2}}}
% \newcommand{\UUt}[1]{\ensuremath{ \UU^{\mathsmaller{\leq} #1}}}
%%% Natural numbers
\newcommand{\N}{\ensuremath{\mathbb{N}}\xspace}
%\newcommand{\N}{\textbf{N}}
\let\nat\N
\newcommand{\suc}{\mathsf{succ}}
\newcommand{\add}{\mathsf{add}}
% \newcommand{\ack}{\mathsf{ack}}
\newcommand{\ass}{\mathsf{ass}}
\newcommand{\ite}{\mathsf{iter}}

%own
\newcommand{\Q}{\ensuremath{\mathbb{Q}}\xspace}
% \newcommand{\R}{\ensuremath{\mathbb{R}}\xspace}



\newcommand{\dbl}{\ensuremath{\mathsf{double}}}


%%% The empty type
\newcommand{\emptyt}{\ensuremath{\mathbf{0}}\xspace}

%%% The unit type
\newcommand{\unit}{\ensuremath{\mathbf{1}}\xspace}
\newcommand{\ttt}{\ensuremath{\star}\xspace}

%%% The two-element type
\newcommand{\bool}{\ensuremath{\mathbf{2}}\xspace}
\newcommand{\btrue}{{1_{\bool}}}
\newcommand{\bfalse}{{0_{\bool}}}
\let\two\bool
\let\trueb\btrue
\let\falseb\bfalse

\newcommand{\swap}{\ensuremath{\textsf{swap}}}
\newcommand{\swapequiv}{\ensuremath{\textsf{swap-equiv}}}


%%% Truncation levels
\newcommand{\iscontr}{\ensuremath{\mathsf{isContr}}}
\newcommand{\isprop}{\ensuremath{\mathsf{isProp}}}
\newcommand{\isset}{\ensuremath{\mathsf{isSet}}}


\def\compare#1#2#3#4{\if#1#3\if#2#41\else0\fi\else0\fi}

\newcommand{\istype}[1]{
  \edef\a{\compare-2#1\empty\empty}
  \if\a1 \iscontr \else
  \edef\b{\compare-1#1\empty\empty}
  \if\b1 \isprop \else
  \edef\c{#1}
  \if0\c \isset \else
  \mathsf{is}\mbox{-}{#1}\mbox{-}\mathsf{type} \fi\fi\fi
}



%%% Univalence
\newcommand{\ua}{\ensuremath{\mathsf{ua}}\xspace} % the inverse of idtoeqv
\newcommand{\idtoeqv}{\ensuremath{\mathsf{idtoeqv}}\xspace}
\newcommand{\univalence}{\ensuremath{\mathsf{univalence}}\xspace} % the full axiom


%%% Circles and spheres
\newcommand{\Sn}{\mathbb{S}}
\newcommand{\base}{\ensuremath{\mathsf{base}}\xspace}
\newcommand{\lloop}{\ensuremath{\mathsf{loop}}\xspace}
\newcommand{\surf}{\ensuremath{\mathsf{surf}}\xspace}

%%% Map on paths %%%
\newcommand{\mapfunc}[1]{\ensuremath{\mathsf{ap}_{#1}}\xspace} % Without argument
\newcommand{\map}[2]{\ensuremath{{#1}\mathopen{}\left({#2}\right)\mathclose{}}\xspace}
\let\Ap\map
%\newcommand{\Ap}[2]{\ensuremath{{#1}\left({#2}\right)}\xspace}
\newcommand{\mapdepfunc}[1]{\ensuremath{\mathsf{apd}_{#1}}\xspace} % Without argument
% \newcommand{\mapdep}[2]{\ensuremath{{#1}\llparenthesis{#2}\rrparenthesis}\xspace}
\newcommand{\mapdep}[2]{\ensuremath{\mapdepfunc{#1}\mathopen{}\left(#2\right)\mathclose{}}\xspace}
\let\apfunc\mapfunc
\let\ap\map
\let\apdfunc\mapdepfunc
\let\apd\mapdep

%own
\newcommand{\happly}{\ensuremath{\mathsf{happly}}}

%% OWN: UIP
\newcommand{\uipplain}{\mathsf{UIP}}
\newcommand{\uipext}[1]{\uipplain_{#1}}

\newcommand{\dec}{\mathsf{decidable}}
\newcommand{\deceq}{\mathsf{discrete}}


% % \newcommand{\trunc}[2]{\left\lVert #1 \right\rVert_{#2}} % note: \| is not symmetric, is it?
% % \newcommand{\truncmap}[2]{\left\lvert #1 \right\rvert_{#2}} % note: \| is not symmetric, is it?
% \newcommand{\trunc}[1]{\left\lVert #1 \right\rVert} % note: \| is not symmetric, is it?
% \newcommand{\truncmap}[1]{\left\lvert #1 \right\rvert} % note: \| is not symmetric, is it?

%%% Bracket/squash/truncation types %%%
\newcommand{\trunc}[2]{\mathopen{}\left\Vert #2\right\Vert_{#1}\mathclose{}}
\newcommand{\ttrunc}[2]{\bigl\Vert #2\bigr\Vert_{#1}}
\newcommand{\Trunc}[2]{\Bigl\Vert #2\Bigr\Vert_{#1}}
\newcommand{\truncf}[1]{\Vert \blank \Vert_{#1}}
\newcommand{\tproj}[3][]{\mathopen{}\left|#3\right|_{#2}^{#1}\mathclose{}}
\newcommand{\tprojf}[2][]{|\blank|_{#2}^{#1}}
\def\pizero{\trunc0}
%\newcommand{\brck}[1]{\trunc{-1}{#1}}
%\newcommand{\Brck}[1]{\Trunc{-1}{#1}}
%\newcommand{\bproj}[1]{\tproj{-1}{#1}}
%\newcommand{\bprojf}{\tprojf{-1}}



\newcommand{\bbbrck}[1]{\Bigl\Vert #1 \Bigr\Vert}
\newcommand{\brck}[1]{\trunc{}{#1}}
\newcommand{\bbrck}[1]{\ttrunc{}{#1}}
\let\bracket\bbrck
\newcommand{\Brck}[1]{\Trunc{}{#1}}
\let\Bracket\Brck
\newcommand{\bproj}[1]{\tproj{}{#1}}
\newcommand{\bprojf}{\tprojf{}}

% OWN
\newcommand{\populated}[1]{\langle \! \langle #1 \rangle \! \rangle}


%OWN
\newcommand{\fst}{\pi_1 \!}
\newcommand{\snd}{\pi_2 \!}
\newcommand{\inl}{\mathsf{inl}\xspace}
\newcommand{\inr}{\mathsf{inr}\xspace}
\newcommand{\inone}{\mathsf{in_1}\xspace}
\newcommand{\intwo}{\mathsf{in_2}\xspace}
\newcommand{\inthree}{\mathsf{in_3}\xspace}


%\newcommand{\hproptype}{\textbf{hProp}}


\newcommand{\LEM}[1]{\ensuremath{\mathsf{LEM}_{#1}}\xspace}
\newcommand{\choice}[1]{\ensuremath{\mathsf{AC}_{#1}}\xspace}

\usepackage{mypack}

\usepackage{tikz}

\hypersetup{urlcolor=blue, colorlinks=true}
% Colors hyperlinks in blue - change to black if annoying

\title{\ttitle}
% Defines the thesis title - don't touch this



\newcommand{\HRule}{\rule{\linewidth}{0.5mm}} % New command to make the lines in the title page




\usepackage{aliascnt}
\usepackage{cleveref}

%\usepackage{hyperref}

\let\lemma\relax
\newaliascnt{lemma}{theorem}
\newtheorem{lemma}[lemma]{Lemma}
\aliascntresetthe{lemma}
\crefname{lemma}{lemma}{lemmas}

\let\definition\relax
\newaliascnt{definition}{theorem}
\newtheorem{definition}[definition]{Definition}
\aliascntresetthe{definition}
\crefname{definition}{definition}{definitions}


\let\proposition\relax
\newaliascnt{proposition}{theorem}
\newtheorem{proposition}[proposition]{Proposition}
\aliascntresetthe{proposition}
\crefname{proposition}{proposition}{propositions}

%\theoremstyle{theorem}
%    \newtheorem{theorem}{Theorem}

%\let\definition\relax
%\theoremstyle{definition}
%    \newaliascnt{definition}{theorem}
%    \newtheorem{definition}[definition]{Definition}
%    \aliascntresetthe{definition}
%    \providecommand*{\definitionautorefname}{Definition}


%\newcommand{\propositionautorefname}{Proposition}




\begin{document}

\frontmatter % Use roman page numbering style (i, ii, iii, iv...) for the pre-content pages

\setstretch{1.3} % Line spacing of 1.3

% Define the page headers using the FancyHdr package and set up for one-sided printing
\fancyhead{} % Clears all page headers and footers
\rhead{\thepage} % Sets the right side header to show the page number
\lhead{} % Clears the left side page header

\pagestyle{fancy} % Finally, use the "fancy" page style to implement the FancyHdr headers

%\newcommand{\HRule}{\rule{\linewidth}{0.5mm}} % New command to make the lines in the title page

% PDF meta-data
\hypersetup{pdftitle={\ttitle}}
\hypersetup{pdfsubject=\subjectname}
\hypersetup{pdfauthor=\authornames}
\hypersetup{pdfkeywords=\keywordnames}


%\listoftodos

%----------------------------------------------------------------------------------------
%	TITLE PAGE
%----------------------------------------------------------------------------------------

\begin{titlepage}

\begin{center}

%\includegraphics[height=1.5cm,scale=0.1]{Pictures/2000px-University_of_Nottingham.png}\\% Logo instead of name
%\textsc{\LARGE \univname}\\[1.5cm] % University name


%\textsc{\Large Doctoral Thesis}\\[0.5cm] % Thesis type

%\HRule \\[0.4cm] % Horizontal line
{\huge \bfseries \ttitle}\\[6cm] % Thesis title
%\HRule \\[1.5cm] % Horizontal line
 

 \large {\authornames, BSc.}
\\[3cm]

%\begin{minipage}{0.4\textwidth}
%\begin{flushleft} \large
%\emph{Author:}\\
%{\authornames, BSc.} % Author name - remove the \href bracket to remove the link
%\end{flushleft}
%\end{minipage}
%\begin{minipage}{0.4\textwidth}
%\begin{flushright} \large
%\emph{Supervisor:} \\
%{\supname} % Supervisor name - remove the \href bracket to remove the link  
%\end{flushright}
%\end{minipage}\\[3cm]
 
\large \textit{Thesis submitted to the University of Nottingham\\  for the degree of Doctor of Philosophy}\\[4cm] % University requirement text
%\textit{in the}\\[0.4cm]
%\groupname\\\deptname\\[2cm] % Research group name and department name
 
{\large September 2014}\\[4cm] % Date
%\includegraphics{Logo} % University/department logo - uncomment to place it
 
\vfill
\end{center}

\end{titlepage}


%\clearpage % Start a new page

%----------------------------------------------------------------------------------------
%	ABSTRACT PAGE
%----------------------------------------------------------------------------------------

%\addtotoc{Abstract} % Add the "Abstract" page entry to the Contents
\cleardoublepage
%\pagestyle{empty}
%\begin{abstract}


%\end{abstract}
\abstract{\addtocontents{toc}{\vspace{1em}} % Add a gap in the Contents, for aesthetics


Martin-L\"{o}f's intuitionistic type theory (Type Theory) is a formal system that serves as not only a foundation of constructive mathematics but also as a dependently typed programming language. Dependent types are types that depend on values of other types. Type Theory is based on the Curry-Howard isomorphism which relates computer programs with mathematical proofs so that we can do computer-aided formal reasoning and write certified programs in programming languages like Agda, Epigram etc. Martin L\"{o}f proposed two kinds of variants of Type Theory which are differentiated on the treatment of equalities. In \itt, propositional equality defined by identity types does not imply definitional equality, and the type checking is decidable. In \ett, propositional equality is identified with definitional equality which makes type checking undecidable. Because of the good computational properties, \itt is usually more popular, however there are some important extensional concepts missing, such as functional extensionality and quotient types etc.

This thesis is about quotient types. A quotient type is a new type whose equality is redefined by an given equivalence relation. However, in the usual formulation of \itt, there is no type former to create a quotient. We will also lose canonicity if we add quotient types into Intensional Type Theory as axioms. In this thesis, we first investigate what are the syntax of quotient types we expect to have, and explain the syntax with categorical counterparts. For quotients which can be represented as a setoid as well as defined as a set without a quotient type former, we propose to define an algebraic structure of quotients called \emph{definable quotients}. It relates the setoid interpretation and the set one via a normalisation function which returns a normal form (canonical choice) for each equivalence class. It can be seen as a simulation of quotient types and it helps theorem proving because we can benefit from both representations. However it is only a compromise approach since it does not apply to all quotients. It seems that we can not define a normalisation function for some quotients in Type Theory, e.g.\ Cauchy reals and finite multisets. Quotient types are indeed essential for formalisation of mathematics and reasoning of programs. Then we consider some models of Type Theory where types are interpreted as structured objects such as setoids, groupoids or \wog. In these models equalities are internalised into types which means that it is possible to redefine equalities. We present an implementation of Altenkirch's \cite{alti:lics99} setoid model  and show that quotient types can be defined within this model. We also investigate a new extension of \mltt called \hott where types are interpreted as \wog. It can be seen as a generalisation of groupoid model and the extensional concepts including quotient types are available. We also introduce a syntactic encoding of \wog which is considered as a first step to build a \wog model in \itt. All these implementations have been done in the dependently typed programming language Agda which is based on intensional \mltt.

}


\clearpage % Start a new page

%\pagestyle{empty}
%----------------------------------------------------------------------------------------
%	ACKNOWLEDGEMENTS
%----------------------------------------------------------------------------------------

\setstretch{1.3} % Reset the line-spacing to 1.3 for body text (if it has changed)


\acknowledgements{\addtocontents{toc}{\vspace{1em}} % Add a gap in the Contents, for aesthetics
%\begin{acknowledgements}

The first person I would like to thank is my supervisor Thorsten Altenkirch. He offered my a great internship on encoding numbers in Agda which finally has been developed into this project. He has always been patient in explaining my questions and taught me a lot on research.
My second supervisor Thomas Anberrée is the one who first introduced me functional programming. He has given me a lot precious advises on my project.
I would also like to thank Venanzio Capretta who examined my first and second year report and provided helpful feedbacks.

My friends Nicolai Kraus, Ambrus Kaposi, Christian Sattler, Paulo Caprotti, Florent Balestrieri, Gabe Dijkstra, Neil Sculthorpe and all other PhD students in our lab has helped me a lot in my project by either teaching me mathematics, discussing on research topics and playing badminton to exercise our bodies. The Functional programming lab is really a lively family and I would like to thank everyone here.
Without their generous help, the thesis would not have been finished.


I would also like to thank the organizers and
other participants of the special year on homotopy type theory at the
Institute for Advanced Study where they had many interesting
discussion topics related to part of work, especially Guillaume
Brunerie whose proposal made it possible. Nordvall Forsberg also provided important discussions on our work.

During the years in Nottingham, My mother Guangfei Lv, and my father Youyuan Li continuously support me from China, talk to me and send me my favourite foods. My aunt Guangshu Lv also helps me a lot, gives me advices by sending mails from Germany. Without their support I would never achieve my goals.

Finally, I would like to thank the School of Computer Science and international office in the University of Nottingham who financially support this project.

%\end{acknowledgements}

}

\clearpage % Start a new page

%----------------------------------------------------------------------------------------
%	LIST OF CONTENTS/FIGURES/TABLES PAGES
%----------------------------------------------------------------------------------------

\pagestyle{fancy} % The page style headers have been "empty" all this time, now use the "fancy" headers as defined before to bring them back

\lhead{\emph{Contents}} % Set the left side page header to "Contents"
\setcounter{tocdepth}{1}
\tableofcontents % Write out the Table of Contents

%\lhead{\emph{List of Figures}} % Set the left side page header to "List of Figures"
%\listoffigures % Write out the List of Figures

%\lhead{\emph{List of Tables}} % Set the left side page header to "List of Tables"
%\listoftables % Write out the List of Tables




%----------------------------------------------------------------------------------------
%	THESIS CONTENT - CHAPTERS
%----------------------------------------------------------------------------------------

\mainmatter % Begin numeric (1,2,3...) page numbering

\fancyfoot{}
\fancyhead[RO,LE]{\thepage}
\fancyhead[LO]{\leftmark}
\fancyhead[RE]{\rightmark}

\pagestyle{fancy} % Return the page headers back to the "fancy" style

% Include the chapters of the thesis as separate files from the Chapters folder
% Uncomment the lines as you write the chapters


\chapter{Introduction}

The aim of this work is to find out the approach to implement quotient
types in intensional \mltt. Type theory serves as both a foundation of
mathematics and a programming language. In set theory quotient set is
a set of the equivalence classes of some equivalence relation on
another set. The word \emph{Quotient} is usually used to capture
similar notions in other abstract branches, such as quotient group,
quotient space, quotient category etc. Move to type theory, it is
expected to have \emph{Quotient types} to construct similar objects in
mathematics. However in current setting of \itt, it is not
available. Motivated by this fact, I start to extend type theory with
quotient types. This thesis \todo{fill in the missing part}







Theories of dependent types have been proposed as a foundation of constructive mathematic and as a framework in which to construct certified programs and to extract programs from proofs. Using implementations of such type theories, substantial pieces of constructive mathematics have been formalised and medium scale program developments and verifications have been carried out.

Type theory is usually considered as an more rigorous foundation of
constructive mathematics that set theory. Per Martin-L\"{o}f's developed several
versions of type theory which we usually call them Martin-Löf's
intuitionistic type theory (or in short Martin-Löf's type theory). A number of
dependently typed functional programmming languages are developed upon
one of his type theories, including NuPRL, LEGO, Coq, Agda, Epigram, Pi-Sigma etc.
The Curry-Howard correspondence (propositions corresponds to types and
proofs corresponds to terms), and the notion of dependent types from
\mltt make it possible to do computer-based reasoning within these
languages.

However there are many important extensional concepts missing in the
intensional variants of \mltt. The equality in mathematics is
subdivided into several different notions in type theory and many
problems arose.
\emph{Quotient types} is one of these extensional
concepts which does not exist in \itt like Agda. 

The main subject of this thesis is to study the approaches to encode
quotients and to implement quotient types in type theory.

\section{Background}

\subsection{Equality and extensional concepts}

We first learn equality for numbers in mathematics. It generally
expressed the sameness of two mathematical objects such as numbers or sets.
$1+2=3$ tells us the mathematical object calculated from $1+2$ is the
same as $3$.

From ages before, mathematicians are arguing about the correct way to
characterise equality. The most common idea is that equality is just
identity. Leibniz's law can be seen as a "definition" of
equality:

Given any $x$ and $y$, $x = y$ if, given any predicate $P$, $P(x)$ if and only
if $P(y)$.

In type theory, equality also plays a very important role.
It is not difficult to imagine that the subject of equality is also
contentious in type theory. Considering a variety of issues, there are
several notions of equality in type theory.
We can compare whether two types are equal or two terms are equal. If two
expressions can be computed to the same object then we claim they are
 \emph{definitionally} equal.  In addition we can define types to
 internalise these propositions. A term of this type is just a witness or evidence of
 this equality. We usually call these types identity types or
 propositional equalities.

The problem also arises from the computational difficulties. We
usually equate functions which gives the same outputs if we feed in the
same inputs. However this equality is extensional because the intension
of functions are definitions which can be different. In \itt, which is
more widely used, the encoding of this extensional equality of
functions is also a problem. A type checker can easily decide whether two
values of inductive types are the same, but not whether two functions
are extensionally equal. Even suppose we encode the method to compute
the definitional equality of the outputs for each given input, it can
only decide the equality for finite types. Because there is no
attached equality for each types other than trivial identity, it is
necessary to use a different model of type theory in which types are
equipped with equalities.

\subsection{Quotient types}


In mathematics, a quotient set is a set of the equivalence classes of
some equivalence relation on another set. Abstracted from this
concept, there are also quotient groups, quotient categories etc.

Naturally, given a type and an equivalence relation, quotient types
are interesting to study. Due to the difference of type theory and set
theory, quotient types can not be constructed via equivalence classes
(or we need some other axioms as we will discussed in detail
later).

Quotient types are very useful in three aspects. 
When the quotients
can be defined without the help of quotient types, it provides us
framework to define functions and properties on the base types and
lift them if they respects the equivalence relation. This sounds
superfluous at first, but in practise, it is much easier to define
functions for the base types (because usually they have a bundle of
library codes) on underlying level, and the ''quotient types'' has no
redundant on the surface level.
Quotient types can also enable us define some types which are not
definable in standard \itt, for example real numbers, multisets etc.
Usually the examples we want to implement by quotient types are cases
of quotient sets in mathematics. However, there are not only sets
(whose equality is propositional), but also other types with higher
level of structures. It is an interesting topic in \hott and quotient
types are extended to cover types which are more than $h-sets$.

The topic of quotient types started very early and there are already
several models of type theory to encode extensional concepts including
quotient types.

\subsection{different models}

If each type comes with its own equality, it looks like it naturally
encodes quotient types.

\todo{short explanation of models or leave it for later discusssion.}




\subsection{Applications}

\todo{mention the definable quotient is very useful}

Quotients are very useful in general. The definable quotient structure
which can be implemented without any extension already provide some
good mechanisms. In a definable quotient structure, the ``quotient''
is defined separately as a self-cotained type and what benefits us are
the lifting functions. In some cases, for example, to use a pair of
natural number to represent an integer, it helps us prove properties
on the setoids then lift them to the ones on the ``quotient'' type. To
be more precise, it provides a bridge between the types which we know
a lot (have plenty of theorems and functions) and the new
``quotient'' types.


In a type theory with quotient type, real numbers can be encoded as
the cauchy sequences with an equivalence relation.


\section{Programming in Agda}

In this thesis we use Agda as the only programming language. Agda is a dependently typed, functional programming language, which is
based on the intensional version of \mltt. It is mainly developed by
Ulf Norell, Nils Anders Danielsson, and Andreas Abel.
It is a functional language like Haskell, but has dependent types. It has dependent pattern
matching, type checker, coverage checker and termination checker. It
supports a bundle of scheme of defing data types such as inductive, inductive-recursive, mutually inductive,
coinductive types. Its syntax is close to Haskell, in addition, it has
unicode support and mixfix operators. All these features including
type checker, unicode typing and interactive programming present in
its emacs interface. It is quite popular in the field of mathematical reasoning and programs
verifications. Many researchers implement their work of type theory in
Agda. It is also one of the popular languages in the community of the recently
developed field \hott and the library of \hott is under construction.
The adequate reasoning library code also helps in writing certified programs.

In chapter \ref{bg} we present a brief introduction to it for someone
who meet this language for the first time. In chapter 3 we use Agda to
show...\todo{Write where we use Agda and how}

Agda wiki\cite{agdawiki:main} is a good reference to it and there are
several good tutorials to Agda for example Ulf Norell's
\cite{tutorial}, Ana Bove, Peter Dybjer, and Ulf Norell's \cite{bove2009brief}.

\section{Related work}


\section{Overview}


\todo{compact, comprehensive Overview: \\Add overview of each part, as much as you can, and compact}



In \autoref{bg}, we will discuss the backgroud of this research. Type theory is a popular topic in
theoretical computer science. It is quite powerful not only a a theory
but also as a programming language. We use a dependent functional
programming language called Agda which is design based on \mltt. The
 related work will also be discussed in this chapter.


In \autoref{qt}, we will discuss quotient types which is the topic of
this thesis in detail. Quotient types
can be understood as a interpretation of quotient set in set
theory. It is an extensional concept which is also related to other extensional concepts. It can be encoded in different ways. Categorically speaking it
is a coequalizer, and a split quotient is a just a split coequalizer.


In \autoref{dq}, we start introducing one of our achievements, the
definable quotients. It is usually very unreadable, unorganised and
complicated to write some programs without abstracting. It is also
applied to quotient types. If we have some types that can be abstract
as a quotient type of some common types, then it will be easily
encoded and manipulated. As a example, integers can be encoded as the
quotients types of paired natural numbers over the equivalence
relation that two pairs are equal if they represent the same
subtraction.

In \autoref{rl}, we discuss the undefinable quotients, specially the
real numbers. Also multisets and partiality monads are mentioned as
examples of quotients which can be implemented as long as we have
quotient types. 


In \autoref{sm}, we will talk about the setoid model approach to encode
extensional concepts. The work is mainly extending the setoid model
done by Altenkirch in \cite{alti:lics99} to
quotient types.


In \autoref{HITs}, we will discuss the new area between mathematics and
computer science -- \hott. We will talk about the higher inductive
types and also the \wog-model which is used to interpret
homotopy types in \itt. Quotient types can be encoded \hott simply.
%%

\chapter{Type Theory}
\label{bg}

Type theory is usually used to refer to a formal systems in which every term always has its type. It is initially invented as a foundation for mathematics as an alternative to set theory, and later works well in computer science as programming languages which we can write certified programs with type system. There are various type theories, but here we mainly study Per Martin-L\"{o}f's intuitionistic type theory (N.B.\ we will use "type theory'' specially for it if not ambiguous). It is one of the most widely studied type theory in computer science and has been implemented as programming languages like Agda, Coq, NuPRL, Epigram etc. Another type theory will play important role in this thesis is Homotopy Type Theory which is a comparatively new field between mathematics and computer science which interprets type theory using notions in homotopy type theory.

In this chapter we will introduce type theory, especially \itt, from history to rules. We will also introduce one of the implementation of intensional type theory -- Agda briefly. We will also discuss the topic of extensional concepts which are missing from \itt as a prerequisite for the topic of our thesis -- quotient types.



\section{From set theory to type theory}

The concept of \emph{sets} has been used since thounsands of years ago. However mathematician have not study the theory of sets until 1870s when George Cantor and Richard Dedekind founded set theory as a branch of mathematical logic. Since then set theory is used as a language to describe definitions of most mathematical objects, namely it is a foundational system for mathematics.



However, in the 1900s, Bertrand Russell discovered a paradox in their system. In this naive set theory, there was no distinction between small sets like the set of natural numbers or the set of real numbers and "larger" sets like the set of all sets. This lead to Russell's paradox.

\begin{example}[Russell's Paradox]
Let $R$ be the set of all sets which do not contain themselves
$R = \{x ~| ~x \not\in  x\}$
Then we got a cotradiction
$R \in R \iff R \not\in R$
\end{example}

To avoid this paradox, Russell
proposed the theory of types \cite{rus:1903} as an alternative to
naïve set theory. Each mathematical object is assigned a type. This is done in a hierarchical structure such that "larger" sets and small sets reside in different levels. The set of all
sets is no longer on the same level as its elements and the
paradox disappears.

The elementary notion of type theory is \emph{type} which plays a similar role to set in set theory, but differs fundamentally. Every object in type theory comes with its unique type, while an object in set theory can appear in multiple sets and we can talk about an object without knowing which set it belongs to.
To explain the difference, we use the the number $\mathsf{2}$ as an example. In set theory, 2 is not an element of only one specific set, it belongs to the set of natural numbers $\N$ and also the set of integers $\Z$. While in type
theory, it is impossible to avoid mentioning the type of object $\mathsf{2}$. Usually the term $\AgdaInductiveConstructor{suc} \, \AgdaSymbol{(}\AgdaInductiveConstructor{suc} \, \AgdaInductiveConstructor{zero}\AgdaSymbol{)}$ stands for $\mathsf{2}$ of type $\N$ and we have a different term of type $\Z$ constructed by constructors of $\Z$ which is a different object to the one of $\N$. 

Since Russell's type theory, a variety of type theories have been developed by mathematicians and computer scientists, for example Gödel's System T \cite{gdl:1931}. There are two families of famous type theories building the bridges between mathematics and computer science, \emph{lambda calculus} and \emph{\mltt}.


\subsection{Lambda Calculus}

Alonzo Church introduced lambda calculus in the 1930s. He first introduced
an untyped lambda calculus which turned out to be inconsistent due to
the Kleene-Rosser paradox\cite{kleene1935inconsistency}.

\begin{example}[Kleene-Rosser paradox]
Suppose we have a function $f = \lambda x . \neg (x ~ x) $, then we can deduce a contradiction by applying it to itself:

$f f = (\lambda x . \neg (x ~ x)) f = \neg (f ~ f)$
\end{example}

Then he refined it with adding types. This theory is also called
Church's theory of types or simply typed lambda calculus is
introduced as a foundation of mathematics. An important change is that functions become primitive objects which means functions are types defined inductively using the $\rightarrow$ type former. It is widely applied to
various fields especially computer science. Some languages are extentions of lambda calculus, for example Haskell. Haskell belongs to one of the variants of lambda calculus called System F, although it has evolved into System FC recently. There are also other refinements of lambda calculus which is illustrated by the $\lambda$-cube \cite{barendregt1991introduction}.



\subsection{Per Martin-L\"{o}f's Type Theory}

In 1970s, Per Martin-L\"{o}f \cite{per:71,per:82}  developed his profound intuitionistic type theory. His 1971's formulation which is impredicative was proved to be inconsistent because of Girard's paradox \cite{hurkens1995simplification}. The impredicativty means that for a universe $\mathsf{U}$, there is an axiom $\mathsf{U} \in \mathsf{U}$. The later version is predicative and is more widely used.


It serves as a foundation of constructive mathematics \cite{martin1984intuitionistic}. Different to set theory whose axioms are based on first-order logic, \mltt provides a means of implementing intuitionistic logic. This is achieved by the Curry-Howard
isomorphism\emph{``propositions can be interpreted as types and their
  proofs are inhabitants of these types''}. It relates computer programs with mathematical proofs. Technically, a proposition can be encoded as a type, and then a proof of it can be given by constructing a program (or term) of it. It internalises the  Brouwer–Heyting–Kolmogorov (BHK) interpretation of intuitionistic logic, for example a proof of $P \wedge Q$ is a pair of $p : P$ and $q : Q$.
Started from these, implications are functions, negations are functions into empty types, modus ponens is function application etc. This makes it a programming makes it a programming languages in which we can certify programs within the same language.
Compared to set theory, type theory has less axioms and gives us the possiblity of construct mathematics in computers so that we can do theorem proving and verification with the help of computers.

Another important feature of \mltt is dependent types. The type systems of simply lambda calculus like Haskell are not expressive enough to encode predicate logic without dependent types.

\begin{definition}\label{dpty}
\textit{Dependent type}. Dependent types are types that depends on values of other types \cite{dtw}. 
\end{definition}

As a example, it is possible to write a type for a list of natural number of length 3 as $List~\N ~3$. 

With dependent types, the quantifiers like $\forall$ and $\exists$ can be encoded.
The Curry-Howard isomorphism is then extended to predicate logic. 
A predicate on $X$ can be written as a dependent type $P ~x$ where $x : X$. 


\textbf{Equalities in Type Theory}
The notion of equality is one of the most profound topic in type theory.
we have two kinds of equality, one is definitional equality, the other is propositional equality.

\begin{definition}
\textit{Definitional equality} is a judgement-level equality, which holds when two objects have the same normal forms\cite{nor:90}.
\end{definition}

% Objects are definitional equal if they normalise to the same form. 
% Usually types like $a \equiv b$ stands for a definitional
% equality between $a$ and $b$. It is some primitive jundgements which are part of the
% meta-theory rather than construction like types or terms . Definitional
% equality can be judged and decided by type-checker. 


With dependent types, it is possible to write a type to encode the equality of objects.

\begin{definition}
\textit{Propositional equality} is a type which represents a propostion that two objects of the same type are equal.
\end{definition}

Intuitively if two objects are definitionally equal, they must be propositionally equal.

\begin{equation*}
\infer[\text{Id-intro}]{a = b}{a \equiv b}
\end{equation*}

But how about the other way around (we call it the \emph{equality reflecition rule})? Are two propositional equal objects definitional equal?
\begin{equation}
\label{reflection}
\infer[\text{Reflection}]{a \equiv b}{a = b}
\end{equation}
 Actually
the treatment of equality creates two different versions of \mltt, \emph{intensional} one and \emph{extensional} one.

In \itt, the answer is no. Propositional equality (also called intensional equality  \cite{nor:90}) is different to definitional equality. 
The definitional equality is always decidable hence type checking that depends on definitional equality is
decidable as well~\cite{alti:lics99}. Therefore \itt has better computational behaviors.
Types like $\mathsf{a = b}$ which stands for a
propostion that $\mathsf{a}$ equals $\mathsf{b}$ are propositional equalities. They are some types which we need to prove
or disprove by construction. Each of them has an unique element $\mathsf{refl}$ which only exists if $\mathsf{a}$ and $\mathsf{b}$ are
definitionally equal in all cases. However it is not enough for other extensional equalities, for example the equality of functions.


\ett adopted the equality reflection rule which means the propositional equality is extensional and is undistinguished with definitional equality, in other words, two propositional equal objects are judgementally equal.

There are some extensional concepts exists in \ett. For example the functional extensionality.

\begin{definition}[Functional extensionality]\label{fun-ext}
Objects of different normal forms, for example point-wise equal functions or different proofs of the same proposition, may be definitionally equal. This is called functional extensionality.

\begin{equation}
\infer[\text{fun-ext}]{f = g}{f g : A \rightarrow B, \forall a : A, f \, a = g \, a}
\end{equation}

\end{definition}



\begin{lemma}
\label{functional extensionality is available in ett}
Functional extensionality \ref{fun-ext} is derivable from equality refleciton rule \ref{reflection} (in \ett).
\end{lemma}
 
\begin{proof}
Suppose $\Gamma \vdash f \,a = g \,a$, with reflection rule we have $\Gamma \vdash f \,a \equiv g \,a$.
Then using $\xi$-rule, we know that $\Gamma \vdash \lambda a . f \,a \equiv \lambda a . g \,a$.
From $\eta$-equivalence, we know that $\Gamma \vdash f \equiv g$. We can conclude that $\Gamma \vdash f = g$.
\end{proof}

In \itt, this is not provable. If we add it as an axiom, we will lose the canonicity since we can construct a natural number with this extensionality and substitution for propositional equality.

\begin{example}[non-canonical construction]
Suppose we define $id := \lambda x . x$ and $p0 := \lambda x . x + 0$, recursively it is provable that $p': \forall x , id~ x = p0~x$, from functional extensionality we obtain a propositional equality $p: id = p0$ such that $p \equiv ext ~p'$. We can use substitution to construct non-canonical natural numbers, for instance $subst~ (\lambda f. \N) ~p ~0$. This is non-canonical because the equality is non-canonical so that the expression cannot be normalised.
\end{example}


The non-canonicity makes the definitional equality undecidable and so is type checking. The termination of type checker is not assured.


The choice whether adopt equality reflection rule or not decides the version of \mltt. Agda chooses the intensional one while NuPRL chooses the extensional one.  However we consider intensional version more as a better choice because it ihas better computational behaviors. Even though we do not get the extensional concepts automatically, we still want them in \itt and people are keeping on studying the way to introduce these into \itt.

Altenkirch and McBride introduced a variant of \ett called
\emph{Observational Type Theory}  \cite{alt:06} in which definitional equality is
decidable and propositional equality is extensional.


We already know that \mltt can be encoded as programming languages in
which the evaluation of a well-typed program always terminates \cite{nor:90}.


\section{Agda}

Agda is a dependently typed functional programming language which is designed based on intensional version
of \mltt \cite{agdawiki:main}. It is used as the main tool to study the topic of this thesis.

As we have seen, \mltt is based on the Curry-Howard
isomorphism: types are identified with propositions and terms (or programs) are identified with proofs. It turns Agda is into a proof assistant like Coq, which allows users to do mathematical reasoning and also computer program reasoning. 
Usually to prove the correctness of programs, we need to state some theorems of programming languages on the meta-level, but in Agda we can prove and use these theorems alongwith writing programs.  As Nordström et al. \cite{nps} pointed out that we could express both specifications and programs at the same time when using the type theory to construct proofs using programs.


\todo{following paragraph needs to be rewritten or added with code}
We can prove a proposition following steps below:
First we give the name of the proposition and encode it as the type. Then we can gradually refine the goal to formalise a type-correct program namely the proof. As long as we have the proof, it can be used as a lemma in other proofs or programs. Usually, there are no tactics like in Coq (it may be implemented in the future). But with the gradually refinement mechanism, the process of building proofs is very similar to conceiving proofs in regular mathematics.


There are more features of Agda as follows:

\begin{itemize}
\item \textit{Dependent type}. 
As mentioned in \ref{dpty}, dependent types are types that depends on values of other types \cite{dtw}. They enable us to write more expressive types as program speficication or propositions in order to reduce bugs. In Haskell and other Hindley-Milner style languages, types and values are clearly distinct \cite{tutorial}, In Agda, we can define types depending on values which means the border between types and values is vague. To illustrate what this means, the most common example is $\mathsf{Vector A n}$ where we can length-explicit lists called vectors. It is a data type which represents a vector containing elements of type \textbf{A} and depends on a natural number \textbf{n} which is the length of the list. We can specify types with more constraints such that the we can express what programs we can better and leave the checking work to the type chekcer. For instance, to use the length-explicit vector, we will not encounter exceptions like out of bounds in Java, since it is impossible to define such functions before compiling.

\item \textit{Functional programming language}. As the name indicates that, functional programming languages emphasizes the application of functions rather than changing data in the imperative style like C{}\verb!++! and Java. The base of functional programming is lambda calculus. The key motivation to develop functional programmming language is to eliminating the side effects which means we can ensure the result will be the same no matter how many times we input the same data. There are several generations of functional programming languages, for example Lisp, Erlang, Haskell etc. Most of the applications of them are currently in the academic fields, however as the functional programming developed, more applications will be explored.

% \item \textit{Per Martin-Löf Type Theory}. It has different names like Intuitionistic type theory or Constructive type theory and is developed by Per Martin-Löf in 1980s. It associated functional programs with proofs of mathematical propositions written as dependent types. That means we can now represent propositions we want to prove as types in Agda by dependent types and Curry-Howard isomorphism \cite{aboa}. Then we only need to construct a program of the corresponding type to prove that propostion. For example:

\end{itemize}

As a functional programming languages, Agda also has some nice features for theorem proving,

\begin{itemize}

\item \textit{Pattern matching}. The mechanism for dependently typed pattern matching is very powerful \cite{alti:pisigma-new}. We could prove propositions case by case. In fact it is similar to the approach to prove propositions case by case in regular mathematics. Pattern match is a more intuitive way to use the eliminators of types. For example, to define the negation on booleans $neg : Bool \to Bool$, we can write the program in two cases with respect to the value of the argument.

\item \textit{Inductive \& Recursive definition}. In Agda, types are often defined inductively, for example, natural numbers is defined as

\begin{code}\>\<%
\>\AgdaKeyword{data} \AgdaDatatype{ℕ} \AgdaSymbol{:} \AgdaPrimitiveType{Set} \AgdaKeyword{where}\<%
\\
\>[0]\AgdaIndent{2}{}\<[2]%
\>[2]\AgdaInductiveConstructor{zero} \AgdaSymbol{:} \AgdaDatatype{ℕ}\<%
\\
\>[0]\AgdaIndent{2}{}\<[2]%
\>[2]\AgdaInductiveConstructor{suc} \<[7]%
\>[7]\AgdaSymbol{:} \AgdaSymbol{(}\AgdaBound{n} \AgdaSymbol{:} \AgdaDatatype{ℕ}\AgdaSymbol{)} \AgdaSymbol{→} \AgdaDatatype{ℕ}\<%
\>\<\end{code}

The function for inductive types are usually written in recursive style, for example, the double function for natural numbers,

\begin{code}\>\<%
\\
\>\AgdaFunction{double} \AgdaSymbol{:} \AgdaDatatype{ℕ} \AgdaSymbol{→} \AgdaDatatype{ℕ}\<%
\\
\>\AgdaFunction{double} \AgdaInductiveConstructor{zero} \AgdaSymbol{=} \AgdaInductiveConstructor{zero}\<%
\\
\>\AgdaFunction{double} \AgdaSymbol{(}\AgdaInductiveConstructor{suc} \AgdaBound{n}\AgdaSymbol{)} \AgdaSymbol{=} \AgdaInductiveConstructor{suc} \AgdaSymbol{(}\AgdaInductiveConstructor{suc} \AgdaSymbol{(}\AgdaFunction{double} \AgdaBound{n}\AgdaSymbol{))}\<%
\>\<\end{code}

The availability of recursive definition enables programmers to prove propositions in the same manner of mathematical induction. 

\item \textit{Construction of functions}. One of the advantage of using a functional programming language as a theorem prover is the construction of functions which makes the proving more flexible.

In functional programming languages, complicated programs are commonly built gradually using aunxiliary functions and frequently used functions in the library.

Described as a proof assistant, complicated theorems are commonly proved gradually using lemmas and other theorems we have proved.

This decreases the difficulty of interpreting proofs in mathematics into Agda.

\item \textit{Lazy evaluation}. Lazy evaluation could eliminate unecessary operation because Agda is lazy to delay a computation until we need its result. It is often used to handle infinite data structures. \cite{wiki:Lazy_evaluation}

\end{itemize}

Agda also has some special functions in its interactive emacs interface beyond simple functional programming languages which enhance the ease and convenience of this language.

\begin{itemize}
\item \textit{Type Checker}. Type checker is an essential part of Agda. You can use to to type check a file without compiling it. It is the type checker that detect type mismatch problem and for theorem proving, it means the proof is incorrect. It interactively shows the goals, assumptions and variables when buiding a proof. 

The \emph{coverage checker} makes sure that the patterns cover all possible cases \cite{aboa}. 

The \emph{termination checker} will warn possiblily non-terminated error. The missing cases error will be reported by type checker. The suspected non-terminated definition can not be used by other ones. All programs must terminate in Agda so that it will not crash \cite{tutorial}.  The type checker then ensures that the proof is complete and not been proved by itself. 

In Agda, type signatures for functions are essential due to the presence of type checker (which is different to Haskell).
 
\item \textit{Interactive interface}. It has a Emacs-based interface for interactively writing and verifying proofs.  With type checker we can refine our proofs step by step \cite{aboa}. It also has some convenient functions and emacs means the potential to be extended.

\item \textit{Unicode support}. In Haskell and Coq, unicode support is not an essential part. However in Agda, to be a better theorem prover, it reads unicode symbols like: $\beta$, $\forall$ and $\exists$ and supports mixfix operators like: $+$ and $-$, which are very common for mathematics. It provides more meaningful names for types and lemmas and more flexible way to define operators. This also improve the readablity of the Agda proofs. For example, the commutativity of plus for natural numbers can be encoded as follows

\begin{code}
\>\AgdaFunction{comm} \AgdaSymbol{:} \AgdaSymbol{∀} \AgdaSymbol{(}\AgdaBound{a} \AgdaBound{b} \AgdaSymbol{:} \AgdaDatatype{ℕ}\AgdaSymbol{)} \AgdaSymbol{→} \AgdaBound{a} \AgdaFunction{+} \AgdaBound{b} \AgdaDatatype{≡} \AgdaBound{b} \AgdaFunction{+} \AgdaBound{a}\<%
\end{code}

We can use symbols we are familiar in regular mathematics.



Secondly we could use symbols to replace some common-used properties to simply the proofs a lot. The following code was simplied using several symbols,


Finally, we could use some other languages characters to define functions such as Chinese characters.

\item \textit{Code navigation}. As long as a program is loaded, it provides shortcut keys to move to the original definitions of certain object and move back. In real life programming it alleviates a great deal of work of programmers to look up the library.

\item \textit{Implicit arguments}. Sometime it is unnessary to write an argument since it can be inferred from other arguments by the type checker. It can simplify the application of functions and make the programs more concise. For example, to define a polymorphic function $\mathsf{id}$,

\begin{code}\>\<%
\\
\>\AgdaFunction{id} \AgdaSymbol{:} \AgdaSymbol{\{}\AgdaBound{A} \AgdaSymbol{:} \AgdaPrimitiveType{Set}\AgdaSymbol{\}} \AgdaSymbol{→} \AgdaBound{A} \AgdaSymbol{→} \AgdaBound{A}\<%
\\
\>\AgdaFunction{id} \AgdaBound{a} \AgdaSymbol{=} \AgdaBound{a}\<%
\end{code}

Whenever we give an argument $\mathsf{a}$,  its type $\mathsf{A}$ must be inferable.

\item \textit{Module system}. The mechanism of parametrised modules makes it possible to define generic operations and prove a whole set of generic properties.

\item \textit{Coinduction}. We can define coinductive types like streams in Agda which are typically infinite data structures. Coinductive occurences must be labelled with $\infty$ and coninductive types do not need to terminate but has to be productive. It is often used in conjunction with lazy evaluation. \cite{wiki:Coinduction}
	
\end{itemize} 

With these helpful features, Agda is a very powerful proof assisstant. It does not magically prove theorems for people, but it really helps mathematicians and computer scientists to do formalised reasoning with verification by high-performance computers. 


\subsection{Basic syntax}

Agda is a functional programming language and at the same time a theorem prover for mathematician. Its syntax has some similarities with Haskell but there are also many differences. In the other perspective, even though it is an implementation of \mltt, it adopts a syntax more closely to functional programming languages.

\begin{itemize}
\item Since Agda has a type checker, the programmer use typing judngement more often, so the designer decide to use single colon $\AgdaSymbol{:}$ for typing judgement, for example $\AgdaFunction{a} \AgdaSymbol{:} \AgdaDatatype{A}$ means that $\AgdaFunction{a}$ is of type $\AgdaDatatype{A}$, while double colons $::$ for the definition of the \emph{cons} constructor for list.

\item  The symbols for different equalities are contrary to the conventions in text. Like in some other programming languages e.g.\ Java Haskell, the equality symbol "$\AgdaSymbol{=}$" is reserved for function definition. Instead the cogruence symbol "$\AgdaDatatype{≡}$ is chosen for identity type which internalise propositional equality. 
% This is inconsistent with our conventional choices of symbols in articles, but it follows the conventions in Haskell and other programming languages that "$\AgdaSymbol{=}$" is used for definition.

\item Agda provides a more flexible way to define mixfix operators. With the unicode support, it is possible to define infix plus operator $\_+\_$, where the underscore marks the spaces for the explicit arguments in non-prefix operators. Underscores in expressions like $f ~ x ~ \_ z$ represent wildcards in function application for unnecessary  arguments (namely if can be inferred by type checker).

\item We use \textbf{data} to define constructors for inductive and coinductive datatypes. 

\item We have universe levels parameters in a lot of definitions which makes code looks unnecessarily cumbersome. 
The universe of small types is encoded as $\Set_{0}$ or $\Set$ rather than $\Type$, even though it is not a set in set-theoretical sense.
We will follow the \textbf{typical ambiguity} in this thesis which says that we write $\AgdaBound{A} \AgdaSymbol{:} \AgdaPrimitiveType{Set}$ for $\AgdaBound{A} \AgdaSymbol{:} \AgdaPrimitiveType{Set} \AgdaBound{a}$ and $\AgdaPrimitiveType{Set} \AgdaSymbol{:} \AgdaPrimitiveType{Set}$ which stands for $\AgdaPrimitiveType{Set}\AgdaBound{i} \AgdaSymbol{:} \AgdaPrimitiveType{Set}\AgdaBound{(i+1)}$.
The universe of propositions $\Prop$ ($\Prop \subset \Set$) does not exists in Agda because there is no proof-irrelevance in \itt as we will see later \ref{extensionality}. For ease of reading, we will use $\Set$ in replace of $\Prop$ in Agda code. Only necessary, we will explicitly add the proof-irrelevance property for a given proposition $P : \Set$, i.e.\ for all $p~ q : P$, $p = q$.


\item Agda has a more liberal way to define $\Pi$-types. They are often written as special cases of function types, for example $\Pi x : A. B$ can be written as $(x : A) \to B$. $\Sigma$-types are defined in Agda standard library. There is also a generalised $\Sigma$-types called \emph{dependent record type} which can be defined by keyword \textbf{record}.

\item For coinductive types and more generally mixed inductive/coinductive types \cite{txa:mpc2010g}, we adopt a set of operators which are defined in module \textbf{Coinduction}. A infinite list (or stream) can be defined as:

\begin{code}
\>\AgdaKeyword{data} \AgdaDatatype{Stream} \AgdaSymbol{(}\AgdaBound{A} \AgdaSymbol{:} \AgdaPrimitiveType{Set}\AgdaSymbol{)} \AgdaSymbol{:} \AgdaPrimitiveType{Set} \AgdaKeyword{where}\<%
\\
\>[0]\AgdaIndent{2}{}\<[2]%
\>[2]\AgdaInductiveConstructor{\_∷\_} \AgdaSymbol{:} \AgdaBound{A} \AgdaSymbol{→} \AgdaDatatype{∞} \AgdaSymbol{(}\AgdaDatatype{Stream} \AgdaBound{A}\AgdaSymbol{)} \AgdaSymbol{→} \AgdaDatatype{Stream} \AgdaBound{A}\<%
\end{code}

The delay operator $\infty$ denotes an coinductive argument. Given $a:A$, the expression with a delay function $\sharp~a$ is a element of type $\infty~A$. $\beta~x$ will force computation in $x : \infty~A$.

\item If type checker can infer the type for some arguments, we can use implicit arguments which are indicated by curly brackets.
For example a function $f:\{x:A\} \to B$ allows us to omit unnecessary argument which makes the code more readable.




\end{itemize}



\subsection{Identity Type}

Identity type is the type introduced by Martin-L\"{o}f to encode the propositional equality for definitionally equal terms \cite{nor:90}. For any two terms of $\mathsf{a\,b : A}$, we have the type $\mathsf{Id (A , a , b)}$ which is inhabitted when $\mathsf{a}$ and $\mathsf{b}$ are definitionally equal. Here we use an alternative equivalent version named after Paulin-Mohring which is parameterized with the left side of the identity. %This also includes the identity type for arbitrary universe level.

\begin{code}%
\\
\>\AgdaKeyword{data} \AgdaDatatype{\_≡\_} \AgdaSymbol{\{}\AgdaBound{A} \AgdaSymbol{:} \AgdaPrimitiveType{Set}\AgdaSymbol{\}} \AgdaSymbol{(}\AgdaBound{x} \AgdaSymbol{:} \AgdaBound{A}\AgdaSymbol{)} \AgdaSymbol{:} \AgdaBound{A} \AgdaSymbol{→} \AgdaPrimitiveType{Set} \AgdaKeyword{where}\<%
\\
\>[0]\AgdaIndent{2}{}\<[2]%
\>[2]\AgdaInductiveConstructor{refl} \AgdaSymbol{:} \AgdaBound{x} \AgdaDatatype{≡} \AgdaBound{x}\<%
\\
%
\end{code}

In Agda eliminators are not automatically derived for the types defined. Instead we have pattern matching generally which is sometimes stronger than eliminators.
As long as we pattern match on a variable of an identity type with the unique inhabitant $\AgdaInductiveConstructor{refl}$, all occurences of both variables become the same.
It is stronger and it provides the eliminator J.

\begin{code}
%
\\
\>\AgdaFunction{J} \AgdaSymbol{:} \AgdaSymbol{(}\AgdaBound{A} \AgdaSymbol{:} \AgdaPrimitiveType{Set}\AgdaSymbol{)(}\AgdaBound{a} \AgdaSymbol{:} \AgdaBound{A}\AgdaSymbol{)} \AgdaSymbol{→} \AgdaSymbol{(}\AgdaBound{P} \AgdaSymbol{:} \AgdaSymbol{(}\AgdaBound{b} \AgdaSymbol{:} \AgdaBound{A}\AgdaSymbol{)} \AgdaSymbol{→} \AgdaBound{a} \AgdaDatatype{≡} \AgdaBound{b} \AgdaSymbol{→} \AgdaPrimitiveType{Set}\AgdaSymbol{)}\<%
\\
\>[0]\AgdaIndent{2}{}\<[2]%
\>[2]\AgdaSymbol{→} \AgdaBound{P} \AgdaBound{a} \AgdaInductiveConstructor{refl}\<%
\\
\>[0]\AgdaIndent{2}{}\<[2]%
\>[2]\AgdaSymbol{→} \AgdaSymbol{(}\AgdaBound{b} \AgdaSymbol{:} \AgdaBound{A}\AgdaSymbol{)(}\AgdaBound{p} \AgdaSymbol{:} \AgdaBound{a} \AgdaDatatype{≡} \AgdaBound{b}\AgdaSymbol{)} \AgdaSymbol{→} \AgdaBound{P} \AgdaBound{b} \AgdaBound{p}\<%
\\
\>\AgdaFunction{J} \AgdaBound{A} \AgdaSymbol{.}\AgdaBound{b} \AgdaBound{P} \AgdaBound{m} \AgdaBound{b} \AgdaInductiveConstructor{refl} \AgdaSymbol{=} \AgdaBound{m}\<%
\\
%
\end{code}

\subsection{Extensionality}
\label{extensionality}

In regular mathematics, equality does not only exists between intensionally equal terms. Objects are also equal if they have the same extensional external properties, like functions and propositions. 

As Martin Hofmann summarises in \cite{hof:phd}, there are several important extensional concepts which we will also encounter in this thesis, \emph{Functional extensionality}, \emph{Uniqueness of identity}, \emph{Proof-irrelevance}, \emph{Propositional extensionality}, \emph{Quotient types}. 
These principles are not currently available in \itt, but they are valid extensions of Type Theory and worth interpreting to help both Mathematics and programs constructions. Intuitively speaking, the extensionally equal terms can be distingushed by any other terms, hence the extensionality is justifiable. 

\begin{itemize}
\item \textit{Functional extensionality} It has been introduced in \ref{fun-ext}.

\item \textit{Uniqueness of identity} In type theory we define a notion of \emph{set} as follows:

\begin{definition}\label{UIP}
A type $A$ is a \textbf{set} if for all $x,y:A$ and all $p,q:x=y$, we have $p=q$.
\end{definition}

This property is usually called uniqueness of identity (UIP). From the usual definition of identity types, UIP is not a result for every type. UIP is also equivalent to Streicher’s “Axiom K”.

\begin{axiom}[K]
For all $x:A$ and $p: x=x$ we have $p=\text{refl}_{x}$.
\end{axiom}

Another characterisation of a set in type theory is given by Hedberg's Theorem.
\begin{theorem}[Hedberg]
If $A$ has decidable equality, then $A$ is a set.
\end{theorem}

\item \textit{Proof-irrelevance} All proofs of the same proposition are propositionally equal.

\item \textit{Propositional extensionality} Two logically equivalent propositions are propositionally equal.

\item \textit{Quotient types} A quotient type is a type formed by redefing equality on a underlying type with a given equivalence relation on it. 

\item \textit{Univalence Axiom}
In \hott, univalence is a extensional principle which states that ismorphic types are propositionally equal.
\end{itemize}

It is interesting to extend Type Theory with these extensional principles. However, it only makes sense if the type-checking decidability and terms canonicity are not sacrificed. This thesis mainly focus on the extension of quotient types.


\section{Homotopy Type Theory}

\hott is a variant of intensional \mltt{} which is a new branch developed between theoretical computer science
and mathematics. Vladimir Voevodsky found a surprising connection between homotopy theory and type theory \cite{voe:06}. He proposed the univalence axiom, which identifies isomorphic structures, as a univalent foundation for mathematics. 
In \hott, there is an observation that notions in type theory can be interpreted by homotopy-theoretical terms. A type is regarded as a \emph{space} and a term of this type is a \emph{point} of this space. Functions between types are \emph{continous maps} and identity types are usually considered as \emph{paths}. Identity types of identity types are \emph{homotopies}. Although these notions are originally defined with topological bases, we only employ them as homotopical notions on a higher level. 


As univalence axiom states, equality is equivalent to equivalence. Acutally it can be seen as an formal acceptance of the ''common sense'' in Mathematics that isomorphic structures can be identitfied. The higher structures of the equivalence also allows us to study the different ways of identification. Therefore it is more appropriate to interpret types as higher groupoids. People is trying to implement \hott in \itt and one possible way is to interpret \wog first in Agda. The author has done some work in this direction which can be found in Chapter~\ref{wog}.


I will not explain this topic in detail here, but in \autoref{HITs}
we give a more detailed introduction to it.



 % Something without a type makes no sense to us because we are not sure what it stands for and how do we use it. The type definition describe the syntax so that some symbol makes sense and the semantic meaning may be revealed from the construction.

% There is another question, whether mathematics is a collection of patterns and laws which is observed, or it is a system created and built by people to explain the patterns and laws in the world. I think people prefer the second answer usually accept the type theory more easily, although most people (probably 99.9 percent) prefer the first one. When we learn what is natural numbers, we learn it as "numbers like 1, 2, 3 ,4 and perhaps 0", the commutative law, associate law are axioms because there is no way to prove it if we introduce it in this manner. We are convinced by some examples like "2 + 3 = 3 + 2" and we find it works for most of the cases then we accept it by observations. It is some methods physicians used a lot -- to conclude some laws from a number of facts. It is a proper method for physicians because what they research on is world can only be observed. However for mathematics, even though it is applied to the real world, it is a system completely created by people. People used their fingers to count, wrote symbols for results, even though it was very shallow it is obviously a aritificial system. People extend 


% Type theory is strongly connected with computation theory.
% Set


% Type theory has fewer axioms, simpler model than set theory which has mutual foudations: logic and axioms.

\section{Summary}


In mathematics, set theory is still a more popular choice over type theory. However in computer science, instuitionistic type theory is worth more studying. It is more close to program construction and from a computer scientist's point of view, it is very natural to accept intuitionistic logic. 

It provides a foundation of mathematics which can be implemented as a programming language so that proving is just programming and verification is just type checking. The aid of computers saves a lot of work from mathematicians and reduces the chance of making mistakes, although the absence of \emph{principle of excluded middle} in intuitionistic logic makes some mathematicians hard to accept. There is a very good talk given by Andrej Bauer in IAS called "Five Stages of Accepting Constructive Mathematics'' online \footnote{available on Youtube}.






%\input{Chapters/Extensionality}

\chapter{Quotient Types}
\label{qt}
% Quotient type or quotient types?


%Quotient type is a type generated by redefining equality on a given type. 

%Technically we cannot redifine equality

 %Generally speaking, given a setoid $A$ and an 
 %equivalence relation $\sim$ on it, a quotient type denoted as
%$\qset{A}$ is a type generated by redefining equality of $A$ by
%$\sim$. In \itt, we usually use setoids to represent quotients but it
%is not satisfactory. It is a very important topic to extend \itt with
%quotient types.

%From a programmming persepective, there are more tools to define data
%types. 
%In the chapter, we first introduce the quotients in Mathematics and
%then quotient types in type theory with categorically interpretations.


%\section{Quotients in Mathematics}

% "M"athematics must be capital


%structures  used also extended to other branches, such as set
%theory, group theory, topology etc.

%we usually use the same operators for abstractly similar 
%operations. For instance, the product is extended to the cartesian product in set theory.


Quotient is one of the primitive notions in \maths. 
In arithmetic, quotient is the result of division

$$8 \div 4 = 2 ~~ \text{or}~~ 8/4 = 2$$


In abstract algebra quotient is the result of ``dividing'' a set, group, space or another algebraic structure (dividend) with respect to an given equivalence relation on it. We use quotient set as an example here.

\begin{definition}
\textbf{Equivalence relation}.
An equivalence relation is a binary
relation which is reflexive, symmetric and transitive.
\end{definition}

Intuitively, given any equivalence relation, a set can be partitioned into
some equivalence classes,

\begin{definition}
\textbf{Equivalence class}.
The equivalence class of an element $a$ is a set whose elements are
all equivalent to $a$
\begin{equation}
\class a = \{x : A \;| \; a \sim x \}
\end{equation}
\end{definition}

The collection of these equivalence classes is called
the quotient set.

\begin{definition}
\textbf{Quotient set}.
Given a set A equipped with an equivalence relation $\sim$, a quotient
set is denoted as $\qset{A}$,
\begin{equation}
\qset{A} = \{ [ a ] \;|\; a : A \}
\end{equation}
\end{definition}

Similarly in topology, group theory or other branches of \maths there are also quotient space, quotient group and other quotients. 

\section{Quotients in Type Theory}

Naturally one would also expect quotient types (or quotients) in Type Theory. Intuitively speaking, a \emph{quotient type} $\qset{A}$ is a type $A$ whose equality is redefined by an equivalence relation $\sim : A \to A \to \Prop$. Unlike \ett, there is no quotient operator in traditional \itt, but setoids are usually used to represent quotients.

\begin{definition}
\textbf{Setoid}.
\noindent A setoid $(A,\sim,\text{eqv}~\sim)$ (usually written as $(A,\sim)$) consists of
\begin{enumerate}
\item a set (type) $A : \Set$,
\item a relation $\sim : A \to A \to \Prop$, and
\item it is an equivalence, i.e.\ the proofs that it is reflexive, symmetric and transitive.
\end{enumerate}
\end{definition}

However it is not an ideal solution. Since it is an alternative representation of sets, everything defined on $\Set$ has to be redefined on $\Setoid$ again. 
For example, functions on setoids, equalities on setoids,
products on setoids etc. Moreover, considering a setoid $(A,\sim)$ whose $A$ is also encoded as a setoid, 
it is essential to define higher setoids.
From a programming perspective, setoids are also
unsafe because we have access to the underlying
 sets \cite{aan}. We can perform operations which do not respect
 equivalence relation and the result does not make any sense.
Therefore, it is essential to use a quotient of type $\Set$ which can be uniformly treated with other types. It is also the case in various branches of mathematics that the quotient is the same kind of object as the base one.

%As long as we define a set as a type in Type Theory, for example the set of natural numbers $\N$, and an equivalence relation on it, we would like to have a quotient immediately. 

%\footnote{See http://www.cs.cornell.edu/home/sfa/Nuprl/NuprlPrimitives/Xquotient_doc.html}


%For quotient algebraic structures in abstract algebra, usually the quotients are same kind of object as the base one. Hence in type theory it is also expected that the type representing a quotient set is of sort $\Set$ rather that $\Setoid$. 


\subsection{Rules for quotients}\label{iqs}

The quotient types are defined by the following rules as described in \cite{Jacobs94quotientsin,hof:95:sm}. 


\infrule[Q-\bf{Form}]
{ \Gamma \vdash A  \andalso \Gamma ,x : A , y : A \vdash x \sim y : \bf{Prop} }
{\Gamma \vdash \qset{A}}

Given a type $A$ and with a binary relation $\sim$ on $A$, we can form the quotient $\qset{A}$.
Notice that $\sim$ is not required to be only equivalence relation. It is because we can prove that $\qset{A}$ is equivalent to $\qsetc{A}$ where $\sim_{C}$ is the least equivalence relation containing $\sim$ (see \autoref{equivalencerelationiso}). However usually we would use an equivalence relation which makes more sense and we assume our quotients is derived from an equivalence relation if not specially pointed out in this thesis.


\begin{multicols}{2}
\infrule[Q-\bf{Intro}]{\Gamma \vdash a : A}{\Gamma \vdash [ a ] : \qset{A}}
\columnbreak
\infrule[Q-\bf{Ax}]
{\Gamma \vdash a , b :  A  \andalso  \Gamma \vdash  p : a \sim b}
{ \Gamma \vdash \text{Qax} ~p : [a]=_{\qset{A}} [b]}
\end{multicols}


We introduce an ``equivalence class'' for each element of $A$. It is usually denoted as $[ a ]$, or $[ a ]_{\sim}$ for $\sim$ if it is unclear which relation it refers to. Qax states that the ``equivalence classes'' of two terms which are related by $\sim$ are (propositionally) equal.


According to Hofmann's \cite{hof:95:sm} definition, it comes with an eliminator with its computation rule and an induction principle:


\infrule[Q-\bf{elim}]
{\Gamma \vdash  B :  \Set \andalso \Gamma , x : A \vdash f ~ a : B \\
\Gamma, a: A, b : A, p : a \sim b \vdash  f^{=}~a~b~p : f ~a =  f ~ b \andalso \Gamma \vdash  q : \qset{A}}
{\Gamma \vdash  \hat{f} ~ q : B}

\infrule[Q-\bf{comp}]{\Gamma \vdash  a : A}{\Gamma \vdash  \text{Qcomp} ~a  : \hat{f} ~ [ a ] \equiv f~ a }


\infrule[Q-\bf{ind}]
{\Gamma,  x : \qset{A} \vdash P ~ x : \Prop \andalso \Gamma, a : A \vdash h ~ a : P ~ [ a ] \andalso \Gamma \vdash  q : \qset{A}}
{\Gamma \vdash \text{Qind} ~h ~q :P~q}

Given a function $f : A \to B$ which respects $\sim$, we can lift it to be a function on $\qset{A}$ as $\hat{f} : \qset{A} \to B$ such that for any element $a : A$, $\hat{f} ~[ a ]$ computes to the same value as $f ~ a$. It allows us to define functions on quotient types by functions on base types (representatives).
Notice that \emph{function application} is written like "$f ~x ~ y$'' where $x$ and $y$ are two arguments for function $f$. We also omit $f^=$ since the computation rule already implies that it is proof-irrelevant.

The induction principle states that for any proposition $P : \qset{A} \to \Prop$. it is enough to just consider cases $ P ~ [ a ]$ for all $a : A$. In other words, it $\qset{A}$ only consists of "equivalence classes" i.e.\ $[ a ]$.


Alternatively, a \emph{dependent} eliminator (dependent lifting) serves the same purpose:

\infrule[Q-\bf{dep-elim}]
{\Gamma, x : \qset{A} \vdash B ~ x : \Set \andalso \Gamma , a : A \vdash f ~ a : B ~ [ a ] \\
\Gamma, a : A, b : A, p : a \sim b \vdash f^= ~a ~b~ p : f ~a
\stackrel{p}{=}  f ~b \andalso \Gamma \vdash q :
\qset{A}}
{\Gamma \vdash \hat{f} ~q : B~ q}

\infrule[Q-\bf{dep-comp}]{\Gamma \vdash a : A}
{\Gamma \vdash \text{Qdcomp} ~a  : \hat{f} ~ [ a ] \equiv f ~ a }


Notice that $\stackrel{p}{=}$ is an abbreviation for propositional equality which requires substitution in the type of the left hand side by $\text{Qax}~p$ so that both sides have the same type.

\begin{proposition}
The non-dependent eliminator with the induction principle is equivalent to the dependent eliminator.
\end{proposition}
\begin{proof}
1. Assume we have non-dependent eliminator and the induction principle, $B$ is an dependent type on $\qset{A}$, $f$ is a function of type $(a : A) \to B ~ [ a ]$ and it respects $\sim$ under substitution ($f^=$), $q$ is an element of $\qset{A}$.

Set $B'$ as a dependent product as $\Sigma (r : \qset{A}) ~B ~r$,

An independent version of $f : A \to B'$ can be defined as

$$f' a \defeq [ a ] , f ~a$$

Given $p : a \sim b$, we can conclude that $f' a =_{B'} f' b$ is inhabited from Qax and $f^=$.

It allows us to lift the non-dependent function $f'$ as $\hat{f'}$ such that 

\begin{equation}\label{f'comp}
\hat{f'} [ a ] \equiv [ a ] , f ~a
\end{equation}
Applying first projection on both sides, the following propositional equality is inhabited:

 $$\text{proj}_1 (\hat{f'} ~[ a ]) = [ a ]$$

By induction principle, the predicate $P : \qset{A} \to \Prop$ defined as

$$P ~ q \defeq \text{proj}_1 ~(\hat{f'} ~q) = q$$

is inhabited.

Finally, to complete the dependent eliminator, we can construct an element of type $B~q$ by

$$\text{proj}_2 ~(\hat{f'} ~q)$$

which has the correct type because of predicate $P$. The computation rule is derivable from \ref{f'comp}.

2. It is easy to find out that the non-dependent eliminator and induction principle are just special cases of dependent eliminator.

A constructive proof in Agda can be found in \autoref{app:dq}.
\end{proof}


Additionally, a quotient is effective (or exact) if an "equivalence class" only contains terms that are related by $\sim$.

\infrule[Q-\bf{effective}]
{\Gamma \vdash a :  A \andalso \Gamma \vdash b :  A  \andalso p : [a] = [b] }
{\text{Qexact}~{p} : a \sim b}

\begin{proposition}
With propositional extensionality, we can prove that all quotients\footnote{we assume $\sim$ is an equivalence relation} are effective.
\end{proposition}

\begin{proof}\label{PUEF}
Suppose we have a set $A$ with an equivalence relation $\sim : A \to A
\to \Prop$, a quotient set is $\qset{A}$.

Given $a : A$, and a predicate $P_a : A \to \Prop$ defined as 
$$P_a ~ x \defeq a \sim x$$

To lift it we have to check $P_a$ is compatible with $\sim$.

Suppose $x \sim y$

by symmetry and transitivity

$\Rightarrow a \sim x \iff a \sim y$

$\equiv P_a~x \iff P_a~y$

by propositional extensionality

$\Rightarrow P_a~x = P_a~y$


To prove the quotient $\qset{A}$ is effective, suppose $[ a ] = [ b ]$, we can simply prove $ \hat{P} [ a ] = \hat{P} [ b ]$ and then

$$a \sim a \equiv \hat{P} [ a ] = \hat{P} [ b ] \equiv a \sim b$$

Finally with eliminator J and $\text{refl} : a \sim a$ we can easily prove

$$a \sim b$$.

\end{proof}

%Alternatively, we can prove it as follows:

%\begin{proof}

%Firstly we prove the equivalence relation is well-defined on the
%quotient types, namely it respects the equivalence relation:

%Suppose we have $a \sim b$ and $c \sim d$, we can deduce $a \sim c \iff
%b \sim d$. Then applying the propositional univalence axiom, we know
%that $a \sim c = b \sim d$, hence the equivalence relation is
%well-defined.

%Because it is well-defined, we can lift it such that

%$[ a ] ~\hat{\sim}~ [ b ] \equiv a \sim b$


%From reflexivity of the equivalence relation, $\forall x : A, x \sim x$, 
%we know that $\forall x : A, [x]~\hat{\sim}~[x]$.

%Assume $[a]=[b]$, using J-eliminator in $[a]~\hat{\sim}~[a]$
%(reflexivity), $[a]~\hat{\sim}~[b]$ which is definitionally equal to $a \sim b$, Hence the quotient is
%effective.
%\end{proof}


\subsection{An impredicative encoding of quotient types}\label{impredicative}

A natural idea is to imitate the set theoretical construction. We would like to define an equivalence class first.

An equivalence class of $a$ can be defined as a tuple

$$[ a ] \defeq \Sigma (x : A) , x \sim a $$

However, given $a ~ b$, $[ a ]$ is not propositionally equal to $[ b ]$. The reason is that $x \sim a$ is not propositional equal to $x \sim b$ even though they are logically equivalent due to the lack of propositional extensionality.


\begin{definition}
\textbf{propositional extensionality} (propositional univalence).

$\forall P, Q : \Prop, (P \iff Q) \iff P = Q.$
\end{definition}

Vladimir Voevodsky introduces an impredicative definition of quotients
%\footnote{\url{http://www.cse.chalmers.se/~coquand/cirm.pdf}} 
which has been encoded in Coq
\cite{voe:hset}. 

Given a setoid $(A,\sim)$, an equivalence class is a subtype given by a predicate $P : A \to \Prop$.

$\text{EqClass} ~ P \defeq (\exists a : A, P ~ a) \times \forall x ~ y : A, P x \to (x \sim y \iff P ~ y)$


The \textbf{set quotient} is then defined as

$\qset{A} \defeq \Sigma P : A \to \Prop, \text{EqClass} ~P$

Notice that it is impossible to extract an element of $A$ from an proof of $\text{EqClass} ~ P$ because the encoding of $\exists a : A, P ~ a$ is impredicative as truncated $Sigma$-type: $\| \Sigma a : A, P ~ a \|$.
The truncation $\|-\|$ is defined impredicatively as

$\|A\| \defeq \forall P : \Prop \to (A \to P) \to P$

$\|A\|$ is in the universe of $\Set_1$($\mathsf{U}_1$ in some articles) and with resizing rules, $\|A\|$
is moved to the universe of $\Set$ (or $\mathsf{U}$). With resizing rules,
$\qset{A}$ is then in the universe of $\Set$ as expected.

% \footnote{\url{http://www.math.ias.edu/~vladimir/Site3/Univalent_Foundations_files/2011_Bergen.pdf}}

There is a function $[\_] : A \to \qset{A}$ which is compatible with $\sim$ and
we can lift any function $f : A \to B$ compatible with $\sim$ such that $\hat{f} [ a ] \equiv f a$

Assume $x \sim y$, $[ x ]$ and $[ y ]$ have to be propositionally equal which can be proved with propositional extensionality.
There is an alternative type-theoretical encoding which is sometimes called intensional quotient set.


%with the proposional extensionality we can prove that
%\begin{theorem}\label{thm-p-e}
%Given $(P , prf) : \qset{A}$, all proofs of $\exists x : A, P (x)$ are equal
%\end{theorem}

%\begin{proof}
%Given any two proofs of $\exists x : A, P (x)$ written as $(x , px)$ and $(y
%, py)$, apply the $EqClass(P)$ to $(x, y, px, py)$ we know that $x
%\sim y$. Hence the truncation of 
%\end{proof}

%Given a element $a : A$, the equivalence class is


%It is called impredicative because it uses resizing rules \footnote{\url{http://phdsinlogic2014.wp.hum.uu.nl/files/2014/04/thierry-coquand1.pdf}} in the encoding of existential proof.

%$|| A || = (P : \Prop)(A \to P) \to P$




\section{Coequalizers?}

\begin{remark}\label{equivalencerelationiso}

\end{remark}



\subsection{Quotient types are coequalizers}

The rules of quotient set is indeed characterised in category-theoretical way.
As Bart pointed out, quotients can be described as a left adjoint to an equality functor.


Categorically speaking, a quotient is a coequalizer.

\begin{definition}
\textbf{Coequalizer}.
Given two objects $X$ and $Y$ and two parallel morphisms $f, g : \morph{X}{Y}$ , a coequalizer is an object Q with a morphism $q : \morph{Y}{Q}$ such that $q \circ f = q \circ g$. It has to be universal as well. Any pair (Q' , q') $q' \circ f = q' \circ g$ has a unique factorisation u such that $q' = u \circ q$
\begin{displaymath}
    \xymatrix{X \ar@<0.5ex>[rr]^f \ar@<-0.5ex>[rr]_g && Y \ar[rr]^q
      \ar[ddrr]_{q'} && Q
      \ar@{.>}[dd]^u \\ \\
& &&& Q' }
\end{displaymath}
\end{definition}

A quotient is the coequalizer when we have two projections $\pi_1$ and
$\pi_2$ from the relation $R = \{(a_1,a_2) : A \times A ~|~ a_1 \sim a_2\}$
\begin{displaymath}
    \xymatrix{R \ar@<0.5ex>[rr]^{\pi_1} \ar@<-0.5ex>[rr]_{\pi_2} && A \ar[rr]^{ [\_]}
      \ar[ddrr]_{f} && Q
      \ar@{.>}[dd]^{\hat{f}} \\ \\
& &&& B }
\end{displaymath}


The exactness corresponds to effectiveness of coequalizer,

\[\xymatrix{
R\pullbackcorner\ar[r]^\pi_1\ar[d]_\pi_2 & A\ar[d]^{[\_]} \\
A\ar[r]_{[\_]} & Q
}\]


A quotient with a right inverse "emb" for $[\_]$ corresponds to \emph{split coequalizer} which is a fork 

\[\xymatrix{
R\ar@<0.5ex>[r]^{\pi_0}\ar@<-0.5ex>[r]_{\pi_1}& A\ar[r]^{[\_]}
& Q
}\]

with morphisms $\emb : Q \to A$ and $t : A \to R$ such that 

$[\_] \circ emb = 1_Q$

$\emb \circ [\_]  = \pi_2 \circ t$ and 

$\pi_1 \circ t = 1_A$ (i.e.\ $t ~ a = (\emb [ a ] , a)$)



\subsection{Quotient types is a left adjoint functor}

Recalling the definition of adjunction.

\begin{definition}
\text{Adjunction}.
Given two categories $A$ $B$, a functor $F : A \to B$ is left adjoint
to $G : B \to A$ if we have a natural isomorphism
$\Omega : hom_{B}(F ~\_ , \_) \to hom_{A}(\_, G ~\_)$
\end{definition}


Quotient can also be seen as a functor $\textbf{Q} : \textbf{Setoids} \to \textbf{Sets}$ which is
left-adjoint to embedding functor $\nabla$
where

$\textbf{Q} ~ (B , \sim) \defeq \qset{B}$, and

$\nabla A \defeq (A , =)$


The adjunction above can be given by following natural isomorphism

\begin{equation*}
\begin{aligned}
\qset{B} & \to A \\
\midrule
\midrule
(B , \sim) & \to (A , =)
\end{aligned}
\end{equation*}

To prove that this is an isomorphism, assume $f : \qset{B} \to A$, there exists a mapping

$\Omega ~ f = f \circ [\_]$ and its inverse

$\Omega^{-1} ~ g = \hat{g}$ (we omit the property that $g$ respects $\sim$)

The isomorphic properties can be verified as follows,

$\Omega^{-1} (\Omega~f) = \dlift{f \circ [\_]} = f$ by the uniqueness from lifting.

$\Omega (\Omega^{-1}~g) = \hat{g} \circ [\_] = g$ by definition of lifting.


\subsection{Quotient inductive types}

As we have seen in \autoref{hott:ext}, quotient inductive types is an alternative way to define quotients in \hott. In fact, some examples suggests that QITs are more powerful than quotient types.

One of the example is the definition of reals which will be discussed in \autoref{rl}. Our construction of reals by Cauchy sequences of rational numbers is not Cauchy complete because there is no limit of each equivalence class. However, the Cauchy approximation approach in \cite{hott} using quotient inductive types is Cauchy complete due to the fact that the equivalence relation and limits are included in its definition.

Another example is unordered trees (rooted tree). An unordered tree is a tree connected to a multiset of rooted trees, hence there is no ordering on subtrees.

Firstly we define ordered trees as:

\begin{itemize}
\item A leaf $l: \mathsf{Tree}$, or
\item An ordered list of subtrees indexed by $\mathbb{N}$, $st : (\mathbb{N} \rightarrow \mathsf{Tree}) \rightarrow \mathsf{Tree}$,
\end{itemize}

With the following equivalence relation:

\begin{itemize}
\item $l_{eq} : l \sim l$,
\item $st_{eq} : (f , g : \N \to \mathsf{Tree}) \to f \sim_{p} g \to sp~f \sim sp~g$
\end{itemize}

where $f \sim_{p} g$ stands for $f$ is a permutation of $g$. The permutation can be defined using a bijective map $p : \N \to \N$ which relates equivalent subtrees recursively.

%However, if we use quotient type $\mathsf{Tree^{\sim}} := \qset{\mathsf{Tree}}$, the resulting trees have unordered subtrees which are themselves ordered.

If we define unordered trees as a quotient type $\mathsf{Tree^{\sim}} := \qset{\mathsf{Tree}}$, 
there is some problem of lifting the $st$. 
For finitely branching trees like binary trees,$st$ can be lifted by nesting lifting functions,

$$\overline{st}~a~b = \text{lift}~(\text{lift}~st~a)~b$$

because its type is isomorphic to $\mathsf{BTree} \rightarrow \mathsf{BTree} \rightarrow \mathsf{BTree}$.
Intuitively we can apply this approach to trees with finite subtrees. However it fails if have infinite subtrees.

Quotient inductive types does not have such problems:

\begin{itemize}
\item $l: \mathsf{Tree}$, 
\item $st : (\mathbb{N} \rightarrow \mathsf{Tree}) \rightarrow \mathsf{Tree}$,
  and
\item a set of paths relates two permuted trees:

$l_{eq} : l  =_{\mathsf{Tree}} l $

$st_{eq} : \forall (f, g : \N \to \mathsf{Tree}) \rightarrow
f \sim_{p} g \rightarrow  st~f =_{\mathsf{Tree}} st~g$
\end{itemize}

The cumulative hierarchy of all sets introduced in \cite{hott} also suggests that quotient types have some weaknesses compared to quotient inductive types.

A cumulative hierarchy can be given by constructors,

$\{\_\} : (I : \Set) \to (I \to M_0) \to M_0$

along with a subset relation,

$\_\in\_ : M_0 \to M_0 \to \Prop$

which is inhabited if $ f(i) \in \{ I , f \}$

Then we can easily define the equivalence relation using set-theoretical definition

$A \sim B \defeq \forall m : M_0, m \in A \iff m \in B$

Similar to unordered trees, we can not obtain the constructor $\overline{\{\_\}}$ because the index set $I$ can be infinite.


In the above examples, it seems that quotient inductive types are more powerful than quotient types due to the ability of defining term constructors and equivalence relations simultaneously. 
However, quotient inductive types are not available in type theories other than \hott and the computational interpretation of it is still an open problem.

%However, it is still an open problem to find a computational interpretation of higher inductive types.

\section{Literature review}

%\todo{rewrite}

\begin{itemize}

\item In \cite{cab}, Mendler et al. have firstly considered building new types from a
given type using a quotient operator $//$. Their work is done in an
implementation of \ett, NuPRL. 
In NuPRL, every type comes with its own equality relation, so the quotient operator can be
seen as a way of redefining equality in a type. But it is not all
about building new types. They also discuss problems that arise from
defining functions on the new type which can be illustrated using a simple example. 

Assume the base type is $A$ and the new equivalence relation is $E$, the new
type can be formed as $A//E$. 

When we want to define a function $f \,\colon\, A//E \to Bool$,  $f\,a \not= f\,b$ may
exists for $a, b \,\colon A$ such that $E\,a\,b$. This will lead to
inconsistency since $E\,a\,b$ implies $a$ converts to $b$ in \ett{}, hence
the left hand side $f\,a$ can be converted to $f\,b$, namely we get $f\,b \not= f\,b$
which is contradicted with the equality reflection rule. 

Therefore a function is said to be well-defined \cite{cab} on the new type only
if it respects the equivalence relation $E$, namely

$$\forall \, a\,b\,\colon A, E\,a\,b \to f\,a = f\,b$$

We call this \emph{soundness} property in \cite{aan}.

 After the introduction of quotient types, Mendler further investigates
 this topic from a categorical perspective in ~\cite{men:90}. He uses
 the correspondence between quotient types in \mltt{} and coequalizers
 in a category of types to define a notion called \emph{squash types},
 which is further discussed by Nogin \cite{nog:02}.

\item To add quotient types to \mltt, Hofmann proposes three models for
quotient types in his PhD thesis \cite{hof:phd}. The first one is a setoid model for
quotient types. In this model all types are attached with partial
equivalence relations, namely all types are setoids rather than
sets. Types without a specific equivalence relation can be seen as
setoids with the basic intensional equality. This is similar to
\ett in some sense. The second one is groupoid model which solves some problems
but it is not definable in \itt. He also proposes a third model to
combine the advantages of the first two models, but it also has some
disadvantages. Later in \cite{hof:95:sm} he gives a simple model in which we have type dependency only at the propositional level, he also shows that \ett is conservative over \itt extended with quotient types \cite{hof:95:con}.

\item Nogin \cite{nog:02} considers a modular approach to axiomatizing the
same quotient types in NuPRL as well. Despite the ease of constructing new types
from base types, he also discusses some
problems about quotient types. For example, since the equality is
extensional, we cannot recover the
witness of the equality.  He suggests including more axioms to
conceptualise quotients. He decomposes the formalisation of quotient type
into several smaller primitives such that they can be handled much
simpler.

\item Homeier \cite{hom} axiomatises quotient types in Higher Order Logic
(HOL), which is also a theorem prover. He creates a tool package to
construct quotient types as a conservative extension of HOL such that
users are able to define new types in HOL. Next he defines the
normalisation functions and proves several properties of
these. Finally he discussed the issues when quotienting on the
aggregate types such as lists and pairs.


\item Courtieu \cite{cou:01} shows an extension of Calculus of Inductive Constructions
with \emph{Normalised Types} which are similar to quotient types, but equivalence relations are replaced by normalisation functions. 
However not all quotient types have normal forms. Normalised types are
proper subsets of quotient types, because we can easily recover a quotient
type from a normalised type as below
%$$ \[ (A, Q, \class\dotph \colon A \to Q) \to (A, \lambda \,a \,b \to \class a = \class b)\]$$


\item Barthe and Geuvers \cite{bar:96} also propose a new notion called
\emph{congruence types}, which is also a special class of quotient
types, in which the base type are inductively defined and with a set
of reduction rules called the term-rewriting system. The idea behind
it is the $\beta$-equivalence is replaced by a set of
$\beta$-conversion rules. Congruence types can be treated as an
alternative to the pattern matching introduced in \cite{coq:92}. The main
purpose of introducing congruence types is to solve problems in
term rewriting systems rather than to implement quotient types.


\item Barthe and Capretta \cite{bar:03} compare different ways to setoids in Type Theory.
The setoid is classified as partial setoid or total setoid depending
on whether the equality relation is reflexive or not. They also
consider obtain quotients with different kinds of setoids, especially
the ones from partial setoids are difficult to define because the lack
of reflexivity.

\item Abbott, Altenkirch et al. \cite{abb:04} provides the basis for
programming with quotient datatypes polymorphically based on their
works on containers which are datatypes whose instances are
collections of objects, such as arrays, trees and so on. Generalising
the notion of container, they define quotient containers as the
containers quotiented by a collection of isomorphisms on the positions
within the containers.

\item Voevodsky \cite{voe:hset} implements quotients in Coq based on a set
of axioms of \hott. He firstly implement
equivalence class and use it to implement quotients which is an
analogy to the construction of quotient sets in set theory. The detail has been given in \autoref{impredicative}.

\end{itemize}





\chapter{Definable Quotients and Others}

\todo{Before 18th-Dec-2013}

Sometimes types defined by quotienting other
types can also be defined inductively instead, for example the integers,
the rational numbers etc. It seems meaningless
to define them as quotient types, but is it really the case?
We are going to exploit the benefits from the definable quotient structures.
However there are some good
properties if we relate them with the base types and equivalence
relation, for example we can lift functions and them properties from base types to
quotient types. Moreover, if the base types are simpler to manipulate,
it is worthwhile using the base type to define functions and reasoning
and then lifting them. We can achieve more convenience by manipulating base types and then lifting the operators and propositions according to the relation between quotient types and base types.

In this Chapter we will show this using one of the examples, the set
of integers. Some of the work is is conducted by Thorsten Altenkirch,
Thomas Anberr\'{e}e and the author together, and summarised in \cite{aan} .


\subsection{Integers}

From the usual symbols to represent integers, we can easily figure out
one inductive definition for integers,

\begin{code}
\\
%
\\
\>\AgdaKeyword{data} \AgdaDatatype{ℤ} \AgdaSymbol{:} \AgdaPrimitiveType{Set} \AgdaKeyword{where}\<%
\\
\>[0]\AgdaIndent{2}{}\<[2]%
\>[2]\AgdaInductiveConstructor{+\_} \AgdaSymbol{:} \AgdaDatatype{ℕ} \AgdaSymbol{→} \AgdaDatatype{ℤ}\<%
\\
\>[0]\AgdaIndent{2}{}\<[2]%
\>[2]\AgdaInductiveConstructor{zero} \AgdaSymbol{:} \AgdaDatatype{ℤ}\<%
\\
\>[0]\AgdaIndent{2}{}\<[2]%
\>[2]\AgdaInductiveConstructor{-\_} \AgdaSymbol{:} \AgdaDatatype{ℕ} \AgdaSymbol{→} \AgdaDatatype{ℤ}\<%
\\
%
\\
\end{code}

However we face a trade-off: three different representation for zero or
to use code $+0$ for number $+1$. Usually the principle is to not
losing canonicity because it requires unnecessary checking for whether
some functions respect the equivalence or not. Therefore, the second
choice makes more sense and we refine it a bit as:

\begin{code}
\\
%
\\
\>\AgdaKeyword{data} \AgdaDatatype{ℤ} \AgdaSymbol{:} \AgdaPrimitiveType{Set} \AgdaKeyword{where}\<%
\\
\>[0]\AgdaIndent{2}{}\<[2]%
\>[2]\AgdaInductiveConstructor{+suc\_} \AgdaSymbol{:} \AgdaDatatype{ℕ} \AgdaSymbol{→} \AgdaDatatype{ℤ}\<%
\\
\>[0]\AgdaIndent{2}{}\<[2]%
\>[2]\AgdaInductiveConstructor{zero} \<[8]%
\>[8]\AgdaSymbol{:} \AgdaDatatype{ℤ}\<%
\\
\>[0]\AgdaIndent{2}{}\<[2]%
\>[2]\AgdaInductiveConstructor{-suc\_} \AgdaSymbol{:} \AgdaDatatype{ℕ} \AgdaSymbol{→} \AgdaDatatype{ℤ}\<%
\\
%
\\
\>\<\end{code}

This is better, but in practice it is expected to have more cases if
we use pattern matching. Every time we use pattern matching, a case will be
split into three. This becomes worse and worse when we have mutiple integer
arguments and we have to do case analysis on all of them. A simple
refinement is combining the first two constructors:

\begin{code}
\\
%
\\
\>\AgdaKeyword{data} \AgdaDatatype{ℤ} \AgdaSymbol{:} \AgdaPrimitiveType{Set} \AgdaKeyword{where}\<%
\\
\>[0]\AgdaIndent{2}{}\<[2]%
\>[2]\AgdaInductiveConstructor{+\_} \<[8]%
\>[8]\AgdaSymbol{:} \AgdaDatatype{ℕ} \AgdaSymbol{→} \AgdaDatatype{ℤ}\<%
\\
\>[0]\AgdaIndent{2}{}\<[2]%
\>[2]\AgdaInductiveConstructor{-suc\_} \AgdaSymbol{:} \AgdaDatatype{ℕ} \AgdaSymbol{→} \AgdaDatatype{ℤ}\<%
\\
%
\\
\>\<\end{code}

This is the most proper version we decided to use for integers. It is
inductively defined and is readable because it is just an
intepretation of the usual symbols for integers in regular mathematics.

Usually the reason of inventing integers is the lack of symbols to
represent the results of subtraction between two
natural numbers. Integers are used to represent these results, and
vice versa,
every integer can be represented as a pair of natural numbers and the
choice is not unique.
For example, from the equation $1 - 4 = - 3$, it is clear that the
integer $- 3$ can be represented the pair $(1,4)$. 
Therefore we can use the paired natural numbers as an alternative
definition for integers.

$$\Z_0=\N \times \N$$

However since there are different pairs for one integer, we have to
quotient it with an equivalence relation. For any two pairs of natural
numbers $(n_1, n_2)$ and $(n_3, n_4)$, we know they represent the same
integer if

$$ n_1 - n_2 = n_3 - n_4$$

Technically, this does not work because the subtraction defined for natural
numbers only returns zero if the pair is for negative number. We only
need to do some small modifications:

$$ n_1 + n_4 = n_3 + n_2$$

This helps us define a relation but it is not enough. This is an
equation in mathamtics, but in Type Theory we have to prove that it is an
equivalence relation, namely, it is reflexive, symmetric and transitive.

Combining the carrier (the pair of natural numbers), the equivalence
relation and its proof, we have a setoid.

\begin{code}
\\
\>\AgdaFunction{ℤ-Setoid} \AgdaSymbol{:} \AgdaRecord{Setoid}\<%
\\
\>\AgdaFunction{ℤ-Setoid} \AgdaSymbol{=} \AgdaKeyword{record}\<%
\\
\>[2]\AgdaIndent{3}{}\<[3]%
\>[3]\AgdaSymbol{\{} \AgdaField{Carrier} \<[19]%
\>[19]\AgdaSymbol{=} \AgdaFunction{ℤ₀}\<%
\\
\>[2]\AgdaIndent{3}{}\<[3]%
\>[3]\AgdaSymbol{;} \AgdaField{\_≈\_} \<[19]%
\>[19]\AgdaSymbol{=} \AgdaPostulate{\_∼\_}\<%
\\
\>[2]\AgdaIndent{3}{}\<[3]%
\>[3]\AgdaSymbol{;} \AgdaField{isEquivalence} \AgdaSymbol{=} \AgdaPostulate{\_∼\_isEquivalence}\<%
\\
\>[2]\AgdaIndent{3}{}\<[3]%
\>[3]\AgdaSymbol{\}}\<%
\\
\end{code}

Since integers is definable as we discussed before, they can
be seen as the normal forms of the equivalent classes. The normalisation
function can be defined as follows:
 
\begin{code}
\\
%
\\
\>\AgdaFunction{[\_]} \<[18]%
\>[18]\AgdaSymbol{:} \AgdaFunction{ℤ₀} \AgdaSymbol{→} \AgdaDatatype{ℤ}\<%
\\
\>\AgdaFunction{[} \AgdaBound{m} \AgdaInductiveConstructor{,} \AgdaInductiveConstructor{0} \AgdaFunction{]} \<[18]%
\>[18]\AgdaSymbol{=} \AgdaInductiveConstructor{+} \AgdaBound{m}\<%
\\
\>\AgdaFunction{[} \AgdaInductiveConstructor{0} \AgdaInductiveConstructor{,} \AgdaInductiveConstructor{suc} \AgdaBound{n} \AgdaFunction{]} \<[18]%
\>[18]\AgdaSymbol{=} \AgdaInductiveConstructor{-suc} \AgdaBound{n}\<%
\\
\>\AgdaFunction{[} \AgdaInductiveConstructor{suc} \AgdaBound{m} \AgdaInductiveConstructor{,} \AgdaInductiveConstructor{suc} \AgdaBound{n} \AgdaFunction{]} \AgdaSymbol{=} \AgdaFunction{[} \AgdaBound{m} \AgdaInductiveConstructor{,} \AgdaBound{n} \AgdaFunction{]}\<%
\\
%
\\
\end{code}

The function should be proved well-defined on the setoid, namely it
has to respect the equivalence relation. We call it soundness here. 
It is not trivial but easy to observe that the function is
sound\footnote{the formal proof can be found in appendix (we cheat a
  bit by defining embedding function to make it simpler)}.

A setoid and a function respects this equivalence (not necessary to be
a normalisation function) constitute a prequotient.

\begin{definition}
Prequotient.

\noindent
Given a setoid $(A,\sim)$,  a \emph{prequotient} $(Q,[\_],\sound)$ over that setoid consists in
\begin{enumerate}
\item \label{enum:Q} a set $Q$,
\item \label{enum:box}a function $[\_]: A \to Q$,
\item \label{enum:sound} a proof \emph{sound} that  the function $[\_]$ is compatible with the relation $\sim$,
that is \[\sound\colon (a,b : A) \to a\sim b \to [a] = [b],\]
\end{enumerate}


Prequotient only includes the formalisation rules and introduction
rules. To complete a \emph{quotient}, we also need the elimination rule added into such a prequotient

\begin{enumerate}
\setcounter{enumi}{3}
\item \label{enum:elim}
for any $B: Q\to\Set$, an eliminator
 \begin{align*}
 \qelim_B\,\,:\,\,\,&(f\colon (a:A) \to B\,\class a) \\
        {\to}\, &((p:a\sim b) \to f\,a \simeq_{\sound\,p}f\,b)\\
        {\to}\, &((q:Q) \to B\,q)
 \end{align*}
such that $\qelimbeta : \qelim_B\,f \,p\,\class a\equiv f\,a$.

\end{enumerate}

This eliminator is also called dependent lifting function because it
actually lifts a function which is well-defined on the setoid to a
function defined on the quotient type. The result type is also
dependent on the quotient type. There is an equivalent definition
given by Martin Hofmann
which has a non-dependent eliminator with an induction principle
instead.

\[\lift\colon (f\colon A \to B) \to (\forall a,b\cdot a\sim b \to f\,a
\equiv f\,b) \to (Q \to B)\]

Suppose $B$ is a predicate, 
\[\qind \colon((a: A)\to B \,\class a)\to ((q : Q)\to B\,q)\]

However, it is oberservable that given any non-empty set $Q$, all
constant functions fit in this definition. Any element of $Q$ has to
be mapped from at most one equivalence class of the setoid
$(A,\sim)$. This property is called \emph{exact} here 

\begin{enumerate}
\setcounter{enumi}{4}
\item $exact :(\forall a,b : A) \to  \class a \equiv \class b \to a \sim b$.

\end{enumerate}
\end{definition}

The quotient is exact if exactly one equivalence class corresponds to
an element of $Q$.

We already know that the integer is definable and it is plausible to
find a representative in each equivalence classes. Since
we treat elements of $Q$ as the name for the equivalence classes, the
selection function can be defined as an embedding function from  the
quotient type $Q$ to base type $A$. This is an alternative and more
flexible way to eliminate the quotient type $Q$ and if a
prequotient $(Q, \class{\dotph}, \sound)$ on a setoid $(A,\sim)$ has
an embedding function which is specified as
\begin{align*}
\emb &: Q \to A\\
\complete &: (a : A) \to \emb {\class a} \sim a\\
\stable &: (q:Q) \to \class{\emb\,q} \equiv q.\\
\end{align*}

Then it is a \emph{definable quotient}. Composing the ``normalisation'' function $\class\dotph$ with the
embedding function, we obtain the real normalisation function. A
definable quotient is an \emph{exact} quotient which is proved in \cite{aan}.



\paragraph{Operations}

For a definable quotient, we can lift an operation by mixing the
normalisation and embedding functions. For example, given an unary
operator 

\begin{code}
\\
\>\AgdaFunction{lift₁} \AgdaSymbol{:} \AgdaSymbol{(}\AgdaBound{op} \AgdaSymbol{:} \AgdaFunction{ℤ₀} \AgdaSymbol{→} \AgdaFunction{ℤ₀}\AgdaSymbol{)} \AgdaSymbol{→} \AgdaDatatype{ℤ} \AgdaSymbol{→} \AgdaDatatype{ℤ}\<%
\\
\>\AgdaFunction{lift₁} \AgdaBound{op} \AgdaSymbol{=} \AgdaFunction{[\_]} \AgdaFunction{∘} \AgdaBound{op} \AgdaFunction{∘} \AgdaFunction{⌜\_⌝}\<%
\\
\end{code}

To lift binary or n-ary operators, we only need to apply the operator
to the representative for each "equivalence class" and "normalise" the
result so that it becomes a function defined on the set of
"equivalence classes".

But there is a unavoidable problem: not all operations defined on
$\Z_0$ is defind on the setoid, namely respect the equivalence
relation. Therefore it is reasonable to verify if the function is
well-defined on the setoid:

$$a \sim b → op \, a \sim op \, b$$

We will show how to define the addition for the quotient
integers. Given two numbers $(a_1 , b_1)$ and $(a_2 , b_2)$ We
only need to add them pair-wisely together, and it can be verified
easily because we know that 

$(a_1 - b_1) + (a_2 - b_2) = (a_1 + a_2) - (b_1) - (b_2)$

The verification is not necessary here but should be important in
other cases.

\begin{code}
\\
\>\AgdaFunction{\_+\_} \AgdaSymbol{:} \AgdaFunction{ℤ₀} \AgdaSymbol{→} \AgdaFunction{ℤ₀} \AgdaSymbol{→} \AgdaFunction{ℤ₀}\<%
\\
\>\AgdaSymbol{(}\AgdaBound{x+} \AgdaInductiveConstructor{,} \AgdaBound{x-}\AgdaSymbol{)} \AgdaFunction{+} \AgdaSymbol{(}\AgdaBound{y+} \AgdaInductiveConstructor{,} \AgdaBound{y-}\AgdaSymbol{)} \AgdaSymbol{=} \AgdaSymbol{(}\AgdaBound{x+} \AgdaFunction{ℕ+} \AgdaBound{y+}\AgdaSymbol{)} \AgdaInductiveConstructor{,} \AgdaSymbol{(}\AgdaBound{x-} \AgdaFunction{ℕ+} \AgdaBound{y-}\AgdaSymbol{)}\<%
\\
\end{code}


\paragraph{Properties}

We can also define the ring of $\Z$. It contains a lot of properties
to prove which are seemed as axioms in classic mathematics. In
constructive mathematics, the only axioms for integers are the
constructors and the elimination rules.

As what we have menetioned, even though the definition of integers
only has two constructors, it gradually increase the difficulty of
proving when doing case analyses on more and more integers. One
example is the proving of distributivity.

\paragraph{Comparison}

\todo{The advantage of use Quotient algebraic structure: the proving of distributivity}

If we try to prove distributivity for integers using the normal
definition, we have to do case analyses on all the three integers,
namely there are eight cases to prove in total. However, if we prove
the laws for quotient integers, it is much simpler since there is
only one case to prove.



However, if we prove it for the quotient integers, it is much
easier. In fact, it is in generally automatically provable. Since the
equality of any two quotient integers is just the equality of two
natural numbers, after some transformation. 
To prove the distributivity, what we need to do is to use the semiring
solver for natural numbers. $\AgdaFunction{DistributesOverˡ}$ means
that the distributivity of the first operators over the second one.
\todo{explain what is a ring solver?}


\begin{code}
\\
\>\AgdaFunction{distˡ} \AgdaSymbol{:} \<[9]%
\>[9]\AgdaFunction{\_*\_} \AgdaFunction{DistributesOverˡ} \AgdaFunction{\_+\_}\<%
\\
\>\AgdaFunction{distˡ} \AgdaSymbol{(}\AgdaBound{a} \AgdaInductiveConstructor{,} \AgdaBound{b}\AgdaSymbol{)} \AgdaSymbol{(}\AgdaBound{c} \AgdaInductiveConstructor{,} \AgdaBound{d}\AgdaSymbol{)} \AgdaSymbol{(}\AgdaBound{e} \AgdaInductiveConstructor{,} \AgdaBound{f}\AgdaSymbol{)} \AgdaSymbol{=} \AgdaFunction{solve} \AgdaNumber{6} \<[40]%
\>[40]\<%
\\
\>[0]\AgdaIndent{2}{}\<[2]%
\>[2]\AgdaSymbol{(λ} \AgdaBound{a} \AgdaBound{b} \AgdaBound{c} \AgdaBound{d} \AgdaBound{e} \AgdaBound{f} \AgdaSymbol{→} \AgdaBound{a} \AgdaFunction{:*} \AgdaSymbol{(}\AgdaBound{c} \AgdaFunction{:+} \AgdaBound{e}\AgdaSymbol{)} \AgdaFunction{:+} \AgdaBound{b} \AgdaFunction{:*} \AgdaSymbol{(}\AgdaBound{d} \AgdaFunction{:+} \AgdaBound{f}\AgdaSymbol{)} \AgdaFunction{:+}\<%
\\
\>[2]\AgdaIndent{6}{}\<[6]%
\>[6]\AgdaSymbol{(}\AgdaBound{a} \AgdaFunction{:*} \AgdaBound{d} \AgdaFunction{:+} \AgdaBound{b} \AgdaFunction{:*} \AgdaBound{c} \AgdaFunction{:+} \AgdaSymbol{(}\AgdaBound{a} \AgdaFunction{:*} \AgdaBound{f} \AgdaFunction{:+} \AgdaBound{b} \AgdaFunction{:*} \AgdaBound{e}\AgdaSymbol{))}\<%
\\
\>[2]\AgdaIndent{6}{}\<[6]%
\>[6]\AgdaFunction{:=}\<%
\\
\>[2]\AgdaIndent{6}{}\<[6]%
\>[6]\AgdaBound{a} \AgdaFunction{:*} \AgdaBound{c} \AgdaFunction{:+} \AgdaBound{b} \AgdaFunction{:*} \AgdaBound{d} \AgdaFunction{:+} \AgdaSymbol{(}\AgdaBound{a} \AgdaFunction{:*} \AgdaBound{e} \AgdaFunction{:+} \AgdaBound{b} \AgdaFunction{:*} \AgdaBound{f}\AgdaSymbol{)} \AgdaFunction{:+}\<%
\\
\>[2]\AgdaIndent{6}{}\<[6]%
\>[6]\AgdaSymbol{(}\AgdaBound{a} \AgdaFunction{:*} \AgdaSymbol{(}\AgdaBound{d} \AgdaFunction{:+} \AgdaBound{f}\AgdaSymbol{)} \AgdaFunction{:+} \AgdaBound{b} \AgdaFunction{:*} \AgdaSymbol{(}\AgdaBound{c} \AgdaFunction{:+} \AgdaBound{e}\AgdaSymbol{)))} \AgdaInductiveConstructor{refl} \AgdaBound{a} \AgdaBound{b} \AgdaBound{c} \AgdaBound{d} \AgdaBound{e} \AgdaBound{f}\<%
\\
\end{code}


There are some drawbacks of this method. First, we have to explicitly
write the equations which means it is not elegant enough. What we
expect is something like the "ring" tactic in Coq which can solve any
equations automatically. This can be achieved by "reflection" which is
feature that returns the type of the goal as some terms such that we
can process it and using it in ring solver. There is already some work
done by van der Walt \cite{van2012reflection}.

Another disadvantage of this method is the type verification of the
terms automatically derived requires more computations than the terms
we manually construct. As long as we use the distributivity in later
proofs, it may heavily slow the type checking. Therefore although
it is very convenient to use the ring solver to prove any proposition for the quotient
integers, we still prove all the common properties for the commutative
ring of the integers. Luckily it is still much simpler than the ones
for the normal integers.


\subsection{Rational numbers}

The quotient of rational numbers is better known than the previous
quotient. We usually write two integers $m$ and $n$ ($n$ is not zero) in
fractional form $\frac{m}{n}$ to represent a rational number. Alternatively we
can use an integer and a positive natural number such that it is
simpler to exclude 0 in the denominator. Two fractions are equal if
they are reduced to the same irreducible term. If the numerator and
denominator of a fraction are coprime, it is said to be an irreducible
fraction. Based on this observation, it is naturally to form a definable quotient, where the base type is 

$$\Q_0 = \Z \times \N$$

The integer is \emph{numerator} and the natural number is \emph{denominator-1}. This approach avoids invalid fractions from construction. 

In Agda, to make the terms more meaningful we define it as

\begin{code}
\\
\>\AgdaKeyword{data} \AgdaDatatype{ℚ₀} \AgdaSymbol{:} \AgdaPrimitiveType{Set} \AgdaKeyword{where}\<%
\\
\>[-1]\AgdaIndent{2}{}\<[2]%
\>[2]\AgdaInductiveConstructor{\_/suc\_} \AgdaSymbol{:} \AgdaSymbol{(}\AgdaBound{n} \AgdaSymbol{:} \AgdaDatatype{ℤ}\AgdaSymbol{)} \AgdaSymbol{→} \AgdaSymbol{(}\AgdaBound{d} \AgdaSymbol{:} \AgdaDatatype{ℕ}\AgdaSymbol{)} \AgdaSymbol{→} \AgdaDatatype{ℚ₀}\<%
\\
\end{code}

In mathematics, to judge the equality of two fractions, it is easier to conduct the following conversion,

$$ \frac{a}{b} = \frac{c}{d} \iff a \times d = c \times b $$

Therefore the equivalence relation can be defined as,

\begin{code}
\\
\>\AgdaFunction{\_*\_} \AgdaSymbol{:} \AgdaDatatype{ℤ} \AgdaSymbol{→} \AgdaDatatype{ℕ} \AgdaSymbol{→} \AgdaDatatype{ℤ}\<%
\\
\>\AgdaSymbol{(}\AgdaInductiveConstructor{+} \AgdaBound{x}\AgdaSymbol{)} \AgdaFunction{*} \AgdaBound{d} \AgdaSymbol{=} \AgdaInductiveConstructor{+} \AgdaSymbol{(}\AgdaBound{x} \AgdaPrimitive{ℕ*} \AgdaBound{d}\AgdaSymbol{)}\<%
\\
\>\AgdaSymbol{(}\AgdaInductiveConstructor{-suc} \AgdaBound{x}\AgdaSymbol{)} \AgdaFunction{*} \AgdaInductiveConstructor{0} \AgdaSymbol{=} \AgdaInductiveConstructor{+} \AgdaNumber{0}\<%
\\
\>\AgdaSymbol{(}\AgdaInductiveConstructor{-suc} \AgdaBound{x}\AgdaSymbol{)} \AgdaFunction{*} \AgdaInductiveConstructor{suc} \AgdaBound{d} \AgdaSymbol{=} \AgdaInductiveConstructor{-suc} \AgdaSymbol{(}\AgdaBound{x} \AgdaPrimitive{ℕ+} \AgdaInductiveConstructor{suc} \AgdaBound{x} \AgdaPrimitive{ℕ*} \AgdaBound{d}\AgdaSymbol{)}\<%
\\
%
\\
\>\AgdaFunction{\_∼\_} \<[35]%
\>[35]\AgdaSymbol{:} \AgdaFunction{Rel} \AgdaDatatype{ℚ₀} \AgdaSymbol{\_}\<%
\\
\>\AgdaSymbol{(}\AgdaBound{n1} \AgdaInductiveConstructor{/suc} \AgdaBound{d1}\AgdaSymbol{)} \AgdaFunction{∼} \AgdaSymbol{(}\AgdaBound{n2} \AgdaInductiveConstructor{/suc} \AgdaBound{d2}\AgdaSymbol{)} \AgdaSymbol{=} \<[31]%
\>[31]\AgdaBound{n1} \AgdaFunction{*} \AgdaInductiveConstructor{suc} \AgdaBound{d2} \AgdaDatatype{≡} \AgdaBound{n2} \AgdaFunction{*} \AgdaInductiveConstructor{suc} \AgdaBound{d1}\<%
\\
\end{code}

The normal form of rational numbers, namely the quotient type in this quotient is the set of irreducible fractions. We only need to add a restriction that the numerator and denominator is coprime,

$$\Q = \Sigma (n \colon \Z). \Sigma (d \colon \N). \mathsf{coprime} \,n \,(d +1)$$

It can be defined as follows which is available in standard library,

\begin{code}
\\
\>\AgdaKeyword{record} \AgdaRecord{ℚ} \AgdaSymbol{:} \AgdaPrimitiveType{Set} \AgdaKeyword{where}\<%
\\
\>[0]\AgdaIndent{2}{}\<[2]%
\>[2]\AgdaKeyword{field}\<%
\\
\>[2]\AgdaIndent{4}{}\<[4]%
\>[4]\AgdaField{numerator} \<[18]%
\>[18]\AgdaSymbol{:} \AgdaDatatype{ℤ}\<%
\\
\>[2]\AgdaIndent{4}{}\<[4]%
\>[4]\AgdaField{denominator-1} \AgdaSymbol{:} \AgdaDatatype{ℕ}\<%
\\
\>[2]\AgdaIndent{4}{}\<[4]%
\>[4]\AgdaField{isCoprime} \<[18]%
\>[18]\AgdaSymbol{:} \AgdaFunction{True} \AgdaSymbol{(}\AgdaFunction{C.coprime?} \AgdaFunction{∣} \AgdaBound{numerator} \AgdaFunction{∣} \AgdaSymbol{(}\AgdaInductiveConstructor{suc} \AgdaBound{denominator-1}\AgdaSymbol{))}\<%
\\
%
\\
\>[0]\AgdaIndent{2}{}\<[2]%
\>[2]\AgdaFunction{denominator} \AgdaSymbol{:} \AgdaDatatype{ℤ}\<%
\\
\>[0]\AgdaIndent{2}{}\<[2]%
\>[2]\AgdaFunction{denominator} \AgdaSymbol{=} \AgdaInductiveConstructor{+} \AgdaInductiveConstructor{suc} \AgdaFunction{denominator-1}\<%
\\
%
\\
\>[0]\AgdaIndent{2}{}\<[2]%
\>[2]\AgdaFunction{coprime} \AgdaSymbol{:} \AgdaFunction{Coprime} \AgdaFunction{numerator} \AgdaFunction{denominator}\<%
\\
\>[0]\AgdaIndent{2}{}\<[2]%
\>[2]\AgdaFunction{coprime} \AgdaSymbol{=} \AgdaFunction{toWitness} \AgdaFunction{isCoprime}\<%
\\
\end{code}

The normalisation function is an implementation of the reducing
process. But first we need to  the |gcd| function which calculates the greatest common
divisor can help us reduce the fraction and give us the proof of
coprime. First we need to define the conversion from the results of GCD
to normal rational numbers,

\begin{code}
\\
\>\AgdaFunction{GCD′→ℚ} \AgdaSymbol{:} \AgdaSymbol{∀} \AgdaBound{x} \AgdaBound{y} \AgdaBound{di} \AgdaSymbol{→} \AgdaBound{y} \AgdaFunction{≢} \AgdaNumber{0} \AgdaSymbol{→} \AgdaDatatype{C.GCD′} \AgdaBound{x} \AgdaBound{y} \AgdaBound{di} \AgdaSymbol{→} \AgdaRecord{ℚ}\<%
\\
\>\AgdaFunction{GCD′→ℚ} \AgdaSymbol{.(}\AgdaBound{q₁} \AgdaPrimitive{ℕ*} \AgdaBound{di}\AgdaSymbol{)} \AgdaSymbol{.(}\AgdaBound{q₂} \AgdaPrimitive{ℕ*} \AgdaBound{di}\AgdaSymbol{)} \AgdaBound{di} \AgdaBound{neo} \AgdaSymbol{(}\AgdaInductiveConstructor{C.gcd-*} \AgdaBound{q₁} \AgdaBound{q₂} \AgdaBound{c}\AgdaSymbol{)} \AgdaSymbol{=} \AgdaKeyword{record} \AgdaSymbol{\{} \AgdaField{numerator} \AgdaSymbol{=} \AgdaFunction{numr}\<%
\\
\>[0]\AgdaIndent{10}{}\<[10]%
\>[10]\AgdaSymbol{;} \AgdaField{denominator-1} \AgdaSymbol{=} \AgdaFunction{pred} \AgdaBound{q₂}\<%
\\
\>[0]\AgdaIndent{10}{}\<[10]%
\>[10]\AgdaSymbol{;} \AgdaField{isCoprime} \AgdaSymbol{=} \AgdaFunction{iscoprime} \AgdaSymbol{\}}\<%
\\
\>[0]\AgdaIndent{3}{}\<[3]%
\>[3]\AgdaKeyword{where}\<%
\\
\>[0]\AgdaIndent{5}{}\<[5]%
\>[5]\AgdaFunction{numr} \AgdaSymbol{=} \AgdaInductiveConstructor{ℤ.+\_} \AgdaBound{q₁}\<%
\\
\>[0]\AgdaIndent{5}{}\<[5]%
\>[5]\AgdaFunction{deno} \AgdaSymbol{=} \AgdaInductiveConstructor{suc} \AgdaSymbol{(}\AgdaFunction{pred} \AgdaBound{q₂}\AgdaSymbol{)}\<%
\\
\>[0]\AgdaIndent{5}{}\<[5]%
\>[5]\<%
\\
\>[0]\AgdaIndent{5}{}\<[5]%
\>[5]\AgdaFunction{numr≡q1} \AgdaSymbol{:} \AgdaFunction{∣} \AgdaFunction{numr} \AgdaFunction{∣} \AgdaDatatype{≡} \AgdaBound{q₁}\<%
\\
\>[0]\AgdaIndent{5}{}\<[5]%
\>[5]\AgdaFunction{numr≡q1} \AgdaSymbol{=} \AgdaInductiveConstructor{refl}\<%
\\
%
\\
\>[0]\AgdaIndent{5}{}\<[5]%
\>[5]\AgdaFunction{lzero} \AgdaSymbol{:} \AgdaSymbol{∀} \AgdaBound{x} \AgdaBound{y} \AgdaSymbol{→} \AgdaBound{x} \AgdaDatatype{≡} \AgdaNumber{0} \AgdaSymbol{→} \AgdaBound{x} \AgdaPrimitive{ℕ*} \AgdaBound{y} \AgdaDatatype{≡} \AgdaNumber{0}\<%
\\
\>[0]\AgdaIndent{5}{}\<[5]%
\>[5]\AgdaFunction{lzero} \AgdaSymbol{.}\AgdaNumber{0} \AgdaBound{y} \AgdaInductiveConstructor{refl} \AgdaSymbol{=} \AgdaInductiveConstructor{refl}\<%
\\
%
\\
\>[0]\AgdaIndent{5}{}\<[5]%
\>[5]\AgdaFunction{q2≢0} \AgdaSymbol{:} \AgdaBound{q₂} \AgdaFunction{≢} \AgdaNumber{0}\<%
\\
\>[0]\AgdaIndent{5}{}\<[5]%
\>[5]\AgdaFunction{q2≢0} \AgdaBound{qe} \AgdaSymbol{=} \AgdaBound{neo} \AgdaSymbol{(}\AgdaFunction{lzero} \AgdaBound{q₂} \AgdaBound{di} \AgdaBound{qe}\AgdaSymbol{)}\<%
\\
%
\\
\>[0]\AgdaIndent{5}{}\<[5]%
\>[5]\AgdaFunction{invsuc} \AgdaSymbol{:} \AgdaSymbol{∀} \AgdaBound{n} \AgdaSymbol{→} \AgdaBound{n} \AgdaFunction{≢} \AgdaNumber{0} \AgdaSymbol{→} \AgdaInductiveConstructor{suc} \AgdaSymbol{(}\AgdaFunction{pred} \AgdaBound{n}\AgdaSymbol{)} \AgdaDatatype{≡} \AgdaBound{n}\<%
\\
\>[0]\AgdaIndent{5}{}\<[5]%
\>[5]\AgdaFunction{invsuc} \AgdaInductiveConstructor{zero} \AgdaBound{nz} \AgdaKeyword{with} \AgdaBound{nz} \AgdaInductiveConstructor{refl}\<%
\\
\>[0]\AgdaIndent{5}{}\<[5]%
\>[5]\AgdaSymbol{...} \AgdaSymbol{|} \AgdaSymbol{()}\<%
\\
\>[0]\AgdaIndent{5}{}\<[5]%
\>[5]\AgdaFunction{invsuc} \AgdaSymbol{(}\AgdaInductiveConstructor{suc} \AgdaBound{n}\AgdaSymbol{)} \AgdaBound{nz} \AgdaSymbol{=} \AgdaInductiveConstructor{refl}\<%
\\
\>[0]\AgdaIndent{5}{}\<[5]%
\>[5]\<%
\\
\>[0]\AgdaIndent{5}{}\<[5]%
\>[5]\AgdaFunction{deno≡q2} \AgdaSymbol{:} \AgdaFunction{deno} \AgdaDatatype{≡} \AgdaBound{q₂}\<%
\\
\>[0]\AgdaIndent{5}{}\<[5]%
\>[5]\AgdaFunction{deno≡q2} \AgdaSymbol{=} \AgdaFunction{invsuc} \AgdaBound{q₂} \AgdaFunction{q2≢0}\<%
\\
\>[0]\AgdaIndent{5}{}\<[5]%
\>[5]\<%
\\
\>[0]\AgdaIndent{5}{}\<[5]%
\>[5]\AgdaFunction{transCop} \AgdaSymbol{:} \AgdaSymbol{∀} \AgdaSymbol{\{}\AgdaBound{a} \AgdaBound{b} \AgdaBound{c} \AgdaBound{d}\AgdaSymbol{\}} \AgdaSymbol{→} \AgdaBound{c} \AgdaDatatype{≡} \AgdaBound{a} \AgdaSymbol{→} \AgdaBound{d} \AgdaDatatype{≡} \AgdaBound{b} \AgdaSymbol{→} \AgdaFunction{C.Coprime} \AgdaBound{a} \AgdaBound{b} \AgdaSymbol{→} \AgdaFunction{C.Coprime} \AgdaBound{c} \AgdaBound{d}\<%
\\
\>[0]\AgdaIndent{5}{}\<[5]%
\>[5]\AgdaFunction{transCop} \AgdaInductiveConstructor{refl} \AgdaInductiveConstructor{refl} \AgdaBound{c} \AgdaSymbol{=} \AgdaBound{c}\<%
\\
%
\\
\>[0]\AgdaIndent{5}{}\<[5]%
\>[5]\AgdaFunction{copnd} \AgdaSymbol{:} \AgdaFunction{C.Coprime} \AgdaFunction{∣} \AgdaFunction{numr} \AgdaFunction{∣} \AgdaFunction{deno}\<%
\\
\>[0]\AgdaIndent{5}{}\<[5]%
\>[5]\AgdaFunction{copnd} \AgdaSymbol{=} \AgdaFunction{transCop} \AgdaFunction{numr≡q1} \AgdaFunction{deno≡q2} \AgdaBound{c}\<%
\\
%
\\
\>[0]\AgdaIndent{5}{}\<[5]%
\>[5]\AgdaFunction{witProp} \AgdaSymbol{:} \AgdaSymbol{∀} \AgdaBound{a} \AgdaBound{b} \AgdaSymbol{→} \AgdaRecord{GCD} \AgdaBound{a} \AgdaBound{b} \AgdaNumber{1} \AgdaSymbol{→} \AgdaFunction{True} \AgdaSymbol{(}\AgdaFunction{C.coprime?} \AgdaBound{a} \AgdaBound{b}\AgdaSymbol{)}\<%
\\
\>[0]\AgdaIndent{5}{}\<[5]%
\>[5]\AgdaFunction{witProp} \AgdaBound{a} \AgdaBound{b} \AgdaBound{gcd1} \AgdaKeyword{with} \AgdaFunction{gcd} \AgdaBound{a} \AgdaBound{b}\<%
\\
\>[0]\AgdaIndent{5}{}\<[5]%
\>[5]\AgdaFunction{witProp} \AgdaBound{a} \AgdaBound{b} \AgdaBound{gcd1} \AgdaSymbol{|} \AgdaInductiveConstructor{zero} \AgdaInductiveConstructor{,} \AgdaBound{y} \AgdaKeyword{with} \AgdaFunction{GCD.unique} \AgdaBound{gcd1} \AgdaBound{y}\<%
\\
\>[0]\AgdaIndent{5}{}\<[5]%
\>[5]\AgdaFunction{witProp} \AgdaBound{a} \AgdaBound{b} \AgdaBound{gcd1} \AgdaSymbol{|} \AgdaInductiveConstructor{zero} \AgdaInductiveConstructor{,} \AgdaBound{y} \AgdaSymbol{|} \AgdaSymbol{()}\<%
\\
\>[0]\AgdaIndent{5}{}\<[5]%
\>[5]\AgdaFunction{witProp} \AgdaBound{a} \AgdaBound{b} \AgdaBound{gcd1} \AgdaSymbol{|} \AgdaInductiveConstructor{suc} \AgdaInductiveConstructor{zero} \AgdaInductiveConstructor{,} \AgdaBound{y} \AgdaSymbol{=} \AgdaInductiveConstructor{tt}\<%
\\
\>[0]\AgdaIndent{5}{}\<[5]%
\>[5]\AgdaFunction{witProp} \AgdaBound{a} \AgdaBound{b} \AgdaBound{gcd1} \AgdaSymbol{|} \AgdaInductiveConstructor{suc} \AgdaSymbol{(}\AgdaInductiveConstructor{suc} \AgdaBound{n}\AgdaSymbol{)} \AgdaInductiveConstructor{,} \AgdaBound{y} \AgdaKeyword{with} \AgdaFunction{GCD.unique} \AgdaBound{gcd1} \AgdaBound{y}\<%
\\
\>[0]\AgdaIndent{5}{}\<[5]%
\>[5]\AgdaFunction{witProp} \AgdaBound{a} \AgdaBound{b} \AgdaBound{gcd1} \AgdaSymbol{|} \AgdaInductiveConstructor{suc} \AgdaSymbol{(}\AgdaInductiveConstructor{suc} \AgdaBound{n}\AgdaSymbol{)} \AgdaInductiveConstructor{,} \AgdaBound{y} \AgdaSymbol{|} \AgdaSymbol{()}\<%
\\
%
\\
\>[0]\AgdaIndent{5}{}\<[5]%
\>[5]\AgdaFunction{iscoprime} \AgdaSymbol{:} \AgdaFunction{True} \AgdaSymbol{(}\AgdaFunction{C.coprime?} \AgdaFunction{∣} \AgdaFunction{numr} \AgdaFunction{∣} \AgdaFunction{deno}\AgdaSymbol{)}\<%
\\
\>[0]\AgdaIndent{5}{}\<[5]%
\>[5]\AgdaFunction{iscoprime} \AgdaSymbol{=} \AgdaFunction{witProp} \AgdaFunction{∣} \AgdaFunction{numr} \AgdaFunction{∣} \AgdaFunction{deno} \AgdaSymbol{(}\AgdaFunction{C.coprime-gcd} \AgdaFunction{copnd}\AgdaSymbol{)}\<%
\\
\end{code}

To normalise a fractional, we split it into 3 cases with respect to
the numerator. The idea is to calculate the "gcd'' and then use the
above function to get the normalised rational number.

\begin{code}
\\
\>\AgdaFunction{[\_]} \AgdaSymbol{:} \AgdaDatatype{ℚ₀} \AgdaSymbol{→} \AgdaRecord{ℚ}\<%
\\
\>\AgdaFunction{[} \AgdaSymbol{(}\AgdaInductiveConstructor{+} \AgdaInductiveConstructor{0}\AgdaSymbol{)} \AgdaInductiveConstructor{/suc} \AgdaBound{d} \AgdaFunction{]} \AgdaSymbol{=} \AgdaInductiveConstructor{ℤ.+\_} \AgdaNumber{0} \AgdaFunction{÷} \AgdaNumber{1}\<%
\\
\>\AgdaFunction{[} \AgdaSymbol{(}\AgdaInductiveConstructor{+} \AgdaSymbol{(}\AgdaInductiveConstructor{suc} \AgdaBound{n}\AgdaSymbol{))} \AgdaInductiveConstructor{/suc} \AgdaBound{d} \AgdaFunction{]} \AgdaKeyword{with} \AgdaFunction{gcd} \AgdaSymbol{(}\AgdaInductiveConstructor{suc} \AgdaBound{n}\AgdaSymbol{)} \AgdaSymbol{(}\AgdaInductiveConstructor{suc} \AgdaBound{d}\AgdaSymbol{)}\<%
\\
\>\AgdaFunction{[} \AgdaSymbol{(}\AgdaInductiveConstructor{+} \AgdaInductiveConstructor{suc} \AgdaBound{n}\AgdaSymbol{)} \AgdaInductiveConstructor{/suc} \AgdaBound{d} \AgdaFunction{]} \AgdaSymbol{|} \AgdaBound{di} \AgdaInductiveConstructor{,} \AgdaBound{g} \AgdaSymbol{=} \AgdaFunction{GCD′→ℚ} \AgdaSymbol{(}\AgdaInductiveConstructor{suc} \AgdaBound{n}\AgdaSymbol{)} \AgdaSymbol{(}\AgdaInductiveConstructor{suc} \AgdaBound{d}\AgdaSymbol{)} \AgdaBound{di} \AgdaSymbol{(λ} \AgdaSymbol{())} \AgdaSymbol{(}\AgdaFunction{C.gcd-gcd′} \AgdaBound{g}\AgdaSymbol{)}\<%
\\
\end{code}


The embedding function is simple. We only need to forget the coprime proof in the normal form,

\begin{code}
\\
\>\AgdaFunction{⌜\_⌝} \AgdaSymbol{:} \AgdaRecord{ℚ} \AgdaSymbol{→} \AgdaDatatype{ℚ₀}\<%
\\
\>\AgdaFunction{⌜} \AgdaBound{x} \AgdaFunction{⌝} \AgdaSymbol{=} \AgdaSymbol{(}\AgdaFunction{ℤcon} \AgdaSymbol{(}\AgdaFunction{ℚ.numerator} \AgdaBound{x}\AgdaSymbol{))} \AgdaInductiveConstructor{/suc} \AgdaSymbol{(}\AgdaFunction{ℚ.denominator-1} \AgdaBound{x}\AgdaSymbol{)}\<%
\\
\end{code} 

Similarly, we are able to construct the setoid, the prequotient and
then the definable quotient of rational numbers. We can benefit from
the ease of defining operators and proving theorems on setoids while
still using the normal form of rational numbers, the lifted operators
and properties which are safer.

The same approach works here as well. Since we can easily embed the natural numbers into integers, the
equations of the quotient rational numbers are degraded to equations of the
integers.  The commutative ring of integers also enable us to prove
all properties of rational numbers automatically.


\subsection{Real numbers}

The previous quotient types are all definable in \itt{}, so we can
construct the definable quotients for them. However, there are some
types undefinable in \itt{}. The set of real numbers $\R$ has been proved to be undefinable in \cite{aan}.

We have several choices to represent real numbers. We choose Cauchy
sequences of rational numbers to represent real numbers \cite{bis:85}.

$$\R_{0} = \set{s : \N\to\Q \mid \forall\varepsilon
  :\Q,\varepsilon>0\to\exists m:\N, \forall i:\N, i>m\to \vert  s_i -
  s_m \vert  <\varepsilon}$$

It is implementable in Type Theory, but there is a problem of the
choice of type of the second part of the cauchy sequence, namely the property
that it is a cauchy sequence. Do we distinguish the same sequences
with different proofs? Logically speaking we should not. It means that
we need to truncate it to proposition but we will lose the important
tool to guess what the real number is. To avoid this problem, we could
use an alternative equivalent definition which is a subset of $\R_{0}$:

$$\R_0' = \set{f: \N\to\Q \mid \forall k
  :\N,\forall m , n > k, \to \vert  f_m -
  f_n \vert  < \frac{1}{k}}$$

With the definition, the condition part is propositional and we can
guess the number by applying any number k to the sequence and we know
the interval where it should be located.

Different cauchy sequences can represent the same number. Therefore an equivalence relation\footnote{
The Agda version is in Appendix} is expected. In mathematics two Cauchy sequences $\R_0$ are said to be
equal if their pointwise difference converges to zero,

$$r \sim s = \forall\varepsilon :\Q,\varepsilon>0\to\exists m:\N,
\forall i:\N, i>m\to \vert  r_i - s_i \vert <\varepsilon$$

\subsection{Non-normalizability of Cauchy Sequences}

To prove that it is impossible to give a full definable quotient
structure of real numbers with the setoid of cauchy sequences, we
could show it by proving that it is impossible to define a
normalisation function for the cauchy sequences.

\begin{definition}\label{def:nor}
We say that a quotient structure $\qset{A}$ is definable via a
normalisation, if we have a normalisation function
 \begin{equation}
  nf : A \to A
 \end{equation}
 with the property that it respects $\sim$
\begin{equation}
 p : \prd{c_1,c_2 : A} c_1 \sim c_2 \to nf(c_1) = nf(c_2).
\end{equation}
such that
 \begin{equation}
 q : \prd{c : A}  {nf(c)} \sim c.
 \end{equation}
\end{definition}

It is equivalent to say that we have a definable quotient structure in
the sense of \cite{aan}, because we can form the set of equivalence
classes as

\begin{equation*}
Q \defeq \Sigma_{c : \R_0} nf(c) = c
\end{equation*}

where the second part is propositional, and the "normalisation''
function can be defined as

\begin{equation*}
[c] \defeq nf(c) , p(q)
\end{equation*}

and the embedding function is just the first projection. The properties can be verified
easily.

In the other way around, the true normalisation function is just

\begin{equation*}
n \defeq emb \circ [\_]
\end{equation*}

and the properties are holded as well.


We have made an attempt in the original version of our \cite{aan}
draft, but there is something important problem pointed out by Martin Escardo. Laterly, Nicolai
Kraus suggests to fix the proof by proving it as a meta-theoretical
property. We will show an adaption of his proof here.

\paragraph{Some preliminaries}

In fact the proof is mainly conducted using topological tools. The
following definitions are helpful for someone who are not so familiar
with topological concepts.


\begin{definition}
Metric space. In mathematics, a metric space is a set where a notion
of distance (called a metric) between elements of the set is defined.

A metric space is an ordered pair $(M , d)$ where M is a set and d is a metric on M:
\begin{enumerate}
\item M is a set,
\item and $d : M \times M \rightarrow \R$ s.t.
\item $d (x , y) = 0 \iff x = y$
\item $d(x,y)=d(y,x)$
\item $d(x,y)+d(y,z) \ge d(x,z)$ 
\end{enumerate}
\end{definition}

We usually define a standard topological structure for discrete types.

\begin{itemize}

\item $(\bool , h)$ where 
$h(m,n) =
\left\{
	\begin{array}{ll}
		0  & \mbox{if } m = n \\
		1 & \mbox{if } m \neq n
	\end{array}
\right.
$

\item $(\N , d)$ where 
$d(m,n) =
\left\{
	\begin{array}{ll}
		0  & \mbox{if } m = n \\
		1 & \mbox{if } m \neq n
	\end{array}
\right.
$

\item $(\Q , e)$ where 
$e(m,n) =
\left\{
	\begin{array}{ll}
		0  & \mbox{if } m = n \\
		1 & \mbox{if } m \neq n
	\end{array}
\right.
$

\end{itemize}

We use a slightly different definition of cauchy sequences of rational
numbers here:

\begin{definition}
 We call a function $f : \PN \to \Q$\footnote{we
   use $\PN$ instead of $\N$ because $n$ must be positive in $\frac 1 n$} a \emph{Cauchy Sequence} if it satisfies
 \begin{equation}
  \iscauchy (f) \defeq \fa{n}{\PN} \fa{m} {\N+} m > n \to |f(n) - f(m)| < \frac 1 n. \label{eq:prop-property-short}
 \end{equation}
 The type of Cauchy Sequences is thus
 \begin{equation*}
  \R_0 \defeq \sm{f : \PN \to \Q} \iscauchy (f).
 \end{equation*}
\end{definition}

And the standard metric space for the sequences $\PN \rightarrow \Q $
is defined by the distance function

\begin{equation}
g(f_1,f_2) = 2^{-\mathsf{inf}\{k \in \N \; | \; f_1(k) \not= f_2(k)\}}
\end{equation}

Among all these standard metric spaces, It is a folklore that all
definable functions are continous.

\begin{theorem}\label{defcon}
All definable functions are continous.
\end{theorem}

Let us introduce the following auxiliary definition:
\begin{definition}
 For a sequence $f : \N \to \Q$, we say that $f$ is \emph{Cauchy with
   factor $k$}, written as $\iscauchy_k$, for some $k \in \Q$, $k > 0$, if
 \begin{equation}
  \iscauchy_k(f) \; \defeq \; \fapairs{n}{m}{\N} m > n \to |f(n) - f(m)| < \frac 1 {k \cdot n}. \label{eq:cauchy-aux}
 \end{equation}
 The usual Cauchy condition $\iscauchy$ is therefore ``Cauchy with factor $1$''.
\end{definition}

\begin{remark} If we claim a function $f$ is defined on $\R_0$ that
  respects $\sim$, it means that we have a proof
\begin{equation}
 p : \prd{c_1,c_2 : \R_0} c_1 \sim c_2 \to f(c_1) = f(c_2).
\end{equation}

\end{remark}

Now we have enough tools to prove the following proposition.

\begin{proposition} \label{prop:main}
 \textbf{"$\R_0 / \sim$" is connected.} It means that any continuous function $f$ 
 \begin{equation}
  f : \R_0 \to \bool
 \end{equation} that respects $\sim$ is constant. We prove that it is impossible to find $c_1, c_2:
\R_0$ such that $f(c_1) \not= f(c_2)$ meta-theoretically, instead of
deriving a proof term of this in type theory.

The definable of function implies that it is a continous function \ref{defcon}
between the standard metric spaces for $\R_0$ and $\bool$\footnote{The
  metric of $\R$ comes from the first component. Technically, if
  $\R_0$ is defined by \ref{eq:usedbytxa}, this would not make it a
  metric space (as the distance between non-equal elements could be
  $0$); however, this would not matter, and for our definition, there
  is no problem anyway.}.

\end{proposition}



\begin{proof}

The general idea is to interpret our definition in classical mathematics,
assume we have a non-constant function and deduce a
contradiction.
 
Consider the ``naive'' set model (with ``classical standard mathematics'' as meta-theory). This clearly works if we are in a minimalisitic type theory with $\Pi$, $\Sigma$, $\mathsf{W}$, $=$, $\mathbb N$; however, if we restrict ourselves to the types in the lowest universe of homotopy type theory (which is enough), it also works for HoTT.
 We use $\tometa \cdot$ as an interpreation function; 
 for example, we write $\R$ for the field of real numbers which can be defined as $\tometa {\R_0} / \tometa{\sim}$. 
 By abuse of notation, we write $\tometa {\R_0}$ for the set of Cauchy sequences in the model that fulfill the Cauchy condition, without the actual proof thereof. This is justified as this property is propositional. 
 
 For readability, we use the symbol $=$ for equality in the theory as well in the model, and we do not use the semantic brackets for natural numbers such as $2$ or $4$.
 In the model, we use $\clim \cdot : \tometa {\R_0} \to \R$ as the
 function that gives us the limit of a Cauchy sequence (Not all
 functions in the ``naive'' set model have to be continuous).
 Thus, for $r : \R_0$, we write $\climeta r \in \R$ for the real number it represents.

Assume $f, p$ are given. We prove that $\tometa f : \tometa {\R_0} \to \tometa \bool$ is constant in the model, which implies the statement of Proposition \ref{prop:main}.

Thus, assume $\tometa f$ is non-constant,  there are $c_1, c_2 : \tometa {\R_0}$ with $\tometa f(c_1) \not= \tometa f(c_2)$. 

 Define
 \begin{align}
  m_1 & \defeq \mathsf{sup} \{\clim d \in \R \; | \; d \in \tometa {\R_0}, \clim d \leq \mathsf{max}(\clim{c_1}, \clim{c_2}), \tometa f(d) = \tometa \btrue \}\\
  m_2 & \defeq \mathsf{sup} \{\clim d \in \R \; | \; d \in \tometa {\R_0}, \clim d \leq \mathsf{max}(\clim{c_1}, \clim{c_2}), \tometa f(d) = \tometa \bfalse \}
 \end{align}
 (note that one of these two necessarily has to be $\clim{c_1}$ or $\clim{c_2}$, whichever is bigger).
 Set $m \defeq \mathsf{min}(m_1,m_2)$, and we can observe that in
 \emph{every} neighborhood $U$ of $m$, given any $t$,
we can always find another point $x \in U$ such that $x = \clim e$
(for some $e$) with $\tometa f(e) \not= \tometa f(t)$.
 
 Let $c \in \tometa {\R_0}$ be a Cauchy sequence such that $\clim c$ is equal to $m$. 
 We may assume that $c$ satisfies the condition
 $\tometa{\iscauchy_5}$.\footnote{the factor 5 is chosen due to the
   need of a later proof.} 
%  The slight difference ($\frac 1 n$ is replaced by $\frac 1 {4n}$) will be important.
%  Without loss of generality, assume $\tometa{f}(c) = \tometa \btrue$. % not needed.
 
 As $f$ (and thereby $\tometa f$) is continuous (remember the metric
 spaces), there exists $n_0 \in \tometa {\N}$ such that for any Cauchy
 sequence $c' \in \tometa {\R_0}$, if the first $n_0$ sequence
 elements of $c'$ coincide with those of $c$ (namely the distance
 $g(c, c') = 2^{-\mathsf{inf}\{k \in \N \; | \; c(k) \not= c'(k)\}}
 \leq 2^{-{n_0}}$ ), then $\tometa f(c') = \tometa f(c)$. 
 Write $U \subset \tometa {\R_0}$ for the set of Cauchy sequences which fulfill this property, and $\clim U \defeq \{\clim d \; | \; d \in U\}$ for the set of reals that $U$ corresponds to.
 
 We claim that $\clim U$ is a neighborhood of $m$. 
 More precisely, we prove: 
 The interval $I \defeq (m - \frac 1 {2n_0} , m+ \frac 1 {2n_0})$ is contained in $\clim U$.
 Let $x \in \R$ be in $I$. There is a sequence $t : \tometa {\N \to \Q}$ such that $\tometa {\iscauchy_{5n_0}}(t)$ and $\clim t = x$. 
 Let us now ``concatenate'' the first $n_0$ elements of the sequence $c$ with $t$, that is, define
 \begin{align}
  &g : \tometa{\N \to \Q} \\
  &g (n) = \begin{cases}
            c(n) & \text{if $n \leq n_0$} \\
            t(n-n_0) & \text{else}.
           \end{cases}
 \end{align}

 Observe that $g$ is also a Cauchy sequence, i.e. $\tometa \iscauchy(g)$: The only thing that needs to be checked is whether the two ``parts'' of $g$ work well together. Let $0 < n \leq n_0$ and $m > n_0$ be two natural numbers. We need to show that
 \begin{equation}
  |g(n) - g(m)| < \frac 1 n.
 \end{equation}
 Calculate
 \begin{align}
  & \quad |g(n) - g(m)| \\
  = & \quad|c(n) - t(m - n_0)| \\
  = & \quad |c(n) - \clim c + \clim c - \clim t + \clim t - t(m - n_0)| \\
  \leq & \quad | c(n) - \clim c | + |\clim c - \clim t| + |\clim t - t(m - n_0)| \\
  \leq &  \quad  \frac{1}{5n}  + \frac{1}{2n_0} + \frac{1}{5n_0 \cdot (m-n_0)} \\
  \leq &  \quad  \frac{1}{5n}  + \frac{1}{2n} + \frac{1}{5n_0} \\
  <  & \quad \frac 1 n.
 \end{align}

From the continuity property of $f$ and the definition of
$g$ we know that $\tometa f (g) = \tometa f(c)$. Clearly, $\clim g =
\clim t = x \in I$. Therefore \emph{all}  $s \in \tometa {\R_0}$ with $\clim
s \in I$, we can use the "concatenation'' approach to find a $g$
satisfies $s \, \tometa \sim \,  g$ (namely $\clim s = \clim g$), and by the condition that $f$ (and
thereby $\tometa f$) repects $\sim$, we can conclude that $\tometa f
(s) = \tometa f (g) = \tometa f(c)$.
 
%  It is easy to see that $g$ is a Cauchy sequence, i.e. $\tometa \iscauchy(g)$,\footnote{Of course, this is why we needed $\iscauchy_5$ instead of $\iscauchy$ earlier. The $5$ is somewhat arbitrary (todo: think about what is actually needed).} with $\clim g = x$. Further, the first $n_0$ elements of $g$ coincide with those of $c$, and thus, $\tometa f(g) = \tometa f(c)$.
 
However, as we have seen, in \emph{every} neighborhood of $m$, and thus in particular in $(m - \frac 1 {2n_0} , m+ \frac 1 {2n_0})$, there is an $x$ such that $x = \clim e$ (for some $e$) with $\tometa f(e) \not= \tometa f(c)$, in contradiction to the just established statement.
\end{proof}

The proposition that "$\R_0 / \sim$ is connected'' implies the
following corollary:

\begin{corollary}\label{dis:con}
Any continous function from $\R_0$ to any discrete type that respects $\sim$ is constant.
\end{corollary}



\begin{theorem}
Any continuous endofunction $f$ on $\R_0$ that respects $\sim$ which means
 \begin{equation}
  p : \prd{c_1,c_2 : \R_0} c_1 \sim c_2 \to f(c_1) = f(c_2).
 \end{equation}
 is constant (in the sense of corollary \ref{dis:con}).
\end{theorem}
\begin{proof}

Assume we have $f,p$ as required. 

To prove $f$ is constant, it is enough to show that the sequence part is
constant because the proof part is propositional. Again, by slight abuse of notation, we write $\tometa f : \tometa{\R_0} \to \tometa {\R_0}$, omitting the proof part of $f$.
 

Given a postive natural number $n : \tometa \PN$, we have a projection
function $\pi_{n} : \tometa{ \R_0} \to \tometa \Q$. Define a function
$h_n : \tometa {\R_0} \to \tometa \Q$ as

 \begin{equation*}
 h_n(c) \defeq \pi_{n} \circ f
 \end{equation*}

By corollary \ref{dis:con} we know that $h_n$ is constant, hence $f$
is constant everywhere, it is enough to show that $f$ is constant.


% By Proposition \ref{prop:main} we know it is a constant function and
% we can observe that $h (\mathsf 0) = true$, can we infer that $\forall
% c : \tometa {\R_0}, \tometa f c = \tometa f (\mathsf 0)$


%  We only need to show that $\pi_1 \circ f$ (the actual sequence) is constant as the proof of being a Cauchy sequence is propositional.\footnote{Even if $\iscauchy$ is not a propositional predicate (as in \ref{eq:usedbytxa}), it will still be true that $m$ is constant. This is simply because $\sim$ is defined only in terms of the actual sequence part.} Again, by slight abuse of notation, we write $\tometa f : \tometa{\R_0} \to \tometa {\R_0}$, omitting the proof part of $f$.
 
%  Given $c : \tometa {\R_0}$, we want to show $\tometa f (c) = \tometa f (\mathsf 0)$, where $\mathsf 0$ is the sequence that is constantly $0$. 
%  To do so, it is enough to show that, for a given $k : \tometa \N$, we have $\tometa{f}(c)(k) = \tometa{f}(\mathsf 0)(k)$. 
%  If this was not true, we would have a function $\tometa \R \to \tometa \bool$, defined by 
%  \begin{equation*}
%   \lambda c . \mathsf{isEquval }\left(\tometa f (c)(k)\right) \left(\tometa f (\mathsf 0)(k)\right)
%  \end{equation*}
% that is not constant, contradicting Proposition \ref{prop:main}.
\end{proof}


\begin{corollary}
 There is no definable normalisation function on $\R_0$ in the sense
 of \ref{def:nor}
\end{corollary}

\begin{corollary}
 $\R_0 / \sim$ is not definable in the sense of \cite{aan}.
\end{corollary}


However, it doesn't imply that we cannot define the set of real numbers in minimalisitic type theory with $\Pi, \Sigma, \mathsf{W}, =,
\N$. The meaning of definability of real numbers is still not clear enough. To make it more precise, we define
it as whether there is a type $A$ in $\mathsf{TT}$ (minimalisitic type
theory) such that its embedding $\tometa A$ in $\mathsf{TT} + \mathsf{Q}$ (type theory
extended with quotient types) is isomorphic to $\tometa \R_0
/ \tometa \sim$ (where it is a valid type). We conjecture that it is still not
definable.

\begin{proof}
Assume the set of real numbers is definable, we have a type $A$ and
its embedding in $\mathsf{TT} + \mathsf{Q}$  is $\tometa A$. It also
gives us a normalisation function and a representative function
between $\tometa \R_0$ and $\tometa A$.

\end{proof}

\subsection{Cauchy completeness of the cauchy reals}

Whether our definition of cauchy sequence is cauchy complete? In other
words, is there a representative cauchy sequence as a limit for every
equivalence class? The answer is no.

In the HoTT book \cite{hott}, an alternative definition is used
instead which is called cauchy approximation. Because every
approximation has been proven to have a limit, it is cauchy complete.
The definition uses the higher inductive types which will be discussed
in later Chapter \ref{HITs}.

\subsection{Multisets}

\begin{definition}

Multiset. A multiset (or bag) is a set without the constraint that there is no repetitive elements.

\end{definition}

A set is just a special case of multiset when the
\emph{multiplicity}(the number of the occurences) of every element is one.
Multisets (or bags) are believed to be used in ancient times, but it
is only 
studied by mathematicans from 20th century.

In set theory, a multiset is defined as a pair of a set $A$ and a
occurences counting function $m : A \rightarrow \N$.

However in type theory, the set-theoretical "Set" is not available so
we choose another way to define multisets. A multiset can be
represented by some list which accepts multiple occurences like
multiset but is ordered which is a redundant property. In another
word, permutations of a list should represent the same multiset. That
is enough to define an equivalence relation,

Given two lists of type $A$, $p \, q : List A$,

$$ p \sim q = \Sigma f g : \N \rightarrow \N, \forall n, p_n = q_{f \, n}
\wedge q_n = p_{g \, n} $$


\chapter{Real numbers and other undefinable quotients}
\label{rl}

If you mean something specific, always write "the"

In the previous chapter, only definable quotient types are
investigated. But some other
types are undefinable in \itt{} without quotients. In this chapter, the real numbers,
the multisets and
the partiality monad and also the proofs that
the undefinablity of them will be disccussed in detail.


\section{Definability}


\todo{decidable order -> of course definable, A-> N definable, give a
  explicit proof}

\todo{definable -> split quotient because definable is too general}


\section{Real numbers}

We have several choices to represent real numbers. We choose Cauchy
sequences of rational numbers to represent real numbers \cite{bis:85}.

$$\R_{0} = \set{s : \N\to\Q \mid \forall\varepsilon
  :\Q,\varepsilon>0\to\exists m:\N, \forall i:\N, i>m\to \vert  s_i -
  s_m \vert  <\varepsilon}$$

It is implementable in Type Theory, but there is a problem of the
choice of type of the second part of the cauchy sequence, namely the property
that it is a cauchy sequence. Do we distinguish the same sequences
with different proofs? Logically speaking we should not. It means that
we need to truncate it to proposition but we will lose the important
tool to guess what the real number is. To avoid this problem, we could
use an alternative equivalent definition which is a subset of $\R_{0}$:

$$\R_0' = \set{f: \N\to\Q \mid \forall k
  :\N,\forall m , n > k, \to \vert  f_m -
  f_n \vert  < \frac{1}{k}}$$

With the definition, the condition part is propositional and we can
guess the number by applying any number k to the sequence and we know
the interval where it should be located.

Different cauchy sequences can represent the same number. Therefore an equivalence relation\footnote{
The Agda version is in Appendix} is expected. In mathematics two Cauchy sequences $\R_0$ are said to be
equal if their pointwise difference converges to zero,

$$r \sim s = \forall\varepsilon :\Q,\varepsilon>0\to\exists m:\N,
\forall i:\N, i>m\to \vert  r_i - s_i \vert <\varepsilon$$

\subsection{Non-normalizability of Cauchy Sequences}

To prove that it is impossible to give a full definable quotient
structure of real numbers with the setoid of cauchy sequences, we
could show it by proving that it is impossible to define a
normalisation function for the cauchy sequences.

\begin{definition}\label{def:nor}
We say that a quotient structure $\qset{A}$ is definable via a
normalisation, if we have a normalisation function
 \begin{equation}
  nf : A \to A
 \end{equation}
 with the property that it respects $\sim$
\begin{equation}
 p : \prd{c_1,c_2 : A} c_1 \sim c_2 \to nf(c_1) = nf(c_2).
\end{equation}
such that
 \begin{equation}
 q : \prd{c : A}  {nf(c)} \sim c.
 \end{equation}
\end{definition}

It is equivalent to say that we have a definable quotient structure in
the sense of \cite{aan}, because we can form the set of equivalence
classes as

\begin{equation*}
Q \defeq \Sigma_{c : \R_0} nf(c) = c
\end{equation*}

where the second part is propositional, and the "normalisation''
function can be defined as

\begin{equation*}
[c] \defeq nf(c) , p(q)
\end{equation*}

and the embedding function is just the first projection. The properties can be verified
easily.

In the other way around, the true normalisation function is just

\begin{equation*}
n \defeq emb \circ [\_]
\end{equation*}

and the properties hold as well.


We have made an attempt in the original version of our \cite{aan}
draft, but there is something important problem pointed out by Martin Escardo. Laterly, Nicolai
Kraus suggests to fix the proof by proving it as a meta-theoretical
property. We will show an adaption of his proof here.

\textbf{Some preliminaries}

In fact the proof is mainly conducted using topological tools. The
following definitions are helpful for someone who are not so familiar
with topological concepts.


\begin{definition}
Metric space. In mathematics, a metric space is a set where a notion
of distance (called a metric) between elements of the set is defined.

A metric space is an ordered pair $(M , d)$ where M is a set and d is a metric on M:
\begin{enumerate}
\item M is a set,
\item and $d : M \times M \rightarrow \R$ s.t.
\item $d (x , y) = 0 \iff x = y$
\item $d(x,y)=d(y,x)$
\item $d(x,y)+d(y,z) \ge d(x,z)$ 
\end{enumerate}
\end{definition}

We usually define a standard topological structure for discrete types.

\begin{itemize}

\item $(\bool , h)$ where 
$h(m,n) =
\left\{
	\begin{array}{ll}
		0  & \mbox{if } m = n \\
		1 & \mbox{if } m \neq n
	\end{array}
\right.
$

\item $(\N , d)$ where 
$d(m,n) =
\left\{
	\begin{array}{ll}
		0  & \mbox{if } m = n \\
		1 & \mbox{if } m \neq n
	\end{array}
\right.
$

\item $(\Q , e)$ where 
$e(m,n) =
\left\{
	\begin{array}{ll}
		0  & \mbox{if } m = n \\
		1 & \mbox{if } m \neq n
	\end{array}
\right.
$

\end{itemize}

We use a slightly different definition of cauchy sequences of rational
numbers here:

\begin{definition}
 We call a function $f : \PN \to \Q$\footnote{we
   use $\PN$ instead of $\N$ because $n$ must be positive in $\frac 1 n$} a \emph{Cauchy Sequence} if it satisfies
 \begin{equation}
  \iscauchy (f) \defeq \fa{n}{\PN} \fa{m} {\PN} m > n \to |f(n) - f(m)| < \frac 1 n. \label{eq:prop-property-short}
 \end{equation}
 The type of Cauchy Sequences is thus
 \begin{equation*}
  \R_0 \defeq \sm{f : \PN \to \Q} \iscauchy (f).
 \end{equation*}
\end{definition}

And the standard metric space for the sequences $\PN \rightarrow \Q $
is defined by the distance function

\begin{equation}
g(f_1,f_2) = 2^{-\mathsf{inf}\{k \in \N \; | \; f_1(k) \not= f_2(k)\}}
\end{equation}

Among all these standard metric spaces, It is a folklore that all
definable functions are continous.

\begin{theorem}\label{defcon}
All definable functions are continous.
\end{theorem}

Let us introduce the following auxiliary definition:
\begin{definition}
 For a sequence $f : \N \to \Q$, we say that $f$ is \emph{Cauchy with
   factor $k$}, written as $\iscauchy_k$, for some $k \in \Q$, $k > 0$, if
 \begin{equation}
  \iscauchy_k(f) \; \defeq \; \fapairs{n}{m}{\N} m > n \to |f(n) - f(m)| < \frac 1 {k \cdot n}. \label{eq:cauchy-aux}
 \end{equation}
 The usual Cauchy condition $\iscauchy$ is therefore ``Cauchy with factor $1$''.
\end{definition}

\begin{remark} If we claim a function $f$ is defined on $\R_0$ that
  respects $\sim$, it means that we have a proof
\begin{equation}
 p : \prd{c_1,c_2 : \R_0} c_1 \sim c_2 \to f(c_1) = f(c_2).
\end{equation}

\end{remark}

Now we have enough tools to prove the following proposition.

\begin{proposition} \label{prop:main}
 \textbf{"$\R_0 / \sim$" is connected.} It means that any continuous function $f$ 
 \begin{equation}
  f : \R_0 \to \bool
 \end{equation} that respects $\sim$ is constant. We prove that it is impossible to find $c_1, c_2:
\R_0$ such that $f(c_1) \not= f(c_2)$ meta-theoretically, instead of
deriving a proof term of this in type theory.

The definability of function implies that it is a continous function \ref{defcon}
between the standard metric spaces for $\R_0$ and $\bool$\footnote{The
  metric of $\R$ comes from the first component. Technically, if
  $\R_0$ is defined by \ref{eq:usedbytxa}, this would not make it a
  metric space (as the distance between non-equal elements could be
  $0$); however, this would not matter, and for our definition, there
  is no problem anyway.}.

\end{proposition}



\begin{proof}

The general idea is to interpret our definition in classical mathematics,
assume we have a non-constant function and deduce a
contradiction.
 
Consider the ``naive'' set model (with ``classical standard mathematics'' as meta-theory). This clearly works if we are in a minimalisitic type theory with $\Pi$, $\Sigma$, $\mathsf{W}$, $=$, $\mathbb N$; however, if we restrict ourselves to the types in the lowest universe of homotopy type theory (which is enough), it also works for HoTT.
 We use $\tometa \cdot$ as an interpreation function; 
 for example, we write $\R$ for the field of real numbers which can be defined as $\tometa {\R_0} / \tometa{\sim}$. 
 By abuse of notation, we write $\tometa {\R_0}$ for the set of Cauchy sequences in the model that fulfill the Cauchy condition, without the actual proof thereof. This is justified as this property is propositional. 
 
 For readability, we use the symbol $=$ for equality in the theory as well in the model, and we do not use the semantic brackets for natural numbers such as $2$ or $4$.
 In the model, we use $\clim \cdot : \tometa {\R_0} \to \R$ as the
 function that gives us the limit of a Cauchy sequence (Not all
 functions in the ``naive'' set model have to be continuous).
 Thus, for $r : \R_0$, we write $\climeta r \in \R$ for the real number it represents.

Assume $f, p$ are given. We prove that $\tometa f : \tometa {\R_0} \to \tometa \bool$ is constant in the model, which implies the statement of Proposition \ref{prop:main}.

Thus, assume $\tometa f$ is non-constant,  there are $c_1, c_2 : \tometa {\R_0}$ with $\tometa f(c_1) \not= \tometa f(c_2)$. 

 Define
 \begin{align}
  m_1 & \defeq \mathsf{sup} \{\clim d \in \R \; | \; d \in \tometa {\R_0}, \clim d \leq \mathsf{max}(\clim{c_1}, \clim{c_2}), \tometa f(d) = \tometa \btrue \}\\
  m_2 & \defeq \mathsf{sup} \{\clim d \in \R \; | \; d \in \tometa {\R_0}, \clim d \leq \mathsf{max}(\clim{c_1}, \clim{c_2}), \tometa f(d) = \tometa \bfalse \}
 \end{align}
 (note that one of these two necessarily has to be $\clim{c_1}$ or $\clim{c_2}$, whichever is bigger).
 Set $m \defeq \mathsf{min}(m_1,m_2)$, and we can observe that in
 \emph{every} neighborhood $U$ of $m$, given any $t$,
we can always find another point $x \in U$ such that $x = \clim e$
(for some $e$) with $\tometa f(e) \not= \tometa f(t)$.
 
 Let $c \in \tometa {\R_0}$ be a Cauchy sequence such that $\clim c$ is equal to $m$. 
 We may assume that $c$ satisfies the condition
 $\tometa{\iscauchy_5}$.\footnote{the factor 5 is chosen due to the
   need of a later proof.} 
%  The slight difference ($\frac 1 n$ is replaced by $\frac 1 {4n}$) will be important.
%  Without loss of generality, assume $\tometa{f}(c) = \tometa \btrue$. % not needed.
 
 As $f$ (and thereby $\tometa f$) is continuous (remember the metric
 spaces), there exists $n_0 \in \tometa {\N}$ such that for any Cauchy
 sequence $c' \in \tometa {\R_0}$, if the first $n_0$ sequence
 elements of $c'$ coincide with those of $c$ (namely the distance
 $g(c, c') = 2^{-\mathsf{inf}\{k \in \N \; | \; c(k) \not= c'(k)\}}
 \leq 2^{-{n_0}}$ ), then $\tometa f(c') = \tometa f(c)$. 
 Write $U \subset \tometa {\R_0}$ for the set of Cauchy sequences which fulfill this property, and $\clim U \defeq \{\clim d \; | \; d \in U\}$ for the set of reals that $U$ corresponds to.
 
 We claim that $\clim U$ is a neighborhood of $m$. 
 More precisely, we prove: 
 The interval $I \defeq (m - \frac 1 {2n_0} , m+ \frac 1 {2n_0})$ is contained in $\clim U$.
 Let $x \in \R$ be in $I$. There is a sequence $t : \tometa {\N \to \Q}$ such that $\tometa {\iscauchy_{5n_0}}(t)$ and $\clim t = x$. 
 Let us now ``concatenate'' the first $n_0$ elements of the sequence $c$ with $t$, that is, define
 \begin{align}
  &g : \tometa{\N \to \Q} \\
  &g (n) = \begin{cases}
            c(n) & \text{if $n \leq n_0$} \\
            t(n-n_0) & \text{else}.
           \end{cases}
 \end{align}

 Observe that $g$ is also a Cauchy sequence, i.e. $\tometa \iscauchy(g)$: The only thing that needs to be checked is whether the two ``parts'' of $g$ work well together. Let $0 < n \leq n_0$ and $m > n_0$ be two natural numbers. We need to show that
 \begin{equation}
  |g(n) - g(m)| < \frac 1 n.
 \end{equation}
 Calculate
 \begin{align}
  & \quad |g(n) - g(m)| \\
  = & \quad|c(n) - t(m - n_0)| \\
  = & \quad |c(n) - \clim c + \clim c - \clim t + \clim t - t(m - n_0)| \\
  \leq & \quad | c(n) - \clim c | + |\clim c - \clim t| + |\clim t - t(m - n_0)| \\
  \leq &  \quad  \frac{1}{5n}  + \frac{1}{2n_0} + \frac{1}{5n_0 \cdot (m-n_0)} \\
  \leq &  \quad  \frac{1}{5n}  + \frac{1}{2n} + \frac{1}{5n_0} \\
  <  & \quad \frac 1 n.
 \end{align}

From the continuity property of $f$ and the definition of
$g$ we know that $\tometa f (g) = \tometa f(c)$. Clearly, $\clim g =
\clim t = x \in I$. Therefore \emph{all}  $s \in \tometa {\R_0}$ with $\clim
s \in I$, we can use the "concatenation'' approach to find a $g$
satisfies $s \, \tometa \sim \,  g$ (namely $\clim s = \clim g$), and by the condition that $f$ (and
thereby $\tometa f$) repects $\sim$, we can conclude that $\tometa f
(s) = \tometa f (g) = \tometa f(c)$.
 
%  It is easy to see that $g$ is a Cauchy sequence, i.e. $\tometa \iscauchy(g)$,\footnote{Of course, this is why we needed $\iscauchy_5$ instead of $\iscauchy$ earlier. The $5$ is somewhat arbitrary (todo: think about what is actually needed).} with $\clim g = x$. Further, the first $n_0$ elements of $g$ coincide with those of $c$, and thus, $\tometa f(g) = \tometa f(c)$.
 
However, as we have seen, in \emph{every} neighborhood of $m$, and thus in particular in $(m - \frac 1 {2n_0} , m+ \frac 1 {2n_0})$, there is an $x$ such that $x = \clim e$ (for some $e$) with $\tometa f(e) \not= \tometa f(c)$, in contradiction to the just established statement.
\end{proof}

The proposition that "$\R_0 / \sim$ is connected'' implies the
following corollary:

\begin{corollary}\label{dis:con}
Any continous function from $\R_0$ to any discrete type that respects $\sim$ is constant.
\end{corollary}



\begin{theorem}
Any continuous endofunction $f$ on $\R_0$ that respects $\sim$ which means
 \begin{equation}
  p : \prd{c_1,c_2 : \R_0} c_1 \sim c_2 \to f(c_1) = f(c_2).
 \end{equation}
 is constant (in the sense of corollary \ref{dis:con}).
\end{theorem}
\begin{proof}

Assume we have $f,p$ as required. 

To prove $f$ is constant, it is enough to show that the sequence part is
constant because the proof part is propositional. Again, by slight abuse of notation, we write $\tometa f : \tometa{\R_0} \to \tometa {\R_0}$, omitting the proof part of $f$.
 

Given a postive natural number $n : \tometa \PN$, we have a projection
function $\pi_{n} : \tometa{ \R_0} \to \tometa \Q$. Define a function
$h_n : \tometa {\R_0} \to \tometa \Q$ as

 \begin{equation*}
 h_n(c) \defeq \pi_{n} \circ f
 \end{equation*}

By corollary \ref{dis:con} we know that $h_n$ is constant, hence $f$
is constant everywhere, it is enough to show that $f$ is constant.


% By Proposition \ref{prop:main} we know it is a constant function and
% we can observe that $h (\mathsf 0) = true$, can we infer that $\forall
% c : \tometa {\R_0}, \tometa f c = \tometa f (\mathsf 0)$


%  We only need to show that $\pi_1 \circ f$ (the actual sequence) is constant as the proof of being a Cauchy sequence is propositional.\footnote{Even if $\iscauchy$ is not a propositional predicate (as in \ref{eq:usedbytxa}), it will still be true that $m$ is constant. This is simply because $\sim$ is defined only in terms of the actual sequence part.} Again, by slight abuse of notation, we write $\tometa f : \tometa{\R_0} \to \tometa {\R_0}$, omitting the proof part of $f$.
 
%  Given $c : \tometa {\R_0}$, we want to show $\tometa f (c) = \tometa f (\mathsf 0)$, where $\mathsf 0$ is the sequence that is constantly $0$. 
%  To do so, it is enough to show that, for a given $k : \tometa \N$, we have $\tometa{f}(c)(k) = \tometa{f}(\mathsf 0)(k)$. 
%  If this was not true, we would have a function $\tometa \R \to \tometa \bool$, defined by 
%  \begin{equation*}
%   \lambda c . \mathsf{isEquval }\left(\tometa f (c)(k)\right) \left(\tometa f (\mathsf 0)(k)\right)
%  \end{equation*}
% that is not constant, contradicting Proposition \ref{prop:main}.
\end{proof}


\begin{corollary}
 There is no definable normalisation function on $\R_0$ in the sense
 of \ref{def:nor}
\end{corollary}

\begin{corollary}
 $\R_0 / \sim$ is not definable in the sense of \cite{aan}.
\end{corollary}


However, it doesn't imply that we cannot define the set of real numbers in minimalisitic type theory with $\Pi, \Sigma, \mathsf{W}, =,
\N$. The meaning of definability of real numbers is still not clear enough. To make it more precise, we define
it as whether there is a type $A$ in $\mathsf{TT}$ (minimalisitic type
theory) such that its embedding $\tometa A$ in $\mathsf{TT} + \mathsf{Q}$ (type theory
extended with quotient types) is isomorphic to $\tometa \R_0
/ \tometa \sim$ (where it is a valid type). We conjecture that it is still not
definable.

\begin{proof}
Assume the set of real numbers is definable, we have a type $A$ and
its embedding in $\mathsf{TT} + \mathsf{Q}$  is $\tometa A$. It also
gives us a normalisation function and a representative function
between $\tometa \R_0$ and $\tometa A$.
\end{proof}

\begin{proposition}
It is not true that given type $T$, if $T$ is connected then $T$ is contractible.
\end{proposition}
\begin{proof}
As we have seen $ \R_0 / \sim$ is connected. However, it is not
contractible.
Assume it is contractible then there exists $x : \R_0 / \sim$, for
all $y : \R_0 / \sim$, we know $x = y$. Given two uniform sequences
$\bar{0} , \bar{1} : \R_0 / \sim$, we have $\bar{0}  = x = \bar{1}$
which negates $\bar{0} \ne \bar{1}$.
\end{proof}

\subsection{Cauchy completeness of the Cauchy reals}


Is our definition of cauchy reals $\R_0$ Cauchy complete? In other
words, is there a representative Cauchy sequence as a limit for every
equivalence class? The answer is no.

In classical logic, the real number is defined to be the limits. The
limit can be built via a kind of diagonalization
\cite{DBLP:journals/entcs/Lubarsky07}.
While in a constructive setting like type theory here, it fails since
we cannot find a canonical representative for each equivalence
class. Intuitively speaking it is easy to find a canonical choice for
any rational number but it is impossible to find one for any
irrational number. It has been proved by Robert S. Lubarsky in
\cite{DBLP:journals/entcs/Lubarsky07}.

Given the axiom of Countable Choice ($AC_{\omega}$), it is easy to see that we can
find the limit for each equivalence class, namely the Cauchy reals 
are Cauchy complete. However $AC_{\omega}$ is an classical result
which is stronger than the premise "in classical logic''.

In the HoTT book \cite{hott}, an alternative definition is used
instead which is called cauchy approximation. Because every
approximation has been proven to have a limit, it is Cauchy complete.
The definition uses the higher inductive types which will be discussed
in chapter \ref{HITs}.

\section{Multisets}

In mathematics, a multiset (bag) can be seen as a generalised set which
allows multiple occurence of same elements.

\begin{definition}
Multiset. 
A multiset (or bag) is a set without the constraint that there is no repetitive elements.
\end{definition}

In programming languages, a multiset can be seen as an unordered
list. Two lists are bag equivalent \cite{DBLP:conf/itp/Danielsson12} if they are equal up to
reordering. For example, $[1, 2 , 2, 5 ,1]$ is equivalent to
$[2,2,1,1,5]$ since we can find a bijjective mapping between them.
A simple equivalence relation can be defined as follows:

Given two length-$n$ lists (vector) of type $A$, $p \, q : \text{Vec} ~A~n$,
they are equivalent if we have an bijection between them.

$$ p \sim q = \Sigma ~\phi : \text{Fin}\,n \to \text{Fin}\,n, \phi~\text{is a
bijection} \wedge \forall x, p_x = q_{\phi(x)}$$

This equivalence can give rise to a quotient which is just finite
multiset. It is also not definable as a split quotient if it has no well-ordering.

In fact a multiset can be defined as a pair of a set $A$ and a
occurences counting function $m : A \to \N$. However from a practical
perspective, it is not useful because we cannot count the cardinality
of the multiset because $A$ is not necessary to be enumerable. But the
unoroder list represent is countable.


\section{Partiality monad}

The partiality monad is a coninductive type which is available in the
standard library of Agda. It is used to encode delayed computation.

\begin{code}
\\
\>\AgdaKeyword{data} \AgdaDatatype{Delay} \AgdaSymbol{(}\AgdaBound{A} \AgdaSymbol{:} \AgdaPrimitiveType{Set}\AgdaSymbol{)} \AgdaSymbol{:} \AgdaPrimitiveType{Set} \AgdaKeyword{where}\<%
\\
\>[0]\AgdaIndent{2}{}\<[2]%
\>[2]\AgdaInductiveConstructor{now} \AgdaSymbol{:} \AgdaBound{A} \AgdaSymbol{→} \AgdaDatatype{Delay} \AgdaBound{A}\<%
\\
\>[0]\AgdaIndent{2}{}\<[2]%
\>[2]\AgdaInductiveConstructor{later} \AgdaSymbol{:} \AgdaDatatype{∞} \AgdaSymbol{(}\AgdaDatatype{Delay} \AgdaBound{A}\AgdaSymbol{)} \AgdaSymbol{→} \AgdaDatatype{Delay} \AgdaBound{A}\<%
\\
\end{code}

A non-terminating program can be defined in a coinductive way.

\begin{code}
\\
\>\AgdaFunction{never} \AgdaSymbol{:} \AgdaSymbol{\{}\AgdaBound{A} \AgdaSymbol{:} \AgdaPrimitiveType{Set}\AgdaSymbol{\}} \AgdaSymbol{→} \AgdaDatatype{Delay} \AgdaBound{A}\<%
\\
\>\AgdaFunction{never} \AgdaSymbol{=} \AgdaInductiveConstructor{later} \AgdaSymbol{(}\AgdaCoinductiveConstructor{♯} \AgdaFunction{never}\AgdaSymbol{)}\<%
\\
\end{code}

We have two equality for the Delay type: strong bisimilarity and weak
bisimilarity. 
Two computation are strongly bisimilar if they are the
same after the same number of steps delay (there can be infinite
steps).

\begin{code}
\\
\>\AgdaKeyword{data} \AgdaDatatype{\_∼\_} \AgdaSymbol{\{}\AgdaBound{A} \AgdaSymbol{:} \AgdaPrimitiveType{Set}\AgdaSymbol{\}} \AgdaSymbol{:} \AgdaDatatype{Delay} \AgdaBound{A} \AgdaSymbol{→} \AgdaDatatype{Delay} \AgdaBound{A} \AgdaSymbol{→} \AgdaPrimitiveType{Set} \AgdaKeyword{where}\<%
\\
\>[2]\AgdaIndent{4}{}\<[4]%
\>[4]\AgdaInductiveConstructor{now} \<[11]%
\>[11]\AgdaSymbol{:} \AgdaSymbol{∀} \AgdaSymbol{\{}\AgdaBound{x}\AgdaSymbol{\}} \AgdaSymbol{→} \AgdaSymbol{(}\AgdaInductiveConstructor{now} \AgdaBound{x}\AgdaSymbol{)} \AgdaDatatype{∼} \AgdaSymbol{(}\AgdaInductiveConstructor{now} \AgdaBound{x}\AgdaSymbol{)}\<%
\\
\>[2]\AgdaIndent{4}{}\<[4]%
\>[4]\AgdaInductiveConstructor{later} \<[11]%
\>[11]\AgdaSymbol{:} \AgdaSymbol{∀} \AgdaSymbol{\{}\AgdaBound{x} \AgdaBound{y}\AgdaSymbol{\}} \AgdaSymbol{(}\AgdaBound{x∼y} \AgdaSymbol{:} \AgdaDatatype{∞} \AgdaSymbol{((}\AgdaFunction{♭} \AgdaBound{x}\AgdaSymbol{)} \AgdaDatatype{∼} \AgdaSymbol{(}\AgdaFunction{♭} \AgdaBound{y}\AgdaSymbol{)))} \AgdaSymbol{→} \AgdaSymbol{(}\AgdaInductiveConstructor{later} \AgdaBound{x}\AgdaSymbol{)} \AgdaDatatype{∼} \AgdaSymbol{(}\AgdaInductiveConstructor{later} \AgdaBound{y}\AgdaSymbol{)}\<%
\\
\end{code}

We inductively define an operator which states that "x terminates with
y" if we write $\AgdaBound{x}\AgdaSymbol{))} \AgdaDatatype{↓} \AgdaBound{y}$.

\begin{code}
\\
\>\AgdaKeyword{infix} \AgdaNumber{4} \_↓\_\<%
\\
%
\\
\>\AgdaKeyword{data} \AgdaDatatype{\_↓\_} \AgdaSymbol{\{}\AgdaBound{A} \AgdaSymbol{:} \AgdaPrimitiveType{Set}\AgdaSymbol{\}} \AgdaSymbol{:} \AgdaDatatype{Delay} \AgdaBound{A} \AgdaSymbol{→} \AgdaBound{A} \AgdaSymbol{→} \AgdaPrimitiveType{Set} \AgdaKeyword{where}\<%
\\
\>[0]\AgdaIndent{2}{}\<[2]%
\>[2]\AgdaInductiveConstructor{nowT} \<[9]%
\>[9]\AgdaSymbol{:} \AgdaSymbol{∀\{}\AgdaBound{a}\AgdaSymbol{\}} \AgdaSymbol{→} \AgdaSymbol{(}\AgdaInductiveConstructor{now} \AgdaBound{a}\AgdaSymbol{)} \AgdaDatatype{↓} \AgdaBound{a}\<%
\\
\>[0]\AgdaIndent{2}{}\<[2]%
\>[2]\AgdaInductiveConstructor{laterT} \AgdaSymbol{:} \AgdaSymbol{∀\{}\AgdaBound{d} \AgdaBound{a}\AgdaSymbol{\}} \AgdaSymbol{→} \AgdaBound{d} \AgdaDatatype{↓} \AgdaBound{a} \AgdaSymbol{→} \AgdaSymbol{(}\AgdaInductiveConstructor{later} \AgdaSymbol{(}\AgdaCoinductiveConstructor{♯} \AgdaBound{d}\AgdaSymbol{))} \AgdaDatatype{↓} \AgdaBound{a}\<%
\\
\end{code}

And two computation are weakly bisimilar if they terminates with the
same value.

\begin{code}
\\
\>\AgdaKeyword{data} \AgdaDatatype{\_≈\_} \AgdaSymbol{\{}\AgdaBound{A} \AgdaSymbol{:} \AgdaPrimitiveType{Set}\AgdaSymbol{\}} \AgdaSymbol{:} \AgdaDatatype{Delay} \AgdaBound{A} \AgdaSymbol{→} \AgdaDatatype{Delay} \AgdaBound{A} \AgdaSymbol{→} \AgdaPrimitiveType{Set} \AgdaKeyword{where}\<%
\\
\>[2]\AgdaIndent{4}{}\<[4]%
\>[4]\AgdaInductiveConstructor{now} \<[11]%
\>[11]\AgdaSymbol{:} \AgdaSymbol{∀} \AgdaSymbol{\{}\AgdaBound{x} \AgdaBound{y} \AgdaBound{a}\AgdaSymbol{\}} \AgdaSymbol{→} \AgdaBound{x} \AgdaDatatype{↓} \AgdaBound{a} \AgdaSymbol{→} \AgdaBound{y} \AgdaDatatype{↓} \AgdaBound{a} \AgdaSymbol{→} \AgdaBound{x} \AgdaDatatype{≈} \AgdaBound{y}\<%
\\
\>[2]\AgdaIndent{4}{}\<[4]%
\>[4]\AgdaInductiveConstructor{later} \<[11]%
\>[11]\AgdaSymbol{:} \AgdaSymbol{∀} \AgdaSymbol{\{}\AgdaBound{x} \AgdaBound{y}\AgdaSymbol{\}} \AgdaSymbol{(}\AgdaBound{x∼y} \AgdaSymbol{:} \AgdaDatatype{∞} \AgdaSymbol{((}\AgdaFunction{♭} \AgdaBound{x}\AgdaSymbol{)} \AgdaDatatype{≈} \AgdaSymbol{(}\AgdaFunction{♭} \AgdaBound{y}\AgdaSymbol{)))} \AgdaSymbol{→} \AgdaSymbol{(}\AgdaInductiveConstructor{later} \AgdaBound{x}\AgdaSymbol{)} \AgdaDatatype{≈} \AgdaSymbol{(}\AgdaInductiveConstructor{later} \AgdaBound{y}\AgdaSymbol{)}\<%
\\
\end{code}


The quotient derived from the equivalent relation (weak bisimilarity) which represent
the set of all computations can also be a good
example of undefinable quotient.

It is equivalent to another definition inductively on either left side
or right side.


\section{Not all connected type is contractible}

Martin's example is that we can define a type which is proved
different but we cannot find a function from it to \emph{2}.

\todo{Martin Escardo's
  \url{http://www.cs.bham.ac.uk/~mhe/agda/FailureOfTotalSeparatedness.html}
should be considered and discussed here, couterexample of Connected ->
contractible(?)}

conjecture: every quotient definable in pure type theory without
quotient is split

one way to prove it is any connected type is contractible.

the question is still open.


This will implies that reals are not only non-split but also
undefianble in general.


1. why quotients simplifies reasoning. (writened up)
2. Undefinablity
3. omega-groupoids model


%\chapter{The Setoid Model}
\label{sm}
% always forgets the "the"





Altenkirch investigates this issue and gives a solution in
\cite{alti:lics99}. He proposes an extension of \itt by a universe of
propositions $\Prop$ in which all proofs of same propositions are
definitionally equal, namely the theory is proof irrelevant. At the same time,
a setoid model where types are interpreted by a type and an equivalence relation acts as the metatheory and $\eta$-rules for
$\Pi$-types and $\Sigma$-types hold in the metatheory. The extended type
theory generated from the metatheory is decidable and adequate, $Ext$ is
inhabited and it permits large elimination (defining a dependent type by recursion). Within this type theory,
introduction of quotient types is straightforward. 
The set of functions are naturally quotient types, the hidden information is the
definition of the functions and the equivalence relation is the
functional extensionality.







\todo{the Setoid Model}

Quotient types are one of the extensional concepts in Type Theory \cite{hof:phd}. There are several existing intensional models for extensional
concepts. The first one we are going to work with is Altenkirch's
setoid model. To introduce an extensional propositional equality in \itt{}, 
Altenkirch \cite{alti:lics99} proposes an intensional setoid model
with a proof-irrelevant universe of propositions \textbf{Prop}.


\begin{equation}[proof-irr]
\begin{aligned}
\Gamma \vdash P : \textbf{Prop} & & \Gamma \vdash p,q : P & \\
\midrule
& \Gamma \vdash p=q : P & \\
\end{aligned}
\end{equation}

It only contains "propositional'' sets which has at most one
inhabitant. Notice that it is not a definition of types, which means
that we cannot conclude a type is of type \textbf{Prop} if we have a
proof that all
inhabitants are definitionally equal.

The propositional universe is closed under "$\Pi$" and "$\Sigma$", namely dependent functions
and dependent products.

\begin{equation}[\Pi-Prop]
\begin{aligned}
\Gamma \vdash A : \textbf{Set} & & \Gamma,x : A \vdash P \in \textbf{Prop} & \\
\midrule
& \Gamma \vdash \Pi\, x : A.P & \\
\end{aligned}
\end{equation}



\begin{equation}[\Sigma-Prop]
\begin{aligned}
\Gamma \vdash P : \textbf{Prop} & & \Gamma,x : P \vdash Q \in \textbf{Prop} & \\
\midrule
& \Gamma \vdash \Sigma\, x : P.Q & \\
\end{aligned}
\end{equation}



 It is called a setoid model since types are interpreted as setoids.
The solution to introduce the extensional equality is an object type theory defined inside the setoid model which serves as the metatheory. He also proved that the extended type theory generated from the metatheory is decidable and adequate, functional extensionality is
inhabited and it permits large elimination (defining a dependent type by recursion). Within this type theory,
introduction of quotient types is straightforward.

This model is different to a setoid model as an E-category, for instance
the one introduced by Hofmann \cite{hofmann1995interpretation} . An E-category is a category equipped with
an equivalence relation for homsets. To distinguish them, we call this
category \textbf{E-setoids}.  All morphisms of \textbf{E-setoids}
gives rise to types and they are cartesian closed, namely it is a a locally
cartesian closed category (LCCC). Not all morphisms in our category of setoids give
rise to types and it is not an LCCC. Every LCCC can serve as a model for categories with
families but not every category with families has to be an
LCCC. 

\todo{write why this model is not lccc explicitly. refer to Nicolais's
result}



\paragraph{The category of setoids is not a LCCC}

The pullback functor.

\begin{displaymath}
    \xymatrix{X' \ar[r]^{p} \ar[d]_{f^{*}(a)} & Y' \ar[d]^a \\
      X \ar[r]^f& Y }
\end{displaymath}


Observe that $X \rightarrow 1 \cong X$, therefore the pullback of y which is
$X/1 \rightarrow X \times Y / Y$ can be seen as a pullback of X of type $X \rightarrow
X/Y$.

The left adjoint to the pullback functor $f*$ is just the post
composition of $f$ written as $f \circ\_$ or $\Sigma_f$.

\begin{displaymath}
    \xymatrix{X' \ar@{=}[r] \ar[d]_{a} & X' \ar[d]^{\Sigma_f a} \\
      X \ar[r]^f& Y }
\end{displaymath}


However this setoid model is still a model for Type
Theory just like the groupoid model which is a generalisation of it.
To develop this model of type theory in Agda, we have implemented the
categories with families of setoids. 
We build a category with families of setoids to accommodate the types theory described in
\cite{alti:lics99}  so that it is possible to define quotient types
following Martin Hofmann's Paper \cite{hof:95:sm}.  Only necessary
part for the Setoid model will be present here.


\section{An implementation of categories with Families in Agda}

Following the work in \cite{alti:99}, we first define a
proof-irrelevant universe of propositions. We name it as \textbf{hProp}
since \textbf{Prop} is a  reserved word which can't be used and
\textbf{hProp} is a notion from Homotopy Type Theory which we will introduce later.

\section{hProp}

\AgdaHide{

\begin{code}\>\<%
\\
%
\\
\>\AgdaKeyword{open} \AgdaKeyword{import} \AgdaModule{Level}\<%
\\
\>\AgdaKeyword{open} \AgdaKeyword{import} \AgdaModule{Relation.Binary.PropositionalEquality}\<%
\\
%
\\
\>\AgdaKeyword{module} \AgdaModule{hProp} \AgdaSymbol{(}\AgdaBound{ext} \AgdaSymbol{:} \AgdaFunction{Extensionality} \AgdaPrimitive{zero} \AgdaPrimitive{zero}\AgdaSymbol{)} \AgdaKeyword{where}\<%
\\
%
\\
\>\AgdaKeyword{open} \AgdaKeyword{import} \AgdaModule{Relation.Nullary}\<%
\\
\>\AgdaKeyword{open} \AgdaKeyword{import} \AgdaModule{Data.Unit}\<%
\\
\>\AgdaKeyword{open} \AgdaKeyword{import} \AgdaModule{Data.Empty}\<%
\\
\>\AgdaKeyword{open} \AgdaKeyword{import} \AgdaModule{Data.Nat}\<%
\\
\>\AgdaKeyword{open} \AgdaKeyword{import} \AgdaModule{Data.Product}\<%
\\
%
\\
\>\AgdaKeyword{infixr} \AgdaNumber{2} \_⇒\_\<%
\\
%
\\
\>\AgdaKeyword{infixr} \AgdaNumber{3} \_∧\_\<%
\\
%
\\
%
\\
\>\<\end{code}
}

A proof-irrelvant universe only contains sets with at most one inhabitant. 

\begin{code}\>\<%
\\
%
\\
\>\AgdaKeyword{record} \AgdaRecord{hProp} \AgdaSymbol{:} \AgdaPrimitiveType{Set₁} \AgdaKeyword{where}\<%
\\
\>[0]\AgdaIndent{2}{}\<[2]%
\>[2]\AgdaKeyword{constructor} \AgdaInductiveConstructor{hp}\<%
\\
\>[0]\AgdaIndent{2}{}\<[2]%
\>[2]\AgdaKeyword{field}\<%
\\
\>[2]\AgdaIndent{4}{}\<[4]%
\>[4]\AgdaField{prf} \AgdaSymbol{:} \AgdaPrimitiveType{Set}\<%
\\
\>[2]\AgdaIndent{4}{}\<[4]%
\>[4]\AgdaField{Uni} \AgdaSymbol{:} \AgdaSymbol{\{}\AgdaBound{p} \AgdaBound{q} \AgdaSymbol{:} \AgdaBound{prf}\AgdaSymbol{\}} \AgdaSymbol{→} \AgdaBound{p} \AgdaDatatype{≡} \AgdaBound{q}\<%
\\
%
\\
\>\AgdaKeyword{open} \AgdaModule{hProp} \AgdaKeyword{public} \AgdaKeyword{renaming} \AgdaSymbol{(}prf \AgdaSymbol{to} <\_>\AgdaSymbol{)}\<%
\\
%
\\
\>\<\end{code}

We can extract the proof of any propostion $A : hProp$ by using $<>$ and there is always a proof that all inhabitants of it are the same, in other words, if there is any proof of it, the proof is unique. This is not exactly the same as the $Prop$ universe in Altenkirch's approach which is judgemental. It is just a judgement whether a set behaves like a $Proposition$. The $hProp$ we define above is propositional since we can extract the proof of uniqueness.

We would like to have some basic propositions $\top$ and $\bot$. To distinguish them with the ones for non-proof irrelevant propositions which are already available in Agda library, we add a prime to all similar symbols.

\begin{code}\>\<%
\\
%
\\
\>\AgdaFunction{⊤'} \AgdaSymbol{:} \AgdaRecord{hProp}\<%
\\
\>\AgdaFunction{⊤'} \AgdaSymbol{=} \AgdaInductiveConstructor{hp} \AgdaRecord{⊤} \AgdaInductiveConstructor{refl}\<%
\\
%
\\
\>\AgdaFunction{⊥'} \AgdaSymbol{:} \AgdaRecord{hProp}\<%
\\
\>\AgdaFunction{⊥'} \AgdaSymbol{=} \AgdaInductiveConstructor{hp} \AgdaDatatype{⊥} \AgdaSymbol{(λ} \AgdaSymbol{\{}\AgdaBound{p}\AgdaSymbol{\}} \AgdaSymbol{→} \AgdaFunction{⊥-elim} \AgdaBound{p}\AgdaSymbol{)}\<%
\\
%
\\
\>\<\end{code}

We also want the universal and existential quantifier for $hProp$, namely it is closed under $\Pi$-types and $\Sigma$-types.
The universal quantifier of $hProp$ can be axiomitised but we decide to explicitly state that we 
require the functional extensionality to use this module. The reason is that functional extensionality is actually equivalent to the closure under $\Pi$-types.

\begin{code}\>\<%
\\
%
\\
\>\AgdaFunction{∀'} \AgdaSymbol{:} \AgdaSymbol{(}\AgdaBound{A} \AgdaSymbol{:} \AgdaPrimitiveType{Set}\AgdaSymbol{)(}\AgdaBound{P} \AgdaSymbol{:} \AgdaBound{A} \AgdaSymbol{→} \AgdaRecord{hProp}\AgdaSymbol{)} \AgdaSymbol{→} \AgdaRecord{hProp}\<%
\\
\>\AgdaFunction{∀'} \AgdaBound{A} \AgdaBound{P} \AgdaSymbol{=} \AgdaInductiveConstructor{hp} \AgdaSymbol{((}\AgdaBound{x} \AgdaSymbol{:} \AgdaBound{A}\AgdaSymbol{)} \AgdaSymbol{→} \AgdaFunction{<} \AgdaBound{P} \AgdaBound{x} \AgdaFunction{>}\AgdaSymbol{)} \AgdaSymbol{(}\AgdaBound{ext} \AgdaSymbol{(λ} \AgdaBound{x} \AgdaSymbol{→} \AgdaFunction{Uni} \AgdaSymbol{(}\AgdaBound{P} \AgdaBound{x}\AgdaSymbol{)))}\<%
\\
%
\\
\>\<\end{code}


\AgdaHide{
\begin{code}\>\<%
\\
%
\\
\>\AgdaFunction{sig-eq} \AgdaSymbol{:} \AgdaSymbol{\{}\AgdaBound{A} \AgdaSymbol{:} \AgdaPrimitiveType{Set}\AgdaSymbol{\}\{}\AgdaBound{B} \AgdaSymbol{:} \AgdaBound{A} \AgdaSymbol{→} \AgdaPrimitiveType{Set}\AgdaSymbol{\}\{}\AgdaBound{a} \AgdaBound{b} \AgdaSymbol{:} \AgdaBound{A}\AgdaSymbol{\}} \AgdaSymbol{→} \<[43]%
\>[43]\<%
\\
\>[4]\AgdaIndent{9}{}\<[9]%
\>[9]\AgdaSymbol{(}\AgdaBound{p} \AgdaSymbol{:} \AgdaBound{a} \AgdaDatatype{≡} \AgdaBound{b}\AgdaSymbol{)} \AgdaSymbol{→} \<[23]%
\>[23]\<%
\\
\>[4]\AgdaIndent{9}{}\<[9]%
\>[9]\AgdaSymbol{\{}\AgdaBound{c} \AgdaSymbol{:} \AgdaBound{B} \AgdaBound{a}\AgdaSymbol{\}\{}\AgdaBound{d} \AgdaSymbol{:} \AgdaBound{B} \AgdaBound{b}\AgdaSymbol{\}} \AgdaSymbol{→} \<[30]%
\>[30]\<%
\\
\>[4]\AgdaIndent{9}{}\<[9]%
\>[9]\AgdaSymbol{(}\AgdaFunction{subst} \AgdaSymbol{(λ} \AgdaBound{x} \AgdaSymbol{→} \AgdaBound{B} \AgdaBound{x}\AgdaSymbol{)} \AgdaBound{p} \AgdaBound{c} \AgdaDatatype{≡} \AgdaBound{d}\AgdaSymbol{)} \<[37]%
\>[37]\<%
\\
\>[4]\AgdaIndent{9}{}\<[9]%
\>[9]\AgdaSymbol{→} \AgdaDatatype{\_≡\_} \AgdaSymbol{\{\_\}} \AgdaSymbol{\{}\AgdaRecord{Σ} \AgdaBound{A} \AgdaBound{B}\AgdaSymbol{\}} \AgdaSymbol{(}\AgdaBound{a} \AgdaInductiveConstructor{,} \AgdaBound{c}\AgdaSymbol{)} \AgdaSymbol{(}\AgdaBound{b} \AgdaInductiveConstructor{,} \AgdaBound{d}\AgdaSymbol{)}\<%
\\
\>\AgdaFunction{sig-eq} \AgdaInductiveConstructor{refl} \AgdaInductiveConstructor{refl} \AgdaSymbol{=} \AgdaInductiveConstructor{refl}\<%
\\
%
\\
\>\<\end{code}
}

\begin{code}\>\<%
\\
%
\\
%
\\
\>\AgdaFunction{Σ'} \AgdaSymbol{:} \AgdaSymbol{(}\AgdaBound{P} \AgdaSymbol{:} \AgdaRecord{hProp}\AgdaSymbol{)(}\AgdaBound{Q} \AgdaSymbol{:} \AgdaFunction{<} \AgdaBound{P} \AgdaFunction{>} \AgdaSymbol{→} \AgdaRecord{hProp}\AgdaSymbol{)} \AgdaSymbol{→} \AgdaRecord{hProp}\<%
\\
\>\AgdaFunction{Σ'} \AgdaBound{P} \AgdaBound{Q} \AgdaSymbol{=} \AgdaInductiveConstructor{hp} \AgdaSymbol{(}\AgdaRecord{Σ} \AgdaFunction{<} \AgdaBound{P} \AgdaFunction{>} \AgdaSymbol{(λ} \AgdaBound{x} \AgdaSymbol{→} \AgdaFunction{<} \AgdaBound{Q} \AgdaBound{x} \AgdaFunction{>}\AgdaSymbol{))} \<[38]%
\>[38]\<%
\\
\>[9]\AgdaIndent{12}{}\<[12]%
\>[12]\AgdaSymbol{(λ} \AgdaSymbol{\{}\AgdaBound{p}\AgdaSymbol{\}} \AgdaSymbol{\{}\AgdaBound{q}\AgdaSymbol{\}} \AgdaSymbol{→} \<[25]%
\>[25]\<%
\\
\>[9]\AgdaIndent{12}{}\<[12]%
\>[12]\AgdaFunction{sig-eq} \AgdaSymbol{(}\AgdaFunction{Uni} \AgdaBound{P}\AgdaSymbol{)} \AgdaSymbol{(}\AgdaFunction{Uni} \AgdaSymbol{(}\AgdaBound{Q} \AgdaSymbol{(}\AgdaFunction{proj₁} \AgdaBound{q}\AgdaSymbol{))))}\<%
\\
%
\\
\>\<\end{code}

Implication and conjuction which are independent ones of them follow simply.

\begin{code}\>\<%
\\
%
\\
\>\AgdaFunction{\_⇒\_} \AgdaSymbol{:} \AgdaSymbol{(}\AgdaBound{P} \AgdaBound{Q} \AgdaSymbol{:} \AgdaRecord{hProp}\AgdaSymbol{)} \AgdaSymbol{→} \AgdaRecord{hProp}\<%
\\
\>\AgdaBound{P} \AgdaFunction{⇒} \AgdaBound{Q} \AgdaSymbol{=} \<[9]%
\>[9]\AgdaFunction{∀'} \AgdaFunction{<} \AgdaBound{P} \AgdaFunction{>} \AgdaSymbol{(λ} \AgdaBound{\_} \AgdaSymbol{→} \AgdaBound{Q}\AgdaSymbol{)}\<%
\\
%
\\
\>\AgdaFunction{\_∧\_} \AgdaSymbol{:} \AgdaSymbol{(}\AgdaBound{P} \AgdaBound{Q} \AgdaSymbol{:} \AgdaRecord{hProp}\AgdaSymbol{)} \AgdaSymbol{→} \AgdaRecord{hProp}\<%
\\
\>\AgdaBound{P} \AgdaFunction{∧} \AgdaBound{Q} \AgdaSymbol{=} \AgdaFunction{Σ'} \AgdaBound{P} \AgdaSymbol{(λ} \AgdaBound{\_} \AgdaSymbol{→} \AgdaBound{Q}\AgdaSymbol{)}\<%
\\
%
\\
\>\<\end{code}

\AgdaHide{
\begin{code}\>\<%
\\
\>\AgdaKeyword{syntax} ∀' A \AgdaSymbol{(λ} x \AgdaSymbol{→} B\AgdaSymbol{)} \AgdaSymbol{=} ∀'[ x ∶ A ] B


\AgdaKeyword{syntax} Σ' A \AgdaSymbol{(λ} x \AgdaSymbol{→} B\AgdaSymbol{)} \AgdaSymbol{=} Σ'[ x ∶ A ] B

\<\end{code}
}

As long as we have implication and conjuction, more operators on proposition can be defined, for instances negation and logical equivalence.

\begin{code}\>\<%
\\
%
\\
\>\AgdaFunction{¬} \AgdaSymbol{:} \AgdaRecord{hProp} \AgdaSymbol{→} \AgdaRecord{hProp}\<%
\\
\>\AgdaFunction{¬} \AgdaBound{P} \AgdaSymbol{=} \AgdaBound{P} \AgdaFunction{⇒} \AgdaFunction{⊥'} \<[13]%
\>[13]\<%
\\
%
\\
\>\AgdaFunction{\_↔\_} \<[6]%
\>[6]\AgdaSymbol{:} \AgdaSymbol{(}\AgdaBound{P} \AgdaBound{Q} \AgdaSymbol{:} \AgdaRecord{hProp}\AgdaSymbol{)} \AgdaSymbol{→} \AgdaRecord{hProp}\<%
\\
\>\AgdaBound{P} \AgdaFunction{↔} \AgdaBound{Q} \AgdaSymbol{=} \AgdaSymbol{(}\AgdaBound{P} \AgdaFunction{⇒} \AgdaBound{Q}\AgdaSymbol{)} \AgdaFunction{∧} \AgdaSymbol{(}\AgdaBound{Q} \AgdaFunction{⇒} \AgdaBound{P}\AgdaSymbol{)}\<%
\\
%
\\
\>\<\end{code}

\section{Category}

To define category of setoids we should define category first.


\AgdaHide{
\begin{code}\>\<%
\\
%
\\
\>\AgdaSymbol{\{-\#} \AgdaKeyword{OPTIONS} --type-in-type \AgdaSymbol{\#-\}}\<%
\\
%
\\
\>\AgdaKeyword{module} \AgdaModule{Category} \AgdaKeyword{where}\<%
\\
%
\\
\>\AgdaKeyword{open} \AgdaKeyword{import} \AgdaModule{Data.Product}\<%
\\
\>\AgdaKeyword{open} \AgdaKeyword{import} \AgdaModule{Relation.Binary.PropositionalEquality}\<%
\\
%
\\
\>\AgdaKeyword{open} \AgdaKeyword{import} \AgdaModule{Level}\<%
\\
%
\\
\>\<\end{code}
}

\AgdaHide{
\begin{code}\>\<%
\\
%
\\
\>\AgdaKeyword{record} \AgdaRecord{IsCategory}\<%
\\
\>[0]\AgdaIndent{2}{}\<[2]%
\>[2]\AgdaSymbol{(}\AgdaBound{obj} \<[12]%
\>[12]\AgdaSymbol{:} \AgdaPrimitiveType{Set}\AgdaSymbol{)}\<%
\\
%
\\
\>[0]\AgdaIndent{2}{}\<[2]%
\>[2]\AgdaSymbol{(}\AgdaBound{hom} \<[12]%
\>[12]\AgdaSymbol{:} \AgdaBound{obj} \AgdaSymbol{→} \AgdaBound{obj} \AgdaSymbol{→} \AgdaPrimitiveType{Set}\AgdaSymbol{)}\<%
\\
%
\\
\>[0]\AgdaIndent{2}{}\<[2]%
\>[2]\AgdaSymbol{(}\AgdaBound{id} \<[12]%
\>[12]\AgdaSymbol{:} \AgdaSymbol{∀} \AgdaBound{α} \AgdaSymbol{→} \AgdaBound{hom} \AgdaBound{α} \AgdaBound{α}\AgdaSymbol{)}\<%
\\
%
\\
\>[0]\AgdaIndent{2}{}\<[2]%
\>[2]\AgdaSymbol{(}\AgdaBound{[\_⇒\_]\_∘\_} \AgdaSymbol{:} \AgdaSymbol{∀} \AgdaBound{α} \AgdaSymbol{\{}\AgdaBound{β}\AgdaSymbol{\}} \AgdaBound{γ}\<%
\\
\>[2]\AgdaIndent{12}{}\<[12]%
\>[12]\AgdaSymbol{→} \AgdaBound{hom} \AgdaBound{β} \AgdaBound{γ}\<%
\\
\>[2]\AgdaIndent{12}{}\<[12]%
\>[12]\AgdaSymbol{→} \AgdaBound{hom} \AgdaBound{α} \AgdaBound{β}\<%
\\
\>[2]\AgdaIndent{12}{}\<[12]%
\>[12]\AgdaSymbol{→} \AgdaBound{hom} \AgdaBound{α} \AgdaBound{γ}\AgdaSymbol{)}\<%
\\
\>[0]\AgdaIndent{2}{}\<[2]%
\>[2]\AgdaSymbol{:} \AgdaPrimitiveType{Set}\<%
\\
\>[0]\AgdaIndent{2}{}\<[2]%
\>[2]\AgdaKeyword{where}\<%
\\
\>[2]\AgdaIndent{4}{}\<[4]%
\>[4]\AgdaKeyword{constructor} \AgdaInductiveConstructor{IsCatC}\<%
\\
\>[2]\AgdaIndent{4}{}\<[4]%
\>[4]\AgdaKeyword{field}\<%
\\
\>[4]\AgdaIndent{6}{}\<[6]%
\>[6]\AgdaField{id₁} \<[11]%
\>[11]\AgdaSymbol{:} \AgdaSymbol{∀} \AgdaBound{α} \AgdaBound{β} \AgdaSymbol{(}\AgdaBound{f} \AgdaSymbol{:} \AgdaBound{hom} \AgdaBound{α} \AgdaBound{β}\AgdaSymbol{)}\<%
\\
\>[6]\AgdaIndent{11}{}\<[11]%
\>[11]\AgdaSymbol{→} \AgdaBound{[} \AgdaBound{α} \AgdaBound{⇒} \AgdaBound{β} \AgdaBound{]} \AgdaBound{f} \AgdaBound{∘} \AgdaSymbol{(}\AgdaBound{id} \AgdaBound{α}\AgdaSymbol{)} \AgdaDatatype{≡} \AgdaBound{f}\<%
\\
%
\\
\>[0]\AgdaIndent{6}{}\<[6]%
\>[6]\AgdaField{id₂} \<[11]%
\>[11]\AgdaSymbol{:} \AgdaSymbol{∀} \AgdaBound{α} \AgdaBound{β} \AgdaSymbol{(}\AgdaBound{f} \AgdaSymbol{:} \AgdaBound{hom} \AgdaBound{α} \AgdaBound{β}\AgdaSymbol{)}\<%
\\
\>[0]\AgdaIndent{11}{}\<[11]%
\>[11]\AgdaSymbol{→} \AgdaBound{[} \AgdaBound{α} \AgdaBound{⇒} \AgdaBound{β} \AgdaBound{]} \AgdaSymbol{(}\AgdaBound{id} \AgdaBound{β}\AgdaSymbol{)} \AgdaBound{∘} \AgdaBound{f} \AgdaDatatype{≡} \AgdaBound{f}\<%
\\
%
\\
\>[0]\AgdaIndent{6}{}\<[6]%
\>[6]\AgdaField{comp} \AgdaSymbol{:} \AgdaSymbol{∀} \AgdaBound{α} \AgdaSymbol{\{}\AgdaBound{β} \AgdaBound{γ}\AgdaSymbol{\}} \AgdaBound{δ} \AgdaSymbol{(}\AgdaBound{f} \AgdaSymbol{:} \AgdaBound{hom} \AgdaBound{α} \AgdaBound{β}\AgdaSymbol{)} \AgdaSymbol{(}\AgdaBound{g} \AgdaSymbol{:} \AgdaBound{hom} \AgdaBound{β} \AgdaBound{γ}\AgdaSymbol{)} \AgdaSymbol{(}\AgdaBound{h} \AgdaSymbol{:} \AgdaBound{hom} \AgdaBound{γ} \AgdaBound{δ}\AgdaSymbol{)}\<%
\\
\>[0]\AgdaIndent{11}{}\<[11]%
\>[11]\AgdaSymbol{→} \AgdaBound{[} \AgdaBound{α} \AgdaBound{⇒} \AgdaBound{δ} \AgdaBound{]} \AgdaBound{[} \AgdaBound{β} \AgdaBound{⇒} \AgdaBound{δ} \AgdaBound{]} \AgdaBound{h} \AgdaBound{∘} \AgdaBound{g} \AgdaBound{∘} \AgdaBound{f} \AgdaDatatype{≡} \AgdaBound{[} \AgdaBound{α} \AgdaBound{⇒} \AgdaBound{δ} \AgdaBound{]} \AgdaBound{h} \AgdaBound{∘} \AgdaSymbol{(}\AgdaBound{[} \AgdaBound{α} \AgdaBound{⇒} \AgdaBound{γ} \AgdaBound{]} \AgdaBound{g} \AgdaBound{∘} \AgdaBound{f}\AgdaSymbol{)}\<%
\\
%
\\
%
\\
\>\<\end{code}
}


\begin{code}\>\<%
\\
\>\AgdaKeyword{record} \AgdaRecord{Category} \AgdaSymbol{:} \AgdaPrimitiveType{Set} \AgdaKeyword{where}\<%
\\
\>[0]\AgdaIndent{2}{}\<[2]%
\>[2]\AgdaKeyword{constructor} \AgdaInductiveConstructor{CatC}\<%
\\
\>[0]\AgdaIndent{2}{}\<[2]%
\>[2]\AgdaKeyword{field}\<%
\\
\>[2]\AgdaIndent{4}{}\<[4]%
\>[4]\AgdaField{obj} \<[15]%
\>[15]\AgdaSymbol{:} \AgdaPrimitiveType{Set}\<%
\\
%
\\
\>[2]\AgdaIndent{4}{}\<[4]%
\>[4]\AgdaField{hom} \<[15]%
\>[15]\AgdaSymbol{:} \AgdaBound{obj} \AgdaSymbol{→} \AgdaBound{obj} \AgdaSymbol{→} \AgdaPrimitiveType{Set}\<%
\\
%
\\
\>[2]\AgdaIndent{4}{}\<[4]%
\>[4]\AgdaField{id} \<[15]%
\>[15]\AgdaSymbol{:} \AgdaSymbol{∀} \AgdaBound{α}\<%
\\
\>[4]\AgdaIndent{15}{}\<[15]%
\>[15]\AgdaSymbol{→} \AgdaBound{hom} \AgdaBound{α} \AgdaBound{α}\<%
\\
%
\\
\>[0]\AgdaIndent{4}{}\<[4]%
\>[4]\AgdaField{[\_⇒\_]\_∘\_} \<[15]%
\>[15]\AgdaSymbol{:} \AgdaSymbol{∀} \AgdaBound{α} \AgdaSymbol{\{}\AgdaBound{β}\AgdaSymbol{\}} \AgdaBound{γ}\<%
\\
\>[0]\AgdaIndent{15}{}\<[15]%
\>[15]\AgdaSymbol{→} \AgdaBound{hom} \AgdaBound{β} \AgdaBound{γ}\<%
\\
\>[0]\AgdaIndent{15}{}\<[15]%
\>[15]\AgdaSymbol{→} \AgdaBound{hom} \AgdaBound{α} \AgdaBound{β}\<%
\\
\>[0]\AgdaIndent{15}{}\<[15]%
\>[15]\AgdaSymbol{→} \AgdaBound{hom} \AgdaBound{α} \AgdaBound{γ}\<%
\\
%
\\
\>[0]\AgdaIndent{4}{}\<[4]%
\>[4]\AgdaField{isCategory} \AgdaSymbol{:} \AgdaRecord{IsCategory} \AgdaBound{obj} \AgdaBound{hom} \AgdaBound{id} \AgdaBound{[\_⇒\_]\_∘\_}\<%
\\
%
\\
%
\\
\>\<\end{code}

\AgdaHide{
\begin{code}\>\<%
\\
%
\\
\>[0]\AgdaIndent{2}{}\<[2]%
\>[2]\AgdaKeyword{open} \AgdaModule{IsCategory} \AgdaKeyword{public}\<%
\\
%
\\
\>\<\end{code}
}

$isCategory$ contains all the laws for this structure to be a category, for instance the
associativity laws for composition.

\section{Category of setoids}



\AgdaHide{

\begin{code}\>\<%
\\
%
\\
%
\\
\>\AgdaSymbol{\{-\#} \AgdaKeyword{OPTIONS} --type-in-type \AgdaSymbol{\#-\}}\<%
\\
%
\\
\>\AgdaKeyword{open} \AgdaKeyword{import} \AgdaModule{Level}\<%
\\
\>\AgdaKeyword{open} \AgdaKeyword{import} \AgdaModule{Relation.Binary.PropositionalEquality} \AgdaSymbol{as} \AgdaModule{PE} \AgdaKeyword{hiding} \AgdaSymbol{(}refl\AgdaSymbol{;} sym \AgdaSymbol{;} trans\AgdaSymbol{;} isEquivalence\AgdaSymbol{)}\<%
\\
%
\\
\>\AgdaKeyword{module} \AgdaModule{CategoryOfSetoid} \<[25]%
\>[25]\AgdaSymbol{(}\AgdaBound{ext} \AgdaSymbol{:} \AgdaFunction{Extensionality} \AgdaPrimitive{zero} \AgdaPrimitive{zero}\AgdaSymbol{)} \AgdaKeyword{where}\<%
\\
%
\\
\>\AgdaKeyword{open} \AgdaKeyword{import} \AgdaModule{Cats.Category}\<%
\\
\>\AgdaKeyword{open} \AgdaKeyword{import} \AgdaModule{Function}\<%
\\
\>\AgdaKeyword{open} \AgdaKeyword{import} \AgdaModule{Relation.Binary.Core} \AgdaKeyword{using} \AgdaSymbol{(}\_≡\_\AgdaSymbol{)} \AgdaKeyword{renaming} \AgdaSymbol{(}\_⇒\_ \AgdaSymbol{to} \_⇒'\_\AgdaSymbol{)}\<%
\\
\>\AgdaKeyword{open} \AgdaKeyword{import} \AgdaModule{Data.Unit}\<%
\\
\>\AgdaKeyword{open} \AgdaKeyword{import} \AgdaModule{Data.Empty}\<%
\\
\>\AgdaKeyword{import} \AgdaModule{hProp}\<%
\\
\>\AgdaKeyword{open} \AgdaKeyword{module} \AgdaModule{hpx} \AgdaSymbol{=} \AgdaModule{hProp} \AgdaBound{ext}\<%
\\
%
\\
%
\\
\>\AgdaComment{-- Arrow between HSetoid}\<%
\\
%
\\
\>\AgdaKeyword{infix} \AgdaNumber{5} \_⇉\_\<%
\\
%
\\
\>\AgdaComment{-- composition}\<%
\\
%
\\
\>\AgdaKeyword{infixl} \AgdaNumber{5} \_∘c\_\<%
\\
%
\\
\>\<\end{code}
}

Then we could define setoids using \textbf{hProp}. An equivalence relation has three properties reflexivity, symmetry and transitivity. Since we have $refl$ here, we call the reflexivity for propositional equality from the library with prefix as $PE.refl$. 

\begin{code}\>\<%
\\
%
\\
\>\AgdaKeyword{record} \AgdaRecord{ishEquivalence} \AgdaSymbol{\{}\AgdaBound{A} \AgdaSymbol{:} \AgdaPrimitiveType{Set}\AgdaSymbol{\}(}\AgdaBound{\_≈h\_} \AgdaSymbol{:} \AgdaBound{A} \AgdaSymbol{→} \AgdaBound{A} \AgdaSymbol{→} \AgdaRecord{hProp}\AgdaSymbol{)} \AgdaSymbol{:} \AgdaPrimitiveType{Set₁} \AgdaKeyword{where}\<%
\\
\>[0]\AgdaIndent{2}{}\<[2]%
\>[2]\AgdaKeyword{constructor} \AgdaInductiveConstructor{\_,\_,\_}\<%
\\
\>[0]\AgdaIndent{2}{}\<[2]%
\>[2]\AgdaKeyword{field}\<%
\\
\>[2]\AgdaIndent{4}{}\<[4]%
\>[4]\AgdaField{refl} \<[12]%
\>[12]\AgdaSymbol{:} \AgdaSymbol{\{}\AgdaBound{x} \AgdaSymbol{:} \AgdaBound{A}\AgdaSymbol{\}} \AgdaSymbol{→} \AgdaFunction{<} \AgdaBound{x} \AgdaBound{≈h} \AgdaBound{x} \AgdaFunction{>}\<%
\\
\>[2]\AgdaIndent{4}{}\<[4]%
\>[4]\AgdaField{sym} \<[12]%
\>[12]\AgdaSymbol{:} \AgdaSymbol{\{}\AgdaBound{x} \AgdaBound{y} \AgdaSymbol{:} \AgdaBound{A}\AgdaSymbol{\}} \AgdaSymbol{→} \AgdaFunction{<} \AgdaBound{x} \AgdaBound{≈h} \AgdaBound{y} \AgdaFunction{>} \AgdaSymbol{→} \AgdaFunction{<} \AgdaBound{y} \AgdaBound{≈h} \AgdaBound{x} \AgdaFunction{>}\<%
\\
\>[2]\AgdaIndent{4}{}\<[4]%
\>[4]\AgdaField{trans} \<[12]%
\>[12]\AgdaSymbol{:} \AgdaSymbol{\{}\AgdaBound{x} \AgdaBound{y} \AgdaBound{z} \AgdaSymbol{:} \AgdaBound{A}\AgdaSymbol{\}} \AgdaSymbol{→} \AgdaFunction{<} \AgdaBound{x} \AgdaBound{≈h} \AgdaBound{y} \AgdaFunction{>} \AgdaSymbol{→} \AgdaFunction{<} \AgdaBound{y} \AgdaBound{≈h} \AgdaBound{z} \AgdaFunction{>} \AgdaSymbol{→} \AgdaFunction{<} \AgdaBound{x} \AgdaBound{≈h} \AgdaBound{z} \AgdaFunction{>}\<%
\\
%
\\
\>\<\end{code}

Here we use \textbf{hSetoid} as the name because \textbf{Setoid} is
already used for non-proof-irrelvant setoids in the library.
For each setoid, we have a carrier type and an equivalence relation.

\begin{code}\>\<%
\\
\>\AgdaKeyword{record} \AgdaRecord{hSetoid} \AgdaSymbol{:} \AgdaPrimitiveType{Set₁} \AgdaKeyword{where}\<%
\\
\>[0]\AgdaIndent{2}{}\<[2]%
\>[2]\AgdaKeyword{constructor} \AgdaInductiveConstructor{\_,\_,\_}\<%
\\
\>[0]\AgdaIndent{2}{}\<[2]%
\>[2]\AgdaKeyword{infix} \AgdaNumber{4} \_≈h\_ \_≈\_\<%
\\
\>[0]\AgdaIndent{2}{}\<[2]%
\>[2]\AgdaKeyword{field}\<%
\\
\>[2]\AgdaIndent{4}{}\<[4]%
\>[4]\AgdaField{Carrier} \AgdaSymbol{:} \AgdaPrimitiveType{Set}\<%
\\
\>[2]\AgdaIndent{4}{}\<[4]%
\>[4]\AgdaField{\_≈h\_} \<[12]%
\>[12]\AgdaSymbol{:} \AgdaBound{Carrier} \AgdaSymbol{→} \AgdaBound{Carrier} \AgdaSymbol{→} \AgdaRecord{hProp}\<%
\\
\>[2]\AgdaIndent{4}{}\<[4]%
\>[4]\AgdaField{isEquiv} \AgdaSymbol{:} \AgdaRecord{ishEquivalence} \AgdaBound{\_≈h\_}\<%
\\
%
\\
%
\\
\>\<\end{code}

\AgdaHide{
\begin{code}\>\<%
\\
%
\\
\>[0]\AgdaIndent{2}{}\<[2]%
\>[2]\AgdaKeyword{open} \<[8]%
\>[8]\AgdaModule{ishEquivalence} \AgdaFunction{isEquiv} \AgdaKeyword{public}\<%
\\
%
\\
\>[0]\AgdaIndent{2}{}\<[2]%
\>[2]\AgdaFunction{\_≈\_} \AgdaSymbol{:} \AgdaFunction{Carrier} \AgdaSymbol{→} \AgdaFunction{Carrier} \AgdaSymbol{→} \AgdaPrimitiveType{Set}\<%
\\
\>[0]\AgdaIndent{2}{}\<[2]%
\>[2]\AgdaBound{a} \AgdaFunction{≈} \AgdaBound{b} \AgdaSymbol{=} \AgdaFunction{<} \AgdaBound{a} \AgdaFunction{≈h} \AgdaBound{b} \AgdaFunction{>}\<%
\\
\>[0]\AgdaIndent{1}{}\<[1]%
\>[1]\<%
\\
\>[0]\AgdaIndent{2}{}\<[2]%
\>[2]\AgdaFunction{PI} \AgdaSymbol{:} \AgdaSymbol{\{}\AgdaBound{x} \AgdaBound{y} \AgdaSymbol{:} \AgdaFunction{Carrier}\AgdaSymbol{\}\{}\AgdaBound{B} \AgdaSymbol{:} \AgdaPrimitiveType{Set}\AgdaSymbol{\}}\<%
\\
\>[2]\AgdaIndent{7}{}\<[7]%
\>[7]\AgdaSymbol{(}\AgdaBound{A} \AgdaSymbol{:} \AgdaBound{x} \AgdaFunction{≈} \AgdaBound{y} \AgdaSymbol{→} \AgdaBound{B}\AgdaSymbol{)\{}\AgdaBound{p} \AgdaBound{q} \AgdaSymbol{:} \AgdaBound{x} \AgdaFunction{≈} \AgdaBound{y}\AgdaSymbol{\}} \<[36]%
\>[36]\<%
\\
\>[0]\AgdaIndent{5}{}\<[5]%
\>[5]\AgdaSymbol{→} \AgdaBound{A} \AgdaBound{p} \AgdaDatatype{≡} \AgdaBound{A} \AgdaBound{q}\<%
\\
\>[0]\AgdaIndent{2}{}\<[2]%
\>[2]\AgdaFunction{PI} \AgdaSymbol{\{}\AgdaBound{x}\AgdaSymbol{\}} \AgdaSymbol{\{}\AgdaBound{y}\AgdaSymbol{\}} \AgdaBound{A} \AgdaSymbol{\{}\AgdaBound{p}\AgdaSymbol{\}} \AgdaSymbol{\{}\AgdaBound{q}\AgdaSymbol{\}} \AgdaKeyword{with} \AgdaFunction{Uni} \AgdaSymbol{(}\AgdaBound{x} \AgdaFunction{≈h} \AgdaBound{y}\AgdaSymbol{)} \AgdaSymbol{\{}\AgdaBound{p}\AgdaSymbol{\}} \AgdaSymbol{\{}\AgdaBound{q}\AgdaSymbol{\}}\<%
\\
\>[0]\AgdaIndent{2}{}\<[2]%
\>[2]\AgdaFunction{PI} \AgdaBound{A} \AgdaSymbol{|} \AgdaInductiveConstructor{PE.refl} \AgdaSymbol{=} \AgdaInductiveConstructor{PE.refl}\<%
\\
%
\\
\>[0]\AgdaIndent{2}{}\<[2]%
\>[2]\AgdaFunction{reflexive} \AgdaSymbol{:} \AgdaDatatype{\_≡\_} \AgdaFunction{⇒'} \AgdaFunction{\_≈\_}\<%
\\
\>[0]\AgdaIndent{2}{}\<[2]%
\>[2]\AgdaFunction{reflexive} \AgdaInductiveConstructor{PE.refl} \AgdaSymbol{=} \AgdaFunction{refl}\<%
\\
%
\\
\>\AgdaKeyword{open} \AgdaModule{hSetoid} \AgdaKeyword{public} \AgdaKeyword{renaming} \AgdaSymbol{(}refl \AgdaSymbol{to} [\_]refl\AgdaSymbol{;}
     sym \AgdaSymbol{to} [\_]sym\AgdaSymbol{;} \_≈\_ \AgdaSymbol{to} [\_]\_≈\_ \AgdaSymbol{;} \_≈h\_ \AgdaSymbol{to} [\_]\_≈h\_ \AgdaSymbol{;}
     Carrier \AgdaSymbol{to} ∣\_∣ \AgdaSymbol{;} trans \AgdaSymbol{to} [\_]trans\AgdaSymbol{)}\<%
\\
%
\\
%
\\
\>\AgdaFunction{[\_]uip} \AgdaSymbol{:} \AgdaSymbol{∀(}\AgdaBound{Γ} \AgdaSymbol{:} \AgdaRecord{hSetoid}\AgdaSymbol{)\{}\AgdaBound{a} \AgdaBound{b} \AgdaSymbol{:} \AgdaFunction{∣} \AgdaBound{Γ} \AgdaFunction{∣}\AgdaSymbol{\}\{}\AgdaBound{p} \AgdaBound{q} \AgdaSymbol{:} \AgdaFunction{[} \AgdaBound{Γ} \AgdaFunction{]} \AgdaBound{a} \AgdaFunction{≈} \AgdaBound{b}\AgdaSymbol{\}} \AgdaSymbol{→} \AgdaBound{p} \AgdaDatatype{≡} \AgdaBound{q}\<%
\\
\>\AgdaFunction{[} \AgdaBound{Γ} \AgdaFunction{]uip} \AgdaSymbol{\{}\AgdaBound{a}\AgdaSymbol{\}} \AgdaSymbol{\{}\AgdaBound{b}\AgdaSymbol{\}} \AgdaSymbol{=} \AgdaFunction{Uni} \AgdaSymbol{(}\AgdaFunction{[} \AgdaBound{Γ} \AgdaFunction{]} \AgdaBound{a} \AgdaFunction{≈h} \AgdaBound{b}\AgdaSymbol{)}\<%
\\
%
\\
%
\\
\>\<\end{code}
}

A morphism in this category is a function of the underlying sets which respects the equivalence relation. We don't identify the extensional equal functions in the homsets as in \textbf{E-setoids}.

\begin{code}\>\<%
\\
%
\\
\>\AgdaKeyword{record} \AgdaRecord{\_⇉\_} \AgdaSymbol{(}\AgdaBound{A} \AgdaBound{B} \AgdaSymbol{:} \AgdaRecord{hSetoid}\AgdaSymbol{)} \AgdaSymbol{:} \AgdaPrimitiveType{Set₁} \AgdaKeyword{where}\<%
\\
\>[0]\AgdaIndent{2}{}\<[2]%
\>[2]\AgdaKeyword{constructor} \AgdaInductiveConstructor{fn:\_resp:\_}\<%
\\
\>[0]\AgdaIndent{2}{}\<[2]%
\>[2]\AgdaKeyword{field}\<%
\\
\>[2]\AgdaIndent{4}{}\<[4]%
\>[4]\AgdaField{fn} \<[9]%
\>[9]\AgdaSymbol{:} \AgdaFunction{∣} \AgdaBound{A} \AgdaFunction{∣} \AgdaSymbol{→} \AgdaFunction{∣} \AgdaBound{B} \AgdaFunction{∣}\<%
\\
\>[2]\AgdaIndent{4}{}\<[4]%
\>[4]\AgdaField{resp} \AgdaSymbol{:} \AgdaSymbol{\{}\AgdaBound{x} \AgdaBound{y} \AgdaSymbol{:} \AgdaFunction{∣} \AgdaBound{A} \AgdaFunction{∣}\AgdaSymbol{\}} \AgdaSymbol{→} \<[27]%
\>[27]\<%
\\
\>[4]\AgdaIndent{11}{}\<[11]%
\>[11]\AgdaFunction{[} \AgdaBound{A} \AgdaFunction{]} \AgdaBound{x} \AgdaFunction{≈} \AgdaBound{y} \AgdaSymbol{→} \<[25]%
\>[25]\<%
\\
\>[4]\AgdaIndent{11}{}\<[11]%
\>[11]\AgdaFunction{[} \AgdaBound{B} \AgdaFunction{]} \AgdaBound{fn} \AgdaBound{x} \AgdaFunction{≈} \AgdaBound{fn} \AgdaBound{y}\<%
\\
%
\\
\>\<\end{code}

\AgdaHide{
\begin{code}\>\<%
\\
%
\\
\>\AgdaKeyword{open} \AgdaModule{\_⇉\_} \AgdaKeyword{public} \AgdaKeyword{renaming} \AgdaSymbol{(}fn \AgdaSymbol{to} [\_]fn \AgdaSymbol{;} resp \AgdaSymbol{to} [\_]resp\AgdaSymbol{)}\<%
\\
%
\\
\>\<\end{code}
}


The definitions of identity morphism and composition are straightforward and the categorical laws hold trivially as follows.

\begin{code}\>\<%
\\
%
\\
\>\AgdaFunction{id'} \AgdaSymbol{:} \AgdaSymbol{\{}\AgdaBound{Γ} \AgdaSymbol{:} \AgdaRecord{hSetoid}\AgdaSymbol{\}} \AgdaSymbol{→} \AgdaBound{Γ} \AgdaRecord{⇉} \AgdaBound{Γ} \<[28]%
\>[28]\<%
\\
\>\AgdaFunction{id'} \AgdaSymbol{=} \AgdaKeyword{record} \AgdaSymbol{\{} \AgdaField{fn} \AgdaSymbol{=} \AgdaFunction{id}\AgdaSymbol{;} \AgdaField{resp} \AgdaSymbol{=} \AgdaFunction{id}\AgdaSymbol{\}}\<%
\\
%
\\
\>\AgdaFunction{\_∘c\_} \AgdaSymbol{:} \AgdaSymbol{∀\{}\AgdaBound{Γ} \AgdaBound{Δ} \AgdaBound{Z}\AgdaSymbol{\}} \AgdaSymbol{→} \AgdaBound{Δ} \AgdaRecord{⇉} \AgdaBound{Z} \AgdaSymbol{→} \AgdaBound{Γ} \AgdaRecord{⇉} \AgdaBound{Δ} \AgdaSymbol{→} \AgdaBound{Γ} \AgdaRecord{⇉} \AgdaBound{Z}\<%
\\
\>\AgdaBound{yz} \AgdaFunction{∘c} \AgdaBound{xy} \AgdaSymbol{=} \AgdaKeyword{record} \<[18]%
\>[18]\<%
\\
\>[4]\AgdaIndent{11}{}\<[11]%
\>[11]\AgdaSymbol{\{} \AgdaField{fn} \AgdaSymbol{=} \AgdaFunction{[} \AgdaBound{yz} \AgdaFunction{]fn} \AgdaFunction{∘} \AgdaFunction{[} \AgdaBound{xy} \AgdaFunction{]fn}\<%
\\
\>[4]\AgdaIndent{11}{}\<[11]%
\>[11]\AgdaSymbol{;} \AgdaField{resp} \AgdaSymbol{=} \AgdaFunction{[} \AgdaBound{yz} \AgdaFunction{]resp} \AgdaFunction{∘} \AgdaFunction{[} \AgdaBound{xy} \AgdaFunction{]resp}\<%
\\
\>[4]\AgdaIndent{11}{}\<[11]%
\>[11]\AgdaSymbol{\}}\<%
\\
%
\\
\>\AgdaFunction{id₁} \AgdaSymbol{:} \AgdaSymbol{∀} \AgdaBound{Γ} \AgdaBound{Δ} \AgdaSymbol{(}\AgdaBound{ch} \AgdaSymbol{:} \AgdaBound{Γ} \AgdaRecord{⇉} \AgdaBound{Δ}\AgdaSymbol{)} \AgdaSymbol{→} \AgdaBound{ch} \AgdaFunction{∘c} \AgdaFunction{id'} \AgdaDatatype{≡} \AgdaBound{ch}\<%
\\
\>\AgdaFunction{id₁} \AgdaSymbol{\_} \AgdaSymbol{\_} \AgdaBound{ch} \AgdaSymbol{=} \AgdaInductiveConstructor{PE.refl}\<%
\\
%
\\
\>\AgdaFunction{id₂} \AgdaSymbol{:} \AgdaSymbol{∀} \AgdaBound{Γ} \AgdaBound{Δ} \AgdaSymbol{(}\AgdaBound{ch} \AgdaSymbol{:} \AgdaBound{Γ} \AgdaRecord{⇉} \AgdaBound{Δ}\AgdaSymbol{)} \AgdaSymbol{→} \AgdaFunction{id'} \AgdaFunction{∘c} \AgdaBound{ch} \AgdaDatatype{≡} \AgdaBound{ch}\<%
\\
\>\AgdaFunction{id₂} \AgdaSymbol{\_} \AgdaSymbol{\_} \AgdaBound{ch} \AgdaSymbol{=} \AgdaInductiveConstructor{PE.refl}\<%
\\
%
\\
\>\AgdaFunction{comp} \AgdaSymbol{:} \AgdaSymbol{∀} \AgdaBound{Γ} \AgdaSymbol{\{}\AgdaBound{Δ} \AgdaBound{Φ}\AgdaSymbol{\}} \AgdaBound{Ψ} \<[19]%
\>[19]\<%
\\
\>[-2]\AgdaIndent{9}{}\<[9]%
\>[9]\AgdaSymbol{(}\AgdaBound{f} \AgdaSymbol{:} \AgdaBound{Γ} \AgdaRecord{⇉} \AgdaBound{Δ}\AgdaSymbol{)}\<%
\\
\>[0]\AgdaIndent{9}{}\<[9]%
\>[9]\AgdaSymbol{(}\AgdaBound{g} \AgdaSymbol{:} \AgdaBound{Δ} \AgdaRecord{⇉} \AgdaBound{Φ}\AgdaSymbol{)}\<%
\\
\>[0]\AgdaIndent{9}{}\<[9]%
\>[9]\AgdaSymbol{(}\AgdaBound{h} \AgdaSymbol{:} \AgdaBound{Φ} \AgdaRecord{⇉} \AgdaBound{Ψ}\AgdaSymbol{)}\<%
\\
\>[0]\AgdaIndent{7}{}\<[7]%
\>[7]\AgdaSymbol{→} \AgdaBound{h} \AgdaFunction{∘c} \AgdaBound{g} \AgdaFunction{∘c} \AgdaBound{f} \AgdaDatatype{≡} \AgdaBound{h} \AgdaFunction{∘c} \AgdaSymbol{(}\AgdaBound{g} \AgdaFunction{∘c} \AgdaBound{f}\AgdaSymbol{)}\<%
\\
\>\AgdaFunction{comp} \AgdaSymbol{\_} \AgdaSymbol{\_} \AgdaBound{f} \AgdaBound{g} \AgdaBound{h} \AgdaSymbol{=} \AgdaInductiveConstructor{PE.refl}\<%
\\
%
\\
\>\<\end{code}

\AgdaHide{
\begin{code}\>\<%
\\
\>\AgdaComment{\{-
\_f≈\_ :  ∀\{Γ Δ : hSetoid\} → (f g : Γ ⇉ Δ) → hProp
\_f≈\_ \{Γ , \_≈h\_ , (refl , sym , trans)\} \{Δ , \_≈h₁\_ , (refl₁ , sym₁ , trans₁)\} (fn: fn resp: fresp) (fn: gn resp: gresp) 
  = record 
           \{ prf = (g : Γ) → < fn g ≈h₁ gn g >
           ; Uni = ext (λ g → Uni (fn g ≈h₁ gn g))
           \}
-\}}\<%
\\
%
\\
%
\\
\>\<\end{code}
}

Combined all components we obtain the category of setoids.

\begin{code}\>\<%
\\
%
\\
\>\AgdaFunction{setoid-Cat} \AgdaSymbol{:} \AgdaRecord{Category}\<%
\\
\>\AgdaFunction{setoid-Cat} \AgdaSymbol{=} \AgdaInductiveConstructor{CatC} \AgdaRecord{hSetoid} \AgdaRecord{\_⇉\_} \AgdaSymbol{(λ} \AgdaBound{\_} \AgdaSymbol{→} \AgdaFunction{id'}\AgdaSymbol{)} \AgdaSymbol{(λ} \AgdaBound{\_} \AgdaBound{\_} \AgdaSymbol{→} \AgdaFunction{\_∘c\_}\AgdaSymbol{)} \<[57]%
\>[57]\<%
\\
\>[0]\AgdaIndent{13}{}\<[13]%
\>[13]\AgdaSymbol{(}\AgdaInductiveConstructor{IsCatC} \AgdaFunction{id₁} \AgdaFunction{id₂} \AgdaFunction{comp}\AgdaSymbol{)}\<%
\\
%
\\
\>\<\end{code}

This category has a terminal object which is just the unit set with trivial equality. As a terminal object there is precisely one morphism from every object to it.

\begin{code}\>\<%
\\
%
\\
\>\AgdaFunction{⊤-setoid} \AgdaSymbol{:} \AgdaRecord{hSetoid}\<%
\\
\>\AgdaFunction{⊤-setoid} \AgdaSymbol{=} \AgdaKeyword{record} \AgdaSymbol{\{}\<%
\\
\>[0]\AgdaIndent{6}{}\<[6]%
\>[6]\AgdaField{Carrier} \AgdaSymbol{=} \AgdaRecord{⊤}\AgdaSymbol{;}\<%
\\
\>[0]\AgdaIndent{6}{}\<[6]%
\>[6]\AgdaField{\_≈h\_} \<[14]%
\>[14]\AgdaSymbol{=} \AgdaSymbol{λ} \AgdaBound{\_} \AgdaBound{\_} \AgdaSymbol{→} \AgdaFunction{⊤'}\AgdaSymbol{;}\<%
\\
\>[0]\AgdaIndent{6}{}\<[6]%
\>[6]\AgdaField{isEquiv} \AgdaSymbol{=} \AgdaKeyword{record} \AgdaSymbol{\{}\<%
\\
\>[6]\AgdaIndent{8}{}\<[8]%
\>[8]\AgdaField{refl} \<[16]%
\>[16]\AgdaSymbol{=} \AgdaInductiveConstructor{tt}\AgdaSymbol{;}\<%
\\
\>[6]\AgdaIndent{8}{}\<[8]%
\>[8]\AgdaField{sym} \<[16]%
\>[16]\AgdaSymbol{=} \AgdaSymbol{λ} \AgdaBound{\_} \AgdaSymbol{→} \AgdaInductiveConstructor{tt}\AgdaSymbol{;}\<%
\\
\>[6]\AgdaIndent{8}{}\<[8]%
\>[8]\AgdaField{trans} \<[16]%
\>[16]\AgdaSymbol{=} \AgdaSymbol{λ} \AgdaBound{\_} \AgdaBound{\_} \AgdaSymbol{→} \AgdaInductiveConstructor{tt} \AgdaSymbol{\}} \AgdaSymbol{\}}\<%
\\
%
\\
\>\AgdaFunction{⋆} \AgdaSymbol{:} \AgdaSymbol{\{}\AgdaBound{Δ} \AgdaSymbol{:} \AgdaRecord{hSetoid}\AgdaSymbol{\}} \AgdaSymbol{→} \AgdaBound{Δ} \AgdaRecord{⇉} \AgdaFunction{⊤-setoid}\<%
\\
\>\AgdaFunction{⋆} \AgdaSymbol{=} \AgdaKeyword{record} \<[11]%
\>[11]\<%
\\
\>[0]\AgdaIndent{6}{}\<[6]%
\>[6]\AgdaSymbol{\{} \AgdaField{fn} \AgdaSymbol{=} \AgdaSymbol{λ} \AgdaBound{\_} \AgdaSymbol{→} \AgdaInductiveConstructor{tt}\<%
\\
\>[0]\AgdaIndent{6}{}\<[6]%
\>[6]\AgdaSymbol{;} \AgdaField{resp} \AgdaSymbol{=} \AgdaSymbol{λ} \AgdaBound{\_} \AgdaSymbol{→} \AgdaInductiveConstructor{tt} \AgdaSymbol{\}}\<%
\\
%
\\
\>\AgdaFunction{unique⋆} \AgdaSymbol{:} \AgdaSymbol{\{}\AgdaBound{Δ} \AgdaSymbol{:} \AgdaRecord{hSetoid}\AgdaSymbol{\}} \AgdaSymbol{→} \AgdaSymbol{(}\AgdaBound{f} \AgdaSymbol{:} \AgdaBound{Δ} \AgdaRecord{⇉} \AgdaFunction{⊤-setoid}\AgdaSymbol{)} \AgdaSymbol{→} \AgdaBound{f} \AgdaDatatype{≡} \AgdaFunction{⋆}\<%
\\
\>\AgdaFunction{unique⋆} \AgdaBound{f} \AgdaSymbol{=} \AgdaInductiveConstructor{PE.refl}\<%
\\
%
\\
\>\<\end{code}



\section{categories with families of setoids}


A Category with families consists of a base category and a functor
\cite{clairambault2005categories}. We firstly define the categories with
families of sets in Agda  as a guidance for the one for setoids. We
would present the setoid one here since it is relevant.


\AgdaHide{

\begin{code}\>\<%
\\
\>\AgdaSymbol{\{-\#} \AgdaKeyword{OPTIONS} --type-in-type \AgdaSymbol{\#-\}}\<%
\\
%
\\
%
\\
\>\AgdaKeyword{open} \AgdaKeyword{import} \AgdaModule{Level} \AgdaKeyword{hiding} \AgdaSymbol{(}lift\AgdaSymbol{)}\<%
\\
\>\AgdaKeyword{open} \AgdaKeyword{import} \AgdaModule{Relation.Binary.PropositionalEquality} \AgdaSymbol{as} \AgdaModule{PE} \AgdaKeyword{hiding} \AgdaSymbol{(}refl \AgdaSymbol{;} sym \AgdaSymbol{;} trans\AgdaSymbol{;} isEquivalence\AgdaSymbol{;} [\_]\AgdaSymbol{)}\<%
\\
%
\\
\>\AgdaKeyword{module} \AgdaModule{CwF-setoid} \AgdaSymbol{(}\AgdaBound{ext} \AgdaSymbol{:} \AgdaFunction{Extensionality} \AgdaPrimitive{zero} \AgdaPrimitive{zero}\AgdaSymbol{)} \AgdaKeyword{where}\<%
\\
%
\\
%
\\
\>\AgdaKeyword{open} \AgdaKeyword{import} \AgdaModule{Cats.Category}\<%
\\
\>\AgdaKeyword{open} \AgdaKeyword{import} \AgdaModule{Cats.Functor}\<%
\\
\>\AgdaKeyword{open} \AgdaKeyword{import} \AgdaModule{Cats.Duality}\<%
\\
\>\AgdaKeyword{open} \AgdaKeyword{import} \AgdaModule{Data.Product} \AgdaKeyword{renaming} \AgdaSymbol{(}<\_,\_> \AgdaSymbol{to} ⟨\_,\_⟩\AgdaSymbol{)}\<%
\\
\>\AgdaKeyword{open} \AgdaKeyword{import} \AgdaModule{Function}\<%
\\
%
\\
\>\AgdaKeyword{open} \AgdaKeyword{import} \AgdaModule{Relation.Binary.Core} \AgdaKeyword{using} \AgdaSymbol{(}\_≡\_\AgdaSymbol{;} \_≢\_\AgdaSymbol{)}\<%
\\
\>\AgdaKeyword{open} \AgdaKeyword{import} \AgdaModule{Data.Unit}\<%
\\
%
\\
\>\AgdaKeyword{import} \AgdaModule{CategoryOfSetoid}\<%
\\
\>\AgdaKeyword{module} \AgdaModule{cos} \AgdaSymbol{=} \AgdaModule{CategoryOfSetoid} \AgdaBound{ext}\<%
\\
\>\AgdaKeyword{open} \AgdaModule{cos}\<%
\\
%
\\
\>\AgdaKeyword{import} \AgdaModule{hProp}\<%
\\
\>\AgdaKeyword{module} \AgdaModule{hp} \AgdaSymbol{=} \AgdaModule{hProp} \AgdaBound{ext}\<%
\\
\>\AgdaKeyword{open} \AgdaModule{hp}\<%
\\
%
\\
\>\AgdaKeyword{infixl} \AgdaNumber{5} \_\&\_\<%
\\
%
\\
\>\<\end{code}
}

We would like to show two formalisation of category with families for setoids here. The first one is simple and short but not comprehensive. We have to extract all complicated components from the simple definition. However the second one gives these components one by one so that it more understandable and convenient.

The category with families works as a model for type theory. So we will introduce them from a type theoretical point of view.

The base category is the category for contexts. In the setoid version we interpret a context as a setoid as well.

To define the second component, namely the presheaf functor, it is necessary to construct the target category first. The objects of this category are families of setoids.The index setoids are the semantic types and the indexed families of setoids are terms. The morphisms are component-wise morphisms between setoids. All the categorical laws hold trivially.

\begin{code}\>\<%
\\
%
\\
\>\AgdaFunction{inxSetoids} \AgdaSymbol{:} \AgdaPrimitiveType{Set₁}\<%
\\
\>\AgdaFunction{inxSetoids} \AgdaSymbol{=} \AgdaRecord{Σ[} \AgdaBound{I} \AgdaRecord{∶} \AgdaRecord{hSetoid} \AgdaRecord{]} \AgdaSymbol{(}\AgdaFunction{∣} \AgdaBound{I} \AgdaFunction{∣} \AgdaSymbol{→} \AgdaRecord{hSetoid}\AgdaSymbol{)}\<%
\\
%
\\
\>\AgdaFunction{\_⇉setoid\_} \AgdaSymbol{:} \AgdaFunction{inxSetoids} \AgdaSymbol{→} \AgdaFunction{inxSetoids} \AgdaSymbol{→} \AgdaPrimitiveType{Set₁}\<%
\\
\>\AgdaSymbol{(}\AgdaBound{I} \AgdaInductiveConstructor{,} \AgdaBound{f}\AgdaSymbol{)} \AgdaFunction{⇉setoid} \AgdaSymbol{(}\AgdaBound{J} \AgdaInductiveConstructor{,} \AgdaBound{g}\AgdaSymbol{)} \AgdaSymbol{=} \<[26]%
\>[26]\<%
\\
\>[0]\AgdaIndent{2}{}\<[2]%
\>[2]\AgdaRecord{Σ[} \AgdaBound{i-map} \AgdaRecord{∶} \AgdaBound{I} \AgdaRecord{⇉} \AgdaBound{J} \AgdaRecord{]}\<%
\\
\>[2]\AgdaIndent{4}{}\<[4]%
\>[4]\AgdaSymbol{((}\AgdaBound{i} \AgdaSymbol{:} \AgdaFunction{∣} \AgdaBound{I} \AgdaFunction{∣}\AgdaSymbol{)} \AgdaSymbol{→} \AgdaBound{f} \AgdaBound{i} \AgdaRecord{⇉} \AgdaBound{g} \AgdaSymbol{(} \AgdaFunction{[} \AgdaBound{i-map} \AgdaFunction{]fn} \AgdaBound{i}\AgdaSymbol{))}\<%
\\
%
\\
\>\AgdaFunction{Fam-setoid} \AgdaSymbol{:} \AgdaRecord{Category}\<%
\\
\>\AgdaFunction{Fam-setoid} \AgdaSymbol{=} \AgdaInductiveConstructor{CatC} \<[18]%
\>[18]\<%
\\
\>[4]\AgdaIndent{15}{}\<[15]%
\>[15]\AgdaFunction{inxSetoids} \<[26]%
\>[26]\<%
\\
\>[4]\AgdaIndent{15}{}\<[15]%
\>[15]\AgdaFunction{\_⇉setoid\_} \<[25]%
\>[25]\<%
\\
\>[4]\AgdaIndent{15}{}\<[15]%
\>[15]\AgdaSymbol{(λ} \AgdaBound{\_} \AgdaSymbol{→} \AgdaFunction{id'} \AgdaInductiveConstructor{,} \AgdaSymbol{(λ} \AgdaBound{\_} \AgdaSymbol{→} \AgdaFunction{id'}\AgdaSymbol{))} \<[41]%
\>[41]\<%
\\
\>[4]\AgdaIndent{15}{}\<[15]%
\>[15]\AgdaSymbol{(λ} \AgdaSymbol{\{} \AgdaSymbol{\_} \AgdaSymbol{\_} \AgdaSymbol{(}\AgdaBound{fty} \AgdaInductiveConstructor{,} \AgdaBound{ftm}\AgdaSymbol{)} \AgdaSymbol{(}\AgdaBound{gty} \AgdaInductiveConstructor{,} \AgdaBound{gtm}\AgdaSymbol{)} \AgdaSymbol{→} \AgdaBound{fty} \AgdaFunction{∘c} \AgdaBound{gty} \AgdaInductiveConstructor{,}\<%
\\
\>[15]\AgdaIndent{17}{}\<[17]%
\>[17]\AgdaSymbol{(λ} \AgdaBound{i} \AgdaSymbol{→} \AgdaBound{ftm} \AgdaSymbol{(}\AgdaFunction{[} \AgdaBound{gty} \AgdaFunction{]fn} \AgdaBound{i}\AgdaSymbol{)} \AgdaFunction{∘c} \AgdaBound{gtm} \AgdaBound{i}\AgdaSymbol{)\})}\<%
\\
\>[0]\AgdaIndent{15}{}\<[15]%
\>[15]\AgdaSymbol{(}\AgdaInductiveConstructor{IsCatC} \<[23]%
\>[23]\<%
\\
\>[0]\AgdaIndent{17}{}\<[17]%
\>[17]\AgdaSymbol{(λ} \AgdaBound{α} \AgdaBound{β} \AgdaBound{f} \AgdaSymbol{→} \AgdaInductiveConstructor{PE.refl}\AgdaSymbol{)} \<[37]%
\>[37]\<%
\\
\>[0]\AgdaIndent{17}{}\<[17]%
\>[17]\AgdaSymbol{(λ} \AgdaBound{α} \AgdaBound{β} \AgdaBound{f} \AgdaSymbol{→} \AgdaInductiveConstructor{PE.refl}\AgdaSymbol{)} \<[37]%
\>[37]\<%
\\
\>[0]\AgdaIndent{17}{}\<[17]%
\>[17]\AgdaSymbol{(λ} \AgdaBound{α} \AgdaBound{δ} \AgdaBound{f} \AgdaBound{g} \AgdaBound{h} \AgdaSymbol{→} \AgdaInductiveConstructor{PE.refl}\AgdaSymbol{))}\<%
\\
%
\\
\>\<\end{code}

Since we already specify the category of contexts, we only need the presheaf which is a contravariant functor from the category of contexts to the category we defined above. The definition of category with families of setoids could be as simple as follows.

\begin{code}\>\<%
\\
%
\\
\>\AgdaKeyword{record} \AgdaRecord{CWF-setoid} \AgdaSymbol{:} \AgdaPrimitiveType{Set₁} \AgdaKeyword{where}\<%
\\
\>[0]\AgdaIndent{2}{}\<[2]%
\>[2]\AgdaKeyword{field}\<%
\\
\>[0]\AgdaIndent{4}{}\<[4]%
\>[4]\AgdaField{T} \AgdaSymbol{:} \AgdaRecord{Functor} \AgdaSymbol{(}\AgdaFunction{Op} \AgdaFunction{setoid-Cat}\AgdaSymbol{)} \AgdaFunction{Fam-setoid}\<%
\\
%
\\
\>\<\end{code}

All details of this definition are hidden including the functor laws. Therefore we will show the details as the second version.

The semantic contexts are setoids and the terminal object is just the empty context. 

\begin{code}\>\<%
\\
%
\\
\>\AgdaFunction{Con} \AgdaSymbol{=} \AgdaRecord{hSetoid}\<%
\\
%
\\
\>\AgdaFunction{emptyCon} \AgdaSymbol{=} \AgdaFunction{⊤-setoid}\<%
\\
%
\\
\>\AgdaFunction{emptysub} \AgdaSymbol{=} \AgdaFunction{⋆}\<%
\\
%
\\
\>\<\end{code}

A semantic type has following components. $fm$ is a setoid of all types. $substT$ is the substitution between types within the context. It should be a morphism between setoids so it has to preserve the equivalence relation. We also need to specify the computation rules for substitution.

\begin{code}\>\<%
\\
%
\\
\>\AgdaKeyword{record} \AgdaRecord{Ty} \AgdaSymbol{(}\AgdaBound{Γ} \AgdaSymbol{:} \AgdaFunction{Con}\AgdaSymbol{)} \AgdaSymbol{:} \AgdaPrimitiveType{Set₁} \AgdaKeyword{where}\<%
\\
\>[0]\AgdaIndent{2}{}\<[2]%
\>[2]\AgdaKeyword{field}\<%
\\
\>[0]\AgdaIndent{4}{}\<[4]%
\>[4]\AgdaField{fm} \<[11]%
\>[11]\AgdaSymbol{:} \AgdaFunction{∣} \AgdaBound{Γ} \AgdaFunction{∣} \AgdaSymbol{→} \AgdaRecord{hSetoid}\<%
\\
%
\\
\>[0]\AgdaIndent{4}{}\<[4]%
\>[4]\AgdaField{substT} \AgdaSymbol{:} \AgdaSymbol{\{}\AgdaBound{x} \AgdaBound{y} \AgdaSymbol{:} \AgdaFunction{∣} \AgdaBound{Γ} \AgdaFunction{∣}\AgdaSymbol{\}} \AgdaSymbol{→} \<[29]%
\>[29]\<%
\\
\>[4]\AgdaIndent{13}{}\<[13]%
\>[13]\AgdaFunction{[} \AgdaBound{Γ} \AgdaFunction{]} \AgdaBound{x} \AgdaFunction{≈} \AgdaBound{y} \AgdaSymbol{→}\<%
\\
\>[4]\AgdaIndent{13}{}\<[13]%
\>[13]\AgdaFunction{∣} \AgdaBound{fm} \AgdaBound{x} \AgdaFunction{∣} \AgdaSymbol{→}\<%
\\
\>[4]\AgdaIndent{13}{}\<[13]%
\>[13]\AgdaFunction{∣} \AgdaBound{fm} \AgdaBound{y} \AgdaFunction{∣}\<%
\\
\>[0]\AgdaIndent{4}{}\<[4]%
\>[4]\AgdaField{subst*} \AgdaSymbol{:} \AgdaSymbol{∀\{}\AgdaBound{x} \AgdaBound{y} \AgdaSymbol{:} \AgdaFunction{∣} \AgdaBound{Γ} \AgdaFunction{∣}\AgdaSymbol{\}}\<%
\\
\>[0]\AgdaIndent{13}{}\<[13]%
\>[13]\AgdaSymbol{(}\AgdaBound{p} \AgdaSymbol{:} \AgdaFunction{[} \AgdaBound{Γ} \AgdaFunction{]} \AgdaBound{x} \AgdaFunction{≈} \AgdaBound{y}\AgdaSymbol{)}\<%
\\
\>[0]\AgdaIndent{13}{}\<[13]%
\>[13]\AgdaSymbol{\{}\AgdaBound{a} \AgdaBound{b} \AgdaSymbol{:} \AgdaFunction{∣} \AgdaBound{fm} \AgdaBound{x} \AgdaFunction{∣}\AgdaSymbol{\}} \AgdaSymbol{→}\<%
\\
\>[0]\AgdaIndent{13}{}\<[13]%
\>[13]\AgdaFunction{[} \AgdaBound{fm} \AgdaBound{x} \AgdaFunction{]} \AgdaBound{a} \AgdaFunction{≈} \AgdaBound{b} \AgdaSymbol{→}\<%
\\
\>[0]\AgdaIndent{13}{}\<[13]%
\>[13]\AgdaFunction{[} \AgdaBound{fm} \AgdaBound{y} \AgdaFunction{]} \AgdaBound{substT} \AgdaBound{p} \AgdaBound{a} \AgdaFunction{≈} \AgdaBound{substT} \AgdaBound{p} \AgdaBound{b}\<%
\\
%
\\
\>[0]\AgdaIndent{4}{}\<[4]%
\>[4]\AgdaField{refl*} \<[11]%
\>[11]\AgdaSymbol{:} \AgdaSymbol{∀(}\AgdaBound{x} \AgdaSymbol{:} \AgdaFunction{∣} \AgdaBound{Γ} \AgdaFunction{∣}\AgdaSymbol{)}\<%
\\
\>[0]\AgdaIndent{13}{}\<[13]%
\>[13]\AgdaSymbol{(}\AgdaBound{a} \AgdaSymbol{:} \AgdaFunction{∣} \AgdaBound{fm} \AgdaBound{x} \AgdaFunction{∣}\AgdaSymbol{)} \AgdaSymbol{→} \<[30]%
\>[30]\<%
\\
\>[0]\AgdaIndent{13}{}\<[13]%
\>[13]\AgdaFunction{[} \AgdaBound{fm} \AgdaBound{x} \AgdaFunction{]} \AgdaBound{substT} \AgdaFunction{[} \AgdaBound{Γ} \AgdaFunction{]refl} \AgdaBound{a} \AgdaFunction{≈} \AgdaBound{a}\<%
\\
\>[0]\AgdaIndent{4}{}\<[4]%
\>[4]\AgdaField{trans*} \AgdaSymbol{:} \AgdaSymbol{∀\{}\AgdaBound{x} \AgdaBound{y} \AgdaBound{z} \AgdaSymbol{:} \AgdaFunction{∣} \AgdaBound{Γ} \AgdaFunction{∣}\AgdaSymbol{\}}\<%
\\
\>[0]\AgdaIndent{13}{}\<[13]%
\>[13]\AgdaSymbol{(}\AgdaBound{p} \AgdaSymbol{:} \AgdaFunction{[} \AgdaBound{Γ} \AgdaFunction{]} \AgdaBound{x} \AgdaFunction{≈} \AgdaBound{y}\AgdaSymbol{)}\<%
\\
\>[0]\AgdaIndent{13}{}\<[13]%
\>[13]\AgdaSymbol{(}\AgdaBound{q} \AgdaSymbol{:} \AgdaFunction{[} \AgdaBound{Γ} \AgdaFunction{]} \AgdaBound{y} \AgdaFunction{≈} \AgdaBound{z}\AgdaSymbol{)}\<%
\\
\>[0]\AgdaIndent{13}{}\<[13]%
\>[13]\AgdaSymbol{(}\AgdaBound{a} \AgdaSymbol{:} \AgdaFunction{∣} \AgdaBound{fm} \AgdaBound{x} \AgdaFunction{∣}\AgdaSymbol{)} \<[28]%
\>[28]\<%
\\
\>[0]\AgdaIndent{13}{}\<[13]%
\>[13]\AgdaSymbol{→} \AgdaFunction{[} \AgdaBound{fm} \AgdaBound{z} \AgdaFunction{]} \AgdaBound{substT} \AgdaBound{q} \AgdaSymbol{(}\AgdaBound{substT} \AgdaBound{p} \AgdaBound{a}\AgdaSymbol{)} \<[46]%
\>[46]\<%
\\
\>[13]\AgdaIndent{17}{}\<[17]%
\>[17]\AgdaFunction{≈} \AgdaBound{substT} \AgdaSymbol{(}\AgdaFunction{[} \AgdaBound{Γ} \AgdaFunction{]trans} \AgdaBound{p} \AgdaBound{q}\AgdaSymbol{)} \AgdaBound{a}\<%
\\
%
\\
%
\\
\>\<\end{code}

Some other lemmas on the proof irrelevance derived from these fields are not shown here since they are just auxiliary functions.

\AgdaHide{
\begin{code}\>\<%
\\
\>\AgdaComment{-- the proof-irrelevance lemmas for substT}\<%
\\
%
\\
\>[2]\AgdaIndent{2}{}\<[2]%
\>[2]\AgdaFunction{subst-pi} \AgdaSymbol{:} \AgdaSymbol{∀\{}\AgdaBound{x} \AgdaBound{y} \AgdaSymbol{:} \AgdaFunction{∣} \AgdaBound{Γ} \AgdaFunction{∣}\AgdaSymbol{\}}\<%
\\
\>[0]\AgdaIndent{14}{}\<[14]%
\>[14]\AgdaSymbol{\{}\AgdaBound{p} \AgdaBound{q} \AgdaSymbol{:} \AgdaFunction{[} \AgdaBound{Γ} \AgdaFunction{]} \AgdaBound{x} \AgdaFunction{≈} \AgdaBound{y}\AgdaSymbol{\}}\<%
\\
\>[0]\AgdaIndent{14}{}\<[14]%
\>[14]\AgdaSymbol{\{}\AgdaBound{a} \AgdaSymbol{:} \AgdaFunction{∣} \AgdaFunction{fm} \AgdaBound{x} \AgdaFunction{∣}\AgdaSymbol{\}} \AgdaSymbol{→} \AgdaFunction{[} \AgdaFunction{fm} \AgdaBound{y} \AgdaFunction{]} \AgdaFunction{substT} \AgdaBound{p} \AgdaBound{a} \AgdaFunction{≈} \AgdaFunction{substT} \AgdaBound{q} \AgdaBound{a}\<%
\\
\>[0]\AgdaIndent{2}{}\<[2]%
\>[2]\AgdaFunction{subst-pi} \AgdaSymbol{\{}\AgdaBound{x}\AgdaSymbol{\}} \AgdaSymbol{\{}\AgdaBound{y}\AgdaSymbol{\}} \AgdaSymbol{\{}\AgdaBound{p}\AgdaSymbol{\}} \AgdaSymbol{\{}\AgdaBound{q}\AgdaSymbol{\}} \AgdaSymbol{\{}\AgdaBound{a}\AgdaSymbol{\}} \AgdaSymbol{=} \AgdaFunction{reflexive} \AgdaSymbol{(}\AgdaFunction{fm} \AgdaBound{y}\AgdaSymbol{)} \AgdaSymbol{(}\AgdaFunction{PI} \AgdaBound{Γ} \AgdaSymbol{(λ} \AgdaBound{x} \AgdaSymbol{→} \AgdaFunction{substT} \AgdaBound{x} \AgdaBound{a}\AgdaSymbol{))}\<%
\\
%
\\
\>[0]\AgdaIndent{2}{}\<[2]%
\>[2]\AgdaFunction{subst-pi'} \AgdaSymbol{:} \AgdaSymbol{∀\{}\AgdaBound{x} \AgdaSymbol{:} \AgdaFunction{∣} \AgdaBound{Γ} \AgdaFunction{∣}\AgdaSymbol{\}}\<%
\\
\>[2]\AgdaIndent{15}{}\<[15]%
\>[15]\AgdaSymbol{\{}\AgdaBound{p} \AgdaSymbol{:} \AgdaFunction{[} \AgdaBound{Γ} \AgdaFunction{]} \AgdaBound{x} \AgdaFunction{≈} \AgdaBound{x}\AgdaSymbol{\}}\<%
\\
\>[2]\AgdaIndent{15}{}\<[15]%
\>[15]\AgdaSymbol{\{}\AgdaBound{a} \AgdaSymbol{:} \AgdaFunction{∣} \AgdaFunction{fm} \AgdaBound{x} \AgdaFunction{∣}\AgdaSymbol{\}} \AgdaSymbol{→} \AgdaFunction{[} \AgdaFunction{fm} \AgdaBound{x} \AgdaFunction{]} \AgdaFunction{substT} \AgdaBound{p} \AgdaBound{a} \AgdaFunction{≈} \AgdaBound{a}\<%
\\
\>[0]\AgdaIndent{2}{}\<[2]%
\>[2]\AgdaFunction{subst-pi'} \AgdaSymbol{=} \AgdaFunction{[} \AgdaFunction{fm} \AgdaSymbol{\_} \AgdaFunction{]trans} \AgdaFunction{subst-pi} \AgdaSymbol{(}\AgdaFunction{refl*} \AgdaSymbol{\_} \AgdaSymbol{\_)}\<%
\\
%
\\
\>[0]\AgdaIndent{2}{}\<[2]%
\>[2]\AgdaFunction{subst-pi*} \AgdaSymbol{:} \AgdaSymbol{∀\{}\AgdaBound{x} \AgdaBound{y} \AgdaSymbol{:} \AgdaFunction{∣} \AgdaBound{Γ} \AgdaFunction{∣}\AgdaSymbol{\}}\<%
\\
\>[2]\AgdaIndent{16}{}\<[16]%
\>[16]\AgdaSymbol{\{}\AgdaBound{p} \AgdaBound{q} \AgdaSymbol{:} \AgdaFunction{[} \AgdaBound{Γ} \AgdaFunction{]} \AgdaBound{x} \AgdaFunction{≈} \AgdaBound{y}\AgdaSymbol{\}}\<%
\\
\>[2]\AgdaIndent{16}{}\<[16]%
\>[16]\AgdaSymbol{\{}\AgdaBound{a} \AgdaBound{b} \AgdaSymbol{:} \AgdaFunction{∣} \AgdaFunction{fm} \AgdaBound{x} \AgdaFunction{∣}\AgdaSymbol{\}} \AgdaSymbol{→} \AgdaFunction{[} \AgdaFunction{fm} \AgdaBound{x} \AgdaFunction{]} \AgdaBound{a} \AgdaFunction{≈} \AgdaBound{b} \AgdaSymbol{→} \AgdaFunction{[} \AgdaFunction{fm} \AgdaBound{y} \AgdaFunction{]} \AgdaFunction{substT} \AgdaBound{p} \AgdaBound{a} \AgdaFunction{≈} \AgdaFunction{substT} \AgdaBound{q} \AgdaBound{b}\<%
\\
\>[0]\AgdaIndent{2}{}\<[2]%
\>[2]\AgdaFunction{subst-pi*} \AgdaBound{eq} \AgdaSymbol{=} \AgdaFunction{[} \AgdaFunction{fm} \AgdaSymbol{\_} \AgdaFunction{]trans} \AgdaSymbol{(}\AgdaFunction{subst*} \AgdaSymbol{\_} \AgdaBound{eq}\AgdaSymbol{)} \AgdaFunction{subst-pi}\<%
\\
%
\\
%
\\
\>\AgdaComment{-- simplify proofs of trans of inverse equality (including groupoid laws?)}\<%
\\
%
\\
\>[0]\AgdaIndent{2}{}\<[2]%
\>[2]\AgdaFunction{trans-refl} \AgdaSymbol{:} \AgdaSymbol{∀\{}\AgdaBound{x} \AgdaBound{y} \AgdaSymbol{:} \AgdaFunction{∣} \AgdaBound{Γ} \AgdaFunction{∣}\AgdaSymbol{\}}\<%
\\
\>[2]\AgdaIndent{14}{}\<[14]%
\>[14]\AgdaSymbol{\{}\AgdaBound{p} \AgdaSymbol{:} \AgdaFunction{[} \AgdaBound{Γ} \AgdaFunction{]} \AgdaBound{x} \AgdaFunction{≈} \AgdaBound{y}\AgdaSymbol{\}\{}\AgdaBound{q} \AgdaSymbol{:} \AgdaFunction{[} \AgdaBound{Γ} \AgdaFunction{]} \AgdaBound{y} \AgdaFunction{≈} \AgdaBound{x}\AgdaSymbol{\}}\<%
\\
\>[2]\AgdaIndent{14}{}\<[14]%
\>[14]\AgdaSymbol{\{}\AgdaBound{a} \AgdaSymbol{:} \AgdaFunction{∣} \AgdaFunction{fm} \AgdaBound{x} \AgdaFunction{∣}\AgdaSymbol{\}} \AgdaSymbol{→} \<[31]%
\>[31]\<%
\\
\>[2]\AgdaIndent{14}{}\<[14]%
\>[14]\AgdaFunction{[} \AgdaFunction{fm} \AgdaBound{x} \AgdaFunction{]} \AgdaFunction{substT} \AgdaBound{q} \AgdaSymbol{(}\AgdaFunction{substT} \AgdaBound{p} \AgdaBound{a}\AgdaSymbol{)} \AgdaFunction{≈} \AgdaBound{a}\<%
\\
\>[0]\AgdaIndent{2}{}\<[2]%
\>[2]\AgdaFunction{trans-refl} \AgdaSymbol{=} \AgdaFunction{[} \AgdaFunction{fm} \AgdaSymbol{\_} \AgdaFunction{]trans} \AgdaSymbol{(}\AgdaFunction{trans*} \AgdaSymbol{\_} \AgdaSymbol{\_} \AgdaSymbol{\_)} \AgdaFunction{subst-pi'}\<%
\\
%
\\
\>\AgdaComment{-- some more theorems}\<%
\\
\>[0]\AgdaIndent{2}{}\<[2]%
\>[2]\<%
\\
\>[0]\AgdaIndent{2}{}\<[2]%
\>[2]\AgdaFunction{subst-mir1} \AgdaSymbol{:} \AgdaSymbol{∀\{}\AgdaBound{x} \AgdaBound{y} \AgdaSymbol{:} \AgdaFunction{∣} \AgdaBound{Γ} \AgdaFunction{∣}\AgdaSymbol{\}}\<%
\\
\>[2]\AgdaIndent{14}{}\<[14]%
\>[14]\AgdaSymbol{\{}\AgdaBound{p} \AgdaSymbol{:} \AgdaFunction{[} \AgdaBound{Γ} \AgdaFunction{]} \AgdaBound{x} \AgdaFunction{≈} \AgdaBound{y}\AgdaSymbol{\}\{}\AgdaBound{q} \AgdaSymbol{:} \AgdaFunction{[} \AgdaBound{Γ} \AgdaFunction{]} \AgdaBound{y} \AgdaFunction{≈} \AgdaBound{x}\AgdaSymbol{\}}\<%
\\
\>[2]\AgdaIndent{14}{}\<[14]%
\>[14]\AgdaSymbol{\{}\AgdaBound{a} \AgdaSymbol{:} \AgdaFunction{∣} \AgdaFunction{fm} \AgdaBound{x} \AgdaFunction{∣}\AgdaSymbol{\}\{}\AgdaBound{b} \AgdaSymbol{:} \AgdaFunction{∣} \AgdaFunction{fm} \AgdaBound{y} \AgdaFunction{∣}\AgdaSymbol{\}} \AgdaSymbol{→} \<[45]%
\>[45]\<%
\\
\>[2]\AgdaIndent{14}{}\<[14]%
\>[14]\AgdaFunction{[} \AgdaFunction{fm} \AgdaBound{x} \AgdaFunction{]} \AgdaBound{a} \AgdaFunction{≈} \AgdaFunction{substT} \AgdaBound{q} \AgdaBound{b} \AgdaSymbol{→} \AgdaFunction{[} \AgdaFunction{fm} \AgdaBound{y} \AgdaFunction{]} \AgdaFunction{substT} \AgdaBound{p} \AgdaBound{a} \AgdaFunction{≈} \AgdaBound{b}\<%
\\
\>[0]\AgdaIndent{2}{}\<[2]%
\>[2]\AgdaFunction{subst-mir1} \AgdaBound{eq} \AgdaSymbol{=} \AgdaFunction{[} \AgdaFunction{fm} \AgdaSymbol{\_} \AgdaFunction{]trans} \AgdaSymbol{(}\AgdaFunction{subst*} \AgdaSymbol{\_} \AgdaBound{eq}\AgdaSymbol{)} \AgdaFunction{trans-refl}\<%
\\
%
\\
\>[0]\AgdaIndent{2}{}\<[2]%
\>[2]\AgdaFunction{subst-mir2} \AgdaSymbol{:} \AgdaSymbol{∀\{}\AgdaBound{x} \AgdaBound{y} \AgdaSymbol{:} \AgdaFunction{∣} \AgdaBound{Γ} \AgdaFunction{∣}\AgdaSymbol{\}}\<%
\\
\>[2]\AgdaIndent{14}{}\<[14]%
\>[14]\AgdaSymbol{\{}\AgdaBound{p} \AgdaSymbol{:} \AgdaFunction{[} \AgdaBound{Γ} \AgdaFunction{]} \AgdaBound{x} \AgdaFunction{≈} \AgdaBound{y}\AgdaSymbol{\}\{}\AgdaBound{q} \AgdaSymbol{:} \AgdaFunction{[} \AgdaBound{Γ} \AgdaFunction{]} \AgdaBound{y} \AgdaFunction{≈} \AgdaBound{x}\AgdaSymbol{\}}\<%
\\
\>[2]\AgdaIndent{14}{}\<[14]%
\>[14]\AgdaSymbol{\{}\AgdaBound{a} \AgdaSymbol{:} \AgdaFunction{∣} \AgdaFunction{fm} \AgdaBound{x} \AgdaFunction{∣}\AgdaSymbol{\}\{}\AgdaBound{b} \AgdaSymbol{:} \AgdaFunction{∣} \AgdaFunction{fm} \AgdaBound{y} \AgdaFunction{∣}\AgdaSymbol{\}} \AgdaSymbol{→} \<[45]%
\>[45]\<%
\\
\>[2]\AgdaIndent{14}{}\<[14]%
\>[14]\AgdaFunction{[} \AgdaFunction{fm} \AgdaBound{y} \AgdaFunction{]} \AgdaFunction{substT} \AgdaBound{p} \AgdaBound{a} \AgdaFunction{≈} \AgdaBound{b} \AgdaSymbol{→} \AgdaFunction{[} \AgdaFunction{fm} \AgdaBound{x} \AgdaFunction{]} \AgdaBound{a} \AgdaFunction{≈} \AgdaFunction{substT} \AgdaBound{q} \AgdaBound{b}\<%
\\
\>[0]\AgdaIndent{2}{}\<[2]%
\>[2]\AgdaFunction{subst-mir2} \AgdaBound{eq} \AgdaSymbol{=} \AgdaFunction{[} \AgdaFunction{fm} \AgdaSymbol{\_} \AgdaFunction{]sym} \AgdaSymbol{(}\AgdaFunction{subst-mir1} \AgdaSymbol{(}\AgdaFunction{[} \AgdaFunction{fm} \AgdaSymbol{\_} \AgdaFunction{]sym} \AgdaBound{eq}\AgdaSymbol{))}\<%
\\
%
\\
\>\AgdaKeyword{open} \AgdaModule{Ty} \AgdaKeyword{public} \<[15]%
\>[15]\<%
\\
\>[0]\AgdaIndent{2}{}\<[2]%
\>[2]\AgdaKeyword{renaming} \AgdaSymbol{(}substT \AgdaSymbol{to} [\_]subst\AgdaSymbol{;} subst* \AgdaSymbol{to} [\_]subst*\AgdaSymbol{;} fm \AgdaSymbol{to} [\_]fm \AgdaSymbol{;}
            refl* \AgdaSymbol{to} [\_]refl* \AgdaSymbol{;} trans* \AgdaSymbol{to} [\_]trans*\AgdaSymbol{;} subst-pi \AgdaSymbol{to} [\_]subst-pi \AgdaSymbol{;}
            subst-pi' \AgdaSymbol{to} [\_]subst-pi' \AgdaSymbol{;} subst-pi* \AgdaSymbol{to} [\_]subst-pi* \AgdaSymbol{;}
            trans-refl \AgdaSymbol{to} [\_]trans-refl \AgdaSymbol{;} subst-mir1 \AgdaSymbol{to} [\_]subst-mir1 \AgdaSymbol{;}
            subst-mir2 \AgdaSymbol{to} [\_]subst-mir2\AgdaSymbol{)}\<%
\\
%
\\
\>\<\end{code}
}

Then we have to define the substituting in a type given a context morphism and verify it preserves equivalence relation as well.

\begin{code}\>\<%
\\
%
\\
\>\AgdaFunction{\_[\_]T} \AgdaSymbol{:} \AgdaSymbol{∀} \AgdaSymbol{\{}\AgdaBound{Γ} \AgdaBound{Δ} \AgdaSymbol{:} \AgdaFunction{Con}\AgdaSymbol{\}} \AgdaSymbol{→} \AgdaRecord{Ty} \AgdaBound{Δ} \AgdaSymbol{→} \AgdaBound{Γ} \AgdaRecord{⇉} \AgdaBound{Δ} \AgdaSymbol{→} \AgdaRecord{Ty} \AgdaBound{Γ}\<%
\\
\>\AgdaBound{A} \AgdaFunction{[} \AgdaBound{f} \AgdaFunction{]T}\<%
\\
\>[2]\AgdaIndent{5}{}\<[5]%
\>[5]\AgdaSymbol{=} \AgdaKeyword{record}\<%
\\
\>[2]\AgdaIndent{5}{}\<[5]%
\>[5]\AgdaSymbol{\{} \AgdaField{fm} \<[14]%
\>[14]\AgdaSymbol{=} \AgdaFunction{fm} \AgdaFunction{∘} \AgdaFunction{fn}\<%
\\
\>[2]\AgdaIndent{5}{}\<[5]%
\>[5]\AgdaSymbol{;} \AgdaField{substT} \AgdaSymbol{=} \AgdaFunction{substT} \AgdaFunction{∘} \AgdaFunction{resp}\<%
\\
\>[2]\AgdaIndent{5}{}\<[5]%
\>[5]\AgdaSymbol{;} \AgdaField{subst*} \AgdaSymbol{=} \AgdaFunction{subst*} \AgdaFunction{∘} \AgdaFunction{resp}\<%
\\
\>[2]\AgdaIndent{5}{}\<[5]%
\>[5]\AgdaSymbol{;} \AgdaField{refl*} \<[14]%
\>[14]\AgdaSymbol{=} \AgdaSymbol{λ} \AgdaBound{\_} \AgdaBound{\_} \AgdaSymbol{→} \AgdaFunction{subst-pi'}\<%
\\
\>[2]\AgdaIndent{5}{}\<[5]%
\>[5]\AgdaSymbol{;} \AgdaField{trans*} \AgdaSymbol{=} \AgdaSymbol{λ} \AgdaBound{\_} \AgdaBound{\_} \AgdaBound{\_} \AgdaSymbol{→} \<[26]%
\>[26]\<%
\\
\>[5]\AgdaIndent{16}{}\<[16]%
\>[16]\AgdaFunction{[} \AgdaFunction{fm} \AgdaSymbol{(}\AgdaFunction{fn} \AgdaSymbol{\_)} \AgdaFunction{]trans} \AgdaSymbol{(}\AgdaFunction{trans*} \AgdaSymbol{\_} \AgdaSymbol{\_} \AgdaSymbol{\_)} \AgdaFunction{subst-pi}\<%
\\
\>[0]\AgdaIndent{5}{}\<[5]%
\>[5]\AgdaSymbol{\}}\<%
\\
\>[0]\AgdaIndent{5}{}\<[5]%
\>[5]\AgdaKeyword{where} \<[11]%
\>[11]\<%
\\
\>[5]\AgdaIndent{7}{}\<[7]%
\>[7]\AgdaKeyword{open} \AgdaModule{Ty} \AgdaBound{A}\<%
\\
\>[5]\AgdaIndent{7}{}\<[7]%
\>[7]\AgdaKeyword{open} \AgdaModule{\_⇉\_} \AgdaBound{f}\<%
\\
%
\\
\>\<\end{code}

The semantic terms are simpler. It should also preserve the equivalence relation on the elements of contexts.

\begin{code}\>\<%
\\
%
\\
\>\AgdaKeyword{record} \AgdaRecord{Tm} \AgdaSymbol{\{}\AgdaBound{Γ} \AgdaSymbol{:} \AgdaFunction{Con}\AgdaSymbol{\}(}\AgdaBound{A} \AgdaSymbol{:} \AgdaRecord{Ty} \AgdaBound{Γ}\AgdaSymbol{)} \AgdaSymbol{:} \AgdaPrimitiveType{Set} \AgdaKeyword{where}\<%
\\
\>[0]\AgdaIndent{2}{}\<[2]%
\>[2]\AgdaKeyword{constructor} \AgdaInductiveConstructor{tm:\_resp:\_}\<%
\\
\>[0]\AgdaIndent{2}{}\<[2]%
\>[2]\AgdaKeyword{field}\<%
\\
\>[2]\AgdaIndent{4}{}\<[4]%
\>[4]\AgdaField{tm} \<[10]%
\>[10]\AgdaSymbol{:} \AgdaSymbol{(}\AgdaBound{x} \AgdaSymbol{:} \AgdaFunction{∣} \AgdaBound{Γ} \AgdaFunction{∣}\AgdaSymbol{)} \AgdaSymbol{→} \AgdaFunction{∣} \AgdaFunction{[} \AgdaBound{A} \AgdaFunction{]fm} \AgdaBound{x} \AgdaFunction{∣}\<%
\\
\>[2]\AgdaIndent{4}{}\<[4]%
\>[4]\AgdaField{respt} \AgdaSymbol{:} \AgdaSymbol{∀} \AgdaSymbol{\{}\AgdaBound{x} \AgdaBound{y} \AgdaSymbol{:} \AgdaFunction{∣} \AgdaBound{Γ} \AgdaFunction{∣}\AgdaSymbol{\}} \AgdaSymbol{→} \<[30]%
\>[30]\<%
\\
\>[4]\AgdaIndent{14}{}\<[14]%
\>[14]\AgdaSymbol{(}\AgdaBound{p} \AgdaSymbol{:} \AgdaFunction{[} \AgdaBound{Γ} \AgdaFunction{]} \AgdaBound{x} \AgdaFunction{≈} \AgdaBound{y}\AgdaSymbol{)} \AgdaSymbol{→} \<[34]%
\>[34]\<%
\\
\>[4]\AgdaIndent{14}{}\<[14]%
\>[14]\AgdaFunction{[} \AgdaFunction{[} \AgdaBound{A} \AgdaFunction{]fm} \AgdaBound{y} \AgdaFunction{]} \AgdaFunction{[} \AgdaBound{A} \AgdaFunction{]subst} \AgdaBound{p} \AgdaSymbol{(}\AgdaBound{tm} \AgdaBound{x}\AgdaSymbol{)} \AgdaFunction{≈} \AgdaBound{tm} \AgdaBound{y}\<%
\\
%
\\
\>\<\end{code}

\AgdaHide{
\begin{code}\>\<%
\\
\>\AgdaKeyword{open} \AgdaModule{Tm} \AgdaKeyword{public} \AgdaKeyword{renaming} \AgdaSymbol{(}tm \AgdaSymbol{to} [\_]tm \AgdaSymbol{;} respt \AgdaSymbol{to} [\_]respt\AgdaSymbol{)}\<%
\\
%
\\
\>\<\end{code}
}

Substitution for terms can be defined as

\begin{code}\>\<%
\\
%
\\
\>\AgdaFunction{\_[\_]m} \AgdaSymbol{:} \AgdaSymbol{∀} \AgdaSymbol{\{}\AgdaBound{Γ} \AgdaBound{Δ} \AgdaSymbol{:} \AgdaFunction{Con}\AgdaSymbol{\}\{}\AgdaBound{A} \AgdaSymbol{:} \AgdaRecord{Ty} \AgdaBound{Δ}\AgdaSymbol{\}} \AgdaSymbol{→} \<[34]%
\>[34]\<%
\\
\>[0]\AgdaIndent{10}{}\<[10]%
\>[10]\AgdaRecord{Tm} \AgdaBound{A} \AgdaSymbol{→} \<[17]%
\>[17]\<%
\\
\>[0]\AgdaIndent{10}{}\<[10]%
\>[10]\AgdaSymbol{(}\AgdaBound{f} \AgdaSymbol{:} \AgdaBound{Γ} \AgdaRecord{⇉} \AgdaBound{Δ}\AgdaSymbol{)} \<[22]%
\>[22]\<%
\\
\>[0]\AgdaIndent{10}{}\<[10]%
\>[10]\AgdaSymbol{→} \AgdaRecord{Tm} \AgdaSymbol{(}\AgdaBound{A} \AgdaFunction{[} \AgdaBound{f} \AgdaFunction{]T}\AgdaSymbol{)}\<%
\\
\>\AgdaFunction{\_[\_]m} \AgdaBound{t} \AgdaBound{f} \AgdaSymbol{=} \AgdaKeyword{record} \<[19]%
\>[19]\<%
\\
\>[0]\AgdaIndent{10}{}\<[10]%
\>[10]\AgdaSymbol{\{} \AgdaField{tm} \AgdaSymbol{=} \AgdaFunction{[} \AgdaBound{t} \AgdaFunction{]tm} \AgdaFunction{∘} \AgdaFunction{[} \AgdaBound{f} \AgdaFunction{]fn}\<%
\\
\>[0]\AgdaIndent{10}{}\<[10]%
\>[10]\AgdaSymbol{;} \AgdaField{respt} \AgdaSymbol{=} \AgdaFunction{[} \AgdaBound{t} \AgdaFunction{]respt} \AgdaFunction{∘} \AgdaFunction{[} \AgdaBound{f} \AgdaFunction{]resp} \<[43]%
\>[43]\<%
\\
\>[0]\AgdaIndent{10}{}\<[10]%
\>[10]\AgdaSymbol{\}}\<%
\\
%
\\
\>\<\end{code}

Syntactically we can form a new context by using a context $\Gamma$ and a type $A : Ty \:\Gamma$. To introduce a term of it, we need a term of the semantic context $\Gamma$ and a term of semantic type $A$. It is called context comprehension. 

\begin{code}\>\<%
\\
%
\\
\>\AgdaFunction{\_\&\_} \AgdaSymbol{:} \AgdaSymbol{(}\AgdaBound{Γ} \AgdaSymbol{:} \AgdaFunction{Con}\AgdaSymbol{)} \AgdaSymbol{→} \AgdaRecord{Ty} \AgdaBound{Γ} \AgdaSymbol{→} \AgdaFunction{Con}\<%
\\
\>\AgdaBound{Γ} \AgdaFunction{\&} \AgdaBound{A} \AgdaSymbol{=} \AgdaKeyword{record} \<[15]%
\>[15]\<%
\\
\>[0]\AgdaIndent{7}{}\<[7]%
\>[7]\AgdaSymbol{\{} \AgdaField{Carrier} \AgdaSymbol{=} \AgdaRecord{Σ[} \AgdaBound{x} \AgdaRecord{∶} \AgdaFunction{∣} \AgdaBound{Γ} \AgdaFunction{∣} \AgdaRecord{]} \AgdaFunction{∣} \AgdaFunction{fm} \AgdaBound{x} \AgdaFunction{∣}\<%
\\
\>[0]\AgdaIndent{7}{}\<[7]%
\>[7]\AgdaSymbol{;} \AgdaField{\_≈h\_} \<[17]%
\>[17]\AgdaSymbol{=} \AgdaSymbol{λ\{(}\AgdaBound{x} \AgdaInductiveConstructor{,} \AgdaBound{a}\AgdaSymbol{)} \AgdaSymbol{(}\AgdaBound{y} \AgdaInductiveConstructor{,} \AgdaBound{b}\AgdaSymbol{)} \AgdaSymbol{→} \<[39]%
\>[39]\<%
\\
\>[7]\AgdaIndent{19}{}\<[19]%
\>[19]\AgdaFunction{Σ'[} \AgdaBound{p} \AgdaFunction{∶} \AgdaBound{x} \AgdaFunction{≈h} \AgdaBound{y} \AgdaFunction{]} \AgdaFunction{[} \AgdaFunction{fm} \AgdaBound{y} \AgdaFunction{]} \AgdaFunction{substT} \AgdaBound{p} \AgdaBound{a} \AgdaFunction{≈h} \AgdaBound{b}\AgdaSymbol{\}}\<%
\\
\>[0]\AgdaIndent{7}{}\<[7]%
\>[7]\AgdaSymbol{;} \AgdaField{isEquiv} \AgdaSymbol{=} \<[19]%
\>[19]\<%
\\
\>[0]\AgdaIndent{10}{}\<[10]%
\>[10]\AgdaKeyword{record} \<[17]%
\>[17]\<%
\\
\>[0]\AgdaIndent{10}{}\<[10]%
\>[10]\AgdaSymbol{\{} \AgdaField{refl} \<[18]%
\>[18]\AgdaSymbol{=} \AgdaFunction{refl} \AgdaInductiveConstructor{,} \AgdaSymbol{(}\AgdaFunction{refl*} \AgdaSymbol{\_} \AgdaSymbol{\_)}\<%
\\
\>[0]\AgdaIndent{10}{}\<[10]%
\>[10]\AgdaSymbol{;} \AgdaField{sym} \<[18]%
\>[18]\AgdaSymbol{=} \AgdaSymbol{λ} \AgdaSymbol{\{(}\AgdaBound{p} \AgdaInductiveConstructor{,} \AgdaBound{q}\AgdaSymbol{)} \AgdaSymbol{→} \AgdaSymbol{(}\AgdaFunction{sym} \AgdaBound{p}\AgdaSymbol{)} \AgdaInductiveConstructor{,} \<[43]%
\>[43]\<%
\\
\>[10]\AgdaIndent{20}{}\<[20]%
\>[20]\AgdaFunction{[} \AgdaFunction{fm} \AgdaSymbol{\_} \AgdaFunction{]trans} \<[34]%
\>[34]\<%
\\
\>[10]\AgdaIndent{20}{}\<[20]%
\>[20]\AgdaSymbol{(}\AgdaFunction{subst*} \AgdaSymbol{\_} \AgdaSymbol{(}\AgdaFunction{[} \AgdaFunction{fm} \AgdaSymbol{\_} \AgdaFunction{]sym} \AgdaBound{q}\AgdaSymbol{))} \<[47]%
\>[47]\<%
\\
\>[10]\AgdaIndent{20}{}\<[20]%
\>[20]\AgdaFunction{trans-refl} \AgdaSymbol{\}}\<%
\\
\>[0]\AgdaIndent{10}{}\<[10]%
\>[10]\AgdaSymbol{;} \AgdaField{trans} \AgdaSymbol{=} \AgdaSymbol{λ} \AgdaSymbol{\{(}\AgdaBound{p} \AgdaInductiveConstructor{,} \AgdaBound{q}\AgdaSymbol{)} \AgdaSymbol{(}\AgdaBound{m} \AgdaInductiveConstructor{,} \AgdaBound{n}\AgdaSymbol{)} \AgdaSymbol{→}\<%
\\
\>[0]\AgdaIndent{20}{}\<[20]%
\>[20]\AgdaFunction{trans} \AgdaBound{p} \AgdaBound{m} \AgdaInductiveConstructor{,} \<[32]%
\>[32]\<%
\\
\>[0]\AgdaIndent{20}{}\<[20]%
\>[20]\AgdaFunction{[} \AgdaFunction{fm} \AgdaSymbol{\_} \AgdaFunction{]trans} \<[34]%
\>[34]\<%
\\
\>[0]\AgdaIndent{20}{}\<[20]%
\>[20]\AgdaSymbol{(}\AgdaFunction{[} \AgdaFunction{fm} \AgdaSymbol{\_} \AgdaFunction{]trans} \<[35]%
\>[35]\<%
\\
\>[0]\AgdaIndent{20}{}\<[20]%
\>[20]\AgdaSymbol{(}\AgdaFunction{[} \AgdaFunction{fm} \AgdaSymbol{\_} \AgdaFunction{]sym} \AgdaSymbol{(}\AgdaFunction{trans*} \AgdaSymbol{\_} \AgdaSymbol{\_} \AgdaSymbol{\_))} \AgdaSymbol{(}\AgdaFunction{subst*} \AgdaSymbol{\_} \AgdaBound{q}\AgdaSymbol{))} \AgdaBound{n} \AgdaSymbol{\}}\<%
\\
\>[0]\AgdaIndent{10}{}\<[10]%
\>[10]\AgdaSymbol{\}}\<%
\\
\>[0]\AgdaIndent{7}{}\<[7]%
\>[7]\AgdaSymbol{\}}\<%
\\
%
\\
%
\\
\>\<\end{code}

\AgdaHide{
\begin{code}\>\<%
\\
%
\\
\>[0]\AgdaIndent{7}{}\<[7]%
\>[7]\AgdaKeyword{where} \<[13]%
\>[13]\<%
\\
\>[7]\AgdaIndent{9}{}\<[9]%
\>[9]\AgdaKeyword{open} \AgdaModule{hSetoid} \AgdaBound{Γ}\<%
\\
\>[7]\AgdaIndent{9}{}\<[9]%
\>[9]\AgdaKeyword{open} \AgdaModule{Ty} \AgdaBound{A} \<[23]%
\>[23]\<%
\\
%
\\
\>\<\end{code}
}

There are also some other morphisms come with it. Any morphism from a context $\Gamma$ to a context $\Delta \& A$ consists of a morphism from $\Gamma$ to $\Delta$ and a term of type $A$ substituted. In other words, There is an isomorphism between $Hom(\Gamma , \Delta \& A)$ and $\Sigma \gamma : Hom(\Gamma , \Delta) A [ \gamma ] $.

$fst$ projects the morphism and  $snd$ projects the term.
Indeed the $fst$ operation provides weakening for types, and the $snd$ projection enables us to interpret variables. $fst\&$ defines a morphism for each type $A$ which is a canonical projection of $A$.
We need to use $id'$ which are identity context morphisms to achieve these.

\begin{code}\>\<%
\\
%
\\
\>\AgdaFunction{fst} \AgdaSymbol{:} \AgdaSymbol{\{}\AgdaBound{Γ} \AgdaBound{Δ} \AgdaSymbol{:} \AgdaFunction{Con}\AgdaSymbol{\}(}\AgdaBound{A} \AgdaSymbol{:} \AgdaRecord{Ty} \AgdaBound{Δ}\AgdaSymbol{)} \AgdaSymbol{→} \AgdaBound{Γ} \AgdaRecord{⇉} \AgdaSymbol{(}\AgdaBound{Δ} \AgdaFunction{\&} \AgdaBound{A}\AgdaSymbol{)} \AgdaSymbol{→} \AgdaBound{Γ} \AgdaRecord{⇉} \AgdaBound{Δ}\<%
\\
\>\AgdaFunction{fst} \AgdaBound{A} \AgdaBound{f} \AgdaSymbol{=} \AgdaKeyword{record} \<[17]%
\>[17]\<%
\\
\>[-6]\AgdaIndent{8}{}\<[8]%
\>[8]\AgdaSymbol{\{} \AgdaField{fn} \AgdaSymbol{=} \AgdaFunction{proj₁} \AgdaFunction{∘} \AgdaFunction{[} \AgdaBound{f} \AgdaFunction{]fn}\<%
\\
\>[0]\AgdaIndent{8}{}\<[8]%
\>[8]\AgdaSymbol{;} \AgdaField{resp} \AgdaSymbol{=} \AgdaFunction{proj₁} \AgdaFunction{∘} \AgdaFunction{[} \AgdaBound{f} \AgdaFunction{]resp} \<[35]%
\>[35]\<%
\\
\>[0]\AgdaIndent{8}{}\<[8]%
\>[8]\AgdaSymbol{\}}\<%
\\
%
\\
\>\AgdaFunction{fst\&} \AgdaSymbol{:} \AgdaSymbol{\{}\AgdaBound{Γ} \AgdaSymbol{:} \AgdaFunction{Con}\AgdaSymbol{\}(}\AgdaBound{A} \AgdaSymbol{:} \AgdaRecord{Ty} \AgdaBound{Γ}\AgdaSymbol{)} \AgdaSymbol{→} \AgdaBound{Γ} \AgdaFunction{\&} \AgdaBound{A} \AgdaRecord{⇉} \AgdaBound{Γ}\<%
\\
\>\AgdaFunction{fst\&} \AgdaBound{A} \AgdaSymbol{=} \AgdaFunction{fst} \AgdaBound{A} \AgdaFunction{id'}\<%
\\
%
\\
\>\AgdaFunction{\_+T\_} \AgdaSymbol{:} \AgdaSymbol{\{}\AgdaBound{Γ} \AgdaSymbol{:} \AgdaFunction{Con}\AgdaSymbol{\}} \AgdaSymbol{→} \AgdaRecord{Ty} \AgdaBound{Γ} \AgdaSymbol{→} \AgdaSymbol{(}\AgdaBound{A} \AgdaSymbol{:} \AgdaRecord{Ty} \AgdaBound{Γ}\AgdaSymbol{)} \AgdaSymbol{→} \AgdaRecord{Ty} \AgdaSymbol{(}\AgdaBound{Γ} \AgdaFunction{\&} \AgdaBound{A}\AgdaSymbol{)}\<%
\\
\>\AgdaBound{B} \AgdaFunction{+T} \AgdaBound{A} \AgdaSymbol{=} \AgdaBound{B} \AgdaFunction{[} \AgdaFunction{fst\&} \AgdaBound{A} \AgdaFunction{]T}\<%
\\
%
\\
\>\AgdaFunction{snd} \AgdaSymbol{:} \AgdaSymbol{\{}\AgdaBound{Γ} \AgdaBound{Δ} \AgdaSymbol{:} \AgdaFunction{Con}\AgdaSymbol{\}(}\AgdaBound{A} \AgdaSymbol{:} \AgdaRecord{Ty} \AgdaBound{Δ}\AgdaSymbol{)} \AgdaSymbol{→} \<[30]%
\>[30]\<%
\\
\>[0]\AgdaIndent{6}{}\<[6]%
\>[6]\AgdaSymbol{(}\AgdaBound{f} \AgdaSymbol{:} \AgdaBound{Γ} \AgdaRecord{⇉} \AgdaSymbol{(}\AgdaBound{Δ} \AgdaFunction{\&} \AgdaBound{A}\AgdaSymbol{))} \<[24]%
\>[24]\<%
\\
\>[0]\AgdaIndent{6}{}\<[6]%
\>[6]\AgdaSymbol{→} \AgdaRecord{Tm} \AgdaSymbol{(}\AgdaBound{A} \AgdaFunction{[} \AgdaFunction{fst} \AgdaBound{A} \AgdaBound{f} \AgdaFunction{]T}\AgdaSymbol{)}\<%
\\
\>\AgdaFunction{snd} \AgdaBound{A} \AgdaBound{f} \AgdaSymbol{=} \AgdaKeyword{record} \<[17]%
\>[17]\<%
\\
\>[6]\AgdaIndent{8}{}\<[8]%
\>[8]\AgdaSymbol{\{} \AgdaField{tm} \AgdaSymbol{=} \AgdaFunction{proj₂} \AgdaFunction{∘} \AgdaFunction{[} \AgdaBound{f} \AgdaFunction{]fn}\<%
\\
\>[6]\AgdaIndent{8}{}\<[8]%
\>[8]\AgdaSymbol{;} \AgdaField{respt} \AgdaSymbol{=} \AgdaFunction{proj₂} \AgdaFunction{∘} \AgdaFunction{[} \AgdaBound{f} \AgdaFunction{]resp} \<[36]%
\>[36]\<%
\\
\>[6]\AgdaIndent{8}{}\<[8]%
\>[8]\AgdaSymbol{\}}\<%
\\
%
\\
\>\AgdaFunction{v0} \AgdaSymbol{:} \AgdaSymbol{\{}\AgdaBound{Γ} \AgdaSymbol{:} \AgdaFunction{Con}\AgdaSymbol{\}(}\AgdaBound{A} \AgdaSymbol{:} \AgdaRecord{Ty} \AgdaBound{Γ}\AgdaSymbol{)} \AgdaSymbol{→} \AgdaRecord{Tm} \AgdaSymbol{(}\AgdaBound{A} \AgdaFunction{+T} \AgdaBound{A}\AgdaSymbol{)}\<%
\\
\>\AgdaFunction{v0} \AgdaBound{A} \AgdaSymbol{=} \AgdaFunction{snd} \AgdaBound{A} \AgdaFunction{id'}\<%
\\
%
\\
\>\<\end{code}

Inversely we could define a pairing operation to combine a context morphism with a term. The $\eta$-law for the projection and pairing holds trivially.

\begin{code}\>\<%
\\
%
\\
\>\AgdaFunction{\_,,\_} \AgdaSymbol{:} \AgdaSymbol{\{}\AgdaBound{Γ} \AgdaBound{Δ} \AgdaSymbol{:} \AgdaFunction{Con}\AgdaSymbol{\}\{}\AgdaBound{A} \AgdaSymbol{:} \AgdaRecord{Ty} \AgdaBound{Δ}\AgdaSymbol{\}(}\AgdaBound{f} \AgdaSymbol{:} \AgdaBound{Γ} \AgdaRecord{⇉} \AgdaBound{Δ}\AgdaSymbol{)} \AgdaSymbol{→} \<[42]%
\>[42]\<%
\\
\>[-5]\AgdaIndent{7}{}\<[7]%
\>[7]\AgdaSymbol{(}\AgdaRecord{Tm} \AgdaSymbol{(}\AgdaBound{A} \AgdaFunction{[} \AgdaBound{f} \AgdaFunction{]T}\AgdaSymbol{))} \<[23]%
\>[23]\<%
\\
\>[0]\AgdaIndent{7}{}\<[7]%
\>[7]\AgdaSymbol{→} \AgdaBound{Γ} \AgdaRecord{⇉} \AgdaSymbol{(}\AgdaBound{Δ} \AgdaFunction{\&} \AgdaBound{A}\AgdaSymbol{)}\<%
\\
\>\AgdaBound{f} \AgdaFunction{,,} \AgdaBound{t} \AgdaSymbol{=} \AgdaKeyword{record} \<[16]%
\>[16]\<%
\\
\>[7]\AgdaIndent{9}{}\<[9]%
\>[9]\AgdaSymbol{\{} \AgdaField{fn} \AgdaSymbol{=} \AgdaFunction{⟨} \AgdaFunction{[} \AgdaBound{f} \AgdaFunction{]fn} \AgdaFunction{,} \AgdaFunction{[} \AgdaBound{t} \AgdaFunction{]tm} \AgdaFunction{⟩}\<%
\\
\>[7]\AgdaIndent{9}{}\<[9]%
\>[9]\AgdaSymbol{;} \AgdaField{resp} \AgdaSymbol{=} \AgdaFunction{⟨} \AgdaFunction{[} \AgdaBound{f} \AgdaFunction{]resp} \AgdaFunction{,} \AgdaFunction{[} \AgdaBound{t} \AgdaFunction{]respt} \AgdaFunction{⟩}\<%
\\
\>[7]\AgdaIndent{9}{}\<[9]%
\>[9]\AgdaSymbol{\}}\<%
\\
%
\\
\>\AgdaFunction{\&-eta} \AgdaSymbol{:} \AgdaSymbol{\{}\AgdaBound{Γ} \AgdaBound{Δ} \AgdaSymbol{:} \AgdaFunction{Con}\AgdaSymbol{\}\{}\AgdaBound{A} \AgdaSymbol{:} \AgdaRecord{Ty} \AgdaBound{Δ}\AgdaSymbol{\}(}\AgdaBound{f} \AgdaSymbol{:} \AgdaBound{Γ} \AgdaRecord{⇉} \AgdaSymbol{(}\AgdaBound{Δ} \AgdaFunction{\&} \AgdaBound{A}\AgdaSymbol{))} \<[47]%
\>[47]\<%
\\
\>[-4]\AgdaIndent{6}{}\<[6]%
\>[6]\AgdaSymbol{→} \AgdaFunction{\_,,\_} \AgdaSymbol{\{}A \AgdaSymbol{=} \AgdaBound{A}\AgdaSymbol{\}} \AgdaSymbol{(}\AgdaFunction{fst} \AgdaBound{A} \AgdaBound{f}\AgdaSymbol{)} \AgdaSymbol{(}\AgdaFunction{snd} \AgdaBound{A} \AgdaBound{f}\AgdaSymbol{)} \AgdaDatatype{≡} \AgdaBound{f}\<%
\\
\>\AgdaFunction{\&-eta} \AgdaBound{f} \AgdaSymbol{=} \AgdaInductiveConstructor{PE.refl}\<%
\\
%
\\
%
\\
\>\<\end{code}

Then a lifting operation could help us define $\Pi$-types.

\begin{code}\>\<%
\\
%
\\
\>\AgdaFunction{lift} \AgdaSymbol{:} \AgdaSymbol{\{}\AgdaBound{Γ} \AgdaBound{Δ} \AgdaSymbol{:} \AgdaFunction{Con}\AgdaSymbol{\}(}\AgdaBound{f} \AgdaSymbol{:} \AgdaBound{Γ} \AgdaRecord{⇉} \AgdaBound{Δ}\AgdaSymbol{)(}\AgdaBound{A} \AgdaSymbol{:} \AgdaRecord{Ty} \AgdaBound{Δ}\AgdaSymbol{)} \AgdaSymbol{→} \AgdaBound{Γ} \AgdaFunction{\&} \AgdaBound{A} \AgdaFunction{[} \AgdaBound{f} \AgdaFunction{]T} \AgdaRecord{⇉} \AgdaBound{Δ} \AgdaFunction{\&} \AgdaBound{A}\<%
\\
\>\AgdaFunction{lift} \AgdaBound{f} \AgdaBound{A} \AgdaSymbol{=} \AgdaKeyword{record} \<[18]%
\>[18]\<%
\\
\>[0]\AgdaIndent{8}{}\<[8]%
\>[8]\AgdaSymbol{\{} \AgdaField{fn} \AgdaSymbol{=} \AgdaFunction{⟨} \AgdaFunction{[} \AgdaBound{f} \AgdaFunction{]fn} \AgdaFunction{∘} \AgdaFunction{proj₁} \AgdaFunction{,} \AgdaFunction{proj₂} \AgdaFunction{⟩}\<%
\\
\>[0]\AgdaIndent{8}{}\<[8]%
\>[8]\AgdaSymbol{;} \AgdaField{resp} \AgdaSymbol{=} \AgdaFunction{⟨} \AgdaFunction{[} \AgdaBound{f} \AgdaFunction{]resp} \AgdaFunction{∘} \AgdaFunction{proj₁} \AgdaFunction{,} \AgdaFunction{proj₂} \AgdaFunction{⟩}\<%
\\
\>[0]\AgdaIndent{8}{}\<[8]%
\>[8]\AgdaSymbol{\}}\<%
\\
%
\\
\>\AgdaFunction{lift-eta} \AgdaSymbol{:} \AgdaSymbol{\{}\AgdaBound{Γ} \AgdaBound{Δ} \AgdaSymbol{:} \AgdaFunction{Con}\AgdaSymbol{\}}\<%
\\
\>[8]\AgdaIndent{9}{}\<[9]%
\>[9]\AgdaSymbol{(}\AgdaBound{f} \AgdaSymbol{:} \AgdaBound{Γ} \AgdaRecord{⇉} \AgdaBound{Δ}\AgdaSymbol{)(}\AgdaBound{A} \AgdaSymbol{:} \AgdaRecord{Ty} \AgdaBound{Δ}\AgdaSymbol{)(}\AgdaBound{x} \AgdaSymbol{:} \AgdaFunction{∣} \AgdaBound{Γ} \AgdaFunction{∣}\AgdaSymbol{)}\<%
\\
\>[8]\AgdaIndent{9}{}\<[9]%
\>[9]\AgdaSymbol{(}\AgdaBound{a} \AgdaSymbol{:} \AgdaFunction{∣} \AgdaFunction{[} \AgdaBound{A} \AgdaFunction{]fm} \AgdaSymbol{(}\AgdaFunction{[} \AgdaBound{f} \AgdaFunction{]fn} \AgdaBound{x}\AgdaSymbol{)} \AgdaFunction{∣}\AgdaSymbol{)} \<[39]%
\>[39]\<%
\\
\>[8]\AgdaIndent{9}{}\<[9]%
\>[9]\AgdaSymbol{→} \AgdaFunction{[} \AgdaFunction{lift} \AgdaBound{f} \AgdaBound{A} \AgdaFunction{]fn} \AgdaSymbol{(}\AgdaBound{x} \AgdaInductiveConstructor{,} \AgdaBound{a}\AgdaSymbol{)} \AgdaDatatype{≡} \AgdaSymbol{(} \AgdaFunction{[} \AgdaBound{f} \AgdaFunction{]fn} \AgdaBound{x} \AgdaInductiveConstructor{,} \AgdaBound{a}\AgdaSymbol{)}\<%
\\
\>\AgdaFunction{lift-eta} \AgdaBound{f} \AgdaBound{A} \AgdaBound{x} \AgdaBound{a} \AgdaSymbol{=} \AgdaInductiveConstructor{PE.refl}\<%
\\
%
\\
%
\\
\>\<\end{code}

One of the most complicated part of this definition is the $\Pi$-types.
$\Pi$-types is also called dependent function types. Semantically it is a function type on the underlying semantic types with a proof that the the functions respect the equivalence relation. 

%f-resp on the paper ignores refl*

\begin{code}\>\<%
\\
%
\\
\>\AgdaFunction{Π} \AgdaSymbol{:} \AgdaSymbol{\{}\AgdaBound{Γ} \AgdaSymbol{:} \AgdaFunction{Con}\AgdaSymbol{\}(}\AgdaBound{A} \AgdaSymbol{:} \AgdaRecord{Ty} \AgdaBound{Γ}\AgdaSymbol{)(}\AgdaBound{B} \AgdaSymbol{:} \AgdaRecord{Ty} \AgdaSymbol{(}\AgdaBound{Γ} \AgdaFunction{\&} \AgdaBound{A}\AgdaSymbol{))} \AgdaSymbol{→} \AgdaRecord{Ty} \AgdaBound{Γ}\<%
\\
%
\\
\>\<\end{code}

\AgdaHide{
\begin{code}\>\<%
\\
%
\\
\>\AgdaFunction{Π} \AgdaSymbol{\{}\AgdaBound{Γ}\AgdaSymbol{\}} \AgdaBound{A} \AgdaBound{B} \AgdaSymbol{=} \AgdaKeyword{record} \<[19]%
\>[19]\<%
\\
\>[-1]\AgdaIndent{2}{}\<[2]%
\>[2]\AgdaSymbol{\{} \AgdaField{fm} \AgdaSymbol{=} \AgdaSymbol{λ} \AgdaBound{x} \AgdaSymbol{→} \AgdaKeyword{let} \AgdaBound{Ax} \AgdaSymbol{=} \AgdaFunction{[} \AgdaBound{A} \AgdaFunction{]fm} \AgdaBound{x} \AgdaKeyword{in}\<%
\\
\>[0]\AgdaIndent{15}{}\<[15]%
\>[15]\AgdaKeyword{let} \AgdaBound{Bx} \AgdaSymbol{=} \AgdaSymbol{λ} \AgdaBound{a} \AgdaSymbol{→} \AgdaFunction{[} \AgdaBound{B} \AgdaFunction{]fm} \AgdaSymbol{(}\AgdaBound{x} \AgdaInductiveConstructor{,} \AgdaBound{a}\AgdaSymbol{)} \AgdaKeyword{in}\<%
\\
\>[0]\AgdaIndent{9}{}\<[9]%
\>[9]\AgdaKeyword{record}\<%
\\
\>[0]\AgdaIndent{9}{}\<[9]%
\>[9]\AgdaSymbol{\{} \AgdaField{Carrier} \AgdaSymbol{=} \AgdaRecord{Σ[} \AgdaBound{fn} \AgdaRecord{∶} \AgdaSymbol{((}\AgdaBound{a} \AgdaSymbol{:} \AgdaFunction{∣} \AgdaBound{Ax} \AgdaFunction{∣}\AgdaSymbol{)} \AgdaSymbol{→} \AgdaFunction{∣} \AgdaBound{Bx} \AgdaBound{a} \AgdaFunction{∣}\AgdaSymbol{)} \AgdaRecord{]}\<%
\\
\>[9]\AgdaIndent{21}{}\<[21]%
\>[21]\AgdaSymbol{((}\AgdaBound{a} \AgdaBound{b} \AgdaSymbol{:} \AgdaFunction{∣} \AgdaBound{Ax} \AgdaFunction{∣}\AgdaSymbol{)}\<%
\\
\>[9]\AgdaIndent{21}{}\<[21]%
\>[21]\AgdaSymbol{(}\AgdaBound{p} \AgdaSymbol{:} \AgdaFunction{[} \AgdaBound{Ax} \AgdaFunction{]} \AgdaBound{a} \AgdaFunction{≈} \AgdaBound{b}\AgdaSymbol{)} \AgdaSymbol{→}\<%
\\
\>[9]\AgdaIndent{21}{}\<[21]%
\>[21]\AgdaFunction{[} \AgdaBound{Bx} \AgdaBound{b} \AgdaFunction{]} \AgdaFunction{[} \AgdaBound{B} \AgdaFunction{]subst} \AgdaSymbol{(}\AgdaFunction{[} \AgdaBound{Γ} \AgdaFunction{]refl} \AgdaInductiveConstructor{,}\<%
\\
\>[-7]\AgdaIndent{19}{}\<[19]%
\>[19]\AgdaFunction{[} \AgdaBound{Ax} \AgdaFunction{]trans} \AgdaSymbol{(}\AgdaFunction{[} \AgdaBound{A} \AgdaFunction{]refl*} \AgdaBound{x} \AgdaBound{a}\AgdaSymbol{)} \AgdaBound{p}\AgdaSymbol{)} \AgdaSymbol{(}\AgdaBound{fn} \AgdaBound{a}\AgdaSymbol{)} \AgdaFunction{≈} \AgdaBound{fn} \AgdaBound{b}\AgdaSymbol{)} \<[66]%
\>[66]\<%
\\
\>[0]\AgdaIndent{1}{}\<[1]%
\>[1]\<%
\\
\>[1]\AgdaIndent{9}{}\<[9]%
\>[9]\AgdaSymbol{;} \AgdaField{\_≈h\_} \<[19]%
\>[19]\AgdaSymbol{=} \AgdaSymbol{λ\{(}\AgdaBound{f} \AgdaInductiveConstructor{,} \AgdaSymbol{\_)} \AgdaSymbol{(}\AgdaBound{g} \AgdaInductiveConstructor{,} \AgdaSymbol{\_)} \AgdaSymbol{→} \<[41]%
\>[41]\<%
\\
\>[9]\AgdaIndent{24}{}\<[24]%
\>[24]\AgdaFunction{∀'[} \AgdaBound{a} \AgdaFunction{∶} \AgdaSymbol{\_} \AgdaFunction{]} \AgdaFunction{[} \AgdaBound{Bx} \AgdaBound{a} \AgdaFunction{]} \AgdaBound{f} \AgdaBound{a} \AgdaFunction{≈h} \AgdaBound{g} \AgdaBound{a} \AgdaSymbol{\}}\<%
\\
\>[0]\AgdaIndent{9}{}\<[9]%
\>[9]\AgdaSymbol{;} \AgdaField{isEquiv} \AgdaSymbol{=} \AgdaKeyword{record} \AgdaSymbol{\{}\<%
\\
\>[0]\AgdaIndent{19}{}\<[19]%
\>[19]\AgdaField{refl} \<[25]%
\>[25]\AgdaSymbol{=} \AgdaSymbol{λ} \AgdaBound{a} \AgdaSymbol{→} \AgdaFunction{[} \AgdaBound{Bx} \AgdaBound{a} \AgdaFunction{]refl} \<[46]%
\>[46]\<%
\\
\>[0]\AgdaIndent{17}{}\<[17]%
\>[17]\AgdaSymbol{;} \AgdaField{sym} \<[25]%
\>[25]\AgdaSymbol{=} \AgdaSymbol{λ} \AgdaBound{f} \AgdaBound{a} \AgdaSymbol{→} \AgdaFunction{[} \AgdaBound{Bx} \AgdaBound{a} \AgdaFunction{]sym} \AgdaSymbol{(}\AgdaBound{f} \AgdaBound{a}\AgdaSymbol{)}\<%
\\
\>[0]\AgdaIndent{17}{}\<[17]%
\>[17]\AgdaSymbol{;} \AgdaField{trans} \AgdaSymbol{=} \AgdaSymbol{λ} \AgdaBound{f} \AgdaBound{g} \AgdaBound{a} \AgdaSymbol{→} \AgdaFunction{[} \AgdaBound{Bx} \AgdaBound{a} \AgdaFunction{]trans} \AgdaSymbol{(}\AgdaBound{f} \AgdaBound{a}\AgdaSymbol{)} \AgdaSymbol{(}\AgdaBound{g} \AgdaBound{a}\AgdaSymbol{)}\<%
\\
\>[17]\AgdaIndent{29}{}\<[29]%
\>[29]\AgdaSymbol{\}}\<%
\\
\>[3]\AgdaIndent{9}{}\<[9]%
\>[9]\AgdaSymbol{\}}\<%
\\
%
\\
\>[0]\AgdaIndent{2}{}\<[2]%
\>[2]\AgdaSymbol{;} \AgdaField{substT} \AgdaSymbol{=} \AgdaSymbol{λ} \AgdaSymbol{\{}\AgdaBound{x}\AgdaSymbol{\}} \AgdaSymbol{\{}\AgdaBound{y}\AgdaSymbol{\}} \AgdaBound{p} \AgdaSymbol{→}\<%
\\
\>[2]\AgdaIndent{19}{}\<[19]%
\>[19]\AgdaKeyword{let} \AgdaBound{y2x} \AgdaSymbol{=} \AgdaSymbol{λ} \AgdaBound{a} \AgdaSymbol{→} \AgdaFunction{[} \AgdaBound{A} \AgdaFunction{]subst} \AgdaSymbol{(}\AgdaFunction{[} \AgdaBound{Γ} \AgdaFunction{]sym} \AgdaBound{p}\AgdaSymbol{)} \AgdaBound{a} \AgdaKeyword{in}\<%
\\
\>[19]\AgdaIndent{20}{}\<[20]%
\>[20]\AgdaKeyword{let} \AgdaBound{x2y} \AgdaSymbol{=} \AgdaSymbol{λ} \AgdaBound{a} \AgdaSymbol{→} \AgdaFunction{[} \AgdaBound{A} \AgdaFunction{]subst} \AgdaBound{p} \AgdaBound{a} \AgdaKeyword{in}\<%
\\
\>[0]\AgdaIndent{19}{}\<[19]%
\>[19]\AgdaKeyword{let} \AgdaBound{p'} \AgdaSymbol{=} \AgdaSymbol{λ} \AgdaBound{a} \AgdaSymbol{→} \AgdaFunction{[} \AgdaBound{A} \AgdaFunction{]trans-refl} \AgdaKeyword{in}\<%
\\
\>[0]\AgdaIndent{13}{}\<[13]%
\>[13]\AgdaSymbol{λ\{(}\AgdaBound{f} \AgdaInductiveConstructor{,} \AgdaBound{rsp}\AgdaSymbol{)} \AgdaSymbol{→} \<[28]%
\>[28]\<%
\\
\>[13]\AgdaIndent{15}{}\<[15]%
\>[15]\AgdaSymbol{(λ} \AgdaBound{a} \AgdaSymbol{→} \AgdaFunction{[} \AgdaBound{B} \AgdaFunction{]subst} \AgdaSymbol{(}\AgdaBound{p} \AgdaInductiveConstructor{,} \AgdaBound{p'} \AgdaBound{a}\AgdaSymbol{)} \AgdaSymbol{(}\AgdaBound{f} \AgdaSymbol{(}\AgdaBound{y2x} \AgdaBound{a}\AgdaSymbol{)))}\<%
\\
\>[13]\AgdaIndent{15}{}\<[15]%
\>[15]\AgdaInductiveConstructor{,} \<[49]%
\>[49]\<%
\\
\>[13]\AgdaIndent{15}{}\<[15]%
\>[15]\AgdaSymbol{(λ} \AgdaBound{a} \AgdaBound{b} \AgdaBound{q} \AgdaSymbol{→} \<[26]%
\>[26]\<%
\\
\>[15]\AgdaIndent{16}{}\<[16]%
\>[16]\AgdaKeyword{let} \AgdaBound{a'} \AgdaSymbol{=} \AgdaBound{y2x} \AgdaBound{a} \AgdaKeyword{in} \<[34]%
\>[34]\<%
\\
\>[15]\AgdaIndent{16}{}\<[16]%
\>[16]\AgdaKeyword{let} \AgdaBound{b'} \AgdaSymbol{=} \AgdaBound{y2x} \AgdaBound{b} \AgdaKeyword{in}\<%
\\
\>[15]\AgdaIndent{16}{}\<[16]%
\>[16]\AgdaKeyword{let} \AgdaBound{q'} \AgdaSymbol{=} \AgdaFunction{[} \AgdaBound{A} \AgdaFunction{]subst*} \AgdaSymbol{(}\AgdaFunction{[} \AgdaBound{Γ} \AgdaFunction{]sym} \AgdaBound{p}\AgdaSymbol{)} \AgdaBound{q} \AgdaKeyword{in}\<%
\\
\>[15]\AgdaIndent{16}{}\<[16]%
\>[16]\AgdaKeyword{let} \AgdaBound{H} \AgdaSymbol{=} \AgdaBound{rsp} \AgdaBound{a'} \AgdaBound{b'} \AgdaBound{q'} \AgdaKeyword{in}\<%
\\
\>[15]\AgdaIndent{16}{}\<[16]%
\>[16]\AgdaKeyword{let} \AgdaBound{r} \AgdaSymbol{:} \AgdaFunction{[} \AgdaBound{Γ} \AgdaFunction{\&} \AgdaBound{A} \AgdaFunction{]} \AgdaSymbol{(}\AgdaBound{x} \AgdaInductiveConstructor{,} \AgdaBound{b'}\AgdaSymbol{)} \AgdaFunction{≈} \AgdaSymbol{(}\AgdaBound{y} \AgdaInductiveConstructor{,} \AgdaBound{b}\AgdaSymbol{)}
                    r \AgdaSymbol{=} \AgdaSymbol{(}\AgdaBound{p} \AgdaInductiveConstructor{,} \AgdaBound{p'} \AgdaBound{b}\AgdaSymbol{)} \AgdaKeyword{in}\<%
\\
\>[15]\AgdaIndent{16}{}\<[16]%
\>[16]\AgdaKeyword{let} \AgdaBound{pre} \AgdaSymbol{=} \AgdaFunction{[} \AgdaBound{B} \AgdaFunction{]subst*} \AgdaBound{r} \AgdaBound{H} \AgdaKeyword{in}\<%
\\
\>[15]\AgdaIndent{16}{}\<[16]%
\>[16]\<%
\\
\>[15]\AgdaIndent{16}{}\<[16]%
\>[16]\AgdaFunction{[} \AgdaFunction{[} \AgdaBound{B} \AgdaFunction{]fm} \AgdaSymbol{(}\AgdaBound{y} \AgdaInductiveConstructor{,} \AgdaBound{b}\AgdaSymbol{)} \AgdaFunction{]trans} \<[41]%
\>[41]\<%
\\
\>[15]\AgdaIndent{16}{}\<[16]%
\>[16]\AgdaSymbol{(}\AgdaFunction{[} \AgdaBound{B} \AgdaFunction{]trans*} \AgdaSymbol{\_} \AgdaSymbol{\_} \AgdaSymbol{\_)} \<[52]%
\>[52]\<%
\\
\>[15]\AgdaIndent{16}{}\<[16]%
\>[16]\AgdaSymbol{(}\AgdaFunction{[} \AgdaFunction{[} \AgdaBound{B} \AgdaFunction{]fm} \AgdaSymbol{(}\AgdaBound{y} \AgdaInductiveConstructor{,} \AgdaBound{b}\AgdaSymbol{)} \AgdaFunction{]trans} \<[42]%
\>[42]\<%
\\
\>[15]\AgdaIndent{16}{}\<[16]%
\>[16]\AgdaFunction{[} \AgdaBound{B} \AgdaFunction{]subst-pi} \<[30]%
\>[30]\<%
\\
\>[15]\AgdaIndent{16}{}\<[16]%
\>[16]\AgdaSymbol{(}\AgdaFunction{[} \AgdaFunction{[} \AgdaBound{B} \AgdaFunction{]fm} \AgdaSymbol{(}\AgdaBound{y} \AgdaInductiveConstructor{,} \AgdaBound{b}\AgdaSymbol{)} \AgdaFunction{]trans} \<[42]%
\>[42]\<%
\\
\>[15]\AgdaIndent{16}{}\<[16]%
\>[16]\AgdaSymbol{(}\AgdaFunction{[} \AgdaFunction{[} \AgdaBound{B} \AgdaFunction{]fm} \AgdaSymbol{(}\AgdaBound{y} \AgdaInductiveConstructor{,} \AgdaBound{b}\AgdaSymbol{)} \AgdaFunction{]sym} \<[40]%
\>[40]\<%
\\
\>[15]\AgdaIndent{16}{}\<[16]%
\>[16]\AgdaSymbol{(}\AgdaFunction{[} \AgdaBound{B} \AgdaFunction{]trans*} \AgdaSymbol{\_} \AgdaSymbol{\_} \AgdaSymbol{\_))} \<[37]%
\>[37]\<%
\\
\>[15]\AgdaIndent{16}{}\<[16]%
\>[16]\AgdaBound{pre}\AgdaSymbol{))} \<[23]%
\>[23]\<%
\\
\>[15]\AgdaIndent{16}{}\<[16]%
\>[16]\AgdaSymbol{)} \<[22]%
\>[22]\<%
\\
\>[-12]\AgdaIndent{13}{}\<[13]%
\>[13]\AgdaSymbol{\}}\<%
\\
\>[0]\AgdaIndent{2}{}\<[2]%
\>[2]\AgdaSymbol{;} \AgdaField{subst*} \AgdaSymbol{=} \AgdaSymbol{λ} \AgdaBound{\_} \AgdaBound{q} \AgdaBound{\_} \AgdaSymbol{→} \AgdaFunction{[} \AgdaBound{B} \AgdaFunction{]subst*} \AgdaSymbol{\_} \AgdaSymbol{(}\AgdaBound{q} \AgdaSymbol{\_)}\<%
\\
\>[0]\AgdaIndent{2}{}\<[2]%
\>[2]\AgdaSymbol{;} \AgdaField{refl*} \AgdaSymbol{=} \AgdaSymbol{λ} \AgdaSymbol{\{}\AgdaBound{x} \AgdaSymbol{(}\AgdaBound{f} \AgdaInductiveConstructor{,} \AgdaBound{rsp}\AgdaSymbol{)} \AgdaBound{a} \AgdaSymbol{→} \<[32]%
\>[32]\AgdaFunction{[} \AgdaFunction{[} \AgdaBound{B} \AgdaFunction{]fm} \AgdaSymbol{\_} \AgdaFunction{]trans} \<[51]%
\>[51]\<%
\\
\>[2]\AgdaIndent{17}{}\<[17]%
\>[17]\AgdaFunction{[} \AgdaBound{B} \AgdaFunction{]subst-pi} \AgdaSymbol{(}\AgdaBound{rsp} \AgdaSymbol{(}\AgdaFunction{[} \AgdaBound{A} \AgdaFunction{]subst} \<[48]%
\>[48]\<%
\\
\>[17]\AgdaIndent{21}{}\<[21]%
\>[21]\AgdaSymbol{(}\AgdaFunction{[} \AgdaBound{Γ} \AgdaFunction{]sym} \AgdaFunction{[} \AgdaBound{Γ} \AgdaFunction{]refl}\AgdaSymbol{)} \AgdaBound{a}\AgdaSymbol{)} \AgdaBound{a} \AgdaFunction{[} \AgdaBound{A} \AgdaFunction{]subst-pi'}\AgdaSymbol{)} \<[64]%
\>[64]\AgdaSymbol{\}}\<%
\\
\>[2]\AgdaIndent{2}{}\<[2]%
\>[2]\AgdaSymbol{;} \AgdaField{trans*} \AgdaSymbol{=} \AgdaSymbol{λ} \AgdaBound{p} \AgdaBound{q} \AgdaSymbol{→} \AgdaSymbol{λ} \AgdaSymbol{\{(}\AgdaBound{f} \AgdaInductiveConstructor{,} \AgdaBound{rsp}\AgdaSymbol{)} \AgdaBound{a} \AgdaSymbol{→}\<%
\\
\>[0]\AgdaIndent{13}{}\<[13]%
\>[13]\AgdaFunction{[} \AgdaFunction{[} \AgdaBound{B} \AgdaFunction{]fm} \AgdaSymbol{\_} \AgdaFunction{]trans} \<[32]%
\>[32]\<%
\\
\>[0]\AgdaIndent{13}{}\<[13]%
\>[13]\AgdaSymbol{(}\AgdaFunction{[} \AgdaFunction{[} \AgdaBound{B} \AgdaFunction{]fm} \AgdaSymbol{\_} \AgdaFunction{]trans} \<[33]%
\>[33]\<%
\\
\>[0]\AgdaIndent{13}{}\<[13]%
\>[13]\AgdaSymbol{(}\AgdaFunction{[} \AgdaBound{B} \AgdaFunction{]trans*} \AgdaSymbol{\_} \AgdaSymbol{\_} \AgdaSymbol{\_)} \<[33]%
\>[33]\<%
\\
\>[0]\AgdaIndent{13}{}\<[13]%
\>[13]\AgdaSymbol{(}\AgdaFunction{[} \AgdaFunction{[} \AgdaBound{B} \AgdaFunction{]fm} \AgdaSymbol{\_} \AgdaFunction{]sym} \<[31]%
\>[31]\<%
\\
\>[0]\AgdaIndent{13}{}\<[13]%
\>[13]\AgdaSymbol{(}\AgdaFunction{[} \AgdaFunction{[} \AgdaBound{B} \AgdaFunction{]fm} \AgdaSymbol{\_} \AgdaFunction{]trans} \<[33]%
\>[33]\<%
\\
\>[0]\AgdaIndent{13}{}\<[13]%
\>[13]\AgdaSymbol{(}\AgdaFunction{[} \AgdaBound{B} \AgdaFunction{]trans*} \AgdaSymbol{\_} \AgdaSymbol{\_} \AgdaSymbol{\_)} \AgdaFunction{[} \AgdaBound{B} \AgdaFunction{]subst-pi}\AgdaSymbol{)))} \<[50]%
\>[50]\<%
\\
\>[0]\AgdaIndent{13}{}\<[13]%
\>[13]\AgdaSymbol{(}\AgdaFunction{[} \AgdaBound{B} \AgdaFunction{]subst*} \AgdaSymbol{\_} \AgdaSymbol{(}\AgdaBound{rsp} \AgdaSymbol{\_} \AgdaSymbol{\_} \<[37]%
\>[37]\<%
\\
\>[0]\AgdaIndent{13}{}\<[13]%
\>[13]\AgdaSymbol{(}\AgdaFunction{[} \AgdaFunction{[} \AgdaBound{A} \AgdaFunction{]fm} \AgdaSymbol{\_} \AgdaFunction{]trans} \<[33]%
\>[33]\<%
\\
\>[0]\AgdaIndent{13}{}\<[13]%
\>[13]\AgdaSymbol{(}\AgdaFunction{[} \AgdaBound{A} \AgdaFunction{]trans*} \AgdaSymbol{\_} \AgdaSymbol{\_} \AgdaSymbol{\_)} \AgdaFunction{[} \AgdaBound{A} \AgdaFunction{]subst-pi}\AgdaSymbol{)))} \AgdaSymbol{\}} \<[52]%
\>[52]\<%
\\
\>[0]\AgdaIndent{2}{}\<[2]%
\>[2]\AgdaSymbol{\}}\<%
\\
%
\\
\>\<\end{code}
}

It also comes with two necessary operation on the terms of $Pi$-types, $\lambda$-abstraction and application.
There are $\beta-\eta$ laws to verfify for them so that we could form an isomorphism with these two operations. however technically it causes stack overflow. We may simplify these definition in the future so that we could verify them in Agda.

%to do : verification of β and η
%cause stack overflow

\begin{code}\>\<%
\\
%
\\
\>\AgdaFunction{lam} \AgdaSymbol{:} \AgdaSymbol{\{}\AgdaBound{Γ} \AgdaSymbol{:} \AgdaFunction{Con}\AgdaSymbol{\}\{}\AgdaBound{A} \AgdaSymbol{:} \AgdaRecord{Ty} \AgdaBound{Γ}\AgdaSymbol{\}\{}\AgdaBound{B} \AgdaSymbol{:} \AgdaRecord{Ty} \AgdaSymbol{(}\AgdaBound{Γ} \AgdaFunction{\&} \AgdaBound{A}\AgdaSymbol{)\}} \AgdaSymbol{→} \AgdaRecord{Tm} \AgdaBound{B} \AgdaSymbol{→} \AgdaRecord{Tm} \AgdaSymbol{(}\AgdaFunction{Π} \AgdaBound{A} \AgdaBound{B}\AgdaSymbol{)}\<%
\\
\>\<\end{code}

\AgdaHide{
\begin{code}\>\<%
\\
\>\AgdaFunction{lam} \AgdaSymbol{\{}\AgdaBound{Γ}\AgdaSymbol{\}} \AgdaSymbol{\{}\AgdaBound{A}\AgdaSymbol{\}} \AgdaSymbol{(}\AgdaInductiveConstructor{tm:} \AgdaBound{tm} \AgdaInductiveConstructor{resp:} \AgdaBound{respt}\AgdaSymbol{)} \AgdaSymbol{=} \<[35]%
\>[35]\<%
\\
\>[0]\AgdaIndent{2}{}\<[2]%
\>[2]\AgdaKeyword{record} \AgdaSymbol{\{} \AgdaField{tm} \AgdaSymbol{=} \AgdaSymbol{λ} \AgdaBound{x} \AgdaSymbol{→} \AgdaSymbol{(λ} \AgdaBound{a} \AgdaSymbol{→} \AgdaBound{tm} \AgdaSymbol{(}\AgdaBound{x} \AgdaInductiveConstructor{,} \AgdaBound{a}\AgdaSymbol{))} \AgdaInductiveConstructor{,} \<[43]%
\>[43]\<%
\\
\>[2]\AgdaIndent{11}{}\<[11]%
\>[11]\AgdaSymbol{(λ} \AgdaBound{a} \AgdaBound{b} \AgdaBound{p} \AgdaSymbol{→} \AgdaBound{respt} \AgdaSymbol{(}\AgdaFunction{[} \AgdaBound{Γ} \AgdaFunction{]refl} \AgdaInductiveConstructor{,}\<%
\\
\>[11]\AgdaIndent{13}{}\<[13]%
\>[13]\AgdaFunction{[} \AgdaFunction{[} \AgdaBound{A} \AgdaFunction{]fm} \AgdaBound{x} \AgdaFunction{]trans} \AgdaSymbol{(}\AgdaFunction{[} \AgdaBound{A} \AgdaFunction{]refl*} \AgdaSymbol{\_} \AgdaSymbol{\_)} \AgdaBound{p}\AgdaSymbol{))}\<%
\\
\>[-7]\AgdaIndent{9}{}\<[9]%
\>[9]\AgdaSymbol{;} \AgdaField{respt} \AgdaSymbol{=} \AgdaSymbol{λ} \AgdaBound{p} \AgdaBound{\_} \AgdaSymbol{→} \AgdaBound{respt} \AgdaSymbol{(}\AgdaBound{p} \AgdaInductiveConstructor{,} \AgdaFunction{[} \AgdaBound{A} \AgdaFunction{]trans-refl}\AgdaSymbol{)} \<[55]%
\>[55]\<%
\\
\>[0]\AgdaIndent{9}{}\<[9]%
\>[9]\AgdaSymbol{\}}\<%
\\
%
\\
\>\<\end{code}
}


\begin{code}\>\<%
\\
\>\AgdaFunction{app} \AgdaSymbol{:} \AgdaSymbol{\{}\AgdaBound{Γ} \AgdaSymbol{:} \AgdaFunction{Con}\AgdaSymbol{\}\{}\AgdaBound{A} \AgdaSymbol{:} \AgdaRecord{Ty} \AgdaBound{Γ}\AgdaSymbol{\}\{}\AgdaBound{B} \AgdaSymbol{:} \AgdaRecord{Ty} \AgdaSymbol{(}\AgdaBound{Γ} \AgdaFunction{\&} \AgdaBound{A}\AgdaSymbol{)\}} \AgdaSymbol{→} \AgdaRecord{Tm} \AgdaSymbol{(}\AgdaFunction{Π} \AgdaBound{A} \AgdaBound{B}\AgdaSymbol{)} \AgdaSymbol{→} \AgdaRecord{Tm} \AgdaBound{B}\<%
\\
%
\\
\>\<\end{code}

\AgdaHide{
\begin{code}\>\<%
\\
\>\AgdaFunction{app} \AgdaSymbol{\{}\AgdaBound{Γ}\AgdaSymbol{\}} \AgdaSymbol{\{}\AgdaBound{A}\AgdaSymbol{\}} \AgdaSymbol{\{}\AgdaBound{B}\AgdaSymbol{\}} \AgdaSymbol{(}\AgdaInductiveConstructor{tm:} \AgdaBound{tm} \AgdaInductiveConstructor{resp:} \AgdaBound{respt}\AgdaSymbol{)} \AgdaSymbol{=} \<[39]%
\>[39]\<%
\\
\>[0]\AgdaIndent{2}{}\<[2]%
\>[2]\AgdaKeyword{record} \AgdaSymbol{\{} \AgdaField{tm} \AgdaSymbol{=} \AgdaSymbol{λ} \AgdaSymbol{\{(}\AgdaBound{x} \AgdaInductiveConstructor{,} \AgdaBound{a}\AgdaSymbol{)} \AgdaSymbol{→} \AgdaFunction{proj₁} \AgdaSymbol{(}\AgdaBound{tm} \AgdaBound{x}\AgdaSymbol{)} \AgdaBound{a}\AgdaSymbol{\}}\<%
\\
\>[0]\AgdaIndent{9}{}\<[9]%
\>[9]\AgdaSymbol{;} \AgdaField{respt} \AgdaSymbol{=} \AgdaSymbol{λ} \AgdaSymbol{\{}\AgdaBound{x}\AgdaSymbol{\}} \AgdaSymbol{\{}\AgdaBound{y}\AgdaSymbol{\}} \AgdaSymbol{→} \AgdaSymbol{λ} \AgdaSymbol{\{(}\AgdaBound{p} \AgdaInductiveConstructor{,} \AgdaBound{tr}\AgdaSymbol{)} \AgdaSymbol{→} \<[45]%
\>[45]\<%
\\
\>[9]\AgdaIndent{13}{}\<[13]%
\>[13]\AgdaKeyword{let} \AgdaBound{fresp} \AgdaSymbol{=} \AgdaFunction{proj₂} \AgdaSymbol{(}\AgdaBound{tm} \AgdaSymbol{(}\AgdaFunction{proj₁} \AgdaBound{x}\AgdaSymbol{))} \AgdaKeyword{in}\<%
\\
\>[13]\AgdaIndent{16}{}\<[16]%
\>[16]\AgdaFunction{[} \AgdaFunction{[} \AgdaBound{B} \AgdaFunction{]fm} \AgdaSymbol{\_} \AgdaFunction{]trans} \<[35]%
\>[35]\<%
\\
\>[13]\AgdaIndent{16}{}\<[16]%
\>[16]\AgdaSymbol{(}\AgdaFunction{[} \AgdaBound{B} \AgdaFunction{]subst*} \AgdaSymbol{(}\AgdaBound{p} \AgdaInductiveConstructor{,} \AgdaBound{tr}\AgdaSymbol{)} \AgdaSymbol{(}\AgdaFunction{[} \AgdaFunction{[} \AgdaBound{B} \AgdaFunction{]fm} \AgdaSymbol{\_} \AgdaFunction{]sym} \AgdaFunction{[} \AgdaBound{B} \AgdaFunction{]subst-pi'}\AgdaSymbol{))} \<[73]%
\>[73]\<%
\\
\>[13]\AgdaIndent{16}{}\<[16]%
\>[16]\AgdaSymbol{(}\AgdaFunction{[} \AgdaFunction{[} \AgdaBound{B} \AgdaFunction{]fm} \AgdaSymbol{\_} \AgdaFunction{]trans}\<%
\\
\>[13]\AgdaIndent{16}{}\<[16]%
\>[16]\AgdaSymbol{(}\AgdaFunction{[} \AgdaBound{B} \AgdaFunction{]trans*} \AgdaSymbol{(}\AgdaFunction{[} \AgdaBound{Γ} \AgdaFunction{]refl} \AgdaInductiveConstructor{,} \AgdaFunction{[} \AgdaBound{A} \AgdaFunction{]refl*} \AgdaSymbol{\_} \AgdaSymbol{\_)} \AgdaSymbol{\_} \AgdaSymbol{\_)} \<[63]%
\>[63]\<%
\\
\>[13]\AgdaIndent{16}{}\<[16]%
\>[16]\AgdaSymbol{(}\AgdaFunction{[} \AgdaFunction{[} \AgdaBound{B} \AgdaFunction{]fm} \AgdaSymbol{\_} \AgdaFunction{]trans} \<[36]%
\>[36]\<%
\\
\>[13]\AgdaIndent{16}{}\<[16]%
\>[16]\AgdaFunction{[} \AgdaBound{B} \AgdaFunction{]subst-pi} \<[30]%
\>[30]\<%
\\
\>[13]\AgdaIndent{16}{}\<[16]%
\>[16]\AgdaSymbol{(}\AgdaFunction{[} \AgdaFunction{[} \AgdaBound{B} \AgdaFunction{]fm} \AgdaSymbol{\_} \AgdaFunction{]trans} \<[36]%
\>[36]\<%
\\
\>[13]\AgdaIndent{16}{}\<[16]%
\>[16]\AgdaSymbol{(}\AgdaFunction{[} \AgdaFunction{[} \AgdaBound{B} \AgdaFunction{]fm} \AgdaSymbol{\_} \AgdaFunction{]sym} \AgdaSymbol{(}\AgdaFunction{[} \AgdaBound{B} \AgdaFunction{]trans*} \AgdaSymbol{\_} \AgdaSymbol{(}\AgdaBound{p} \AgdaInductiveConstructor{,} \AgdaFunction{[} \AgdaBound{A} \AgdaFunction{]trans-refl}\AgdaSymbol{)} \AgdaSymbol{\_))}\<%
\\
\>[13]\AgdaIndent{16}{}\<[16]%
\>[16]\AgdaSymbol{(}\AgdaFunction{[} \AgdaFunction{[} \AgdaBound{B} \AgdaFunction{]fm} \AgdaSymbol{\_} \AgdaFunction{]trans} \<[36]%
\>[36]\<%
\\
\>[13]\AgdaIndent{16}{}\<[16]%
\>[16]\AgdaSymbol{(}\AgdaFunction{[} \AgdaBound{B} \AgdaFunction{]subst-pi*} \AgdaSymbol{(}\AgdaBound{fresp} \AgdaSymbol{\_} \AgdaSymbol{\_} \AgdaSymbol{(}\AgdaFunction{[} \AgdaBound{A} \AgdaFunction{]subst-mir2} \AgdaBound{tr}\AgdaSymbol{)))} \<[66]%
\>[66]\<%
\\
\>[13]\AgdaIndent{16}{}\<[16]%
\>[16]\AgdaSymbol{(}\AgdaBound{respt} \AgdaBound{p} \AgdaSymbol{\_)))))} \AgdaSymbol{\}}\<%
\\
\>[-6]\AgdaIndent{9}{}\<[9]%
\>[9]\AgdaSymbol{\}}\<%
\\
%
\\
%
\\
\>\<\end{code}
}

Non-dependent version of $\Pi$-types namely function types can be defined with type weakening. Since the dependence disappears, it is possible to define it straightforwardly without using $\Pi$-types.

\begin{code}\>\<%
\\
%
\\
\>\AgdaFunction{\_⇒'\_} \AgdaSymbol{:} \AgdaSymbol{\{}\AgdaBound{Γ} \AgdaSymbol{:} \AgdaFunction{Con}\AgdaSymbol{\}(}\AgdaBound{A} \AgdaBound{B} \AgdaSymbol{:} \AgdaRecord{Ty} \AgdaBound{Γ}\AgdaSymbol{)} \AgdaSymbol{→} \AgdaRecord{Ty} \AgdaBound{Γ}\<%
\\
\>\AgdaBound{A} \AgdaFunction{⇒'} \AgdaBound{B} \AgdaSymbol{=} \AgdaFunction{Π} \AgdaBound{A} \AgdaSymbol{(}\AgdaBound{B} \AgdaFunction{+T} \AgdaBound{A}\AgdaSymbol{)}\<%
\\
%
\\
\>\<\end{code}


\AgdaHide{
\begin{code}\>\<%
\\
%
\\
\>\AgdaFunction{[\_,\_]\_⇒fm\_} \AgdaSymbol{:} \AgdaSymbol{(}\AgdaBound{Γ} \AgdaSymbol{:} \AgdaFunction{Con}\AgdaSymbol{)(}\AgdaBound{x} \AgdaSymbol{:} \AgdaFunction{∣} \AgdaBound{Γ} \AgdaFunction{∣}\AgdaSymbol{)} \<[34]%
\>[34]\<%
\\
\>[0]\AgdaIndent{11}{}\<[11]%
\>[11]\AgdaSymbol{→} \AgdaRecord{hSetoid} \AgdaSymbol{→} \AgdaRecord{hSetoid} \AgdaSymbol{→} \AgdaRecord{hSetoid}\<%
\\
\>\AgdaFunction{[} \AgdaBound{Γ} \AgdaFunction{,} \AgdaBound{x} \AgdaFunction{]} \AgdaBound{Ax} \AgdaFunction{⇒fm} \AgdaBound{Bx} \<[20]%
\>[20]\<%
\\
\>[0]\AgdaIndent{2}{}\<[2]%
\>[2]\AgdaSymbol{=} \AgdaKeyword{record}\<%
\\
\>[0]\AgdaIndent{6}{}\<[6]%
\>[6]\AgdaSymbol{\{} \AgdaField{Carrier} \AgdaSymbol{=} \AgdaRecord{Σ[} \AgdaBound{fn} \AgdaRecord{∶} \AgdaSymbol{(}\AgdaFunction{∣} \AgdaBound{Ax} \AgdaFunction{∣} \AgdaSymbol{→} \AgdaFunction{∣} \AgdaBound{Bx} \AgdaFunction{∣}\AgdaSymbol{)} \AgdaRecord{]} \<[46]%
\>[46]\<%
\\
\>[6]\AgdaIndent{16}{}\<[16]%
\>[16]\AgdaSymbol{((}\AgdaBound{a} \AgdaBound{b} \AgdaSymbol{:} \AgdaFunction{∣} \AgdaBound{Ax} \AgdaFunction{∣}\AgdaSymbol{)(}\AgdaBound{p} \AgdaSymbol{:} \AgdaFunction{[} \AgdaBound{Ax} \AgdaFunction{]} \AgdaBound{a} \AgdaFunction{≈} \AgdaBound{b}\AgdaSymbol{)} \<[50]%
\>[50]\<%
\\
\>[16]\AgdaIndent{18}{}\<[18]%
\>[18]\AgdaSymbol{→} \AgdaFunction{[} \AgdaBound{Bx} \AgdaFunction{]} \AgdaBound{fn} \AgdaBound{a} \AgdaFunction{≈} \AgdaBound{fn} \AgdaBound{b}\AgdaSymbol{)}\<%
\\
\>[-4]\AgdaIndent{6}{}\<[6]%
\>[6]\AgdaSymbol{;} \AgdaField{\_≈h\_} \<[16]%
\>[16]\AgdaSymbol{=} \AgdaSymbol{λ\{(}\AgdaBound{f} \AgdaInductiveConstructor{,} \AgdaSymbol{\_)} \AgdaSymbol{(}\AgdaBound{g} \AgdaInductiveConstructor{,} \AgdaSymbol{\_)} \<[36]%
\>[36]\<%
\\
\>[0]\AgdaIndent{18}{}\<[18]%
\>[18]\AgdaSymbol{→} \AgdaFunction{∀'[} \AgdaBound{a} \AgdaFunction{∶} \AgdaSymbol{\_} \AgdaFunction{]} \AgdaFunction{[} \AgdaBound{Bx} \AgdaFunction{]} \AgdaBound{f} \AgdaBound{a} \AgdaFunction{≈h} \AgdaBound{g} \AgdaBound{a} \AgdaSymbol{\}}\<%
\\
\>[0]\AgdaIndent{6}{}\<[6]%
\>[6]\AgdaSymbol{;} \AgdaField{isEquiv} \AgdaSymbol{=} \AgdaKeyword{record} \AgdaSymbol{\{}\<%
\\
\>[0]\AgdaIndent{16}{}\<[16]%
\>[16]\AgdaField{refl} \<[22]%
\>[22]\AgdaSymbol{=} \AgdaSymbol{λ} \AgdaBound{\_} \AgdaSymbol{→} \AgdaFunction{[} \AgdaBound{Bx} \AgdaFunction{]refl} \<[41]%
\>[41]\<%
\\
\>[0]\AgdaIndent{14}{}\<[14]%
\>[14]\AgdaSymbol{;} \AgdaField{sym} \<[22]%
\>[22]\AgdaSymbol{=} \AgdaSymbol{λ} \AgdaBound{f} \AgdaBound{a} \AgdaSymbol{→} \AgdaFunction{[} \AgdaBound{Bx} \AgdaFunction{]sym} \AgdaSymbol{(}\AgdaBound{f} \AgdaBound{a}\AgdaSymbol{)}\<%
\\
\>[0]\AgdaIndent{14}{}\<[14]%
\>[14]\AgdaSymbol{;} \AgdaField{trans} \AgdaSymbol{=} \AgdaSymbol{λ} \AgdaBound{f} \AgdaBound{g} \AgdaBound{a} \AgdaSymbol{→} \AgdaFunction{[} \AgdaBound{Bx} \AgdaFunction{]trans} \AgdaSymbol{(}\AgdaBound{f} \AgdaBound{a}\AgdaSymbol{)} \AgdaSymbol{(}\AgdaBound{g} \AgdaBound{a}\AgdaSymbol{)}\<%
\\
\>[14]\AgdaIndent{26}{}\<[26]%
\>[26]\AgdaSymbol{\}}\<%
\\
\>[6]\AgdaIndent{6}{}\<[6]%
\>[6]\AgdaSymbol{\}}\<%
\\
%
\\
%
\\
\>\<\end{code}

}

%to do: verification

%verification of functor laws (do we have extensional equality for record types? or eta equality?)
%define equality with respect to propositions which are proof irrelevant


\section{What we can do in this model}

\section{Examples of types}

\AgdaHide{
\begin{code}\>\<%
\\
%
\\
\>\AgdaSymbol{\{-\#} \AgdaKeyword{OPTIONS} --type-in-type \AgdaSymbol{\#-\}}\<%
\\
%
\\
\>\AgdaKeyword{import} \AgdaModule{Level}\<%
\\
\>\AgdaKeyword{open} \AgdaKeyword{import} \AgdaModule{Relation.Binary.PropositionalEquality} \AgdaSymbol{as} \AgdaModule{PE} \AgdaKeyword{hiding} \AgdaSymbol{(}refl \AgdaSymbol{;} sym \AgdaSymbol{;} trans\AgdaSymbol{;} isEquivalence\AgdaSymbol{;} [\_]\AgdaSymbol{)}\<%
\\
%
\\
\>\AgdaKeyword{module} \AgdaModule{CwF-ctd} \AgdaSymbol{(}\AgdaBound{ext} \AgdaSymbol{:} \AgdaFunction{Extensionality} \AgdaPrimitive{Level.zero} \AgdaPrimitive{Level.zero}\AgdaSymbol{)} \AgdaKeyword{where}\<%
\\
%
\\
\>\AgdaKeyword{open} \AgdaKeyword{import} \AgdaModule{Data.Unit}\<%
\\
\>\AgdaKeyword{open} \AgdaKeyword{import} \AgdaModule{Function}\<%
\\
\>\AgdaKeyword{open} \AgdaKeyword{import} \AgdaModule{Data.Product}\<%
\\
%
\\
\>\AgdaKeyword{open} \AgdaKeyword{import} \AgdaModule{CwF-setoidwo} \AgdaBound{ext} \AgdaKeyword{public}\<%
\\
%
\\
\>\AgdaKeyword{open} \AgdaKeyword{import} \AgdaModule{Data.Nat}\<%
\\
%
\\
\>\<\end{code}
}

Binary relation

\begin{code}\>\<%
\\
%
\\
\>\AgdaFunction{Rel} \AgdaSymbol{:} \AgdaSymbol{\{}\AgdaBound{Γ} \AgdaSymbol{:} \AgdaFunction{Con}\AgdaSymbol{\}} \AgdaSymbol{→} \AgdaRecord{Ty} \AgdaBound{Γ} \AgdaSymbol{→} \AgdaPrimitiveType{Set₁}\<%
\\
\>\AgdaFunction{Rel} \AgdaSymbol{\{}\AgdaBound{Γ}\AgdaSymbol{\}} \AgdaBound{A} \AgdaSymbol{=} \AgdaRecord{Ty} \AgdaSymbol{(}\AgdaBound{Γ} \AgdaFunction{\&} \AgdaBound{A} \AgdaFunction{\&} \AgdaBound{A} \AgdaFunction{[} \AgdaFunction{fst\&} \AgdaSymbol{\{}A \AgdaSymbol{=} \AgdaBound{A}\AgdaSymbol{\}} \AgdaFunction{]T}\AgdaSymbol{)}\<%
\\
%
\\
\>\<\end{code}

Natural numbers

\begin{code}\>\<%
\\
%
\\
\>\AgdaKeyword{module} \AgdaModule{Natural} \AgdaSymbol{(}\AgdaBound{Γ} \AgdaSymbol{:} \AgdaFunction{Con}\AgdaSymbol{)} \AgdaKeyword{where}\<%
\\
%
\\
\>[0]\AgdaIndent{2}{}\<[2]%
\>[2]\AgdaFunction{\_≈nat\_} \AgdaSymbol{:} \AgdaDatatype{ℕ} \AgdaSymbol{→} \AgdaDatatype{ℕ} \AgdaSymbol{→} \AgdaRecord{HProp}\<%
\\
\>[0]\AgdaIndent{2}{}\<[2]%
\>[2]\AgdaInductiveConstructor{zero} \AgdaFunction{≈nat} \AgdaInductiveConstructor{zero} \AgdaSymbol{=} \AgdaFunction{⊤'}\<%
\\
\>[0]\AgdaIndent{2}{}\<[2]%
\>[2]\AgdaInductiveConstructor{zero} \AgdaFunction{≈nat} \AgdaInductiveConstructor{suc} \AgdaBound{n} \AgdaSymbol{=} \AgdaFunction{⊥'}\<%
\\
\>[0]\AgdaIndent{2}{}\<[2]%
\>[2]\AgdaInductiveConstructor{suc} \AgdaBound{m} \AgdaFunction{≈nat} \AgdaInductiveConstructor{zero} \AgdaSymbol{=} \AgdaFunction{⊥'}\<%
\\
\>[0]\AgdaIndent{2}{}\<[2]%
\>[2]\AgdaInductiveConstructor{suc} \AgdaBound{m} \AgdaFunction{≈nat} \AgdaInductiveConstructor{suc} \AgdaBound{n} \AgdaSymbol{=} \AgdaBound{m} \AgdaFunction{≈nat} \AgdaBound{n}\<%
\\
\>[0]\AgdaIndent{2}{}\<[2]%
\>[2]\<%
\\
\>[0]\AgdaIndent{2}{}\<[2]%
\>[2]\AgdaFunction{reflNat} \AgdaSymbol{:} \AgdaSymbol{\{}\AgdaBound{x} \AgdaSymbol{:} \AgdaDatatype{ℕ}\AgdaSymbol{\}} \AgdaSymbol{→} \AgdaFunction{<} \AgdaBound{x} \AgdaFunction{≈nat} \AgdaBound{x} \AgdaFunction{>} \<[35]%
\>[35]\<%
\\
\>[0]\AgdaIndent{2}{}\<[2]%
\>[2]\AgdaFunction{reflNat} \AgdaSymbol{\{}\AgdaInductiveConstructor{zero}\AgdaSymbol{\}} \AgdaSymbol{=} \AgdaInductiveConstructor{tt}\<%
\\
\>[0]\AgdaIndent{2}{}\<[2]%
\>[2]\AgdaFunction{reflNat} \AgdaSymbol{\{}\AgdaInductiveConstructor{suc} \AgdaBound{n}\AgdaSymbol{\}} \AgdaSymbol{=} \AgdaFunction{reflNat} \AgdaSymbol{\{}\AgdaBound{n}\AgdaSymbol{\}}\<%
\\
%
\\
\>[0]\AgdaIndent{2}{}\<[2]%
\>[2]\AgdaFunction{symNat} \AgdaSymbol{:} \AgdaSymbol{\{}\AgdaBound{x} \AgdaBound{y} \AgdaSymbol{:} \AgdaDatatype{ℕ}\AgdaSymbol{\}} \AgdaSymbol{→} \AgdaFunction{<} \AgdaBound{x} \AgdaFunction{≈nat} \AgdaBound{y} \AgdaFunction{>} \AgdaSymbol{→} \AgdaFunction{<} \AgdaBound{y} \AgdaFunction{≈nat} \AgdaBound{x} \AgdaFunction{>}\<%
\\
\>[0]\AgdaIndent{2}{}\<[2]%
\>[2]\AgdaFunction{symNat} \AgdaSymbol{\{}\AgdaInductiveConstructor{zero}\AgdaSymbol{\}} \AgdaSymbol{\{}\AgdaInductiveConstructor{zero}\AgdaSymbol{\}} \AgdaBound{eq} \AgdaSymbol{=} \AgdaInductiveConstructor{tt}\<%
\\
\>[0]\AgdaIndent{2}{}\<[2]%
\>[2]\AgdaFunction{symNat} \AgdaSymbol{\{}\AgdaInductiveConstructor{zero}\AgdaSymbol{\}} \AgdaSymbol{\{}\AgdaInductiveConstructor{suc} \AgdaSymbol{\_\}} \AgdaBound{eq} \AgdaSymbol{=} \AgdaBound{eq}\<%
\\
\>[0]\AgdaIndent{2}{}\<[2]%
\>[2]\AgdaFunction{symNat} \AgdaSymbol{\{}\AgdaInductiveConstructor{suc} \AgdaSymbol{\_\}} \AgdaSymbol{\{}\AgdaInductiveConstructor{zero}\AgdaSymbol{\}} \AgdaBound{eq} \AgdaSymbol{=} \AgdaBound{eq}\<%
\\
\>[0]\AgdaIndent{2}{}\<[2]%
\>[2]\AgdaFunction{symNat} \AgdaSymbol{\{}\AgdaInductiveConstructor{suc} \AgdaBound{x}\AgdaSymbol{\}} \AgdaSymbol{\{}\AgdaInductiveConstructor{suc} \AgdaBound{y}\AgdaSymbol{\}} \AgdaBound{eq} \AgdaSymbol{=} \AgdaFunction{symNat} \AgdaSymbol{\{}\AgdaBound{x}\AgdaSymbol{\}} \AgdaSymbol{\{}\AgdaBound{y}\AgdaSymbol{\}} \AgdaBound{eq}\<%
\\
%
\\
\>[0]\AgdaIndent{2}{}\<[2]%
\>[2]\AgdaFunction{transNat} \AgdaSymbol{:} \AgdaSymbol{\{}\AgdaBound{x} \AgdaBound{y} \AgdaBound{z} \AgdaSymbol{:} \AgdaDatatype{ℕ}\AgdaSymbol{\}} \AgdaSymbol{→} \AgdaFunction{<} \AgdaBound{x} \AgdaFunction{≈nat} \AgdaBound{y} \AgdaFunction{>} \AgdaSymbol{→} \AgdaFunction{<} \AgdaBound{y} \AgdaFunction{≈nat} \AgdaBound{z} \AgdaFunction{>} \AgdaSymbol{→} \AgdaFunction{<} \AgdaBound{x} \AgdaFunction{≈nat} \AgdaBound{z} \AgdaFunction{>}\<%
\\
\>[0]\AgdaIndent{2}{}\<[2]%
\>[2]\AgdaFunction{transNat} \AgdaSymbol{\{}\AgdaInductiveConstructor{zero}\AgdaSymbol{\}} \AgdaSymbol{\{}\AgdaInductiveConstructor{zero}\AgdaSymbol{\}} \AgdaBound{xy} \AgdaBound{yz} \AgdaSymbol{=} \AgdaBound{yz}\<%
\\
\>[0]\AgdaIndent{2}{}\<[2]%
\>[2]\AgdaFunction{transNat} \AgdaSymbol{\{}\AgdaInductiveConstructor{zero}\AgdaSymbol{\}} \AgdaSymbol{\{}\AgdaInductiveConstructor{suc} \AgdaSymbol{\_\}} \AgdaSymbol{()} \AgdaBound{yz}\<%
\\
\>[0]\AgdaIndent{2}{}\<[2]%
\>[2]\AgdaFunction{transNat} \AgdaSymbol{\{}\AgdaInductiveConstructor{suc} \AgdaSymbol{\_\}} \AgdaSymbol{\{}\AgdaInductiveConstructor{zero}\AgdaSymbol{\}} \AgdaSymbol{()} \AgdaBound{yz}\<%
\\
\>[0]\AgdaIndent{2}{}\<[2]%
\>[2]\AgdaFunction{transNat} \AgdaSymbol{\{}\AgdaInductiveConstructor{suc} \AgdaSymbol{\_\}} \AgdaSymbol{\{}\AgdaInductiveConstructor{suc} \AgdaSymbol{\_\}} \AgdaSymbol{\{}\AgdaInductiveConstructor{zero}\AgdaSymbol{\}} \AgdaBound{xy} \AgdaBound{yz} \AgdaSymbol{=} \AgdaBound{yz}\<%
\\
\>[0]\AgdaIndent{2}{}\<[2]%
\>[2]\AgdaFunction{transNat} \AgdaSymbol{\{}\AgdaInductiveConstructor{suc} \AgdaBound{x}\AgdaSymbol{\}} \AgdaSymbol{\{}\AgdaInductiveConstructor{suc} \AgdaBound{y}\AgdaSymbol{\}} \AgdaSymbol{\{}\AgdaInductiveConstructor{suc} \AgdaBound{z}\AgdaSymbol{\}} \AgdaBound{xy} \AgdaBound{yz} \AgdaSymbol{=} \AgdaFunction{transNat} \AgdaSymbol{\{}\AgdaBound{x}\AgdaSymbol{\}} \AgdaSymbol{\{}\AgdaBound{y}\AgdaSymbol{\}} \AgdaSymbol{\{}\AgdaBound{z}\AgdaSymbol{\}} \AgdaBound{xy} \AgdaBound{yz}\<%
\\
%
\\
\>[0]\AgdaIndent{2}{}\<[2]%
\>[2]\AgdaFunction{⟦Nat⟧} \AgdaSymbol{:} \AgdaRecord{Ty} \AgdaBound{Γ}\<%
\\
\>[0]\AgdaIndent{2}{}\<[2]%
\>[2]\AgdaFunction{⟦Nat⟧} \AgdaSymbol{=} \AgdaKeyword{record} \<[17]%
\>[17]\<%
\\
\>[2]\AgdaIndent{4}{}\<[4]%
\>[4]\AgdaSymbol{\{} \AgdaField{fm} \AgdaSymbol{=} \AgdaSymbol{λ} \AgdaBound{γ} \AgdaSymbol{→} \AgdaKeyword{record}\<%
\\
\>[4]\AgdaIndent{9}{}\<[9]%
\>[9]\AgdaSymbol{\{} \AgdaField{Carrier} \AgdaSymbol{=} \AgdaDatatype{ℕ}\<%
\\
\>[4]\AgdaIndent{9}{}\<[9]%
\>[9]\AgdaSymbol{;} \AgdaField{\_≈h\_} \AgdaSymbol{=} \AgdaFunction{\_≈nat\_}\<%
\\
\>[4]\AgdaIndent{9}{}\<[9]%
\>[9]\AgdaSymbol{;} \AgdaField{refl} \AgdaSymbol{=} \AgdaSymbol{λ} \AgdaSymbol{\{}\AgdaBound{n}\AgdaSymbol{\}} \AgdaSymbol{→} \AgdaFunction{reflNat} \AgdaSymbol{\{}\AgdaBound{n}\AgdaSymbol{\}}\<%
\\
\>[4]\AgdaIndent{9}{}\<[9]%
\>[9]\AgdaSymbol{;} \AgdaField{sym} \AgdaSymbol{=} \AgdaSymbol{λ} \AgdaSymbol{\{}\AgdaBound{x}\AgdaSymbol{\}} \AgdaSymbol{\{}\AgdaBound{y}\AgdaSymbol{\}} \AgdaSymbol{→} \AgdaFunction{symNat} \AgdaSymbol{\{}\AgdaBound{x}\AgdaSymbol{\}} \AgdaSymbol{\{}\AgdaBound{y}\AgdaSymbol{\}}\<%
\\
\>[4]\AgdaIndent{9}{}\<[9]%
\>[9]\AgdaSymbol{;} \AgdaField{trans} \AgdaSymbol{=} \AgdaSymbol{λ} \AgdaSymbol{\{}\AgdaBound{x}\AgdaSymbol{\}} \AgdaSymbol{\{}\AgdaBound{y}\AgdaSymbol{\}} \AgdaSymbol{\{}\AgdaBound{z}\AgdaSymbol{\}} \AgdaSymbol{→} \AgdaFunction{transNat} \AgdaSymbol{\{}\AgdaBound{x}\AgdaSymbol{\}} \AgdaSymbol{\{}\AgdaBound{y}\AgdaSymbol{\}} \AgdaSymbol{\{}\AgdaBound{z}\AgdaSymbol{\}}\<%
\\
\>[4]\AgdaIndent{9}{}\<[9]%
\>[9]\AgdaSymbol{\}}\<%
\\
\>[0]\AgdaIndent{4}{}\<[4]%
\>[4]\AgdaSymbol{;} \AgdaField{substT} \AgdaSymbol{=} \AgdaSymbol{λ} \AgdaBound{\_} \AgdaSymbol{→} \AgdaFunction{id}\<%
\\
\>[0]\AgdaIndent{4}{}\<[4]%
\>[4]\AgdaSymbol{;} \AgdaField{subst*} \AgdaSymbol{=} \AgdaSymbol{λ} \AgdaBound{\_} \AgdaSymbol{→} \AgdaFunction{id}\<%
\\
\>[0]\AgdaIndent{4}{}\<[4]%
\>[4]\AgdaSymbol{;} \AgdaField{refl*} \AgdaSymbol{=} \AgdaSymbol{λ} \AgdaBound{x} \AgdaBound{a} \AgdaSymbol{→} \AgdaFunction{reflNat} \AgdaSymbol{\{}\AgdaBound{a}\AgdaSymbol{\}}\<%
\\
\>[0]\AgdaIndent{4}{}\<[4]%
\>[4]\AgdaSymbol{;} \AgdaField{trans*} \AgdaSymbol{=} \AgdaSymbol{λ} \AgdaBound{a} \AgdaSymbol{→} \AgdaFunction{reflNat} \AgdaSymbol{\{}\AgdaBound{a}\AgdaSymbol{\}} \<[33]%
\>[33]\<%
\\
\>[0]\AgdaIndent{4}{}\<[4]%
\>[4]\AgdaSymbol{\}}\<%
\\
%
\\
\>[0]\AgdaIndent{2}{}\<[2]%
\>[2]\AgdaFunction{⟦0⟧} \AgdaSymbol{:} \AgdaRecord{Tm} \AgdaFunction{⟦Nat⟧}\<%
\\
\>[0]\AgdaIndent{2}{}\<[2]%
\>[2]\AgdaFunction{⟦0⟧} \AgdaSymbol{=} \AgdaKeyword{record}\<%
\\
\>[2]\AgdaIndent{6}{}\<[6]%
\>[6]\AgdaSymbol{\{} \AgdaField{tm} \AgdaSymbol{=} \AgdaSymbol{λ} \AgdaBound{\_} \AgdaSymbol{→} \AgdaNumber{0}\<%
\\
\>[2]\AgdaIndent{6}{}\<[6]%
\>[6]\AgdaSymbol{;} \AgdaField{respt} \AgdaSymbol{=} \AgdaSymbol{λ} \AgdaBound{p} \AgdaSymbol{→} \AgdaInductiveConstructor{tt}\<%
\\
\>[2]\AgdaIndent{6}{}\<[6]%
\>[6]\AgdaSymbol{\}}\<%
\\
%
\\
\>[0]\AgdaIndent{2}{}\<[2]%
\>[2]\AgdaFunction{⟦s⟧} \AgdaSymbol{:} \AgdaRecord{Tm} \AgdaFunction{⟦Nat⟧} \AgdaSymbol{→} \AgdaRecord{Tm} \AgdaFunction{⟦Nat⟧}\<%
\\
\>[0]\AgdaIndent{2}{}\<[2]%
\>[2]\AgdaFunction{⟦s⟧} \AgdaSymbol{(}\AgdaInductiveConstructor{tm:} \AgdaBound{t} \AgdaInductiveConstructor{resp:} \AgdaBound{respt}\AgdaSymbol{)} \<[26]%
\>[26]\<%
\\
\>[2]\AgdaIndent{6}{}\<[6]%
\>[6]\AgdaSymbol{=} \AgdaKeyword{record}\<%
\\
\>[2]\AgdaIndent{6}{}\<[6]%
\>[6]\AgdaSymbol{\{} \AgdaField{tm} \AgdaSymbol{=} \AgdaInductiveConstructor{suc} \AgdaFunction{∘} \AgdaBound{t}\<%
\\
\>[2]\AgdaIndent{6}{}\<[6]%
\>[6]\AgdaSymbol{;} \AgdaField{respt} \AgdaSymbol{=} \AgdaBound{respt}\<%
\\
\>[2]\AgdaIndent{6}{}\<[6]%
\>[6]\AgdaSymbol{\}}\<%
\\
%
\\
\>\<\end{code}

Simply typed universe

\AgdaHide{
\begin{code}\>\<%
\\
%
\\
\>\AgdaComment{\{-
  data  ⟦U⟧⁰ : Set where
    nat : ⟦U⟧⁰
    arr<\_,\_> : (a b : ⟦U⟧⁰) → ⟦U⟧⁰

  \_\textasciitilde⟦U⟧\_ : ⟦U⟧⁰ → ⟦U⟧⁰ → HProp
  nat \textasciitilde⟦U⟧ nat = ⊤'
  nat \textasciitilde⟦U⟧ arr< a , b > = ⊥'
  arr< a , b > \textasciitilde⟦U⟧ nat = ⊥'
  arr< a , b > \textasciitilde⟦U⟧ arr< a' , b' > = a \textasciitilde⟦U⟧ a' ∧ b \textasciitilde⟦U⟧ b'

  reflU :  \{x : ⟦U⟧⁰\} → < x \textasciitilde⟦U⟧ x >
  reflU \{nat\} = tt
  reflU \{arr< a , b >\} = reflU \{a\} , reflU \{b\}

  symU : \{x y : ⟦U⟧⁰\} → < x \textasciitilde⟦U⟧ y > → < y \textasciitilde⟦U⟧ x >
  symU \{nat\} \{nat\} eq = tt
  symU \{nat\} \{arr< a , b >\} eq = eq
  symU \{arr< a , b >\} \{nat\} eq = eq
  symU \{arr< a , b >\} \{arr< a' , b' >\} (p , q) = (symU \{a\} \{a'\} p) 
                                               , (symU \{b\} \{b'\} q)

  transU : \{x y z : ⟦U⟧⁰\} → < x \textasciitilde⟦U⟧ y > → < y \textasciitilde⟦U⟧ z > → < x \textasciitilde⟦U⟧ z >
  transU \{nat\} \{nat\} eq1 eq2 = eq2
  transU \{nat\} \{arr< a , b >\} () eq2
  transU \{arr< a , b >\} \{nat\} () eq2
  transU \{arr< a , b >\} \{arr< a' , b' >\} \{nat\} eq1 eq2 = eq2
  transU \{arr< a , b >\} \{arr< a' , b' >\} \{arr< a0 , b0 >\} (p1 , q1) 
         (p2 , q2) = (transU \{a\} \{a'\} \{a0\} p1 p2) 
         , transU \{b\} \{b'\} \{b0\} q1 q2

  ⟦U⟧ : Ty Γ
  ⟦U⟧ = record 
    \{ fm = λ γ → record
         \{ Carrier = ⟦U⟧⁰
         ; \_≈h\_ = \_\textasciitilde⟦U⟧\_
         ; refl = λ \{x\} → reflU \{x\}
         ; sym = λ \{x\} \{y\} → symU \{x\} \{y\}
         ; trans = λ \{x\} \{y\} \{z\} → transU \{x\} \{y\} \{z\}
         \}
    ; substT = λ \_ → id
    ; subst* = λ \_ → id
    ; refl* = λ x a → reflU \{a\}
    ; trans* = λ a → reflU \{a\}
    \}

  elfm : Σ ∣ Γ ∣ (λ x → ⟦U⟧⁰) → HSetoid
  elfm (γ , nat) = [ ⟦Nat⟧ ]fm γ
  elfm (γ , arr< a , b >) = [ Γ , γ ] elfm (γ , a) ⇒fm elfm (γ , b)
-\}}\<%
\\
%
\\
\>\<\end{code}
}

\AgdaHide{
\begin{code}\>\<%
\\
%
\\
\>\AgdaComment{\{- To do : To find the way to extract the substT from ->

  elsubstT : \{x y : Σ ∣ Γ ∣ (λ x' → ⟦U⟧⁰)\} →
      Σ < [ Γ ] proj₁ x ≈h proj₁ y > (λ x' → < proj₂ x \textasciitilde⟦U⟧ proj₂ y >) →
      ∣ elfm x ∣ → ∣ elfm y ∣
  elsubstT \{\_ , nat\} \{\_ , nat\} \_ x' = x'
  elsubstT \{\_ , nat\} \{\_ , arr< a , b >\} (p , ()) x'
  elsubstT \{\_ , arr< a , b >\} \{\_ , nat\} (p , ()) x'
  elsubstT \{γ , arr< a , b >\} \{γ' , arr< a' , b' >\} (p , qa , qb) (s1 , s2) = 
   \{!!\}

  ⟦El⟧ : Ty (Γ \& ⟦U⟧)
  ⟦El⟧ = record 
       \{ fm = elfm
       ; substT = elsubstT
       ; subst* = \{!!\}
       ; refl* = \{!!\}
       ; trans* = \{!!\} 
       \}

-\}}\<%
\\
\>\<\end{code}
}

The equality type

\begin{code}\>\<%
\\
%
\\
\>\AgdaKeyword{module} \AgdaModule{Equality-Type} \AgdaSymbol{(}\AgdaBound{Γ} \AgdaSymbol{:} \AgdaFunction{Con}\AgdaSymbol{)(}\AgdaBound{A} \AgdaSymbol{:} \AgdaRecord{Ty} \AgdaBound{Γ}\AgdaSymbol{)} \AgdaKeyword{where}\<%
\\
%
\\
\>[0]\AgdaIndent{2}{}\<[2]%
\>[2]\AgdaFunction{⟦Id⟧} \AgdaSymbol{:} \AgdaFunction{Rel} \AgdaBound{A}\<%
\\
\>[0]\AgdaIndent{2}{}\<[2]%
\>[2]\AgdaFunction{⟦Id⟧} \AgdaSymbol{=} \AgdaKeyword{record} \<[16]%
\>[16]\<%
\\
\>[2]\AgdaIndent{4}{}\<[4]%
\>[4]\AgdaSymbol{\{} \AgdaField{fm} \AgdaSymbol{=} \AgdaSymbol{λ} \AgdaSymbol{\{((}\AgdaBound{x} \AgdaInductiveConstructor{,} \AgdaBound{a}\AgdaSymbol{)} \AgdaInductiveConstructor{,} \AgdaBound{b}\AgdaSymbol{)} \AgdaSymbol{→} \<[30]%
\>[30]\<%
\\
\>[4]\AgdaIndent{13}{}\<[13]%
\>[13]\AgdaKeyword{record} \<[20]%
\>[20]\<%
\\
\>[4]\AgdaIndent{13}{}\<[13]%
\>[13]\AgdaSymbol{\{} \AgdaField{Carrier} \AgdaSymbol{=} \AgdaFunction{[} \AgdaFunction{[} \AgdaBound{A} \AgdaFunction{]fm} \AgdaBound{x} \AgdaFunction{]} \AgdaBound{a} \AgdaFunction{≈} \AgdaBound{b}\<%
\\
\>[4]\AgdaIndent{13}{}\<[13]%
\>[13]\AgdaSymbol{;} \AgdaField{\_≈h\_} \AgdaSymbol{=} \AgdaSymbol{λ} \AgdaBound{\_} \AgdaBound{\_} \AgdaSymbol{→} \AgdaKeyword{record} \AgdaSymbol{\{} \AgdaField{prf} \AgdaSymbol{=} \AgdaRecord{⊤} \AgdaSymbol{;} \AgdaField{Uni} \AgdaSymbol{=} \AgdaInductiveConstructor{PE.refl} \AgdaSymbol{\}}\<%
\\
\>[4]\AgdaIndent{13}{}\<[13]%
\>[13]\AgdaSymbol{;} \AgdaField{refl} \AgdaSymbol{=} \AgdaInductiveConstructor{tt} \<[25]%
\>[25]\<%
\\
\>[4]\AgdaIndent{13}{}\<[13]%
\>[13]\AgdaSymbol{;} \AgdaField{sym} \AgdaSymbol{=} \AgdaFunction{id}\<%
\\
\>[4]\AgdaIndent{13}{}\<[13]%
\>[13]\AgdaSymbol{;} \AgdaField{trans} \AgdaSymbol{=} \AgdaSymbol{λ} \AgdaBound{\_} \AgdaBound{\_} \AgdaSymbol{→} \AgdaInductiveConstructor{tt}\<%
\\
\>[4]\AgdaIndent{13}{}\<[13]%
\>[13]\AgdaSymbol{\}}\<%
\\
\>[4]\AgdaIndent{13}{}\<[13]%
\>[13]\AgdaSymbol{\}}\<%
\\
\>[0]\AgdaIndent{4}{}\<[4]%
\>[4]\AgdaSymbol{;} \AgdaField{substT} \AgdaSymbol{=} \AgdaSymbol{λ} \AgdaSymbol{\{((}\AgdaBound{x} \AgdaInductiveConstructor{,} \AgdaBound{a}\AgdaSymbol{)} \AgdaInductiveConstructor{,} \AgdaBound{b}\AgdaSymbol{)} \AgdaBound{x0} \AgdaSymbol{→} \<[37]%
\>[37]\<%
\\
\>[0]\AgdaIndent{15}{}\<[15]%
\>[15]\AgdaFunction{[} \AgdaFunction{[} \AgdaBound{A} \AgdaFunction{]fm} \AgdaSymbol{\_} \AgdaFunction{]trans} \<[34]%
\>[34]\<%
\\
\>[0]\AgdaIndent{15}{}\<[15]%
\>[15]\AgdaSymbol{(}\AgdaFunction{[} \AgdaFunction{[} \AgdaBound{A} \AgdaFunction{]fm} \AgdaSymbol{\_} \AgdaFunction{]sym} \AgdaBound{a}\AgdaSymbol{)} \<[36]%
\>[36]\<%
\\
\>[0]\AgdaIndent{15}{}\<[15]%
\>[15]\AgdaSymbol{(}\AgdaFunction{[} \AgdaFunction{[} \AgdaBound{A} \AgdaFunction{]fm} \AgdaSymbol{\_} \AgdaFunction{]trans} \<[35]%
\>[35]\<%
\\
\>[0]\AgdaIndent{15}{}\<[15]%
\>[15]\AgdaSymbol{(}\AgdaFunction{[} \AgdaBound{A} \AgdaFunction{]subst*} \AgdaSymbol{\_} \AgdaBound{x0}\AgdaSymbol{)} \AgdaBound{b}\AgdaSymbol{)} \<[37]%
\>[37]\<%
\\
\>[0]\AgdaIndent{15}{}\<[15]%
\>[15]\AgdaSymbol{\}}\<%
\\
\>[0]\AgdaIndent{4}{}\<[4]%
\>[4]\AgdaSymbol{;} \AgdaField{subst*} \AgdaSymbol{=} \AgdaSymbol{λ} \AgdaBound{\_} \AgdaBound{\_} \AgdaSymbol{→} \AgdaInductiveConstructor{tt}\<%
\\
\>[0]\AgdaIndent{4}{}\<[4]%
\>[4]\AgdaSymbol{;} \AgdaField{refl*} \AgdaSymbol{=} \AgdaSymbol{λ} \AgdaBound{\_} \AgdaBound{\_} \AgdaSymbol{→} \AgdaInductiveConstructor{tt}\<%
\\
\>[0]\AgdaIndent{4}{}\<[4]%
\>[4]\AgdaSymbol{;} \AgdaField{trans*} \AgdaSymbol{=} \AgdaSymbol{λ} \AgdaBound{\_} \AgdaSymbol{→} \AgdaInductiveConstructor{tt} \<[24]%
\>[24]\<%
\\
\>[0]\AgdaIndent{4}{}\<[4]%
\>[4]\AgdaSymbol{\}}\<%
\\
%
\\
%
\\
\>[0]\AgdaIndent{2}{}\<[2]%
\>[2]\AgdaFunction{⟦refl⟧⁰} \AgdaSymbol{:} \AgdaRecord{Tm} \AgdaSymbol{\{}\AgdaBound{Γ} \AgdaFunction{\&} \AgdaBound{A}\AgdaSymbol{\}} \AgdaSymbol{(}\AgdaFunction{⟦Id⟧} \AgdaFunction{[} \AgdaKeyword{record} \AgdaSymbol{\{} \AgdaField{fn} \AgdaSymbol{=} \AgdaSymbol{λ} \AgdaBound{x'} \AgdaSymbol{→} \AgdaBound{x'} \AgdaInductiveConstructor{,} \AgdaFunction{proj₂} \AgdaBound{x'} \<[66]%
\>[66]\<%
\\
\>[0]\AgdaIndent{23}{}\<[23]%
\>[23]\AgdaSymbol{;} \AgdaField{resp} \AgdaSymbol{=} \AgdaSymbol{λ} \AgdaBound{x'} \AgdaSymbol{→} \AgdaBound{x'} \AgdaInductiveConstructor{,} \AgdaFunction{proj₂} \AgdaBound{x'} \AgdaSymbol{\}} \AgdaFunction{]T}\AgdaSymbol{)} \<[59]%
\>[59]\<%
\\
\>[0]\AgdaIndent{2}{}\<[2]%
\>[2]\AgdaFunction{⟦refl⟧⁰} \AgdaSymbol{=} \AgdaKeyword{record}\<%
\\
\>[0]\AgdaIndent{11}{}\<[11]%
\>[11]\AgdaSymbol{\{} \AgdaField{tm} \AgdaSymbol{=} \AgdaSymbol{λ} \AgdaSymbol{\{(}\AgdaBound{x} \AgdaInductiveConstructor{,} \AgdaBound{a}\AgdaSymbol{)} \AgdaSymbol{→} \AgdaFunction{[} \AgdaFunction{[} \AgdaBound{A} \AgdaFunction{]fm} \AgdaBound{x} \AgdaFunction{]refl} \AgdaSymbol{\{}\AgdaBound{a}\AgdaSymbol{\}} \AgdaSymbol{\}}\<%
\\
\>[0]\AgdaIndent{11}{}\<[11]%
\>[11]\AgdaSymbol{;} \AgdaField{respt} \AgdaSymbol{=} \AgdaSymbol{λ} \AgdaBound{p} \AgdaSymbol{→} \AgdaInductiveConstructor{tt}\<%
\\
\>[0]\AgdaIndent{11}{}\<[11]%
\>[11]\AgdaSymbol{\}}\<%
\\
%
\\
\>[0]\AgdaIndent{2}{}\<[2]%
\>[2]\AgdaFunction{⟦refl⟧} \AgdaSymbol{=} \<[12]%
\>[12]\AgdaFunction{lam} \AgdaSymbol{\{}\AgdaBound{Γ}\AgdaSymbol{\}} \AgdaSymbol{\{}\AgdaBound{A}\AgdaSymbol{\}} \AgdaFunction{⟦refl⟧⁰}\<%
\\
%
\\
\>\<\end{code}

Subst using equality types

\begin{code}\>\<%
\\
%
\\
\>[0]\AgdaIndent{2}{}\<[2]%
\>[2]\AgdaKeyword{module} \AgdaModule{substIn} \AgdaSymbol{(}\AgdaBound{B} \AgdaSymbol{:} \AgdaRecord{Ty} \AgdaSymbol{(}\AgdaBound{Γ} \AgdaFunction{\&} \AgdaBound{A}\AgdaSymbol{))} \AgdaKeyword{where}\<%
\\
\>[0]\AgdaIndent{2}{}\<[2]%
\>[2]\<%
\\
\>[2]\AgdaIndent{4}{}\<[4]%
\>[4]\AgdaFunction{⟦subst⟧⁰} \AgdaSymbol{:} \AgdaRecord{Tm} \AgdaSymbol{\{}\AgdaBound{Γ} \AgdaFunction{\&} \AgdaBound{A} \AgdaFunction{\&} \AgdaSymbol{(}\AgdaBound{A} \AgdaFunction{[} \AgdaFunction{fst\&} \AgdaSymbol{\{}A \AgdaSymbol{=} \AgdaBound{A}\AgdaSymbol{\}} \AgdaFunction{]T}\AgdaSymbol{)} \<[49]%
\>[49]\<%
\\
\>[4]\AgdaIndent{15}{}\<[15]%
\>[15]\AgdaFunction{\&} \AgdaFunction{⟦Id⟧} \AgdaFunction{\&} \AgdaBound{B} \AgdaFunction{[} \AgdaFunction{fst\&} \AgdaSymbol{\{}A \AgdaSymbol{=} \AgdaBound{A} \AgdaFunction{[} \AgdaFunction{fst\&} \AgdaSymbol{\{}A \AgdaSymbol{=} \AgdaBound{A}\AgdaSymbol{\}} \AgdaFunction{]T}\AgdaSymbol{\}} \<[60]%
\>[60]\AgdaFunction{]T} \<[63]%
\>[63]\<%
\\
\>[4]\AgdaIndent{15}{}\<[15]%
\>[15]\AgdaFunction{[} \AgdaFunction{fst\&} \AgdaSymbol{\{}A \AgdaSymbol{=} \AgdaFunction{⟦Id⟧}\AgdaSymbol{\}} \AgdaFunction{]T}\AgdaSymbol{\}} \<[37]%
\>[37]\<%
\\
\>[-2]\AgdaIndent{13}{}\<[13]%
\>[13]\AgdaSymbol{(}\AgdaBound{B} \AgdaFunction{[} \AgdaKeyword{record} \AgdaSymbol{\{} \AgdaField{fn} \AgdaSymbol{=} \AgdaSymbol{λ} \AgdaBound{x} \AgdaSymbol{→} \AgdaSymbol{(}\AgdaFunction{proj₁} \AgdaSymbol{(}\AgdaFunction{proj₁} \AgdaSymbol{(}\AgdaFunction{proj₁} \AgdaSymbol{(}\AgdaFunction{proj₁} \AgdaBound{x}\AgdaSymbol{))))} \<[72]%
\>[72]\<%
\\
\>[0]\AgdaIndent{13}{}\<[13]%
\>[13]\AgdaInductiveConstructor{,} \AgdaSymbol{(}\AgdaFunction{proj₂} \AgdaSymbol{(}\AgdaFunction{proj₁} \AgdaSymbol{(}\AgdaFunction{proj₁} \AgdaBound{x}\AgdaSymbol{)))} \<[41]%
\>[41]\<%
\\
\>[0]\AgdaIndent{13}{}\<[13]%
\>[13]\AgdaSymbol{;} \AgdaField{resp} \AgdaSymbol{=} \AgdaSymbol{λ} \AgdaBound{x} \AgdaSymbol{→} \AgdaFunction{proj₁} \AgdaSymbol{(}\AgdaFunction{proj₁} \AgdaSymbol{(}\AgdaFunction{proj₁} \AgdaSymbol{(}\AgdaFunction{proj₁} \AgdaBound{x}\AgdaSymbol{)))} \<[60]%
\>[60]\<%
\\
\>[0]\AgdaIndent{13}{}\<[13]%
\>[13]\AgdaInductiveConstructor{,} \AgdaFunction{proj₂} \AgdaSymbol{(}\AgdaFunction{proj₁} \AgdaSymbol{(}\AgdaFunction{proj₁} \AgdaBound{x}\AgdaSymbol{))} \AgdaSymbol{\}} \AgdaFunction{]T}\AgdaSymbol{)}\<%
\\
%
\\
\>[0]\AgdaIndent{4}{}\<[4]%
\>[4]\AgdaFunction{⟦subst⟧⁰} \AgdaSymbol{=} \AgdaKeyword{record}\<%
\\
\>[0]\AgdaIndent{11}{}\<[11]%
\>[11]\AgdaSymbol{\{} \AgdaField{tm} \AgdaSymbol{=} \AgdaSymbol{λ} \AgdaSymbol{\{((((}\AgdaBound{x} \AgdaInductiveConstructor{,} \AgdaBound{a}\AgdaSymbol{)} \AgdaInductiveConstructor{,} \AgdaBound{b}\AgdaSymbol{)} \AgdaInductiveConstructor{,} \AgdaBound{p}\AgdaSymbol{)} \AgdaInductiveConstructor{,} \AgdaBound{PA}\AgdaSymbol{)} \AgdaSymbol{→} \AgdaFunction{[} \AgdaBound{B} \AgdaFunction{]subst} \<[61]%
\>[61]\<%
\\
\>[11]\AgdaIndent{18}{}\<[18]%
\>[18]\AgdaSymbol{(}\AgdaFunction{[} \AgdaBound{Γ} \AgdaFunction{]refl} \AgdaInductiveConstructor{,} \AgdaFunction{[} \AgdaFunction{[} \AgdaBound{A} \AgdaFunction{]fm} \AgdaSymbol{\_} \AgdaFunction{]trans} \<[50]%
\>[50]\<%
\\
\>[11]\AgdaIndent{18}{}\<[18]%
\>[18]\AgdaSymbol{(}\AgdaFunction{[} \AgdaBound{A} \AgdaFunction{]refl*} \AgdaSymbol{\_} \AgdaSymbol{\_)} \AgdaBound{p}\AgdaSymbol{)} \AgdaBound{PA} \AgdaSymbol{\}}\<%
\\
\>[0]\AgdaIndent{11}{}\<[11]%
\>[11]\AgdaSymbol{;} \AgdaField{respt} \AgdaSymbol{=} \AgdaSymbol{λ} \AgdaSymbol{\{((((}\AgdaBound{m} \AgdaInductiveConstructor{,} \AgdaBound{a}\AgdaSymbol{)} \AgdaInductiveConstructor{,} \AgdaBound{b}\AgdaSymbol{)} \AgdaInductiveConstructor{,} \AgdaBound{p}\AgdaSymbol{)} \AgdaInductiveConstructor{,} \AgdaBound{PA}\AgdaSymbol{)} \AgdaSymbol{→} \<[53]%
\>[53]\<%
\\
\>[0]\AgdaIndent{13}{}\<[13]%
\>[13]\AgdaFunction{[} \AgdaFunction{[} \AgdaBound{B} \AgdaFunction{]fm} \AgdaSymbol{\_} \AgdaFunction{]trans} \<[32]%
\>[32]\<%
\\
\>[0]\AgdaIndent{13}{}\<[13]%
\>[13]\AgdaSymbol{(}\AgdaFunction{[} \AgdaBound{B} \AgdaFunction{]trans*} \AgdaSymbol{\_)} \<[29]%
\>[29]\<%
\\
\>[13]\AgdaIndent{14}{}\<[14]%
\>[14]\AgdaSymbol{(}\AgdaFunction{[} \AgdaFunction{[} \AgdaBound{B} \AgdaFunction{]fm} \AgdaSymbol{\_} \AgdaFunction{]trans} \<[34]%
\>[34]\<%
\\
\>[0]\AgdaIndent{13}{}\<[13]%
\>[13]\AgdaFunction{[} \AgdaBound{B} \AgdaFunction{]subst-pi} \<[27]%
\>[27]\<%
\\
\>[0]\AgdaIndent{13}{}\<[13]%
\>[13]\AgdaSymbol{(}\AgdaFunction{[} \AgdaFunction{[} \AgdaBound{B} \AgdaFunction{]fm} \AgdaSymbol{\_} \AgdaFunction{]trans} \<[33]%
\>[33]\<%
\\
\>[0]\AgdaIndent{13}{}\<[13]%
\>[13]\AgdaSymbol{(}\AgdaFunction{[} \AgdaFunction{[} \AgdaBound{B} \AgdaFunction{]fm} \AgdaSymbol{\_} \AgdaFunction{]sym} \AgdaSymbol{(}\AgdaFunction{[} \AgdaBound{B} \AgdaFunction{]trans*} \AgdaSymbol{\_))}\<%
\\
\>[0]\AgdaIndent{13}{}\<[13]%
\>[13]\AgdaSymbol{(}\AgdaFunction{[} \AgdaBound{B} \AgdaFunction{]subst*} \AgdaSymbol{\_} \AgdaBound{PA}\AgdaSymbol{)} \AgdaSymbol{))} \AgdaSymbol{\}}\<%
\\
\>[0]\AgdaIndent{11}{}\<[11]%
\>[11]\AgdaSymbol{\}}\<%
\\
\>\<\end{code}


\AgdaHide{
\begin{code}\>\<%
\\
%
\\
\>\AgdaComment{--    ⟦subst⟧ = lam (lam (lam ⟦subst⟧⁰))}\<%
\\
\>[0]\AgdaIndent{4}{}\<[4]%
\>[4]\<%
\\
\>\<\end{code}
}


\AgdaHide{
\begin{code}\>\<%
\\
%
\\
\>\AgdaSymbol{\{-\#} \AgdaKeyword{OPTIONS} --type-in-type \AgdaSymbol{\#-\}}\<%
\\
%
\\
\>\AgdaKeyword{import} \AgdaModule{Level}\<%
\\
\>\AgdaKeyword{open} \AgdaKeyword{import} \AgdaModule{Relation.Binary.PropositionalEquality} \AgdaSymbol{as} \AgdaModule{PE} \AgdaKeyword{hiding} \AgdaSymbol{(}refl \AgdaSymbol{;} sym \AgdaSymbol{;} trans\AgdaSymbol{;} isEquivalence\AgdaSymbol{;} [\_]\AgdaSymbol{)}\<%
\\
%
\\
\>\AgdaKeyword{module} \AgdaModule{CwF-quotient} \AgdaSymbol{(}\AgdaBound{ext} \AgdaSymbol{:} \AgdaFunction{Extensionality} \AgdaPrimitive{Level.zero} \AgdaPrimitive{Level.zero}\AgdaSymbol{)} \AgdaKeyword{where}\<%
\\
%
\\
\>\AgdaKeyword{open} \AgdaKeyword{import} \AgdaModule{Data.Unit}\<%
\\
\>\AgdaKeyword{open} \AgdaKeyword{import} \AgdaModule{Function}\<%
\\
\>\AgdaKeyword{open} \AgdaKeyword{import} \AgdaModule{Data.Product}\<%
\\
%
\\
%
\\
\>\AgdaComment{-- importing other CWF files}\<%
\\
%
\\
\>\AgdaKeyword{import} \AgdaModule{CwF-setoid}\<%
\\
%
\\
\>\AgdaKeyword{open} \AgdaModule{CwF-setoid} \AgdaBound{ext}\<%
\\
%
\\
\>\AgdaKeyword{import} \AgdaModule{CategoryOfSetoid}\<%
\\
\>\AgdaKeyword{module} \AgdaModule{cos'} \AgdaSymbol{=} \AgdaModule{CategoryOfSetoid} \AgdaBound{ext}\<%
\\
\>\AgdaKeyword{open} \AgdaModule{cos'}\<%
\\
%
\\
\>\AgdaKeyword{import} \AgdaModule{hProp}\<%
\\
\>\AgdaKeyword{module} \AgdaModule{hp'} \AgdaSymbol{=} \AgdaModule{hProp} \AgdaBound{ext}\<%
\\
\>\AgdaKeyword{open} \AgdaModule{hp'}\<%
\\
%
\\
\>\AgdaKeyword{import} \AgdaModule{CwF-ctd}\<%
\\
\>\AgdaKeyword{module} \AgdaModule{cc} \AgdaSymbol{=} \AgdaModule{CwF-ctd} \AgdaBound{ext}\<%
\\
\>\AgdaKeyword{open} \AgdaModule{cc}\<%
\\
%
\\
%
\\
\>\<\end{code}
}


The equality type is an essential part of a type theory. We could define it by using the equivalence relation from the setoid representation of type A. The equivalence relation is trivial since it is proof-irrelevant.

\begin{code}\>\<%
\\
%
\\
\>\AgdaFunction{Rel} \AgdaSymbol{:} \AgdaSymbol{\{}\AgdaBound{Γ} \AgdaSymbol{:} \AgdaFunction{Con}\AgdaSymbol{\}} \AgdaSymbol{→} \AgdaRecord{Ty} \AgdaBound{Γ} \AgdaSymbol{→} \AgdaPrimitiveType{Set₁}\<%
\\
\>\AgdaFunction{Rel} \AgdaSymbol{\{}\AgdaBound{Γ}\AgdaSymbol{\}} \AgdaBound{A} \AgdaSymbol{=} \AgdaRecord{Ty} \AgdaSymbol{(}\AgdaBound{Γ} \AgdaFunction{\&} \AgdaBound{A} \AgdaFunction{\&} \AgdaBound{A} \AgdaFunction{+T} \AgdaBound{A}\AgdaSymbol{)}\<%
\\
%
\\
\>\AgdaFunction{⟦Id⟧} \AgdaSymbol{:} \AgdaSymbol{\{}\AgdaBound{Γ} \AgdaSymbol{:} \AgdaFunction{Con}\AgdaSymbol{\}(}\AgdaBound{A} \AgdaSymbol{:} \AgdaRecord{Ty} \AgdaBound{Γ}\AgdaSymbol{)} \AgdaSymbol{→} \AgdaFunction{Rel} \AgdaBound{A}\<%
\\
\>\AgdaFunction{⟦Id⟧} \AgdaBound{A}\<%
\\
\>[0]\AgdaIndent{3}{}\<[3]%
\>[3]\AgdaSymbol{=} \AgdaKeyword{record} \<[12]%
\>[12]\<%
\\
\>[3]\AgdaIndent{7}{}\<[7]%
\>[7]\AgdaSymbol{\{} \AgdaField{fm} \AgdaSymbol{=} \AgdaSymbol{λ} \AgdaSymbol{\{((}\AgdaBound{x} \AgdaInductiveConstructor{,} \AgdaBound{a}\AgdaSymbol{)} \AgdaInductiveConstructor{,} \AgdaBound{b}\AgdaSymbol{)} \AgdaSymbol{→} \<[34]%
\>[34]\AgdaKeyword{record}\<%
\\
\>[7]\AgdaIndent{9}{}\<[9]%
\>[9]\AgdaSymbol{\{} \AgdaField{Carrier} \AgdaSymbol{=} \AgdaFunction{[} \AgdaFunction{[} \AgdaBound{A} \AgdaFunction{]fm} \AgdaBound{x} \AgdaFunction{]} \AgdaBound{a} \AgdaFunction{≈} \AgdaBound{b}\<%
\\
\>[7]\AgdaIndent{9}{}\<[9]%
\>[9]\AgdaSymbol{;} \AgdaField{\_≈h\_} \AgdaSymbol{=} \AgdaSymbol{λ} \AgdaBound{x₁} \AgdaBound{x₂} \AgdaSymbol{→} \AgdaFunction{⊤'}\<%
\\
\>[7]\AgdaIndent{9}{}\<[9]%
\>[9]\AgdaSymbol{;} \AgdaField{isEquiv} \AgdaSymbol{=} \AgdaKeyword{record}\<%
\\
\>[9]\AgdaIndent{13}{}\<[13]%
\>[13]\AgdaSymbol{\{} \AgdaField{refl} \AgdaSymbol{=} \AgdaSymbol{λ} \AgdaSymbol{\{}\AgdaBound{x₁}\AgdaSymbol{\}} \AgdaSymbol{→} \AgdaInductiveConstructor{tt}\<%
\\
\>[9]\AgdaIndent{13}{}\<[13]%
\>[13]\AgdaSymbol{;} \AgdaField{sym} \AgdaSymbol{=} \AgdaSymbol{λ} \AgdaBound{x₂} \AgdaSymbol{→} \AgdaInductiveConstructor{tt}\<%
\\
\>[9]\AgdaIndent{13}{}\<[13]%
\>[13]\AgdaSymbol{;} \AgdaField{trans} \AgdaSymbol{=} \AgdaSymbol{λ} \AgdaBound{x₂} \AgdaBound{x₃} \AgdaSymbol{→} \AgdaInductiveConstructor{tt}\<%
\\
\>[9]\AgdaIndent{13}{}\<[13]%
\>[13]\AgdaSymbol{\}}\<%
\\
\>[0]\AgdaIndent{9}{}\<[9]%
\>[9]\AgdaSymbol{\}} \AgdaSymbol{\}}\<%
\\
\>[0]\AgdaIndent{7}{}\<[7]%
\>[7]\AgdaSymbol{;} \AgdaField{substT} \AgdaSymbol{=} \AgdaSymbol{λ} \AgdaSymbol{\{((}\AgdaBound{x} \AgdaInductiveConstructor{,} \AgdaBound{a}\AgdaSymbol{)} \AgdaInductiveConstructor{,} \AgdaBound{b}\AgdaSymbol{)} \AgdaBound{x0} \AgdaSymbol{→} \<[40]%
\>[40]\<%
\\
\>[7]\AgdaIndent{15}{}\<[15]%
\>[15]\AgdaFunction{[} \AgdaFunction{[} \AgdaBound{A} \AgdaFunction{]fm} \AgdaSymbol{\_} \AgdaFunction{]trans} \<[34]%
\>[34]\<%
\\
\>[7]\AgdaIndent{15}{}\<[15]%
\>[15]\AgdaSymbol{(}\AgdaFunction{[} \AgdaFunction{[} \AgdaBound{A} \AgdaFunction{]fm} \AgdaSymbol{\_} \AgdaFunction{]sym} \AgdaBound{a}\AgdaSymbol{)} \<[36]%
\>[36]\<%
\\
\>[7]\AgdaIndent{15}{}\<[15]%
\>[15]\AgdaSymbol{(}\AgdaFunction{[} \AgdaFunction{[} \AgdaBound{A} \AgdaFunction{]fm} \AgdaSymbol{\_} \AgdaFunction{]trans} \<[35]%
\>[35]\<%
\\
\>[7]\AgdaIndent{15}{}\<[15]%
\>[15]\AgdaSymbol{(}\AgdaFunction{[} \AgdaBound{A} \AgdaFunction{]subst*} \AgdaSymbol{\_} \AgdaBound{x0}\AgdaSymbol{)} \AgdaBound{b}\AgdaSymbol{)} \<[37]%
\>[37]\<%
\\
\>[7]\AgdaIndent{15}{}\<[15]%
\>[15]\AgdaSymbol{\}}\<%
\\
\>[0]\AgdaIndent{7}{}\<[7]%
\>[7]\AgdaSymbol{;} \AgdaField{subst*} \AgdaSymbol{=} \AgdaSymbol{λ} \AgdaBound{p} \AgdaBound{x₁} \AgdaSymbol{→} \AgdaInductiveConstructor{tt}\<%
\\
\>[0]\AgdaIndent{7}{}\<[7]%
\>[7]\AgdaSymbol{;} \AgdaField{refl*} \AgdaSymbol{=} \AgdaSymbol{λ} \AgdaBound{x} \AgdaBound{a} \AgdaSymbol{→} \AgdaInductiveConstructor{tt}\<%
\\
\>[0]\AgdaIndent{7}{}\<[7]%
\>[7]\AgdaSymbol{;} \AgdaField{trans*} \AgdaSymbol{=} \AgdaSymbol{λ} \AgdaBound{p} \AgdaBound{q} \AgdaBound{a} \AgdaSymbol{→} \AgdaInductiveConstructor{tt} \AgdaSymbol{\}}\<%
\\
%
\\
\>\<\end{code}

The unique inhabitant $refl$ is defined as

\begin{code}\>\<%
\\
%
\\
%
\\
\>\AgdaFunction{cm-refl} \AgdaSymbol{:} \AgdaSymbol{\{}\AgdaBound{Γ} \AgdaSymbol{:} \AgdaFunction{Con}\AgdaSymbol{\}(}\AgdaBound{A} \AgdaSymbol{:} \AgdaRecord{Ty} \AgdaBound{Γ}\AgdaSymbol{)} \AgdaSymbol{→} \AgdaBound{Γ} \AgdaFunction{\&} \AgdaBound{A} \AgdaRecord{⇉} \AgdaSymbol{(}\AgdaBound{Γ} \AgdaFunction{\&} \AgdaBound{A} \AgdaFunction{\&} \AgdaBound{A} \AgdaFunction{+T} \AgdaBound{A}\AgdaSymbol{)}\<%
\\
\>\AgdaFunction{cm-refl} \AgdaBound{A} \AgdaSymbol{=} \AgdaKeyword{record} \AgdaSymbol{\{} \AgdaField{fn} \AgdaSymbol{=} \AgdaSymbol{λ} \AgdaBound{x'} \AgdaSymbol{→} \AgdaBound{x'} \AgdaInductiveConstructor{,} \AgdaFunction{proj₂} \AgdaBound{x'} \<[47]%
\>[47]\<%
\\
\>[7]\AgdaIndent{19}{}\<[19]%
\>[19]\AgdaSymbol{;} \AgdaField{resp} \AgdaSymbol{=} \AgdaSymbol{λ} \AgdaBound{x'} \AgdaSymbol{→} \AgdaBound{x'} \AgdaInductiveConstructor{,} \AgdaFunction{proj₂} \AgdaBound{x'} \AgdaSymbol{\}}\<%
\\
%
\\
\>\AgdaFunction{⟦refl⟧⁰} \AgdaSymbol{:} \AgdaSymbol{\{}\AgdaBound{Γ} \AgdaSymbol{:} \AgdaFunction{Con}\AgdaSymbol{\}(}\AgdaBound{A} \AgdaSymbol{:} \AgdaRecord{Ty} \AgdaBound{Γ}\AgdaSymbol{)} \<[30]%
\>[30]\<%
\\
\>[0]\AgdaIndent{7}{}\<[7]%
\>[7]\AgdaSymbol{→} \AgdaRecord{Tm} \AgdaSymbol{\{}\AgdaBound{Γ} \AgdaFunction{\&} \AgdaBound{A}\AgdaSymbol{\}} \AgdaSymbol{(}\AgdaFunction{⟦Id⟧} \AgdaBound{A}\<%
\\
\>[0]\AgdaIndent{10}{}\<[10]%
\>[10]\AgdaFunction{[} \AgdaFunction{cm-refl} \AgdaBound{A} \AgdaFunction{]T}\AgdaSymbol{)} \<[26]%
\>[26]\<%
\\
\>\AgdaFunction{⟦refl⟧⁰} \AgdaBound{A} \AgdaSymbol{=} \AgdaKeyword{record}\<%
\\
\>[10]\AgdaIndent{11}{}\<[11]%
\>[11]\AgdaSymbol{\{} \AgdaField{tm} \AgdaSymbol{=} \AgdaSymbol{λ} \AgdaSymbol{\{(}\AgdaBound{x} \AgdaInductiveConstructor{,} \AgdaBound{a}\AgdaSymbol{)} \AgdaSymbol{→} \AgdaFunction{[} \AgdaFunction{[} \AgdaBound{A} \AgdaFunction{]fm} \AgdaBound{x} \AgdaFunction{]refl} \AgdaSymbol{\{}\AgdaBound{a}\AgdaSymbol{\}} \AgdaSymbol{\}}\<%
\\
\>[10]\AgdaIndent{11}{}\<[11]%
\>[11]\AgdaSymbol{;} \AgdaField{respt} \AgdaSymbol{=} \AgdaSymbol{λ} \AgdaBound{p} \AgdaSymbol{→} \AgdaInductiveConstructor{tt}\<%
\\
\>[10]\AgdaIndent{11}{}\<[11]%
\>[11]\AgdaSymbol{\}}\<%
\\
%
\\
\>\AgdaFunction{⟦refl⟧} \AgdaSymbol{:} \AgdaSymbol{\{}\AgdaBound{Γ} \AgdaSymbol{:} \AgdaFunction{Con}\AgdaSymbol{\}(}\AgdaBound{A} \AgdaSymbol{:} \AgdaRecord{Ty} \AgdaBound{Γ}\AgdaSymbol{)} \<[29]%
\>[29]\<%
\\
\>[-6]\AgdaIndent{7}{}\<[7]%
\>[7]\AgdaSymbol{→} \AgdaRecord{Tm} \AgdaSymbol{\{}\AgdaBound{Γ}\AgdaSymbol{\}} \AgdaSymbol{(}\AgdaFunction{Π} \AgdaBound{A} \AgdaSymbol{(}\AgdaFunction{⟦Id⟧} \AgdaBound{A} \<[29]%
\>[29]\<%
\\
\>[0]\AgdaIndent{10}{}\<[10]%
\>[10]\AgdaFunction{[} \AgdaFunction{cm-refl} \AgdaBound{A} \AgdaFunction{]T}\AgdaSymbol{)} \AgdaSymbol{)}\<%
\\
\>\AgdaFunction{⟦refl⟧} \AgdaSymbol{\{}\AgdaBound{Γ}\AgdaSymbol{\}} \AgdaBound{A} \AgdaSymbol{=} \<[16]%
\>[16]\AgdaFunction{lam} \AgdaSymbol{\{}\AgdaBound{Γ}\AgdaSymbol{\}} \AgdaSymbol{\{}\AgdaBound{A}\AgdaSymbol{\}} \AgdaSymbol{(}\AgdaFunction{⟦refl⟧⁰} \AgdaBound{A}\AgdaSymbol{)}\<%
\\
%
\\
%
\\
\>\<\end{code}

We have an abstracted $refl$ term as well. Using $\Pi$-types we could define the eliminator for $Id$, but it is more involved.

We have done the basics for category of families of setoids. There are more types can be interpreted in this model so that we could show that it is a valid model for Type Theory. We would like to interpret quotient types in this model by following Hofmann's method in \cite{hof:95:sm} or by ourselves.


\AgdaHide{
\begin{code}\>\<%
\\
\>\AgdaComment{\{-

-- substIn (B : Ty (Γ \& A))

⟦subst⟧⁰ : \{Γ : Con\}(A : Ty Γ)(B : Ty (Γ \& A)) → 
           Tm \{Γ \& A \& (A [ fst\& A ]T) 
           \& (⟦Id⟧ A) \& B [ fst\& (A [ fst\& A ]T) ]T [ fst\& (⟦Id⟧ A) ]T\} 
         (B [ record \{ fn = λ x → (proj₁ (proj₁ (proj₁ (proj₁ x)))) , (proj₂ (proj₁ (proj₁ x))) ; resp = λ x → proj₁ (proj₁ (proj₁ (proj₁ x))) , proj₂ (proj₁ (proj₁ x)) \} ]T)

⟦subst⟧⁰ \{Γ\} A B = record
       \{ tm = λ \{((((x , a) , b) , p) , PA) → [ B ]subst ([ Γ ]refl , [ [ A ]fm \_ ]trans ([ A ]refl* \_ \_) p) PA \}
       ; respt = λ \{((((m , a) , b) , p) , PA) → 
         [ [ B ]fm \_ ]trans 
         ([ B ]trans* \_ \_ \_) 
          ([ [ B ]fm \_ ]trans 
         [ B ]subst-pi 
         ([ [ B ]fm \_ ]trans 
         ([ [ B ]fm \_ ]sym ([ B ]trans* \_ \_ \_))
         ([ B ]subst* \_ PA) )) \}
       \}


-\}}\<%
\\
%
\\
\>\AgdaComment{-- The mechanism used in Martin Hofmann's Paper}\<%
\\
%
\\
\>\AgdaKeyword{record} \AgdaRecord{Prop-Uni} \AgdaSymbol{(}\AgdaBound{Γ} \AgdaSymbol{:} \AgdaFunction{Con}\AgdaSymbol{)} \AgdaSymbol{:} \AgdaPrimitiveType{Set} \AgdaKeyword{where}\<%
\\
\>[0]\AgdaIndent{2}{}\<[2]%
\>[2]\AgdaKeyword{field}\<%
\\
%
\\
\>[0]\AgdaIndent{4}{}\<[4]%
\>[4]\AgdaField{prf} \AgdaSymbol{:} \AgdaRecord{Ty} \AgdaBound{Γ}\<%
\\
\>[0]\AgdaIndent{4}{}\<[4]%
\>[4]\AgdaField{uni} \AgdaSymbol{:} \AgdaSymbol{∀} \AgdaBound{γ} \AgdaBound{x} \AgdaBound{y} \AgdaSymbol{→} \AgdaFunction{[} \AgdaFunction{[} \AgdaBound{prf} \AgdaFunction{]fm} \AgdaBound{γ} \AgdaFunction{]} \AgdaBound{x} \AgdaFunction{≈h} \AgdaBound{y} \AgdaDatatype{≡} \AgdaFunction{⊤'}\<%
\\
\>\AgdaKeyword{open} \AgdaModule{Prop-Uni}\<%
\\
%
\\
\>\AgdaComment{-- Is it correct to write  Tm A → Tm B for dependent types?}\<%
\\
%
\\
%
\\
%
\\
\>\AgdaFunction{Id-is-prop} \AgdaSymbol{:} \AgdaSymbol{\{}\AgdaBound{Γ} \AgdaSymbol{:} \AgdaFunction{Con}\AgdaSymbol{\}(}\AgdaBound{A} \AgdaSymbol{:} \AgdaRecord{Ty} \AgdaBound{Γ}\AgdaSymbol{)} \AgdaSymbol{→} \AgdaRecord{Prop-Uni} \AgdaSymbol{(}\AgdaBound{Γ} \AgdaFunction{\&} \AgdaBound{A} \AgdaFunction{\&} \AgdaSymbol{(}\AgdaBound{A} \AgdaFunction{[} \AgdaFunction{fst\&} \AgdaBound{A} \AgdaFunction{]T}\AgdaSymbol{))}\<%
\\
\>\AgdaFunction{Id-is-prop} \AgdaBound{A} \AgdaSymbol{=} \AgdaKeyword{record} \AgdaSymbol{\{} \AgdaField{prf} \AgdaSymbol{=} \AgdaFunction{⟦Id⟧} \AgdaBound{A} \AgdaSymbol{;} \AgdaField{uni} \AgdaSymbol{=} \AgdaSymbol{λ} \AgdaBound{γ} \AgdaBound{x} \AgdaBound{y} \AgdaSymbol{→} \AgdaInductiveConstructor{PE.refl} \AgdaSymbol{\}}\<%
\\
%
\\
\>\AgdaComment{\{-
record Quo \{Γ : Con\}(A : Ty Γ)(R : Prop-Uni (Γ \& A \& (A [ fst\& \{Γ\} \{A\} ]T))) : Set where
  field
    Q : Ty Γ
    [\_] : Tm A → Tm Q
    Q-elim : ∀ (B : Ty Γ)
                 (M : Tm \{Γ \& A\} (B [ fst\& \{Γ\} \{A\} ]T))
                 (N : Tm Q)
                 (H : Tm \{Γ \& A \& A [ fst\& \{Γ\} \{A\} ]T \& prf R\} (prf (Id-is-prop B) [ fst\& \{Γ \& A \& A [ fst\& \{Γ\} \{A\} ]T\} \{prf R\} ]T)) -- (prf (Id-is-prop (B [ fst\& \{Γ\} \{A\} ]T)))
               → Tm B

-\}}\<%
\\
%
\\
%
\\
%
\\
\>\<\end{code}
}





\section{Quotient types in setoid model}





\section{Observational equality}

%definitional distinct types

Later in in \cite{alti:ott-conf}, Altenkirch and McBride further
simplifies the setoid model by adopting McBride's heterogeneous
approach to equality. They identifies values up to observation rather than
  construction which is called observational equality. It is the
  propositional equality induced by the Setoid model.  In general we have a heterogeneous equality which
  compares terms of types which are different in construction. It only
  make sense when we can prove the types are the same. It helps us
  avoids the heavy use of $subst$ which makes formalisation and
  reasoning involved. We could simplify the setoid model by adapting this
  approach and the implementation could be easier.


%\chapter{Homotopy Type Theory and higher inductive types}\label{HITs}


\chapter{Models of Type Theory}\label{models}


To introduce extensional concepts in \itt, one can simply postulate them as axioms but the good computational properties of Type Theory will be lost. Therefore it is crucial that these axioms have 
a computational interpretation. One solution is to construct a model of theses axioms in \itt or in a constructive setting such that extensional concepts like functional extensionality, quotient types, univalence are automatically derivable. Types as usually interpreted as structured objects rather than sets, and the model is named after the structured objects, for example, setoid, groupoid, simplicial sets. 


In this chapter, we discuss several models of extensional concepts, and mainly introduce Altenkirch's setoid model. We build a category with families of setoids to accommodate the types theory described in
\cite{alti:lics99}  so that it is possible to define quotient types following Martin Hofmann's Paper \cite{hof:95:sm}. We also briefly introduce several models of \hott where quotient types are also available.

\section{Setoid model}


One of the intensional models of extensional concepts inside \itt is setoid model where types are interpreted as setoids. Martin Hofmann has studied this approach in \cite{hof:phd}, but a naive version of setoid model does not satisfy all definitional equalities. The version in \cite{DBLP:conf/tlca/Hofmann95} which is a model for quotient types does not allow large eliminations (defining a dependent type by recursion).

Altenkirch then proposes \cite{alti:lics99} a different approach in which the setoid model serves as the metatheory.
He uses an extension of \itt by a universe of propositions $\Prop$ as metatheory, and the $\eta$-rules for $\Pi$-types and $\Sigma$-types hold. 

%In \cite{DBLP:conf/csl/Hofmann94}, Martin Hofmann first proposes a setoid model where types are interpreted as setoids, and the function are equivalence preserving. Cateogically, a setoid is a special groupoid where every isomorphism is unique, and then functions between two setoids are just functors.


\infrule[proof-irr]{\Gamma \vdash P : \Prop \andalso \Gamma \vdash p,q : P}{\Gamma \vdash p \equiv q : P}



 $\Prop$ only contains "propositional'' sets which has at most one
inhabitant. Notice that it is not a definition of types, which means
that we cannot conclude a type is of type \textbf{Prop} if we have a
proof that all inhabitants of it are definitionally equal.

The propositional universe is closed under $\Pi$-types and $\Sigma$-types:



\infrule[$\Pi$-Prop]{\Gamma \vdash A : \Set \andalso \Gamma, x : A \vdash P : \Prop}
{\Gamma \vdash \Pi~ (x : A) \to P : \Prop}



\infrule[$\Sigma$-Prop]{\Gamma \vdash P : \Prop \andalso \Gamma,x : P \vdash Q : \Prop}
{\Gamma \vdash \Sigma ~(x : P) ~ Q : \Prop}


The metatheory is proved:

\begin{itemize}
\item Decidable. The definitional equality is decideable, hence type checking is decidable.

\item Consistent. Not all types inhabited and not all well typed definitionality equality holds. 

\item Adequate. All terms of type $\N$ are reducible to numerals.
\end{itemize}


And then Altenkirch constrct an intensional model within this metatheory which is decidable and adequate, functional extensionality is inhabited and it permits large elimination. 


% Within this type theory, introduction of quotient types is straightforward. 
%The set of functions are naturally quotient types, the hidden information is the definition of the functions and the equivalence relation is the functional extensionality.



\begin{remark}[The setoid model is not LCCC]
This model is different to a setoid model as an E-category, for instance
the one introduced by Hofmann \cite{hofmann1995interpretation}. An E-category is a category equipped with
an equivalence relation for homsets. To distinguish them, we call this
category \textbf{E-setoids}.  All morphisms of \textbf{E-setoids}
gives rise to types and they are cartesian closed, namely it is a a locally
cartesian closed category (LCCC). However not all morphisms in our category of setoids give rise to types which means that it is not an LCCC. Every LCCC can serve as a model for categories with
families but not every category with families has to be a LCCC. 

This setoid model is the category of setoids \textbf{Std} which is a full subcategory of \textbf{Gpd} (the category of small groupoids). Every object of \textbf{Gpd} whose all homsets contain at most one morphism are in this subcategory. Altenkirch and Klaus \cite{Altenkirch12setoidsare} prove that both \textbf{Gpd} and \textbf{Std} are cartesian closed but not locally cartesian closed.
\end{remark}


\section{Category with families}


The setoid model is defined as categories with families as introduced by Dybjer \cite{Dyb:96} and Hofmann
\cite{hof:97}. The object theory is decidable because its definitional equalities are interpreted by definitional equality in the metatheory which is decidable.

We implement categories with families in Agda, and the detailed code can be found in Appendix \ref{cwf}.

We first define \textbf{hProp} which serves as the proof-irrelevant universe of propositions although it is not exactly the same as the $Prop$ universe which is judgemental. Any set behaves like a $Proposition$ belongs to $HProp$ but not $Prop$.


\begin{code}\>\<%
\\
\>\AgdaKeyword{record} \AgdaRecord{HProp} \AgdaSymbol{:} \AgdaPrimitiveType{Set₁} \AgdaKeyword{where}\<%
\\
\>[0]\AgdaIndent{2}{}\<[2]%
\\
\>[0]\AgdaIndent{2}{}\<[2]%
\>[2]\AgdaKeyword{field}\<%
\\
\>[2]\AgdaIndent{4}{}\<[4]%
\>[4]\AgdaField{prf} \AgdaSymbol{:} \AgdaPrimitiveType{Set}\<%
\\
\>[2]\AgdaIndent{4}{}\<[4]%
\>[4]\AgdaField{Uni} \AgdaSymbol{:} \AgdaSymbol{\{}\AgdaBound{p} \AgdaBound{q} \AgdaSymbol{:} \AgdaBound{prf}\AgdaSymbol{\}} \AgdaSymbol{→} \AgdaBound{p} \AgdaDatatype{≡} \AgdaBound{q}\<
%
\>\<\end{code}

We also define basic propositions $\top$ and $\bot$ and the universal and existential quantifier, namely it is closed under $\Pi$-types and $\Sigma$-types. Notice that, the $\Pi$-closure requires functional extensionality which can be postulated since $\Pi$-closure is itself an axiom in this model.

valent to the closure under $\Pi$-types.

\begin{code}\>\<
%
\\
\>\AgdaFunction{∀'} \AgdaSymbol{:} \AgdaSymbol{(}\AgdaBound{A} \AgdaSymbol{:} \AgdaPrimitiveType{Set}\AgdaSymbol{)(}\AgdaBound{P} \AgdaSymbol{:} \AgdaBound{A} \AgdaSymbol{→} \AgdaRecord{hProp}\AgdaSymbol{)} \AgdaSymbol{→} \AgdaRecord{hProp}\<%
\\
\>\AgdaFunction{∀'} \AgdaBound{A} \AgdaBound{P} \AgdaSymbol{=} \AgdaInductiveConstructor{hp} \AgdaSymbol{((}\AgdaBound{x} \AgdaSymbol{:} \AgdaBound{A}\AgdaSymbol{)} \AgdaSymbol{→} \AgdaFunction{<} \AgdaBound{P} \AgdaBound{x} \AgdaFunction{>}\AgdaSymbol{)} \AgdaSymbol{(}\AgdaBound{ext} \AgdaSymbol{(λ} \AgdaBound{x} \AgdaSymbol{→} \AgdaFunction{Uni} \AgdaSymbol{(}\AgdaBound{P} \AgdaBound{x}\AgdaSymbol{)))}\<%
\\
\>\<\end{code}

Then setoids can be defined as follows considering \textbf{HProp}.

\begin{code}\>\<%
\\
\>\AgdaKeyword{record} \AgdaRecord{hSetoid} \AgdaSymbol{:} \AgdaPrimitiveType{Set₁} \AgdaKeyword{where}\<%
\\
\>[0]\AgdaIndent{2}{}\<[2]%
\>[2]\AgdaKeyword{constructor} \AgdaInductiveConstructor{\_,\_,\_}\<%
\\
\>[0]\AgdaIndent{2}{}\<[2]%
\>[2]\AgdaKeyword{infix} \AgdaNumber{4} \_≈h\_ \_≈\_\<%
\\
\>[0]\AgdaIndent{2}{}\<[2]%
\>[2]\AgdaKeyword{field}\<%
\\
\>[2]\AgdaIndent{4}{}\<[4]%
\>[4]\AgdaField{Carrier} \AgdaSymbol{:} \AgdaPrimitiveType{Set}\<%
\\
\>[2]\AgdaIndent{4}{}\<[4]%
\>[4]\AgdaField{\_≈h\_} \<[12]%
\>[12]\AgdaSymbol{:} \AgdaBound{Carrier} \AgdaSymbol{→} \AgdaBound{Carrier} \AgdaSymbol{→} \AgdaRecord{hProp}\<%
\\
\>[2]\AgdaIndent{4}{}\<[4]%
\>[4]\AgdaField{isEquiv} \AgdaSymbol{:} \AgdaRecord{ishEquivalence} \AgdaBound{\_≈h\_}\<%
\\
%
\\
%
\\
\>\<\end{code}


As long as we define all ingredients, we can build the category of setoids of $\textbf{Std}$. However we still follow the Altenkirch's approach to define types and terms separately instead of a categorical presheaf construction.

The main thing we have done is to define more types within this setoid model, including $\Pi$-types and $\Sigma$-types, equality types, universe, natural numbers and most importantly the quotient types.






\section{Quotient types in setoid model}





\section{Observational equality}

%definitional distinct types

Later in in \cite{alti:ott-conf}, Altenkirch and McBride further
simplifies the setoid model by adopting McBride's heterogeneous
approach to equality. They identifies values up to observation rather than
  construction which is called observational equality. It is the
  propositional equality induced by the Setoid model.  In general we have a heterogeneous equality which
  compares terms of types which are different in construction. It only
  make sense when we can prove the types are the same. It helps us
  avoids the heavy use of $subst$ which makes formalisation and
  reasoning involved. We could simplify the setoid model by adapting this
  approach and the implementation could be easier.











\section{Groupoid model}




\section{Models of \hott}




\cite{bezem2013model}

\hott can be seen as Type Theory together with several axioms, like the univalence axiom and higher inductive types. We can postulate these axioms but then we can not keep the good computational properties. Therefore it is crucial that axioms have a computational interpretation. One solution is 









There are some models of \hott currently being studied extensively.
To interpret types as weak $\omega$-groupoids, one main problems is
the complexity of the definition of weak $\omega$-groupoid. The
coherence conditions are very difficult to specified.
It is much simpler to interpret types as Kan simplicial sets.
Voevodsky's univalent model\cite{klv:ssetmodel} is based on Kan simplicial sets. 
 Streicher wrote a concise introduction to this model \cite{DBLP:dblp_journals/japll/Streicher14}. 



\iffalse % comment out multiple line

\begin{definition}
A simplicial set $X$ is a functor from $\Delta^{op}$ to $\Set$ where
$\Delta$ is the simplex category.
\end{definition}

$\Delta^{op}$ is a category whose objects are non-empty totally ordered
finite sets. The morphisms are order-preserving functions. 
Face maps and degeneracy maps are the most important morphisms in this
category.

A simplexes is a generalisation of a triangle to arbitrary
dimensions. $3$-dimensional simplex is tetrahedron and $k+1$-simplex can
be obtained by adding one point to $k$-simplex which does not lie in the
dimension where the $k$-simplex is.

A simplicial complex is a collection of simplexes. Topologically speaking, it
is constructed by gluing n-dimensional simplexes together. 
\todo{show an example graph}

A simplicial set, therefore, can be illustrate by the same graph where
the set of points is given by $X_0$, the set of lines is $X_1$ and so
on. The graph looks very similar to n-groupoids. In fact simplical
sets can interpret types in a similar way (?).

\fi




However the simplicial set model is not constructive as Coquand showed
that it requires classical logic in an essential way \cite{TC:sset}.


The distinction of semi-simplicial set is there is only face maps
but no degeneracy maps. We can denote a semi-simplicial set as a
functor $X : \Delta_{inj} \rightarrow |Set$. The morphisms in $\Delta_{inj}$ are not only order preserving
but also injective.
A “iterated dependency” approach is believed to solve the coherence
issues.


\subsection{semi-simplicial sets}

\todo{cite https://uf-ias-2012.wikispaces.com/Semi-simplicial+types}


%Klaus and me were trying to implement semi-simplical set in Agda. 


\subsection{Cubical sets}

Someone can deduce what is a cubical set from the name literaly. It is
also a functor $S : \Box^{op} \rightarrow \Set$ (or a presheaf on the
cube category $ \Box^{op}$).

\section{Summary}

In this chapter we introduce the basic notions in \hott, discuss the
various implementations of \hott. In the next chapter we will focus on
thesyntactic implementation of \wog following
Brunerie's approach. We attempt to formalise the groupoid model of
\hott in intensional type theory, specially in Agda.




\section{Summary}




% In \itt, the uniqueness of identity types is not
% accepted in general, but derivable for types whose propositional
% equality is decidable. The homotopy interpretation fits
% nicely by provides higher levels structures which are weaker
% equivalence relation (compared to strict equality) between identity types.


However the implementation of \hott in \itt is still an open problem. We
work on defining semi-simplicial types and \wog to solve this problem. There
is also the very new model using cubical sets proposed by Bezem,
Coquand and Huber in \cite{bezem2013model}.



\hott does not only help us model type theory with a focus on the equality, but also provides mathematicians type theoretical tools to study homotopy theory.



\chapter{\hott and \og Model}

\todo{Before 15th-Jan-2013}

\section{What is \hott?}

\section{Quotient types in \hott}


\section{Syntax of \wog}


\section{Semantics}

%\chapter{Conclusion and Future Work}

We have presented the evolution of theories of types especially \mltt (Type Theory) and discussed different variants of it.  We have compared two versions of Type Theory: \ett (ETT) and \itt (ITT). ITT has decidable type checking but lacks some extensional concepts such as functional extensionality and quotient types. On the other hand,  ETT has equality reflection which gives us these extensional concepts but its type checking is undecidable due to the identification of propositional equality and definitional equality.


The notion of quotient types is one of the important extensional concepts which can facilitate mathematical and programming constructions. Because of the good computational properties such as decidable type checking, ITT is usually preferable to be implemented as a programming language compared to ETT and we would like to have quotient types in it. We have presented a definition of quotient types in a type theory with a proof-irrelevant universe, and shown that simply adding it into \itt as an axiom results in the loss of the $\N$-canonicity property. We have also made clear its correspondence to coequalizers in $\Set$ and a left adjoint functor in category theory.


We have discussed the definability of a normalisation function for a given quotient represented as a setoid. For quotients which can be defined inductively with a normalisation function e.g.\ the set of integers and the set of rational numbers, we have proposed an algebraic structure to bridge the setoid representations and set definitions.
We have shown that the application of definable quotient structure can improve the constructions by keeping good properties of both representations by providing some applications. Because it can be seen as a simulation of quotient types, we can also expect similar benefits from the applications of quotient types.


For the definable quotient structure, a potential future project would be to complete the implementation of numbers in Agda with the help of definable quotients. There are also other definable quotients implementable in our algebraic quotient structures. It can enrich the library of Agda for more potential mathematical proofs. We can also extend Agda with normalised types \cite{cou:01}, namely to build a special case of quotient types with respect to a normalisation function in the sense of \Cref{def:nor}. 

Although a quotient type former is unnecessary for definable quotients, it seems indispensable for some other quotients whose normalisation functions are not definable. With the assumption that Brouwer's continuity holds in meta-theory, we have shown a proof that there is no definable normalisation function for Cauchy reals $\qset{\R_{0}}$. There are also other examples like the partiality monad, finite multisets. 
In the future, we would like to investigate the definability of quotients in general, and in particular, we would like to find out whether the non-existence of a normalisation function for a quotient implies that it is not definable as a set in general.


The solution to introduce quotient types in \itt without losing good computational properties is to build models where types are interpreted as sets with equality internally defined, such as setoids, groupoids or \wog. We have developed an implementation of Altenkirch's setoid model in Agda, and explained 
 our construction of quotient types inside of it.


For the setoid model, there are more details to work out. For example the verification of properties, to define a type for propositions such that we can write the type of equivalence relations using $\Pi$-types. We can also simplify the setoid model by using heterogeneous equality as we discussed in \Cref{models}. We have also considered to use h-propositions in place of the universe of propositions in the metatheory. However it requires functional extensionality to prove the $\Pi$-closure of h-propositions. It would still be interesting to compare this approach with the one we have presented. It is also worthwhile to extend the setoid model with examples of quotients like the set of real numbers and finite multisets which are not definable via normalisation. 
Other extensional concepts and coinductive types can also be considered in the setoid model.


We have also investigated the new extension of \mltt --  \hott. In \hott, types are interpreted as \wog which is a generalization of a groupoid. We have discussed quotients in \hott, With univalence, quotients can be defined impredicatively. We can also define quotients using higher inductive types (HITs), and in fact HITs can be seen as "generalized quotient types".
Therefore a computation interpretation of \hott can also be seen as a solution to quotient types in \itt. 

We have shown a syntactic construction of \wog in Agda as a first step to build a weak $\omega$-groupoid model of Type Theory. We have defined a type theory \tig to describe the coherence conditions for a globular set to become a \wogs. We have shown some constructions of the coherences laws, for example groupoid laws inside this theory by using suspensions and replacement techniques. We also use heterogeneous equality for terms to overcome technical issues in implementation.

There is still a lot of work to do in the syntactic framework.
For instance, we would like to investigate the relation between the \tig and a type theory with equality types and $\J$ eliminator which is called $\mathcal{T}_{eq}$. One direction is to simulate the $\J$ eliminator syntactically in \tig as we mentioned before, the other direction is to derive $\J$ using $\mathsf{coh}$ if we can prove that the $\mathcal{T}_{eq}$ is a weak $\omega$-groupoid. 
The syntax could be simplified by adopting categories with families. An alternative may be to use higher inductive types directly to formalize the syntax of type theory. 

We would like to formalise a proof of that $\AgdaFunction{Idω}$ is a weak $\omega$-groupoid, but the base set in a globular set is an h-set which is incompatible with $\AgdaFunction{Idω}$. Perhaps as Altenkirch suggests \cite{CoherenceProblem}, we can solve the problem by using a universe with extensional equality, and Agda's propositional equality as strict equality so that we can define $\AgdaFunction{Idω}$ as a globular set in this universe.
Finally to model Type Theory with weak $\omega$-groupoids and to eliminate the univalence axiom would be the most challenging task to do in the future.

It would also be interesting to consider quotient \emph{types} in \hott. 
The notion of quotient types we considered in this thesis refers to the quotients with a \emph{propositional} equivalence relation. However in a type theory with higher dimensions, like \hott, the notion of quotient types can be more general and we would like to consider non-propositional quotients, for example, the quotient of a set by a groupoid.



%----------------------------------------------------------------------------------------
%	THESIS CONTENT - APPENDICES
%----------------------------------------------------------------------------------------

\addtocontents{toc}{\vspace{2em}} % Add a gap in the Contents, for aesthetics

\appendix % Cue to tell LaTeX that the following 'chapters' are Appendices

% Include the appendices of the thesis as separate files from the Appendices folder
% Uncomment the lines as you write the Appendices

\chapter{Definable quotient structures}\label{app:dq}


\AgdaHide{
\begin{code}\>\<%
\\
%
\\
\>\AgdaKeyword{module} \AgdaModule{Setoids} \AgdaKeyword{where}\<%
\\
%
\\
\>\AgdaKeyword{open} \AgdaKeyword{import} \AgdaModule{Relation.Binary.Core} \AgdaSymbol{as} \AgdaModule{Core} \AgdaKeyword{using} \AgdaSymbol{(}\_≡\_\AgdaSymbol{)}\<%
\\
%
\\
\>\AgdaKeyword{open} \AgdaKeyword{import} \AgdaModule{Data.Product}\<%
\\
%
\\
\>\AgdaKeyword{open} \AgdaModule{Core} \AgdaKeyword{public} \AgdaKeyword{hiding} \AgdaSymbol{(}\_≡\_\AgdaSymbol{;} refl\AgdaSymbol{;} \_≢\_\AgdaSymbol{)}\<%
\\
%
\\
%
\\
\>\AgdaFunction{\_⇔\_} \AgdaSymbol{:} \AgdaSymbol{(}\AgdaBound{A} \AgdaBound{B} \AgdaSymbol{:} \AgdaPrimitiveType{Set}\AgdaSymbol{)} \AgdaSymbol{→} \AgdaPrimitiveType{Set}\<%
\\
\>\AgdaBound{A} \AgdaFunction{⇔} \AgdaBound{B} \AgdaSymbol{=} \AgdaSymbol{(}\AgdaBound{A} \AgdaSymbol{→} \AgdaBound{B}\AgdaSymbol{)} \AgdaFunction{×} \AgdaSymbol{(}\AgdaBound{B} \AgdaSymbol{→} \AgdaBound{A}\AgdaSymbol{)}\<%
\\
%
\\
\>\<\end{code}
}

\begin{code}\>\<%
\\
\>\AgdaKeyword{record} \AgdaRecord{Setoid} \AgdaSymbol{:} \AgdaPrimitiveType{Set₁} \AgdaKeyword{where}\<%
\\
\>[0]\AgdaIndent{2}{}\<[2]%
\>[2]\AgdaKeyword{infix} \AgdaNumber{4} \_\textasciitilde\_\<%
\\
\>[0]\AgdaIndent{2}{}\<[2]%
\>[2]\AgdaKeyword{field}\<%
\\
\>[2]\AgdaIndent{4}{}\<[4]%
\>[4]\AgdaField{Carrier} \<[18]%
\>[18]\AgdaSymbol{:} \AgdaPrimitiveType{Set}\<%
\\
\>[2]\AgdaIndent{4}{}\<[4]%
\>[4]\AgdaField{\_\textasciitilde\_} \<[18]%
\>[18]\AgdaSymbol{:} \AgdaBound{Carrier} \AgdaSymbol{→} \AgdaBound{Carrier} \AgdaSymbol{→} \AgdaPrimitiveType{Set}\<%
\\
\>[2]\AgdaIndent{4}{}\<[4]%
\>[4]\AgdaField{isEquivalence} \AgdaSymbol{:} \AgdaRecord{IsEquivalence} \AgdaBound{\_\textasciitilde\_}\<%
\\
%
\\
\>[0]\AgdaIndent{2}{}\<[2]%
\>[2]\AgdaKeyword{open} \AgdaModule{IsEquivalence} \AgdaFunction{isEquivalence} \AgdaKeyword{public}\<%
\\
\>\<\end{code}


\AgdaHide{
\begin{code}\>\<%
\\
%
\\
\>\AgdaKeyword{module} \AgdaModule{Quotient} \AgdaKeyword{where}\<%
\\
%
\\
\>\AgdaKeyword{open} \AgdaKeyword{import} \AgdaModule{Data.Product}\<%
\\
\>\AgdaKeyword{open} \AgdaKeyword{import} \AgdaModule{Function}\<%
\\
\>\AgdaKeyword{open} \AgdaKeyword{import} \AgdaModule{Level} \AgdaKeyword{using} \AgdaSymbol{(}\_⊔\_\AgdaSymbol{)}\<%
\\
\>\AgdaComment{-- open import Relation.Binary}\<%
\\
\>\AgdaKeyword{open} \AgdaKeyword{import} \AgdaModule{Data.Nat} \AgdaKeyword{hiding} \AgdaSymbol{(}\_⊔\_\AgdaSymbol{)}\<%
\\
\>\AgdaKeyword{open} \AgdaKeyword{import} \AgdaModule{Setoids}\<%
\\
\>\AgdaComment{-- Setoid = RB.Setoid Level.zero Level.zero}\<%
\\
%
\\
\>\AgdaKeyword{open} \AgdaKeyword{import} \AgdaModule{Relation.Binary.PropositionalEquality} \AgdaSymbol{as} \AgdaModule{PE}\<%
\\
\>[0]\AgdaIndent{2}{}\<[2]%
\>[2]\AgdaKeyword{hiding} \AgdaSymbol{(}[\_]\AgdaSymbol{)}\<%
\\
%
\\
\>\AgdaKeyword{open} \AgdaKeyword{import} \AgdaModule{ThomasProperties}\<%
\\
%
\\
\>\<\end{code}
}

We first define the relation that "$f$ respects $\sim$" (f is compatible with $\sim$)

\begin{code}\>\<%
\\
\>\AgdaFunction{\_respects\_} \AgdaSymbol{:} \AgdaSymbol{\{}\AgdaBound{A} \AgdaSymbol{:} \AgdaPrimitiveType{Set}\AgdaSymbol{\}\{}\AgdaBound{B} \AgdaSymbol{:} \AgdaPrimitiveType{Set}\AgdaSymbol{\}(}\AgdaBound{f} \AgdaSymbol{:} \AgdaBound{A} \AgdaSymbol{→} \AgdaBound{B}\AgdaSymbol{)} \<[43]%
\>[43]\<%
\\
\>[2]\AgdaIndent{11}{}\<[11]%
\>[11]\AgdaSymbol{→} \AgdaSymbol{(}\AgdaBound{\_\textasciitilde\_} \AgdaSymbol{:} \AgdaBound{A} \AgdaSymbol{→} \AgdaBound{A} \AgdaSymbol{→} \AgdaPrimitiveType{Set}\AgdaSymbol{)} \AgdaSymbol{→} \AgdaPrimitiveType{Set}\<%
\\
\>\AgdaBound{f} \AgdaFunction{respects} \AgdaBound{\_\textasciitilde\_} \AgdaSymbol{=} \AgdaSymbol{∀} \AgdaSymbol{\{}\AgdaBound{a} \AgdaBound{a'}\AgdaSymbol{\}} \AgdaSymbol{→} \AgdaBound{a} \AgdaBound{\textasciitilde} \AgdaBound{a'} \AgdaSymbol{→} \AgdaBound{f} \AgdaBound{a} \AgdaDatatype{≡} \AgdaBound{f} \AgdaBound{a'}\<%
\\
\>\<\end{code}

Prequotient

\begin{code}\>\<%
\\
\>\AgdaKeyword{record} \AgdaRecord{pre-Quotient} \AgdaSymbol{(}\AgdaBound{S} \AgdaSymbol{:} \AgdaRecord{Setoid}\AgdaSymbol{)} \AgdaSymbol{:} \AgdaPrimitiveType{Set₁} \AgdaKeyword{where}\<%
\\
\>[0]\AgdaIndent{2}{}\<[2]%
\>[2]\AgdaKeyword{open} \AgdaModule{Setoid} \AgdaBound{S} \AgdaKeyword{renaming} \AgdaSymbol{(}Carrier \AgdaSymbol{to} A\AgdaSymbol{)}\<%
\\
\>[0]\AgdaIndent{2}{}\<[2]%
\>[2]\AgdaKeyword{field}\<%
\\
\>[2]\AgdaIndent{4}{}\<[4]%
\>[4]\AgdaField{Q} \<[8]%
\>[8]\AgdaSymbol{:} \AgdaPrimitiveType{Set}\<%
\\
\>[2]\AgdaIndent{4}{}\<[4]%
\>[4]\AgdaField{[\_]} \AgdaSymbol{:} \AgdaFunction{A} \AgdaSymbol{→} \AgdaBound{Q}\<%
\\
\>[2]\AgdaIndent{4}{}\<[4]%
\>[4]\AgdaField{[\_]⁼} \AgdaSymbol{:} \AgdaBound{[\_]} \AgdaFunction{respects} \AgdaFunction{\_\textasciitilde\_}\<%
\\
\>\<\end{code}
\AgdaHide{
\begin{code}\>\<%
\\
\>[0]\AgdaIndent{2}{}\<[2]%
\>[2]\AgdaKeyword{open} \AgdaModule{Setoid} \AgdaBound{S} \AgdaKeyword{public} \AgdaKeyword{renaming} \<[32]%
\>[32]\<%
\\
\>[0]\AgdaIndent{7}{}\<[7]%
\>[7]\AgdaSymbol{(}Carrier \AgdaSymbol{to} A 
       \AgdaSymbol{;} refl \AgdaSymbol{to} \textasciitilde-refl\AgdaSymbol{;} sym \AgdaSymbol{to} \textasciitilde-sym\AgdaSymbol{;}
       trans \AgdaSymbol{to} \textasciitilde-trans\AgdaSymbol{)}\<%
\\
\>\<\end{code}
}

We can assume UIP which will only be applied on quotient sets

\begin{code}\>\<%
\\
\>\AgdaFunction{≡prop} \AgdaSymbol{:} \AgdaSymbol{\{}\AgdaBound{A} \AgdaSymbol{:} \AgdaPrimitiveType{Set}\AgdaSymbol{\}\{}\AgdaBound{a} \AgdaBound{b} \AgdaSymbol{:} \AgdaBound{A}\AgdaSymbol{\}} \AgdaSymbol{→} \AgdaSymbol{(}\AgdaBound{p} \AgdaBound{q} \AgdaSymbol{:} \AgdaBound{a} \AgdaDatatype{≡} \AgdaBound{b}\AgdaSymbol{)} \AgdaSymbol{→} \AgdaBound{p} \AgdaDatatype{≡} \AgdaBound{q}\<%
\\
\>\AgdaFunction{≡prop} \AgdaSymbol{\{}\AgdaBound{A}\AgdaSymbol{\}} \AgdaSymbol{\{}\AgdaBound{a}\AgdaSymbol{\}} \AgdaSymbol{\{}\AgdaSymbol{.}\AgdaBound{a}\AgdaSymbol{\}} \AgdaInductiveConstructor{refl} \AgdaInductiveConstructor{refl} \AgdaSymbol{=} \AgdaInductiveConstructor{refl}\<%
\\
%
\\
\>\AgdaFunction{subIrr} \AgdaSymbol{:} \AgdaSymbol{\{}\AgdaBound{S} \AgdaSymbol{:} \AgdaPrimitiveType{Set}\AgdaSymbol{\}\{}\AgdaBound{A} \AgdaSymbol{:} \AgdaBound{S} \AgdaSymbol{→} \AgdaPrimitiveType{Set}\AgdaSymbol{\}\{}\AgdaBound{a} \AgdaBound{b} \AgdaSymbol{:} \AgdaBound{S}\AgdaSymbol{\}(}\AgdaBound{p} \AgdaBound{q} \AgdaSymbol{:} \AgdaBound{a} \AgdaDatatype{≡} \AgdaBound{b}\AgdaSymbol{)\{}\AgdaBound{m} \AgdaSymbol{:} \AgdaBound{A} \AgdaBound{a}\AgdaSymbol{\}}\<%
\\
\>[0]\AgdaIndent{7}{}\<[7]%
\>[7]\AgdaSymbol{→} \AgdaFunction{subst} \AgdaBound{A} \AgdaBound{p} \AgdaBound{m} \AgdaDatatype{≡} \AgdaFunction{subst} \AgdaBound{A} \AgdaBound{q} \AgdaBound{m}\<%
\\
\>\AgdaFunction{subIrr} \AgdaBound{p} \AgdaBound{q} \AgdaKeyword{with} \AgdaFunction{≡prop} \AgdaBound{p} \AgdaBound{q}\<%
\\
\>\AgdaFunction{subIrr} \AgdaBound{p} \AgdaSymbol{.}\AgdaBound{p} \AgdaSymbol{|} \AgdaInductiveConstructor{refl} \AgdaSymbol{=} \AgdaInductiveConstructor{refl}\<%
\\
%
\\
\>\AgdaFunction{subIrr2} \AgdaSymbol{:} \AgdaSymbol{\{}\AgdaBound{S} \AgdaSymbol{:} \AgdaPrimitiveType{Set}\AgdaSymbol{\}\{}\AgdaBound{A} \AgdaSymbol{:} \AgdaPrimitiveType{Set}\AgdaSymbol{\}\{}\AgdaBound{a} \AgdaBound{b} \AgdaSymbol{:} \AgdaBound{S}\AgdaSymbol{\}(}\AgdaBound{p} \AgdaSymbol{:} \AgdaBound{a} \AgdaDatatype{≡} \AgdaBound{b}\AgdaSymbol{)\{}\AgdaBound{m} \AgdaSymbol{:} \AgdaBound{A}\AgdaSymbol{\}}\<%
\\
\>[0]\AgdaIndent{7}{}\<[7]%
\>[7]\AgdaSymbol{→} \AgdaFunction{subst} \AgdaSymbol{(λ} \AgdaBound{\_} \AgdaSymbol{→} \AgdaBound{A}\AgdaSymbol{)} \AgdaBound{p} \AgdaBound{m} \AgdaDatatype{≡} \AgdaBound{m}\<%
\\
\>\AgdaFunction{subIrr2} \AgdaInductiveConstructor{refl} \AgdaSymbol{=} \AgdaInductiveConstructor{refl}\<%
\\
\>\<\end{code}

Quotient with dependent eliminator

\begin{code}\>\<%
\\
\>\AgdaKeyword{record} \AgdaRecord{Quotient} \AgdaSymbol{\{}\AgdaBound{S} \AgdaSymbol{:} \AgdaRecord{Setoid}\AgdaSymbol{\}}\<%
\\
\>[0]\AgdaIndent{7}{}\<[7]%
\>[7]\AgdaSymbol{(}\AgdaBound{PQ} \AgdaSymbol{:} \AgdaRecord{pre-Quotient} \AgdaBound{S}\AgdaSymbol{)} \AgdaSymbol{:} \AgdaPrimitiveType{Set₁} \AgdaKeyword{where}\<%
\\
\>[0]\AgdaIndent{2}{}\<[2]%
\>[2]\AgdaKeyword{open} \AgdaModule{pre-Quotient} \AgdaBound{PQ}\<%
\\
\>[0]\AgdaIndent{2}{}\<[2]%
\>[2]\AgdaKeyword{field}\<%
\\
\>[2]\AgdaIndent{4}{}\<[4]%
\>[4]\AgdaField{qelim} \<[12]%
\>[12]\AgdaSymbol{:} \AgdaSymbol{\{}\AgdaBound{B} \AgdaSymbol{:} \AgdaFunction{Q} \AgdaSymbol{→} \AgdaPrimitiveType{Set}\AgdaSymbol{\}}\<%
\\
\>[4]\AgdaIndent{12}{}\<[12]%
\>[12]\AgdaSymbol{→} \AgdaSymbol{(}\AgdaBound{f} \AgdaSymbol{:} \AgdaSymbol{(}\AgdaBound{a} \AgdaSymbol{:} \AgdaFunction{A}\AgdaSymbol{)} \AgdaSymbol{→} \AgdaBound{B} \AgdaFunction{[} \AgdaBound{a} \AgdaFunction{]}\AgdaSymbol{)}\<%
\\
\>[4]\AgdaIndent{12}{}\<[12]%
\>[12]\AgdaSymbol{→} \AgdaSymbol{(∀} \AgdaSymbol{\{}\AgdaBound{a} \AgdaBound{a'}\AgdaSymbol{\}} \AgdaSymbol{→} \AgdaSymbol{(}\AgdaBound{p} \AgdaSymbol{:} \AgdaBound{a} \AgdaFunction{\textasciitilde} \AgdaBound{a'}\AgdaSymbol{)} \<[39]%
\>[39]\<%
\\
\>[4]\AgdaIndent{12}{}\<[12]%
\>[12]\AgdaSymbol{→} \AgdaFunction{subst} \AgdaBound{B} \AgdaFunction{[} \AgdaBound{p} \AgdaFunction{]⁼} \AgdaSymbol{(}\AgdaBound{f} \AgdaBound{a}\AgdaSymbol{)} \AgdaDatatype{≡} \AgdaBound{f} \AgdaBound{a'}\AgdaSymbol{)}\<%
\\
\>[4]\AgdaIndent{12}{}\<[12]%
\>[12]\AgdaSymbol{→} \AgdaSymbol{(}\AgdaBound{q} \AgdaSymbol{:} \AgdaFunction{Q}\AgdaSymbol{)} \AgdaSymbol{→} \AgdaBound{B} \AgdaBound{q}\<%
\\
\>[0]\AgdaIndent{4}{}\<[4]%
\>[4]\AgdaField{qelim-β} \AgdaSymbol{:} \AgdaSymbol{∀} \AgdaSymbol{\{}\AgdaBound{B} \AgdaBound{a} \AgdaBound{f}\AgdaSymbol{\}}\<%
\\
\>[0]\AgdaIndent{12}{}\<[12]%
\>[12]\AgdaSymbol{(}\AgdaBound{resp} \AgdaSymbol{:} \AgdaSymbol{(∀} \AgdaSymbol{\{}\AgdaBound{a} \AgdaBound{a'}\AgdaSymbol{\}} \AgdaSymbol{→} \AgdaSymbol{(}\AgdaBound{p} \AgdaSymbol{:} \AgdaBound{a} \AgdaFunction{\textasciitilde} \AgdaBound{a'}\AgdaSymbol{)} \<[45]%
\>[45]\<%
\\
\>[0]\AgdaIndent{12}{}\<[12]%
\>[12]\AgdaSymbol{→} \AgdaFunction{subst} \AgdaBound{B} \AgdaFunction{[} \AgdaBound{p} \AgdaFunction{]⁼} \AgdaSymbol{(}\AgdaBound{f} \AgdaBound{a}\AgdaSymbol{)} \AgdaDatatype{≡} \AgdaBound{f} \AgdaBound{a'}\AgdaSymbol{))}\<%
\\
\>[0]\AgdaIndent{12}{}\<[12]%
\>[12]\AgdaSymbol{→} \AgdaBound{qelim} \AgdaSymbol{\{}\AgdaBound{B}\AgdaSymbol{\}} \AgdaBound{f} \AgdaBound{resp} \AgdaFunction{[} \AgdaBound{a} \AgdaFunction{]} \AgdaDatatype{≡} \AgdaBound{f} \AgdaBound{a}\<%
\\
\>\<\end{code}

Quotient (Hofmann's)

\begin{code}\>\<%
\\
\>\AgdaKeyword{record} \AgdaRecord{Hof-Quotient} \AgdaSymbol{\{}\AgdaBound{S} \AgdaSymbol{:} \AgdaRecord{Setoid}\AgdaSymbol{\}}\<%
\\
\>[0]\AgdaIndent{7}{}\<[7]%
\>[7]\AgdaSymbol{(}\AgdaBound{PQ} \AgdaSymbol{:} \AgdaRecord{pre-Quotient} \AgdaBound{S}\AgdaSymbol{)} \AgdaSymbol{:} \AgdaPrimitiveType{Set₁} \AgdaKeyword{where}\<%
\\
\>[0]\AgdaIndent{2}{}\<[2]%
\>[2]\AgdaKeyword{open} \AgdaModule{pre-Quotient} \AgdaBound{PQ}\<%
\\
\>[0]\AgdaIndent{2}{}\<[2]%
\>[2]\AgdaKeyword{field}\<%
\\
\>[2]\AgdaIndent{4}{}\<[4]%
\>[4]\AgdaField{lift} \<[11]%
\>[11]\AgdaSymbol{:} \AgdaSymbol{\{}\AgdaBound{B} \AgdaSymbol{:} \AgdaPrimitiveType{Set}\AgdaSymbol{\}}\<%
\\
\>[4]\AgdaIndent{11}{}\<[11]%
\>[11]\AgdaSymbol{→} \AgdaSymbol{(}\AgdaBound{f} \AgdaSymbol{:} \AgdaFunction{A} \AgdaSymbol{→} \AgdaBound{B}\AgdaSymbol{)}\<%
\\
\>[4]\AgdaIndent{11}{}\<[11]%
\>[11]\AgdaSymbol{→} \AgdaBound{f} \AgdaFunction{respects} \AgdaFunction{\_\textasciitilde\_}\<%
\\
\>[4]\AgdaIndent{11}{}\<[11]%
\>[11]\AgdaSymbol{→} \AgdaFunction{Q} \AgdaSymbol{→} \AgdaBound{B}\<%
\\
%
\\
\>[0]\AgdaIndent{4}{}\<[4]%
\>[4]\AgdaField{lift-β} \AgdaSymbol{:} \AgdaSymbol{∀} \AgdaSymbol{\{}\AgdaBound{B} \AgdaBound{a} \AgdaBound{f}\AgdaSymbol{\}(}\AgdaBound{resp} \AgdaSymbol{:} \AgdaBound{f} \AgdaFunction{respects} \AgdaFunction{\_\textasciitilde\_}\AgdaSymbol{)} \<[46]%
\>[46]\<%
\\
\>[0]\AgdaIndent{11}{}\<[11]%
\>[11]\AgdaSymbol{→} \AgdaBound{lift} \AgdaSymbol{\{}\AgdaBound{B}\AgdaSymbol{\}} \AgdaBound{f} \AgdaBound{resp} \AgdaFunction{[} \AgdaBound{a} \AgdaFunction{]} \AgdaDatatype{≡} \AgdaBound{f} \AgdaBound{a}\<%
\\
%
\\
\>[0]\AgdaIndent{4}{}\<[4]%
\>[4]\AgdaField{qind} \<[11]%
\>[11]\AgdaSymbol{:} \AgdaSymbol{∀} \AgdaSymbol{(}\AgdaBound{P} \AgdaSymbol{:} \AgdaFunction{Q} \AgdaSymbol{→} \AgdaPrimitiveType{Set}\AgdaSymbol{)}\<%
\\
\>[0]\AgdaIndent{11}{}\<[11]%
\>[11]\AgdaSymbol{→} \AgdaSymbol{(∀\{}\AgdaBound{x}\AgdaSymbol{\}} \AgdaSymbol{→} \AgdaSymbol{(}\AgdaBound{p} \AgdaBound{q} \AgdaSymbol{:} \AgdaBound{P} \AgdaBound{x}\AgdaSymbol{)} \AgdaSymbol{→} \AgdaBound{p} \AgdaDatatype{≡} \AgdaBound{q}\AgdaSymbol{)}\<%
\\
\>[0]\AgdaIndent{11}{}\<[11]%
\>[11]\AgdaSymbol{→} \AgdaSymbol{(∀} \AgdaBound{a} \AgdaSymbol{→} \AgdaBound{P} \AgdaFunction{[} \AgdaBound{a} \AgdaFunction{]}\AgdaSymbol{)}\<%
\\
\>[0]\AgdaIndent{11}{}\<[11]%
\>[11]\AgdaSymbol{→} \AgdaSymbol{(∀} \AgdaBound{x} \AgdaSymbol{→} \AgdaBound{P} \AgdaBound{x}\AgdaSymbol{)}\<%
\\
\>\<\end{code}


\begin{code}\>\<%
\\
\>\AgdaKeyword{record} \AgdaRecord{Hof-Quotient'} \AgdaSymbol{\{}\AgdaBound{S} \AgdaSymbol{:} \AgdaRecord{Setoid}\AgdaSymbol{\}}\<%
\\
\>[0]\AgdaIndent{7}{}\<[7]%
\>[7]\AgdaSymbol{(}\AgdaBound{PQ} \AgdaSymbol{:} \AgdaRecord{pre-Quotient} \AgdaBound{S}\AgdaSymbol{)} \AgdaSymbol{:} \AgdaPrimitiveType{Set₁} \AgdaKeyword{where}\<%
\\
\>[0]\AgdaIndent{2}{}\<[2]%
\>[2]\AgdaKeyword{open} \AgdaModule{pre-Quotient} \AgdaBound{PQ}\<%
\\
\>[0]\AgdaIndent{2}{}\<[2]%
\>[2]\AgdaKeyword{field}\<%
\\
\>[2]\AgdaIndent{4}{}\<[4]%
\>[4]\AgdaField{lift} \<[11]%
\>[11]\AgdaSymbol{:} \AgdaSymbol{\{}\AgdaBound{B} \AgdaSymbol{:} \AgdaPrimitiveType{Set}\AgdaSymbol{\}}\<%
\\
\>[4]\AgdaIndent{11}{}\<[11]%
\>[11]\AgdaSymbol{→} \AgdaSymbol{(}\AgdaBound{f} \AgdaSymbol{:} \AgdaFunction{A} \AgdaSymbol{→} \AgdaBound{B}\AgdaSymbol{)}\<%
\\
\>[4]\AgdaIndent{11}{}\<[11]%
\>[11]\AgdaSymbol{→} \AgdaBound{f} \AgdaFunction{respects} \AgdaFunction{\_\textasciitilde\_}\<%
\\
\>[4]\AgdaIndent{11}{}\<[11]%
\>[11]\AgdaSymbol{→} \AgdaFunction{Q} \AgdaSymbol{→} \AgdaBound{B}\<%
\\
%
\\
\>[0]\AgdaIndent{4}{}\<[4]%
\>[4]\AgdaField{lift-β} \AgdaSymbol{:} \AgdaSymbol{∀} \AgdaSymbol{\{}\AgdaBound{B} \AgdaBound{a} \AgdaBound{f}\AgdaSymbol{\}(}\AgdaBound{resp} \AgdaSymbol{:} \AgdaBound{f} \AgdaFunction{respects} \AgdaFunction{\_\textasciitilde\_}\AgdaSymbol{)} \<[46]%
\>[46]\<%
\\
\>[0]\AgdaIndent{11}{}\<[11]%
\>[11]\AgdaSymbol{→} \AgdaBound{lift} \AgdaSymbol{\{}\AgdaBound{B}\AgdaSymbol{\}} \AgdaBound{f} \AgdaBound{resp} \AgdaFunction{[} \AgdaBound{a} \AgdaFunction{]} \AgdaDatatype{≡} \AgdaBound{f} \AgdaBound{a}\<%
\\
%
\\
\>[0]\AgdaIndent{4}{}\<[4]%
\>[4]\AgdaField{qind} \<[11]%
\>[11]\AgdaSymbol{:} \AgdaSymbol{∀} \AgdaSymbol{(}\AgdaBound{P} \AgdaSymbol{:} \AgdaFunction{Q} \AgdaSymbol{→} \AgdaPrimitiveType{Set}\AgdaSymbol{)}\<%
\\
\>[0]\AgdaIndent{11}{}\<[11]%
\>[11]\AgdaSymbol{→} \AgdaSymbol{(∀\{}\AgdaBound{x}\AgdaSymbol{\}} \AgdaSymbol{→} \AgdaSymbol{(}\AgdaBound{p} \AgdaBound{q} \AgdaSymbol{:} \AgdaBound{P} \AgdaBound{x}\AgdaSymbol{)} \AgdaSymbol{→} \AgdaBound{p} \AgdaDatatype{≡} \AgdaBound{q}\AgdaSymbol{)}\<%
\\
\>[0]\AgdaIndent{11}{}\<[11]%
\>[11]\AgdaSymbol{→} \AgdaSymbol{(∀} \AgdaBound{a} \AgdaSymbol{→} \AgdaBound{P} \AgdaFunction{[} \AgdaBound{a} \AgdaFunction{]}\AgdaSymbol{)}\<%
\\
\>[0]\AgdaIndent{11}{}\<[11]%
\>[11]\AgdaSymbol{→} \AgdaSymbol{(∀} \AgdaBound{x} \AgdaSymbol{→} \AgdaBound{P} \AgdaBound{x}\AgdaSymbol{)}\<%
\\
\>\<\end{code}

Exact quotient

\begin{code}\>\<%
\\
\>\AgdaKeyword{record} \AgdaRecord{exact-Quotient} \AgdaSymbol{\{}\AgdaBound{S} \AgdaSymbol{:} \AgdaRecord{Setoid}\AgdaSymbol{\}}\<%
\\
\>[0]\AgdaIndent{7}{}\<[7]%
\>[7]\AgdaSymbol{(}\AgdaBound{PQ} \AgdaSymbol{:} \AgdaRecord{pre-Quotient} \AgdaBound{S}\AgdaSymbol{)} \AgdaSymbol{:} \AgdaPrimitiveType{Set₁} \AgdaKeyword{where}\<%
\\
\>[0]\AgdaIndent{2}{}\<[2]%
\>[2]\AgdaKeyword{open} \AgdaModule{pre-Quotient} \AgdaBound{PQ}\<%
\\
\>[0]\AgdaIndent{2}{}\<[2]%
\>[2]\AgdaKeyword{field}\<%
\\
\>[2]\AgdaIndent{4}{}\<[4]%
\>[4]\AgdaField{Qu} \<[10]%
\>[10]\AgdaSymbol{:} \AgdaRecord{Quotient} \AgdaBound{PQ}\<%
\\
\>[2]\AgdaIndent{4}{}\<[4]%
\>[4]\AgdaField{exact} \AgdaSymbol{:} \AgdaSymbol{∀} \AgdaSymbol{\{}\AgdaBound{a} \AgdaBound{b} \AgdaSymbol{:} \AgdaFunction{A}\AgdaSymbol{\}} \AgdaSymbol{→} \AgdaFunction{[} \AgdaBound{a} \AgdaFunction{]} \AgdaDatatype{≡} \AgdaFunction{[} \AgdaBound{b} \AgdaFunction{]} \AgdaSymbol{→} \AgdaBound{a} \AgdaFunction{\textasciitilde} \AgdaBound{b}\<%
\\
\>\<\end{code}

Definable quotient

\begin{code}\>\<%
\\
\>\AgdaKeyword{record} \AgdaRecord{def-Quotient} \AgdaSymbol{\{}\AgdaBound{S} \AgdaSymbol{:} \AgdaRecord{Setoid}\AgdaSymbol{\}}\<%
\\
\>[4]\AgdaIndent{7}{}\<[7]%
\>[7]\AgdaSymbol{(}\AgdaBound{PQ} \AgdaSymbol{:} \AgdaRecord{pre-Quotient} \AgdaBound{S}\AgdaSymbol{)} \AgdaSymbol{:} \AgdaPrimitiveType{Set₁} \AgdaKeyword{where}\<%
\\
\>[1]\AgdaIndent{2}{}\<[2]%
\>[2]\AgdaKeyword{open} \AgdaModule{pre-Quotient} \AgdaBound{PQ}\<%
\\
\>[0]\AgdaIndent{2}{}\<[2]%
\>[2]\AgdaKeyword{field}\<%
\\
\>[2]\AgdaIndent{4}{}\<[4]%
\>[4]\AgdaField{emb} \<[13]%
\>[13]\AgdaSymbol{:} \AgdaFunction{Q} \AgdaSymbol{→} \AgdaFunction{A}\<%
\\
\>[2]\AgdaIndent{4}{}\<[4]%
\>[4]\AgdaField{complete} \AgdaSymbol{:} \AgdaSymbol{∀} \AgdaBound{a} \AgdaSymbol{→} \AgdaBound{emb} \AgdaFunction{[} \AgdaBound{a} \AgdaFunction{]} \AgdaFunction{\textasciitilde} \AgdaBound{a}\<%
\\
\>[2]\AgdaIndent{4}{}\<[4]%
\>[4]\AgdaField{stable} \<[13]%
\>[13]\AgdaSymbol{:} \AgdaSymbol{∀} \AgdaBound{q} \AgdaSymbol{→} \AgdaFunction{[} \AgdaBound{emb} \AgdaBound{q} \AgdaFunction{]} \AgdaDatatype{≡} \AgdaBound{q}\<%
\\
\>\<\end{code}

\textbf{Proof :} Definable quotients are exact.

\begin{code}\>\<%
\\
\>[0]\AgdaIndent{2}{}\<[2]%
\>[2]\AgdaFunction{exact} \AgdaSymbol{:} \AgdaSymbol{∀\{}\AgdaBound{a} \AgdaBound{b}\AgdaSymbol{\}} \AgdaSymbol{→} \AgdaFunction{[} \AgdaBound{a} \AgdaFunction{]} \AgdaDatatype{≡} \AgdaFunction{[} \AgdaBound{b} \AgdaFunction{]} \AgdaSymbol{→} \AgdaBound{a} \AgdaFunction{\textasciitilde} \AgdaBound{b}\<%
\\
\>[0]\AgdaIndent{2}{}\<[2]%
\>[2]\AgdaFunction{exact} \AgdaSymbol{\{}\AgdaBound{a}\AgdaSymbol{\}} \AgdaSymbol{\{}\AgdaBound{b}\AgdaSymbol{\}} \AgdaBound{p} \AgdaSymbol{=} \<[20]%
\>[20]\<%
\\
\>[2]\AgdaIndent{4}{}\<[4]%
\>[4]\AgdaFunction{\textasciitilde-trans} \AgdaSymbol{(}\AgdaFunction{\textasciitilde-sym} \AgdaSymbol{(}\AgdaFunction{complete} \AgdaBound{a}\AgdaSymbol{))} \<[33]%
\>[33]\<%
\\
\>[2]\AgdaIndent{4}{}\<[4]%
\>[4]\AgdaSymbol{(}\AgdaFunction{\textasciitilde-trans} \AgdaSymbol{(}\AgdaFunction{subst} \AgdaSymbol{(λ} \AgdaBound{x} \AgdaSymbol{→} \<[27]%
\>[27]\<%
\\
\>[2]\AgdaIndent{4}{}\<[4]%
\>[4]\AgdaFunction{emb} \AgdaFunction{[} \AgdaBound{a} \AgdaFunction{]} \AgdaFunction{\textasciitilde} \AgdaFunction{emb} \AgdaBound{x}\AgdaSymbol{)} \<[23]%
\>[23]\<%
\\
\>[2]\AgdaIndent{4}{}\<[4]%
\>[4]\AgdaBound{p} \AgdaFunction{\textasciitilde-refl}\AgdaSymbol{)} \AgdaSymbol{(}\AgdaFunction{complete} \AgdaBound{b}\AgdaSymbol{))}\<%
\\
\>\<\end{code}


\textbf{Equivalences and conversions among the quotient structures}

\AgdaHide{
\begin{code}\>\<%
\\
\>\AgdaFunction{Σeq} \AgdaSymbol{:} \AgdaSymbol{\{}\AgdaBound{A} \AgdaSymbol{:} \AgdaPrimitiveType{Set}\AgdaSymbol{\}\{}\AgdaBound{B} \AgdaSymbol{:} \AgdaBound{A} \AgdaSymbol{→} \AgdaPrimitiveType{Set}\AgdaSymbol{\}\{}\AgdaBound{a} \AgdaBound{a'} \AgdaSymbol{:} \AgdaBound{A}\AgdaSymbol{\}}\<%
\\
\>[4]\AgdaIndent{6}{}\<[6]%
\>[6]\AgdaSymbol{\{}\AgdaBound{b} \AgdaSymbol{:} \AgdaBound{B} \AgdaBound{a}\AgdaSymbol{\}\{}\AgdaBound{b'} \AgdaSymbol{:} \AgdaBound{B} \AgdaBound{a'}\AgdaSymbol{\}(}\AgdaBound{p} \AgdaSymbol{:} \AgdaBound{a} \AgdaDatatype{≡} \AgdaBound{a'}\AgdaSymbol{)} \<[39]%
\>[39]\<%
\\
\>[0]\AgdaIndent{4}{}\<[4]%
\>[4]\AgdaSymbol{→} \AgdaFunction{subst} \AgdaBound{B} \AgdaBound{p} \AgdaBound{b} \AgdaDatatype{≡} \AgdaBound{b'} \AgdaSymbol{→} \AgdaSymbol{(}\AgdaBound{a} \AgdaInductiveConstructor{,} \AgdaBound{b}\AgdaSymbol{)} \AgdaDatatype{≡} \AgdaSymbol{(}\AgdaBound{a'} \AgdaInductiveConstructor{,} \AgdaBound{b'}\AgdaSymbol{)}\<%
\\
\>\AgdaFunction{Σeq} \AgdaInductiveConstructor{refl} \AgdaInductiveConstructor{refl} \AgdaSymbol{=} \AgdaInductiveConstructor{refl}\<%
\\
%
\\
%
\\
\>\AgdaFunction{ind2dep} \AgdaSymbol{:} \AgdaSymbol{∀} \AgdaSymbol{\{}\AgdaBound{Q} \AgdaSymbol{:} \AgdaPrimitiveType{Set}\AgdaSymbol{\}\{}\AgdaBound{B} \AgdaSymbol{:} \AgdaBound{Q} \AgdaSymbol{→} \AgdaPrimitiveType{Set}\AgdaSymbol{\}}\<%
\\
\>[0]\AgdaIndent{8}{}\<[8]%
\>[8]\AgdaSymbol{→} \AgdaSymbol{(}\AgdaBound{f} \AgdaSymbol{:} \AgdaBound{Q} \AgdaSymbol{→} \AgdaRecord{Σ} \AgdaBound{Q} \AgdaBound{B}\AgdaSymbol{)}\<%
\\
\>[0]\AgdaIndent{8}{}\<[8]%
\>[8]\AgdaSymbol{→} \AgdaSymbol{(∀} \AgdaBound{q} \AgdaSymbol{→} \AgdaFunction{proj₁} \AgdaSymbol{(}\AgdaBound{f} \AgdaBound{q}\AgdaSymbol{)} \AgdaDatatype{≡} \AgdaBound{q}\AgdaSymbol{)}\<%
\\
\>[0]\AgdaIndent{8}{}\<[8]%
\>[8]\AgdaSymbol{→} \AgdaSymbol{(}\AgdaBound{q} \AgdaSymbol{:} \AgdaBound{Q}\AgdaSymbol{)} \AgdaSymbol{→} \AgdaBound{B} \AgdaBound{q}\<%
\\
\>\AgdaFunction{ind2dep} \AgdaSymbol{\{}\AgdaBound{Q}\AgdaSymbol{\}} \AgdaSymbol{\{}\AgdaBound{B}\AgdaSymbol{\}} \AgdaBound{f} \AgdaBound{id'} \AgdaBound{q} \AgdaSymbol{=} \AgdaFunction{subst} \AgdaBound{B} \AgdaSymbol{(}\AgdaBound{id'} \AgdaBound{q}\AgdaSymbol{)} \AgdaSymbol{(}\AgdaFunction{proj₂} \AgdaSymbol{(}\AgdaBound{f} \AgdaBound{q}\AgdaSymbol{))}\<%
\\
\>\<\end{code}
}

\textbf{Proof :} Hofmann's definition of quotient is equivalent to Quotient.

\begin{code}\>\<%
\\
\>\AgdaFunction{Hof-Quotient→Quotient} \AgdaSymbol{:} \AgdaSymbol{\{}\AgdaBound{S} \AgdaSymbol{:} \AgdaRecord{Setoid}\AgdaSymbol{\}\{}\AgdaBound{PQ} \AgdaSymbol{:} \AgdaRecord{pre-Quotient} \AgdaBound{S}\AgdaSymbol{\}} \AgdaSymbol{→}\<%
\\
\>[0]\AgdaIndent{2}{}\<[2]%
\>[2]\AgdaSymbol{(}\AgdaRecord{Hof-Quotient} \AgdaBound{PQ}\AgdaSymbol{)} \AgdaSymbol{→} \AgdaSymbol{(}\AgdaRecord{Quotient} \AgdaBound{PQ}\AgdaSymbol{)}\<%
\\
\>\AgdaFunction{Hof-Quotient→Quotient} \AgdaSymbol{\{}\AgdaBound{S}\AgdaSymbol{\}} \AgdaSymbol{\{}\AgdaBound{PQ}\AgdaSymbol{\}} \AgdaBound{QuH} \AgdaSymbol{=} \<[37]%
\>[37]\<%
\\
\>[0]\AgdaIndent{2}{}\<[2]%
\>[2]\AgdaKeyword{record} \<[9]%
\>[9]\<%
\\
\>[2]\AgdaIndent{4}{}\<[4]%
\>[4]\AgdaSymbol{\{} \AgdaField{qelim} \<[14]%
\>[14]\AgdaSymbol{=} \AgdaSymbol{λ} \AgdaSymbol{\{}\AgdaBound{B}\AgdaSymbol{\}} \AgdaBound{f} \AgdaBound{resp} \<[29]%
\>[29]\<%
\\
\>[2]\AgdaIndent{4}{}\<[4]%
\>[4]\AgdaSymbol{→} \AgdaFunction{proj₁} \AgdaSymbol{(}\AgdaFunction{qelim'} \AgdaBound{f} \AgdaBound{resp}\AgdaSymbol{)}\<%
\\
\>[2]\AgdaIndent{4}{}\<[4]%
\>[4]\AgdaSymbol{;} \AgdaField{qelim-β} \AgdaSymbol{=} \AgdaSymbol{λ} \AgdaSymbol{\{}\AgdaBound{B}\AgdaSymbol{\}} \AgdaSymbol{\{}\AgdaBound{a}\AgdaSymbol{\}} \AgdaSymbol{\{}\AgdaBound{f}\AgdaSymbol{\}} \AgdaBound{resp} \<[35]%
\>[35]\<%
\\
\>[2]\AgdaIndent{4}{}\<[4]%
\>[4]\AgdaSymbol{→} \AgdaFunction{proj₂} \AgdaSymbol{(}\AgdaFunction{qelim'} \AgdaBound{f} \AgdaBound{resp}\AgdaSymbol{)}\<%
\\
\>[2]\AgdaIndent{4}{}\<[4]%
\>[4]\AgdaSymbol{\}}\<%
\\
\>[0]\AgdaIndent{2}{}\<[2]%
\>[2]\AgdaKeyword{where}\<%
\\
\>[0]\AgdaIndent{4}{}\<[4]%
\>[4]\AgdaKeyword{open} \AgdaModule{pre-Quotient} \AgdaBound{PQ}\<%
\\
\>[0]\AgdaIndent{4}{}\<[4]%
\>[4]\AgdaKeyword{open} \AgdaModule{Hof-Quotient} \AgdaBound{QuH}\<%
\\
%
\\
\>[0]\AgdaIndent{4}{}\<[4]%
\>[4]\AgdaFunction{qelim'} \AgdaSymbol{:} \AgdaSymbol{\{}\AgdaBound{B} \AgdaSymbol{:} \AgdaFunction{Q} \AgdaSymbol{→} \AgdaPrimitiveType{Set}\AgdaSymbol{\}}\<%
\\
\>[4]\AgdaIndent{11}{}\<[11]%
\>[11]\AgdaSymbol{→} \AgdaSymbol{(}\AgdaBound{f} \AgdaSymbol{:} \AgdaSymbol{(}\AgdaBound{a} \AgdaSymbol{:} \AgdaFunction{A}\AgdaSymbol{)} \AgdaSymbol{→} \AgdaBound{B} \AgdaFunction{[} \AgdaBound{a} \AgdaFunction{]}\AgdaSymbol{)}\<%
\\
\>[4]\AgdaIndent{11}{}\<[11]%
\>[11]\AgdaSymbol{→} \AgdaSymbol{(∀} \AgdaSymbol{\{}\AgdaBound{a} \AgdaBound{a'}\AgdaSymbol{\}} \AgdaSymbol{→} \AgdaSymbol{(}\AgdaBound{p} \AgdaSymbol{:} \AgdaBound{a} \AgdaFunction{\textasciitilde} \AgdaBound{a'}\AgdaSymbol{)} \<[38]%
\>[38]\<%
\\
\>[4]\AgdaIndent{11}{}\<[11]%
\>[11]\AgdaSymbol{→} \AgdaFunction{subst} \AgdaBound{B} \AgdaFunction{[} \AgdaBound{p} \AgdaFunction{]⁼} \AgdaSymbol{(}\AgdaBound{f} \AgdaBound{a}\AgdaSymbol{)} \AgdaDatatype{≡} \AgdaBound{f} \AgdaBound{a'}\AgdaSymbol{)}\<%
\\
\>[4]\AgdaIndent{11}{}\<[11]%
\>[11]\AgdaSymbol{→} \AgdaRecord{Σ[} \AgdaBound{f\textasciicircum} \AgdaRecord{∶} \AgdaSymbol{((}\AgdaBound{q} \AgdaSymbol{:} \AgdaFunction{Q}\AgdaSymbol{)} \AgdaSymbol{→} \AgdaBound{B} \AgdaBound{q}\AgdaSymbol{)} \AgdaRecord{]} \<[39]%
\>[39]\<%
\\
\>[11]\AgdaIndent{14}{}\<[14]%
\>[14]\AgdaSymbol{(∀} \AgdaSymbol{\{}\AgdaBound{a}\AgdaSymbol{\}} \AgdaSymbol{→} \AgdaBound{f\textasciicircum} \AgdaFunction{[} \AgdaBound{a} \AgdaFunction{]} \AgdaDatatype{≡} \AgdaBound{f} \AgdaBound{a}\AgdaSymbol{)}\<%
\\
\>[-1]\AgdaIndent{4}{}\<[4]%
\>[4]\AgdaFunction{qelim'} \AgdaSymbol{\{}\AgdaBound{B}\AgdaSymbol{\}} \AgdaBound{f} \AgdaBound{resp} \AgdaSymbol{=} \<[25]%
\>[25]\AgdaFunction{f\textasciicircum} \AgdaInductiveConstructor{,} \AgdaFunction{f\textasciicircum-β}\<%
\\
\>[0]\AgdaIndent{10}{}\<[10]%
\>[10]\AgdaKeyword{where}\<%
\\
%
\\
\>[10]\AgdaIndent{11}{}\<[11]%
\>[11]\AgdaFunction{f₀} \AgdaSymbol{:} \AgdaFunction{A} \AgdaSymbol{→} \AgdaRecord{Σ} \AgdaFunction{Q} \AgdaBound{B}\<%
\\
\>[10]\AgdaIndent{11}{}\<[11]%
\>[11]\AgdaFunction{f₀} \AgdaBound{a} \AgdaSymbol{=} \AgdaFunction{[} \AgdaBound{a} \AgdaFunction{]} \AgdaInductiveConstructor{,} \AgdaBound{f} \AgdaBound{a}\<%
\\
\>[-3]\AgdaIndent{4}{}\<[4]%
\>[4]\<%
\\
\>[0]\AgdaIndent{11}{}\<[11]%
\>[11]\AgdaFunction{resp₀} \AgdaSymbol{:} \AgdaFunction{f₀} \AgdaFunction{respects} \AgdaFunction{\_\textasciitilde\_}\<%
\\
\>[0]\AgdaIndent{11}{}\<[11]%
\>[11]\AgdaFunction{resp₀} \AgdaBound{p} \AgdaSymbol{=} \AgdaFunction{Σeq} \AgdaFunction{[} \AgdaBound{p} \AgdaFunction{]⁼} \AgdaSymbol{(}\AgdaBound{resp} \AgdaBound{p}\AgdaSymbol{)}\<%
\\
%
\\
%
\\
\>[0]\AgdaIndent{11}{}\<[11]%
\>[11]\AgdaFunction{f'} \AgdaSymbol{:} \AgdaFunction{Q} \AgdaSymbol{→} \AgdaRecord{Σ} \AgdaFunction{Q} \AgdaBound{B}\<%
\\
\>[0]\AgdaIndent{11}{}\<[11]%
\>[11]\AgdaFunction{f'} \AgdaSymbol{=} \AgdaFunction{lift} \AgdaFunction{f₀} \AgdaFunction{resp₀}\<%
\\
%
\\
\>[0]\AgdaIndent{11}{}\<[11]%
\>[11]\AgdaFunction{id'} \AgdaSymbol{:} \AgdaFunction{Q} \AgdaSymbol{→} \AgdaFunction{Q}\<%
\\
\>[0]\AgdaIndent{11}{}\<[11]%
\>[11]\AgdaFunction{id'} \AgdaSymbol{=} \AgdaFunction{proj₁} \AgdaFunction{∘} \AgdaFunction{f'}\<%
\\
\>[0]\AgdaIndent{11}{}\<[11]%
\>[11]\<%
\\
\>[0]\AgdaIndent{11}{}\<[11]%
\>[11]\AgdaFunction{P} \AgdaSymbol{:} \AgdaFunction{Q} \AgdaSymbol{→} \AgdaPrimitiveType{Set}\<%
\\
\>[0]\AgdaIndent{11}{}\<[11]%
\>[11]\AgdaFunction{P} \AgdaBound{q} \AgdaSymbol{=} \AgdaFunction{id'} \AgdaBound{q} \AgdaDatatype{≡} \AgdaBound{q}\<%
\\
%
\\
\>[0]\AgdaIndent{11}{}\<[11]%
\>[11]\AgdaFunction{f'-β} \AgdaSymbol{:} \AgdaSymbol{\{}\AgdaBound{a} \AgdaSymbol{:} \AgdaFunction{A}\AgdaSymbol{\}} \AgdaSymbol{→} \AgdaFunction{f'} \AgdaFunction{[} \AgdaBound{a} \AgdaFunction{]} \AgdaDatatype{≡} \AgdaFunction{[} \AgdaBound{a} \AgdaFunction{]} \AgdaInductiveConstructor{,} \AgdaBound{f} \AgdaBound{a}\<%
\\
\>[0]\AgdaIndent{11}{}\<[11]%
\>[11]\AgdaFunction{f'-β} \AgdaSymbol{=} \AgdaFunction{lift-β} \AgdaSymbol{\_}\<%
\\
%
\\
\>[0]\AgdaIndent{11}{}\<[11]%
\>[11]\AgdaFunction{isIda} \AgdaSymbol{:} \AgdaSymbol{∀} \AgdaSymbol{\{}\AgdaBound{a}\AgdaSymbol{\}} \AgdaSymbol{→} \AgdaFunction{id'} \AgdaFunction{[} \AgdaBound{a} \AgdaFunction{]} \AgdaDatatype{≡} \AgdaFunction{[} \AgdaBound{a} \AgdaFunction{]}\<%
\\
\>[0]\AgdaIndent{11}{}\<[11]%
\>[11]\AgdaFunction{isIda} \AgdaSymbol{=} \AgdaFunction{cong} \AgdaFunction{proj₁} \AgdaFunction{f'-β}\<%
\\
%
\\
\>[0]\AgdaIndent{11}{}\<[11]%
\>[11]\AgdaFunction{isIdq} \AgdaSymbol{:} \AgdaSymbol{∀} \AgdaSymbol{\{}\AgdaBound{q}\AgdaSymbol{\}} \AgdaSymbol{→} \AgdaFunction{id'} \AgdaBound{q} \AgdaDatatype{≡} \AgdaBound{q}\<%
\\
\>[0]\AgdaIndent{11}{}\<[11]%
\>[11]\AgdaFunction{isIdq} \AgdaSymbol{\{}\AgdaBound{q}\AgdaSymbol{\}} \AgdaSymbol{=} \AgdaFunction{qind} \AgdaFunction{P} \AgdaFunction{≡prop} \AgdaSymbol{(λ} \AgdaBound{\_} \AgdaSymbol{→} \AgdaFunction{isIda}\AgdaSymbol{)} \AgdaBound{q}\<%
\\
%
\\
\>[0]\AgdaIndent{11}{}\<[11]%
\>[11]\AgdaFunction{f\textasciicircum} \AgdaSymbol{:} \AgdaSymbol{(}\AgdaBound{q} \AgdaSymbol{:} \AgdaFunction{Q}\AgdaSymbol{)} \AgdaSymbol{→} \AgdaBound{B} \AgdaBound{q}\<%
\\
\>[0]\AgdaIndent{11}{}\<[11]%
\>[11]\AgdaFunction{f\textasciicircum} \AgdaBound{q} \AgdaSymbol{=} \AgdaFunction{subst} \AgdaBound{B} \AgdaFunction{isIdq} \AgdaSymbol{(}\AgdaFunction{proj₂} \AgdaSymbol{(}\AgdaFunction{f'} \AgdaBound{q}\AgdaSymbol{))}\<%
\\
%
\\
\>[0]\AgdaIndent{11}{}\<[11]%
\>[11]\AgdaFunction{f'-sound2} \AgdaSymbol{:} \AgdaSymbol{∀} \AgdaSymbol{\{}\AgdaBound{a}\AgdaSymbol{\}} \AgdaSymbol{→} \<[31]%
\>[31]\<%
\\
\>[11]\AgdaIndent{21}{}\<[21]%
\>[21]\AgdaFunction{subst} \AgdaBound{B} \AgdaFunction{isIda} \AgdaSymbol{(}\AgdaFunction{proj₂} \AgdaSymbol{(}\AgdaFunction{f'} \AgdaFunction{[} \AgdaBound{a} \AgdaFunction{]}\AgdaSymbol{))} \AgdaDatatype{≡} \AgdaBound{f} \AgdaBound{a}\<%
\\
\>[0]\AgdaIndent{11}{}\<[11]%
\>[11]\AgdaFunction{f'-sound2} \AgdaSymbol{=} \AgdaFunction{cong-proj₂} \AgdaSymbol{\_} \AgdaSymbol{\_} \AgdaFunction{f'-β}\<%
\\
\>[0]\AgdaIndent{11}{}\<[11]%
\>[11]\<%
\\
\>[0]\AgdaIndent{11}{}\<[11]%
\>[11]\AgdaFunction{f\textasciicircum-β} \AgdaSymbol{:} \AgdaSymbol{∀} \AgdaSymbol{\{}\AgdaBound{a}\AgdaSymbol{\}} \AgdaSymbol{→} \AgdaFunction{f\textasciicircum} \AgdaFunction{[} \AgdaBound{a} \AgdaFunction{]} \AgdaDatatype{≡} \AgdaBound{f} \AgdaBound{a}\<%
\\
\>[0]\AgdaIndent{11}{}\<[11]%
\>[11]\AgdaFunction{f\textasciicircum-β} \AgdaSymbol{\{}\AgdaBound{a}\AgdaSymbol{\}} \AgdaSymbol{=} \AgdaFunction{trans} \AgdaSymbol{(}\AgdaFunction{subIrr} \AgdaFunction{isIdq} \AgdaFunction{isIda}\AgdaSymbol{)} \AgdaFunction{f'-sound2}\<%
\\
\>\<\end{code}

\begin{code}\>\<%
\\
\>\AgdaFunction{Quotient→Hof-Quotient} \AgdaSymbol{:} \<[24]%
\>[24]\<%
\\
\>[0]\AgdaIndent{2}{}\<[2]%
\>[2]\AgdaSymbol{\{}\AgdaBound{S} \AgdaSymbol{:} \AgdaRecord{Setoid}\AgdaSymbol{\}\{}\AgdaBound{PQ} \AgdaSymbol{:} \AgdaRecord{pre-Quotient} \AgdaBound{S}\AgdaSymbol{\}}\<%
\\
\>[0]\AgdaIndent{2}{}\<[2]%
\>[2]\AgdaSymbol{→} \AgdaSymbol{(}\AgdaRecord{Quotient} \AgdaBound{PQ}\AgdaSymbol{)}\<%
\\
\>[0]\AgdaIndent{2}{}\<[2]%
\>[2]\AgdaSymbol{→} \AgdaSymbol{(}\AgdaRecord{Hof-Quotient} \AgdaBound{PQ}\AgdaSymbol{)}\<%
\\
\>\AgdaFunction{Quotient→Hof-Quotient} \AgdaSymbol{\{}\AgdaBound{S}\AgdaSymbol{\}} \AgdaSymbol{\{}\AgdaBound{PQ}\AgdaSymbol{\}} \AgdaBound{QU} \AgdaSymbol{=}\<%
\\
\>[0]\AgdaIndent{2}{}\<[2]%
\>[2]\AgdaKeyword{record}\<%
\\
\>[0]\AgdaIndent{2}{}\<[2]%
\>[2]\AgdaSymbol{\{} \AgdaField{lift} \<[11]%
\>[11]\AgdaSymbol{=} \AgdaSymbol{λ} \AgdaBound{f} \AgdaBound{resp} \<[22]%
\>[22]\<%
\\
\>[2]\AgdaIndent{13}{}\<[13]%
\>[13]\AgdaSymbol{→} \AgdaFunction{qelim} \AgdaBound{f} \AgdaSymbol{(}\AgdaFunction{resp'} \AgdaBound{resp}\AgdaSymbol{)}\<%
\\
\>[0]\AgdaIndent{2}{}\<[2]%
\>[2]\AgdaSymbol{;} \AgdaField{lift-β} \AgdaSymbol{=} \AgdaSymbol{λ} \AgdaBound{resp} \<[20]%
\>[20]\<%
\\
\>[0]\AgdaIndent{13}{}\<[13]%
\>[13]\AgdaSymbol{→} \AgdaFunction{qelim-β} \AgdaSymbol{(}\AgdaFunction{resp'} \AgdaBound{resp}\AgdaSymbol{)}\<%
\\
\>[0]\AgdaIndent{2}{}\<[2]%
\>[2]\AgdaSymbol{;} \AgdaField{qind} \AgdaSymbol{=} \AgdaSymbol{λ} \AgdaBound{P} \AgdaBound{isP} \AgdaBound{f} \<[21]%
\>[21]\<%
\\
\>[0]\AgdaIndent{11}{}\<[11]%
\>[11]\AgdaSymbol{→} \AgdaFunction{qelim} \AgdaSymbol{\{}\AgdaBound{P}\AgdaSymbol{\}} \AgdaBound{f} \AgdaSymbol{(λ} \AgdaBound{\_} \AgdaSymbol{→} \AgdaBound{isP} \AgdaSymbol{\_} \AgdaSymbol{\_)}\<%
\\
\>[0]\AgdaIndent{2}{}\<[2]%
\>[2]\AgdaSymbol{\}}\<%
\\
\>[0]\AgdaIndent{2}{}\<[2]%
\>[2]\AgdaKeyword{where}\<%
\\
\>[2]\AgdaIndent{4}{}\<[4]%
\>[4]\AgdaKeyword{open} \AgdaModule{pre-Quotient} \AgdaBound{PQ}\<%
\\
\>[2]\AgdaIndent{4}{}\<[4]%
\>[4]\AgdaKeyword{open} \AgdaModule{Quotient} \AgdaBound{QU}\<%
\\
%
\\
\>[2]\AgdaIndent{4}{}\<[4]%
\>[4]\AgdaFunction{resp'} \AgdaSymbol{:} \AgdaSymbol{\{}\AgdaBound{B} \AgdaSymbol{:} \AgdaPrimitiveType{Set}\AgdaSymbol{\}\{}\AgdaBound{a} \AgdaBound{a'} \AgdaSymbol{:} \AgdaFunction{A}\AgdaSymbol{\}}\<%
\\
\>[4]\AgdaIndent{10}{}\<[10]%
\>[10]\AgdaSymbol{\{}\AgdaBound{f} \AgdaSymbol{:} \AgdaFunction{A} \AgdaSymbol{→} \AgdaBound{B}\AgdaSymbol{\}}\<%
\\
\>[4]\AgdaIndent{10}{}\<[10]%
\>[10]\AgdaSymbol{(}\AgdaBound{resp} \AgdaSymbol{:} \AgdaBound{f} \AgdaFunction{respects} \AgdaFunction{\_\textasciitilde\_}\AgdaSymbol{)}\<%
\\
\>[4]\AgdaIndent{10}{}\<[10]%
\>[10]\AgdaSymbol{(}\AgdaBound{p} \AgdaSymbol{:} \AgdaBound{a} \AgdaFunction{\textasciitilde} \AgdaBound{a'}\AgdaSymbol{)}\<%
\\
\>[4]\AgdaIndent{10}{}\<[10]%
\>[10]\AgdaSymbol{→} \AgdaFunction{subst} \AgdaSymbol{(λ} \AgdaBound{\_} \AgdaSymbol{→} \AgdaBound{B}\AgdaSymbol{)} \AgdaFunction{[} \AgdaBound{p} \AgdaFunction{]⁼} \AgdaSymbol{(}\AgdaBound{f} \AgdaBound{a}\AgdaSymbol{)} \<[41]%
\>[41]\<%
\\
\>[4]\AgdaIndent{10}{}\<[10]%
\>[10]\AgdaDatatype{≡} \AgdaBound{f} \AgdaBound{a'}\<%
\\
\>[0]\AgdaIndent{4}{}\<[4]%
\>[4]\AgdaFunction{resp'} \AgdaBound{resp} \AgdaBound{p} \AgdaSymbol{=} \<[19]%
\>[19]\<%
\\
\>[0]\AgdaIndent{10}{}\<[10]%
\>[10]\AgdaFunction{trans} \AgdaSymbol{(}\AgdaFunction{subIrr2} \AgdaFunction{[} \AgdaBound{p} \AgdaFunction{]⁼}\AgdaSymbol{)}\<%
\\
\>[0]\AgdaIndent{10}{}\<[10]%
\>[10]\AgdaSymbol{(}\AgdaBound{resp} \AgdaBound{p}\AgdaSymbol{)}\<%
\\
\>\<\end{code}

\textbf{Proof :} A definable quotient gives rise to a \emph{quotient}.

\begin{code}\>\<%
\\
\>\AgdaFunction{def-Quotient→Quotient} \AgdaSymbol{:} \<[24]%
\>[24]\<%
\\
\>[0]\AgdaIndent{2}{}\<[2]%
\>[2]\AgdaSymbol{\{}\AgdaBound{S} \AgdaSymbol{:} \AgdaRecord{Setoid}\AgdaSymbol{\}\{}\AgdaBound{PQ} \AgdaSymbol{:} \AgdaRecord{pre-Quotient} \AgdaBound{S}\AgdaSymbol{\}}\<%
\\
\>[0]\AgdaIndent{2}{}\<[2]%
\>[2]\AgdaSymbol{→} \AgdaSymbol{(}\AgdaRecord{def-Quotient} \AgdaBound{PQ}\AgdaSymbol{)} \AgdaSymbol{→} \AgdaSymbol{(}\AgdaRecord{Quotient} \AgdaBound{PQ}\AgdaSymbol{)}\<%
\\
\>\AgdaFunction{def-Quotient→Quotient} \AgdaSymbol{\{}\AgdaBound{S}\AgdaSymbol{\}} \AgdaSymbol{\{}\AgdaBound{PQ}\AgdaSymbol{\}} \AgdaBound{QuD} \AgdaSymbol{=} \<[37]%
\>[37]\<%
\\
\>[0]\AgdaIndent{2}{}\<[2]%
\>[2]\AgdaKeyword{record} \AgdaSymbol{\{} \AgdaField{qelim} \AgdaSymbol{=} \<[19]%
\>[19]\<%
\\
\>[2]\AgdaIndent{9}{}\<[9]%
\>[9]\AgdaSymbol{λ} \AgdaSymbol{\{}\AgdaBound{B}\AgdaSymbol{\}} \AgdaBound{f} \AgdaBound{resp} \AgdaBound{q} \AgdaSymbol{→} \AgdaFunction{subst} \AgdaBound{B} \AgdaSymbol{(}\AgdaFunction{stable} \AgdaBound{q}\AgdaSymbol{)} \AgdaSymbol{(}\AgdaBound{f} \AgdaSymbol{(}\AgdaFunction{emb} \AgdaBound{q}\AgdaSymbol{))}\<%
\\
\>[2]\AgdaIndent{9}{}\<[9]%
\>[9]\AgdaSymbol{;} \AgdaField{qelim-β} \AgdaSymbol{=} \<[21]%
\>[21]\<%
\\
\>[2]\AgdaIndent{9}{}\<[9]%
\>[9]\AgdaSymbol{λ} \AgdaSymbol{\{}\AgdaBound{B}\AgdaSymbol{\}} \AgdaSymbol{\{}\AgdaBound{a}\AgdaSymbol{\}} \AgdaSymbol{\{}\AgdaBound{f}\AgdaSymbol{\}} \AgdaBound{resp} \AgdaSymbol{→} \<[30]%
\>[30]\<%
\\
\>[2]\AgdaIndent{9}{}\<[9]%
\>[9]\AgdaFunction{trans} \AgdaSymbol{(}\AgdaFunction{subIrr} \AgdaSymbol{(}\AgdaFunction{stable} \AgdaFunction{[} \AgdaBound{a} \AgdaFunction{]}\AgdaSymbol{)} \<[38]%
\>[38]\<%
\\
\>[2]\AgdaIndent{9}{}\<[9]%
\>[9]\AgdaFunction{[} \AgdaFunction{complete} \AgdaBound{a} \AgdaFunction{]⁼}\AgdaSymbol{)} \AgdaSymbol{(}\AgdaBound{resp} \AgdaSymbol{(}\AgdaFunction{complete} \AgdaBound{a}\AgdaSymbol{))}\<%
\\
%
\\
\>[0]\AgdaIndent{2}{}\<[2]%
\>[2]\AgdaSymbol{\}}\<%
\\
\>[0]\AgdaIndent{4}{}\<[4]%
\>[4]\AgdaKeyword{where}\<%
\\
\>[0]\AgdaIndent{4}{}\<[4]%
\>[4]\AgdaKeyword{open} \AgdaModule{pre-Quotient} \AgdaBound{PQ}\<%
\\
\>[0]\AgdaIndent{4}{}\<[4]%
\>[4]\AgdaKeyword{open} \AgdaModule{def-Quotient} \AgdaBound{QuD}\<%
\\
\>\<\end{code}


\textbf{Proof :} A definable quotients gives rise to an \emph{exact (effective) quotient}.

\begin{code}\>\<%
\\
\>\AgdaFunction{def-Quotient→exact-Quotient} \AgdaSymbol{:} \<[30]%
\>[30]\<%
\\
\>[0]\AgdaIndent{2}{}\<[2]%
\>[2]\AgdaSymbol{\{}\AgdaBound{S} \AgdaSymbol{:} \AgdaRecord{Setoid}\AgdaSymbol{\}\{}\AgdaBound{PQ} \AgdaSymbol{:} \AgdaRecord{pre-Quotient} \AgdaBound{S}\AgdaSymbol{\}}\<%
\\
\>[0]\AgdaIndent{2}{}\<[2]%
\>[2]\AgdaSymbol{→} \AgdaRecord{def-Quotient} \AgdaBound{PQ} \AgdaSymbol{→} \AgdaRecord{exact-Quotient} \AgdaBound{PQ}\<%
\\
\>\AgdaFunction{def-Quotient→exact-Quotient} \AgdaSymbol{\{}\AgdaBound{S}\AgdaSymbol{\}} \AgdaSymbol{\{}\AgdaBound{PQ}\AgdaSymbol{\}} \AgdaBound{QuD} \AgdaSymbol{=}\<%
\\
\>[0]\AgdaIndent{2}{}\<[2]%
\>[2]\AgdaKeyword{record} \AgdaSymbol{\{} \AgdaField{Qu} \AgdaSymbol{=} \AgdaFunction{def-Quotient→Quotient} \AgdaBound{QuD}\<%
\\
\>[2]\AgdaIndent{9}{}\<[9]%
\>[9]\AgdaSymbol{;} \AgdaField{exact} \AgdaSymbol{=} \AgdaFunction{exact}\<%
\\
\>[2]\AgdaIndent{9}{}\<[9]%
\>[9]\AgdaSymbol{\}}\<%
\\
\>[0]\AgdaIndent{2}{}\<[2]%
\>[2]\AgdaKeyword{where}\<%
\\
\>[0]\AgdaIndent{4}{}\<[4]%
\>[4]\AgdaKeyword{open} \AgdaModule{pre-Quotient} \AgdaBound{PQ}\<%
\\
\>[0]\AgdaIndent{4}{}\<[4]%
\>[4]\AgdaKeyword{open} \AgdaModule{def-Quotient} \AgdaBound{QuD}\<%
\\
\>\<\end{code}

\begin{code}\>\<%
\\
\>\AgdaFunction{def-Quotient→Hof-Quotient} \<[26]%
\>[26]\<%
\\
\>[0]\AgdaIndent{2}{}\<[2]%
\>[2]\AgdaSymbol{:} \AgdaSymbol{\{}\AgdaBound{S} \AgdaSymbol{:} \AgdaRecord{Setoid}\AgdaSymbol{\}} \<[17]%
\>[17]\<%
\\
\>[0]\AgdaIndent{2}{}\<[2]%
\>[2]\AgdaSymbol{→} \AgdaSymbol{\{}\AgdaBound{PQ} \AgdaSymbol{:} \AgdaRecord{pre-Quotient} \AgdaBound{S}\AgdaSymbol{\}}\<%
\\
\>[0]\AgdaIndent{2}{}\<[2]%
\>[2]\AgdaSymbol{→} \AgdaSymbol{(}\AgdaRecord{def-Quotient} \AgdaBound{PQ}\AgdaSymbol{)} \<[22]%
\>[22]\<%
\\
\>[0]\AgdaIndent{2}{}\<[2]%
\>[2]\AgdaSymbol{→} \AgdaSymbol{(}\AgdaRecord{Hof-Quotient} \AgdaBound{PQ}\AgdaSymbol{)}\<%
\\
\>\AgdaFunction{def-Quotient→Hof-Quotient} \AgdaSymbol{\{}\AgdaBound{S}\AgdaSymbol{\}} \AgdaSymbol{\{}\AgdaBound{PQ}\AgdaSymbol{\}} \AgdaBound{QuD} \AgdaSymbol{=}\<%
\\
\>[0]\AgdaIndent{2}{}\<[2]%
\>[2]\AgdaKeyword{record} \<[9]%
\>[9]\<%
\\
\>[0]\AgdaIndent{2}{}\<[2]%
\>[2]\AgdaSymbol{\{} \AgdaField{lift} \<[11]%
\>[11]\AgdaSymbol{=} \AgdaSymbol{λ} \AgdaBound{f} \AgdaBound{\_} \AgdaSymbol{→} \AgdaBound{f} \AgdaFunction{∘} \AgdaFunction{emb}\<%
\\
\>[0]\AgdaIndent{2}{}\<[2]%
\>[2]\AgdaSymbol{;} \AgdaField{lift-β} \AgdaSymbol{=} \AgdaSymbol{λ} \AgdaBound{resp} \AgdaSymbol{→} \AgdaBound{resp} \AgdaSymbol{(}\AgdaFunction{complete} \AgdaSymbol{\_)}\<%
\\
\>[0]\AgdaIndent{2}{}\<[2]%
\>[2]\AgdaSymbol{;} \AgdaField{qind} \<[11]%
\>[11]\AgdaSymbol{=} \AgdaSymbol{λ} \AgdaBound{P} \AgdaBound{\_} \AgdaBound{f} \AgdaBound{\_} \AgdaSymbol{→} \<[25]%
\>[25]\<%
\\
\>[2]\AgdaIndent{11}{}\<[11]%
\>[11]\AgdaFunction{subst} \AgdaBound{P} \AgdaSymbol{(}\AgdaFunction{stable} \AgdaSymbol{\_)} \AgdaSymbol{(}\AgdaBound{f} \AgdaSymbol{(}\AgdaFunction{emb} \AgdaSymbol{\_))}\<%
\\
\>[0]\AgdaIndent{2}{}\<[2]%
\>[2]\AgdaSymbol{\}}\<%
\\
\>[0]\AgdaIndent{2}{}\<[2]%
\>[2]\AgdaKeyword{where}\<%
\\
\>[2]\AgdaIndent{4}{}\<[4]%
\>[4]\AgdaKeyword{open} \AgdaModule{pre-Quotient} \AgdaBound{PQ}\<%
\\
\>[2]\AgdaIndent{4}{}\<[4]%
\>[4]\AgdaKeyword{open} \AgdaModule{def-Quotient} \AgdaBound{QuD}\<%
\\
\>\<\end{code}


\begin{code}\>\<%
\\
\>\AgdaFunction{def-Quotient→Hof-Quotient'} \AgdaSymbol{:} \<[29]%
\>[29]\<%
\\
\>[0]\AgdaIndent{2}{}\<[2]%
\>[2]\AgdaSymbol{\{}\AgdaBound{S} \AgdaSymbol{:} \AgdaRecord{Setoid}\AgdaSymbol{\}\{}\AgdaBound{PQ} \AgdaSymbol{:} \AgdaRecord{pre-Quotient} \AgdaBound{S}\AgdaSymbol{\}}\<%
\\
\>[0]\AgdaIndent{2}{}\<[2]%
\>[2]\AgdaSymbol{→} \AgdaSymbol{(}\AgdaRecord{def-Quotient} \AgdaBound{PQ}\AgdaSymbol{)} \AgdaSymbol{→} \AgdaSymbol{(}\AgdaRecord{Hof-Quotient} \AgdaBound{PQ}\AgdaSymbol{)}\<%
\\
\>\AgdaFunction{def-Quotient→Hof-Quotient'} \AgdaSymbol{=} \<[29]%
\>[29]\<%
\\
\>[0]\AgdaIndent{2}{}\<[2]%
\>[2]\AgdaFunction{Quotient→Hof-Quotient} \AgdaFunction{∘} \AgdaFunction{def-Quotient→Quotient}\<%
\\
\>\<\end{code}




\textbf{Proof :} The propositional univalence (propositional extensionality) implies that a quotient is always exact.

\AgdaHide{
\begin{code}\>\<%
\\
%
\\
\>\AgdaSymbol{\{-\#} \AgdaKeyword{OPTIONS} --type-in-type \AgdaSymbol{\#-\}}\<%
\\
%
\\
\>\AgdaKeyword{open} \AgdaKeyword{import} \AgdaModule{Setoids}\<%
\\
\>\AgdaKeyword{open} \AgdaKeyword{import} \AgdaModule{Relation.Binary.PropositionalEquality} \AgdaSymbol{as} \AgdaModule{PE}\<%
\\
\>[0]\AgdaIndent{2}{}\<[2]%
\>[2]\AgdaKeyword{hiding} \AgdaSymbol{(}[\_]\AgdaSymbol{)}\<%
\\
\>\AgdaKeyword{open} \AgdaKeyword{import} \AgdaModule{Quotient}\<%
\\
\>\AgdaKeyword{open} \AgdaKeyword{import} \AgdaModule{Data.Product}\<%
\\
%
\\
\>\AgdaKeyword{module} \AgdaModule{PuImpEff}\<%
\\
\>\<\end{code}
}

Assume we have the propositional univalence (the other direction trivial holds)

\begin{code}\>\<%
\\
\>[0]\AgdaIndent{2}{}\<[2]%
\>[2]\AgdaSymbol{(}\AgdaBound{PropUni₁} \AgdaSymbol{:} \AgdaSymbol{∀} \AgdaSymbol{\{}\AgdaBound{p} \AgdaBound{q} \AgdaSymbol{:} \AgdaPrimitiveType{Set}\AgdaSymbol{\}} \AgdaSymbol{→} \AgdaSymbol{(}\AgdaBound{p} \AgdaFunction{⇔} \AgdaBound{q}\AgdaSymbol{)} \AgdaSymbol{→} \AgdaBound{p} \AgdaDatatype{≡} \AgdaBound{q}\AgdaSymbol{)}\<%
\\
\>[0]\AgdaIndent{2}{}\<[2]%
\>[2]\AgdaSymbol{\{}\AgdaBound{S} \AgdaSymbol{:} \AgdaRecord{Setoid}\AgdaSymbol{\}}\<%
\\
\>[0]\AgdaIndent{2}{}\<[2]%
\>[2]\AgdaSymbol{\{}\AgdaBound{PQ} \AgdaSymbol{:} \AgdaRecord{pre-Quotient} \AgdaBound{S}\AgdaSymbol{\}}\<%
\\
\>[0]\AgdaIndent{2}{}\<[2]%
\>[2]\AgdaSymbol{\{}\AgdaBound{Qu} \AgdaSymbol{:} \AgdaRecord{Hof-Quotient} \AgdaBound{PQ}\AgdaSymbol{\}}\<%
\\
\>[2]\AgdaIndent{4}{}\<[4]%
\>[4]\AgdaKeyword{where}\<%
\\
\>[0]\AgdaIndent{2}{}\<[2]%
\>[2]\AgdaKeyword{open} \AgdaModule{pre-Quotient} \AgdaBound{PQ}\<%
\\
\>[0]\AgdaIndent{2}{}\<[2]%
\>[2]\AgdaKeyword{open} \AgdaModule{Hof-Quotient} \AgdaBound{Qu}\<%
\\
%
\\
\>[0]\AgdaIndent{2}{}\<[2]%
\>[2]\AgdaFunction{coerce} \AgdaSymbol{:} \AgdaSymbol{\{}\AgdaBound{A} \AgdaBound{B} \AgdaSymbol{:} \AgdaPrimitiveType{Set}\AgdaSymbol{\}} \AgdaSymbol{→} \AgdaBound{A} \AgdaDatatype{≡} \AgdaBound{B} \AgdaSymbol{→} \AgdaBound{A} \AgdaSymbol{→} \AgdaBound{B}\<%
\\
\>[0]\AgdaIndent{2}{}\<[2]%
\>[2]\AgdaFunction{coerce} \AgdaInductiveConstructor{refl} \AgdaBound{m} \AgdaSymbol{=} \AgdaBound{m}\<%
\\
%
\\
\>[0]\AgdaIndent{2}{}\<[2]%
\>[2]\AgdaFunction{exact} \AgdaSymbol{:} \AgdaSymbol{∀} \AgdaBound{a} \AgdaBound{a'} \AgdaSymbol{→} \AgdaFunction{[} \AgdaBound{a} \AgdaFunction{]} \AgdaDatatype{≡} \AgdaFunction{[} \AgdaBound{a'} \AgdaFunction{]} \AgdaSymbol{→} \AgdaBound{a} \AgdaFunction{\textasciitilde} \AgdaBound{a'}\<%
\\
\>[0]\AgdaIndent{2}{}\<[2]%
\>[2]\AgdaFunction{exact} \AgdaBound{a} \AgdaBound{a'} \AgdaBound{p} \AgdaSymbol{=} \AgdaFunction{coerce} \AgdaFunction{P\textasciicircum-β} \AgdaSymbol{(}\AgdaFunction{\textasciitilde-refl} \AgdaSymbol{\{}\AgdaBound{a}\AgdaSymbol{\})}\<%
\\
\>[2]\AgdaIndent{8}{}\<[8]%
\>[8]\AgdaKeyword{where}\<%
\\
\>[8]\AgdaIndent{10}{}\<[10]%
\>[10]\AgdaFunction{P} \AgdaSymbol{:} \AgdaFunction{A} \AgdaSymbol{→} \AgdaPrimitiveType{Set}\<%
\\
\>[8]\AgdaIndent{10}{}\<[10]%
\>[10]\AgdaFunction{P} \AgdaBound{x} \AgdaSymbol{=} \AgdaBound{a} \AgdaFunction{\textasciitilde} \AgdaBound{x}\<%
\\
%
\\
\>[8]\AgdaIndent{10}{}\<[10]%
\>[10]\AgdaFunction{isEqClass} \AgdaSymbol{:} \AgdaSymbol{∀} \AgdaSymbol{\{}\AgdaBound{a} \AgdaBound{b}\AgdaSymbol{\}} \AgdaSymbol{→} \AgdaBound{a} \AgdaFunction{\textasciitilde} \AgdaBound{b} \AgdaSymbol{→} \AgdaFunction{P} \AgdaBound{a} \AgdaFunction{⇔} \AgdaFunction{P} \AgdaBound{b}\<%
\\
\>[8]\AgdaIndent{10}{}\<[10]%
\>[10]\AgdaFunction{isEqClass} \AgdaBound{p} \AgdaSymbol{=} \AgdaSymbol{(λ} \AgdaBound{q} \AgdaSymbol{→} \AgdaFunction{\textasciitilde-trans} \AgdaBound{q} \AgdaBound{p}\AgdaSymbol{)} \AgdaInductiveConstructor{,} \<[46]%
\>[46]\<%
\\
\>[10]\AgdaIndent{28}{}\<[28]%
\>[28]\AgdaSymbol{(λ} \AgdaBound{q} \AgdaSymbol{→} \AgdaFunction{\textasciitilde-trans} \AgdaBound{q} \AgdaSymbol{(}\AgdaFunction{\textasciitilde-sym} \AgdaBound{p}\AgdaSymbol{))}\<%
\\
%
\\
\>[0]\AgdaIndent{10}{}\<[10]%
\>[10]\AgdaFunction{P-resp} \AgdaSymbol{:} \AgdaFunction{P} \AgdaFunction{respects} \AgdaFunction{\_\textasciitilde\_}\<%
\\
\>[0]\AgdaIndent{10}{}\<[10]%
\>[10]\AgdaFunction{P-resp} \AgdaBound{p} \AgdaSymbol{=} \AgdaBound{PropUni₁} \AgdaSymbol{(}\AgdaFunction{isEqClass} \AgdaBound{p}\AgdaSymbol{)}\<%
\\
%
\\
\>[0]\AgdaIndent{10}{}\<[10]%
\>[10]\AgdaFunction{P\textasciicircum} \AgdaSymbol{:} \AgdaFunction{Q} \AgdaSymbol{→} \AgdaPrimitiveType{Set}\<%
\\
\>[0]\AgdaIndent{10}{}\<[10]%
\>[10]\AgdaFunction{P\textasciicircum} \AgdaSymbol{=} \AgdaFunction{lift} \AgdaFunction{P} \AgdaFunction{P-resp}\<%
\\
%
\\
\>[0]\AgdaIndent{10}{}\<[10]%
\>[10]\AgdaFunction{P\textasciicircum-β} \AgdaSymbol{:} \AgdaFunction{P} \AgdaBound{a} \AgdaDatatype{≡} \AgdaFunction{P} \AgdaBound{a'}\<%
\\
\>[0]\AgdaIndent{10}{}\<[10]%
\>[10]\AgdaFunction{P\textasciicircum-β} \AgdaSymbol{=} \AgdaFunction{trans} \AgdaSymbol{(}\AgdaFunction{sym} \AgdaSymbol{(}\AgdaFunction{lift-β} \AgdaSymbol{\_))} \<[40]%
\>[40]\<%
\\
\>[10]\AgdaIndent{17}{}\<[17]%
\>[17]\AgdaSymbol{(}\AgdaFunction{trans} \AgdaSymbol{(}\AgdaFunction{cong} \AgdaFunction{P\textasciicircum} \AgdaBound{p}\AgdaSymbol{)} \AgdaSymbol{(}\AgdaFunction{lift-β} \AgdaSymbol{\_))}\<%
\\
\>\<\end{code}


\AgdaHide{
\begin{code}\>\<%
\\
%
\\
\>\AgdaKeyword{module} \AgdaModule{QInteger} \AgdaKeyword{where}\<%
\\
%
\\
%
\\
\>\AgdaKeyword{open} \AgdaKeyword{import} \AgdaModule{Data.Nat}\<%
\\
\>\AgdaKeyword{open} \AgdaKeyword{import} \AgdaModule{Data.Product}\<%
\\
\>\AgdaKeyword{open} \AgdaKeyword{import} \AgdaModule{Function}\<%
\\
\>\AgdaKeyword{open} \AgdaKeyword{import} \AgdaModule{Data.Nat.Properties}\<%
\\
\>\AgdaKeyword{open} \AgdaKeyword{import} \AgdaModule{Nat.Properties+}\<%
\\
\>\AgdaKeyword{open} \AgdaKeyword{import} \AgdaModule{Relation.Binary.PropositionalEquality} \AgdaSymbol{as} \AgdaModule{PE} \AgdaKeyword{hiding} \AgdaSymbol{(}[\_]\AgdaSymbol{)}\<%
\\
%
\\
\>\AgdaKeyword{open} \AgdaKeyword{import} \AgdaModule{Relation.Binary} \AgdaKeyword{hiding} \AgdaSymbol{(}Setoid\AgdaSymbol{)}\<%
\\
%
\\
\>\AgdaKeyword{open} \AgdaKeyword{import} \AgdaModule{Symbols}\<%
\\
\>\AgdaKeyword{open} \AgdaKeyword{import} \AgdaModule{Setoids}\<%
\\
\>\AgdaKeyword{open} \AgdaKeyword{import} \AgdaModule{Quotient}\<%
\\
\>\<\end{code}
}

\textbf{Setoid Integer}

Base set

\begin{code}\>\<%
\\
\>\AgdaKeyword{infix} \AgdaNumber{4} \_,\_\<%
\\
%
\\
\>\AgdaKeyword{data} \AgdaDatatype{ℤ₀} \AgdaSymbol{:} \AgdaPrimitiveType{Set} \AgdaKeyword{where}\<%
\\
\>[0]\AgdaIndent{2}{}\<[2]%
\>[2]\AgdaInductiveConstructor{\_,\_} \AgdaSymbol{:} \AgdaDatatype{ℕ} \AgdaSymbol{→} \AgdaDatatype{ℕ} \AgdaSymbol{→} \AgdaDatatype{ℤ₀}\<%
\\
\>\<\end{code}

Equivalence relation

\begin{code}\>\<%
\\
\>\AgdaKeyword{infixl} \AgdaNumber{2} \_∼\_\<%
\\
%
\\
\>\AgdaFunction{\_∼\_} \AgdaSymbol{:} \AgdaDatatype{ℤ₀} \AgdaSymbol{→} \AgdaDatatype{ℤ₀} \AgdaSymbol{→} \AgdaPrimitiveType{Set}\<%
\\
\>\AgdaSymbol{(}\AgdaBound{x+} \AgdaInductiveConstructor{,} \AgdaBound{x-}\AgdaSymbol{)} \AgdaFunction{∼} \AgdaSymbol{(}\AgdaBound{y+} \AgdaInductiveConstructor{,} \AgdaBound{y-}\AgdaSymbol{)} \AgdaSymbol{=} \AgdaSymbol{(}\AgdaBound{x+} \AgdaPrimitive{+} \AgdaBound{y-}\AgdaSymbol{)} \AgdaDatatype{≡} \AgdaSymbol{(}\AgdaBound{y+} \AgdaPrimitive{+} \AgdaBound{x-}\AgdaSymbol{)}\<%
\\
\>\<\end{code}

Equivalence properties

\begin{code}\>\<%
\\
\>\AgdaFunction{∼refl} \AgdaSymbol{:} \AgdaSymbol{∀} \AgdaSymbol{\{}\AgdaBound{a}\AgdaSymbol{\}} \AgdaSymbol{→} \AgdaBound{a} \AgdaFunction{∼} \AgdaBound{a}\<%
\\
\>\AgdaFunction{∼refl} \AgdaSymbol{\{}\AgdaBound{x+} \AgdaInductiveConstructor{,} \AgdaBound{x-}\AgdaSymbol{\}} \AgdaSymbol{=} \AgdaInductiveConstructor{refl}\<%
\\
%
\\
\>\AgdaFunction{∼sym} \AgdaSymbol{:} \AgdaSymbol{∀} \AgdaSymbol{\{}\AgdaBound{a} \AgdaBound{b}\AgdaSymbol{\}} \AgdaSymbol{→} \AgdaBound{a} \AgdaFunction{∼} \AgdaBound{b} \AgdaSymbol{→} \AgdaBound{b} \AgdaFunction{∼} \AgdaBound{a}\<%
\\
\>\AgdaFunction{∼sym} \AgdaSymbol{\{}\AgdaBound{x+} \AgdaInductiveConstructor{,} \AgdaBound{x-}\AgdaSymbol{\}} \AgdaSymbol{\{}\AgdaBound{y+} \AgdaInductiveConstructor{,} \AgdaBound{y-}\AgdaSymbol{\}} \AgdaSymbol{=} \AgdaFunction{sym}\<%
\\
%
\\
\>\AgdaFunction{∼trans} \AgdaSymbol{:} \<[10]%
\>[10]\AgdaSymbol{∀} \AgdaSymbol{\{}\AgdaBound{a} \AgdaBound{b} \AgdaBound{c}\AgdaSymbol{\}} \AgdaSymbol{→} \AgdaBound{a} \AgdaFunction{∼} \AgdaBound{b} \AgdaSymbol{→} \AgdaBound{b} \AgdaFunction{∼} \AgdaBound{c} \AgdaSymbol{→} \AgdaBound{a} \AgdaFunction{∼} \AgdaBound{c}\<%
\\
\>\AgdaFunction{∼trans} \AgdaSymbol{\{}\AgdaBound{x+} \AgdaInductiveConstructor{,} \AgdaBound{x-}\AgdaSymbol{\}} \AgdaSymbol{\{}\AgdaBound{y+} \AgdaInductiveConstructor{,} \AgdaBound{y-}\AgdaSymbol{\}} \AgdaSymbol{\{}\AgdaBound{z+} \AgdaInductiveConstructor{,} \AgdaBound{z-}\AgdaSymbol{\}} \AgdaBound{x=y} \AgdaBound{y=z} \AgdaSymbol{=} \<[47]%
\>[47]\<%
\\
\>[2]\AgdaIndent{7}{}\<[7]%
\>[7]\AgdaFunction{cancel-+-left} \AgdaSymbol{(}\AgdaBound{y+} \AgdaPrimitive{+} \AgdaBound{y-}\AgdaSymbol{)} \<[31]%
\>[31]\<%
\\
\>[2]\AgdaIndent{7}{}\<[7]%
\>[7]\AgdaSymbol{(}\AgdaFunction{swap24} \AgdaBound{y+} \AgdaBound{y-} \AgdaBound{x+} \AgdaBound{z-} \<[27]%
\>[27]\<%
\\
\>[2]\AgdaIndent{7}{}\<[7]%
\>[7]\AgdaFunction{>≡<} \AgdaSymbol{((}\AgdaBound{y=z} \AgdaFunction{+=} \AgdaBound{x=y}\AgdaSymbol{)} \AgdaFunction{>≡<} \AgdaFunction{swap13} \AgdaBound{z+} \AgdaBound{y-} \AgdaBound{y+} \AgdaBound{x-}\AgdaSymbol{))}\<%
\\
%
\\
\>\AgdaFunction{\_∼\_isEquivalence} \AgdaSymbol{:} \AgdaRecord{IsEquivalence} \AgdaFunction{\_∼\_}\<%
\\
\>\AgdaFunction{\_∼\_isEquivalence} \AgdaSymbol{=} \AgdaKeyword{record}\<%
\\
\>[0]\AgdaIndent{2}{}\<[2]%
\>[2]\AgdaSymbol{\{} \AgdaField{refl} \<[10]%
\>[10]\AgdaSymbol{=} \AgdaFunction{∼refl}\<%
\\
\>[0]\AgdaIndent{2}{}\<[2]%
\>[2]\AgdaSymbol{;} \AgdaField{sym} \<[10]%
\>[10]\AgdaSymbol{=} \AgdaFunction{∼sym}\<%
\\
\>[0]\AgdaIndent{2}{}\<[2]%
\>[2]\AgdaSymbol{;} \AgdaField{trans} \AgdaSymbol{=} \AgdaFunction{∼trans}\<%
\\
\>[0]\AgdaIndent{2}{}\<[2]%
\>[2]\AgdaSymbol{\}}\<%
\\
\>\<\end{code}

(ℤ₀, ∼) is a setoid

\begin{code}\>\<%
\\
\>\AgdaFunction{ℤ-Setoid} \AgdaSymbol{:} \AgdaRecord{Setoid}\<%
\\
\>\AgdaFunction{ℤ-Setoid} \AgdaSymbol{=} \AgdaKeyword{record}\<%
\\
\>[0]\AgdaIndent{2}{}\<[2]%
\>[2]\AgdaSymbol{\{} \AgdaField{Carrier} \<[18]%
\>[18]\AgdaSymbol{=} \AgdaDatatype{ℤ₀}\<%
\\
\>[0]\AgdaIndent{2}{}\<[2]%
\>[2]\AgdaSymbol{;} \AgdaField{\_\textasciitilde\_} \<[18]%
\>[18]\AgdaSymbol{=} \AgdaFunction{\_∼\_}\<%
\\
\>[0]\AgdaIndent{2}{}\<[2]%
\>[2]\AgdaSymbol{;} \AgdaField{isEquivalence} \AgdaSymbol{=} \AgdaFunction{\_∼\_isEquivalence}\<%
\\
\>[0]\AgdaIndent{2}{}\<[2]%
\>[2]\AgdaSymbol{\}}\<%
\\
\>\<\end{code}


Definition of ℤ

\begin{code}\>\<%
\\
\>\AgdaKeyword{data} \AgdaDatatype{ℤ} \AgdaSymbol{:} \AgdaPrimitiveType{Set} \AgdaKeyword{where}\<%
\\
\>[0]\AgdaIndent{2}{}\<[2]%
\>[2]\AgdaInductiveConstructor{+\_} \<[8]%
\>[8]\AgdaSymbol{:} \AgdaSymbol{(}\AgdaBound{n} \AgdaSymbol{:} \AgdaDatatype{ℕ}\AgdaSymbol{)} \AgdaSymbol{→} \AgdaDatatype{ℤ}\<%
\\
\>[0]\AgdaIndent{2}{}\<[2]%
\>[2]\AgdaInductiveConstructor{-suc\_} \AgdaSymbol{:} \AgdaSymbol{(}\AgdaBound{n} \AgdaSymbol{:} \AgdaDatatype{ℕ}\AgdaSymbol{)} \AgdaSymbol{→} \AgdaDatatype{ℤ}\<%
\\
%
\\
\>\<\end{code}

Normalisation function

\begin{code}\>\<%
\\
\>\AgdaFunction{[\_]} \<[22]%
\>[22]\AgdaSymbol{:} \AgdaDatatype{ℤ₀} \AgdaSymbol{→} \AgdaDatatype{ℤ}\<%
\\
\>\AgdaFunction{[} \AgdaBound{m} \AgdaInductiveConstructor{,} \AgdaInductiveConstructor{0} \AgdaFunction{]} \<[22]%
\>[22]\AgdaSymbol{=} \AgdaInductiveConstructor{+} \AgdaBound{m}\<%
\\
\>\AgdaFunction{[} \AgdaInductiveConstructor{0} \AgdaInductiveConstructor{,} \AgdaInductiveConstructor{suc} \AgdaBound{n} \AgdaFunction{]} \<[20]%
\>[20]\AgdaSymbol{=} \AgdaInductiveConstructor{-suc} \AgdaBound{n}\<%
\\
\>\AgdaFunction{[} \AgdaInductiveConstructor{suc} \AgdaBound{m} \AgdaInductiveConstructor{,} \AgdaInductiveConstructor{suc} \AgdaBound{n} \AgdaFunction{]} \AgdaSymbol{=} \AgdaFunction{[} \AgdaBound{m} \AgdaInductiveConstructor{,} \AgdaBound{n} \AgdaFunction{]}\<%
\\
\>\<\end{code}

Embedding function

\begin{code}\>\<%
\\
\>\AgdaFunction{⌜\_⌝} \<[11]%
\>[11]\AgdaSymbol{:} \AgdaDatatype{ℤ} \AgdaSymbol{→} \AgdaDatatype{ℤ₀}\<%
\\
\>\AgdaFunction{⌜} \AgdaInductiveConstructor{+} \AgdaBound{n} \AgdaFunction{⌝} \<[11]%
\>[11]\AgdaSymbol{=} \AgdaBound{n} \AgdaInductiveConstructor{,} \AgdaNumber{0}\<%
\\
\>\AgdaFunction{⌜} \AgdaInductiveConstructor{-suc} \AgdaBound{n} \AgdaFunction{⌝} \AgdaSymbol{=} \AgdaNumber{0} \AgdaInductiveConstructor{,} \AgdaInductiveConstructor{ℕ.suc} \AgdaBound{n}\<%
\\
\>\<\end{code}

Stability

\begin{code}\>\<%
\\
%
\\
\>\AgdaFunction{stable} \<[18]%
\>[18]\AgdaSymbol{:} \AgdaSymbol{∀} \AgdaSymbol{\{}\AgdaBound{n}\AgdaSymbol{\}} \AgdaSymbol{→} \AgdaFunction{[} \AgdaFunction{⌜} \AgdaBound{n} \AgdaFunction{⌝} \AgdaFunction{]} \AgdaDatatype{≡} \AgdaBound{n}\<%
\\
\>\AgdaFunction{stable} \AgdaSymbol{\{}\AgdaInductiveConstructor{+} \AgdaBound{n}\AgdaSymbol{\}} \<[18]%
\>[18]\AgdaSymbol{=} \AgdaInductiveConstructor{refl}\<%
\\
\>\AgdaFunction{stable} \AgdaSymbol{\{} \AgdaInductiveConstructor{-suc} \AgdaBound{n} \AgdaSymbol{\}} \AgdaSymbol{=} \AgdaInductiveConstructor{refl}\<%
\\
%
\\
\>\<\end{code}

Completeness

\begin{code}\>\<%
\\
%
\\
\>\AgdaFunction{compl} \AgdaSymbol{:} \AgdaSymbol{∀} \AgdaBound{n} \AgdaSymbol{→} \AgdaFunction{⌜} \AgdaFunction{[} \AgdaBound{n} \AgdaFunction{]} \AgdaFunction{⌝} \AgdaFunction{∼} \AgdaBound{n}\<%
\\
\>\AgdaFunction{compl} \AgdaSymbol{(}\AgdaBound{x} \AgdaInductiveConstructor{,} \AgdaInductiveConstructor{0}\AgdaSymbol{)} \<[22]%
\>[22]\AgdaSymbol{=} \AgdaInductiveConstructor{refl}\<%
\\
\>\AgdaFunction{compl} \AgdaSymbol{(}\AgdaInductiveConstructor{0} \AgdaInductiveConstructor{,} \AgdaInductiveConstructor{suc} \AgdaBound{y}\AgdaSymbol{)} \<[22]%
\>[22]\AgdaSymbol{=} \AgdaInductiveConstructor{refl}\<%
\\
\>\AgdaFunction{compl} \AgdaSymbol{(}\AgdaInductiveConstructor{suc} \AgdaBound{x} \AgdaInductiveConstructor{,} \AgdaInductiveConstructor{suc} \AgdaBound{y}\AgdaSymbol{)} \AgdaSymbol{=} \AgdaFunction{∼trans} \AgdaSymbol{(}\AgdaFunction{compl} \AgdaSymbol{(}\AgdaBound{x} \AgdaInductiveConstructor{,} \AgdaBound{y}\AgdaSymbol{))} \<[47]%
\>[47]\<%
\\
\>[2]\AgdaIndent{26}{}\<[26]%
\>[26]\AgdaSymbol{(}\AgdaFunction{sym} \AgdaSymbol{(}\AgdaFunction{sm+n≡m+sn} \AgdaBound{x}\AgdaSymbol{))}\<%
\\
%
\\
%
\\
\>\AgdaFunction{sound'} \AgdaSymbol{:} \AgdaSymbol{∀} \AgdaSymbol{\{}\AgdaBound{i} \AgdaBound{j}\AgdaSymbol{\}} \AgdaSymbol{→} \AgdaFunction{⌜} \AgdaBound{i} \AgdaFunction{⌝} \AgdaFunction{∼} \AgdaFunction{⌜} \AgdaBound{j} \AgdaFunction{⌝} \<[34]%
\>[34]\AgdaSymbol{→} \AgdaBound{i} \AgdaDatatype{≡} \AgdaBound{j}\<%
\\
\>\AgdaFunction{sound'} \<[8]%
\>[8]\AgdaSymbol{\{}\AgdaInductiveConstructor{+} \AgdaBound{i}\AgdaSymbol{\}} \AgdaSymbol{\{}\AgdaInductiveConstructor{+} \AgdaBound{j}\AgdaSymbol{\}} \AgdaBound{eqt} \<[29]%
\>[29]\AgdaSymbol{=} \AgdaInductiveConstructor{+\_} \AgdaFunction{⋆} \AgdaSymbol{(}\AgdaFunction{+r-cancel} \AgdaNumber{0} \AgdaBound{eqt}\AgdaSymbol{)}\<%
\\
\>\AgdaFunction{sound'} \<[8]%
\>[8]\AgdaSymbol{\{}\AgdaInductiveConstructor{+} \AgdaBound{i}\AgdaSymbol{\}} \AgdaSymbol{\{} \AgdaInductiveConstructor{-suc} \AgdaBound{j} \AgdaSymbol{\}} \AgdaBound{eqt} \AgdaKeyword{with} \AgdaBound{i} \AgdaFunction{+suc} \AgdaBound{j} \AgdaFunction{≢0} \AgdaBound{eqt}\<%
\\
\>\AgdaSymbol{...} \AgdaSymbol{|} \AgdaSymbol{()}\<%
\\
\>\AgdaFunction{sound'} \<[8]%
\>[8]\AgdaSymbol{\{} \AgdaInductiveConstructor{-suc} \AgdaBound{i} \AgdaSymbol{\}} \AgdaSymbol{\{} \AgdaInductiveConstructor{+} \AgdaBound{j} \AgdaSymbol{\}} \AgdaBound{eqt} \AgdaKeyword{with} \AgdaBound{j} \AgdaFunction{+suc} \AgdaBound{i} \AgdaFunction{≢0} \AgdaFunction{⟨} \AgdaBound{eqt} \AgdaFunction{⟩}\<%
\\
\>\AgdaSymbol{...} \AgdaSymbol{|} \AgdaSymbol{()}\<%
\\
\>\AgdaFunction{sound'} \<[8]%
\>[8]\AgdaSymbol{\{} \AgdaInductiveConstructor{-suc} \AgdaBound{i} \AgdaSymbol{\}} \AgdaSymbol{\{} \AgdaInductiveConstructor{-suc} \AgdaBound{j} \AgdaSymbol{\}} \AgdaBound{eqt} \AgdaSymbol{=} \AgdaInductiveConstructor{-suc\_} \AgdaFunction{⋆} \AgdaFunction{pred} \AgdaFunction{⋆} \AgdaFunction{⟨} \AgdaBound{eqt} \AgdaFunction{⟩}\<%
\\
\>\<\end{code}

Soundness

\begin{code}\>\<%
\\
\>\AgdaFunction{sound} \AgdaSymbol{:} \AgdaSymbol{∀} \AgdaSymbol{\{}\AgdaBound{x} \AgdaBound{y}\AgdaSymbol{\}} \AgdaSymbol{→} \AgdaBound{x} \AgdaFunction{∼} \AgdaBound{y} \AgdaSymbol{→} \AgdaFunction{[} \AgdaBound{x} \AgdaFunction{]} \AgdaDatatype{≡} \AgdaFunction{[} \AgdaBound{y} \AgdaFunction{]}\<%
\\
\>\AgdaFunction{sound} \AgdaSymbol{\{} \AgdaBound{x} \AgdaSymbol{\}} \AgdaSymbol{\{} \AgdaBound{y} \AgdaSymbol{\}} \AgdaBound{x∼y} \AgdaSymbol{=} \AgdaFunction{sound'} \AgdaSymbol{(}\AgdaFunction{∼trans} \AgdaSymbol{(}\AgdaFunction{compl} \AgdaSymbol{\_)} \<[49]%
\>[49]\<%
\\
\>[18]\AgdaIndent{6}{}\<[6]%
\>[6]\AgdaSymbol{(}\AgdaFunction{∼trans} \AgdaSymbol{(}\AgdaBound{x∼y}\AgdaSymbol{)} \AgdaSymbol{(}\AgdaFunction{∼sym} \AgdaSymbol{(}\AgdaFunction{compl} \AgdaSymbol{\_))))} \<[39]%
\>[39]\<%
\\
\>\<\end{code}

The quotient definitions for ℤ

\begin{code}\>\<%
\\
\>\AgdaFunction{ℤ-PreQu} \AgdaSymbol{:} \AgdaRecord{pre-Quotient} \AgdaFunction{ℤ-Setoid}\<%
\\
\>\AgdaFunction{ℤ-PreQu} \AgdaSymbol{=} \AgdaKeyword{record}\<%
\\
\>[0]\AgdaIndent{2}{}\<[2]%
\>[2]\AgdaSymbol{\{} \AgdaField{Q} \<[12]%
\>[12]\AgdaSymbol{=} \AgdaDatatype{ℤ}\<%
\\
\>[0]\AgdaIndent{2}{}\<[2]%
\>[2]\AgdaSymbol{;} \AgdaField{[\_]} \<[12]%
\>[12]\AgdaSymbol{=} \<[15]%
\>[15]\AgdaFunction{[\_]}\<%
\\
\>[0]\AgdaIndent{2}{}\<[2]%
\>[2]\AgdaSymbol{;} \AgdaField{[\_]⁼} \<[11]%
\>[11]\AgdaSymbol{=} \AgdaFunction{sound}\<%
\\
\>[0]\AgdaIndent{2}{}\<[2]%
\>[2]\AgdaSymbol{\}}\<%
\\
%
\\
\>\AgdaFunction{ℤ-QuD} \AgdaSymbol{:} \AgdaRecord{def-Quotient} \AgdaFunction{ℤ-PreQu}\<%
\\
\>\AgdaFunction{ℤ-QuD} \AgdaSymbol{=} \AgdaKeyword{record}\<%
\\
\>[0]\AgdaIndent{2}{}\<[2]%
\>[2]\AgdaSymbol{\{} \AgdaField{emb} \<[14]%
\>[14]\AgdaSymbol{=} \AgdaFunction{⌜\_⌝}\<%
\\
\>[0]\AgdaIndent{2}{}\<[2]%
\>[2]\AgdaSymbol{;} \AgdaField{complete} \<[14]%
\>[14]\AgdaSymbol{=} \AgdaSymbol{λ} \AgdaBound{z} \AgdaSymbol{→} \AgdaFunction{compl} \AgdaSymbol{\_}\<%
\\
\>[0]\AgdaIndent{2}{}\<[2]%
\>[2]\AgdaSymbol{;} \AgdaField{stable} \<[14]%
\>[14]\AgdaSymbol{=} \AgdaSymbol{λ} \AgdaBound{z} \AgdaSymbol{→} \AgdaFunction{stable}\<%
\\
\>[0]\AgdaIndent{2}{}\<[2]%
\>[2]\AgdaSymbol{\}}\<%
\\
%
\\
\>\AgdaFunction{ℤ-Qu} \AgdaSymbol{=} \AgdaFunction{def-Quotient→Quotient} \AgdaFunction{ℤ-QuD}\<%
\\
\>\<\end{code}
\label{defQInt}

\AgdaHide{
\begin{code}\>\<%
\\
%
\\
\>\AgdaKeyword{module} \AgdaModule{Rational} \AgdaKeyword{where}\<%
\\
%
\\
\>\AgdaKeyword{import} \AgdaModule{Data.Rational} \AgdaSymbol{as} \AgdaModule{Rt}\<%
\\
\>\AgdaComment{-- open import Data.Rational' as Rt₀ hiding (-\_ ; \_÷\_ ; ∣\_∣)}\<%
\\
\>\AgdaKeyword{open} \AgdaKeyword{import} \AgdaModule{Data.Product}\<%
\\
\>\AgdaKeyword{open} \AgdaKeyword{import} \AgdaModule{Data.Integer} \AgdaKeyword{hiding} \AgdaSymbol{(}suc\AgdaSymbol{;} pred\AgdaSymbol{)} \AgdaKeyword{renaming} \AgdaSymbol{(}-[1+\_] \AgdaSymbol{to} -suc\_\AgdaSymbol{;} \_*\_ \AgdaSymbol{to} \_ℤ*\_\AgdaSymbol{;}-\_ \AgdaSymbol{to} ℤ-\_\AgdaSymbol{)}\<%
\\
\>\AgdaKeyword{open} \AgdaKeyword{import} \AgdaModule{QInteger} \<[22]%
\>[22]\AgdaKeyword{hiding} \AgdaSymbol{(}\_∼\_\AgdaSymbol{;} [\_]\AgdaSymbol{;} ⌜\_⌝\AgdaSymbol{;} sound \AgdaSymbol{;} stable\AgdaSymbol{)} \AgdaComment{-- as ℤ' hiding (-\_ ; suc ; [\_] ; ∣\_∣; \_◃\_ ; pred ; ⌜\_⌝) renaming (p to ∣\_∣')}\<%
\\
\>\AgdaComment{-- import Data.Integer.Properties' as ℤ'}\<%
\\
\>\AgdaComment{-- import Data.Integer.Setoid as ℤ₀}\<%
\\
\>\AgdaKeyword{open} \AgdaKeyword{import} \AgdaModule{Data.Nat} \AgdaSymbol{as} \AgdaModule{ℕ} \AgdaKeyword{renaming} \AgdaSymbol{(}\_+\_ \AgdaSymbol{to} \_ℕ+\_ \AgdaSymbol{;} \_*\_ \AgdaSymbol{to} \_ℕ*\_\AgdaSymbol{)}\<%
\\
\>\AgdaKeyword{open} \AgdaKeyword{import} \AgdaModule{Data.Nat.GCD}\<%
\\
\>\AgdaKeyword{open} \AgdaKeyword{import} \AgdaModule{Data.Nat.Divisibility} \AgdaKeyword{using} \AgdaSymbol{(}\_∣\_ \AgdaSymbol{;} 1∣\_ \AgdaSymbol{;} divides\AgdaSymbol{)}\<%
\\
\>\AgdaKeyword{open} \AgdaKeyword{import} \AgdaModule{Data.Nat.Coprimality}\<%
\\
\>\AgdaKeyword{open} \AgdaKeyword{import} \AgdaModule{Data.Unit}\<%
\\
\>\AgdaKeyword{open} \AgdaKeyword{import} \AgdaModule{Relation.Binary.PropositionalEquality} \AgdaKeyword{hiding} \AgdaSymbol{(}[\_]\AgdaSymbol{)}\<%
\\
\>\AgdaKeyword{open} \AgdaKeyword{import} \AgdaModule{Relation.Nullary.Decidable}\<%
\\
\>\<\end{code}
}

\section{Rational numbers}

\begin{code}\>\<%
\\
\>\AgdaKeyword{data} \AgdaDatatype{ℚ₀} \AgdaSymbol{:} \AgdaPrimitiveType{Set} \AgdaKeyword{where}\<%
\\
\>[0]\AgdaIndent{2}{}\<[2]%
\>[2]\AgdaInductiveConstructor{\_/suc\_} \AgdaSymbol{:} \AgdaSymbol{(}\AgdaBound{n} \AgdaSymbol{:} \AgdaDatatype{ℤ}\AgdaSymbol{)} \AgdaSymbol{→} \AgdaSymbol{(}\AgdaBound{d} \AgdaSymbol{:} \AgdaDatatype{ℕ}\AgdaSymbol{)} \AgdaSymbol{→} \AgdaDatatype{ℚ₀}\<%
\\
\>\<\end{code}

\textbf{Extractions}

\begin{code}\>\<%
\\
\>\AgdaFunction{num} \AgdaSymbol{:} \AgdaDatatype{ℚ₀} \AgdaSymbol{→} \AgdaDatatype{ℤ}\<%
\\
\>\AgdaFunction{num} \AgdaSymbol{(}\AgdaBound{n} \AgdaInductiveConstructor{/suc} \AgdaSymbol{\_)} \AgdaSymbol{=} \AgdaBound{n}\<%
\\
%
\\
\>\AgdaFunction{den} \AgdaSymbol{:} \AgdaDatatype{ℚ₀} \AgdaSymbol{→} \AgdaDatatype{ℕ}\<%
\\
\>\AgdaFunction{den} \AgdaSymbol{(\_} \AgdaInductiveConstructor{/suc} \AgdaBound{d}\AgdaSymbol{)} \AgdaSymbol{=} \AgdaInductiveConstructor{suc} \AgdaBound{d}\<%
\\
\>\<\end{code}

\textbf{Equivalence relation}

\begin{code}\>\<%
\\
\>\AgdaKeyword{infixl} \AgdaNumber{2} \_∼\_\<%
\\
%
\\
\>\AgdaFunction{\_∼\_} \<[6]%
\>[6]\AgdaSymbol{:} \AgdaDatatype{ℚ₀} \AgdaSymbol{→} \AgdaDatatype{ℚ₀} \AgdaSymbol{→} \AgdaPrimitiveType{Set}\<%
\\
\>\AgdaBound{n1} \AgdaInductiveConstructor{/suc} \AgdaBound{d1} \AgdaFunction{∼} \AgdaBound{n2} \AgdaInductiveConstructor{/suc} \AgdaBound{d2} \AgdaSymbol{=} \<[27]%
\>[27]\AgdaBound{n1} \AgdaFunction{ℤ*} \AgdaSymbol{(}\AgdaInductiveConstructor{+} \AgdaInductiveConstructor{suc} \AgdaBound{d2}\AgdaSymbol{)} \AgdaDatatype{≡} \AgdaBound{n2} \AgdaFunction{ℤ*} \AgdaSymbol{(}\AgdaInductiveConstructor{+} \AgdaInductiveConstructor{suc} \AgdaBound{d1}\AgdaSymbol{)}\<%
\\
\>\<\end{code}

Property: a fraction is reduced

i.e. the absolute value of the numerator is comprime to the denominator

\begin{code}\>\<%
\\
\>\AgdaFunction{IsReduced} \AgdaSymbol{:} \AgdaDatatype{ℚ₀} \AgdaSymbol{→} \AgdaPrimitiveType{Set}\<%
\\
\>\AgdaFunction{IsReduced} \AgdaSymbol{(}\AgdaBound{n} \AgdaInductiveConstructor{/suc} \AgdaBound{d}\AgdaSymbol{)} \AgdaSymbol{=} \AgdaFunction{True} \AgdaSymbol{(}\AgdaFunction{coprime?} \AgdaFunction{∣} \AgdaBound{n} \AgdaFunction{∣} \AgdaSymbol{(}\AgdaInductiveConstructor{suc} \AgdaBound{d}\AgdaSymbol{))}\<%
\\
\>\<\end{code}

The Definition of $\Q$ which is equivalent to the one in standard library

\begin{code}\>\<%
\\
\>\AgdaFunction{ℚ} \AgdaSymbol{:} \AgdaPrimitiveType{Set}\<%
\\
\>\AgdaFunction{ℚ} \AgdaSymbol{=} \AgdaRecord{Σ[} \AgdaBound{q} \AgdaRecord{∶} \AgdaDatatype{ℚ₀} \AgdaRecord{]} \AgdaFunction{IsReduced} \AgdaBound{q}\<%
\\
\>\<\end{code}

\textbf{Normalisation function}:

1. Calculate a reduced fraction for $\frac{x}{y}$ with a condition that y is not zero.

\begin{code}\>\<%
\\
\>\AgdaFunction{calℚ} \AgdaSymbol{:} \AgdaSymbol{∀(}\AgdaBound{x} \AgdaBound{y} \AgdaSymbol{:} \AgdaDatatype{ℕ}\AgdaSymbol{)} \AgdaSymbol{→} \AgdaBound{y} \AgdaFunction{≢} \AgdaNumber{0} \AgdaSymbol{→} \AgdaFunction{ℚ}\<%
\\
\>\AgdaFunction{calℚ} \AgdaBound{x} \AgdaBound{y} \AgdaBound{neo} \AgdaKeyword{with} \AgdaFunction{gcd′} \AgdaBound{x} \AgdaBound{y}\<%
\\
\>\AgdaFunction{calℚ} \AgdaSymbol{.(}\AgdaBound{q₁} \AgdaPrimitive{ℕ*} \AgdaBound{di}\AgdaSymbol{)} \AgdaSymbol{.(}\AgdaBound{q₂} \AgdaPrimitive{ℕ*} \AgdaBound{di}\AgdaSymbol{)} \AgdaBound{neo} \<[33]%
\>[33]\<%
\\
\>[0]\AgdaIndent{2}{}\<[2]%
\>[2]\AgdaSymbol{|} \AgdaBound{di} \AgdaInductiveConstructor{,} \AgdaInductiveConstructor{gcd-*} \AgdaBound{q₁} \AgdaBound{q₂} \AgdaBound{c} \AgdaSymbol{=} \AgdaSymbol{(}\AgdaFunction{numr} \AgdaInductiveConstructor{/suc} \AgdaFunction{pred} \AgdaBound{q₂}\AgdaSymbol{)} \AgdaInductiveConstructor{,} \AgdaFunction{iscoprime}\<%
\\
\>[2]\AgdaIndent{3}{}\<[3]%
\>[3]\AgdaKeyword{where}\<%
\\
\>[3]\AgdaIndent{5}{}\<[5]%
\>[5]\AgdaFunction{numr} \AgdaSymbol{=} \AgdaInductiveConstructor{+} \AgdaBound{q₁}\<%
\\
\>[3]\AgdaIndent{5}{}\<[5]%
\>[5]\AgdaFunction{deno} \AgdaSymbol{=} \AgdaInductiveConstructor{suc} \AgdaSymbol{(}\AgdaFunction{pred} \AgdaBound{q₂}\AgdaSymbol{)}\<%
\\
%
\\
\>[3]\AgdaIndent{5}{}\<[5]%
\>[5]\AgdaFunction{lzero} \AgdaSymbol{:} \AgdaSymbol{∀} \AgdaBound{x} \AgdaBound{y} \AgdaSymbol{→} \AgdaBound{x} \AgdaDatatype{≡} \AgdaNumber{0} \AgdaSymbol{→} \AgdaBound{x} \AgdaPrimitive{ℕ*} \AgdaBound{y} \AgdaDatatype{≡} \AgdaNumber{0}\<%
\\
\>[3]\AgdaIndent{5}{}\<[5]%
\>[5]\AgdaFunction{lzero} \AgdaSymbol{.}\AgdaNumber{0} \AgdaBound{y} \AgdaInductiveConstructor{refl} \AgdaSymbol{=} \AgdaInductiveConstructor{refl}\<%
\\
%
\\
\>[3]\AgdaIndent{5}{}\<[5]%
\>[5]\AgdaFunction{q2≢0} \AgdaSymbol{:} \AgdaBound{q₂} \AgdaFunction{≢} \AgdaNumber{0}\<%
\\
\>[3]\AgdaIndent{5}{}\<[5]%
\>[5]\AgdaFunction{q2≢0} \AgdaBound{qe} \AgdaSymbol{=} \AgdaBound{neo} \AgdaSymbol{(}\AgdaFunction{lzero} \AgdaBound{q₂} \AgdaBound{di} \AgdaBound{qe}\AgdaSymbol{)}\<%
\\
%
\\
\>[3]\AgdaIndent{5}{}\<[5]%
\>[5]\AgdaFunction{invsuc} \AgdaSymbol{:} \AgdaSymbol{∀} \AgdaBound{n} \AgdaSymbol{→} \AgdaBound{n} \AgdaFunction{≢} \AgdaNumber{0} \AgdaSymbol{→} \AgdaBound{n} \AgdaDatatype{≡} \AgdaInductiveConstructor{suc} \AgdaSymbol{(}\AgdaFunction{pred} \AgdaBound{n}\AgdaSymbol{)}\<%
\\
\>[3]\AgdaIndent{5}{}\<[5]%
\>[5]\AgdaFunction{invsuc} \AgdaInductiveConstructor{zero} \AgdaBound{nz} \AgdaKeyword{with} \AgdaBound{nz} \AgdaInductiveConstructor{refl}\<%
\\
\>[3]\AgdaIndent{5}{}\<[5]%
\>[5]\AgdaSymbol{...} \AgdaSymbol{|} \AgdaSymbol{()}\<%
\\
\>[3]\AgdaIndent{5}{}\<[5]%
\>[5]\AgdaFunction{invsuc} \AgdaSymbol{(}\AgdaInductiveConstructor{suc} \AgdaBound{n}\AgdaSymbol{)} \AgdaBound{nz} \AgdaSymbol{=} \AgdaInductiveConstructor{refl}\<%
\\
\>[3]\AgdaIndent{5}{}\<[5]%
\>[5]\<%
\\
\>[3]\AgdaIndent{5}{}\<[5]%
\>[5]\AgdaFunction{deno≡q2} \AgdaSymbol{:} \AgdaBound{q₂} \AgdaDatatype{≡} \AgdaFunction{deno}\<%
\\
\>[3]\AgdaIndent{5}{}\<[5]%
\>[5]\AgdaFunction{deno≡q2} \AgdaSymbol{=} \AgdaFunction{invsuc} \AgdaBound{q₂} \AgdaFunction{q2≢0}\<%
\\
%
\\
\>[3]\AgdaIndent{5}{}\<[5]%
\>[5]\AgdaFunction{copnd} \AgdaSymbol{:} \AgdaFunction{Coprime} \AgdaBound{q₁} \AgdaFunction{deno}\<%
\\
\>[3]\AgdaIndent{5}{}\<[5]%
\>[5]\AgdaFunction{copnd} \AgdaSymbol{=} \AgdaFunction{subst} \AgdaSymbol{(λ} \AgdaBound{x} \AgdaSymbol{→} \AgdaFunction{Coprime} \AgdaBound{q₁} \AgdaBound{x}\AgdaSymbol{)} \AgdaFunction{deno≡q2} \AgdaBound{c}\<%
\\
%
\\
\>[3]\AgdaIndent{5}{}\<[5]%
\>[5]\AgdaFunction{witProp} \AgdaSymbol{:} \AgdaSymbol{∀} \AgdaBound{a} \AgdaBound{b} \AgdaSymbol{→} \AgdaRecord{GCD} \AgdaBound{a} \AgdaBound{b} \AgdaNumber{1} \<[33]%
\>[33]\<%
\\
\>[5]\AgdaIndent{13}{}\<[13]%
\>[13]\AgdaSymbol{→} \AgdaFunction{True} \AgdaSymbol{(}\AgdaFunction{coprime?} \AgdaBound{a} \AgdaBound{b}\AgdaSymbol{)}\<%
\\
\>[0]\AgdaIndent{5}{}\<[5]%
\>[5]\AgdaFunction{witProp} \AgdaBound{a} \AgdaBound{b} \AgdaBound{gcd1} \AgdaKeyword{with} \AgdaFunction{gcd} \AgdaBound{a} \AgdaBound{b}\<%
\\
\>[0]\AgdaIndent{5}{}\<[5]%
\>[5]\AgdaFunction{witProp} \AgdaBound{a} \AgdaBound{b} \AgdaBound{gcd1} \AgdaSymbol{|} \AgdaInductiveConstructor{zero} \AgdaInductiveConstructor{,} \AgdaBound{y} \AgdaKeyword{with} \AgdaFunction{GCD.unique} \AgdaBound{gcd1} \AgdaBound{y}\<%
\\
\>[0]\AgdaIndent{5}{}\<[5]%
\>[5]\AgdaFunction{witProp} \AgdaBound{a} \AgdaBound{b} \AgdaBound{gcd1} \AgdaSymbol{|} \AgdaInductiveConstructor{zero} \AgdaInductiveConstructor{,} \AgdaBound{y} \AgdaSymbol{|} \AgdaSymbol{()}\<%
\\
\>[0]\AgdaIndent{5}{}\<[5]%
\>[5]\AgdaFunction{witProp} \AgdaBound{a} \AgdaBound{b} \AgdaBound{gcd1} \AgdaSymbol{|} \AgdaInductiveConstructor{suc} \AgdaInductiveConstructor{zero} \AgdaInductiveConstructor{,} \AgdaBound{y} \AgdaSymbol{=} \AgdaInductiveConstructor{tt}\<%
\\
\>[0]\AgdaIndent{5}{}\<[5]%
\>[5]\AgdaFunction{witProp} \AgdaBound{a} \AgdaBound{b} \AgdaBound{gcd1} \AgdaSymbol{|} \AgdaInductiveConstructor{suc} \AgdaSymbol{(}\AgdaInductiveConstructor{suc} \AgdaBound{n}\AgdaSymbol{)} \AgdaInductiveConstructor{,} \AgdaBound{y} \<[40]%
\>[40]\<%
\\
\>[5]\AgdaIndent{33}{}\<[33]%
\>[33]\AgdaKeyword{with} \AgdaFunction{GCD.unique} \AgdaBound{gcd1} \AgdaBound{y}\<%
\\
\>[0]\AgdaIndent{5}{}\<[5]%
\>[5]\AgdaFunction{witProp} \AgdaBound{a} \AgdaBound{b} \AgdaBound{gcd1} \AgdaSymbol{|} \AgdaInductiveConstructor{suc} \AgdaSymbol{(}\AgdaInductiveConstructor{suc} \AgdaBound{n}\AgdaSymbol{)} \AgdaInductiveConstructor{,} \AgdaBound{y} \AgdaSymbol{|} \AgdaSymbol{()}\<%
\\
%
\\
\>[0]\AgdaIndent{5}{}\<[5]%
\>[5]\AgdaFunction{iscoprime} \AgdaSymbol{:} \AgdaFunction{True} \AgdaSymbol{(}\AgdaFunction{coprime?} \AgdaFunction{∣} \AgdaFunction{numr} \AgdaFunction{∣} \AgdaFunction{deno}\AgdaSymbol{)}\<%
\\
\>[0]\AgdaIndent{5}{}\<[5]%
\>[5]\AgdaFunction{iscoprime} \AgdaSymbol{=} \AgdaFunction{witProp} \AgdaSymbol{\_} \AgdaSymbol{\_} \AgdaSymbol{(}\AgdaFunction{coprime-gcd} \AgdaFunction{copnd}\AgdaSymbol{)}\<%
\\
\>\<\end{code}

2.Negation

\begin{code}\>\<%
\\
\>\AgdaFunction{-\_} \AgdaSymbol{:} \AgdaFunction{ℚ} \AgdaSymbol{→} \AgdaFunction{ℚ}\<%
\\
\>\AgdaFunction{-\_} \AgdaSymbol{((}\AgdaBound{n} \AgdaInductiveConstructor{/suc} \AgdaBound{d}\AgdaSymbol{)} \AgdaInductiveConstructor{,} \AgdaBound{isC}\AgdaSymbol{)} \AgdaSymbol{=} \AgdaSymbol{((}\AgdaFunction{ℤ-} \AgdaBound{n}\AgdaSymbol{)} \AgdaInductiveConstructor{/suc} \AgdaBound{d}\AgdaSymbol{)} \AgdaInductiveConstructor{,}\<%
\\
\>[0]\AgdaIndent{3}{}\<[3]%
\>[3]\AgdaFunction{subst} \AgdaSymbol{(λ} \AgdaBound{x} \AgdaSymbol{→} \AgdaFunction{True} \AgdaSymbol{(}\AgdaFunction{coprime?} \AgdaBound{x} \AgdaSymbol{(}\AgdaInductiveConstructor{suc} \AgdaBound{d}\AgdaSymbol{)))} \<[43]%
\>[43]\<%
\\
\>[0]\AgdaIndent{5}{}\<[5]%
\>[5]\AgdaSymbol{(}\AgdaFunction{forgetSign} \AgdaBound{n}\AgdaSymbol{)} \AgdaBound{isC}\<%
\\
\>[0]\AgdaIndent{3}{}\<[3]%
\>[3]\AgdaKeyword{where}\<%
\\
\>[0]\AgdaIndent{5}{}\<[5]%
\>[5]\AgdaFunction{forgetSign} \AgdaSymbol{:} \AgdaSymbol{∀} \AgdaBound{x} \AgdaSymbol{→} \AgdaFunction{∣} \AgdaBound{x} \AgdaFunction{∣} \AgdaDatatype{≡} \AgdaFunction{∣} \<[35]%
\>[35]\AgdaFunction{ℤ-} \AgdaBound{x} \AgdaFunction{∣}\<%
\\
\>[0]\AgdaIndent{5}{}\<[5]%
\>[5]\AgdaFunction{forgetSign} \AgdaSymbol{(}\AgdaInductiveConstructor{-suc} \AgdaBound{n}\AgdaSymbol{)} \AgdaSymbol{=} \AgdaInductiveConstructor{refl}\<%
\\
\>[0]\AgdaIndent{5}{}\<[5]%
\>[5]\AgdaFunction{forgetSign} \AgdaSymbol{(}\AgdaInductiveConstructor{+} \AgdaInductiveConstructor{zero}\AgdaSymbol{)} \AgdaSymbol{=} \AgdaInductiveConstructor{refl}\<%
\\
\>[0]\AgdaIndent{5}{}\<[5]%
\>[5]\AgdaFunction{forgetSign} \AgdaSymbol{(}\AgdaInductiveConstructor{+} \AgdaSymbol{(}\AgdaInductiveConstructor{suc} \AgdaBound{n}\AgdaSymbol{))} \AgdaSymbol{=} \AgdaInductiveConstructor{refl}\<%
\\
\>\<\end{code}

3.Normalisation function

\begin{code}\>\<%
\\
\>\AgdaFunction{[\_]} \AgdaSymbol{:} \AgdaDatatype{ℚ₀} \AgdaSymbol{→} \AgdaFunction{ℚ}\<%
\\
\>\AgdaFunction{[} \AgdaSymbol{(}\AgdaInductiveConstructor{+} \AgdaBound{n}\AgdaSymbol{)} \AgdaInductiveConstructor{/suc} \AgdaBound{d} \AgdaFunction{]} \AgdaSymbol{=} \AgdaFunction{calℚ} \AgdaBound{n} \AgdaSymbol{(}\AgdaInductiveConstructor{suc} \AgdaBound{d}\AgdaSymbol{)} \AgdaSymbol{(λ} \AgdaSymbol{())}\<%
\\
\>\AgdaFunction{[} \AgdaSymbol{(}\AgdaInductiveConstructor{-suc} \AgdaBound{n}\AgdaSymbol{)} \AgdaInductiveConstructor{/suc} \AgdaBound{d} \AgdaFunction{]} \AgdaSymbol{=} \AgdaFunction{-} \AgdaFunction{calℚ} \AgdaSymbol{(}\AgdaInductiveConstructor{suc} \AgdaBound{n}\AgdaSymbol{)} \AgdaSymbol{(}\AgdaInductiveConstructor{suc} \AgdaBound{d}\AgdaSymbol{)} \AgdaSymbol{(λ} \AgdaSymbol{())}\<%
\\
\>\<\end{code}

Embedding function

\begin{code}\>\<%
\\
\>\AgdaFunction{⌜\_⌝} \AgdaSymbol{:} \AgdaFunction{ℚ} \AgdaSymbol{→} \AgdaDatatype{ℚ₀}\<%
\\
\>\AgdaFunction{⌜\_⌝} \AgdaSymbol{=} \AgdaFunction{proj₁}\<%
\\
\>\<\end{code}



\AgdaHide{
\begin{code}\>\<%
\\
\>\AgdaComment{\{-
GCD′→ℚ : ∀ x y di → y ≢ 0 → C.GCD′ x y di → ℚ
GCD′→ℚ .(q₁ ℕ* di) .(q₂ ℕ* di) di neo (C.gcd-* q₁ q₂ c) 
  = record \{ numerator = numr
           ; denominator-1 = pred q₂
           ; isCoprime = iscoprime \}
   where
     numr = ℤ.+\_ q₁
     deno = suc (pred q₂)
     
     numr≡q1 : ∣ numr ∣ ≡ q₁
     numr≡q1 = refl

     lzero : ∀ x y → x ≡ 0 → x ℕ* y ≡ 0
     lzero .0 y refl = refl

     q2≢0 : q₂ ≢ 0
     q2≢0 qe = neo (lzero q₂ di qe)

     invsuc : ∀ n → n ≢ 0 → suc (pred n) ≡ n
     invsuc zero nz with nz refl
     ... | ()
     invsuc (suc n) nz = refl
     
     deno≡q2 : deno ≡ q₂
     deno≡q2 = invsuc q₂ q2≢0
     
     transCop : ∀ \{a b c d\} → c ≡ a → d ≡ b 
              → C.Coprime a b → C.Coprime c d
     transCop refl refl c = c

     copnd : C.Coprime ∣ numr ∣ deno
     copnd = transCop numr≡q1 deno≡q2 c

     witProp : ∀ a b → GCD a b 1 
             → True (C.coprime? a b)
     witProp a b gcd1 with gcd a b
     witProp a b gcd1 | zero , y with GCD.unique gcd1 y
     witProp a b gcd1 | zero , y | ()
     witProp a b gcd1 | suc zero , y = tt
     witProp a b gcd1 | suc (suc n) , y 
                                 with GCD.unique gcd1 y
     witProp a b gcd1 | suc (suc n) , y | ()

     iscoprime : True (C.coprime? ∣ numr ∣ deno)
     iscoprime = witProp ∣ numr ∣ deno 
                 (C.coprime-gcd copnd)
}\<\end{code}

2.Negation

\begin{code}\>\AgdaComment{
-\_ : ℚ → ℚ
-\_ q = record \{ numerator = ℤ- numr
              ; denominator-1 = deno-1
              ; isCoprime = iscoprime- \}
   where
     numr = ℚ.numerator q
     deno-1 = ℚ.denominator-1 q

     iscoprime : True (C.coprime? ∣ numr ∣ (suc deno-1))
     iscoprime = ℚ.isCoprime q

     forgetSign : ∀ x → ∣ x ∣ ≡ ∣  ℤ- x ∣
     forgetSign (-suc n) = refl
     forgetSign (+ zero) = refl
     forgetSign (+ (suc n)) = refl

     iscoprime- : True (C.coprime? ∣ ℤ- numr ∣ (suc deno-1))
     iscoprime- = subst (λ x → True (C.coprime? x (suc deno-1))) 
                  (forgetSign numr) iscoprime
}\<\end{code}

3.Normalisation function

\begin{code}\>\AgdaComment{
[\_] : ℚ₀ → ℚ
[ (+ 0) /suc d ] = ℤ.+\_ 0 ÷ 1

[ (+ (suc n)) /suc d ] with gcd (suc n) (suc d)

[ (+ suc n) /suc d ] | di , g = GCD′→ℚ (suc n) (suc d) 
                              di (λ ()) (C.gcd-gcd′ g)

[ (-suc n) /suc d ] with gcd (suc n) (suc d)
... | di , g = - GCD′→ℚ (suc n) (suc d) di (λ ()) 
             (C.gcd-gcd′ g)
}\<\end{code}

Embedding function

\begin{code}\>\AgdaComment{
⌜\_⌝ : ℚ → ℚ₀
⌜ q ⌝ = (ℚ.numerator q) /suc (ℚ.denominator-1 q)
-\}}\<%
\\
\>\<\end{code}

}
%\chapter{Fractions reduction}

\begin{code}\label{rational-gcd}
\\
\>\AgdaFunction{GCD′→ℚ} \AgdaSymbol{:} \AgdaSymbol{∀} \AgdaBound{x} \AgdaBound{y} \AgdaBound{di} \AgdaSymbol{→} \AgdaBound{y} \AgdaFunction{≢} \AgdaNumber{0} \AgdaSymbol{→} \AgdaDatatype{C.GCD′} \AgdaBound{x} \AgdaBound{y} \AgdaBound{di} \AgdaSymbol{→} \AgdaRecord{ℚ}\<%
\\
\>\AgdaFunction{GCD′→ℚ} \AgdaSymbol{.(}\AgdaBound{q₁} \AgdaPrimitive{ℕ*} \AgdaBound{di}\AgdaSymbol{)} \AgdaSymbol{.(}\AgdaBound{q₂} \AgdaPrimitive{ℕ*} \AgdaBound{di}\AgdaSymbol{)} \AgdaBound{di} \AgdaBound{neo} \AgdaSymbol{(}\AgdaInductiveConstructor{C.gcd-*} \AgdaBound{q₁} \AgdaBound{q₂} \AgdaBound{c}\AgdaSymbol{)} \AgdaSymbol{=} \AgdaKeyword{record} \AgdaSymbol{\{} \AgdaField{numerator} \AgdaSymbol{=} \AgdaFunction{numr}\<%
\\
\>[0]\AgdaIndent{10}{}\<[10]%
\>[10]\AgdaSymbol{;} \AgdaField{denominator-1} \AgdaSymbol{=} \AgdaFunction{pred} \AgdaBound{q₂}\<%
\\
\>[0]\AgdaIndent{10}{}\<[10]%
\>[10]\AgdaSymbol{;} \AgdaField{isCoprime} \AgdaSymbol{=} \AgdaFunction{iscoprime} \AgdaSymbol{\}}\<%
\\
\>[0]\AgdaIndent{3}{}\<[3]%
\>[3]\AgdaKeyword{where}\<%
\\
\>[0]\AgdaIndent{5}{}\<[5]%
\>[5]\AgdaFunction{numr} \AgdaSymbol{=} \AgdaInductiveConstructor{ℤ.+\_} \AgdaBound{q₁}\<%
\\
\>[0]\AgdaIndent{5}{}\<[5]%
\>[5]\AgdaFunction{deno} \AgdaSymbol{=} \AgdaInductiveConstructor{suc} \AgdaSymbol{(}\AgdaFunction{pred} \AgdaBound{q₂}\AgdaSymbol{)}\<%
\\
\>[0]\AgdaIndent{5}{}\<[5]%
\>[5]\<%
\\
\>[0]\AgdaIndent{5}{}\<[5]%
\>[5]\AgdaFunction{numr≡q1} \AgdaSymbol{:} \AgdaFunction{∣} \AgdaFunction{numr} \AgdaFunction{∣} \AgdaDatatype{≡} \AgdaBound{q₁}\<%
\\
\>[0]\AgdaIndent{5}{}\<[5]%
\>[5]\AgdaFunction{numr≡q1} \AgdaSymbol{=} \AgdaInductiveConstructor{refl}\<%
\\
%
\\
\>[0]\AgdaIndent{5}{}\<[5]%
\>[5]\AgdaFunction{lzero} \AgdaSymbol{:} \AgdaSymbol{∀} \AgdaBound{x} \AgdaBound{y} \AgdaSymbol{→} \AgdaBound{x} \AgdaDatatype{≡} \AgdaNumber{0} \AgdaSymbol{→} \AgdaBound{x} \AgdaPrimitive{ℕ*} \AgdaBound{y} \AgdaDatatype{≡} \AgdaNumber{0}\<%
\\
\>[0]\AgdaIndent{5}{}\<[5]%
\>[5]\AgdaFunction{lzero} \AgdaSymbol{.}\AgdaNumber{0} \AgdaBound{y} \AgdaInductiveConstructor{refl} \AgdaSymbol{=} \AgdaInductiveConstructor{refl}\<%
\\
%
\\
\>[0]\AgdaIndent{5}{}\<[5]%
\>[5]\AgdaFunction{q2≢0} \AgdaSymbol{:} \AgdaBound{q₂} \AgdaFunction{≢} \AgdaNumber{0}\<%
\\
\>[0]\AgdaIndent{5}{}\<[5]%
\>[5]\AgdaFunction{q2≢0} \AgdaBound{qe} \AgdaSymbol{=} \AgdaBound{neo} \AgdaSymbol{(}\AgdaFunction{lzero} \AgdaBound{q₂} \AgdaBound{di} \AgdaBound{qe}\AgdaSymbol{)}\<%
\\
%
\\
\>[0]\AgdaIndent{5}{}\<[5]%
\>[5]\AgdaFunction{invsuc} \AgdaSymbol{:} \AgdaSymbol{∀} \AgdaBound{n} \AgdaSymbol{→} \AgdaBound{n} \AgdaFunction{≢} \AgdaNumber{0} \AgdaSymbol{→} \AgdaInductiveConstructor{suc} \AgdaSymbol{(}\AgdaFunction{pred} \AgdaBound{n}\AgdaSymbol{)} \AgdaDatatype{≡} \AgdaBound{n}\<%
\\
\>[0]\AgdaIndent{5}{}\<[5]%
\>[5]\AgdaFunction{invsuc} \AgdaInductiveConstructor{zero} \AgdaBound{nz} \AgdaKeyword{with} \AgdaBound{nz} \AgdaInductiveConstructor{refl}\<%
\\
\>[0]\AgdaIndent{5}{}\<[5]%
\>[5]\AgdaSymbol{...} \AgdaSymbol{|} \AgdaSymbol{()}\<%
\\
\>[0]\AgdaIndent{5}{}\<[5]%
\>[5]\AgdaFunction{invsuc} \AgdaSymbol{(}\AgdaInductiveConstructor{suc} \AgdaBound{n}\AgdaSymbol{)} \AgdaBound{nz} \AgdaSymbol{=} \AgdaInductiveConstructor{refl}\<%
\\
\>[0]\AgdaIndent{5}{}\<[5]%
\>[5]\<%
\\
\>[0]\AgdaIndent{5}{}\<[5]%
\>[5]\AgdaFunction{deno≡q2} \AgdaSymbol{:} \AgdaFunction{deno} \AgdaDatatype{≡} \AgdaBound{q₂}\<%
\\
\>[0]\AgdaIndent{5}{}\<[5]%
\>[5]\AgdaFunction{deno≡q2} \AgdaSymbol{=} \AgdaFunction{invsuc} \AgdaBound{q₂} \AgdaFunction{q2≢0}\<%
\\
\>[0]\AgdaIndent{5}{}\<[5]%
\>[5]\<%
\\
\>[0]\AgdaIndent{5}{}\<[5]%
\>[5]\AgdaFunction{transCop} \AgdaSymbol{:} \AgdaSymbol{∀} \AgdaSymbol{\{}\AgdaBound{a} \AgdaBound{b} \AgdaBound{c} \AgdaBound{d}\AgdaSymbol{\}} \AgdaSymbol{→} \AgdaBound{c} \AgdaDatatype{≡} \AgdaBound{a} \AgdaSymbol{→} \AgdaBound{d} \AgdaDatatype{≡} \AgdaBound{b} \AgdaSymbol{→} \AgdaFunction{C.Coprime} \AgdaBound{a} \AgdaBound{b} \AgdaSymbol{→} \AgdaFunction{C.Coprime} \AgdaBound{c} \AgdaBound{d}\<%
\\
\>[0]\AgdaIndent{5}{}\<[5]%
\>[5]\AgdaFunction{transCop} \AgdaInductiveConstructor{refl} \AgdaInductiveConstructor{refl} \AgdaBound{c} \AgdaSymbol{=} \AgdaBound{c}\<%
\\
%
\\
\>[0]\AgdaIndent{5}{}\<[5]%
\>[5]\AgdaFunction{copnd} \AgdaSymbol{:} \AgdaFunction{C.Coprime} \AgdaFunction{∣} \AgdaFunction{numr} \AgdaFunction{∣} \AgdaFunction{deno}\<%
\\
\>[0]\AgdaIndent{5}{}\<[5]%
\>[5]\AgdaFunction{copnd} \AgdaSymbol{=} \AgdaFunction{transCop} \AgdaFunction{numr≡q1} \AgdaFunction{deno≡q2} \AgdaBound{c}\<%
\\
%
\\
\>[0]\AgdaIndent{5}{}\<[5]%
\>[5]\AgdaFunction{witProp} \AgdaSymbol{:} \AgdaSymbol{∀} \AgdaBound{a} \AgdaBound{b} \AgdaSymbol{→} \AgdaRecord{GCD} \AgdaBound{a} \AgdaBound{b} \AgdaNumber{1} \AgdaSymbol{→} \AgdaFunction{True} \AgdaSymbol{(}\AgdaFunction{C.coprime?} \AgdaBound{a} \AgdaBound{b}\AgdaSymbol{)}\<%
\\
\>[0]\AgdaIndent{5}{}\<[5]%
\>[5]\AgdaFunction{witProp} \AgdaBound{a} \AgdaBound{b} \AgdaBound{gcd1} \AgdaKeyword{with} \AgdaFunction{gcd} \AgdaBound{a} \AgdaBound{b}\<%
\\
\>[0]\AgdaIndent{5}{}\<[5]%
\>[5]\AgdaFunction{witProp} \AgdaBound{a} \AgdaBound{b} \AgdaBound{gcd1} \AgdaSymbol{|} \AgdaInductiveConstructor{zero} \AgdaInductiveConstructor{,} \AgdaBound{y} \AgdaKeyword{with} \AgdaFunction{GCD.unique} \AgdaBound{gcd1} \AgdaBound{y}\<%
\\
\>[0]\AgdaIndent{5}{}\<[5]%
\>[5]\AgdaFunction{witProp} \AgdaBound{a} \AgdaBound{b} \AgdaBound{gcd1} \AgdaSymbol{|} \AgdaInductiveConstructor{zero} \AgdaInductiveConstructor{,} \AgdaBound{y} \AgdaSymbol{|} \AgdaSymbol{()}\<%
\\
\>[0]\AgdaIndent{5}{}\<[5]%
\>[5]\AgdaFunction{witProp} \AgdaBound{a} \AgdaBound{b} \AgdaBound{gcd1} \AgdaSymbol{|} \AgdaInductiveConstructor{suc} \AgdaInductiveConstructor{zero} \AgdaInductiveConstructor{,} \AgdaBound{y} \AgdaSymbol{=} \AgdaInductiveConstructor{tt}\<%
\\
\>[0]\AgdaIndent{5}{}\<[5]%
\>[5]\AgdaFunction{witProp} \AgdaBound{a} \AgdaBound{b} \AgdaBound{gcd1} \AgdaSymbol{|} \AgdaInductiveConstructor{suc} \AgdaSymbol{(}\AgdaInductiveConstructor{suc} \AgdaBound{n}\AgdaSymbol{)} \AgdaInductiveConstructor{,} \AgdaBound{y} \AgdaKeyword{with} \AgdaFunction{GCD.unique} \AgdaBound{gcd1} \AgdaBound{y}\<%
\\
\>[0]\AgdaIndent{5}{}\<[5]%
\>[5]\AgdaFunction{witProp} \AgdaBound{a} \AgdaBound{b} \AgdaBound{gcd1} \AgdaSymbol{|} \AgdaInductiveConstructor{suc} \AgdaSymbol{(}\AgdaInductiveConstructor{suc} \AgdaBound{n}\AgdaSymbol{)} \AgdaInductiveConstructor{,} \AgdaBound{y} \AgdaSymbol{|} \AgdaSymbol{()}\<%
\\
%
\\
\>[0]\AgdaIndent{5}{}\<[5]%
\>[5]\AgdaFunction{iscoprime} \AgdaSymbol{:} \AgdaFunction{True} \AgdaSymbol{(}\AgdaFunction{C.coprime?} \AgdaFunction{∣} \AgdaFunction{numr} \AgdaFunction{∣} \AgdaFunction{deno}\AgdaSymbol{)}\<%
\\
\>[0]\AgdaIndent{5}{}\<[5]%
\>[5]\AgdaFunction{iscoprime} \AgdaSymbol{=} \AgdaFunction{witProp} \AgdaFunction{∣} \AgdaFunction{numr} \AgdaFunction{∣} \AgdaFunction{deno} \AgdaSymbol{(}\AgdaFunction{C.coprime-gcd} \AgdaFunction{copnd}\AgdaSymbol{)}\<%
\\
\end{code}

%

\begin{code}\>\<%
\\
%
\\
\>\AgdaKeyword{module} \AgdaModule{GroupoidQuotient} \AgdaKeyword{where}\<%
\\
%
\\
\>\AgdaKeyword{open} \AgdaKeyword{import} \AgdaModule{Relation.Binary.PropositionalEquality} \AgdaSymbol{as} \AgdaModule{PE}\<%
\\
\>[0]\AgdaIndent{2}{}\<[2]%
\>[2]\AgdaKeyword{hiding} \AgdaSymbol{(}[\_]\AgdaSymbol{)}\<%
\\
\>\AgdaKeyword{open} \AgdaKeyword{import} \AgdaModule{Quotient}\<%
\\
\>\AgdaKeyword{open} \AgdaKeyword{import} \AgdaModule{Data.Product}\<%
\\
\>\AgdaKeyword{open} \AgdaModule{≡-Reasoning}\<%
\\
\>\<\end{code}

The equivalence relation is not propositional but h-set.

\begin{code}\>\<%
\\
%
\\
\>\AgdaKeyword{record} \AgdaRecord{Groupoid} \AgdaSymbol{:} \AgdaPrimitiveType{Set₁} \AgdaKeyword{where}\<%
\\
\>[0]\AgdaIndent{2}{}\<[2]%
\>[2]\AgdaKeyword{infix} \AgdaNumber{4} \_\textasciitilde\_\<%
\\
\>[0]\AgdaIndent{2}{}\<[2]%
\>[2]\AgdaKeyword{infixl} \AgdaNumber{5} \_*\_\<%
\\
\>[0]\AgdaIndent{2}{}\<[2]%
\>[2]\AgdaKeyword{field}\<%
\\
\>[2]\AgdaIndent{4}{}\<[4]%
\>[4]\AgdaField{Carrier} \AgdaSymbol{:} \AgdaPrimitiveType{Set}\<%
\\
\>[2]\AgdaIndent{4}{}\<[4]%
\>[4]\AgdaField{\_\textasciitilde\_} \<[12]%
\>[12]\AgdaSymbol{:} \AgdaBound{Carrier} \AgdaSymbol{→} \AgdaBound{Carrier} \AgdaSymbol{→} \AgdaPrimitiveType{Set}\<%
\\
\>[2]\AgdaIndent{4}{}\<[4]%
\>[4]\AgdaField{id} \<[12]%
\>[12]\AgdaSymbol{:} \AgdaSymbol{∀\{}\AgdaBound{a}\AgdaSymbol{\}} \AgdaSymbol{→} \AgdaBound{a} \AgdaBound{\textasciitilde} \AgdaBound{a}\<%
\\
\>[2]\AgdaIndent{4}{}\<[4]%
\>[4]\AgdaField{\_⁻¹} \<[13]%
\>[13]\AgdaSymbol{:} \AgdaSymbol{∀\{}\AgdaBound{a} \AgdaBound{b}\AgdaSymbol{\}} \AgdaSymbol{→} \AgdaBound{a} \AgdaBound{\textasciitilde} \AgdaBound{b} \AgdaSymbol{→} \AgdaBound{b} \AgdaBound{\textasciitilde} \AgdaBound{a}\<%
\\
\>[2]\AgdaIndent{4}{}\<[4]%
\>[4]\AgdaField{\_*\_} \<[12]%
\>[12]\AgdaSymbol{:} \AgdaSymbol{∀\{}\AgdaBound{a} \AgdaBound{b} \AgdaBound{c}\AgdaSymbol{\}} \AgdaSymbol{→} \AgdaBound{a} \AgdaBound{\textasciitilde} \AgdaBound{b} \AgdaSymbol{→} \AgdaBound{b} \AgdaBound{\textasciitilde} \AgdaBound{c} \AgdaSymbol{→} \AgdaBound{a} \AgdaBound{\textasciitilde} \AgdaBound{c}\<%
\\
\>[2]\AgdaIndent{4}{}\<[4]%
\>[4]\AgdaField{id₁} \<[12]%
\>[12]\AgdaSymbol{:} \AgdaSymbol{∀\{}\AgdaBound{a} \AgdaBound{b}\AgdaSymbol{\}\{}\AgdaBound{p} \AgdaSymbol{:} \AgdaBound{a} \AgdaBound{\textasciitilde} \AgdaBound{b}\AgdaSymbol{\}} \AgdaSymbol{→} \AgdaBound{id} \AgdaBound{*} \AgdaBound{p} \AgdaDatatype{≡} \AgdaBound{p}\<%
\\
\>[2]\AgdaIndent{4}{}\<[4]%
\>[4]\AgdaField{id₂} \<[12]%
\>[12]\AgdaSymbol{:} \AgdaSymbol{∀\{}\AgdaBound{a} \AgdaBound{b}\AgdaSymbol{\}\{}\AgdaBound{p} \AgdaSymbol{:} \AgdaBound{a} \AgdaBound{\textasciitilde} \AgdaBound{b}\AgdaSymbol{\}} \AgdaSymbol{→} \AgdaBound{p} \AgdaBound{*} \AgdaBound{id} \AgdaDatatype{≡} \AgdaBound{p}\<%
\\
\>[2]\AgdaIndent{4}{}\<[4]%
\>[4]\AgdaField{inv₁} \<[12]%
\>[12]\AgdaSymbol{:} \AgdaSymbol{∀\{}\AgdaBound{a} \AgdaBound{b}\AgdaSymbol{\}\{}\AgdaBound{p} \AgdaSymbol{:} \AgdaBound{a} \AgdaBound{\textasciitilde} \AgdaBound{b}\AgdaSymbol{\}} \AgdaSymbol{→} \AgdaBound{p} \AgdaBound{*} \AgdaBound{p} \AgdaBound{⁻¹} \AgdaDatatype{≡} \AgdaBound{id}\<%
\\
\>[2]\AgdaIndent{4}{}\<[4]%
\>[4]\AgdaField{inv₂} \<[12]%
\>[12]\AgdaSymbol{:} \AgdaSymbol{∀\{}\AgdaBound{a} \AgdaBound{b}\AgdaSymbol{\}\{}\AgdaBound{p} \AgdaSymbol{:} \AgdaBound{a} \AgdaBound{\textasciitilde} \AgdaBound{b}\AgdaSymbol{\}} \AgdaSymbol{→} \<[35]%
\>[35]\AgdaBound{p} \AgdaBound{⁻¹} \AgdaBound{*} \AgdaBound{p} \AgdaDatatype{≡} \AgdaBound{id}\<%
\\
\>[2]\AgdaIndent{4}{}\<[4]%
\>[4]\AgdaField{assoc} \<[12]%
\>[12]\AgdaSymbol{:} \AgdaSymbol{∀\{}\AgdaBound{a} \AgdaBound{b} \AgdaBound{c} \AgdaBound{d}\AgdaSymbol{\}\{}\AgdaBound{p} \AgdaSymbol{:} \AgdaBound{a} \AgdaBound{\textasciitilde} \AgdaBound{b}\AgdaSymbol{\}\{}\AgdaBound{q} \AgdaSymbol{:} \AgdaBound{b} \AgdaBound{\textasciitilde} \AgdaBound{c}\AgdaSymbol{\}\{}\AgdaBound{r} \AgdaSymbol{:} \AgdaBound{c} \AgdaBound{\textasciitilde} \AgdaBound{d}\AgdaSymbol{\}}\<%
\\
\>[4]\AgdaIndent{12}{}\<[12]%
\>[12]\AgdaSymbol{→} \AgdaSymbol{(}\AgdaBound{p} \AgdaBound{*} \AgdaBound{q}\AgdaSymbol{)} \AgdaBound{*} \AgdaBound{r} \AgdaDatatype{≡} \AgdaBound{p} \AgdaBound{*} \AgdaSymbol{(}\AgdaBound{q} \AgdaBound{*} \AgdaBound{r}\AgdaSymbol{)}\<%
\\
\>[0]\AgdaIndent{4}{}\<[4]%
\>[4]\AgdaField{isGrp} \<[12]%
\>[12]\AgdaSymbol{:} \AgdaSymbol{∀\{}\AgdaBound{a} \AgdaBound{b}\AgdaSymbol{\}\{}\AgdaBound{p} \AgdaBound{q} \AgdaSymbol{:} \AgdaBound{a} \AgdaBound{\textasciitilde} \AgdaBound{b}\AgdaSymbol{\}} \AgdaSymbol{→} \AgdaFunction{isProp} \AgdaSymbol{(}\AgdaBound{p} \AgdaDatatype{≡} \AgdaBound{q}\AgdaSymbol{)}\<%
\\
%
\\
\>[0]\AgdaIndent{2}{}\<[2]%
\>[2]\AgdaFunction{inv₂-id₁} \AgdaSymbol{:} \AgdaSymbol{∀\{}\AgdaBound{a} \AgdaBound{b} \AgdaBound{c}\AgdaSymbol{\}\{}\AgdaBound{p} \AgdaSymbol{:} \AgdaBound{a} \AgdaFunction{\textasciitilde} \AgdaBound{b}\AgdaSymbol{\}\{}\AgdaBound{q} \AgdaSymbol{:} \AgdaBound{b} \AgdaFunction{\textasciitilde} \AgdaBound{c}\AgdaSymbol{\}} \AgdaSymbol{→} \AgdaBound{p} \AgdaFunction{⁻¹} \AgdaFunction{*} \AgdaBound{p} \AgdaFunction{*} \AgdaBound{q} \AgdaDatatype{≡} \AgdaBound{q}\<%
\\
\>[0]\AgdaIndent{2}{}\<[2]%
\>[2]\AgdaFunction{inv₂-id₁} \AgdaSymbol{=} \AgdaFunction{trans} \AgdaSymbol{(}\AgdaFunction{cong} \AgdaSymbol{(λ} \AgdaBound{x} \AgdaSymbol{→} \AgdaBound{x} \AgdaFunction{*} \AgdaSymbol{\_)} \AgdaFunction{inv₂}\AgdaSymbol{)} \AgdaFunction{id₁}\<%
\\
%
\\
\>[0]\AgdaIndent{2}{}\<[2]%
\>[2]\AgdaFunction{inv₁-id₂} \AgdaSymbol{:} \AgdaSymbol{∀\{}\AgdaBound{a} \AgdaBound{b} \AgdaBound{c}\AgdaSymbol{\}\{}\AgdaBound{p} \AgdaSymbol{:} \AgdaBound{a} \AgdaFunction{\textasciitilde} \AgdaBound{b}\AgdaSymbol{\}\{}\AgdaBound{q} \AgdaSymbol{:} \AgdaBound{b} \AgdaFunction{\textasciitilde} \AgdaBound{c}\AgdaSymbol{\}} \AgdaSymbol{→} \AgdaBound{p} \AgdaFunction{*} \AgdaSymbol{(}\AgdaBound{q} \AgdaFunction{*} \AgdaBound{q} \AgdaFunction{⁻¹}\AgdaSymbol{)} \AgdaDatatype{≡} \AgdaBound{p}\<%
\\
\>[0]\AgdaIndent{2}{}\<[2]%
\>[2]\AgdaFunction{inv₁-id₂} \AgdaSymbol{=} \AgdaFunction{trans} \AgdaSymbol{(}\AgdaFunction{cong} \AgdaSymbol{(λ} \AgdaBound{x} \AgdaSymbol{→} \AgdaSymbol{\_} \AgdaFunction{*} \AgdaBound{x}\AgdaSymbol{)} \AgdaFunction{inv₁}\AgdaSymbol{)} \AgdaFunction{id₂}\<%
\\
%
\\
\>[0]\AgdaIndent{2}{}\<[2]%
\>[2]\AgdaFunction{id-sym} \AgdaSymbol{:} \AgdaSymbol{∀} \AgdaSymbol{\{}\AgdaBound{a}\AgdaSymbol{\}} \AgdaSymbol{→} \AgdaFunction{id} \AgdaFunction{⁻¹} \AgdaDatatype{≡} \AgdaFunction{id} \AgdaSymbol{\{}\AgdaBound{a}\AgdaSymbol{\}} \<[34]%
\>[34]\<%
\\
\>[0]\AgdaIndent{2}{}\<[2]%
\>[2]\AgdaFunction{id-sym} \AgdaSymbol{=} \AgdaFunction{begin}\<%
\\
\>[2]\AgdaIndent{14}{}\<[14]%
\>[14]\AgdaFunction{id} \AgdaFunction{⁻¹} \<[27]%
\>[27]\AgdaFunction{≡⟨} \AgdaFunction{sym} \AgdaFunction{id₂} \AgdaFunction{⟩}\<%
\\
\>[2]\AgdaIndent{14}{}\<[14]%
\>[14]\AgdaFunction{id} \AgdaFunction{⁻¹} \AgdaFunction{*} \AgdaFunction{id} \<[27]%
\>[27]\AgdaFunction{≡⟨} \AgdaFunction{inv₂} \AgdaFunction{⟩}\<%
\\
\>[2]\AgdaIndent{14}{}\<[14]%
\>[14]\AgdaFunction{id} \AgdaFunction{∎}\<%
\\
%
\\
\>[0]\AgdaIndent{2}{}\<[2]%
\>[2]\AgdaFunction{sym-comp} \AgdaSymbol{:} \AgdaSymbol{∀} \AgdaSymbol{\{}\AgdaBound{a} \AgdaBound{b} \AgdaBound{c}\AgdaSymbol{\}(}\AgdaBound{p} \AgdaSymbol{:} \AgdaBound{a} \AgdaFunction{\textasciitilde} \AgdaBound{b}\AgdaSymbol{)(}\AgdaBound{q} \AgdaSymbol{:} \AgdaBound{b} \AgdaFunction{\textasciitilde} \AgdaBound{c}\AgdaSymbol{)} \AgdaSymbol{→} \AgdaSymbol{(}\AgdaBound{p} \AgdaFunction{*} \AgdaBound{q}\AgdaSymbol{)} \AgdaFunction{⁻¹} \AgdaDatatype{≡} \AgdaBound{q} \AgdaFunction{⁻¹} \AgdaFunction{*} \AgdaBound{p} \AgdaFunction{⁻¹}\<%
\\
\>[0]\AgdaIndent{2}{}\<[2]%
\>[2]\AgdaFunction{sym-comp} \AgdaBound{p} \AgdaBound{q} \AgdaSymbol{=} \AgdaFunction{begin}\<%
\\
\>[2]\AgdaIndent{14}{}\<[14]%
\>[14]\AgdaSymbol{(}\AgdaBound{p} \AgdaFunction{*} \AgdaBound{q}\AgdaSymbol{)} \AgdaFunction{⁻¹} \<[51]%
\>[51]\AgdaFunction{≡⟨} \AgdaFunction{sym} \AgdaFunction{inv₁-id₂} \AgdaFunction{⟩}\<%
\\
\>[2]\AgdaIndent{14}{}\<[14]%
\>[14]\AgdaSymbol{(}\AgdaBound{p} \AgdaFunction{*} \AgdaBound{q}\AgdaSymbol{)} \AgdaFunction{⁻¹} \AgdaFunction{*} \AgdaSymbol{(}\AgdaBound{p} \AgdaFunction{*} \AgdaBound{p} \AgdaFunction{⁻¹}\AgdaSymbol{)} \<[52]%
\>[52]\AgdaFunction{≡⟨} \AgdaFunction{cong} \AgdaSymbol{(λ} \AgdaBound{x} \AgdaSymbol{→} \AgdaSymbol{(}\AgdaBound{p} \AgdaFunction{*} \AgdaBound{q}\AgdaSymbol{)} \AgdaFunction{⁻¹} \AgdaFunction{*} \AgdaSymbol{(}\AgdaBound{x} \AgdaFunction{*} \AgdaBound{p} \AgdaFunction{⁻¹}\AgdaSymbol{))} \AgdaSymbol{(}\AgdaFunction{sym} \AgdaFunction{inv₁-id₂}\AgdaSymbol{)} \AgdaFunction{⟩}\<%
\\
\>[2]\AgdaIndent{14}{}\<[14]%
\>[14]\AgdaSymbol{(}\AgdaBound{p} \AgdaFunction{*} \AgdaBound{q}\AgdaSymbol{)} \AgdaFunction{⁻¹} \AgdaFunction{*} \AgdaSymbol{(}\AgdaBound{p} \AgdaFunction{*} \AgdaSymbol{(}\AgdaBound{q} \AgdaFunction{*} \AgdaBound{q} \AgdaFunction{⁻¹}\AgdaSymbol{)} \AgdaFunction{*} \AgdaBound{p} \AgdaFunction{⁻¹}\AgdaSymbol{)} \<[53]%
\>[53]\AgdaFunction{≡⟨} \AgdaFunction{cong} \AgdaSymbol{(λ} \AgdaBound{x} \AgdaSymbol{→} \AgdaSymbol{(}\AgdaBound{p} \AgdaFunction{*} \AgdaBound{q}\AgdaSymbol{)} \AgdaFunction{⁻¹} \AgdaFunction{*} \AgdaSymbol{(}\AgdaBound{x} \AgdaFunction{*} \AgdaBound{p} \AgdaFunction{⁻¹}\AgdaSymbol{))} \AgdaSymbol{(}\AgdaFunction{sym} \AgdaFunction{assoc}\AgdaSymbol{)} \AgdaFunction{⟩}\<%
\\
\>[2]\AgdaIndent{14}{}\<[14]%
\>[14]\AgdaSymbol{(}\AgdaBound{p} \AgdaFunction{*} \AgdaBound{q}\AgdaSymbol{)} \AgdaFunction{⁻¹} \AgdaFunction{*} \AgdaSymbol{(}\AgdaBound{p} \AgdaFunction{*} \AgdaBound{q} \AgdaFunction{*} \AgdaBound{q} \AgdaFunction{⁻¹} \AgdaFunction{*} \AgdaBound{p} \AgdaFunction{⁻¹}\AgdaSymbol{)} \<[53]%
\>[53]\AgdaFunction{≡⟨} \AgdaFunction{sym} \AgdaFunction{assoc} \AgdaFunction{⟩}\<%
\\
\>[2]\AgdaIndent{14}{}\<[14]%
\>[14]\AgdaSymbol{(}\AgdaBound{p} \AgdaFunction{*} \AgdaBound{q}\AgdaSymbol{)} \AgdaFunction{⁻¹} \AgdaFunction{*} \AgdaSymbol{(}\AgdaBound{p} \AgdaFunction{*} \AgdaBound{q} \AgdaFunction{*} \AgdaBound{q} \AgdaFunction{⁻¹}\AgdaSymbol{)} \AgdaFunction{*} \AgdaBound{p} \AgdaFunction{⁻¹} \<[53]%
\>[53]\AgdaFunction{≡⟨} \AgdaFunction{cong} \AgdaSymbol{(λ} \AgdaBound{x} \AgdaSymbol{→} \AgdaBound{x} \AgdaFunction{*} \AgdaBound{p} \AgdaFunction{⁻¹}\AgdaSymbol{)} \AgdaSymbol{(}\AgdaFunction{sym} \AgdaFunction{assoc}\AgdaSymbol{)} \AgdaFunction{⟩}\<%
\\
\>[2]\AgdaIndent{14}{}\<[14]%
\>[14]\AgdaSymbol{(}\AgdaBound{p} \AgdaFunction{*} \AgdaBound{q}\AgdaSymbol{)} \AgdaFunction{⁻¹} \AgdaFunction{*} \AgdaSymbol{(}\AgdaBound{p} \AgdaFunction{*} \AgdaBound{q}\AgdaSymbol{)} \AgdaFunction{*} \AgdaBound{q} \AgdaFunction{⁻¹} \AgdaFunction{*} \AgdaBound{p} \AgdaFunction{⁻¹} \<[53]%
\>[53]\AgdaFunction{≡⟨} \AgdaFunction{cong} \AgdaSymbol{(λ} \AgdaBound{x} \AgdaSymbol{→} \AgdaBound{x} \AgdaFunction{*} \AgdaBound{p} \AgdaFunction{⁻¹}\AgdaSymbol{)} \AgdaFunction{inv₂-id₁} \AgdaFunction{⟩}\<%
\\
\>[2]\AgdaIndent{14}{}\<[14]%
\>[14]\AgdaBound{q} \AgdaFunction{⁻¹} \AgdaFunction{*} \AgdaBound{p} \AgdaFunction{⁻¹} \AgdaFunction{∎}\<%
\\
%
\\
\>\AgdaFunction{isGroupoid} \AgdaSymbol{:} \AgdaSymbol{(}\AgdaBound{S} \AgdaSymbol{:} \AgdaPrimitiveType{Set}\AgdaSymbol{)} \AgdaSymbol{→} \AgdaPrimitiveType{Set}\<%
\\
\>\AgdaFunction{isGroupoid} \AgdaBound{S} \AgdaSymbol{=} \AgdaSymbol{\{}\AgdaBound{a} \AgdaBound{b} \AgdaSymbol{:} \AgdaBound{S}\AgdaSymbol{\}} \AgdaSymbol{→} \AgdaFunction{isSet} \AgdaSymbol{(}\AgdaBound{a} \AgdaDatatype{≡} \AgdaBound{b}\AgdaSymbol{)}\<%
\\
%
\\
\>\AgdaKeyword{record} \AgdaRecord{pre-GQuotient} \AgdaSymbol{(}\AgdaBound{G} \AgdaSymbol{:} \AgdaRecord{Groupoid}\AgdaSymbol{)} \AgdaSymbol{:} \AgdaPrimitiveType{Set₁} \AgdaKeyword{where}\<%
\\
\>[0]\AgdaIndent{2}{}\<[2]%
\>[2]\AgdaKeyword{open} \AgdaModule{Groupoid} \AgdaBound{G} \AgdaKeyword{renaming} \AgdaSymbol{(}Carrier \AgdaSymbol{to} A\AgdaSymbol{)}\<%
\\
\>[0]\AgdaIndent{2}{}\<[2]%
\>[2]\AgdaKeyword{field}\<%
\\
\>[2]\AgdaIndent{4}{}\<[4]%
\>[4]\AgdaField{Q} \<[8]%
\>[8]\AgdaSymbol{:} \AgdaPrimitiveType{Set}\<%
\\
\>[2]\AgdaIndent{4}{}\<[4]%
\>[4]\AgdaField{[\_]} \AgdaSymbol{:} \AgdaFunction{A} \AgdaSymbol{→} \AgdaBound{Q}\<%
\\
\>[2]\AgdaIndent{4}{}\<[4]%
\>[4]\AgdaField{[\_]⁼} \AgdaSymbol{:} \AgdaBound{[\_]} \AgdaFunction{respects} \AgdaFunction{\_\textasciitilde\_}\<%
\\
\>[2]\AgdaIndent{4}{}\<[4]%
\>[4]\AgdaField{QisGroupoid} \AgdaSymbol{:} \AgdaFunction{isGroupoid} \AgdaBound{Q}\<%
\\
\>[0]\AgdaIndent{2}{}\<[2]%
\>[2]\AgdaKeyword{open} \AgdaModule{Groupoid} \AgdaBound{G} \AgdaKeyword{public} \AgdaKeyword{renaming} \AgdaSymbol{(}Carrier \AgdaSymbol{to} A\AgdaSymbol{)}\<%
\\
%
\\
\>\AgdaKeyword{record} \AgdaRecord{GQuotient} \AgdaSymbol{\{}\AgdaBound{G} \AgdaSymbol{:} \AgdaRecord{Groupoid}\AgdaSymbol{\}(}\AgdaBound{PGQ} \AgdaSymbol{:} \AgdaRecord{pre-GQuotient} \AgdaBound{G}\AgdaSymbol{)} \AgdaSymbol{:} \AgdaPrimitiveType{Set₁} \AgdaKeyword{where}\<%
\\
\>[0]\AgdaIndent{2}{}\<[2]%
\>[2]\AgdaKeyword{open} \AgdaModule{pre-GQuotient} \AgdaBound{PGQ}\<%
\\
\>[0]\AgdaIndent{2}{}\<[2]%
\>[2]\AgdaKeyword{field}\<%
\\
\>[2]\AgdaIndent{4}{}\<[4]%
\>[4]\AgdaField{qelim} \<[12]%
\>[12]\AgdaSymbol{:} \AgdaSymbol{\{}\AgdaBound{B} \AgdaSymbol{:} \AgdaFunction{Q} \AgdaSymbol{→} \AgdaPrimitiveType{Set}\AgdaSymbol{\}}\<%
\\
\>[4]\AgdaIndent{12}{}\<[12]%
\>[12]\AgdaSymbol{→} \AgdaSymbol{(}\AgdaBound{f} \AgdaSymbol{:} \AgdaSymbol{(}\AgdaBound{a} \AgdaSymbol{:} \AgdaFunction{A}\AgdaSymbol{)} \AgdaSymbol{→} \AgdaBound{B} \AgdaFunction{[} \AgdaBound{a} \AgdaFunction{]}\AgdaSymbol{)}\<%
\\
\>[4]\AgdaIndent{12}{}\<[12]%
\>[12]\AgdaSymbol{→} \AgdaSymbol{(∀} \AgdaSymbol{\{}\AgdaBound{a} \AgdaBound{a'}\AgdaSymbol{\}} \AgdaSymbol{→} \AgdaSymbol{(}\AgdaBound{p} \AgdaSymbol{:} \AgdaBound{a} \AgdaFunction{\textasciitilde} \AgdaBound{a'}\AgdaSymbol{)} \AgdaSymbol{→} \AgdaFunction{subst} \AgdaBound{B} \AgdaFunction{[} \AgdaBound{p} \AgdaFunction{]⁼} \AgdaSymbol{(}\AgdaBound{f} \AgdaBound{a}\AgdaSymbol{)} \AgdaDatatype{≡} \AgdaBound{f} \AgdaBound{a'}\AgdaSymbol{)}\<%
\\
\>[4]\AgdaIndent{12}{}\<[12]%
\>[12]\AgdaSymbol{→} \AgdaSymbol{(}\AgdaBound{q} \AgdaSymbol{:} \AgdaFunction{Q}\AgdaSymbol{)} \AgdaSymbol{→} \AgdaBound{B} \AgdaBound{q}\<%
\\
\>[0]\AgdaIndent{4}{}\<[4]%
\>[4]\AgdaField{qelim-β} \AgdaSymbol{:} \AgdaSymbol{∀} \AgdaSymbol{\{}\AgdaBound{B} \AgdaBound{a} \AgdaBound{f}\AgdaSymbol{\}(}\AgdaBound{resp} \AgdaSymbol{:} \AgdaSymbol{(∀} \AgdaSymbol{\{}\AgdaBound{a} \AgdaBound{a'}\AgdaSymbol{\}} \AgdaSymbol{→} \AgdaSymbol{(}\AgdaBound{p} \AgdaSymbol{:} \AgdaBound{a} \AgdaFunction{\textasciitilde} \AgdaBound{a'}\AgdaSymbol{)} \AgdaSymbol{→} \AgdaFunction{subst} \AgdaBound{B} \AgdaFunction{[} \AgdaBound{p} \AgdaFunction{]⁼} \AgdaSymbol{(}\AgdaBound{f} \AgdaBound{a}\AgdaSymbol{)} \AgdaDatatype{≡} \AgdaBound{f} \AgdaBound{a'}\AgdaSymbol{))}\<%
\\
\>[0]\AgdaIndent{12}{}\<[12]%
\>[12]\AgdaSymbol{→} \AgdaBound{qelim} \AgdaSymbol{\{}\AgdaBound{B}\AgdaSymbol{\}} \AgdaBound{f} \AgdaBound{resp} \AgdaFunction{[} \AgdaBound{a} \AgdaFunction{]} \AgdaDatatype{≡} \AgdaBound{f} \AgdaBound{a}\<%
\\
%
\\
%
\\
%
\\
\>\AgdaKeyword{record} \AgdaRecord{ExactGQuotient} \AgdaSymbol{\{}\AgdaBound{G} \AgdaSymbol{:} \AgdaRecord{Groupoid}\AgdaSymbol{\}\{}\AgdaBound{PGQ} \AgdaSymbol{:} \AgdaRecord{pre-GQuotient} \AgdaBound{G}\AgdaSymbol{\}(}\AgdaBound{GQu} \AgdaSymbol{:} \AgdaRecord{GQuotient} \AgdaBound{PGQ}\AgdaSymbol{)} \AgdaSymbol{:} \AgdaPrimitiveType{Set₁} \AgdaKeyword{where}\<%
\\
\>[0]\AgdaIndent{2}{}\<[2]%
\>[2]\AgdaKeyword{open} \AgdaModule{pre-GQuotient} \AgdaBound{PGQ}\<%
\\
\>[0]\AgdaIndent{2}{}\<[2]%
\>[2]\AgdaKeyword{field}\<%
\\
\>[2]\AgdaIndent{4}{}\<[4]%
\>[4]\AgdaField{exact} \<[11]%
\>[11]\AgdaSymbol{:} \AgdaSymbol{∀\{}\AgdaBound{a} \AgdaBound{b} \AgdaSymbol{:} \AgdaFunction{A}\AgdaSymbol{\}} \AgdaSymbol{→} \AgdaFunction{[} \AgdaBound{a} \AgdaFunction{]} \AgdaDatatype{≡} \AgdaFunction{[} \AgdaBound{b} \AgdaFunction{]} \AgdaSymbol{→} \AgdaBound{a} \AgdaFunction{\textasciitilde} \AgdaBound{b}\<%
\\
\>[2]\AgdaIndent{4}{}\<[4]%
\>[4]\AgdaField{isInv₁} \AgdaSymbol{:} \AgdaSymbol{∀\{}\AgdaBound{a} \AgdaBound{b}\AgdaSymbol{\}\{}\AgdaBound{p} \AgdaSymbol{:} \AgdaBound{a} \AgdaFunction{\textasciitilde} \AgdaBound{b}\AgdaSymbol{\}} \AgdaSymbol{→} \AgdaBound{exact} \AgdaFunction{[} \AgdaBound{p} \AgdaFunction{]⁼} \AgdaDatatype{≡} \AgdaBound{p}\<%
\\
\>[2]\AgdaIndent{4}{}\<[4]%
\>[4]\AgdaField{isInv₂} \AgdaSymbol{:} \AgdaSymbol{∀\{}\AgdaBound{a} \AgdaBound{b}\AgdaSymbol{\}\{}\AgdaBound{p} \AgdaSymbol{:} \AgdaFunction{[} \AgdaBound{a} \AgdaFunction{]} \AgdaDatatype{≡} \AgdaFunction{[} \AgdaBound{b} \AgdaFunction{]}\AgdaSymbol{\}} \AgdaSymbol{→} \AgdaFunction{[} \AgdaBound{exact} \AgdaBound{p} \AgdaFunction{]⁼} \AgdaDatatype{≡} \AgdaBound{p}\<%
\\
%
\\
%
\\
%
\\
\>\AgdaKeyword{module} \AgdaModule{UAImpEff}\<%
\\
\>[0]\AgdaIndent{2}{}\<[2]%
\>[2]\AgdaSymbol{(}\AgdaBound{UA} \AgdaSymbol{:} \AgdaSymbol{∀} \AgdaSymbol{\{}\AgdaBound{p} \AgdaBound{q} \AgdaSymbol{:} \AgdaPrimitiveType{Set}\AgdaSymbol{\}} \AgdaSymbol{→} \AgdaSymbol{(}\AgdaBound{p} \AgdaFunction{⇔} \AgdaBound{q}\AgdaSymbol{)} \AgdaSymbol{→} \AgdaBound{p} \AgdaDatatype{≡} \AgdaBound{q}\AgdaSymbol{)}\<%
\\
\>[0]\AgdaIndent{2}{}\<[2]%
\>[2]\AgdaSymbol{(}\AgdaBound{G} \AgdaSymbol{:} \AgdaRecord{Groupoid}\AgdaSymbol{)}\<%
\\
\>[0]\AgdaIndent{2}{}\<[2]%
\>[2]\AgdaSymbol{(}\AgdaBound{PGQ} \AgdaSymbol{:} \AgdaRecord{pre-GQuotient} \AgdaBound{G}\AgdaSymbol{)}\<%
\\
\>[0]\AgdaIndent{2}{}\<[2]%
\>[2]\AgdaSymbol{(}\AgdaBound{GQu} \AgdaSymbol{:} \AgdaRecord{GQuotient} \AgdaBound{PGQ}\AgdaSymbol{)}\<%
\\
\>[2]\AgdaIndent{4}{}\<[4]%
\>[4]\AgdaKeyword{where}\<%
\\
\>[0]\AgdaIndent{2}{}\<[2]%
\>[2]\AgdaKeyword{open} \AgdaModule{pre-GQuotient} \AgdaBound{PGQ}\<%
\\
\>[0]\AgdaIndent{2}{}\<[2]%
\>[2]\AgdaKeyword{open} \AgdaModule{GQuotient} \AgdaBound{GQu}\<%
\\
\>\<\end{code}

There is also an equivalence on the arrow level $\_\sim_{1}\_$ derived from $\_\sim\_$ (higher levels are trivial)

\begin{code}\>\<%
\\
\>[0]\AgdaIndent{2}{}\<[2]%
\>[2]\AgdaFunction{\_\textasciitilde₁\_} \AgdaSymbol{:} \AgdaSymbol{∀\{}\AgdaBound{a} \AgdaBound{a'} \AgdaBound{b} \AgdaBound{b'}\AgdaSymbol{\}} \AgdaSymbol{→} \AgdaBound{a} \AgdaFunction{\textasciitilde} \AgdaBound{a'} \AgdaSymbol{→} \AgdaBound{b} \AgdaFunction{\textasciitilde} \AgdaBound{b'} \AgdaSymbol{→} \AgdaSymbol{(}\AgdaBound{a} \AgdaFunction{\textasciitilde} \AgdaBound{b} \AgdaSymbol{→} \AgdaBound{a'} \AgdaFunction{\textasciitilde} \AgdaBound{b'}\AgdaSymbol{)}\<%
\\
\>[0]\AgdaIndent{2}{}\<[2]%
\>[2]\AgdaFunction{\_\textasciitilde₁\_} \AgdaBound{p} \AgdaBound{q} \AgdaBound{r} \AgdaSymbol{=} \AgdaBound{p} \AgdaFunction{⁻¹} \AgdaFunction{*} \AgdaBound{r} \AgdaFunction{*} \AgdaBound{q}\<%
\\
%
\\
\>[0]\AgdaIndent{2}{}\<[2]%
\>[2]\AgdaFunction{\_\textasciitilde₁I\_} \AgdaSymbol{:} \AgdaSymbol{∀\{}\AgdaBound{a} \AgdaBound{a'} \AgdaBound{b} \AgdaBound{b'}\AgdaSymbol{\}} \AgdaSymbol{→} \AgdaBound{a} \AgdaFunction{\textasciitilde} \AgdaBound{a'} \AgdaSymbol{→} \AgdaBound{b} \AgdaFunction{\textasciitilde} \AgdaBound{b'} \AgdaSymbol{→} \AgdaSymbol{(}\AgdaBound{a'} \AgdaFunction{\textasciitilde} \AgdaBound{b'} \AgdaSymbol{→} \AgdaBound{a} \AgdaFunction{\textasciitilde} \AgdaBound{b}\AgdaSymbol{)}\<%
\\
\>[0]\AgdaIndent{2}{}\<[2]%
\>[2]\AgdaFunction{\_\textasciitilde₁I\_} \AgdaBound{p} \AgdaBound{q} \AgdaBound{r} \AgdaSymbol{=} \AgdaBound{p} \AgdaFunction{*} \AgdaBound{r} \AgdaFunction{*} \AgdaBound{q} \AgdaFunction{⁻¹}\<%
\\
%
\\
\>[0]\AgdaIndent{2}{}\<[2]%
\>[2]\AgdaFunction{\textasciitilde₁-equiv₁} \AgdaSymbol{:} \AgdaSymbol{∀\{}\AgdaBound{a} \AgdaBound{a'} \AgdaBound{b} \AgdaBound{b'}\AgdaSymbol{\}(}\AgdaBound{p} \AgdaSymbol{:} \AgdaBound{a} \AgdaFunction{\textasciitilde} \AgdaBound{a'}\AgdaSymbol{)(}\AgdaBound{q} \AgdaSymbol{:} \AgdaBound{b} \AgdaFunction{\textasciitilde} \AgdaBound{b'}\AgdaSymbol{)(}\AgdaBound{r} \AgdaSymbol{:} \AgdaBound{a} \AgdaFunction{\textasciitilde} \AgdaBound{b}\AgdaSymbol{)}\<%
\\
\>[2]\AgdaIndent{11}{}\<[11]%
\>[11]\AgdaSymbol{→} \AgdaSymbol{(}\AgdaBound{p} \AgdaFunction{\textasciitilde₁I} \AgdaBound{q}\AgdaSymbol{)} \AgdaSymbol{((}\AgdaBound{p} \AgdaFunction{\textasciitilde₁} \AgdaBound{q}\AgdaSymbol{)} \AgdaBound{r}\AgdaSymbol{)} \AgdaDatatype{≡} \AgdaBound{r}\<%
\\
\>[0]\AgdaIndent{2}{}\<[2]%
\>[2]\AgdaFunction{\textasciitilde₁-equiv₁} \AgdaBound{p} \AgdaBound{q} \AgdaBound{r} \AgdaSymbol{=} \AgdaFunction{begin}\<%
\\
\>[0]\AgdaIndent{13}{}\<[13]%
\>[13]\AgdaSymbol{(}\AgdaBound{p} \AgdaFunction{*} \AgdaSymbol{((}\AgdaBound{p} \AgdaFunction{⁻¹} \AgdaFunction{*} \AgdaBound{r}\AgdaSymbol{)} \AgdaFunction{*} \AgdaBound{q}\AgdaSymbol{))} \AgdaFunction{*} \AgdaBound{q} \AgdaFunction{⁻¹} \AgdaFunction{≡⟨} \AgdaFunction{cong} \AgdaSymbol{(λ} \AgdaBound{x} \AgdaSymbol{→} \AgdaBound{x} \AgdaFunction{*} \AgdaBound{q} \AgdaFunction{⁻¹}\AgdaSymbol{)} \AgdaSymbol{(}\AgdaFunction{sym} \AgdaFunction{assoc}\AgdaSymbol{)} \AgdaFunction{⟩}\<%
\\
\>[0]\AgdaIndent{13}{}\<[13]%
\>[13]\AgdaSymbol{((}\AgdaBound{p} \AgdaFunction{*} \AgdaSymbol{(}\AgdaBound{p} \AgdaFunction{⁻¹} \AgdaFunction{*} \AgdaBound{r}\AgdaSymbol{))} \AgdaFunction{*} \AgdaBound{q}\AgdaSymbol{)} \AgdaFunction{*} \AgdaBound{q} \AgdaFunction{⁻¹} \AgdaFunction{≡⟨} \AgdaFunction{cong} \AgdaSymbol{(λ} \AgdaBound{x} \AgdaSymbol{→} \AgdaBound{x} \AgdaFunction{*} \AgdaBound{q} \AgdaFunction{*} \AgdaBound{q} \AgdaFunction{⁻¹}\AgdaSymbol{)} \AgdaSymbol{(}\AgdaFunction{sym} \AgdaFunction{assoc}\AgdaSymbol{)} \AgdaFunction{⟩}\<%
\\
\>[0]\AgdaIndent{13}{}\<[13]%
\>[13]\AgdaSymbol{(((}\AgdaBound{p} \AgdaFunction{*} \AgdaBound{p} \AgdaFunction{⁻¹}\AgdaSymbol{)} \AgdaFunction{*} \AgdaBound{r}\AgdaSymbol{)} \AgdaFunction{*} \AgdaBound{q}\AgdaSymbol{)} \AgdaFunction{*} \AgdaBound{q} \AgdaFunction{⁻¹} \AgdaFunction{≡⟨} \AgdaFunction{cong} \AgdaSymbol{(λ} \AgdaBound{x} \AgdaSymbol{→} \AgdaBound{x} \AgdaFunction{*} \AgdaBound{q} \AgdaFunction{*} \AgdaBound{q} \AgdaFunction{⁻¹}\AgdaSymbol{)} \AgdaSymbol{(}\AgdaFunction{trans} \AgdaSymbol{(}\AgdaFunction{cong} \AgdaSymbol{(λ} \AgdaBound{x} \AgdaSymbol{→} \AgdaBound{x} \AgdaFunction{*} \AgdaSymbol{\_)} \AgdaFunction{inv₁}\AgdaSymbol{)} \AgdaFunction{id₁}\AgdaSymbol{)} \AgdaFunction{⟩}\<%
\\
\>[0]\AgdaIndent{13}{}\<[13]%
\>[13]\AgdaSymbol{(}\AgdaBound{r} \AgdaFunction{*} \AgdaBound{q}\AgdaSymbol{)} \AgdaFunction{*} \AgdaBound{q} \AgdaFunction{⁻¹} \<[42]%
\>[42]\AgdaFunction{≡⟨} \AgdaFunction{assoc} \AgdaFunction{⟩}\<%
\\
\>[0]\AgdaIndent{13}{}\<[13]%
\>[13]\AgdaBound{r} \AgdaFunction{*} \AgdaSymbol{(}\AgdaBound{q} \AgdaFunction{*} \AgdaBound{q} \AgdaFunction{⁻¹}\AgdaSymbol{)} \<[42]%
\>[42]\AgdaFunction{≡⟨} \AgdaFunction{trans} \AgdaSymbol{(}\AgdaFunction{cong} \AgdaSymbol{(λ} \AgdaBound{x} \AgdaSymbol{→} \AgdaBound{r} \AgdaFunction{*} \AgdaBound{x}\AgdaSymbol{)} \AgdaFunction{inv₁}\AgdaSymbol{)} \AgdaFunction{id₂} \AgdaFunction{⟩}\<%
\\
\>[0]\AgdaIndent{13}{}\<[13]%
\>[13]\AgdaBound{r} \AgdaFunction{∎}\<%
\\
%
\\
\>[0]\AgdaIndent{2}{}\<[2]%
\>[2]\AgdaFunction{\textasciitilde₁-equiv₂} \AgdaSymbol{:} \AgdaSymbol{∀\{}\AgdaBound{a} \AgdaBound{a'} \AgdaBound{b} \AgdaBound{b'}\AgdaSymbol{\}(}\AgdaBound{p} \AgdaSymbol{:} \AgdaBound{a} \AgdaFunction{\textasciitilde} \AgdaBound{a'}\AgdaSymbol{)(}\AgdaBound{q} \AgdaSymbol{:} \AgdaBound{b} \AgdaFunction{\textasciitilde} \AgdaBound{b'}\AgdaSymbol{)(}\AgdaBound{r} \AgdaSymbol{:} \AgdaBound{a'} \AgdaFunction{\textasciitilde} \AgdaBound{b'}\AgdaSymbol{)}\<%
\\
\>[0]\AgdaIndent{11}{}\<[11]%
\>[11]\AgdaSymbol{→} \AgdaSymbol{(}\AgdaBound{p} \AgdaFunction{\textasciitilde₁} \AgdaBound{q}\AgdaSymbol{)} \AgdaSymbol{((}\AgdaBound{p} \AgdaFunction{\textasciitilde₁I} \AgdaBound{q}\AgdaSymbol{)} \AgdaBound{r}\AgdaSymbol{)} \AgdaDatatype{≡} \AgdaBound{r}\<%
\\
\>[0]\AgdaIndent{2}{}\<[2]%
\>[2]\AgdaFunction{\textasciitilde₁-equiv₂} \AgdaBound{p} \AgdaBound{q} \AgdaBound{r} \AgdaSymbol{=} \AgdaFunction{begin}\<%
\\
\>[0]\AgdaIndent{13}{}\<[13]%
\>[13]\AgdaSymbol{(}\AgdaBound{p} \AgdaFunction{⁻¹} \AgdaFunction{*} \AgdaSymbol{((}\AgdaBound{p} \AgdaFunction{*} \AgdaBound{r}\AgdaSymbol{)} \AgdaFunction{*} \AgdaBound{q} \AgdaFunction{⁻¹}\AgdaSymbol{))} \AgdaFunction{*} \AgdaBound{q} \AgdaFunction{≡⟨} \AgdaFunction{cong} \AgdaSymbol{(λ} \AgdaBound{x} \AgdaSymbol{→} \AgdaBound{x} \AgdaFunction{*} \AgdaBound{q}\AgdaSymbol{)} \AgdaSymbol{(}\AgdaFunction{sym} \AgdaFunction{assoc}\AgdaSymbol{)} \AgdaFunction{⟩}\<%
\\
\>[0]\AgdaIndent{13}{}\<[13]%
\>[13]\AgdaSymbol{((}\AgdaBound{p} \AgdaFunction{⁻¹} \AgdaFunction{*} \AgdaSymbol{(}\AgdaBound{p} \AgdaFunction{*} \AgdaBound{r}\AgdaSymbol{))} \AgdaFunction{*} \AgdaBound{q} \AgdaFunction{⁻¹}\AgdaSymbol{)} \AgdaFunction{*} \AgdaBound{q} \AgdaFunction{≡⟨} \AgdaFunction{cong} \AgdaSymbol{(λ} \AgdaBound{x} \AgdaSymbol{→} \AgdaBound{x} \AgdaFunction{*} \AgdaBound{q} \<[65]%
\>[65]\AgdaFunction{⁻¹} \AgdaFunction{*} \AgdaBound{q}\AgdaSymbol{)} \AgdaSymbol{(}\AgdaFunction{sym} \AgdaFunction{assoc}\AgdaSymbol{)} \AgdaFunction{⟩}\<%
\\
\>[0]\AgdaIndent{13}{}\<[13]%
\>[13]\AgdaSymbol{(((}\AgdaBound{p} \AgdaFunction{⁻¹} \AgdaFunction{*} \AgdaBound{p}\AgdaSymbol{)} \AgdaFunction{*} \AgdaBound{r}\AgdaSymbol{)} \AgdaFunction{*} \AgdaBound{q} \AgdaFunction{⁻¹}\AgdaSymbol{)} \AgdaFunction{*} \AgdaBound{q} \AgdaFunction{≡⟨} \AgdaFunction{cong} \AgdaSymbol{(λ} \AgdaBound{x} \AgdaSymbol{→} \AgdaBound{x} \AgdaFunction{*} \AgdaBound{q} \AgdaFunction{⁻¹} \AgdaFunction{*} \AgdaBound{q}\AgdaSymbol{)} \AgdaSymbol{(}\AgdaFunction{trans} \AgdaSymbol{(}\AgdaFunction{cong} \AgdaSymbol{(λ} \AgdaBound{x} \AgdaSymbol{→} \AgdaBound{x} \AgdaFunction{*} \AgdaSymbol{\_)} \AgdaFunction{inv₂}\AgdaSymbol{)} \AgdaFunction{id₁}\AgdaSymbol{)} \AgdaFunction{⟩}\<%
\\
\>[0]\AgdaIndent{13}{}\<[13]%
\>[13]\AgdaSymbol{(}\AgdaBound{r} \AgdaFunction{*} \AgdaBound{q} \AgdaFunction{⁻¹}\AgdaSymbol{)} \AgdaFunction{*} \AgdaBound{q} \<[42]%
\>[42]\AgdaFunction{≡⟨} \AgdaFunction{assoc} \AgdaFunction{⟩}\<%
\\
\>[0]\AgdaIndent{13}{}\<[13]%
\>[13]\AgdaBound{r} \AgdaFunction{*} \AgdaSymbol{(}\AgdaBound{q} \AgdaFunction{⁻¹} \AgdaFunction{*} \AgdaBound{q}\AgdaSymbol{)} \<[42]%
\>[42]\AgdaFunction{≡⟨} \AgdaFunction{trans} \AgdaSymbol{(}\AgdaFunction{cong} \AgdaSymbol{(λ} \AgdaBound{x} \AgdaSymbol{→} \AgdaBound{r} \AgdaFunction{*} \AgdaBound{x}\AgdaSymbol{)} \AgdaFunction{inv₂}\AgdaSymbol{)} \AgdaFunction{id₂} \AgdaFunction{⟩}\<%
\\
\>[0]\AgdaIndent{13}{}\<[13]%
\>[13]\AgdaBound{r} \AgdaFunction{∎}\<%
\\
%
\\
\>[0]\AgdaIndent{2}{}\<[2]%
\>[2]\AgdaFunction{\textasciitilde₁-sound} \AgdaSymbol{:} \AgdaSymbol{∀\{}\AgdaBound{a} \AgdaBound{a'} \AgdaBound{b} \AgdaBound{b'}\AgdaSymbol{\}} \AgdaSymbol{→} \AgdaBound{a} \AgdaFunction{\textasciitilde} \AgdaBound{a'} \AgdaSymbol{→} \AgdaBound{b} \AgdaFunction{\textasciitilde} \AgdaBound{b'} \AgdaSymbol{→} \AgdaSymbol{(}\AgdaBound{a} \AgdaFunction{\textasciitilde} \AgdaBound{b}\AgdaSymbol{)} \AgdaDatatype{≡} \AgdaSymbol{(}\AgdaBound{a'} \AgdaFunction{\textasciitilde} \AgdaBound{b'}\AgdaSymbol{)}\<%
\\
\>[0]\AgdaIndent{2}{}\<[2]%
\>[2]\AgdaFunction{\textasciitilde₁-sound} \AgdaBound{p} \AgdaBound{q} \AgdaSymbol{=} \AgdaBound{UA} \AgdaSymbol{((}\AgdaBound{p} \AgdaFunction{\textasciitilde₁} \AgdaBound{q}\AgdaSymbol{)} \AgdaInductiveConstructor{,} \AgdaSymbol{(}\AgdaBound{p} \AgdaFunction{\textasciitilde₁I} \AgdaBound{q}\AgdaSymbol{))}\<%
\\
\>\<\end{code}

Functorial property

\begin{code}\>\<%
\\
\>[0]\AgdaIndent{2}{}\<[2]%
\>[2]\AgdaFunction{\_\textasciitilde₁\_-id} \AgdaSymbol{:} \AgdaSymbol{∀} \AgdaSymbol{\{}\AgdaBound{a}\AgdaSymbol{\}(}\AgdaBound{r} \AgdaSymbol{:} \AgdaBound{a} \AgdaFunction{\textasciitilde} \AgdaBound{a}\AgdaSymbol{)} \AgdaSymbol{→} \AgdaSymbol{(}\AgdaFunction{id} \AgdaFunction{\textasciitilde₁} \AgdaFunction{id}\AgdaSymbol{)} \AgdaBound{r} \AgdaDatatype{≡} \AgdaBound{r}\<%
\\
\>[0]\AgdaIndent{2}{}\<[2]%
\>[2]\AgdaFunction{\_\textasciitilde₁\_-id} \AgdaBound{r} \AgdaSymbol{=} \AgdaFunction{begin}\<%
\\
\>[2]\AgdaIndent{10}{}\<[10]%
\>[10]\AgdaFunction{id} \AgdaFunction{⁻¹} \AgdaFunction{*} \AgdaBound{r} \AgdaFunction{*} \AgdaFunction{id} \AgdaFunction{≡⟨} \AgdaFunction{id₂} \AgdaFunction{⟩}\<%
\\
\>[2]\AgdaIndent{10}{}\<[10]%
\>[10]\AgdaFunction{id} \AgdaFunction{⁻¹} \AgdaFunction{*} \AgdaBound{r} \<[25]%
\>[25]\AgdaFunction{≡⟨} \AgdaFunction{cong} \AgdaSymbol{(λ} \AgdaBound{x} \AgdaSymbol{→} \AgdaBound{x} \AgdaFunction{*} \AgdaBound{r}\AgdaSymbol{)} \AgdaFunction{id-sym} \AgdaFunction{⟩}\<%
\\
\>[2]\AgdaIndent{10}{}\<[10]%
\>[10]\AgdaFunction{id} \AgdaFunction{*} \AgdaBound{r} \<[25]%
\>[25]\AgdaFunction{≡⟨} \AgdaFunction{id₁} \AgdaFunction{⟩}\<%
\\
\>[2]\AgdaIndent{10}{}\<[10]%
\>[10]\AgdaBound{r} \AgdaFunction{∎}\<%
\\
%
\\
\>[0]\AgdaIndent{2}{}\<[2]%
\>[2]\AgdaFunction{\_\textasciitilde₁\_-comp} \AgdaSymbol{:} \AgdaSymbol{∀} \AgdaSymbol{\{}\AgdaBound{a} \AgdaBound{a'} \AgdaBound{a''} \AgdaBound{b} \AgdaBound{b'} \AgdaBound{b''}\AgdaSymbol{\}(}\AgdaBound{p} \AgdaSymbol{:} \AgdaBound{a} \AgdaFunction{\textasciitilde} \AgdaBound{a'}\AgdaSymbol{)(}\AgdaBound{q} \AgdaSymbol{:} \AgdaBound{b} \AgdaFunction{\textasciitilde} \AgdaBound{b'}\AgdaSymbol{)}\<%
\\
\>[0]\AgdaIndent{13}{}\<[13]%
\>[13]\AgdaSymbol{(}\AgdaBound{p'} \AgdaSymbol{:} \AgdaBound{a'} \AgdaFunction{\textasciitilde} \AgdaBound{a''}\AgdaSymbol{)(}\AgdaBound{q'} \AgdaSymbol{:} \AgdaBound{b'} \AgdaFunction{\textasciitilde} \AgdaBound{b''}\AgdaSymbol{)(}\AgdaBound{r} \AgdaSymbol{:} \AgdaBound{a} \AgdaFunction{\textasciitilde} \AgdaBound{b}\AgdaSymbol{)}\<%
\\
\>[0]\AgdaIndent{11}{}\<[11]%
\>[11]\AgdaSymbol{→} \AgdaSymbol{((}\AgdaBound{p} \AgdaFunction{*} \AgdaBound{p'}\AgdaSymbol{)} \AgdaFunction{\textasciitilde₁} \AgdaSymbol{(}\AgdaBound{q} \AgdaFunction{*} \AgdaBound{q'}\AgdaSymbol{))} \AgdaBound{r} \AgdaDatatype{≡} \AgdaSymbol{(}\AgdaBound{p'} \AgdaFunction{\textasciitilde₁} \AgdaBound{q'}\AgdaSymbol{)} \AgdaSymbol{((}\AgdaBound{p} \AgdaFunction{\textasciitilde₁} \AgdaBound{q}\AgdaSymbol{)} \AgdaBound{r}\AgdaSymbol{)}\<%
\\
\>[0]\AgdaIndent{2}{}\<[2]%
\>[2]\AgdaFunction{\_\textasciitilde₁\_-comp} \AgdaBound{p} \AgdaBound{q} \AgdaBound{p'} \AgdaBound{q'} \AgdaBound{r} \AgdaSymbol{=} \AgdaFunction{begin}\<%
\\
\>[2]\AgdaIndent{10}{}\<[10]%
\>[10]\AgdaSymbol{(}\AgdaBound{p} \AgdaFunction{*} \AgdaBound{p'}\AgdaSymbol{)} \AgdaFunction{⁻¹} \AgdaFunction{*} \AgdaBound{r} \AgdaFunction{*} \AgdaSymbol{(}\AgdaBound{q} \AgdaFunction{*} \AgdaBound{q'}\AgdaSymbol{)} \<[39]%
\>[39]\AgdaFunction{≡⟨} \AgdaFunction{cong} \AgdaSymbol{(λ} \AgdaBound{x} \AgdaSymbol{→} \AgdaBound{x} \AgdaFunction{*} \AgdaBound{r} \AgdaFunction{*} \AgdaSymbol{(}\AgdaBound{q} \AgdaFunction{*} \AgdaBound{q'}\AgdaSymbol{))} \AgdaSymbol{(}\AgdaFunction{sym-comp} \AgdaSymbol{\_} \AgdaSymbol{\_)} \AgdaFunction{⟩}\<%
\\
\>[2]\AgdaIndent{10}{}\<[10]%
\>[10]\AgdaSymbol{(}\AgdaBound{p'} \AgdaFunction{⁻¹} \AgdaFunction{*} \AgdaBound{p} \AgdaFunction{⁻¹}\AgdaSymbol{)} \AgdaFunction{*} \AgdaBound{r} \AgdaFunction{*} \AgdaSymbol{(}\AgdaBound{q} \AgdaFunction{*} \AgdaBound{q'}\AgdaSymbol{)} \AgdaFunction{≡⟨} \AgdaFunction{cong} \AgdaSymbol{(λ} \AgdaBound{x} \AgdaSymbol{→} \AgdaBound{x} \AgdaFunction{*} \AgdaSymbol{(}\AgdaBound{q} \AgdaFunction{*} \AgdaBound{q'}\AgdaSymbol{))} \AgdaFunction{assoc} \AgdaFunction{⟩}\<%
\\
\>[2]\AgdaIndent{10}{}\<[10]%
\>[10]\AgdaBound{p'} \AgdaFunction{⁻¹} \AgdaFunction{*} \AgdaSymbol{(}\AgdaBound{p} \AgdaFunction{⁻¹} \AgdaFunction{*} \AgdaBound{r}\AgdaSymbol{)} \AgdaFunction{*} \AgdaSymbol{(}\AgdaBound{q} \AgdaFunction{*} \AgdaBound{q'}\AgdaSymbol{)} \AgdaFunction{≡⟨} \AgdaFunction{sym} \AgdaFunction{assoc} \AgdaFunction{⟩}\<%
\\
\>[2]\AgdaIndent{10}{}\<[10]%
\>[10]\AgdaBound{p'} \AgdaFunction{⁻¹} \AgdaFunction{*} \AgdaSymbol{(}\AgdaBound{p} \AgdaFunction{⁻¹} \AgdaFunction{*} \AgdaBound{r}\AgdaSymbol{)} \AgdaFunction{*} \AgdaBound{q} \AgdaFunction{*} \AgdaBound{q'} \<[40]%
\>[40]\AgdaFunction{≡⟨} \AgdaFunction{cong} \AgdaSymbol{(λ} \AgdaBound{x} \AgdaSymbol{→} \AgdaBound{x} \AgdaFunction{*} \AgdaBound{q'}\AgdaSymbol{)} \AgdaFunction{assoc} \AgdaFunction{⟩}\<%
\\
\>[2]\AgdaIndent{10}{}\<[10]%
\>[10]\AgdaBound{p'} \AgdaFunction{⁻¹} \AgdaFunction{*} \AgdaSymbol{(}\AgdaBound{p} \AgdaFunction{⁻¹} \AgdaFunction{*} \AgdaBound{r} \AgdaFunction{*} \AgdaBound{q}\AgdaSymbol{)} \AgdaFunction{*} \AgdaBound{q'} \<[40]%
\>[40]\AgdaFunction{∎}\<%
\\
%
\\
%
\\
\>[0]\AgdaIndent{2}{}\<[2]%
\>[2]\AgdaKeyword{postulate} \<[12]%
\>[12]\<%
\\
\>[0]\AgdaIndent{4}{}\<[4]%
\>[4]\AgdaPostulate{lift₁} \AgdaSymbol{:} \AgdaSymbol{\{} \AgdaBound{B} \AgdaSymbol{:} \AgdaPrimitiveType{Set₁}\AgdaSymbol{\}} \<[24]%
\>[24]\<%
\\
\>[4]\AgdaIndent{12}{}\<[12]%
\>[12]\AgdaSymbol{(}\AgdaBound{f} \AgdaSymbol{:} \AgdaFunction{A} \AgdaSymbol{→} \AgdaBound{B}\AgdaSymbol{)}\<%
\\
\>[4]\AgdaIndent{12}{}\<[12]%
\>[12]\AgdaSymbol{(}\AgdaBound{f-resp} \AgdaSymbol{:} \AgdaBound{f} \AgdaFunction{respects} \AgdaFunction{\_\textasciitilde\_}\AgdaSymbol{)}\<%
\\
\>[-2]\AgdaIndent{10}{}\<[10]%
\>[10]\AgdaSymbol{→} \AgdaFunction{Q} \AgdaSymbol{→} \AgdaBound{B}\<%
\\
%
\\
\>[0]\AgdaIndent{2}{}\<[2]%
\>[2]\AgdaKeyword{postulate}\<%
\\
\>[2]\AgdaIndent{4}{}\<[4]%
\>[4]\AgdaPostulate{lift-β₁} \AgdaSymbol{:} \AgdaSymbol{∀} \AgdaSymbol{\{}\AgdaBound{B} \AgdaBound{a} \AgdaBound{f}\AgdaSymbol{\}}\<%
\\
\>[4]\AgdaIndent{14}{}\<[14]%
\>[14]\AgdaSymbol{\{}\AgdaBound{f-resp} \AgdaSymbol{:} \AgdaBound{f} \AgdaFunction{respects} \AgdaFunction{\_\textasciitilde\_}\AgdaSymbol{\}}\<%
\\
\>[-2]\AgdaIndent{12}{}\<[12]%
\>[12]\AgdaSymbol{→} \AgdaPostulate{lift₁} \AgdaSymbol{\{}\AgdaBound{B}\AgdaSymbol{\}} \AgdaBound{f} \AgdaBound{f-resp} \AgdaFunction{[} \AgdaBound{a} \AgdaFunction{]} \<[40]%
\>[40]\AgdaDatatype{≡} \AgdaBound{f} \AgdaBound{a}\<%
\\
%
\\
\>[0]\AgdaIndent{2}{}\<[2]%
\>[2]\AgdaFunction{exact} \AgdaSymbol{:} \AgdaSymbol{∀} \AgdaSymbol{\{}\AgdaBound{a} \AgdaBound{a'}\AgdaSymbol{\}} \AgdaSymbol{→} \AgdaFunction{[} \AgdaBound{a} \AgdaFunction{]} \AgdaDatatype{≡} \AgdaFunction{[} \AgdaBound{a'} \AgdaFunction{]} \AgdaSymbol{→} \AgdaBound{a} \AgdaFunction{\textasciitilde} \AgdaBound{a'}\<%
\\
\>[0]\AgdaIndent{2}{}\<[2]%
\>[2]\AgdaFunction{exact} \AgdaSymbol{\{}\AgdaBound{a}\AgdaSymbol{\}} \AgdaSymbol{\{}\AgdaBound{a'}\AgdaSymbol{\}} \AgdaBound{eq} \AgdaSymbol{=} \AgdaFunction{coerce} \AgdaFunction{P\textasciicircum-β} \AgdaSymbol{(}\AgdaFunction{id} \AgdaSymbol{\{}\AgdaBound{a}\AgdaSymbol{\})}\<%
\\
\>[2]\AgdaIndent{8}{}\<[8]%
\>[8]\AgdaKeyword{where}\<%
\\
\>[8]\AgdaIndent{10}{}\<[10]%
\>[10]\AgdaFunction{P} \AgdaSymbol{:} \AgdaFunction{A} \AgdaSymbol{→} \AgdaPrimitiveType{Set}\<%
\\
\>[8]\AgdaIndent{10}{}\<[10]%
\>[10]\AgdaFunction{P} \AgdaBound{x} \AgdaSymbol{=} \AgdaBound{a} \AgdaFunction{\textasciitilde} \AgdaBound{x}\<%
\\
\>\<\end{code}
Functorial
\begin{code}\>\<%
\\
\>[8]\AgdaIndent{10}{}\<[10]%
\>[10]\AgdaFunction{P₁} \AgdaSymbol{:} \AgdaSymbol{∀\{}\AgdaBound{b} \AgdaBound{c}\AgdaSymbol{\}} \AgdaSymbol{→} \AgdaBound{b} \AgdaFunction{\textasciitilde} \AgdaBound{c} \AgdaSymbol{→} \AgdaSymbol{(}\AgdaFunction{P} \AgdaBound{b} \AgdaSymbol{→} \AgdaFunction{P} \AgdaBound{c}\AgdaSymbol{)}\<%
\\
\>[8]\AgdaIndent{10}{}\<[10]%
\>[10]\AgdaFunction{P₁} \AgdaBound{bc} \AgdaBound{ab} \AgdaSymbol{=} \AgdaBound{ab} \AgdaFunction{*} \AgdaBound{bc}\<%
\\
%
\\
\>[8]\AgdaIndent{10}{}\<[10]%
\>[10]\AgdaFunction{P₁-id} \AgdaSymbol{:} \AgdaSymbol{∀\{}\AgdaBound{b}\AgdaSymbol{\}\{}\AgdaBound{r} \AgdaSymbol{:} \AgdaFunction{P} \AgdaBound{b}\AgdaSymbol{\}} \AgdaSymbol{→} \AgdaFunction{P₁} \AgdaFunction{id} \AgdaBound{r} \AgdaDatatype{≡} \AgdaBound{r}\<%
\\
\>[8]\AgdaIndent{10}{}\<[10]%
\>[10]\AgdaFunction{P₁-id} \AgdaSymbol{=} \AgdaFunction{id₂}\<%
\\
%
\\
\>[8]\AgdaIndent{10}{}\<[10]%
\>[10]\AgdaFunction{P₁-comp} \AgdaSymbol{:} \AgdaSymbol{∀\{}\AgdaBound{b} \AgdaBound{c} \AgdaBound{d}\AgdaSymbol{\}\{}\AgdaBound{p} \AgdaSymbol{:} \AgdaBound{b} \AgdaFunction{\textasciitilde} \AgdaBound{c}\AgdaSymbol{\}\{}\AgdaBound{q} \AgdaSymbol{:} \AgdaBound{c} \AgdaFunction{\textasciitilde} \AgdaBound{d}\AgdaSymbol{\}\{}\AgdaBound{r} \AgdaSymbol{:} \AgdaFunction{P} \AgdaBound{b}\AgdaSymbol{\}} \AgdaSymbol{→} \AgdaFunction{P₁} \AgdaSymbol{(}\AgdaBound{p} \AgdaFunction{*} \AgdaBound{q}\AgdaSymbol{)} \AgdaBound{r} \AgdaDatatype{≡} \AgdaFunction{P₁} \AgdaBound{q} \AgdaSymbol{(}\AgdaFunction{P₁} \AgdaBound{p} \AgdaBound{r}\AgdaSymbol{)}\<%
\\
%
\\
\>[8]\AgdaIndent{10}{}\<[10]%
\>[10]\AgdaFunction{P₁-comp} \AgdaSymbol{=} \AgdaFunction{sym} \AgdaFunction{assoc}\<%
\\
\>\<\end{code}


We didn't write the equivalence proof explicitly i.e.\ isEquivalence (P₁ x).

\begin{code}\>\<%
\\
\>[8]\AgdaIndent{10}{}\<[10]%
\>[10]\AgdaFunction{P-resp} \AgdaSymbol{:} \AgdaFunction{P} \AgdaFunction{respects} \AgdaFunction{\_\textasciitilde\_}\<%
\\
\>[8]\AgdaIndent{10}{}\<[10]%
\>[10]\AgdaFunction{P-resp} \AgdaSymbol{\{}\AgdaBound{b}\AgdaSymbol{\}} \AgdaSymbol{\{}\AgdaBound{b'}\AgdaSymbol{\}} \AgdaBound{bb'} \AgdaSymbol{=} \AgdaBound{UA} \AgdaSymbol{(}\AgdaFunction{P₁} \AgdaBound{bb'} \AgdaInductiveConstructor{,} \AgdaFunction{P₁} \AgdaSymbol{(}\AgdaBound{bb'} \AgdaFunction{⁻¹}\AgdaSymbol{))}\<%
\\
%
\\
\>[8]\AgdaIndent{10}{}\<[10]%
\>[10]\AgdaFunction{P\textasciicircum} \AgdaSymbol{:} \AgdaFunction{Q} \AgdaSymbol{→} \AgdaFunction{Prp}\<%
\\
\>[8]\AgdaIndent{10}{}\<[10]%
\>[10]\AgdaFunction{P\textasciicircum} \AgdaSymbol{=} \AgdaPostulate{lift₁} \AgdaFunction{P} \AgdaFunction{P-resp}\<%
\\
%
\\
\>[8]\AgdaIndent{10}{}\<[10]%
\>[10]\AgdaFunction{P\textasciicircum-β} \AgdaSymbol{:} \AgdaFunction{P} \AgdaBound{a} \AgdaDatatype{≡} \AgdaFunction{P} \AgdaBound{a'}\<%
\\
\>[8]\AgdaIndent{10}{}\<[10]%
\>[10]\AgdaFunction{P\textasciicircum-β} \AgdaSymbol{=} \AgdaFunction{trans} \AgdaSymbol{(}\AgdaFunction{sym} \AgdaPostulate{lift-β₁}\AgdaSymbol{)} \AgdaSymbol{(}\AgdaFunction{trans} \AgdaSymbol{(}\AgdaFunction{cong} \AgdaFunction{P\textasciicircum} \AgdaBound{eq}\AgdaSymbol{)} \AgdaPostulate{lift-β₁}\AgdaSymbol{)}\<%
\\
%
\\
%
\\
\>\AgdaComment{-- coerce (trans (sym lift-β₁) (trans (cong (lift₁ P P-resp) [ p ]⁼) lift-β₁)) (id \{a\}) ≡ p}\<%
\\
\>\AgdaComment{--  isInv₁ : ∀\{a b\}\{p : a \textasciitilde b\} → exact [ p ]⁼ ≡ p}\<%
\\
\>\AgdaComment{--  isInv₁ \{p = p\} = \{!!\}}\<%
\\
%
\\
\>\AgdaComment{--  isInv₂ : ∀\{a b\}\{p : [ a ] ≡ [ b ]\} → [ exact p ]⁼ ≡ p}\<%
\\
%
\\
%
\\
\>\<\end{code}

\chapter{Category with families of setoids}\label{app:cwf}

\AgdaHide{
\begin{code}\>\<%
\\
%
\\
\>\AgdaKeyword{module} \AgdaModule{CategoryofSetoid} \AgdaKeyword{where}\<%
\\
%
\\
%
\\
\>\AgdaKeyword{open} \AgdaKeyword{import} \AgdaModule{Relation.Binary.PropositionalEquality} \AgdaSymbol{as} \AgdaModule{PE}\<%
\\
\>[0]\AgdaIndent{2}{}\<[2]%
\>[2]\AgdaKeyword{hiding} \AgdaSymbol{(}[\_] \AgdaSymbol{;} refl\AgdaSymbol{;} sym \AgdaSymbol{;} trans\AgdaSymbol{;} isEquivalence\AgdaSymbol{)}\<%
\\
\>\AgdaKeyword{open} \AgdaKeyword{import} \AgdaModule{Level}\<%
\\
\>\AgdaKeyword{open} \AgdaKeyword{import} \AgdaModule{Data.Unit}\<%
\\
%
\\
\>\<\end{code}
}

\section{Metatheory}

Subset defined by a predicate $B$

\begin{code}\>\<%
\\
\>\AgdaKeyword{record} \AgdaRecord{Subset} \AgdaSymbol{\{}\AgdaBound{a} \AgdaBound{b}\AgdaSymbol{\}} \AgdaSymbol{(}\AgdaBound{A} \AgdaSymbol{:} \AgdaPrimitiveType{Set} \AgdaBound{a}\AgdaSymbol{)} \<[32]%
\>[32]\<%
\\
\>[2]\AgdaIndent{7}{}\<[7]%
\>[7]\AgdaSymbol{(}\AgdaBound{B} \AgdaSymbol{:} \AgdaBound{A} \AgdaSymbol{→} \AgdaPrimitiveType{Set} \AgdaBound{b}\AgdaSymbol{)} \AgdaSymbol{:} \AgdaPrimitiveType{Set} \AgdaSymbol{(}\AgdaBound{a} \AgdaPrimitive{⊔} \AgdaBound{b}\AgdaSymbol{)} \AgdaKeyword{where}\<%
\\
\>[0]\AgdaIndent{2}{}\<[2]%
\>[2]\AgdaKeyword{constructor} \AgdaInductiveConstructor{\_,\_}\<%
\\
\>[0]\AgdaIndent{2}{}\<[2]%
\>[2]\AgdaKeyword{field}\<%
\\
\>[2]\AgdaIndent{4}{}\<[4]%
\>[4]\AgdaField{prj₁} \AgdaSymbol{:} \AgdaBound{A}\<%
\\
\>[2]\AgdaIndent{4}{}\<[4]%
\>[4]\AgdaSymbol{.}\AgdaField{prj₂} \AgdaSymbol{:} \AgdaBound{B} \AgdaBound{prj₁}\<%
\\
\>\AgdaKeyword{open} \AgdaModule{Subset} \AgdaKeyword{public}\<%
\\
\>\<\end{code}

Setoids 

\begin{code}\>\<%
\\
\>\AgdaKeyword{record} \AgdaRecord{Setoid} \AgdaSymbol{:} \AgdaPrimitiveType{Set₁} \AgdaKeyword{where}\<%
\\
\>[0]\AgdaIndent{2}{}\<[2]%
\>[2]\AgdaKeyword{infix} \AgdaNumber{4} \_≈\_\<%
\\
\>[0]\AgdaIndent{2}{}\<[2]%
\>[2]\AgdaKeyword{field}\<%
\\
\>[2]\AgdaIndent{4}{}\<[4]%
\>[4]\AgdaField{Carrier} \AgdaSymbol{:} \AgdaPrimitiveType{Set}\<%
\\
\>[2]\AgdaIndent{4}{}\<[4]%
\>[4]\AgdaField{\_≈\_} \<[12]%
\>[12]\AgdaSymbol{:} \AgdaBound{Carrier} \AgdaSymbol{→} \AgdaBound{Carrier} \AgdaSymbol{→} \AgdaPrimitiveType{Set}\<%
\\
\>[2]\AgdaIndent{4}{}\<[4]%
\>[4]\AgdaSymbol{.}\AgdaField{refl} \<[12]%
\>[12]\AgdaSymbol{:} \AgdaSymbol{∀\{}\AgdaBound{x}\AgdaSymbol{\}} \AgdaSymbol{→} \AgdaBound{x} \AgdaBound{≈} \AgdaBound{x}\<%
\\
\>[2]\AgdaIndent{4}{}\<[4]%
\>[4]\AgdaSymbol{.}\AgdaField{sym} \<[12]%
\>[12]\AgdaSymbol{:} \AgdaSymbol{∀\{}\AgdaBound{x} \AgdaBound{y}\AgdaSymbol{\}} \AgdaSymbol{→} \AgdaBound{x} \AgdaBound{≈} \AgdaBound{y} \AgdaSymbol{→} \AgdaBound{y} \AgdaBound{≈} \AgdaBound{x}\<%
\\
\>[2]\AgdaIndent{4}{}\<[4]%
\>[4]\AgdaSymbol{.}\AgdaField{trans} \<[12]%
\>[12]\AgdaSymbol{:} \AgdaSymbol{∀\{}\AgdaBound{x} \AgdaBound{y} \AgdaBound{z}\AgdaSymbol{\}} \AgdaSymbol{→} \AgdaBound{x} \AgdaBound{≈} \AgdaBound{y} \AgdaSymbol{→} \AgdaBound{y} \AgdaBound{≈} \AgdaBound{z} \AgdaSymbol{→} \AgdaBound{x} \AgdaBound{≈} \AgdaBound{z}\<%
\\
\>\AgdaKeyword{open} \AgdaModule{Setoid} \AgdaKeyword{public} \AgdaKeyword{renaming} \<[28]%
\>[28]\<%
\\
\>[4]\AgdaIndent{5}{}\<[5]%
\>[5]\AgdaSymbol{(}Carrier \AgdaSymbol{to} ∣\_∣ \AgdaSymbol{;} \_≈\_ \AgdaSymbol{to} [\_]\_≈\_ \AgdaSymbol{;} refl \AgdaSymbol{to} [\_]refl\AgdaSymbol{;}\<%
\\
\>[5] trans \AgdaSymbol{to} [\_]trans\AgdaSymbol{;} sym \AgdaSymbol{to} [\_]sym\AgdaSymbol{)} \<[95]%
\>[95]\<%
\\
\>\<\end{code}

Morphisms between Setoids (Functors)


\begin{code}\>\<%
\\
\>\AgdaKeyword{infix} \AgdaNumber{5} \_⇉\_\<%
\\
%
\\
\>\AgdaKeyword{record} \AgdaRecord{\_⇉\_} \AgdaSymbol{(}\AgdaBound{A} \AgdaBound{B} \AgdaSymbol{:} \AgdaRecord{Setoid}\AgdaSymbol{)} \AgdaSymbol{:} \AgdaPrimitiveType{Set} \AgdaKeyword{where}\<%
\\
\>[-1]\AgdaIndent{2}{}\<[2]%
\>[2]\AgdaKeyword{constructor} \AgdaInductiveConstructor{fn:\_resp:\_}\<%
\\
\>[0]\AgdaIndent{2}{}\<[2]%
\>[2]\AgdaKeyword{field}\<%
\\
\>[2]\AgdaIndent{4}{}\<[4]%
\>[4]\AgdaField{fn} \<[9]%
\>[9]\AgdaSymbol{:} \AgdaFunction{∣} \AgdaBound{A} \AgdaFunction{∣} \AgdaSymbol{→} \AgdaFunction{∣} \AgdaBound{B} \AgdaFunction{∣}\<%
\\
\>[2]\AgdaIndent{4}{}\<[4]%
\>[4]\AgdaSymbol{.}\AgdaField{resp} \AgdaSymbol{:} \AgdaSymbol{\{}\AgdaBound{x} \AgdaBound{y} \AgdaSymbol{:} \AgdaFunction{∣} \AgdaBound{A} \AgdaFunction{∣}\AgdaSymbol{\}} \AgdaSymbol{→} \<[28]%
\>[28]\<%
\\
\>[4]\AgdaIndent{11}{}\<[11]%
\>[11]\AgdaSymbol{(}\AgdaFunction{[} \AgdaBound{A} \AgdaFunction{]} \AgdaBound{x} \AgdaFunction{≈} \AgdaBound{y}\AgdaSymbol{)} \AgdaSymbol{→} \<[27]%
\>[27]\<%
\\
\>[4]\AgdaIndent{11}{}\<[11]%
\>[11]\AgdaFunction{[} \AgdaBound{B} \AgdaFunction{]} \AgdaBound{fn} \AgdaBound{x} \AgdaFunction{≈} \AgdaBound{fn} \AgdaBound{y}\<%
\\
\>\AgdaKeyword{open} \AgdaModule{\_⇉\_} \AgdaKeyword{public} \AgdaKeyword{renaming} \AgdaSymbol{(}fn \AgdaSymbol{to} [\_]fn \AgdaSymbol{;} resp \AgdaSymbol{to} [\_]resp\AgdaSymbol{)}\<%
\\
\>\<\end{code}

Terminal object

\begin{code}\>\<%
\\
\>\AgdaFunction{●} \AgdaSymbol{:} \AgdaRecord{Setoid}\<%
\\
\>\AgdaFunction{●} \<[4]%
\>[4]\AgdaSymbol{=} \AgdaKeyword{record} \AgdaSymbol{\{}\<%
\\
\>[1]\AgdaIndent{6}{}\<[6]%
\>[6]\AgdaField{Carrier} \AgdaSymbol{=} \AgdaRecord{⊤}\AgdaSymbol{;}\<%
\\
\>[0]\AgdaIndent{6}{}\<[6]%
\>[6]\AgdaField{\_≈\_} \<[13]%
\>[13]\AgdaSymbol{=} \AgdaSymbol{λ} \AgdaBound{\_} \AgdaBound{\_} \AgdaSymbol{→} \AgdaRecord{⊤}\AgdaSymbol{;}\<%
\\
\>[0]\AgdaIndent{6}{}\<[6]%
\>[6]\AgdaField{refl} \<[14]%
\>[14]\AgdaSymbol{=} \AgdaInductiveConstructor{tt}\AgdaSymbol{;}\<%
\\
\>[0]\AgdaIndent{6}{}\<[6]%
\>[6]\AgdaField{sym} \<[14]%
\>[14]\AgdaSymbol{=} \AgdaSymbol{λ} \AgdaBound{\_} \AgdaSymbol{→} \AgdaInductiveConstructor{tt}\AgdaSymbol{;}\<%
\\
\>[0]\AgdaIndent{6}{}\<[6]%
\>[6]\AgdaField{trans} \<[14]%
\>[14]\AgdaSymbol{=} \AgdaSymbol{λ} \AgdaBound{\_} \AgdaBound{\_} \AgdaSymbol{→} \AgdaInductiveConstructor{tt} \AgdaSymbol{\}}\<%
\\
%
\\
\>\AgdaFunction{⋆} \AgdaSymbol{:} \AgdaSymbol{\{}\AgdaBound{Δ} \AgdaSymbol{:} \AgdaRecord{Setoid}\AgdaSymbol{\}} \AgdaSymbol{→} \AgdaBound{Δ} \AgdaRecord{⇉} \AgdaFunction{●}\<%
\\
\>\AgdaFunction{⋆} \AgdaSymbol{=} \AgdaKeyword{record} \<[11]%
\>[11]\<%
\\
\>[0]\AgdaIndent{6}{}\<[6]%
\>[6]\AgdaSymbol{\{} \AgdaField{fn} \AgdaSymbol{=} \AgdaSymbol{λ} \AgdaBound{\_} \AgdaSymbol{→} \AgdaInductiveConstructor{tt}\<%
\\
\>[0]\AgdaIndent{6}{}\<[6]%
\>[6]\AgdaSymbol{;} \AgdaField{resp} \AgdaSymbol{=} \AgdaSymbol{λ} \AgdaBound{\_} \AgdaSymbol{→} \AgdaInductiveConstructor{tt} \AgdaSymbol{\}}\<%
\\
%
\\
\>\AgdaFunction{uniqueHom} \AgdaSymbol{:} \AgdaSymbol{∀} \AgdaSymbol{(}\AgdaBound{Δ} \AgdaSymbol{:} \AgdaRecord{Setoid}\AgdaSymbol{)} \<[27]%
\>[27]\<%
\\
\>[6]\AgdaIndent{10}{}\<[10]%
\>[10]\AgdaSymbol{→} \AgdaSymbol{(}\AgdaBound{f} \AgdaSymbol{:} \AgdaBound{Δ} \AgdaRecord{⇉} \AgdaFunction{●}\AgdaSymbol{)} \AgdaSymbol{→} \AgdaBound{f} \AgdaDatatype{≡} \AgdaFunction{⋆}\<%
\\
\>\AgdaFunction{uniqueHom} \AgdaBound{Δ} \AgdaBound{f} \AgdaSymbol{=} \AgdaInductiveConstructor{PE.refl}\<%
\\
\>\<\end{code}


\AgdaHide{

\begin{code}\>\<%
\\
\>\AgdaSymbol{\{-\#} \AgdaKeyword{OPTIONS} --type-in-type \AgdaSymbol{\#-\}}\<%
\\
%
\\
%
\\
\>\AgdaKeyword{open} \AgdaKeyword{import} \AgdaModule{Level} \AgdaKeyword{hiding} \AgdaSymbol{(}lift\AgdaSymbol{)}\<%
\\
\>\AgdaKeyword{open} \AgdaKeyword{import} \AgdaModule{Relation.Binary.PropositionalEquality} \AgdaSymbol{as} \AgdaModule{PE} \AgdaKeyword{hiding} \AgdaSymbol{(}refl \AgdaSymbol{;} sym \AgdaSymbol{;} trans\AgdaSymbol{;} isEquivalence\AgdaSymbol{;} [\_]\AgdaSymbol{)}\<%
\\
%
\\
\>\AgdaKeyword{module} \AgdaModule{CwF-setoid} \AgdaSymbol{(}\AgdaBound{ext} \AgdaSymbol{:} \AgdaFunction{Extensionality} \AgdaPrimitive{zero} \AgdaPrimitive{zero}\AgdaSymbol{)} \AgdaKeyword{where}\<%
\\
%
\\
%
\\
\>\AgdaKeyword{open} \AgdaKeyword{import} \AgdaModule{Cats.Category}\<%
\\
\>\AgdaKeyword{open} \AgdaKeyword{import} \AgdaModule{Cats.Functor}\<%
\\
\>\AgdaKeyword{open} \AgdaKeyword{import} \AgdaModule{Cats.Duality}\<%
\\
\>\AgdaKeyword{open} \AgdaKeyword{import} \AgdaModule{Data.Product} \AgdaKeyword{renaming} \AgdaSymbol{(}<\_,\_> \AgdaSymbol{to} ⟨\_,\_⟩\AgdaSymbol{)}\<%
\\
\>\AgdaKeyword{open} \AgdaKeyword{import} \AgdaModule{Function}\<%
\\
%
\\
\>\AgdaKeyword{open} \AgdaKeyword{import} \AgdaModule{Relation.Binary.Core} \AgdaKeyword{using} \AgdaSymbol{(}\_≡\_\AgdaSymbol{;} \_≢\_\AgdaSymbol{)}\<%
\\
\>\AgdaKeyword{open} \AgdaKeyword{import} \AgdaModule{Data.Unit}\<%
\\
%
\\
\>\AgdaKeyword{import} \AgdaModule{CategoryOfSetoid}\<%
\\
\>\AgdaKeyword{module} \AgdaModule{cos} \AgdaSymbol{=} \AgdaModule{CategoryOfSetoid} \AgdaBound{ext}\<%
\\
\>\AgdaKeyword{open} \AgdaModule{cos}\<%
\\
%
\\
\>\AgdaKeyword{import} \AgdaModule{hProp}\<%
\\
\>\AgdaKeyword{module} \AgdaModule{hp} \AgdaSymbol{=} \AgdaModule{hProp} \AgdaBound{ext}\<%
\\
\>\AgdaKeyword{open} \AgdaModule{hp}\<%
\\
%
\\
\>\AgdaKeyword{infixl} \AgdaNumber{5} \_\&\_\<%
\\
%
\\
\>\<\end{code}
}

We would like to show two formalisation of category with families for setoids here. The first one is simple and short but not comprehensive. We have to extract all complicated components from the simple definition. However the second one gives these components one by one so that it more understandable and convenient.

The category with families works as a model for type theory. So we will introduce them from a type theoretical point of view.

The base category is the category for contexts. In the setoid version we interpret a context as a setoid as well.

To define the second component, namely the presheaf functor, it is necessary to construct the target category first. The objects of this category are families of setoids.The index setoids are the semantic types and the indexed families of setoids are terms. The morphisms are component-wise morphisms between setoids. All the categorical laws hold trivially.

\begin{code}\>\<%
\\
%
\\
\>\AgdaFunction{inxSetoids} \AgdaSymbol{:} \AgdaPrimitiveType{Set₁}\<%
\\
\>\AgdaFunction{inxSetoids} \AgdaSymbol{=} \AgdaRecord{Σ[} \AgdaBound{I} \AgdaRecord{∶} \AgdaRecord{hSetoid} \AgdaRecord{]} \AgdaSymbol{(}\AgdaFunction{∣} \AgdaBound{I} \AgdaFunction{∣} \AgdaSymbol{→} \AgdaRecord{hSetoid}\AgdaSymbol{)}\<%
\\
%
\\
\>\AgdaFunction{\_⇉setoid\_} \AgdaSymbol{:} \AgdaFunction{inxSetoids} \AgdaSymbol{→} \AgdaFunction{inxSetoids} \AgdaSymbol{→} \AgdaPrimitiveType{Set₁}\<%
\\
\>\AgdaSymbol{(}\AgdaBound{I} \AgdaInductiveConstructor{,} \AgdaBound{f}\AgdaSymbol{)} \AgdaFunction{⇉setoid} \AgdaSymbol{(}\AgdaBound{J} \AgdaInductiveConstructor{,} \AgdaBound{g}\AgdaSymbol{)} \AgdaSymbol{=} \<[26]%
\>[26]\<%
\\
\>[0]\AgdaIndent{2}{}\<[2]%
\>[2]\AgdaRecord{Σ[} \AgdaBound{i-map} \AgdaRecord{∶} \AgdaBound{I} \AgdaRecord{⇉} \AgdaBound{J} \AgdaRecord{]}\<%
\\
\>[2]\AgdaIndent{4}{}\<[4]%
\>[4]\AgdaSymbol{((}\AgdaBound{i} \AgdaSymbol{:} \AgdaFunction{∣} \AgdaBound{I} \AgdaFunction{∣}\AgdaSymbol{)} \AgdaSymbol{→} \AgdaBound{f} \AgdaBound{i} \AgdaRecord{⇉} \AgdaBound{g} \AgdaSymbol{(} \AgdaFunction{[} \AgdaBound{i-map} \AgdaFunction{]fn} \AgdaBound{i}\AgdaSymbol{))}\<%
\\
%
\\
\>\AgdaFunction{Fam-setoid} \AgdaSymbol{:} \AgdaRecord{Category}\<%
\\
\>\AgdaFunction{Fam-setoid} \AgdaSymbol{=} \AgdaInductiveConstructor{CatC} \<[18]%
\>[18]\<%
\\
\>[4]\AgdaIndent{15}{}\<[15]%
\>[15]\AgdaFunction{inxSetoids} \<[26]%
\>[26]\<%
\\
\>[4]\AgdaIndent{15}{}\<[15]%
\>[15]\AgdaFunction{\_⇉setoid\_} \<[25]%
\>[25]\<%
\\
\>[4]\AgdaIndent{15}{}\<[15]%
\>[15]\AgdaSymbol{(λ} \AgdaBound{\_} \AgdaSymbol{→} \AgdaFunction{id'} \AgdaInductiveConstructor{,} \AgdaSymbol{(λ} \AgdaBound{\_} \AgdaSymbol{→} \AgdaFunction{id'}\AgdaSymbol{))} \<[41]%
\>[41]\<%
\\
\>[4]\AgdaIndent{15}{}\<[15]%
\>[15]\AgdaSymbol{(λ} \AgdaSymbol{\{} \AgdaSymbol{\_} \AgdaSymbol{\_} \AgdaSymbol{(}\AgdaBound{fty} \AgdaInductiveConstructor{,} \AgdaBound{ftm}\AgdaSymbol{)} \AgdaSymbol{(}\AgdaBound{gty} \AgdaInductiveConstructor{,} \AgdaBound{gtm}\AgdaSymbol{)} \AgdaSymbol{→} \AgdaBound{fty} \AgdaFunction{∘c} \AgdaBound{gty} \AgdaInductiveConstructor{,}\<%
\\
\>[15]\AgdaIndent{17}{}\<[17]%
\>[17]\AgdaSymbol{(λ} \AgdaBound{i} \AgdaSymbol{→} \AgdaBound{ftm} \AgdaSymbol{(}\AgdaFunction{[} \AgdaBound{gty} \AgdaFunction{]fn} \AgdaBound{i}\AgdaSymbol{)} \AgdaFunction{∘c} \AgdaBound{gtm} \AgdaBound{i}\AgdaSymbol{)\})}\<%
\\
\>[0]\AgdaIndent{15}{}\<[15]%
\>[15]\AgdaSymbol{(}\AgdaInductiveConstructor{IsCatC} \<[23]%
\>[23]\<%
\\
\>[0]\AgdaIndent{17}{}\<[17]%
\>[17]\AgdaSymbol{(λ} \AgdaBound{α} \AgdaBound{β} \AgdaBound{f} \AgdaSymbol{→} \AgdaInductiveConstructor{PE.refl}\AgdaSymbol{)} \<[37]%
\>[37]\<%
\\
\>[0]\AgdaIndent{17}{}\<[17]%
\>[17]\AgdaSymbol{(λ} \AgdaBound{α} \AgdaBound{β} \AgdaBound{f} \AgdaSymbol{→} \AgdaInductiveConstructor{PE.refl}\AgdaSymbol{)} \<[37]%
\>[37]\<%
\\
\>[0]\AgdaIndent{17}{}\<[17]%
\>[17]\AgdaSymbol{(λ} \AgdaBound{α} \AgdaBound{δ} \AgdaBound{f} \AgdaBound{g} \AgdaBound{h} \AgdaSymbol{→} \AgdaInductiveConstructor{PE.refl}\AgdaSymbol{))}\<%
\\
%
\\
\>\<\end{code}

Since we already specify the category of contexts, we only need the presheaf which is a contravariant functor from the category of contexts to the category we defined above. The definition of category with families of setoids could be as simple as follows.

\begin{code}\>\<%
\\
%
\\
\>\AgdaKeyword{record} \AgdaRecord{CWF-setoid} \AgdaSymbol{:} \AgdaPrimitiveType{Set₁} \AgdaKeyword{where}\<%
\\
\>[0]\AgdaIndent{2}{}\<[2]%
\>[2]\AgdaKeyword{field}\<%
\\
\>[0]\AgdaIndent{4}{}\<[4]%
\>[4]\AgdaField{T} \AgdaSymbol{:} \AgdaRecord{Functor} \AgdaSymbol{(}\AgdaFunction{Op} \AgdaFunction{setoid-Cat}\AgdaSymbol{)} \AgdaFunction{Fam-setoid}\<%
\\
%
\\
\>\<\end{code}

All details of this definition are hidden including the functor laws. Therefore we will show the details as the second version.

The semantic contexts are setoids and the terminal object is just the empty context. 

\begin{code}\>\<%
\\
%
\\
\>\AgdaFunction{Con} \AgdaSymbol{=} \AgdaRecord{hSetoid}\<%
\\
%
\\
\>\AgdaFunction{emptyCon} \AgdaSymbol{=} \AgdaFunction{⊤-setoid}\<%
\\
%
\\
\>\AgdaFunction{emptysub} \AgdaSymbol{=} \AgdaFunction{⋆}\<%
\\
%
\\
\>\<\end{code}

A semantic type has following components. $fm$ is a setoid of all types. $substT$ is the substitution between types within the context. It should be a morphism between setoids so it has to preserve the equivalence relation. We also need to specify the computation rules for substitution.

\begin{code}\>\<%
\\
%
\\
\>\AgdaKeyword{record} \AgdaRecord{Ty} \AgdaSymbol{(}\AgdaBound{Γ} \AgdaSymbol{:} \AgdaFunction{Con}\AgdaSymbol{)} \AgdaSymbol{:} \AgdaPrimitiveType{Set₁} \AgdaKeyword{where}\<%
\\
\>[0]\AgdaIndent{2}{}\<[2]%
\>[2]\AgdaKeyword{field}\<%
\\
\>[0]\AgdaIndent{4}{}\<[4]%
\>[4]\AgdaField{fm} \<[11]%
\>[11]\AgdaSymbol{:} \AgdaFunction{∣} \AgdaBound{Γ} \AgdaFunction{∣} \AgdaSymbol{→} \AgdaRecord{hSetoid}\<%
\\
%
\\
\>[0]\AgdaIndent{4}{}\<[4]%
\>[4]\AgdaField{substT} \AgdaSymbol{:} \AgdaSymbol{\{}\AgdaBound{x} \AgdaBound{y} \AgdaSymbol{:} \AgdaFunction{∣} \AgdaBound{Γ} \AgdaFunction{∣}\AgdaSymbol{\}} \AgdaSymbol{→} \<[29]%
\>[29]\<%
\\
\>[4]\AgdaIndent{13}{}\<[13]%
\>[13]\AgdaFunction{[} \AgdaBound{Γ} \AgdaFunction{]} \AgdaBound{x} \AgdaFunction{≈} \AgdaBound{y} \AgdaSymbol{→}\<%
\\
\>[4]\AgdaIndent{13}{}\<[13]%
\>[13]\AgdaFunction{∣} \AgdaBound{fm} \AgdaBound{x} \AgdaFunction{∣} \AgdaSymbol{→}\<%
\\
\>[4]\AgdaIndent{13}{}\<[13]%
\>[13]\AgdaFunction{∣} \AgdaBound{fm} \AgdaBound{y} \AgdaFunction{∣}\<%
\\
\>[0]\AgdaIndent{4}{}\<[4]%
\>[4]\AgdaField{subst*} \AgdaSymbol{:} \AgdaSymbol{∀\{}\AgdaBound{x} \AgdaBound{y} \AgdaSymbol{:} \AgdaFunction{∣} \AgdaBound{Γ} \AgdaFunction{∣}\AgdaSymbol{\}}\<%
\\
\>[0]\AgdaIndent{13}{}\<[13]%
\>[13]\AgdaSymbol{(}\AgdaBound{p} \AgdaSymbol{:} \AgdaFunction{[} \AgdaBound{Γ} \AgdaFunction{]} \AgdaBound{x} \AgdaFunction{≈} \AgdaBound{y}\AgdaSymbol{)}\<%
\\
\>[0]\AgdaIndent{13}{}\<[13]%
\>[13]\AgdaSymbol{\{}\AgdaBound{a} \AgdaBound{b} \AgdaSymbol{:} \AgdaFunction{∣} \AgdaBound{fm} \AgdaBound{x} \AgdaFunction{∣}\AgdaSymbol{\}} \AgdaSymbol{→}\<%
\\
\>[0]\AgdaIndent{13}{}\<[13]%
\>[13]\AgdaFunction{[} \AgdaBound{fm} \AgdaBound{x} \AgdaFunction{]} \AgdaBound{a} \AgdaFunction{≈} \AgdaBound{b} \AgdaSymbol{→}\<%
\\
\>[0]\AgdaIndent{13}{}\<[13]%
\>[13]\AgdaFunction{[} \AgdaBound{fm} \AgdaBound{y} \AgdaFunction{]} \AgdaBound{substT} \AgdaBound{p} \AgdaBound{a} \AgdaFunction{≈} \AgdaBound{substT} \AgdaBound{p} \AgdaBound{b}\<%
\\
%
\\
\>[0]\AgdaIndent{4}{}\<[4]%
\>[4]\AgdaField{refl*} \<[11]%
\>[11]\AgdaSymbol{:} \AgdaSymbol{∀(}\AgdaBound{x} \AgdaSymbol{:} \AgdaFunction{∣} \AgdaBound{Γ} \AgdaFunction{∣}\AgdaSymbol{)}\<%
\\
\>[0]\AgdaIndent{13}{}\<[13]%
\>[13]\AgdaSymbol{(}\AgdaBound{a} \AgdaSymbol{:} \AgdaFunction{∣} \AgdaBound{fm} \AgdaBound{x} \AgdaFunction{∣}\AgdaSymbol{)} \AgdaSymbol{→} \<[30]%
\>[30]\<%
\\
\>[0]\AgdaIndent{13}{}\<[13]%
\>[13]\AgdaFunction{[} \AgdaBound{fm} \AgdaBound{x} \AgdaFunction{]} \AgdaBound{substT} \AgdaFunction{[} \AgdaBound{Γ} \AgdaFunction{]refl} \AgdaBound{a} \AgdaFunction{≈} \AgdaBound{a}\<%
\\
\>[0]\AgdaIndent{4}{}\<[4]%
\>[4]\AgdaField{trans*} \AgdaSymbol{:} \AgdaSymbol{∀\{}\AgdaBound{x} \AgdaBound{y} \AgdaBound{z} \AgdaSymbol{:} \AgdaFunction{∣} \AgdaBound{Γ} \AgdaFunction{∣}\AgdaSymbol{\}}\<%
\\
\>[0]\AgdaIndent{13}{}\<[13]%
\>[13]\AgdaSymbol{(}\AgdaBound{p} \AgdaSymbol{:} \AgdaFunction{[} \AgdaBound{Γ} \AgdaFunction{]} \AgdaBound{x} \AgdaFunction{≈} \AgdaBound{y}\AgdaSymbol{)}\<%
\\
\>[0]\AgdaIndent{13}{}\<[13]%
\>[13]\AgdaSymbol{(}\AgdaBound{q} \AgdaSymbol{:} \AgdaFunction{[} \AgdaBound{Γ} \AgdaFunction{]} \AgdaBound{y} \AgdaFunction{≈} \AgdaBound{z}\AgdaSymbol{)}\<%
\\
\>[0]\AgdaIndent{13}{}\<[13]%
\>[13]\AgdaSymbol{(}\AgdaBound{a} \AgdaSymbol{:} \AgdaFunction{∣} \AgdaBound{fm} \AgdaBound{x} \AgdaFunction{∣}\AgdaSymbol{)} \<[28]%
\>[28]\<%
\\
\>[0]\AgdaIndent{13}{}\<[13]%
\>[13]\AgdaSymbol{→} \AgdaFunction{[} \AgdaBound{fm} \AgdaBound{z} \AgdaFunction{]} \AgdaBound{substT} \AgdaBound{q} \AgdaSymbol{(}\AgdaBound{substT} \AgdaBound{p} \AgdaBound{a}\AgdaSymbol{)} \<[46]%
\>[46]\<%
\\
\>[13]\AgdaIndent{17}{}\<[17]%
\>[17]\AgdaFunction{≈} \AgdaBound{substT} \AgdaSymbol{(}\AgdaFunction{[} \AgdaBound{Γ} \AgdaFunction{]trans} \AgdaBound{p} \AgdaBound{q}\AgdaSymbol{)} \AgdaBound{a}\<%
\\
%
\\
%
\\
\>\<\end{code}

Some other lemmas on the proof irrelevance derived from these fields are not shown here since they are just auxiliary functions.

\AgdaHide{
\begin{code}\>\<%
\\
\>\AgdaComment{-- the proof-irrelevance lemmas for substT}\<%
\\
%
\\
\>[2]\AgdaIndent{2}{}\<[2]%
\>[2]\AgdaFunction{subst-pi} \AgdaSymbol{:} \AgdaSymbol{∀\{}\AgdaBound{x} \AgdaBound{y} \AgdaSymbol{:} \AgdaFunction{∣} \AgdaBound{Γ} \AgdaFunction{∣}\AgdaSymbol{\}}\<%
\\
\>[0]\AgdaIndent{14}{}\<[14]%
\>[14]\AgdaSymbol{\{}\AgdaBound{p} \AgdaBound{q} \AgdaSymbol{:} \AgdaFunction{[} \AgdaBound{Γ} \AgdaFunction{]} \AgdaBound{x} \AgdaFunction{≈} \AgdaBound{y}\AgdaSymbol{\}}\<%
\\
\>[0]\AgdaIndent{14}{}\<[14]%
\>[14]\AgdaSymbol{\{}\AgdaBound{a} \AgdaSymbol{:} \AgdaFunction{∣} \AgdaFunction{fm} \AgdaBound{x} \AgdaFunction{∣}\AgdaSymbol{\}} \AgdaSymbol{→} \AgdaFunction{[} \AgdaFunction{fm} \AgdaBound{y} \AgdaFunction{]} \AgdaFunction{substT} \AgdaBound{p} \AgdaBound{a} \AgdaFunction{≈} \AgdaFunction{substT} \AgdaBound{q} \AgdaBound{a}\<%
\\
\>[0]\AgdaIndent{2}{}\<[2]%
\>[2]\AgdaFunction{subst-pi} \AgdaSymbol{\{}\AgdaBound{x}\AgdaSymbol{\}} \AgdaSymbol{\{}\AgdaBound{y}\AgdaSymbol{\}} \AgdaSymbol{\{}\AgdaBound{p}\AgdaSymbol{\}} \AgdaSymbol{\{}\AgdaBound{q}\AgdaSymbol{\}} \AgdaSymbol{\{}\AgdaBound{a}\AgdaSymbol{\}} \AgdaSymbol{=} \AgdaFunction{reflexive} \AgdaSymbol{(}\AgdaFunction{fm} \AgdaBound{y}\AgdaSymbol{)} \AgdaSymbol{(}\AgdaFunction{PI} \AgdaBound{Γ} \AgdaSymbol{(λ} \AgdaBound{x} \AgdaSymbol{→} \AgdaFunction{substT} \AgdaBound{x} \AgdaBound{a}\AgdaSymbol{))}\<%
\\
%
\\
\>[0]\AgdaIndent{2}{}\<[2]%
\>[2]\AgdaFunction{subst-pi'} \AgdaSymbol{:} \AgdaSymbol{∀\{}\AgdaBound{x} \AgdaSymbol{:} \AgdaFunction{∣} \AgdaBound{Γ} \AgdaFunction{∣}\AgdaSymbol{\}}\<%
\\
\>[2]\AgdaIndent{15}{}\<[15]%
\>[15]\AgdaSymbol{\{}\AgdaBound{p} \AgdaSymbol{:} \AgdaFunction{[} \AgdaBound{Γ} \AgdaFunction{]} \AgdaBound{x} \AgdaFunction{≈} \AgdaBound{x}\AgdaSymbol{\}}\<%
\\
\>[2]\AgdaIndent{15}{}\<[15]%
\>[15]\AgdaSymbol{\{}\AgdaBound{a} \AgdaSymbol{:} \AgdaFunction{∣} \AgdaFunction{fm} \AgdaBound{x} \AgdaFunction{∣}\AgdaSymbol{\}} \AgdaSymbol{→} \AgdaFunction{[} \AgdaFunction{fm} \AgdaBound{x} \AgdaFunction{]} \AgdaFunction{substT} \AgdaBound{p} \AgdaBound{a} \AgdaFunction{≈} \AgdaBound{a}\<%
\\
\>[0]\AgdaIndent{2}{}\<[2]%
\>[2]\AgdaFunction{subst-pi'} \AgdaSymbol{=} \AgdaFunction{[} \AgdaFunction{fm} \AgdaSymbol{\_} \AgdaFunction{]trans} \AgdaFunction{subst-pi} \AgdaSymbol{(}\AgdaFunction{refl*} \AgdaSymbol{\_} \AgdaSymbol{\_)}\<%
\\
%
\\
\>[0]\AgdaIndent{2}{}\<[2]%
\>[2]\AgdaFunction{subst-pi*} \AgdaSymbol{:} \AgdaSymbol{∀\{}\AgdaBound{x} \AgdaBound{y} \AgdaSymbol{:} \AgdaFunction{∣} \AgdaBound{Γ} \AgdaFunction{∣}\AgdaSymbol{\}}\<%
\\
\>[2]\AgdaIndent{16}{}\<[16]%
\>[16]\AgdaSymbol{\{}\AgdaBound{p} \AgdaBound{q} \AgdaSymbol{:} \AgdaFunction{[} \AgdaBound{Γ} \AgdaFunction{]} \AgdaBound{x} \AgdaFunction{≈} \AgdaBound{y}\AgdaSymbol{\}}\<%
\\
\>[2]\AgdaIndent{16}{}\<[16]%
\>[16]\AgdaSymbol{\{}\AgdaBound{a} \AgdaBound{b} \AgdaSymbol{:} \AgdaFunction{∣} \AgdaFunction{fm} \AgdaBound{x} \AgdaFunction{∣}\AgdaSymbol{\}} \AgdaSymbol{→} \AgdaFunction{[} \AgdaFunction{fm} \AgdaBound{x} \AgdaFunction{]} \AgdaBound{a} \AgdaFunction{≈} \AgdaBound{b} \AgdaSymbol{→} \AgdaFunction{[} \AgdaFunction{fm} \AgdaBound{y} \AgdaFunction{]} \AgdaFunction{substT} \AgdaBound{p} \AgdaBound{a} \AgdaFunction{≈} \AgdaFunction{substT} \AgdaBound{q} \AgdaBound{b}\<%
\\
\>[0]\AgdaIndent{2}{}\<[2]%
\>[2]\AgdaFunction{subst-pi*} \AgdaBound{eq} \AgdaSymbol{=} \AgdaFunction{[} \AgdaFunction{fm} \AgdaSymbol{\_} \AgdaFunction{]trans} \AgdaSymbol{(}\AgdaFunction{subst*} \AgdaSymbol{\_} \AgdaBound{eq}\AgdaSymbol{)} \AgdaFunction{subst-pi}\<%
\\
%
\\
%
\\
\>\AgdaComment{-- simplify proofs of trans of inverse equality (including groupoid laws?)}\<%
\\
%
\\
\>[0]\AgdaIndent{2}{}\<[2]%
\>[2]\AgdaFunction{trans-refl} \AgdaSymbol{:} \AgdaSymbol{∀\{}\AgdaBound{x} \AgdaBound{y} \AgdaSymbol{:} \AgdaFunction{∣} \AgdaBound{Γ} \AgdaFunction{∣}\AgdaSymbol{\}}\<%
\\
\>[2]\AgdaIndent{14}{}\<[14]%
\>[14]\AgdaSymbol{\{}\AgdaBound{p} \AgdaSymbol{:} \AgdaFunction{[} \AgdaBound{Γ} \AgdaFunction{]} \AgdaBound{x} \AgdaFunction{≈} \AgdaBound{y}\AgdaSymbol{\}\{}\AgdaBound{q} \AgdaSymbol{:} \AgdaFunction{[} \AgdaBound{Γ} \AgdaFunction{]} \AgdaBound{y} \AgdaFunction{≈} \AgdaBound{x}\AgdaSymbol{\}}\<%
\\
\>[2]\AgdaIndent{14}{}\<[14]%
\>[14]\AgdaSymbol{\{}\AgdaBound{a} \AgdaSymbol{:} \AgdaFunction{∣} \AgdaFunction{fm} \AgdaBound{x} \AgdaFunction{∣}\AgdaSymbol{\}} \AgdaSymbol{→} \<[31]%
\>[31]\<%
\\
\>[2]\AgdaIndent{14}{}\<[14]%
\>[14]\AgdaFunction{[} \AgdaFunction{fm} \AgdaBound{x} \AgdaFunction{]} \AgdaFunction{substT} \AgdaBound{q} \AgdaSymbol{(}\AgdaFunction{substT} \AgdaBound{p} \AgdaBound{a}\AgdaSymbol{)} \AgdaFunction{≈} \AgdaBound{a}\<%
\\
\>[0]\AgdaIndent{2}{}\<[2]%
\>[2]\AgdaFunction{trans-refl} \AgdaSymbol{=} \AgdaFunction{[} \AgdaFunction{fm} \AgdaSymbol{\_} \AgdaFunction{]trans} \AgdaSymbol{(}\AgdaFunction{trans*} \AgdaSymbol{\_} \AgdaSymbol{\_} \AgdaSymbol{\_)} \AgdaFunction{subst-pi'}\<%
\\
%
\\
\>\AgdaComment{-- some more theorems}\<%
\\
\>[0]\AgdaIndent{2}{}\<[2]%
\>[2]\<%
\\
\>[0]\AgdaIndent{2}{}\<[2]%
\>[2]\AgdaFunction{subst-mir1} \AgdaSymbol{:} \AgdaSymbol{∀\{}\AgdaBound{x} \AgdaBound{y} \AgdaSymbol{:} \AgdaFunction{∣} \AgdaBound{Γ} \AgdaFunction{∣}\AgdaSymbol{\}}\<%
\\
\>[2]\AgdaIndent{14}{}\<[14]%
\>[14]\AgdaSymbol{\{}\AgdaBound{p} \AgdaSymbol{:} \AgdaFunction{[} \AgdaBound{Γ} \AgdaFunction{]} \AgdaBound{x} \AgdaFunction{≈} \AgdaBound{y}\AgdaSymbol{\}\{}\AgdaBound{q} \AgdaSymbol{:} \AgdaFunction{[} \AgdaBound{Γ} \AgdaFunction{]} \AgdaBound{y} \AgdaFunction{≈} \AgdaBound{x}\AgdaSymbol{\}}\<%
\\
\>[2]\AgdaIndent{14}{}\<[14]%
\>[14]\AgdaSymbol{\{}\AgdaBound{a} \AgdaSymbol{:} \AgdaFunction{∣} \AgdaFunction{fm} \AgdaBound{x} \AgdaFunction{∣}\AgdaSymbol{\}\{}\AgdaBound{b} \AgdaSymbol{:} \AgdaFunction{∣} \AgdaFunction{fm} \AgdaBound{y} \AgdaFunction{∣}\AgdaSymbol{\}} \AgdaSymbol{→} \<[45]%
\>[45]\<%
\\
\>[2]\AgdaIndent{14}{}\<[14]%
\>[14]\AgdaFunction{[} \AgdaFunction{fm} \AgdaBound{x} \AgdaFunction{]} \AgdaBound{a} \AgdaFunction{≈} \AgdaFunction{substT} \AgdaBound{q} \AgdaBound{b} \AgdaSymbol{→} \AgdaFunction{[} \AgdaFunction{fm} \AgdaBound{y} \AgdaFunction{]} \AgdaFunction{substT} \AgdaBound{p} \AgdaBound{a} \AgdaFunction{≈} \AgdaBound{b}\<%
\\
\>[0]\AgdaIndent{2}{}\<[2]%
\>[2]\AgdaFunction{subst-mir1} \AgdaBound{eq} \AgdaSymbol{=} \AgdaFunction{[} \AgdaFunction{fm} \AgdaSymbol{\_} \AgdaFunction{]trans} \AgdaSymbol{(}\AgdaFunction{subst*} \AgdaSymbol{\_} \AgdaBound{eq}\AgdaSymbol{)} \AgdaFunction{trans-refl}\<%
\\
%
\\
\>[0]\AgdaIndent{2}{}\<[2]%
\>[2]\AgdaFunction{subst-mir2} \AgdaSymbol{:} \AgdaSymbol{∀\{}\AgdaBound{x} \AgdaBound{y} \AgdaSymbol{:} \AgdaFunction{∣} \AgdaBound{Γ} \AgdaFunction{∣}\AgdaSymbol{\}}\<%
\\
\>[2]\AgdaIndent{14}{}\<[14]%
\>[14]\AgdaSymbol{\{}\AgdaBound{p} \AgdaSymbol{:} \AgdaFunction{[} \AgdaBound{Γ} \AgdaFunction{]} \AgdaBound{x} \AgdaFunction{≈} \AgdaBound{y}\AgdaSymbol{\}\{}\AgdaBound{q} \AgdaSymbol{:} \AgdaFunction{[} \AgdaBound{Γ} \AgdaFunction{]} \AgdaBound{y} \AgdaFunction{≈} \AgdaBound{x}\AgdaSymbol{\}}\<%
\\
\>[2]\AgdaIndent{14}{}\<[14]%
\>[14]\AgdaSymbol{\{}\AgdaBound{a} \AgdaSymbol{:} \AgdaFunction{∣} \AgdaFunction{fm} \AgdaBound{x} \AgdaFunction{∣}\AgdaSymbol{\}\{}\AgdaBound{b} \AgdaSymbol{:} \AgdaFunction{∣} \AgdaFunction{fm} \AgdaBound{y} \AgdaFunction{∣}\AgdaSymbol{\}} \AgdaSymbol{→} \<[45]%
\>[45]\<%
\\
\>[2]\AgdaIndent{14}{}\<[14]%
\>[14]\AgdaFunction{[} \AgdaFunction{fm} \AgdaBound{y} \AgdaFunction{]} \AgdaFunction{substT} \AgdaBound{p} \AgdaBound{a} \AgdaFunction{≈} \AgdaBound{b} \AgdaSymbol{→} \AgdaFunction{[} \AgdaFunction{fm} \AgdaBound{x} \AgdaFunction{]} \AgdaBound{a} \AgdaFunction{≈} \AgdaFunction{substT} \AgdaBound{q} \AgdaBound{b}\<%
\\
\>[0]\AgdaIndent{2}{}\<[2]%
\>[2]\AgdaFunction{subst-mir2} \AgdaBound{eq} \AgdaSymbol{=} \AgdaFunction{[} \AgdaFunction{fm} \AgdaSymbol{\_} \AgdaFunction{]sym} \AgdaSymbol{(}\AgdaFunction{subst-mir1} \AgdaSymbol{(}\AgdaFunction{[} \AgdaFunction{fm} \AgdaSymbol{\_} \AgdaFunction{]sym} \AgdaBound{eq}\AgdaSymbol{))}\<%
\\
%
\\
\>\AgdaKeyword{open} \AgdaModule{Ty} \AgdaKeyword{public} \<[15]%
\>[15]\<%
\\
\>[0]\AgdaIndent{2}{}\<[2]%
\>[2]\AgdaKeyword{renaming} \AgdaSymbol{(}substT \AgdaSymbol{to} [\_]subst\AgdaSymbol{;} subst* \AgdaSymbol{to} [\_]subst*\AgdaSymbol{;} fm \AgdaSymbol{to} [\_]fm \AgdaSymbol{;}
            refl* \AgdaSymbol{to} [\_]refl* \AgdaSymbol{;} trans* \AgdaSymbol{to} [\_]trans*\AgdaSymbol{;} subst-pi \AgdaSymbol{to} [\_]subst-pi \AgdaSymbol{;}
            subst-pi' \AgdaSymbol{to} [\_]subst-pi' \AgdaSymbol{;} subst-pi* \AgdaSymbol{to} [\_]subst-pi* \AgdaSymbol{;}
            trans-refl \AgdaSymbol{to} [\_]trans-refl \AgdaSymbol{;} subst-mir1 \AgdaSymbol{to} [\_]subst-mir1 \AgdaSymbol{;}
            subst-mir2 \AgdaSymbol{to} [\_]subst-mir2\AgdaSymbol{)}\<%
\\
%
\\
\>\<\end{code}
}

Then we have to define the substituting in a type given a context morphism and verify it preserves equivalence relation as well.

\begin{code}\>\<%
\\
%
\\
\>\AgdaFunction{\_[\_]T} \AgdaSymbol{:} \AgdaSymbol{∀} \AgdaSymbol{\{}\AgdaBound{Γ} \AgdaBound{Δ} \AgdaSymbol{:} \AgdaFunction{Con}\AgdaSymbol{\}} \AgdaSymbol{→} \AgdaRecord{Ty} \AgdaBound{Δ} \AgdaSymbol{→} \AgdaBound{Γ} \AgdaRecord{⇉} \AgdaBound{Δ} \AgdaSymbol{→} \AgdaRecord{Ty} \AgdaBound{Γ}\<%
\\
\>\AgdaBound{A} \AgdaFunction{[} \AgdaBound{f} \AgdaFunction{]T}\<%
\\
\>[2]\AgdaIndent{5}{}\<[5]%
\>[5]\AgdaSymbol{=} \AgdaKeyword{record}\<%
\\
\>[2]\AgdaIndent{5}{}\<[5]%
\>[5]\AgdaSymbol{\{} \AgdaField{fm} \<[14]%
\>[14]\AgdaSymbol{=} \AgdaFunction{fm} \AgdaFunction{∘} \AgdaFunction{fn}\<%
\\
\>[2]\AgdaIndent{5}{}\<[5]%
\>[5]\AgdaSymbol{;} \AgdaField{substT} \AgdaSymbol{=} \AgdaFunction{substT} \AgdaFunction{∘} \AgdaFunction{resp}\<%
\\
\>[2]\AgdaIndent{5}{}\<[5]%
\>[5]\AgdaSymbol{;} \AgdaField{subst*} \AgdaSymbol{=} \AgdaFunction{subst*} \AgdaFunction{∘} \AgdaFunction{resp}\<%
\\
\>[2]\AgdaIndent{5}{}\<[5]%
\>[5]\AgdaSymbol{;} \AgdaField{refl*} \<[14]%
\>[14]\AgdaSymbol{=} \AgdaSymbol{λ} \AgdaBound{\_} \AgdaBound{\_} \AgdaSymbol{→} \AgdaFunction{subst-pi'}\<%
\\
\>[2]\AgdaIndent{5}{}\<[5]%
\>[5]\AgdaSymbol{;} \AgdaField{trans*} \AgdaSymbol{=} \AgdaSymbol{λ} \AgdaBound{\_} \AgdaBound{\_} \AgdaBound{\_} \AgdaSymbol{→} \<[26]%
\>[26]\<%
\\
\>[5]\AgdaIndent{16}{}\<[16]%
\>[16]\AgdaFunction{[} \AgdaFunction{fm} \AgdaSymbol{(}\AgdaFunction{fn} \AgdaSymbol{\_)} \AgdaFunction{]trans} \AgdaSymbol{(}\AgdaFunction{trans*} \AgdaSymbol{\_} \AgdaSymbol{\_} \AgdaSymbol{\_)} \AgdaFunction{subst-pi}\<%
\\
\>[0]\AgdaIndent{5}{}\<[5]%
\>[5]\AgdaSymbol{\}}\<%
\\
\>[0]\AgdaIndent{5}{}\<[5]%
\>[5]\AgdaKeyword{where} \<[11]%
\>[11]\<%
\\
\>[5]\AgdaIndent{7}{}\<[7]%
\>[7]\AgdaKeyword{open} \AgdaModule{Ty} \AgdaBound{A}\<%
\\
\>[5]\AgdaIndent{7}{}\<[7]%
\>[7]\AgdaKeyword{open} \AgdaModule{\_⇉\_} \AgdaBound{f}\<%
\\
%
\\
\>\<\end{code}

The semantic terms are simpler. It should also preserve the equivalence relation on the elements of contexts.

\begin{code}\>\<%
\\
%
\\
\>\AgdaKeyword{record} \AgdaRecord{Tm} \AgdaSymbol{\{}\AgdaBound{Γ} \AgdaSymbol{:} \AgdaFunction{Con}\AgdaSymbol{\}(}\AgdaBound{A} \AgdaSymbol{:} \AgdaRecord{Ty} \AgdaBound{Γ}\AgdaSymbol{)} \AgdaSymbol{:} \AgdaPrimitiveType{Set} \AgdaKeyword{where}\<%
\\
\>[0]\AgdaIndent{2}{}\<[2]%
\>[2]\AgdaKeyword{constructor} \AgdaInductiveConstructor{tm:\_resp:\_}\<%
\\
\>[0]\AgdaIndent{2}{}\<[2]%
\>[2]\AgdaKeyword{field}\<%
\\
\>[2]\AgdaIndent{4}{}\<[4]%
\>[4]\AgdaField{tm} \<[10]%
\>[10]\AgdaSymbol{:} \AgdaSymbol{(}\AgdaBound{x} \AgdaSymbol{:} \AgdaFunction{∣} \AgdaBound{Γ} \AgdaFunction{∣}\AgdaSymbol{)} \AgdaSymbol{→} \AgdaFunction{∣} \AgdaFunction{[} \AgdaBound{A} \AgdaFunction{]fm} \AgdaBound{x} \AgdaFunction{∣}\<%
\\
\>[2]\AgdaIndent{4}{}\<[4]%
\>[4]\AgdaField{respt} \AgdaSymbol{:} \AgdaSymbol{∀} \AgdaSymbol{\{}\AgdaBound{x} \AgdaBound{y} \AgdaSymbol{:} \AgdaFunction{∣} \AgdaBound{Γ} \AgdaFunction{∣}\AgdaSymbol{\}} \AgdaSymbol{→} \<[30]%
\>[30]\<%
\\
\>[4]\AgdaIndent{14}{}\<[14]%
\>[14]\AgdaSymbol{(}\AgdaBound{p} \AgdaSymbol{:} \AgdaFunction{[} \AgdaBound{Γ} \AgdaFunction{]} \AgdaBound{x} \AgdaFunction{≈} \AgdaBound{y}\AgdaSymbol{)} \AgdaSymbol{→} \<[34]%
\>[34]\<%
\\
\>[4]\AgdaIndent{14}{}\<[14]%
\>[14]\AgdaFunction{[} \AgdaFunction{[} \AgdaBound{A} \AgdaFunction{]fm} \AgdaBound{y} \AgdaFunction{]} \AgdaFunction{[} \AgdaBound{A} \AgdaFunction{]subst} \AgdaBound{p} \AgdaSymbol{(}\AgdaBound{tm} \AgdaBound{x}\AgdaSymbol{)} \AgdaFunction{≈} \AgdaBound{tm} \AgdaBound{y}\<%
\\
%
\\
\>\<\end{code}

\AgdaHide{
\begin{code}\>\<%
\\
\>\AgdaKeyword{open} \AgdaModule{Tm} \AgdaKeyword{public} \AgdaKeyword{renaming} \AgdaSymbol{(}tm \AgdaSymbol{to} [\_]tm \AgdaSymbol{;} respt \AgdaSymbol{to} [\_]respt\AgdaSymbol{)}\<%
\\
%
\\
\>\<\end{code}
}

Substitution for terms can be defined as

\begin{code}\>\<%
\\
%
\\
\>\AgdaFunction{\_[\_]m} \AgdaSymbol{:} \AgdaSymbol{∀} \AgdaSymbol{\{}\AgdaBound{Γ} \AgdaBound{Δ} \AgdaSymbol{:} \AgdaFunction{Con}\AgdaSymbol{\}\{}\AgdaBound{A} \AgdaSymbol{:} \AgdaRecord{Ty} \AgdaBound{Δ}\AgdaSymbol{\}} \AgdaSymbol{→} \<[34]%
\>[34]\<%
\\
\>[0]\AgdaIndent{10}{}\<[10]%
\>[10]\AgdaRecord{Tm} \AgdaBound{A} \AgdaSymbol{→} \<[17]%
\>[17]\<%
\\
\>[0]\AgdaIndent{10}{}\<[10]%
\>[10]\AgdaSymbol{(}\AgdaBound{f} \AgdaSymbol{:} \AgdaBound{Γ} \AgdaRecord{⇉} \AgdaBound{Δ}\AgdaSymbol{)} \<[22]%
\>[22]\<%
\\
\>[0]\AgdaIndent{10}{}\<[10]%
\>[10]\AgdaSymbol{→} \AgdaRecord{Tm} \AgdaSymbol{(}\AgdaBound{A} \AgdaFunction{[} \AgdaBound{f} \AgdaFunction{]T}\AgdaSymbol{)}\<%
\\
\>\AgdaFunction{\_[\_]m} \AgdaBound{t} \AgdaBound{f} \AgdaSymbol{=} \AgdaKeyword{record} \<[19]%
\>[19]\<%
\\
\>[0]\AgdaIndent{10}{}\<[10]%
\>[10]\AgdaSymbol{\{} \AgdaField{tm} \AgdaSymbol{=} \AgdaFunction{[} \AgdaBound{t} \AgdaFunction{]tm} \AgdaFunction{∘} \AgdaFunction{[} \AgdaBound{f} \AgdaFunction{]fn}\<%
\\
\>[0]\AgdaIndent{10}{}\<[10]%
\>[10]\AgdaSymbol{;} \AgdaField{respt} \AgdaSymbol{=} \AgdaFunction{[} \AgdaBound{t} \AgdaFunction{]respt} \AgdaFunction{∘} \AgdaFunction{[} \AgdaBound{f} \AgdaFunction{]resp} \<[43]%
\>[43]\<%
\\
\>[0]\AgdaIndent{10}{}\<[10]%
\>[10]\AgdaSymbol{\}}\<%
\\
%
\\
\>\<\end{code}

Syntactically we can form a new context by using a context $\Gamma$ and a type $A : Ty \:\Gamma$. To introduce a term of it, we need a term of the semantic context $\Gamma$ and a term of semantic type $A$. It is called context comprehension. 

\begin{code}\>\<%
\\
%
\\
\>\AgdaFunction{\_\&\_} \AgdaSymbol{:} \AgdaSymbol{(}\AgdaBound{Γ} \AgdaSymbol{:} \AgdaFunction{Con}\AgdaSymbol{)} \AgdaSymbol{→} \AgdaRecord{Ty} \AgdaBound{Γ} \AgdaSymbol{→} \AgdaFunction{Con}\<%
\\
\>\AgdaBound{Γ} \AgdaFunction{\&} \AgdaBound{A} \AgdaSymbol{=} \AgdaKeyword{record} \<[15]%
\>[15]\<%
\\
\>[0]\AgdaIndent{7}{}\<[7]%
\>[7]\AgdaSymbol{\{} \AgdaField{Carrier} \AgdaSymbol{=} \AgdaRecord{Σ[} \AgdaBound{x} \AgdaRecord{∶} \AgdaFunction{∣} \AgdaBound{Γ} \AgdaFunction{∣} \AgdaRecord{]} \AgdaFunction{∣} \AgdaFunction{fm} \AgdaBound{x} \AgdaFunction{∣}\<%
\\
\>[0]\AgdaIndent{7}{}\<[7]%
\>[7]\AgdaSymbol{;} \AgdaField{\_≈h\_} \<[17]%
\>[17]\AgdaSymbol{=} \AgdaSymbol{λ\{(}\AgdaBound{x} \AgdaInductiveConstructor{,} \AgdaBound{a}\AgdaSymbol{)} \AgdaSymbol{(}\AgdaBound{y} \AgdaInductiveConstructor{,} \AgdaBound{b}\AgdaSymbol{)} \AgdaSymbol{→} \<[39]%
\>[39]\<%
\\
\>[7]\AgdaIndent{19}{}\<[19]%
\>[19]\AgdaFunction{Σ'[} \AgdaBound{p} \AgdaFunction{∶} \AgdaBound{x} \AgdaFunction{≈h} \AgdaBound{y} \AgdaFunction{]} \AgdaFunction{[} \AgdaFunction{fm} \AgdaBound{y} \AgdaFunction{]} \AgdaFunction{substT} \AgdaBound{p} \AgdaBound{a} \AgdaFunction{≈h} \AgdaBound{b}\AgdaSymbol{\}}\<%
\\
\>[0]\AgdaIndent{7}{}\<[7]%
\>[7]\AgdaSymbol{;} \AgdaField{isEquiv} \AgdaSymbol{=} \<[19]%
\>[19]\<%
\\
\>[0]\AgdaIndent{10}{}\<[10]%
\>[10]\AgdaKeyword{record} \<[17]%
\>[17]\<%
\\
\>[0]\AgdaIndent{10}{}\<[10]%
\>[10]\AgdaSymbol{\{} \AgdaField{refl} \<[18]%
\>[18]\AgdaSymbol{=} \AgdaFunction{refl} \AgdaInductiveConstructor{,} \AgdaSymbol{(}\AgdaFunction{refl*} \AgdaSymbol{\_} \AgdaSymbol{\_)}\<%
\\
\>[0]\AgdaIndent{10}{}\<[10]%
\>[10]\AgdaSymbol{;} \AgdaField{sym} \<[18]%
\>[18]\AgdaSymbol{=} \AgdaSymbol{λ} \AgdaSymbol{\{(}\AgdaBound{p} \AgdaInductiveConstructor{,} \AgdaBound{q}\AgdaSymbol{)} \AgdaSymbol{→} \AgdaSymbol{(}\AgdaFunction{sym} \AgdaBound{p}\AgdaSymbol{)} \AgdaInductiveConstructor{,} \<[43]%
\>[43]\<%
\\
\>[10]\AgdaIndent{20}{}\<[20]%
\>[20]\AgdaFunction{[} \AgdaFunction{fm} \AgdaSymbol{\_} \AgdaFunction{]trans} \<[34]%
\>[34]\<%
\\
\>[10]\AgdaIndent{20}{}\<[20]%
\>[20]\AgdaSymbol{(}\AgdaFunction{subst*} \AgdaSymbol{\_} \AgdaSymbol{(}\AgdaFunction{[} \AgdaFunction{fm} \AgdaSymbol{\_} \AgdaFunction{]sym} \AgdaBound{q}\AgdaSymbol{))} \<[47]%
\>[47]\<%
\\
\>[10]\AgdaIndent{20}{}\<[20]%
\>[20]\AgdaFunction{trans-refl} \AgdaSymbol{\}}\<%
\\
\>[0]\AgdaIndent{10}{}\<[10]%
\>[10]\AgdaSymbol{;} \AgdaField{trans} \AgdaSymbol{=} \AgdaSymbol{λ} \AgdaSymbol{\{(}\AgdaBound{p} \AgdaInductiveConstructor{,} \AgdaBound{q}\AgdaSymbol{)} \AgdaSymbol{(}\AgdaBound{m} \AgdaInductiveConstructor{,} \AgdaBound{n}\AgdaSymbol{)} \AgdaSymbol{→}\<%
\\
\>[0]\AgdaIndent{20}{}\<[20]%
\>[20]\AgdaFunction{trans} \AgdaBound{p} \AgdaBound{m} \AgdaInductiveConstructor{,} \<[32]%
\>[32]\<%
\\
\>[0]\AgdaIndent{20}{}\<[20]%
\>[20]\AgdaFunction{[} \AgdaFunction{fm} \AgdaSymbol{\_} \AgdaFunction{]trans} \<[34]%
\>[34]\<%
\\
\>[0]\AgdaIndent{20}{}\<[20]%
\>[20]\AgdaSymbol{(}\AgdaFunction{[} \AgdaFunction{fm} \AgdaSymbol{\_} \AgdaFunction{]trans} \<[35]%
\>[35]\<%
\\
\>[0]\AgdaIndent{20}{}\<[20]%
\>[20]\AgdaSymbol{(}\AgdaFunction{[} \AgdaFunction{fm} \AgdaSymbol{\_} \AgdaFunction{]sym} \AgdaSymbol{(}\AgdaFunction{trans*} \AgdaSymbol{\_} \AgdaSymbol{\_} \AgdaSymbol{\_))} \AgdaSymbol{(}\AgdaFunction{subst*} \AgdaSymbol{\_} \AgdaBound{q}\AgdaSymbol{))} \AgdaBound{n} \AgdaSymbol{\}}\<%
\\
\>[0]\AgdaIndent{10}{}\<[10]%
\>[10]\AgdaSymbol{\}}\<%
\\
\>[0]\AgdaIndent{7}{}\<[7]%
\>[7]\AgdaSymbol{\}}\<%
\\
%
\\
%
\\
\>\<\end{code}

\AgdaHide{
\begin{code}\>\<%
\\
%
\\
\>[0]\AgdaIndent{7}{}\<[7]%
\>[7]\AgdaKeyword{where} \<[13]%
\>[13]\<%
\\
\>[7]\AgdaIndent{9}{}\<[9]%
\>[9]\AgdaKeyword{open} \AgdaModule{hSetoid} \AgdaBound{Γ}\<%
\\
\>[7]\AgdaIndent{9}{}\<[9]%
\>[9]\AgdaKeyword{open} \AgdaModule{Ty} \AgdaBound{A} \<[23]%
\>[23]\<%
\\
%
\\
\>\<\end{code}
}

There are also some other morphisms come with it. Any morphism from a context $\Gamma$ to a context $\Delta \& A$ consists of a morphism from $\Gamma$ to $\Delta$ and a term of type $A$ substituted. In other words, There is an isomorphism between $Hom(\Gamma , \Delta \& A)$ and $\Sigma \gamma : Hom(\Gamma , \Delta) A [ \gamma ] $.

$fst$ projects the morphism and  $snd$ projects the term.
Indeed the $fst$ operation provides weakening for types, and the $snd$ projection enables us to interpret variables. $fst\&$ defines a morphism for each type $A$ which is a canonical projection of $A$.
We need to use $id'$ which are identity context morphisms to achieve these.

\begin{code}\>\<%
\\
%
\\
\>\AgdaFunction{fst} \AgdaSymbol{:} \AgdaSymbol{\{}\AgdaBound{Γ} \AgdaBound{Δ} \AgdaSymbol{:} \AgdaFunction{Con}\AgdaSymbol{\}(}\AgdaBound{A} \AgdaSymbol{:} \AgdaRecord{Ty} \AgdaBound{Δ}\AgdaSymbol{)} \AgdaSymbol{→} \AgdaBound{Γ} \AgdaRecord{⇉} \AgdaSymbol{(}\AgdaBound{Δ} \AgdaFunction{\&} \AgdaBound{A}\AgdaSymbol{)} \AgdaSymbol{→} \AgdaBound{Γ} \AgdaRecord{⇉} \AgdaBound{Δ}\<%
\\
\>\AgdaFunction{fst} \AgdaBound{A} \AgdaBound{f} \AgdaSymbol{=} \AgdaKeyword{record} \<[17]%
\>[17]\<%
\\
\>[-6]\AgdaIndent{8}{}\<[8]%
\>[8]\AgdaSymbol{\{} \AgdaField{fn} \AgdaSymbol{=} \AgdaFunction{proj₁} \AgdaFunction{∘} \AgdaFunction{[} \AgdaBound{f} \AgdaFunction{]fn}\<%
\\
\>[0]\AgdaIndent{8}{}\<[8]%
\>[8]\AgdaSymbol{;} \AgdaField{resp} \AgdaSymbol{=} \AgdaFunction{proj₁} \AgdaFunction{∘} \AgdaFunction{[} \AgdaBound{f} \AgdaFunction{]resp} \<[35]%
\>[35]\<%
\\
\>[0]\AgdaIndent{8}{}\<[8]%
\>[8]\AgdaSymbol{\}}\<%
\\
%
\\
\>\AgdaFunction{fst\&} \AgdaSymbol{:} \AgdaSymbol{\{}\AgdaBound{Γ} \AgdaSymbol{:} \AgdaFunction{Con}\AgdaSymbol{\}(}\AgdaBound{A} \AgdaSymbol{:} \AgdaRecord{Ty} \AgdaBound{Γ}\AgdaSymbol{)} \AgdaSymbol{→} \AgdaBound{Γ} \AgdaFunction{\&} \AgdaBound{A} \AgdaRecord{⇉} \AgdaBound{Γ}\<%
\\
\>\AgdaFunction{fst\&} \AgdaBound{A} \AgdaSymbol{=} \AgdaFunction{fst} \AgdaBound{A} \AgdaFunction{id'}\<%
\\
%
\\
\>\AgdaFunction{\_+T\_} \AgdaSymbol{:} \AgdaSymbol{\{}\AgdaBound{Γ} \AgdaSymbol{:} \AgdaFunction{Con}\AgdaSymbol{\}} \AgdaSymbol{→} \AgdaRecord{Ty} \AgdaBound{Γ} \AgdaSymbol{→} \AgdaSymbol{(}\AgdaBound{A} \AgdaSymbol{:} \AgdaRecord{Ty} \AgdaBound{Γ}\AgdaSymbol{)} \AgdaSymbol{→} \AgdaRecord{Ty} \AgdaSymbol{(}\AgdaBound{Γ} \AgdaFunction{\&} \AgdaBound{A}\AgdaSymbol{)}\<%
\\
\>\AgdaBound{B} \AgdaFunction{+T} \AgdaBound{A} \AgdaSymbol{=} \AgdaBound{B} \AgdaFunction{[} \AgdaFunction{fst\&} \AgdaBound{A} \AgdaFunction{]T}\<%
\\
%
\\
\>\AgdaFunction{snd} \AgdaSymbol{:} \AgdaSymbol{\{}\AgdaBound{Γ} \AgdaBound{Δ} \AgdaSymbol{:} \AgdaFunction{Con}\AgdaSymbol{\}(}\AgdaBound{A} \AgdaSymbol{:} \AgdaRecord{Ty} \AgdaBound{Δ}\AgdaSymbol{)} \AgdaSymbol{→} \<[30]%
\>[30]\<%
\\
\>[0]\AgdaIndent{6}{}\<[6]%
\>[6]\AgdaSymbol{(}\AgdaBound{f} \AgdaSymbol{:} \AgdaBound{Γ} \AgdaRecord{⇉} \AgdaSymbol{(}\AgdaBound{Δ} \AgdaFunction{\&} \AgdaBound{A}\AgdaSymbol{))} \<[24]%
\>[24]\<%
\\
\>[0]\AgdaIndent{6}{}\<[6]%
\>[6]\AgdaSymbol{→} \AgdaRecord{Tm} \AgdaSymbol{(}\AgdaBound{A} \AgdaFunction{[} \AgdaFunction{fst} \AgdaBound{A} \AgdaBound{f} \AgdaFunction{]T}\AgdaSymbol{)}\<%
\\
\>\AgdaFunction{snd} \AgdaBound{A} \AgdaBound{f} \AgdaSymbol{=} \AgdaKeyword{record} \<[17]%
\>[17]\<%
\\
\>[6]\AgdaIndent{8}{}\<[8]%
\>[8]\AgdaSymbol{\{} \AgdaField{tm} \AgdaSymbol{=} \AgdaFunction{proj₂} \AgdaFunction{∘} \AgdaFunction{[} \AgdaBound{f} \AgdaFunction{]fn}\<%
\\
\>[6]\AgdaIndent{8}{}\<[8]%
\>[8]\AgdaSymbol{;} \AgdaField{respt} \AgdaSymbol{=} \AgdaFunction{proj₂} \AgdaFunction{∘} \AgdaFunction{[} \AgdaBound{f} \AgdaFunction{]resp} \<[36]%
\>[36]\<%
\\
\>[6]\AgdaIndent{8}{}\<[8]%
\>[8]\AgdaSymbol{\}}\<%
\\
%
\\
\>\AgdaFunction{v0} \AgdaSymbol{:} \AgdaSymbol{\{}\AgdaBound{Γ} \AgdaSymbol{:} \AgdaFunction{Con}\AgdaSymbol{\}(}\AgdaBound{A} \AgdaSymbol{:} \AgdaRecord{Ty} \AgdaBound{Γ}\AgdaSymbol{)} \AgdaSymbol{→} \AgdaRecord{Tm} \AgdaSymbol{(}\AgdaBound{A} \AgdaFunction{+T} \AgdaBound{A}\AgdaSymbol{)}\<%
\\
\>\AgdaFunction{v0} \AgdaBound{A} \AgdaSymbol{=} \AgdaFunction{snd} \AgdaBound{A} \AgdaFunction{id'}\<%
\\
%
\\
\>\<\end{code}

Inversely we could define a pairing operation to combine a context morphism with a term. The $\eta$-law for the projection and pairing holds trivially.

\begin{code}\>\<%
\\
%
\\
\>\AgdaFunction{\_,,\_} \AgdaSymbol{:} \AgdaSymbol{\{}\AgdaBound{Γ} \AgdaBound{Δ} \AgdaSymbol{:} \AgdaFunction{Con}\AgdaSymbol{\}\{}\AgdaBound{A} \AgdaSymbol{:} \AgdaRecord{Ty} \AgdaBound{Δ}\AgdaSymbol{\}(}\AgdaBound{f} \AgdaSymbol{:} \AgdaBound{Γ} \AgdaRecord{⇉} \AgdaBound{Δ}\AgdaSymbol{)} \AgdaSymbol{→} \<[42]%
\>[42]\<%
\\
\>[-5]\AgdaIndent{7}{}\<[7]%
\>[7]\AgdaSymbol{(}\AgdaRecord{Tm} \AgdaSymbol{(}\AgdaBound{A} \AgdaFunction{[} \AgdaBound{f} \AgdaFunction{]T}\AgdaSymbol{))} \<[23]%
\>[23]\<%
\\
\>[0]\AgdaIndent{7}{}\<[7]%
\>[7]\AgdaSymbol{→} \AgdaBound{Γ} \AgdaRecord{⇉} \AgdaSymbol{(}\AgdaBound{Δ} \AgdaFunction{\&} \AgdaBound{A}\AgdaSymbol{)}\<%
\\
\>\AgdaBound{f} \AgdaFunction{,,} \AgdaBound{t} \AgdaSymbol{=} \AgdaKeyword{record} \<[16]%
\>[16]\<%
\\
\>[7]\AgdaIndent{9}{}\<[9]%
\>[9]\AgdaSymbol{\{} \AgdaField{fn} \AgdaSymbol{=} \AgdaFunction{⟨} \AgdaFunction{[} \AgdaBound{f} \AgdaFunction{]fn} \AgdaFunction{,} \AgdaFunction{[} \AgdaBound{t} \AgdaFunction{]tm} \AgdaFunction{⟩}\<%
\\
\>[7]\AgdaIndent{9}{}\<[9]%
\>[9]\AgdaSymbol{;} \AgdaField{resp} \AgdaSymbol{=} \AgdaFunction{⟨} \AgdaFunction{[} \AgdaBound{f} \AgdaFunction{]resp} \AgdaFunction{,} \AgdaFunction{[} \AgdaBound{t} \AgdaFunction{]respt} \AgdaFunction{⟩}\<%
\\
\>[7]\AgdaIndent{9}{}\<[9]%
\>[9]\AgdaSymbol{\}}\<%
\\
%
\\
\>\AgdaFunction{\&-eta} \AgdaSymbol{:} \AgdaSymbol{\{}\AgdaBound{Γ} \AgdaBound{Δ} \AgdaSymbol{:} \AgdaFunction{Con}\AgdaSymbol{\}\{}\AgdaBound{A} \AgdaSymbol{:} \AgdaRecord{Ty} \AgdaBound{Δ}\AgdaSymbol{\}(}\AgdaBound{f} \AgdaSymbol{:} \AgdaBound{Γ} \AgdaRecord{⇉} \AgdaSymbol{(}\AgdaBound{Δ} \AgdaFunction{\&} \AgdaBound{A}\AgdaSymbol{))} \<[47]%
\>[47]\<%
\\
\>[-4]\AgdaIndent{6}{}\<[6]%
\>[6]\AgdaSymbol{→} \AgdaFunction{\_,,\_} \AgdaSymbol{\{}A \AgdaSymbol{=} \AgdaBound{A}\AgdaSymbol{\}} \AgdaSymbol{(}\AgdaFunction{fst} \AgdaBound{A} \AgdaBound{f}\AgdaSymbol{)} \AgdaSymbol{(}\AgdaFunction{snd} \AgdaBound{A} \AgdaBound{f}\AgdaSymbol{)} \AgdaDatatype{≡} \AgdaBound{f}\<%
\\
\>\AgdaFunction{\&-eta} \AgdaBound{f} \AgdaSymbol{=} \AgdaInductiveConstructor{PE.refl}\<%
\\
%
\\
%
\\
\>\<\end{code}

Then a lifting operation could help us define $\Pi$-types.

\begin{code}\>\<%
\\
%
\\
\>\AgdaFunction{lift} \AgdaSymbol{:} \AgdaSymbol{\{}\AgdaBound{Γ} \AgdaBound{Δ} \AgdaSymbol{:} \AgdaFunction{Con}\AgdaSymbol{\}(}\AgdaBound{f} \AgdaSymbol{:} \AgdaBound{Γ} \AgdaRecord{⇉} \AgdaBound{Δ}\AgdaSymbol{)(}\AgdaBound{A} \AgdaSymbol{:} \AgdaRecord{Ty} \AgdaBound{Δ}\AgdaSymbol{)} \AgdaSymbol{→} \AgdaBound{Γ} \AgdaFunction{\&} \AgdaBound{A} \AgdaFunction{[} \AgdaBound{f} \AgdaFunction{]T} \AgdaRecord{⇉} \AgdaBound{Δ} \AgdaFunction{\&} \AgdaBound{A}\<%
\\
\>\AgdaFunction{lift} \AgdaBound{f} \AgdaBound{A} \AgdaSymbol{=} \AgdaKeyword{record} \<[18]%
\>[18]\<%
\\
\>[0]\AgdaIndent{8}{}\<[8]%
\>[8]\AgdaSymbol{\{} \AgdaField{fn} \AgdaSymbol{=} \AgdaFunction{⟨} \AgdaFunction{[} \AgdaBound{f} \AgdaFunction{]fn} \AgdaFunction{∘} \AgdaFunction{proj₁} \AgdaFunction{,} \AgdaFunction{proj₂} \AgdaFunction{⟩}\<%
\\
\>[0]\AgdaIndent{8}{}\<[8]%
\>[8]\AgdaSymbol{;} \AgdaField{resp} \AgdaSymbol{=} \AgdaFunction{⟨} \AgdaFunction{[} \AgdaBound{f} \AgdaFunction{]resp} \AgdaFunction{∘} \AgdaFunction{proj₁} \AgdaFunction{,} \AgdaFunction{proj₂} \AgdaFunction{⟩}\<%
\\
\>[0]\AgdaIndent{8}{}\<[8]%
\>[8]\AgdaSymbol{\}}\<%
\\
%
\\
\>\AgdaFunction{lift-eta} \AgdaSymbol{:} \AgdaSymbol{\{}\AgdaBound{Γ} \AgdaBound{Δ} \AgdaSymbol{:} \AgdaFunction{Con}\AgdaSymbol{\}}\<%
\\
\>[8]\AgdaIndent{9}{}\<[9]%
\>[9]\AgdaSymbol{(}\AgdaBound{f} \AgdaSymbol{:} \AgdaBound{Γ} \AgdaRecord{⇉} \AgdaBound{Δ}\AgdaSymbol{)(}\AgdaBound{A} \AgdaSymbol{:} \AgdaRecord{Ty} \AgdaBound{Δ}\AgdaSymbol{)(}\AgdaBound{x} \AgdaSymbol{:} \AgdaFunction{∣} \AgdaBound{Γ} \AgdaFunction{∣}\AgdaSymbol{)}\<%
\\
\>[8]\AgdaIndent{9}{}\<[9]%
\>[9]\AgdaSymbol{(}\AgdaBound{a} \AgdaSymbol{:} \AgdaFunction{∣} \AgdaFunction{[} \AgdaBound{A} \AgdaFunction{]fm} \AgdaSymbol{(}\AgdaFunction{[} \AgdaBound{f} \AgdaFunction{]fn} \AgdaBound{x}\AgdaSymbol{)} \AgdaFunction{∣}\AgdaSymbol{)} \<[39]%
\>[39]\<%
\\
\>[8]\AgdaIndent{9}{}\<[9]%
\>[9]\AgdaSymbol{→} \AgdaFunction{[} \AgdaFunction{lift} \AgdaBound{f} \AgdaBound{A} \AgdaFunction{]fn} \AgdaSymbol{(}\AgdaBound{x} \AgdaInductiveConstructor{,} \AgdaBound{a}\AgdaSymbol{)} \AgdaDatatype{≡} \AgdaSymbol{(} \AgdaFunction{[} \AgdaBound{f} \AgdaFunction{]fn} \AgdaBound{x} \AgdaInductiveConstructor{,} \AgdaBound{a}\AgdaSymbol{)}\<%
\\
\>\AgdaFunction{lift-eta} \AgdaBound{f} \AgdaBound{A} \AgdaBound{x} \AgdaBound{a} \AgdaSymbol{=} \AgdaInductiveConstructor{PE.refl}\<%
\\
%
\\
%
\\
\>\<\end{code}

One of the most complicated part of this definition is the $\Pi$-types.
$\Pi$-types is also called dependent function types. Semantically it is a function type on the underlying semantic types with a proof that the the functions respect the equivalence relation. 

%f-resp on the paper ignores refl*

\begin{code}\>\<%
\\
%
\\
\>\AgdaFunction{Π} \AgdaSymbol{:} \AgdaSymbol{\{}\AgdaBound{Γ} \AgdaSymbol{:} \AgdaFunction{Con}\AgdaSymbol{\}(}\AgdaBound{A} \AgdaSymbol{:} \AgdaRecord{Ty} \AgdaBound{Γ}\AgdaSymbol{)(}\AgdaBound{B} \AgdaSymbol{:} \AgdaRecord{Ty} \AgdaSymbol{(}\AgdaBound{Γ} \AgdaFunction{\&} \AgdaBound{A}\AgdaSymbol{))} \AgdaSymbol{→} \AgdaRecord{Ty} \AgdaBound{Γ}\<%
\\
%
\\
\>\<\end{code}

\AgdaHide{
\begin{code}\>\<%
\\
%
\\
\>\AgdaFunction{Π} \AgdaSymbol{\{}\AgdaBound{Γ}\AgdaSymbol{\}} \AgdaBound{A} \AgdaBound{B} \AgdaSymbol{=} \AgdaKeyword{record} \<[19]%
\>[19]\<%
\\
\>[-1]\AgdaIndent{2}{}\<[2]%
\>[2]\AgdaSymbol{\{} \AgdaField{fm} \AgdaSymbol{=} \AgdaSymbol{λ} \AgdaBound{x} \AgdaSymbol{→} \AgdaKeyword{let} \AgdaBound{Ax} \AgdaSymbol{=} \AgdaFunction{[} \AgdaBound{A} \AgdaFunction{]fm} \AgdaBound{x} \AgdaKeyword{in}\<%
\\
\>[0]\AgdaIndent{15}{}\<[15]%
\>[15]\AgdaKeyword{let} \AgdaBound{Bx} \AgdaSymbol{=} \AgdaSymbol{λ} \AgdaBound{a} \AgdaSymbol{→} \AgdaFunction{[} \AgdaBound{B} \AgdaFunction{]fm} \AgdaSymbol{(}\AgdaBound{x} \AgdaInductiveConstructor{,} \AgdaBound{a}\AgdaSymbol{)} \AgdaKeyword{in}\<%
\\
\>[0]\AgdaIndent{9}{}\<[9]%
\>[9]\AgdaKeyword{record}\<%
\\
\>[0]\AgdaIndent{9}{}\<[9]%
\>[9]\AgdaSymbol{\{} \AgdaField{Carrier} \AgdaSymbol{=} \AgdaRecord{Σ[} \AgdaBound{fn} \AgdaRecord{∶} \AgdaSymbol{((}\AgdaBound{a} \AgdaSymbol{:} \AgdaFunction{∣} \AgdaBound{Ax} \AgdaFunction{∣}\AgdaSymbol{)} \AgdaSymbol{→} \AgdaFunction{∣} \AgdaBound{Bx} \AgdaBound{a} \AgdaFunction{∣}\AgdaSymbol{)} \AgdaRecord{]}\<%
\\
\>[9]\AgdaIndent{21}{}\<[21]%
\>[21]\AgdaSymbol{((}\AgdaBound{a} \AgdaBound{b} \AgdaSymbol{:} \AgdaFunction{∣} \AgdaBound{Ax} \AgdaFunction{∣}\AgdaSymbol{)}\<%
\\
\>[9]\AgdaIndent{21}{}\<[21]%
\>[21]\AgdaSymbol{(}\AgdaBound{p} \AgdaSymbol{:} \AgdaFunction{[} \AgdaBound{Ax} \AgdaFunction{]} \AgdaBound{a} \AgdaFunction{≈} \AgdaBound{b}\AgdaSymbol{)} \AgdaSymbol{→}\<%
\\
\>[9]\AgdaIndent{21}{}\<[21]%
\>[21]\AgdaFunction{[} \AgdaBound{Bx} \AgdaBound{b} \AgdaFunction{]} \AgdaFunction{[} \AgdaBound{B} \AgdaFunction{]subst} \AgdaSymbol{(}\AgdaFunction{[} \AgdaBound{Γ} \AgdaFunction{]refl} \AgdaInductiveConstructor{,}\<%
\\
\>[-7]\AgdaIndent{19}{}\<[19]%
\>[19]\AgdaFunction{[} \AgdaBound{Ax} \AgdaFunction{]trans} \AgdaSymbol{(}\AgdaFunction{[} \AgdaBound{A} \AgdaFunction{]refl*} \AgdaBound{x} \AgdaBound{a}\AgdaSymbol{)} \AgdaBound{p}\AgdaSymbol{)} \AgdaSymbol{(}\AgdaBound{fn} \AgdaBound{a}\AgdaSymbol{)} \AgdaFunction{≈} \AgdaBound{fn} \AgdaBound{b}\AgdaSymbol{)} \<[66]%
\>[66]\<%
\\
\>[0]\AgdaIndent{1}{}\<[1]%
\>[1]\<%
\\
\>[1]\AgdaIndent{9}{}\<[9]%
\>[9]\AgdaSymbol{;} \AgdaField{\_≈h\_} \<[19]%
\>[19]\AgdaSymbol{=} \AgdaSymbol{λ\{(}\AgdaBound{f} \AgdaInductiveConstructor{,} \AgdaSymbol{\_)} \AgdaSymbol{(}\AgdaBound{g} \AgdaInductiveConstructor{,} \AgdaSymbol{\_)} \AgdaSymbol{→} \<[41]%
\>[41]\<%
\\
\>[9]\AgdaIndent{24}{}\<[24]%
\>[24]\AgdaFunction{∀'[} \AgdaBound{a} \AgdaFunction{∶} \AgdaSymbol{\_} \AgdaFunction{]} \AgdaFunction{[} \AgdaBound{Bx} \AgdaBound{a} \AgdaFunction{]} \AgdaBound{f} \AgdaBound{a} \AgdaFunction{≈h} \AgdaBound{g} \AgdaBound{a} \AgdaSymbol{\}}\<%
\\
\>[0]\AgdaIndent{9}{}\<[9]%
\>[9]\AgdaSymbol{;} \AgdaField{isEquiv} \AgdaSymbol{=} \AgdaKeyword{record} \AgdaSymbol{\{}\<%
\\
\>[0]\AgdaIndent{19}{}\<[19]%
\>[19]\AgdaField{refl} \<[25]%
\>[25]\AgdaSymbol{=} \AgdaSymbol{λ} \AgdaBound{a} \AgdaSymbol{→} \AgdaFunction{[} \AgdaBound{Bx} \AgdaBound{a} \AgdaFunction{]refl} \<[46]%
\>[46]\<%
\\
\>[0]\AgdaIndent{17}{}\<[17]%
\>[17]\AgdaSymbol{;} \AgdaField{sym} \<[25]%
\>[25]\AgdaSymbol{=} \AgdaSymbol{λ} \AgdaBound{f} \AgdaBound{a} \AgdaSymbol{→} \AgdaFunction{[} \AgdaBound{Bx} \AgdaBound{a} \AgdaFunction{]sym} \AgdaSymbol{(}\AgdaBound{f} \AgdaBound{a}\AgdaSymbol{)}\<%
\\
\>[0]\AgdaIndent{17}{}\<[17]%
\>[17]\AgdaSymbol{;} \AgdaField{trans} \AgdaSymbol{=} \AgdaSymbol{λ} \AgdaBound{f} \AgdaBound{g} \AgdaBound{a} \AgdaSymbol{→} \AgdaFunction{[} \AgdaBound{Bx} \AgdaBound{a} \AgdaFunction{]trans} \AgdaSymbol{(}\AgdaBound{f} \AgdaBound{a}\AgdaSymbol{)} \AgdaSymbol{(}\AgdaBound{g} \AgdaBound{a}\AgdaSymbol{)}\<%
\\
\>[17]\AgdaIndent{29}{}\<[29]%
\>[29]\AgdaSymbol{\}}\<%
\\
\>[3]\AgdaIndent{9}{}\<[9]%
\>[9]\AgdaSymbol{\}}\<%
\\
%
\\
\>[0]\AgdaIndent{2}{}\<[2]%
\>[2]\AgdaSymbol{;} \AgdaField{substT} \AgdaSymbol{=} \AgdaSymbol{λ} \AgdaSymbol{\{}\AgdaBound{x}\AgdaSymbol{\}} \AgdaSymbol{\{}\AgdaBound{y}\AgdaSymbol{\}} \AgdaBound{p} \AgdaSymbol{→}\<%
\\
\>[2]\AgdaIndent{19}{}\<[19]%
\>[19]\AgdaKeyword{let} \AgdaBound{y2x} \AgdaSymbol{=} \AgdaSymbol{λ} \AgdaBound{a} \AgdaSymbol{→} \AgdaFunction{[} \AgdaBound{A} \AgdaFunction{]subst} \AgdaSymbol{(}\AgdaFunction{[} \AgdaBound{Γ} \AgdaFunction{]sym} \AgdaBound{p}\AgdaSymbol{)} \AgdaBound{a} \AgdaKeyword{in}\<%
\\
\>[19]\AgdaIndent{20}{}\<[20]%
\>[20]\AgdaKeyword{let} \AgdaBound{x2y} \AgdaSymbol{=} \AgdaSymbol{λ} \AgdaBound{a} \AgdaSymbol{→} \AgdaFunction{[} \AgdaBound{A} \AgdaFunction{]subst} \AgdaBound{p} \AgdaBound{a} \AgdaKeyword{in}\<%
\\
\>[0]\AgdaIndent{19}{}\<[19]%
\>[19]\AgdaKeyword{let} \AgdaBound{p'} \AgdaSymbol{=} \AgdaSymbol{λ} \AgdaBound{a} \AgdaSymbol{→} \AgdaFunction{[} \AgdaBound{A} \AgdaFunction{]trans-refl} \AgdaKeyword{in}\<%
\\
\>[0]\AgdaIndent{13}{}\<[13]%
\>[13]\AgdaSymbol{λ\{(}\AgdaBound{f} \AgdaInductiveConstructor{,} \AgdaBound{rsp}\AgdaSymbol{)} \AgdaSymbol{→} \<[28]%
\>[28]\<%
\\
\>[13]\AgdaIndent{15}{}\<[15]%
\>[15]\AgdaSymbol{(λ} \AgdaBound{a} \AgdaSymbol{→} \AgdaFunction{[} \AgdaBound{B} \AgdaFunction{]subst} \AgdaSymbol{(}\AgdaBound{p} \AgdaInductiveConstructor{,} \AgdaBound{p'} \AgdaBound{a}\AgdaSymbol{)} \AgdaSymbol{(}\AgdaBound{f} \AgdaSymbol{(}\AgdaBound{y2x} \AgdaBound{a}\AgdaSymbol{)))}\<%
\\
\>[13]\AgdaIndent{15}{}\<[15]%
\>[15]\AgdaInductiveConstructor{,} \<[49]%
\>[49]\<%
\\
\>[13]\AgdaIndent{15}{}\<[15]%
\>[15]\AgdaSymbol{(λ} \AgdaBound{a} \AgdaBound{b} \AgdaBound{q} \AgdaSymbol{→} \<[26]%
\>[26]\<%
\\
\>[15]\AgdaIndent{16}{}\<[16]%
\>[16]\AgdaKeyword{let} \AgdaBound{a'} \AgdaSymbol{=} \AgdaBound{y2x} \AgdaBound{a} \AgdaKeyword{in} \<[34]%
\>[34]\<%
\\
\>[15]\AgdaIndent{16}{}\<[16]%
\>[16]\AgdaKeyword{let} \AgdaBound{b'} \AgdaSymbol{=} \AgdaBound{y2x} \AgdaBound{b} \AgdaKeyword{in}\<%
\\
\>[15]\AgdaIndent{16}{}\<[16]%
\>[16]\AgdaKeyword{let} \AgdaBound{q'} \AgdaSymbol{=} \AgdaFunction{[} \AgdaBound{A} \AgdaFunction{]subst*} \AgdaSymbol{(}\AgdaFunction{[} \AgdaBound{Γ} \AgdaFunction{]sym} \AgdaBound{p}\AgdaSymbol{)} \AgdaBound{q} \AgdaKeyword{in}\<%
\\
\>[15]\AgdaIndent{16}{}\<[16]%
\>[16]\AgdaKeyword{let} \AgdaBound{H} \AgdaSymbol{=} \AgdaBound{rsp} \AgdaBound{a'} \AgdaBound{b'} \AgdaBound{q'} \AgdaKeyword{in}\<%
\\
\>[15]\AgdaIndent{16}{}\<[16]%
\>[16]\AgdaKeyword{let} \AgdaBound{r} \AgdaSymbol{:} \AgdaFunction{[} \AgdaBound{Γ} \AgdaFunction{\&} \AgdaBound{A} \AgdaFunction{]} \AgdaSymbol{(}\AgdaBound{x} \AgdaInductiveConstructor{,} \AgdaBound{b'}\AgdaSymbol{)} \AgdaFunction{≈} \AgdaSymbol{(}\AgdaBound{y} \AgdaInductiveConstructor{,} \AgdaBound{b}\AgdaSymbol{)}
                    r \AgdaSymbol{=} \AgdaSymbol{(}\AgdaBound{p} \AgdaInductiveConstructor{,} \AgdaBound{p'} \AgdaBound{b}\AgdaSymbol{)} \AgdaKeyword{in}\<%
\\
\>[15]\AgdaIndent{16}{}\<[16]%
\>[16]\AgdaKeyword{let} \AgdaBound{pre} \AgdaSymbol{=} \AgdaFunction{[} \AgdaBound{B} \AgdaFunction{]subst*} \AgdaBound{r} \AgdaBound{H} \AgdaKeyword{in}\<%
\\
\>[15]\AgdaIndent{16}{}\<[16]%
\>[16]\<%
\\
\>[15]\AgdaIndent{16}{}\<[16]%
\>[16]\AgdaFunction{[} \AgdaFunction{[} \AgdaBound{B} \AgdaFunction{]fm} \AgdaSymbol{(}\AgdaBound{y} \AgdaInductiveConstructor{,} \AgdaBound{b}\AgdaSymbol{)} \AgdaFunction{]trans} \<[41]%
\>[41]\<%
\\
\>[15]\AgdaIndent{16}{}\<[16]%
\>[16]\AgdaSymbol{(}\AgdaFunction{[} \AgdaBound{B} \AgdaFunction{]trans*} \AgdaSymbol{\_} \AgdaSymbol{\_} \AgdaSymbol{\_)} \<[52]%
\>[52]\<%
\\
\>[15]\AgdaIndent{16}{}\<[16]%
\>[16]\AgdaSymbol{(}\AgdaFunction{[} \AgdaFunction{[} \AgdaBound{B} \AgdaFunction{]fm} \AgdaSymbol{(}\AgdaBound{y} \AgdaInductiveConstructor{,} \AgdaBound{b}\AgdaSymbol{)} \AgdaFunction{]trans} \<[42]%
\>[42]\<%
\\
\>[15]\AgdaIndent{16}{}\<[16]%
\>[16]\AgdaFunction{[} \AgdaBound{B} \AgdaFunction{]subst-pi} \<[30]%
\>[30]\<%
\\
\>[15]\AgdaIndent{16}{}\<[16]%
\>[16]\AgdaSymbol{(}\AgdaFunction{[} \AgdaFunction{[} \AgdaBound{B} \AgdaFunction{]fm} \AgdaSymbol{(}\AgdaBound{y} \AgdaInductiveConstructor{,} \AgdaBound{b}\AgdaSymbol{)} \AgdaFunction{]trans} \<[42]%
\>[42]\<%
\\
\>[15]\AgdaIndent{16}{}\<[16]%
\>[16]\AgdaSymbol{(}\AgdaFunction{[} \AgdaFunction{[} \AgdaBound{B} \AgdaFunction{]fm} \AgdaSymbol{(}\AgdaBound{y} \AgdaInductiveConstructor{,} \AgdaBound{b}\AgdaSymbol{)} \AgdaFunction{]sym} \<[40]%
\>[40]\<%
\\
\>[15]\AgdaIndent{16}{}\<[16]%
\>[16]\AgdaSymbol{(}\AgdaFunction{[} \AgdaBound{B} \AgdaFunction{]trans*} \AgdaSymbol{\_} \AgdaSymbol{\_} \AgdaSymbol{\_))} \<[37]%
\>[37]\<%
\\
\>[15]\AgdaIndent{16}{}\<[16]%
\>[16]\AgdaBound{pre}\AgdaSymbol{))} \<[23]%
\>[23]\<%
\\
\>[15]\AgdaIndent{16}{}\<[16]%
\>[16]\AgdaSymbol{)} \<[22]%
\>[22]\<%
\\
\>[-12]\AgdaIndent{13}{}\<[13]%
\>[13]\AgdaSymbol{\}}\<%
\\
\>[0]\AgdaIndent{2}{}\<[2]%
\>[2]\AgdaSymbol{;} \AgdaField{subst*} \AgdaSymbol{=} \AgdaSymbol{λ} \AgdaBound{\_} \AgdaBound{q} \AgdaBound{\_} \AgdaSymbol{→} \AgdaFunction{[} \AgdaBound{B} \AgdaFunction{]subst*} \AgdaSymbol{\_} \AgdaSymbol{(}\AgdaBound{q} \AgdaSymbol{\_)}\<%
\\
\>[0]\AgdaIndent{2}{}\<[2]%
\>[2]\AgdaSymbol{;} \AgdaField{refl*} \AgdaSymbol{=} \AgdaSymbol{λ} \AgdaSymbol{\{}\AgdaBound{x} \AgdaSymbol{(}\AgdaBound{f} \AgdaInductiveConstructor{,} \AgdaBound{rsp}\AgdaSymbol{)} \AgdaBound{a} \AgdaSymbol{→} \<[32]%
\>[32]\AgdaFunction{[} \AgdaFunction{[} \AgdaBound{B} \AgdaFunction{]fm} \AgdaSymbol{\_} \AgdaFunction{]trans} \<[51]%
\>[51]\<%
\\
\>[2]\AgdaIndent{17}{}\<[17]%
\>[17]\AgdaFunction{[} \AgdaBound{B} \AgdaFunction{]subst-pi} \AgdaSymbol{(}\AgdaBound{rsp} \AgdaSymbol{(}\AgdaFunction{[} \AgdaBound{A} \AgdaFunction{]subst} \<[48]%
\>[48]\<%
\\
\>[17]\AgdaIndent{21}{}\<[21]%
\>[21]\AgdaSymbol{(}\AgdaFunction{[} \AgdaBound{Γ} \AgdaFunction{]sym} \AgdaFunction{[} \AgdaBound{Γ} \AgdaFunction{]refl}\AgdaSymbol{)} \AgdaBound{a}\AgdaSymbol{)} \AgdaBound{a} \AgdaFunction{[} \AgdaBound{A} \AgdaFunction{]subst-pi'}\AgdaSymbol{)} \<[64]%
\>[64]\AgdaSymbol{\}}\<%
\\
\>[2]\AgdaIndent{2}{}\<[2]%
\>[2]\AgdaSymbol{;} \AgdaField{trans*} \AgdaSymbol{=} \AgdaSymbol{λ} \AgdaBound{p} \AgdaBound{q} \AgdaSymbol{→} \AgdaSymbol{λ} \AgdaSymbol{\{(}\AgdaBound{f} \AgdaInductiveConstructor{,} \AgdaBound{rsp}\AgdaSymbol{)} \AgdaBound{a} \AgdaSymbol{→}\<%
\\
\>[0]\AgdaIndent{13}{}\<[13]%
\>[13]\AgdaFunction{[} \AgdaFunction{[} \AgdaBound{B} \AgdaFunction{]fm} \AgdaSymbol{\_} \AgdaFunction{]trans} \<[32]%
\>[32]\<%
\\
\>[0]\AgdaIndent{13}{}\<[13]%
\>[13]\AgdaSymbol{(}\AgdaFunction{[} \AgdaFunction{[} \AgdaBound{B} \AgdaFunction{]fm} \AgdaSymbol{\_} \AgdaFunction{]trans} \<[33]%
\>[33]\<%
\\
\>[0]\AgdaIndent{13}{}\<[13]%
\>[13]\AgdaSymbol{(}\AgdaFunction{[} \AgdaBound{B} \AgdaFunction{]trans*} \AgdaSymbol{\_} \AgdaSymbol{\_} \AgdaSymbol{\_)} \<[33]%
\>[33]\<%
\\
\>[0]\AgdaIndent{13}{}\<[13]%
\>[13]\AgdaSymbol{(}\AgdaFunction{[} \AgdaFunction{[} \AgdaBound{B} \AgdaFunction{]fm} \AgdaSymbol{\_} \AgdaFunction{]sym} \<[31]%
\>[31]\<%
\\
\>[0]\AgdaIndent{13}{}\<[13]%
\>[13]\AgdaSymbol{(}\AgdaFunction{[} \AgdaFunction{[} \AgdaBound{B} \AgdaFunction{]fm} \AgdaSymbol{\_} \AgdaFunction{]trans} \<[33]%
\>[33]\<%
\\
\>[0]\AgdaIndent{13}{}\<[13]%
\>[13]\AgdaSymbol{(}\AgdaFunction{[} \AgdaBound{B} \AgdaFunction{]trans*} \AgdaSymbol{\_} \AgdaSymbol{\_} \AgdaSymbol{\_)} \AgdaFunction{[} \AgdaBound{B} \AgdaFunction{]subst-pi}\AgdaSymbol{)))} \<[50]%
\>[50]\<%
\\
\>[0]\AgdaIndent{13}{}\<[13]%
\>[13]\AgdaSymbol{(}\AgdaFunction{[} \AgdaBound{B} \AgdaFunction{]subst*} \AgdaSymbol{\_} \AgdaSymbol{(}\AgdaBound{rsp} \AgdaSymbol{\_} \AgdaSymbol{\_} \<[37]%
\>[37]\<%
\\
\>[0]\AgdaIndent{13}{}\<[13]%
\>[13]\AgdaSymbol{(}\AgdaFunction{[} \AgdaFunction{[} \AgdaBound{A} \AgdaFunction{]fm} \AgdaSymbol{\_} \AgdaFunction{]trans} \<[33]%
\>[33]\<%
\\
\>[0]\AgdaIndent{13}{}\<[13]%
\>[13]\AgdaSymbol{(}\AgdaFunction{[} \AgdaBound{A} \AgdaFunction{]trans*} \AgdaSymbol{\_} \AgdaSymbol{\_} \AgdaSymbol{\_)} \AgdaFunction{[} \AgdaBound{A} \AgdaFunction{]subst-pi}\AgdaSymbol{)))} \AgdaSymbol{\}} \<[52]%
\>[52]\<%
\\
\>[0]\AgdaIndent{2}{}\<[2]%
\>[2]\AgdaSymbol{\}}\<%
\\
%
\\
\>\<\end{code}
}

It also comes with two necessary operation on the terms of $Pi$-types, $\lambda$-abstraction and application.
There are $\beta-\eta$ laws to verfify for them so that we could form an isomorphism with these two operations. however technically it causes stack overflow. We may simplify these definition in the future so that we could verify them in Agda.

%to do : verification of β and η
%cause stack overflow

\begin{code}\>\<%
\\
%
\\
\>\AgdaFunction{lam} \AgdaSymbol{:} \AgdaSymbol{\{}\AgdaBound{Γ} \AgdaSymbol{:} \AgdaFunction{Con}\AgdaSymbol{\}\{}\AgdaBound{A} \AgdaSymbol{:} \AgdaRecord{Ty} \AgdaBound{Γ}\AgdaSymbol{\}\{}\AgdaBound{B} \AgdaSymbol{:} \AgdaRecord{Ty} \AgdaSymbol{(}\AgdaBound{Γ} \AgdaFunction{\&} \AgdaBound{A}\AgdaSymbol{)\}} \AgdaSymbol{→} \AgdaRecord{Tm} \AgdaBound{B} \AgdaSymbol{→} \AgdaRecord{Tm} \AgdaSymbol{(}\AgdaFunction{Π} \AgdaBound{A} \AgdaBound{B}\AgdaSymbol{)}\<%
\\
\>\<\end{code}

\AgdaHide{
\begin{code}\>\<%
\\
\>\AgdaFunction{lam} \AgdaSymbol{\{}\AgdaBound{Γ}\AgdaSymbol{\}} \AgdaSymbol{\{}\AgdaBound{A}\AgdaSymbol{\}} \AgdaSymbol{(}\AgdaInductiveConstructor{tm:} \AgdaBound{tm} \AgdaInductiveConstructor{resp:} \AgdaBound{respt}\AgdaSymbol{)} \AgdaSymbol{=} \<[35]%
\>[35]\<%
\\
\>[0]\AgdaIndent{2}{}\<[2]%
\>[2]\AgdaKeyword{record} \AgdaSymbol{\{} \AgdaField{tm} \AgdaSymbol{=} \AgdaSymbol{λ} \AgdaBound{x} \AgdaSymbol{→} \AgdaSymbol{(λ} \AgdaBound{a} \AgdaSymbol{→} \AgdaBound{tm} \AgdaSymbol{(}\AgdaBound{x} \AgdaInductiveConstructor{,} \AgdaBound{a}\AgdaSymbol{))} \AgdaInductiveConstructor{,} \<[43]%
\>[43]\<%
\\
\>[2]\AgdaIndent{11}{}\<[11]%
\>[11]\AgdaSymbol{(λ} \AgdaBound{a} \AgdaBound{b} \AgdaBound{p} \AgdaSymbol{→} \AgdaBound{respt} \AgdaSymbol{(}\AgdaFunction{[} \AgdaBound{Γ} \AgdaFunction{]refl} \AgdaInductiveConstructor{,}\<%
\\
\>[11]\AgdaIndent{13}{}\<[13]%
\>[13]\AgdaFunction{[} \AgdaFunction{[} \AgdaBound{A} \AgdaFunction{]fm} \AgdaBound{x} \AgdaFunction{]trans} \AgdaSymbol{(}\AgdaFunction{[} \AgdaBound{A} \AgdaFunction{]refl*} \AgdaSymbol{\_} \AgdaSymbol{\_)} \AgdaBound{p}\AgdaSymbol{))}\<%
\\
\>[-7]\AgdaIndent{9}{}\<[9]%
\>[9]\AgdaSymbol{;} \AgdaField{respt} \AgdaSymbol{=} \AgdaSymbol{λ} \AgdaBound{p} \AgdaBound{\_} \AgdaSymbol{→} \AgdaBound{respt} \AgdaSymbol{(}\AgdaBound{p} \AgdaInductiveConstructor{,} \AgdaFunction{[} \AgdaBound{A} \AgdaFunction{]trans-refl}\AgdaSymbol{)} \<[55]%
\>[55]\<%
\\
\>[0]\AgdaIndent{9}{}\<[9]%
\>[9]\AgdaSymbol{\}}\<%
\\
%
\\
\>\<\end{code}
}


\begin{code}\>\<%
\\
\>\AgdaFunction{app} \AgdaSymbol{:} \AgdaSymbol{\{}\AgdaBound{Γ} \AgdaSymbol{:} \AgdaFunction{Con}\AgdaSymbol{\}\{}\AgdaBound{A} \AgdaSymbol{:} \AgdaRecord{Ty} \AgdaBound{Γ}\AgdaSymbol{\}\{}\AgdaBound{B} \AgdaSymbol{:} \AgdaRecord{Ty} \AgdaSymbol{(}\AgdaBound{Γ} \AgdaFunction{\&} \AgdaBound{A}\AgdaSymbol{)\}} \AgdaSymbol{→} \AgdaRecord{Tm} \AgdaSymbol{(}\AgdaFunction{Π} \AgdaBound{A} \AgdaBound{B}\AgdaSymbol{)} \AgdaSymbol{→} \AgdaRecord{Tm} \AgdaBound{B}\<%
\\
%
\\
\>\<\end{code}

\AgdaHide{
\begin{code}\>\<%
\\
\>\AgdaFunction{app} \AgdaSymbol{\{}\AgdaBound{Γ}\AgdaSymbol{\}} \AgdaSymbol{\{}\AgdaBound{A}\AgdaSymbol{\}} \AgdaSymbol{\{}\AgdaBound{B}\AgdaSymbol{\}} \AgdaSymbol{(}\AgdaInductiveConstructor{tm:} \AgdaBound{tm} \AgdaInductiveConstructor{resp:} \AgdaBound{respt}\AgdaSymbol{)} \AgdaSymbol{=} \<[39]%
\>[39]\<%
\\
\>[0]\AgdaIndent{2}{}\<[2]%
\>[2]\AgdaKeyword{record} \AgdaSymbol{\{} \AgdaField{tm} \AgdaSymbol{=} \AgdaSymbol{λ} \AgdaSymbol{\{(}\AgdaBound{x} \AgdaInductiveConstructor{,} \AgdaBound{a}\AgdaSymbol{)} \AgdaSymbol{→} \AgdaFunction{proj₁} \AgdaSymbol{(}\AgdaBound{tm} \AgdaBound{x}\AgdaSymbol{)} \AgdaBound{a}\AgdaSymbol{\}}\<%
\\
\>[0]\AgdaIndent{9}{}\<[9]%
\>[9]\AgdaSymbol{;} \AgdaField{respt} \AgdaSymbol{=} \AgdaSymbol{λ} \AgdaSymbol{\{}\AgdaBound{x}\AgdaSymbol{\}} \AgdaSymbol{\{}\AgdaBound{y}\AgdaSymbol{\}} \AgdaSymbol{→} \AgdaSymbol{λ} \AgdaSymbol{\{(}\AgdaBound{p} \AgdaInductiveConstructor{,} \AgdaBound{tr}\AgdaSymbol{)} \AgdaSymbol{→} \<[45]%
\>[45]\<%
\\
\>[9]\AgdaIndent{13}{}\<[13]%
\>[13]\AgdaKeyword{let} \AgdaBound{fresp} \AgdaSymbol{=} \AgdaFunction{proj₂} \AgdaSymbol{(}\AgdaBound{tm} \AgdaSymbol{(}\AgdaFunction{proj₁} \AgdaBound{x}\AgdaSymbol{))} \AgdaKeyword{in}\<%
\\
\>[13]\AgdaIndent{16}{}\<[16]%
\>[16]\AgdaFunction{[} \AgdaFunction{[} \AgdaBound{B} \AgdaFunction{]fm} \AgdaSymbol{\_} \AgdaFunction{]trans} \<[35]%
\>[35]\<%
\\
\>[13]\AgdaIndent{16}{}\<[16]%
\>[16]\AgdaSymbol{(}\AgdaFunction{[} \AgdaBound{B} \AgdaFunction{]subst*} \AgdaSymbol{(}\AgdaBound{p} \AgdaInductiveConstructor{,} \AgdaBound{tr}\AgdaSymbol{)} \AgdaSymbol{(}\AgdaFunction{[} \AgdaFunction{[} \AgdaBound{B} \AgdaFunction{]fm} \AgdaSymbol{\_} \AgdaFunction{]sym} \AgdaFunction{[} \AgdaBound{B} \AgdaFunction{]subst-pi'}\AgdaSymbol{))} \<[73]%
\>[73]\<%
\\
\>[13]\AgdaIndent{16}{}\<[16]%
\>[16]\AgdaSymbol{(}\AgdaFunction{[} \AgdaFunction{[} \AgdaBound{B} \AgdaFunction{]fm} \AgdaSymbol{\_} \AgdaFunction{]trans}\<%
\\
\>[13]\AgdaIndent{16}{}\<[16]%
\>[16]\AgdaSymbol{(}\AgdaFunction{[} \AgdaBound{B} \AgdaFunction{]trans*} \AgdaSymbol{(}\AgdaFunction{[} \AgdaBound{Γ} \AgdaFunction{]refl} \AgdaInductiveConstructor{,} \AgdaFunction{[} \AgdaBound{A} \AgdaFunction{]refl*} \AgdaSymbol{\_} \AgdaSymbol{\_)} \AgdaSymbol{\_} \AgdaSymbol{\_)} \<[63]%
\>[63]\<%
\\
\>[13]\AgdaIndent{16}{}\<[16]%
\>[16]\AgdaSymbol{(}\AgdaFunction{[} \AgdaFunction{[} \AgdaBound{B} \AgdaFunction{]fm} \AgdaSymbol{\_} \AgdaFunction{]trans} \<[36]%
\>[36]\<%
\\
\>[13]\AgdaIndent{16}{}\<[16]%
\>[16]\AgdaFunction{[} \AgdaBound{B} \AgdaFunction{]subst-pi} \<[30]%
\>[30]\<%
\\
\>[13]\AgdaIndent{16}{}\<[16]%
\>[16]\AgdaSymbol{(}\AgdaFunction{[} \AgdaFunction{[} \AgdaBound{B} \AgdaFunction{]fm} \AgdaSymbol{\_} \AgdaFunction{]trans} \<[36]%
\>[36]\<%
\\
\>[13]\AgdaIndent{16}{}\<[16]%
\>[16]\AgdaSymbol{(}\AgdaFunction{[} \AgdaFunction{[} \AgdaBound{B} \AgdaFunction{]fm} \AgdaSymbol{\_} \AgdaFunction{]sym} \AgdaSymbol{(}\AgdaFunction{[} \AgdaBound{B} \AgdaFunction{]trans*} \AgdaSymbol{\_} \AgdaSymbol{(}\AgdaBound{p} \AgdaInductiveConstructor{,} \AgdaFunction{[} \AgdaBound{A} \AgdaFunction{]trans-refl}\AgdaSymbol{)} \AgdaSymbol{\_))}\<%
\\
\>[13]\AgdaIndent{16}{}\<[16]%
\>[16]\AgdaSymbol{(}\AgdaFunction{[} \AgdaFunction{[} \AgdaBound{B} \AgdaFunction{]fm} \AgdaSymbol{\_} \AgdaFunction{]trans} \<[36]%
\>[36]\<%
\\
\>[13]\AgdaIndent{16}{}\<[16]%
\>[16]\AgdaSymbol{(}\AgdaFunction{[} \AgdaBound{B} \AgdaFunction{]subst-pi*} \AgdaSymbol{(}\AgdaBound{fresp} \AgdaSymbol{\_} \AgdaSymbol{\_} \AgdaSymbol{(}\AgdaFunction{[} \AgdaBound{A} \AgdaFunction{]subst-mir2} \AgdaBound{tr}\AgdaSymbol{)))} \<[66]%
\>[66]\<%
\\
\>[13]\AgdaIndent{16}{}\<[16]%
\>[16]\AgdaSymbol{(}\AgdaBound{respt} \AgdaBound{p} \AgdaSymbol{\_)))))} \AgdaSymbol{\}}\<%
\\
\>[-6]\AgdaIndent{9}{}\<[9]%
\>[9]\AgdaSymbol{\}}\<%
\\
%
\\
%
\\
\>\<\end{code}
}

Non-dependent version of $\Pi$-types namely function types can be defined with type weakening. Since the dependence disappears, it is possible to define it straightforwardly without using $\Pi$-types.

\begin{code}\>\<%
\\
%
\\
\>\AgdaFunction{\_⇒'\_} \AgdaSymbol{:} \AgdaSymbol{\{}\AgdaBound{Γ} \AgdaSymbol{:} \AgdaFunction{Con}\AgdaSymbol{\}(}\AgdaBound{A} \AgdaBound{B} \AgdaSymbol{:} \AgdaRecord{Ty} \AgdaBound{Γ}\AgdaSymbol{)} \AgdaSymbol{→} \AgdaRecord{Ty} \AgdaBound{Γ}\<%
\\
\>\AgdaBound{A} \AgdaFunction{⇒'} \AgdaBound{B} \AgdaSymbol{=} \AgdaFunction{Π} \AgdaBound{A} \AgdaSymbol{(}\AgdaBound{B} \AgdaFunction{+T} \AgdaBound{A}\AgdaSymbol{)}\<%
\\
%
\\
\>\<\end{code}


\AgdaHide{
\begin{code}\>\<%
\\
%
\\
\>\AgdaFunction{[\_,\_]\_⇒fm\_} \AgdaSymbol{:} \AgdaSymbol{(}\AgdaBound{Γ} \AgdaSymbol{:} \AgdaFunction{Con}\AgdaSymbol{)(}\AgdaBound{x} \AgdaSymbol{:} \AgdaFunction{∣} \AgdaBound{Γ} \AgdaFunction{∣}\AgdaSymbol{)} \<[34]%
\>[34]\<%
\\
\>[0]\AgdaIndent{11}{}\<[11]%
\>[11]\AgdaSymbol{→} \AgdaRecord{hSetoid} \AgdaSymbol{→} \AgdaRecord{hSetoid} \AgdaSymbol{→} \AgdaRecord{hSetoid}\<%
\\
\>\AgdaFunction{[} \AgdaBound{Γ} \AgdaFunction{,} \AgdaBound{x} \AgdaFunction{]} \AgdaBound{Ax} \AgdaFunction{⇒fm} \AgdaBound{Bx} \<[20]%
\>[20]\<%
\\
\>[0]\AgdaIndent{2}{}\<[2]%
\>[2]\AgdaSymbol{=} \AgdaKeyword{record}\<%
\\
\>[0]\AgdaIndent{6}{}\<[6]%
\>[6]\AgdaSymbol{\{} \AgdaField{Carrier} \AgdaSymbol{=} \AgdaRecord{Σ[} \AgdaBound{fn} \AgdaRecord{∶} \AgdaSymbol{(}\AgdaFunction{∣} \AgdaBound{Ax} \AgdaFunction{∣} \AgdaSymbol{→} \AgdaFunction{∣} \AgdaBound{Bx} \AgdaFunction{∣}\AgdaSymbol{)} \AgdaRecord{]} \<[46]%
\>[46]\<%
\\
\>[6]\AgdaIndent{16}{}\<[16]%
\>[16]\AgdaSymbol{((}\AgdaBound{a} \AgdaBound{b} \AgdaSymbol{:} \AgdaFunction{∣} \AgdaBound{Ax} \AgdaFunction{∣}\AgdaSymbol{)(}\AgdaBound{p} \AgdaSymbol{:} \AgdaFunction{[} \AgdaBound{Ax} \AgdaFunction{]} \AgdaBound{a} \AgdaFunction{≈} \AgdaBound{b}\AgdaSymbol{)} \<[50]%
\>[50]\<%
\\
\>[16]\AgdaIndent{18}{}\<[18]%
\>[18]\AgdaSymbol{→} \AgdaFunction{[} \AgdaBound{Bx} \AgdaFunction{]} \AgdaBound{fn} \AgdaBound{a} \AgdaFunction{≈} \AgdaBound{fn} \AgdaBound{b}\AgdaSymbol{)}\<%
\\
\>[-4]\AgdaIndent{6}{}\<[6]%
\>[6]\AgdaSymbol{;} \AgdaField{\_≈h\_} \<[16]%
\>[16]\AgdaSymbol{=} \AgdaSymbol{λ\{(}\AgdaBound{f} \AgdaInductiveConstructor{,} \AgdaSymbol{\_)} \AgdaSymbol{(}\AgdaBound{g} \AgdaInductiveConstructor{,} \AgdaSymbol{\_)} \<[36]%
\>[36]\<%
\\
\>[0]\AgdaIndent{18}{}\<[18]%
\>[18]\AgdaSymbol{→} \AgdaFunction{∀'[} \AgdaBound{a} \AgdaFunction{∶} \AgdaSymbol{\_} \AgdaFunction{]} \AgdaFunction{[} \AgdaBound{Bx} \AgdaFunction{]} \AgdaBound{f} \AgdaBound{a} \AgdaFunction{≈h} \AgdaBound{g} \AgdaBound{a} \AgdaSymbol{\}}\<%
\\
\>[0]\AgdaIndent{6}{}\<[6]%
\>[6]\AgdaSymbol{;} \AgdaField{isEquiv} \AgdaSymbol{=} \AgdaKeyword{record} \AgdaSymbol{\{}\<%
\\
\>[0]\AgdaIndent{16}{}\<[16]%
\>[16]\AgdaField{refl} \<[22]%
\>[22]\AgdaSymbol{=} \AgdaSymbol{λ} \AgdaBound{\_} \AgdaSymbol{→} \AgdaFunction{[} \AgdaBound{Bx} \AgdaFunction{]refl} \<[41]%
\>[41]\<%
\\
\>[0]\AgdaIndent{14}{}\<[14]%
\>[14]\AgdaSymbol{;} \AgdaField{sym} \<[22]%
\>[22]\AgdaSymbol{=} \AgdaSymbol{λ} \AgdaBound{f} \AgdaBound{a} \AgdaSymbol{→} \AgdaFunction{[} \AgdaBound{Bx} \AgdaFunction{]sym} \AgdaSymbol{(}\AgdaBound{f} \AgdaBound{a}\AgdaSymbol{)}\<%
\\
\>[0]\AgdaIndent{14}{}\<[14]%
\>[14]\AgdaSymbol{;} \AgdaField{trans} \AgdaSymbol{=} \AgdaSymbol{λ} \AgdaBound{f} \AgdaBound{g} \AgdaBound{a} \AgdaSymbol{→} \AgdaFunction{[} \AgdaBound{Bx} \AgdaFunction{]trans} \AgdaSymbol{(}\AgdaBound{f} \AgdaBound{a}\AgdaSymbol{)} \AgdaSymbol{(}\AgdaBound{g} \AgdaBound{a}\AgdaSymbol{)}\<%
\\
\>[14]\AgdaIndent{26}{}\<[26]%
\>[26]\AgdaSymbol{\}}\<%
\\
\>[6]\AgdaIndent{6}{}\<[6]%
\>[6]\AgdaSymbol{\}}\<%
\\
%
\\
%
\\
\>\<\end{code}

}

%to do: verification

%verification of functor laws (do we have extensional equality for record types? or eta equality?)
%define equality with respect to propositions which are proof irrelevant



\AgdaHide{
\begin{code}\>\<%
\\
%
\\
\>\AgdaSymbol{\{-\#} \AgdaKeyword{OPTIONS} --type-in-type \AgdaSymbol{\#-\}}\<%
\\
%
\\
\>\AgdaKeyword{import} \AgdaModule{Level}\<%
\\
\>\AgdaKeyword{open} \AgdaKeyword{import} \AgdaModule{Relation.Binary.PropositionalEquality} \AgdaSymbol{as} \AgdaModule{PE} \AgdaKeyword{hiding} \AgdaSymbol{(}refl \AgdaSymbol{;} sym \AgdaSymbol{;} trans\AgdaSymbol{;} isEquivalence\AgdaSymbol{;} [\_]\AgdaSymbol{)}\<%
\\
%
\\
\>\AgdaKeyword{module} \AgdaModule{CwF-quotient} \AgdaSymbol{(}\AgdaBound{ext} \AgdaSymbol{:} \AgdaFunction{Extensionality} \AgdaPrimitive{Level.zero} \AgdaPrimitive{Level.zero}\AgdaSymbol{)} \AgdaKeyword{where}\<%
\\
%
\\
\>\AgdaKeyword{open} \AgdaKeyword{import} \AgdaModule{Data.Unit}\<%
\\
\>\AgdaKeyword{open} \AgdaKeyword{import} \AgdaModule{Function}\<%
\\
\>\AgdaKeyword{open} \AgdaKeyword{import} \AgdaModule{Data.Product}\<%
\\
%
\\
%
\\
\>\AgdaComment{-- importing other CWF files}\<%
\\
%
\\
\>\AgdaKeyword{import} \AgdaModule{CwF-setoid}\<%
\\
%
\\
\>\AgdaKeyword{open} \AgdaModule{CwF-setoid} \AgdaBound{ext}\<%
\\
%
\\
\>\AgdaKeyword{import} \AgdaModule{CategoryOfSetoid}\<%
\\
\>\AgdaKeyword{module} \AgdaModule{cos'} \AgdaSymbol{=} \AgdaModule{CategoryOfSetoid} \AgdaBound{ext}\<%
\\
\>\AgdaKeyword{open} \AgdaModule{cos'}\<%
\\
%
\\
\>\AgdaKeyword{import} \AgdaModule{hProp}\<%
\\
\>\AgdaKeyword{module} \AgdaModule{hp'} \AgdaSymbol{=} \AgdaModule{hProp} \AgdaBound{ext}\<%
\\
\>\AgdaKeyword{open} \AgdaModule{hp'}\<%
\\
%
\\
\>\AgdaKeyword{import} \AgdaModule{CwF-ctd}\<%
\\
\>\AgdaKeyword{module} \AgdaModule{cc} \AgdaSymbol{=} \AgdaModule{CwF-ctd} \AgdaBound{ext}\<%
\\
\>\AgdaKeyword{open} \AgdaModule{cc}\<%
\\
%
\\
%
\\
\>\<\end{code}
}


The equality type is an essential part of a type theory. We could define it by using the equivalence relation from the setoid representation of type A. The equivalence relation is trivial since it is proof-irrelevant.

\begin{code}\>\<%
\\
%
\\
\>\AgdaFunction{Rel} \AgdaSymbol{:} \AgdaSymbol{\{}\AgdaBound{Γ} \AgdaSymbol{:} \AgdaFunction{Con}\AgdaSymbol{\}} \AgdaSymbol{→} \AgdaRecord{Ty} \AgdaBound{Γ} \AgdaSymbol{→} \AgdaPrimitiveType{Set₁}\<%
\\
\>\AgdaFunction{Rel} \AgdaSymbol{\{}\AgdaBound{Γ}\AgdaSymbol{\}} \AgdaBound{A} \AgdaSymbol{=} \AgdaRecord{Ty} \AgdaSymbol{(}\AgdaBound{Γ} \AgdaFunction{\&} \AgdaBound{A} \AgdaFunction{\&} \AgdaBound{A} \AgdaFunction{+T} \AgdaBound{A}\AgdaSymbol{)}\<%
\\
%
\\
\>\AgdaFunction{⟦Id⟧} \AgdaSymbol{:} \AgdaSymbol{\{}\AgdaBound{Γ} \AgdaSymbol{:} \AgdaFunction{Con}\AgdaSymbol{\}(}\AgdaBound{A} \AgdaSymbol{:} \AgdaRecord{Ty} \AgdaBound{Γ}\AgdaSymbol{)} \AgdaSymbol{→} \AgdaFunction{Rel} \AgdaBound{A}\<%
\\
\>\AgdaFunction{⟦Id⟧} \AgdaBound{A}\<%
\\
\>[0]\AgdaIndent{3}{}\<[3]%
\>[3]\AgdaSymbol{=} \AgdaKeyword{record} \<[12]%
\>[12]\<%
\\
\>[3]\AgdaIndent{7}{}\<[7]%
\>[7]\AgdaSymbol{\{} \AgdaField{fm} \AgdaSymbol{=} \AgdaSymbol{λ} \AgdaSymbol{\{((}\AgdaBound{x} \AgdaInductiveConstructor{,} \AgdaBound{a}\AgdaSymbol{)} \AgdaInductiveConstructor{,} \AgdaBound{b}\AgdaSymbol{)} \AgdaSymbol{→} \<[34]%
\>[34]\AgdaKeyword{record}\<%
\\
\>[7]\AgdaIndent{9}{}\<[9]%
\>[9]\AgdaSymbol{\{} \AgdaField{Carrier} \AgdaSymbol{=} \AgdaFunction{[} \AgdaFunction{[} \AgdaBound{A} \AgdaFunction{]fm} \AgdaBound{x} \AgdaFunction{]} \AgdaBound{a} \AgdaFunction{≈} \AgdaBound{b}\<%
\\
\>[7]\AgdaIndent{9}{}\<[9]%
\>[9]\AgdaSymbol{;} \AgdaField{\_≈h\_} \AgdaSymbol{=} \AgdaSymbol{λ} \AgdaBound{x₁} \AgdaBound{x₂} \AgdaSymbol{→} \AgdaFunction{⊤'}\<%
\\
\>[7]\AgdaIndent{9}{}\<[9]%
\>[9]\AgdaSymbol{;} \AgdaField{isEquiv} \AgdaSymbol{=} \AgdaKeyword{record}\<%
\\
\>[9]\AgdaIndent{13}{}\<[13]%
\>[13]\AgdaSymbol{\{} \AgdaField{refl} \AgdaSymbol{=} \AgdaSymbol{λ} \AgdaSymbol{\{}\AgdaBound{x₁}\AgdaSymbol{\}} \AgdaSymbol{→} \AgdaInductiveConstructor{tt}\<%
\\
\>[9]\AgdaIndent{13}{}\<[13]%
\>[13]\AgdaSymbol{;} \AgdaField{sym} \AgdaSymbol{=} \AgdaSymbol{λ} \AgdaBound{x₂} \AgdaSymbol{→} \AgdaInductiveConstructor{tt}\<%
\\
\>[9]\AgdaIndent{13}{}\<[13]%
\>[13]\AgdaSymbol{;} \AgdaField{trans} \AgdaSymbol{=} \AgdaSymbol{λ} \AgdaBound{x₂} \AgdaBound{x₃} \AgdaSymbol{→} \AgdaInductiveConstructor{tt}\<%
\\
\>[9]\AgdaIndent{13}{}\<[13]%
\>[13]\AgdaSymbol{\}}\<%
\\
\>[0]\AgdaIndent{9}{}\<[9]%
\>[9]\AgdaSymbol{\}} \AgdaSymbol{\}}\<%
\\
\>[0]\AgdaIndent{7}{}\<[7]%
\>[7]\AgdaSymbol{;} \AgdaField{substT} \AgdaSymbol{=} \AgdaSymbol{λ} \AgdaSymbol{\{((}\AgdaBound{x} \AgdaInductiveConstructor{,} \AgdaBound{a}\AgdaSymbol{)} \AgdaInductiveConstructor{,} \AgdaBound{b}\AgdaSymbol{)} \AgdaBound{x0} \AgdaSymbol{→} \<[40]%
\>[40]\<%
\\
\>[7]\AgdaIndent{15}{}\<[15]%
\>[15]\AgdaFunction{[} \AgdaFunction{[} \AgdaBound{A} \AgdaFunction{]fm} \AgdaSymbol{\_} \AgdaFunction{]trans} \<[34]%
\>[34]\<%
\\
\>[7]\AgdaIndent{15}{}\<[15]%
\>[15]\AgdaSymbol{(}\AgdaFunction{[} \AgdaFunction{[} \AgdaBound{A} \AgdaFunction{]fm} \AgdaSymbol{\_} \AgdaFunction{]sym} \AgdaBound{a}\AgdaSymbol{)} \<[36]%
\>[36]\<%
\\
\>[7]\AgdaIndent{15}{}\<[15]%
\>[15]\AgdaSymbol{(}\AgdaFunction{[} \AgdaFunction{[} \AgdaBound{A} \AgdaFunction{]fm} \AgdaSymbol{\_} \AgdaFunction{]trans} \<[35]%
\>[35]\<%
\\
\>[7]\AgdaIndent{15}{}\<[15]%
\>[15]\AgdaSymbol{(}\AgdaFunction{[} \AgdaBound{A} \AgdaFunction{]subst*} \AgdaSymbol{\_} \AgdaBound{x0}\AgdaSymbol{)} \AgdaBound{b}\AgdaSymbol{)} \<[37]%
\>[37]\<%
\\
\>[7]\AgdaIndent{15}{}\<[15]%
\>[15]\AgdaSymbol{\}}\<%
\\
\>[0]\AgdaIndent{7}{}\<[7]%
\>[7]\AgdaSymbol{;} \AgdaField{subst*} \AgdaSymbol{=} \AgdaSymbol{λ} \AgdaBound{p} \AgdaBound{x₁} \AgdaSymbol{→} \AgdaInductiveConstructor{tt}\<%
\\
\>[0]\AgdaIndent{7}{}\<[7]%
\>[7]\AgdaSymbol{;} \AgdaField{refl*} \AgdaSymbol{=} \AgdaSymbol{λ} \AgdaBound{x} \AgdaBound{a} \AgdaSymbol{→} \AgdaInductiveConstructor{tt}\<%
\\
\>[0]\AgdaIndent{7}{}\<[7]%
\>[7]\AgdaSymbol{;} \AgdaField{trans*} \AgdaSymbol{=} \AgdaSymbol{λ} \AgdaBound{p} \AgdaBound{q} \AgdaBound{a} \AgdaSymbol{→} \AgdaInductiveConstructor{tt} \AgdaSymbol{\}}\<%
\\
%
\\
\>\<\end{code}

The unique inhabitant $refl$ is defined as

\begin{code}\>\<%
\\
%
\\
%
\\
\>\AgdaFunction{cm-refl} \AgdaSymbol{:} \AgdaSymbol{\{}\AgdaBound{Γ} \AgdaSymbol{:} \AgdaFunction{Con}\AgdaSymbol{\}(}\AgdaBound{A} \AgdaSymbol{:} \AgdaRecord{Ty} \AgdaBound{Γ}\AgdaSymbol{)} \AgdaSymbol{→} \AgdaBound{Γ} \AgdaFunction{\&} \AgdaBound{A} \AgdaRecord{⇉} \AgdaSymbol{(}\AgdaBound{Γ} \AgdaFunction{\&} \AgdaBound{A} \AgdaFunction{\&} \AgdaBound{A} \AgdaFunction{+T} \AgdaBound{A}\AgdaSymbol{)}\<%
\\
\>\AgdaFunction{cm-refl} \AgdaBound{A} \AgdaSymbol{=} \AgdaKeyword{record} \AgdaSymbol{\{} \AgdaField{fn} \AgdaSymbol{=} \AgdaSymbol{λ} \AgdaBound{x'} \AgdaSymbol{→} \AgdaBound{x'} \AgdaInductiveConstructor{,} \AgdaFunction{proj₂} \AgdaBound{x'} \<[47]%
\>[47]\<%
\\
\>[7]\AgdaIndent{19}{}\<[19]%
\>[19]\AgdaSymbol{;} \AgdaField{resp} \AgdaSymbol{=} \AgdaSymbol{λ} \AgdaBound{x'} \AgdaSymbol{→} \AgdaBound{x'} \AgdaInductiveConstructor{,} \AgdaFunction{proj₂} \AgdaBound{x'} \AgdaSymbol{\}}\<%
\\
%
\\
\>\AgdaFunction{⟦refl⟧⁰} \AgdaSymbol{:} \AgdaSymbol{\{}\AgdaBound{Γ} \AgdaSymbol{:} \AgdaFunction{Con}\AgdaSymbol{\}(}\AgdaBound{A} \AgdaSymbol{:} \AgdaRecord{Ty} \AgdaBound{Γ}\AgdaSymbol{)} \<[30]%
\>[30]\<%
\\
\>[0]\AgdaIndent{7}{}\<[7]%
\>[7]\AgdaSymbol{→} \AgdaRecord{Tm} \AgdaSymbol{\{}\AgdaBound{Γ} \AgdaFunction{\&} \AgdaBound{A}\AgdaSymbol{\}} \AgdaSymbol{(}\AgdaFunction{⟦Id⟧} \AgdaBound{A}\<%
\\
\>[0]\AgdaIndent{10}{}\<[10]%
\>[10]\AgdaFunction{[} \AgdaFunction{cm-refl} \AgdaBound{A} \AgdaFunction{]T}\AgdaSymbol{)} \<[26]%
\>[26]\<%
\\
\>\AgdaFunction{⟦refl⟧⁰} \AgdaBound{A} \AgdaSymbol{=} \AgdaKeyword{record}\<%
\\
\>[10]\AgdaIndent{11}{}\<[11]%
\>[11]\AgdaSymbol{\{} \AgdaField{tm} \AgdaSymbol{=} \AgdaSymbol{λ} \AgdaSymbol{\{(}\AgdaBound{x} \AgdaInductiveConstructor{,} \AgdaBound{a}\AgdaSymbol{)} \AgdaSymbol{→} \AgdaFunction{[} \AgdaFunction{[} \AgdaBound{A} \AgdaFunction{]fm} \AgdaBound{x} \AgdaFunction{]refl} \AgdaSymbol{\{}\AgdaBound{a}\AgdaSymbol{\}} \AgdaSymbol{\}}\<%
\\
\>[10]\AgdaIndent{11}{}\<[11]%
\>[11]\AgdaSymbol{;} \AgdaField{respt} \AgdaSymbol{=} \AgdaSymbol{λ} \AgdaBound{p} \AgdaSymbol{→} \AgdaInductiveConstructor{tt}\<%
\\
\>[10]\AgdaIndent{11}{}\<[11]%
\>[11]\AgdaSymbol{\}}\<%
\\
%
\\
\>\AgdaFunction{⟦refl⟧} \AgdaSymbol{:} \AgdaSymbol{\{}\AgdaBound{Γ} \AgdaSymbol{:} \AgdaFunction{Con}\AgdaSymbol{\}(}\AgdaBound{A} \AgdaSymbol{:} \AgdaRecord{Ty} \AgdaBound{Γ}\AgdaSymbol{)} \<[29]%
\>[29]\<%
\\
\>[-6]\AgdaIndent{7}{}\<[7]%
\>[7]\AgdaSymbol{→} \AgdaRecord{Tm} \AgdaSymbol{\{}\AgdaBound{Γ}\AgdaSymbol{\}} \AgdaSymbol{(}\AgdaFunction{Π} \AgdaBound{A} \AgdaSymbol{(}\AgdaFunction{⟦Id⟧} \AgdaBound{A} \<[29]%
\>[29]\<%
\\
\>[0]\AgdaIndent{10}{}\<[10]%
\>[10]\AgdaFunction{[} \AgdaFunction{cm-refl} \AgdaBound{A} \AgdaFunction{]T}\AgdaSymbol{)} \AgdaSymbol{)}\<%
\\
\>\AgdaFunction{⟦refl⟧} \AgdaSymbol{\{}\AgdaBound{Γ}\AgdaSymbol{\}} \AgdaBound{A} \AgdaSymbol{=} \<[16]%
\>[16]\AgdaFunction{lam} \AgdaSymbol{\{}\AgdaBound{Γ}\AgdaSymbol{\}} \AgdaSymbol{\{}\AgdaBound{A}\AgdaSymbol{\}} \AgdaSymbol{(}\AgdaFunction{⟦refl⟧⁰} \AgdaBound{A}\AgdaSymbol{)}\<%
\\
%
\\
%
\\
\>\<\end{code}

We have an abstracted $refl$ term as well. Using $\Pi$-types we could define the eliminator for $Id$, but it is more involved.

We have done the basics for category of families of setoids. There are more types can be interpreted in this model so that we could show that it is a valid model for Type Theory. We would like to interpret quotient types in this model by following Hofmann's method in \cite{hof:95:sm} or by ourselves.


\AgdaHide{
\begin{code}\>\<%
\\
\>\AgdaComment{\{-

-- substIn (B : Ty (Γ \& A))

⟦subst⟧⁰ : \{Γ : Con\}(A : Ty Γ)(B : Ty (Γ \& A)) → 
           Tm \{Γ \& A \& (A [ fst\& A ]T) 
           \& (⟦Id⟧ A) \& B [ fst\& (A [ fst\& A ]T) ]T [ fst\& (⟦Id⟧ A) ]T\} 
         (B [ record \{ fn = λ x → (proj₁ (proj₁ (proj₁ (proj₁ x)))) , (proj₂ (proj₁ (proj₁ x))) ; resp = λ x → proj₁ (proj₁ (proj₁ (proj₁ x))) , proj₂ (proj₁ (proj₁ x)) \} ]T)

⟦subst⟧⁰ \{Γ\} A B = record
       \{ tm = λ \{((((x , a) , b) , p) , PA) → [ B ]subst ([ Γ ]refl , [ [ A ]fm \_ ]trans ([ A ]refl* \_ \_) p) PA \}
       ; respt = λ \{((((m , a) , b) , p) , PA) → 
         [ [ B ]fm \_ ]trans 
         ([ B ]trans* \_ \_ \_) 
          ([ [ B ]fm \_ ]trans 
         [ B ]subst-pi 
         ([ [ B ]fm \_ ]trans 
         ([ [ B ]fm \_ ]sym ([ B ]trans* \_ \_ \_))
         ([ B ]subst* \_ PA) )) \}
       \}


-\}}\<%
\\
%
\\
\>\AgdaComment{-- The mechanism used in Martin Hofmann's Paper}\<%
\\
%
\\
\>\AgdaKeyword{record} \AgdaRecord{Prop-Uni} \AgdaSymbol{(}\AgdaBound{Γ} \AgdaSymbol{:} \AgdaFunction{Con}\AgdaSymbol{)} \AgdaSymbol{:} \AgdaPrimitiveType{Set} \AgdaKeyword{where}\<%
\\
\>[0]\AgdaIndent{2}{}\<[2]%
\>[2]\AgdaKeyword{field}\<%
\\
%
\\
\>[0]\AgdaIndent{4}{}\<[4]%
\>[4]\AgdaField{prf} \AgdaSymbol{:} \AgdaRecord{Ty} \AgdaBound{Γ}\<%
\\
\>[0]\AgdaIndent{4}{}\<[4]%
\>[4]\AgdaField{uni} \AgdaSymbol{:} \AgdaSymbol{∀} \AgdaBound{γ} \AgdaBound{x} \AgdaBound{y} \AgdaSymbol{→} \AgdaFunction{[} \AgdaFunction{[} \AgdaBound{prf} \AgdaFunction{]fm} \AgdaBound{γ} \AgdaFunction{]} \AgdaBound{x} \AgdaFunction{≈h} \AgdaBound{y} \AgdaDatatype{≡} \AgdaFunction{⊤'}\<%
\\
\>\AgdaKeyword{open} \AgdaModule{Prop-Uni}\<%
\\
%
\\
\>\AgdaComment{-- Is it correct to write  Tm A → Tm B for dependent types?}\<%
\\
%
\\
%
\\
%
\\
\>\AgdaFunction{Id-is-prop} \AgdaSymbol{:} \AgdaSymbol{\{}\AgdaBound{Γ} \AgdaSymbol{:} \AgdaFunction{Con}\AgdaSymbol{\}(}\AgdaBound{A} \AgdaSymbol{:} \AgdaRecord{Ty} \AgdaBound{Γ}\AgdaSymbol{)} \AgdaSymbol{→} \AgdaRecord{Prop-Uni} \AgdaSymbol{(}\AgdaBound{Γ} \AgdaFunction{\&} \AgdaBound{A} \AgdaFunction{\&} \AgdaSymbol{(}\AgdaBound{A} \AgdaFunction{[} \AgdaFunction{fst\&} \AgdaBound{A} \AgdaFunction{]T}\AgdaSymbol{))}\<%
\\
\>\AgdaFunction{Id-is-prop} \AgdaBound{A} \AgdaSymbol{=} \AgdaKeyword{record} \AgdaSymbol{\{} \AgdaField{prf} \AgdaSymbol{=} \AgdaFunction{⟦Id⟧} \AgdaBound{A} \AgdaSymbol{;} \AgdaField{uni} \AgdaSymbol{=} \AgdaSymbol{λ} \AgdaBound{γ} \AgdaBound{x} \AgdaBound{y} \AgdaSymbol{→} \AgdaInductiveConstructor{PE.refl} \AgdaSymbol{\}}\<%
\\
%
\\
\>\AgdaComment{\{-
record Quo \{Γ : Con\}(A : Ty Γ)(R : Prop-Uni (Γ \& A \& (A [ fst\& \{Γ\} \{A\} ]T))) : Set where
  field
    Q : Ty Γ
    [\_] : Tm A → Tm Q
    Q-elim : ∀ (B : Ty Γ)
                 (M : Tm \{Γ \& A\} (B [ fst\& \{Γ\} \{A\} ]T))
                 (N : Tm Q)
                 (H : Tm \{Γ \& A \& A [ fst\& \{Γ\} \{A\} ]T \& prf R\} (prf (Id-is-prop B) [ fst\& \{Γ \& A \& A [ fst\& \{Γ\} \{A\} ]T\} \{prf R\} ]T)) -- (prf (Id-is-prop (B [ fst\& \{Γ\} \{A\} ]T)))
               → Tm B

-\}}\<%
\\
%
\\
%
\\
%
\\
\>\<\end{code}
}

%\AgdaHide{
\begin{code}\>\<%
\\
%
\\
\>\AgdaKeyword{open} \AgdaKeyword{import} \AgdaModule{Level}\<%
\\
\>\AgdaKeyword{open} \AgdaKeyword{import} \AgdaModule{Relation.Binary.PropositionalEquality}\<%
\\
%
\\
\>\AgdaKeyword{module} \AgdaModule{HProp} \AgdaSymbol{(}\AgdaBound{ext} \AgdaSymbol{:} \AgdaFunction{Extensionality} \AgdaPrimitive{zero} \AgdaPrimitive{zero}\AgdaSymbol{)} \AgdaKeyword{where}\<%
\\
%
\\
\>\AgdaKeyword{open} \AgdaKeyword{import} \AgdaModule{Relation.Nullary}\<%
\\
\>\AgdaKeyword{open} \AgdaKeyword{import} \AgdaModule{Data.Unit}\<%
\\
\>\AgdaKeyword{open} \AgdaKeyword{import} \AgdaModule{Data.Empty}\<%
\\
\>\AgdaKeyword{open} \AgdaKeyword{import} \AgdaModule{Data.Nat}\<%
\\
\>\AgdaKeyword{open} \AgdaKeyword{import} \AgdaModule{Function}\<%
\\
%
\\
\>\<\end{code}
}

\section{Metatheory}

\HProp used as propositional universe

\begin{code}\>\<%
\\
%
\\
\>\AgdaKeyword{record} \AgdaRecord{HProp} \AgdaSymbol{:} \AgdaPrimitiveType{Set₁} \AgdaKeyword{where}\<%
\\
\>[0]\AgdaIndent{2}{}\<[2]%
\>[2]\AgdaKeyword{constructor} \AgdaInductiveConstructor{hProp}\<%
\\
\>[0]\AgdaIndent{2}{}\<[2]%
\>[2]\AgdaKeyword{field}\<%
\\
\>[2]\AgdaIndent{4}{}\<[4]%
\>[4]\AgdaField{prf} \AgdaSymbol{:} \AgdaPrimitiveType{Set}\<%
\\
\>[2]\AgdaIndent{4}{}\<[4]%
\>[4]\AgdaField{Uni} \AgdaSymbol{:} \AgdaSymbol{\{}\AgdaBound{p} \AgdaBound{q} \AgdaSymbol{:} \AgdaBound{prf}\AgdaSymbol{\}} \AgdaSymbol{→} \AgdaBound{p} \AgdaDatatype{≡} \AgdaBound{q}\<%
\\
%
\\
\>\AgdaKeyword{open} \AgdaModule{HProp} \AgdaKeyword{public} \AgdaKeyword{renaming} \AgdaSymbol{(}prf \AgdaSymbol{to} <\_>\AgdaSymbol{)}\<%
\\
%
\\
\>\<\end{code}

$\top$ and $\bot$

\begin{code}\>\<%
\\
%
\\
\>\AgdaFunction{⊤'} \AgdaSymbol{:} \AgdaRecord{HProp}\<%
\\
\>\AgdaFunction{⊤'} \AgdaSymbol{=} \AgdaInductiveConstructor{hProp} \AgdaRecord{⊤} \AgdaInductiveConstructor{refl}\<%
\\
%
\\
\>\AgdaFunction{exUni} \AgdaSymbol{:} \AgdaSymbol{\{}\AgdaBound{A} \AgdaSymbol{:} \AgdaPrimitiveType{Set}\AgdaSymbol{\}} \AgdaSymbol{→} \AgdaSymbol{(∀} \AgdaSymbol{(}\AgdaBound{p} \AgdaBound{q} \AgdaSymbol{:} \AgdaBound{A}\AgdaSymbol{)} \AgdaSymbol{→} \AgdaBound{p} \AgdaDatatype{≡} \AgdaBound{q}\AgdaSymbol{)} \<[42]%
\>[42]\<%
\\
\>[4]\AgdaIndent{6}{}\<[6]%
\>[6]\AgdaSymbol{→} \AgdaSymbol{(∀} \AgdaSymbol{\{}\AgdaBound{p} \AgdaBound{q} \AgdaSymbol{:} \AgdaBound{A}\AgdaSymbol{\}} \AgdaSymbol{→} \AgdaBound{p} \AgdaDatatype{≡} \AgdaBound{q}\AgdaSymbol{)}\<%
\\
\>\AgdaFunction{exUni} \AgdaBound{f} \AgdaSymbol{\{}\AgdaBound{p}\AgdaSymbol{\}} \AgdaSymbol{\{}\AgdaBound{q}\AgdaSymbol{\}} \AgdaSymbol{=} \AgdaBound{f} \AgdaBound{p} \AgdaBound{q}\<%
\\
%
\\
\>\AgdaFunction{⊥'} \AgdaSymbol{:} \AgdaRecord{HProp}\<%
\\
\>\AgdaFunction{⊥'} \AgdaSymbol{=} \AgdaInductiveConstructor{hProp} \AgdaDatatype{⊥} \AgdaSymbol{(}\AgdaFunction{exUni} \AgdaSymbol{(λ} \AgdaSymbol{()))}\<%
\\
%
\\
\>\<\end{code}

\HProp is closed under $\Pi$-types

\begin{code}\>\<%
\\
%
\\
\>\AgdaFunction{∀'} \AgdaSymbol{:} \AgdaSymbol{(}\AgdaBound{A} \AgdaSymbol{:} \AgdaPrimitiveType{Set}\AgdaSymbol{)(}\AgdaBound{P} \AgdaSymbol{:} \AgdaBound{A} \AgdaSymbol{→} \AgdaRecord{HProp}\AgdaSymbol{)} \AgdaSymbol{→} \AgdaRecord{HProp}\<%
\\
\>\AgdaFunction{∀'} \AgdaBound{A} \AgdaBound{P} \AgdaSymbol{=} \AgdaInductiveConstructor{hProp} \AgdaSymbol{((}\AgdaBound{x} \AgdaSymbol{:} \AgdaBound{A}\AgdaSymbol{)} \AgdaSymbol{→} \AgdaFunction{<} \AgdaBound{P} \AgdaBound{x} \AgdaFunction{>}\AgdaSymbol{)} \AgdaSymbol{(}\AgdaBound{ext} \AgdaSymbol{(λ} \AgdaBound{x} \AgdaSymbol{→} \AgdaFunction{Uni} \AgdaSymbol{(}\AgdaBound{P} \AgdaBound{x}\AgdaSymbol{)))}\<%
\\
%
\\
\>\AgdaKeyword{syntax} ∀' A \AgdaSymbol{(λ} x \AgdaSymbol{→} B\AgdaSymbol{)} \AgdaSymbol{=} ∀'[ x ∶ A ] B

\AgdaKeyword{infixr} \AgdaNumber{2} \_⇛\_\<%
\\
%
\\
\>\AgdaFunction{\_⇛\_} \AgdaSymbol{:} \AgdaSymbol{(}\AgdaBound{P} \AgdaBound{Q} \AgdaSymbol{:} \AgdaRecord{HProp}\AgdaSymbol{)} \AgdaSymbol{→} \AgdaRecord{HProp}\<%
\\
\>\AgdaBound{P} \AgdaFunction{⇛} \AgdaBound{Q} \AgdaSymbol{=} \<[9]%
\>[9]\AgdaFunction{∀'} \AgdaFunction{<} \AgdaBound{P} \AgdaFunction{>} \AgdaSymbol{(λ} \AgdaBound{\_} \AgdaSymbol{→} \AgdaBound{Q}\AgdaSymbol{)}\<%
\\
%
\\
\>\<\end{code}

\HProp is closed under $\Sigma$-types

\begin{code}\>\<%
\\
%
\\
\>\AgdaKeyword{open} \AgdaKeyword{import} \AgdaModule{Data.Product}\<%
\\
%
\\
\>\AgdaFunction{sig-eq} \AgdaSymbol{:} \AgdaSymbol{\{}\AgdaBound{A} \AgdaSymbol{:} \AgdaPrimitiveType{Set}\AgdaSymbol{\}\{}\AgdaBound{B} \AgdaSymbol{:} \AgdaBound{A} \AgdaSymbol{→} \AgdaPrimitiveType{Set}\AgdaSymbol{\}}\<%
\\
\>[6]\AgdaIndent{9}{}\<[9]%
\>[9]\AgdaSymbol{\{}\AgdaBound{a} \AgdaBound{b} \AgdaSymbol{:} \AgdaBound{A}\AgdaSymbol{\}(}\AgdaBound{p} \AgdaSymbol{:} \AgdaBound{a} \AgdaDatatype{≡} \AgdaBound{b}\AgdaSymbol{)}\<%
\\
\>[6]\AgdaIndent{9}{}\<[9]%
\>[9]\AgdaSymbol{\{}\AgdaBound{c} \AgdaSymbol{:} \AgdaBound{B} \AgdaBound{a}\AgdaSymbol{\}\{}\AgdaBound{d} \AgdaSymbol{:} \AgdaBound{B} \AgdaBound{b}\AgdaSymbol{\}} \AgdaSymbol{→}\<%
\\
\>[6]\AgdaIndent{9}{}\<[9]%
\>[9]\AgdaSymbol{(}\AgdaFunction{subst} \AgdaSymbol{(λ} \AgdaBound{x} \AgdaSymbol{→} \AgdaBound{B} \AgdaBound{x}\AgdaSymbol{)} \AgdaBound{p} \AgdaBound{c} \AgdaDatatype{≡} \AgdaBound{d}\AgdaSymbol{)} \AgdaSymbol{→}\<%
\\
\>[6]\AgdaIndent{9}{}\<[9]%
\>[9]\AgdaDatatype{\_≡\_} \AgdaSymbol{\{\_\}} \AgdaSymbol{\{}\AgdaRecord{Σ} \AgdaBound{A} \AgdaBound{B}\AgdaSymbol{\}} \AgdaSymbol{(}\AgdaBound{a} \AgdaInductiveConstructor{,} \AgdaBound{c}\AgdaSymbol{)} \AgdaSymbol{(}\AgdaBound{b} \AgdaInductiveConstructor{,} \AgdaBound{d}\AgdaSymbol{)}\<%
\\
\>\AgdaFunction{sig-eq} \AgdaInductiveConstructor{refl} \AgdaInductiveConstructor{refl} \AgdaSymbol{=} \AgdaInductiveConstructor{refl}\<%
\\
%
\\
\>\AgdaFunction{Σ'} \AgdaSymbol{:} \AgdaSymbol{(}\AgdaBound{P} \AgdaSymbol{:} \AgdaRecord{HProp}\AgdaSymbol{)(}\AgdaBound{Q} \AgdaSymbol{:} \AgdaFunction{<} \AgdaBound{P} \AgdaFunction{>} \AgdaSymbol{→} \AgdaRecord{HProp}\AgdaSymbol{)} \AgdaSymbol{→} \AgdaRecord{HProp}\<%
\\
\>\AgdaFunction{Σ'} \AgdaBound{P} \AgdaBound{Q} \AgdaSymbol{=} \AgdaInductiveConstructor{hProp} \AgdaSymbol{(}\AgdaRecord{Σ} \AgdaFunction{<} \AgdaBound{P} \AgdaFunction{>} \AgdaSymbol{(λ} \AgdaBound{x} \AgdaSymbol{→} \AgdaFunction{<} \AgdaBound{Q} \AgdaBound{x} \AgdaFunction{>}\AgdaSymbol{))} \<[41]%
\>[41]\<%
\\
\>[6]\AgdaIndent{9}{}\<[9]%
\>[9]\AgdaSymbol{(λ} \AgdaSymbol{\{}\AgdaBound{p}\AgdaSymbol{\}} \AgdaSymbol{\{}\AgdaBound{q}\AgdaSymbol{\}} \AgdaSymbol{→} \AgdaFunction{sig-eq} \AgdaSymbol{(}\AgdaFunction{Uni} \AgdaBound{P}\AgdaSymbol{)} \AgdaSymbol{(}\AgdaFunction{Uni} \AgdaSymbol{(}\AgdaBound{Q} \AgdaSymbol{(}\AgdaFunction{proj₁} \AgdaBound{q}\AgdaSymbol{))))}\<%
\\
%
\\
\>\AgdaKeyword{syntax} Σ' A \AgdaSymbol{(λ} x \AgdaSymbol{→} B\AgdaSymbol{)} \AgdaSymbol{=} Σ'[ x ∶ A ] B

\AgdaKeyword{infixr} \AgdaNumber{3} \_∧\_\<%
\\
%
\\
\>\AgdaFunction{\_∧\_} \AgdaSymbol{:} \AgdaSymbol{(}\AgdaBound{P} \AgdaBound{Q} \AgdaSymbol{:} \AgdaRecord{HProp}\AgdaSymbol{)} \AgdaSymbol{→} \AgdaRecord{HProp}\<%
\\
\>\AgdaBound{P} \AgdaFunction{∧} \AgdaBound{Q} \AgdaSymbol{=} \AgdaFunction{Σ'} \AgdaBound{P} \AgdaSymbol{(λ} \AgdaBound{\_} \AgdaSymbol{→} \AgdaBound{Q}\AgdaSymbol{)}\<%
\\
%
\\
\>\<\end{code}

Negation

\begin{code}\>\<%
\\
%
\\
\>\AgdaFunction{\textasciitilde} \AgdaSymbol{:} \AgdaRecord{HProp} \AgdaSymbol{→} \AgdaRecord{HProp}\<%
\\
\>\AgdaFunction{\textasciitilde} \AgdaBound{P} \AgdaSymbol{=} \AgdaBound{P} \AgdaFunction{⇛} \AgdaFunction{⊥'} \<[13]%
\>[13]\<%
\\
%
\\
\>\<\end{code}

Logical equivalence

\begin{code}\>\<%
\\
%
\\
\>\AgdaFunction{\_⇄\_} \<[6]%
\>[6]\AgdaSymbol{:} \AgdaSymbol{(}\AgdaBound{P} \AgdaBound{Q} \AgdaSymbol{:} \AgdaRecord{HProp}\AgdaSymbol{)} \AgdaSymbol{→} \AgdaRecord{HProp}\<%
\\
\>\AgdaBound{P} \AgdaFunction{⇄} \AgdaBound{Q} \AgdaSymbol{=} \AgdaSymbol{(}\AgdaBound{P} \AgdaFunction{⇛} \AgdaBound{Q}\AgdaSymbol{)} \AgdaFunction{∧} \AgdaSymbol{(}\AgdaBound{Q} \AgdaFunction{⇛} \AgdaBound{P}\AgdaSymbol{)}\<%
\\
%
\\
%
\\
%
\\
\>\<\end{code}

$\eta$-rules for $\Pi$-types and $\Sigma$-types

\begin{code}\>\<%
\\
%
\\
\>\AgdaFunction{Π-eta} \AgdaSymbol{:} \AgdaSymbol{\{}\AgdaBound{A} \AgdaSymbol{:} \AgdaPrimitiveType{Set}\AgdaSymbol{\}\{}\AgdaBound{B} \AgdaSymbol{:} \AgdaBound{A} \AgdaSymbol{→} \AgdaPrimitiveType{Set}\AgdaSymbol{\}(}\AgdaBound{f} \AgdaSymbol{:} \AgdaSymbol{(}\AgdaBound{a} \AgdaSymbol{:} \AgdaBound{A}\AgdaSymbol{)} \AgdaSymbol{→} \AgdaBound{B} \AgdaBound{a}\AgdaSymbol{)} \AgdaSymbol{→} \<[52]%
\>[52]\<%
\\
\>[-5]\AgdaIndent{8}{}\<[8]%
\>[8]\AgdaSymbol{(λ} \AgdaBound{x} \AgdaSymbol{→} \AgdaBound{f} \AgdaBound{x}\AgdaSymbol{)} \AgdaDatatype{≡} \AgdaBound{f}\<%
\\
\>\AgdaFunction{Π-eta} \AgdaBound{f} \AgdaSymbol{=} \AgdaInductiveConstructor{refl}\<%
\\
%
\\
\>\AgdaFunction{Σ-eta} \AgdaSymbol{:} \AgdaSymbol{\{}\AgdaBound{A} \AgdaSymbol{:} \AgdaPrimitiveType{Set}\AgdaSymbol{\}\{}\AgdaBound{B} \AgdaSymbol{:} \AgdaBound{A} \AgdaSymbol{→} \AgdaPrimitiveType{Set}\AgdaSymbol{\}(}\AgdaBound{p} \AgdaSymbol{:} \AgdaRecord{Σ} \AgdaBound{A} \AgdaBound{B}\AgdaSymbol{)} \AgdaSymbol{→} \<[44]%
\>[44]\<%
\\
\>[0]\AgdaIndent{8}{}\<[8]%
\>[8]\AgdaSymbol{(}\AgdaFunction{proj₁} \AgdaBound{p} \AgdaInductiveConstructor{,} \AgdaFunction{proj₂} \AgdaBound{p}\AgdaSymbol{)} \AgdaDatatype{≡} \AgdaBound{p}\<%
\\
\>\AgdaFunction{Σ-eta} \AgdaBound{p} \AgdaSymbol{=} \AgdaInductiveConstructor{refl}\<%
\\
%
\\
\>\<\end{code}

%
\AgdaHide{

\begin{code}\>\<%
\\
%
\\
%
\\
\>\AgdaSymbol{\{-\#} \AgdaKeyword{OPTIONS} --type-in-type \AgdaSymbol{\#-\}}\<%
\\
%
\\
\>\AgdaKeyword{open} \AgdaKeyword{import} \AgdaModule{Level}\<%
\\
\>\AgdaKeyword{open} \AgdaKeyword{import} \AgdaModule{Relation.Binary.PropositionalEquality} \AgdaSymbol{as} \AgdaModule{PE} \AgdaKeyword{hiding} \AgdaSymbol{(}refl\AgdaSymbol{;} sym \AgdaSymbol{;} trans\AgdaSymbol{;} isEquivalence\AgdaSymbol{)}\<%
\\
%
\\
\>\AgdaKeyword{module} \AgdaModule{CategoryOfSetoid} \<[25]%
\>[25]\AgdaSymbol{(}\AgdaBound{ext} \AgdaSymbol{:} \AgdaFunction{Extensionality} \AgdaPrimitive{zero} \AgdaPrimitive{zero}\AgdaSymbol{)} \AgdaKeyword{where}\<%
\\
%
\\
\>\AgdaKeyword{open} \AgdaKeyword{import} \AgdaModule{Cats.Category}\<%
\\
\>\AgdaKeyword{open} \AgdaKeyword{import} \AgdaModule{Function}\<%
\\
\>\AgdaKeyword{open} \AgdaKeyword{import} \AgdaModule{Relation.Binary.Core} \AgdaKeyword{using} \AgdaSymbol{(}\_≡\_\AgdaSymbol{)} \AgdaKeyword{renaming} \AgdaSymbol{(}\_⇒\_ \AgdaSymbol{to} \_⇒'\_\AgdaSymbol{)}\<%
\\
\>\AgdaKeyword{open} \AgdaKeyword{import} \AgdaModule{Data.Unit}\<%
\\
\>\AgdaKeyword{open} \AgdaKeyword{import} \AgdaModule{Data.Empty}\<%
\\
\>\AgdaKeyword{import} \AgdaModule{hProp}\<%
\\
\>\AgdaKeyword{open} \AgdaKeyword{module} \AgdaModule{hpx} \AgdaSymbol{=} \AgdaModule{hProp} \AgdaBound{ext}\<%
\\
%
\\
%
\\
\>\AgdaComment{-- Arrow between HSetoid}\<%
\\
%
\\
\>\AgdaKeyword{infix} \AgdaNumber{5} \_⇉\_\<%
\\
%
\\
\>\AgdaComment{-- composition}\<%
\\
%
\\
\>\AgdaKeyword{infixl} \AgdaNumber{5} \_∘c\_\<%
\\
%
\\
\>\<\end{code}
}

Then we could define setoids using \textbf{hProp}. An equivalence relation has three properties reflexivity, symmetry and transitivity. Since we have $refl$ here, we call the reflexivity for propositional equality from the library with prefix as $PE.refl$. 

\begin{code}\>\<%
\\
%
\\
\>\AgdaKeyword{record} \AgdaRecord{ishEquivalence} \AgdaSymbol{\{}\AgdaBound{A} \AgdaSymbol{:} \AgdaPrimitiveType{Set}\AgdaSymbol{\}(}\AgdaBound{\_≈h\_} \AgdaSymbol{:} \AgdaBound{A} \AgdaSymbol{→} \AgdaBound{A} \AgdaSymbol{→} \AgdaRecord{hProp}\AgdaSymbol{)} \AgdaSymbol{:} \AgdaPrimitiveType{Set₁} \AgdaKeyword{where}\<%
\\
\>[0]\AgdaIndent{2}{}\<[2]%
\>[2]\AgdaKeyword{constructor} \AgdaInductiveConstructor{\_,\_,\_}\<%
\\
\>[0]\AgdaIndent{2}{}\<[2]%
\>[2]\AgdaKeyword{field}\<%
\\
\>[2]\AgdaIndent{4}{}\<[4]%
\>[4]\AgdaField{refl} \<[12]%
\>[12]\AgdaSymbol{:} \AgdaSymbol{\{}\AgdaBound{x} \AgdaSymbol{:} \AgdaBound{A}\AgdaSymbol{\}} \AgdaSymbol{→} \AgdaFunction{<} \AgdaBound{x} \AgdaBound{≈h} \AgdaBound{x} \AgdaFunction{>}\<%
\\
\>[2]\AgdaIndent{4}{}\<[4]%
\>[4]\AgdaField{sym} \<[12]%
\>[12]\AgdaSymbol{:} \AgdaSymbol{\{}\AgdaBound{x} \AgdaBound{y} \AgdaSymbol{:} \AgdaBound{A}\AgdaSymbol{\}} \AgdaSymbol{→} \AgdaFunction{<} \AgdaBound{x} \AgdaBound{≈h} \AgdaBound{y} \AgdaFunction{>} \AgdaSymbol{→} \AgdaFunction{<} \AgdaBound{y} \AgdaBound{≈h} \AgdaBound{x} \AgdaFunction{>}\<%
\\
\>[2]\AgdaIndent{4}{}\<[4]%
\>[4]\AgdaField{trans} \<[12]%
\>[12]\AgdaSymbol{:} \AgdaSymbol{\{}\AgdaBound{x} \AgdaBound{y} \AgdaBound{z} \AgdaSymbol{:} \AgdaBound{A}\AgdaSymbol{\}} \AgdaSymbol{→} \AgdaFunction{<} \AgdaBound{x} \AgdaBound{≈h} \AgdaBound{y} \AgdaFunction{>} \AgdaSymbol{→} \AgdaFunction{<} \AgdaBound{y} \AgdaBound{≈h} \AgdaBound{z} \AgdaFunction{>} \AgdaSymbol{→} \AgdaFunction{<} \AgdaBound{x} \AgdaBound{≈h} \AgdaBound{z} \AgdaFunction{>}\<%
\\
%
\\
\>\<\end{code}

Here we use \textbf{hSetoid} as the name because \textbf{Setoid} is
already used for non-proof-irrelvant setoids in the library.
For each setoid, we have a carrier type and an equivalence relation.

\begin{code}\>\<%
\\
\>\AgdaKeyword{record} \AgdaRecord{hSetoid} \AgdaSymbol{:} \AgdaPrimitiveType{Set₁} \AgdaKeyword{where}\<%
\\
\>[0]\AgdaIndent{2}{}\<[2]%
\>[2]\AgdaKeyword{constructor} \AgdaInductiveConstructor{\_,\_,\_}\<%
\\
\>[0]\AgdaIndent{2}{}\<[2]%
\>[2]\AgdaKeyword{infix} \AgdaNumber{4} \_≈h\_ \_≈\_\<%
\\
\>[0]\AgdaIndent{2}{}\<[2]%
\>[2]\AgdaKeyword{field}\<%
\\
\>[2]\AgdaIndent{4}{}\<[4]%
\>[4]\AgdaField{Carrier} \AgdaSymbol{:} \AgdaPrimitiveType{Set}\<%
\\
\>[2]\AgdaIndent{4}{}\<[4]%
\>[4]\AgdaField{\_≈h\_} \<[12]%
\>[12]\AgdaSymbol{:} \AgdaBound{Carrier} \AgdaSymbol{→} \AgdaBound{Carrier} \AgdaSymbol{→} \AgdaRecord{hProp}\<%
\\
\>[2]\AgdaIndent{4}{}\<[4]%
\>[4]\AgdaField{isEquiv} \AgdaSymbol{:} \AgdaRecord{ishEquivalence} \AgdaBound{\_≈h\_}\<%
\\
%
\\
%
\\
\>\<\end{code}

\AgdaHide{
\begin{code}\>\<%
\\
%
\\
\>[0]\AgdaIndent{2}{}\<[2]%
\>[2]\AgdaKeyword{open} \<[8]%
\>[8]\AgdaModule{ishEquivalence} \AgdaFunction{isEquiv} \AgdaKeyword{public}\<%
\\
%
\\
\>[0]\AgdaIndent{2}{}\<[2]%
\>[2]\AgdaFunction{\_≈\_} \AgdaSymbol{:} \AgdaFunction{Carrier} \AgdaSymbol{→} \AgdaFunction{Carrier} \AgdaSymbol{→} \AgdaPrimitiveType{Set}\<%
\\
\>[0]\AgdaIndent{2}{}\<[2]%
\>[2]\AgdaBound{a} \AgdaFunction{≈} \AgdaBound{b} \AgdaSymbol{=} \AgdaFunction{<} \AgdaBound{a} \AgdaFunction{≈h} \AgdaBound{b} \AgdaFunction{>}\<%
\\
\>[0]\AgdaIndent{1}{}\<[1]%
\>[1]\<%
\\
\>[0]\AgdaIndent{2}{}\<[2]%
\>[2]\AgdaFunction{PI} \AgdaSymbol{:} \AgdaSymbol{\{}\AgdaBound{x} \AgdaBound{y} \AgdaSymbol{:} \AgdaFunction{Carrier}\AgdaSymbol{\}\{}\AgdaBound{B} \AgdaSymbol{:} \AgdaPrimitiveType{Set}\AgdaSymbol{\}}\<%
\\
\>[2]\AgdaIndent{7}{}\<[7]%
\>[7]\AgdaSymbol{(}\AgdaBound{A} \AgdaSymbol{:} \AgdaBound{x} \AgdaFunction{≈} \AgdaBound{y} \AgdaSymbol{→} \AgdaBound{B}\AgdaSymbol{)\{}\AgdaBound{p} \AgdaBound{q} \AgdaSymbol{:} \AgdaBound{x} \AgdaFunction{≈} \AgdaBound{y}\AgdaSymbol{\}} \<[36]%
\>[36]\<%
\\
\>[0]\AgdaIndent{5}{}\<[5]%
\>[5]\AgdaSymbol{→} \AgdaBound{A} \AgdaBound{p} \AgdaDatatype{≡} \AgdaBound{A} \AgdaBound{q}\<%
\\
\>[0]\AgdaIndent{2}{}\<[2]%
\>[2]\AgdaFunction{PI} \AgdaSymbol{\{}\AgdaBound{x}\AgdaSymbol{\}} \AgdaSymbol{\{}\AgdaBound{y}\AgdaSymbol{\}} \AgdaBound{A} \AgdaSymbol{\{}\AgdaBound{p}\AgdaSymbol{\}} \AgdaSymbol{\{}\AgdaBound{q}\AgdaSymbol{\}} \AgdaKeyword{with} \AgdaFunction{Uni} \AgdaSymbol{(}\AgdaBound{x} \AgdaFunction{≈h} \AgdaBound{y}\AgdaSymbol{)} \AgdaSymbol{\{}\AgdaBound{p}\AgdaSymbol{\}} \AgdaSymbol{\{}\AgdaBound{q}\AgdaSymbol{\}}\<%
\\
\>[0]\AgdaIndent{2}{}\<[2]%
\>[2]\AgdaFunction{PI} \AgdaBound{A} \AgdaSymbol{|} \AgdaInductiveConstructor{PE.refl} \AgdaSymbol{=} \AgdaInductiveConstructor{PE.refl}\<%
\\
%
\\
\>[0]\AgdaIndent{2}{}\<[2]%
\>[2]\AgdaFunction{reflexive} \AgdaSymbol{:} \AgdaDatatype{\_≡\_} \AgdaFunction{⇒'} \AgdaFunction{\_≈\_}\<%
\\
\>[0]\AgdaIndent{2}{}\<[2]%
\>[2]\AgdaFunction{reflexive} \AgdaInductiveConstructor{PE.refl} \AgdaSymbol{=} \AgdaFunction{refl}\<%
\\
%
\\
\>\AgdaKeyword{open} \AgdaModule{hSetoid} \AgdaKeyword{public} \AgdaKeyword{renaming} \AgdaSymbol{(}refl \AgdaSymbol{to} [\_]refl\AgdaSymbol{;}
     sym \AgdaSymbol{to} [\_]sym\AgdaSymbol{;} \_≈\_ \AgdaSymbol{to} [\_]\_≈\_ \AgdaSymbol{;} \_≈h\_ \AgdaSymbol{to} [\_]\_≈h\_ \AgdaSymbol{;}
     Carrier \AgdaSymbol{to} ∣\_∣ \AgdaSymbol{;} trans \AgdaSymbol{to} [\_]trans\AgdaSymbol{)}\<%
\\
%
\\
%
\\
\>\AgdaFunction{[\_]uip} \AgdaSymbol{:} \AgdaSymbol{∀(}\AgdaBound{Γ} \AgdaSymbol{:} \AgdaRecord{hSetoid}\AgdaSymbol{)\{}\AgdaBound{a} \AgdaBound{b} \AgdaSymbol{:} \AgdaFunction{∣} \AgdaBound{Γ} \AgdaFunction{∣}\AgdaSymbol{\}\{}\AgdaBound{p} \AgdaBound{q} \AgdaSymbol{:} \AgdaFunction{[} \AgdaBound{Γ} \AgdaFunction{]} \AgdaBound{a} \AgdaFunction{≈} \AgdaBound{b}\AgdaSymbol{\}} \AgdaSymbol{→} \AgdaBound{p} \AgdaDatatype{≡} \AgdaBound{q}\<%
\\
\>\AgdaFunction{[} \AgdaBound{Γ} \AgdaFunction{]uip} \AgdaSymbol{\{}\AgdaBound{a}\AgdaSymbol{\}} \AgdaSymbol{\{}\AgdaBound{b}\AgdaSymbol{\}} \AgdaSymbol{=} \AgdaFunction{Uni} \AgdaSymbol{(}\AgdaFunction{[} \AgdaBound{Γ} \AgdaFunction{]} \AgdaBound{a} \AgdaFunction{≈h} \AgdaBound{b}\AgdaSymbol{)}\<%
\\
%
\\
%
\\
\>\<\end{code}
}

A morphism in this category is a function of the underlying sets which respects the equivalence relation. We don't identify the extensional equal functions in the homsets as in \textbf{E-setoids}.

\begin{code}\>\<%
\\
%
\\
\>\AgdaKeyword{record} \AgdaRecord{\_⇉\_} \AgdaSymbol{(}\AgdaBound{A} \AgdaBound{B} \AgdaSymbol{:} \AgdaRecord{hSetoid}\AgdaSymbol{)} \AgdaSymbol{:} \AgdaPrimitiveType{Set₁} \AgdaKeyword{where}\<%
\\
\>[0]\AgdaIndent{2}{}\<[2]%
\>[2]\AgdaKeyword{constructor} \AgdaInductiveConstructor{fn:\_resp:\_}\<%
\\
\>[0]\AgdaIndent{2}{}\<[2]%
\>[2]\AgdaKeyword{field}\<%
\\
\>[2]\AgdaIndent{4}{}\<[4]%
\>[4]\AgdaField{fn} \<[9]%
\>[9]\AgdaSymbol{:} \AgdaFunction{∣} \AgdaBound{A} \AgdaFunction{∣} \AgdaSymbol{→} \AgdaFunction{∣} \AgdaBound{B} \AgdaFunction{∣}\<%
\\
\>[2]\AgdaIndent{4}{}\<[4]%
\>[4]\AgdaField{resp} \AgdaSymbol{:} \AgdaSymbol{\{}\AgdaBound{x} \AgdaBound{y} \AgdaSymbol{:} \AgdaFunction{∣} \AgdaBound{A} \AgdaFunction{∣}\AgdaSymbol{\}} \AgdaSymbol{→} \<[27]%
\>[27]\<%
\\
\>[4]\AgdaIndent{11}{}\<[11]%
\>[11]\AgdaFunction{[} \AgdaBound{A} \AgdaFunction{]} \AgdaBound{x} \AgdaFunction{≈} \AgdaBound{y} \AgdaSymbol{→} \<[25]%
\>[25]\<%
\\
\>[4]\AgdaIndent{11}{}\<[11]%
\>[11]\AgdaFunction{[} \AgdaBound{B} \AgdaFunction{]} \AgdaBound{fn} \AgdaBound{x} \AgdaFunction{≈} \AgdaBound{fn} \AgdaBound{y}\<%
\\
%
\\
\>\<\end{code}

\AgdaHide{
\begin{code}\>\<%
\\
%
\\
\>\AgdaKeyword{open} \AgdaModule{\_⇉\_} \AgdaKeyword{public} \AgdaKeyword{renaming} \AgdaSymbol{(}fn \AgdaSymbol{to} [\_]fn \AgdaSymbol{;} resp \AgdaSymbol{to} [\_]resp\AgdaSymbol{)}\<%
\\
%
\\
\>\<\end{code}
}


The definitions of identity morphism and composition are straightforward and the categorical laws hold trivially as follows.

\begin{code}\>\<%
\\
%
\\
\>\AgdaFunction{id'} \AgdaSymbol{:} \AgdaSymbol{\{}\AgdaBound{Γ} \AgdaSymbol{:} \AgdaRecord{hSetoid}\AgdaSymbol{\}} \AgdaSymbol{→} \AgdaBound{Γ} \AgdaRecord{⇉} \AgdaBound{Γ} \<[28]%
\>[28]\<%
\\
\>\AgdaFunction{id'} \AgdaSymbol{=} \AgdaKeyword{record} \AgdaSymbol{\{} \AgdaField{fn} \AgdaSymbol{=} \AgdaFunction{id}\AgdaSymbol{;} \AgdaField{resp} \AgdaSymbol{=} \AgdaFunction{id}\AgdaSymbol{\}}\<%
\\
%
\\
\>\AgdaFunction{\_∘c\_} \AgdaSymbol{:} \AgdaSymbol{∀\{}\AgdaBound{Γ} \AgdaBound{Δ} \AgdaBound{Z}\AgdaSymbol{\}} \AgdaSymbol{→} \AgdaBound{Δ} \AgdaRecord{⇉} \AgdaBound{Z} \AgdaSymbol{→} \AgdaBound{Γ} \AgdaRecord{⇉} \AgdaBound{Δ} \AgdaSymbol{→} \AgdaBound{Γ} \AgdaRecord{⇉} \AgdaBound{Z}\<%
\\
\>\AgdaBound{yz} \AgdaFunction{∘c} \AgdaBound{xy} \AgdaSymbol{=} \AgdaKeyword{record} \<[18]%
\>[18]\<%
\\
\>[4]\AgdaIndent{11}{}\<[11]%
\>[11]\AgdaSymbol{\{} \AgdaField{fn} \AgdaSymbol{=} \AgdaFunction{[} \AgdaBound{yz} \AgdaFunction{]fn} \AgdaFunction{∘} \AgdaFunction{[} \AgdaBound{xy} \AgdaFunction{]fn}\<%
\\
\>[4]\AgdaIndent{11}{}\<[11]%
\>[11]\AgdaSymbol{;} \AgdaField{resp} \AgdaSymbol{=} \AgdaFunction{[} \AgdaBound{yz} \AgdaFunction{]resp} \AgdaFunction{∘} \AgdaFunction{[} \AgdaBound{xy} \AgdaFunction{]resp}\<%
\\
\>[4]\AgdaIndent{11}{}\<[11]%
\>[11]\AgdaSymbol{\}}\<%
\\
%
\\
\>\AgdaFunction{id₁} \AgdaSymbol{:} \AgdaSymbol{∀} \AgdaBound{Γ} \AgdaBound{Δ} \AgdaSymbol{(}\AgdaBound{ch} \AgdaSymbol{:} \AgdaBound{Γ} \AgdaRecord{⇉} \AgdaBound{Δ}\AgdaSymbol{)} \AgdaSymbol{→} \AgdaBound{ch} \AgdaFunction{∘c} \AgdaFunction{id'} \AgdaDatatype{≡} \AgdaBound{ch}\<%
\\
\>\AgdaFunction{id₁} \AgdaSymbol{\_} \AgdaSymbol{\_} \AgdaBound{ch} \AgdaSymbol{=} \AgdaInductiveConstructor{PE.refl}\<%
\\
%
\\
\>\AgdaFunction{id₂} \AgdaSymbol{:} \AgdaSymbol{∀} \AgdaBound{Γ} \AgdaBound{Δ} \AgdaSymbol{(}\AgdaBound{ch} \AgdaSymbol{:} \AgdaBound{Γ} \AgdaRecord{⇉} \AgdaBound{Δ}\AgdaSymbol{)} \AgdaSymbol{→} \AgdaFunction{id'} \AgdaFunction{∘c} \AgdaBound{ch} \AgdaDatatype{≡} \AgdaBound{ch}\<%
\\
\>\AgdaFunction{id₂} \AgdaSymbol{\_} \AgdaSymbol{\_} \AgdaBound{ch} \AgdaSymbol{=} \AgdaInductiveConstructor{PE.refl}\<%
\\
%
\\
\>\AgdaFunction{comp} \AgdaSymbol{:} \AgdaSymbol{∀} \AgdaBound{Γ} \AgdaSymbol{\{}\AgdaBound{Δ} \AgdaBound{Φ}\AgdaSymbol{\}} \AgdaBound{Ψ} \<[19]%
\>[19]\<%
\\
\>[-2]\AgdaIndent{9}{}\<[9]%
\>[9]\AgdaSymbol{(}\AgdaBound{f} \AgdaSymbol{:} \AgdaBound{Γ} \AgdaRecord{⇉} \AgdaBound{Δ}\AgdaSymbol{)}\<%
\\
\>[0]\AgdaIndent{9}{}\<[9]%
\>[9]\AgdaSymbol{(}\AgdaBound{g} \AgdaSymbol{:} \AgdaBound{Δ} \AgdaRecord{⇉} \AgdaBound{Φ}\AgdaSymbol{)}\<%
\\
\>[0]\AgdaIndent{9}{}\<[9]%
\>[9]\AgdaSymbol{(}\AgdaBound{h} \AgdaSymbol{:} \AgdaBound{Φ} \AgdaRecord{⇉} \AgdaBound{Ψ}\AgdaSymbol{)}\<%
\\
\>[0]\AgdaIndent{7}{}\<[7]%
\>[7]\AgdaSymbol{→} \AgdaBound{h} \AgdaFunction{∘c} \AgdaBound{g} \AgdaFunction{∘c} \AgdaBound{f} \AgdaDatatype{≡} \AgdaBound{h} \AgdaFunction{∘c} \AgdaSymbol{(}\AgdaBound{g} \AgdaFunction{∘c} \AgdaBound{f}\AgdaSymbol{)}\<%
\\
\>\AgdaFunction{comp} \AgdaSymbol{\_} \AgdaSymbol{\_} \AgdaBound{f} \AgdaBound{g} \AgdaBound{h} \AgdaSymbol{=} \AgdaInductiveConstructor{PE.refl}\<%
\\
%
\\
\>\<\end{code}

\AgdaHide{
\begin{code}\>\<%
\\
\>\AgdaComment{\{-
\_f≈\_ :  ∀\{Γ Δ : hSetoid\} → (f g : Γ ⇉ Δ) → hProp
\_f≈\_ \{Γ , \_≈h\_ , (refl , sym , trans)\} \{Δ , \_≈h₁\_ , (refl₁ , sym₁ , trans₁)\} (fn: fn resp: fresp) (fn: gn resp: gresp) 
  = record 
           \{ prf = (g : Γ) → < fn g ≈h₁ gn g >
           ; Uni = ext (λ g → Uni (fn g ≈h₁ gn g))
           \}
-\}}\<%
\\
%
\\
%
\\
\>\<\end{code}
}

Combined all components we obtain the category of setoids.

\begin{code}\>\<%
\\
%
\\
\>\AgdaFunction{setoid-Cat} \AgdaSymbol{:} \AgdaRecord{Category}\<%
\\
\>\AgdaFunction{setoid-Cat} \AgdaSymbol{=} \AgdaInductiveConstructor{CatC} \AgdaRecord{hSetoid} \AgdaRecord{\_⇉\_} \AgdaSymbol{(λ} \AgdaBound{\_} \AgdaSymbol{→} \AgdaFunction{id'}\AgdaSymbol{)} \AgdaSymbol{(λ} \AgdaBound{\_} \AgdaBound{\_} \AgdaSymbol{→} \AgdaFunction{\_∘c\_}\AgdaSymbol{)} \<[57]%
\>[57]\<%
\\
\>[0]\AgdaIndent{13}{}\<[13]%
\>[13]\AgdaSymbol{(}\AgdaInductiveConstructor{IsCatC} \AgdaFunction{id₁} \AgdaFunction{id₂} \AgdaFunction{comp}\AgdaSymbol{)}\<%
\\
%
\\
\>\<\end{code}

This category has a terminal object which is just the unit set with trivial equality. As a terminal object there is precisely one morphism from every object to it.

\begin{code}\>\<%
\\
%
\\
\>\AgdaFunction{⊤-setoid} \AgdaSymbol{:} \AgdaRecord{hSetoid}\<%
\\
\>\AgdaFunction{⊤-setoid} \AgdaSymbol{=} \AgdaKeyword{record} \AgdaSymbol{\{}\<%
\\
\>[0]\AgdaIndent{6}{}\<[6]%
\>[6]\AgdaField{Carrier} \AgdaSymbol{=} \AgdaRecord{⊤}\AgdaSymbol{;}\<%
\\
\>[0]\AgdaIndent{6}{}\<[6]%
\>[6]\AgdaField{\_≈h\_} \<[14]%
\>[14]\AgdaSymbol{=} \AgdaSymbol{λ} \AgdaBound{\_} \AgdaBound{\_} \AgdaSymbol{→} \AgdaFunction{⊤'}\AgdaSymbol{;}\<%
\\
\>[0]\AgdaIndent{6}{}\<[6]%
\>[6]\AgdaField{isEquiv} \AgdaSymbol{=} \AgdaKeyword{record} \AgdaSymbol{\{}\<%
\\
\>[6]\AgdaIndent{8}{}\<[8]%
\>[8]\AgdaField{refl} \<[16]%
\>[16]\AgdaSymbol{=} \AgdaInductiveConstructor{tt}\AgdaSymbol{;}\<%
\\
\>[6]\AgdaIndent{8}{}\<[8]%
\>[8]\AgdaField{sym} \<[16]%
\>[16]\AgdaSymbol{=} \AgdaSymbol{λ} \AgdaBound{\_} \AgdaSymbol{→} \AgdaInductiveConstructor{tt}\AgdaSymbol{;}\<%
\\
\>[6]\AgdaIndent{8}{}\<[8]%
\>[8]\AgdaField{trans} \<[16]%
\>[16]\AgdaSymbol{=} \AgdaSymbol{λ} \AgdaBound{\_} \AgdaBound{\_} \AgdaSymbol{→} \AgdaInductiveConstructor{tt} \AgdaSymbol{\}} \AgdaSymbol{\}}\<%
\\
%
\\
\>\AgdaFunction{⋆} \AgdaSymbol{:} \AgdaSymbol{\{}\AgdaBound{Δ} \AgdaSymbol{:} \AgdaRecord{hSetoid}\AgdaSymbol{\}} \AgdaSymbol{→} \AgdaBound{Δ} \AgdaRecord{⇉} \AgdaFunction{⊤-setoid}\<%
\\
\>\AgdaFunction{⋆} \AgdaSymbol{=} \AgdaKeyword{record} \<[11]%
\>[11]\<%
\\
\>[0]\AgdaIndent{6}{}\<[6]%
\>[6]\AgdaSymbol{\{} \AgdaField{fn} \AgdaSymbol{=} \AgdaSymbol{λ} \AgdaBound{\_} \AgdaSymbol{→} \AgdaInductiveConstructor{tt}\<%
\\
\>[0]\AgdaIndent{6}{}\<[6]%
\>[6]\AgdaSymbol{;} \AgdaField{resp} \AgdaSymbol{=} \AgdaSymbol{λ} \AgdaBound{\_} \AgdaSymbol{→} \AgdaInductiveConstructor{tt} \AgdaSymbol{\}}\<%
\\
%
\\
\>\AgdaFunction{unique⋆} \AgdaSymbol{:} \AgdaSymbol{\{}\AgdaBound{Δ} \AgdaSymbol{:} \AgdaRecord{hSetoid}\AgdaSymbol{\}} \AgdaSymbol{→} \AgdaSymbol{(}\AgdaBound{f} \AgdaSymbol{:} \AgdaBound{Δ} \AgdaRecord{⇉} \AgdaFunction{⊤-setoid}\AgdaSymbol{)} \AgdaSymbol{→} \AgdaBound{f} \AgdaDatatype{≡} \AgdaFunction{⋆}\<%
\\
\>\AgdaFunction{unique⋆} \AgdaBound{f} \AgdaSymbol{=} \AgdaInductiveConstructor{PE.refl}\<%
\\
%
\\
\>\<\end{code}


%\AgdaHide{
\begin{code}\>\<%
\\
%
\\
\>\AgdaSymbol{\{-\#} \AgdaKeyword{OPTIONS} --type-in-type \AgdaSymbol{\#-\}}\<%
\\
%
\\
%
\\
\>\AgdaKeyword{open} \AgdaKeyword{import} \AgdaModule{Level}\<%
\\
\>\AgdaKeyword{open} \AgdaKeyword{import} \AgdaModule{Relation.Binary.PropositionalEquality} \AgdaSymbol{as} \AgdaModule{PE} \AgdaKeyword{hiding} \AgdaSymbol{(}refl \AgdaSymbol{;} sym \AgdaSymbol{;} trans\AgdaSymbol{;} isEquivalence\AgdaSymbol{;} [\_]\AgdaSymbol{)}\<%
\\
%
\\
\>\AgdaKeyword{module} \AgdaModule{CwF-setoidwo} \AgdaSymbol{(}\AgdaBound{ext} \AgdaSymbol{:} \AgdaFunction{Extensionality} \AgdaPrimitive{zero} \AgdaPrimitive{zero}\AgdaSymbol{)} \AgdaKeyword{where}\<%
\\
%
\\
%
\\
\>\AgdaKeyword{open} \AgdaKeyword{import} \AgdaModule{Cats.Category}\<%
\\
\>\AgdaKeyword{open} \AgdaKeyword{import} \AgdaModule{Cats.Functor} \AgdaKeyword{renaming} \AgdaSymbol{(}Functor \AgdaSymbol{to} \_⇨\_\AgdaSymbol{)}\<%
\\
\>\AgdaKeyword{open} \AgdaKeyword{import} \AgdaModule{Cats.Duality}\<%
\\
\>\AgdaKeyword{open} \AgdaKeyword{import} \AgdaModule{Data.Product} \AgdaKeyword{renaming} \AgdaSymbol{(}<\_,\_> \AgdaSymbol{to} ⟨\_,\_⟩\AgdaSymbol{)}\<%
\\
\>\AgdaKeyword{open} \AgdaKeyword{import} \AgdaModule{Function}\<%
\\
%
\\
\>\AgdaComment{-- open import Relation.Binary}\<%
\\
\>\AgdaKeyword{open} \AgdaKeyword{import} \AgdaModule{Relation.Binary.Core} \AgdaKeyword{using} \AgdaSymbol{(}\_≡\_\AgdaSymbol{;} \_≢\_\AgdaSymbol{)}\<%
\\
\>\AgdaKeyword{open} \AgdaKeyword{import} \AgdaModule{Data.Unit}\<%
\\
%
\\
\>\AgdaKeyword{open} \AgdaKeyword{import} \AgdaModule{CategoryOfSetoid} \AgdaBound{ext} \AgdaKeyword{public}\<%
\\
%
\\
\>\AgdaComment{-- open import HProp ext}\<%
\\
%
\\
\>\AgdaKeyword{open} \AgdaKeyword{import} \AgdaModule{CwF} \AgdaKeyword{hiding} \AgdaSymbol{(}\_⇉\_\AgdaSymbol{)}\<%
\\
%
\\
\>\<\end{code}
}

\section{Categories with families}

Context are interpreted as setoids

\begin{code}\>\<%
\\
%
\\
\>\AgdaFunction{Con} \AgdaSymbol{=} \AgdaRecord{HSetoid}\<%
\\
%
\\
\>\<\end{code}

Types are functors from $\Gamma → \mathbf{Std}$

\begin{code}\>\<%
\\
%
\\
\>\AgdaKeyword{record} \AgdaRecord{Ty} \AgdaSymbol{(}\AgdaBound{Γ} \AgdaSymbol{:} \AgdaFunction{Con}\AgdaSymbol{)} \AgdaSymbol{:} \AgdaPrimitiveType{Set₁} \AgdaKeyword{where}\<%
\\
\>[0]\AgdaIndent{2}{}\<[2]%
\>[2]\AgdaKeyword{field}\<%
\\
\>[2]\AgdaIndent{4}{}\<[4]%
\>[4]\AgdaField{fm} \<[11]%
\>[11]\AgdaSymbol{:} \AgdaFunction{∣} \AgdaBound{Γ} \AgdaFunction{∣} \AgdaSymbol{→} \AgdaRecord{HSetoid}\<%
\\
%
\\
\>[2]\AgdaIndent{4}{}\<[4]%
\>[4]\AgdaField{substT} \AgdaSymbol{:} \AgdaSymbol{\{}\AgdaBound{x} \AgdaBound{y} \AgdaSymbol{:} \AgdaFunction{∣} \AgdaBound{Γ} \AgdaFunction{∣}\AgdaSymbol{\}} \AgdaSymbol{→} \<[29]%
\>[29]\<%
\\
\>[4]\AgdaIndent{13}{}\<[13]%
\>[13]\AgdaFunction{[} \AgdaBound{Γ} \AgdaFunction{]} \AgdaBound{x} \AgdaFunction{≈} \AgdaBound{y} \AgdaSymbol{→}\<%
\\
\>[4]\AgdaIndent{13}{}\<[13]%
\>[13]\AgdaFunction{∣} \AgdaBound{fm} \AgdaBound{x} \AgdaFunction{∣} \AgdaSymbol{→} \AgdaFunction{∣} \AgdaBound{fm} \AgdaBound{y} \AgdaFunction{∣}\<%
\\
%
\\
\>[0]\AgdaIndent{4}{}\<[4]%
\>[4]\AgdaField{subst*} \AgdaSymbol{:} \AgdaSymbol{∀\{}\AgdaBound{x} \AgdaBound{y} \AgdaSymbol{:} \AgdaFunction{∣} \AgdaBound{Γ} \AgdaFunction{∣}\AgdaSymbol{\}}\<%
\\
\>[0]\AgdaIndent{13}{}\<[13]%
\>[13]\AgdaSymbol{(}\AgdaBound{p} \AgdaSymbol{:} \AgdaFunction{[} \AgdaBound{Γ} \AgdaFunction{]} \AgdaBound{x} \AgdaFunction{≈} \AgdaBound{y}\AgdaSymbol{)}\<%
\\
\>[0]\AgdaIndent{13}{}\<[13]%
\>[13]\AgdaSymbol{\{}\AgdaBound{a} \AgdaBound{b} \AgdaSymbol{:} \AgdaFunction{∣} \AgdaBound{fm} \AgdaBound{x} \AgdaFunction{∣}\AgdaSymbol{\}} \AgdaSymbol{→}\<%
\\
\>[0]\AgdaIndent{13}{}\<[13]%
\>[13]\AgdaFunction{[} \AgdaBound{fm} \AgdaBound{x} \AgdaFunction{]} \AgdaBound{a} \AgdaFunction{≈} \AgdaBound{b} \AgdaSymbol{→}\<%
\\
\>[0]\AgdaIndent{13}{}\<[13]%
\>[13]\AgdaFunction{[} \AgdaBound{fm} \AgdaBound{y} \AgdaFunction{]} \AgdaBound{substT} \AgdaBound{p} \AgdaBound{a} \AgdaFunction{≈} \AgdaBound{substT} \AgdaBound{p} \AgdaBound{b}\<%
\\
%
\\
\>[0]\AgdaIndent{4}{}\<[4]%
\>[4]\AgdaField{refl*} \<[11]%
\>[11]\AgdaSymbol{:} \AgdaSymbol{∀(}\AgdaBound{x} \AgdaSymbol{:} \AgdaFunction{∣} \AgdaBound{Γ} \AgdaFunction{∣}\AgdaSymbol{)(}\AgdaBound{a} \AgdaSymbol{:} \AgdaFunction{∣} \AgdaBound{fm} \AgdaBound{x} \AgdaFunction{∣}\AgdaSymbol{)} \AgdaSymbol{→} \<[42]%
\>[42]\<%
\\
\>[0]\AgdaIndent{13}{}\<[13]%
\>[13]\AgdaFunction{[} \AgdaBound{fm} \AgdaBound{x} \AgdaFunction{]} \AgdaBound{substT} \AgdaFunction{[} \AgdaBound{Γ} \AgdaFunction{]refl} \AgdaBound{a} \AgdaFunction{≈} \AgdaBound{a}\<%
\\
%
\\
\>[0]\AgdaIndent{4}{}\<[4]%
\>[4]\AgdaField{trans*} \AgdaSymbol{:} \AgdaSymbol{∀\{}\AgdaBound{x} \AgdaBound{y} \AgdaBound{z} \AgdaSymbol{:} \AgdaFunction{∣} \AgdaBound{Γ} \AgdaFunction{∣}\AgdaSymbol{\}}\<%
\\
\>[0]\AgdaIndent{13}{}\<[13]%
\>[13]\AgdaSymbol{\{}\AgdaBound{p} \AgdaSymbol{:} \AgdaFunction{[} \AgdaBound{Γ} \AgdaFunction{]} \AgdaBound{x} \AgdaFunction{≈} \AgdaBound{y}\AgdaSymbol{\}\{}\AgdaBound{q} \AgdaSymbol{:} \AgdaFunction{[} \AgdaBound{Γ} \AgdaFunction{]} \AgdaBound{y} \AgdaFunction{≈} \AgdaBound{z}\AgdaSymbol{\}}\<%
\\
\>[0]\AgdaIndent{13}{}\<[13]%
\>[13]\AgdaSymbol{(}\AgdaBound{a} \AgdaSymbol{:} \AgdaFunction{∣} \AgdaBound{fm} \AgdaBound{x} \AgdaFunction{∣}\AgdaSymbol{)} \AgdaSymbol{→} \<[30]%
\>[30]\<%
\\
\>[0]\AgdaIndent{13}{}\<[13]%
\>[13]\AgdaFunction{[} \AgdaBound{fm} \AgdaBound{z} \AgdaFunction{]} \AgdaBound{substT} \AgdaBound{q} \AgdaSymbol{(}\AgdaBound{substT} \AgdaBound{p} \AgdaBound{a}\AgdaSymbol{)} \AgdaFunction{≈} \AgdaBound{substT} \AgdaSymbol{(}\AgdaFunction{[} \AgdaBound{Γ} \AgdaFunction{]trans} \AgdaBound{p} \AgdaBound{q}\AgdaSymbol{)} \AgdaBound{a}\<%
\\
%
\\
\>\<\end{code}

Proof-irrelevance lemmas for substT

\begin{code}\>\<%
\\
%
\\
\>[0]\AgdaIndent{2}{}\<[2]%
\>[2]\AgdaFunction{subst-pi} \AgdaSymbol{:} \AgdaSymbol{∀\{}\AgdaBound{x} \AgdaBound{y} \AgdaSymbol{:} \AgdaFunction{∣} \AgdaBound{Γ} \AgdaFunction{∣}\AgdaSymbol{\}}\<%
\\
\>[0]\AgdaIndent{14}{}\<[14]%
\>[14]\AgdaSymbol{\{}\AgdaBound{p} \AgdaBound{q} \AgdaSymbol{:} \AgdaFunction{[} \AgdaBound{Γ} \AgdaFunction{]} \AgdaBound{x} \AgdaFunction{≈} \AgdaBound{y}\AgdaSymbol{\}}\<%
\\
\>[0]\AgdaIndent{14}{}\<[14]%
\>[14]\AgdaSymbol{\{}\AgdaBound{a} \AgdaSymbol{:} \AgdaFunction{∣} \AgdaFunction{fm} \AgdaBound{x} \AgdaFunction{∣}\AgdaSymbol{\}} \<[29]%
\>[29]\<%
\\
\>[0]\AgdaIndent{11}{}\<[11]%
\>[11]\AgdaSymbol{→} \AgdaFunction{[} \AgdaFunction{fm} \AgdaBound{y} \AgdaFunction{]} \AgdaFunction{substT} \AgdaBound{p} \AgdaBound{a} \AgdaFunction{≈} \AgdaFunction{substT} \AgdaBound{q} \AgdaBound{a}\<%
\\
\>[0]\AgdaIndent{2}{}\<[2]%
\>[2]\AgdaFunction{subst-pi} \AgdaSymbol{\{}\AgdaBound{x}\AgdaSymbol{\}} \AgdaSymbol{\{}\AgdaBound{y}\AgdaSymbol{\}} \AgdaSymbol{\{}\AgdaBound{p}\AgdaSymbol{\}} \AgdaSymbol{\{}\AgdaBound{q}\AgdaSymbol{\}} \AgdaSymbol{\{}\AgdaBound{a}\AgdaSymbol{\}} \<[31]%
\>[31]\<%
\\
\>[2]\AgdaIndent{11}{}\<[11]%
\>[11]\AgdaSymbol{=} \AgdaFunction{reflexive} \AgdaSymbol{(}\AgdaFunction{fm} \AgdaBound{y}\AgdaSymbol{)} \AgdaSymbol{(}\AgdaFunction{PI} \AgdaBound{Γ} \AgdaSymbol{(λ} \AgdaBound{x} \AgdaSymbol{→} \AgdaFunction{substT} \AgdaBound{x} \AgdaBound{a}\AgdaSymbol{))}\<%
\\
%
\\
\>[0]\AgdaIndent{2}{}\<[2]%
\>[2]\AgdaFunction{subst-pi'} \AgdaSymbol{:} \AgdaSymbol{∀\{}\AgdaBound{x} \AgdaSymbol{:} \AgdaFunction{∣} \AgdaBound{Γ} \AgdaFunction{∣}\AgdaSymbol{\}}\<%
\\
\>[0]\AgdaIndent{15}{}\<[15]%
\>[15]\AgdaSymbol{\{}\AgdaBound{p} \AgdaSymbol{:} \AgdaFunction{[} \AgdaBound{Γ} \AgdaFunction{]} \AgdaBound{x} \AgdaFunction{≈} \AgdaBound{x}\AgdaSymbol{\}}\<%
\\
\>[0]\AgdaIndent{15}{}\<[15]%
\>[15]\AgdaSymbol{\{}\AgdaBound{a} \AgdaSymbol{:} \AgdaFunction{∣} \AgdaFunction{fm} \AgdaBound{x} \AgdaFunction{∣}\AgdaSymbol{\}} \AgdaSymbol{→} \AgdaFunction{[} \AgdaFunction{fm} \AgdaBound{x} \AgdaFunction{]} \AgdaFunction{substT} \AgdaBound{p} \AgdaBound{a} \AgdaFunction{≈} \AgdaBound{a}\<%
\\
\>[0]\AgdaIndent{2}{}\<[2]%
\>[2]\AgdaFunction{subst-pi'} \AgdaSymbol{=} \AgdaFunction{[} \AgdaFunction{fm} \AgdaSymbol{\_} \AgdaFunction{]trans} \AgdaFunction{subst-pi} \AgdaSymbol{(}\AgdaFunction{refl*} \AgdaSymbol{\_} \AgdaSymbol{\_)}\<%
\\
%
\\
\>[0]\AgdaIndent{2}{}\<[2]%
\>[2]\AgdaFunction{subst-pi*} \AgdaSymbol{:} \AgdaSymbol{∀\{}\AgdaBound{x} \AgdaBound{y} \AgdaSymbol{:} \AgdaFunction{∣} \AgdaBound{Γ} \AgdaFunction{∣}\AgdaSymbol{\}}\<%
\\
\>[2]\AgdaIndent{16}{}\<[16]%
\>[16]\AgdaSymbol{\{}\AgdaBound{p} \AgdaBound{q} \AgdaSymbol{:} \AgdaFunction{[} \AgdaBound{Γ} \AgdaFunction{]} \AgdaBound{x} \AgdaFunction{≈} \AgdaBound{y}\AgdaSymbol{\}}\<%
\\
\>[2]\AgdaIndent{16}{}\<[16]%
\>[16]\AgdaSymbol{\{}\AgdaBound{a} \AgdaBound{b} \AgdaSymbol{:} \AgdaFunction{∣} \AgdaFunction{fm} \AgdaBound{x} \AgdaFunction{∣}\AgdaSymbol{\}} \AgdaSymbol{→} \AgdaFunction{[} \AgdaFunction{fm} \AgdaBound{x} \AgdaFunction{]} \AgdaBound{a} \AgdaFunction{≈} \AgdaBound{b} \AgdaSymbol{→} \AgdaFunction{[} \AgdaFunction{fm} \AgdaBound{y} \AgdaFunction{]} \AgdaFunction{substT} \AgdaBound{p} \AgdaBound{a} \AgdaFunction{≈} \AgdaFunction{substT} \AgdaBound{q} \AgdaBound{b}\<%
\\
\>[0]\AgdaIndent{2}{}\<[2]%
\>[2]\AgdaFunction{subst-pi*} \AgdaBound{eq} \AgdaSymbol{=} \AgdaFunction{[} \AgdaFunction{fm} \AgdaSymbol{\_} \AgdaFunction{]trans} \AgdaSymbol{(}\AgdaFunction{subst*} \AgdaSymbol{\_} \AgdaBound{eq}\AgdaSymbol{)} \AgdaFunction{subst-pi}\<%
\\
%
\\
%
\\
%
\\
\>[0]\AgdaIndent{2}{}\<[2]%
\>[2]\AgdaFunction{trans-refl} \AgdaSymbol{:} \AgdaSymbol{∀\{}\AgdaBound{x} \AgdaBound{y} \AgdaSymbol{:} \AgdaFunction{∣} \AgdaBound{Γ} \AgdaFunction{∣}\AgdaSymbol{\}}\<%
\\
\>[2]\AgdaIndent{14}{}\<[14]%
\>[14]\AgdaSymbol{\{}\AgdaBound{p} \AgdaSymbol{:} \AgdaFunction{[} \AgdaBound{Γ} \AgdaFunction{]} \AgdaBound{x} \AgdaFunction{≈} \AgdaBound{y}\AgdaSymbol{\}\{}\AgdaBound{q} \AgdaSymbol{:} \AgdaFunction{[} \AgdaBound{Γ} \AgdaFunction{]} \AgdaBound{y} \AgdaFunction{≈} \AgdaBound{x}\AgdaSymbol{\}}\<%
\\
\>[2]\AgdaIndent{14}{}\<[14]%
\>[14]\AgdaSymbol{\{}\AgdaBound{a} \AgdaSymbol{:} \AgdaFunction{∣} \AgdaFunction{fm} \AgdaBound{x} \AgdaFunction{∣}\AgdaSymbol{\}} \AgdaSymbol{→} \<[31]%
\>[31]\<%
\\
\>[2]\AgdaIndent{14}{}\<[14]%
\>[14]\AgdaFunction{[} \AgdaFunction{fm} \AgdaBound{x} \AgdaFunction{]} \AgdaFunction{substT} \AgdaBound{q} \AgdaSymbol{(}\AgdaFunction{substT} \AgdaBound{p} \AgdaBound{a}\AgdaSymbol{)} \AgdaFunction{≈} \AgdaBound{a}\<%
\\
\>[0]\AgdaIndent{2}{}\<[2]%
\>[2]\AgdaFunction{trans-refl} \AgdaSymbol{=} \AgdaFunction{[} \AgdaFunction{fm} \AgdaSymbol{\_} \AgdaFunction{]trans} \AgdaSymbol{(}\AgdaFunction{trans*} \AgdaSymbol{\_)} \AgdaFunction{subst-pi'}\<%
\\
%
\\
\>[0]\AgdaIndent{2}{}\<[2]%
\>[2]\AgdaFunction{subst-mir1} \AgdaSymbol{:} \AgdaSymbol{∀\{}\AgdaBound{x} \AgdaBound{y} \AgdaSymbol{:} \AgdaFunction{∣} \AgdaBound{Γ} \AgdaFunction{∣}\AgdaSymbol{\}}\<%
\\
\>[2]\AgdaIndent{14}{}\<[14]%
\>[14]\AgdaSymbol{\{}\AgdaBound{p} \AgdaSymbol{:} \AgdaFunction{[} \AgdaBound{Γ} \AgdaFunction{]} \AgdaBound{x} \AgdaFunction{≈} \AgdaBound{y}\AgdaSymbol{\}\{}\AgdaBound{q} \AgdaSymbol{:} \AgdaFunction{[} \AgdaBound{Γ} \AgdaFunction{]} \AgdaBound{y} \AgdaFunction{≈} \AgdaBound{x}\AgdaSymbol{\}}\<%
\\
\>[2]\AgdaIndent{14}{}\<[14]%
\>[14]\AgdaSymbol{\{}\AgdaBound{a} \AgdaSymbol{:} \AgdaFunction{∣} \AgdaFunction{fm} \AgdaBound{x} \AgdaFunction{∣}\AgdaSymbol{\}\{}\AgdaBound{b} \AgdaSymbol{:} \AgdaFunction{∣} \AgdaFunction{fm} \AgdaBound{y} \AgdaFunction{∣}\AgdaSymbol{\}} \AgdaSymbol{→} \<[45]%
\>[45]\<%
\\
\>[2]\AgdaIndent{14}{}\<[14]%
\>[14]\AgdaFunction{[} \AgdaFunction{fm} \AgdaBound{x} \AgdaFunction{]} \AgdaBound{a} \AgdaFunction{≈} \AgdaFunction{substT} \AgdaBound{q} \AgdaBound{b} \AgdaSymbol{→} \AgdaFunction{[} \AgdaFunction{fm} \AgdaBound{y} \AgdaFunction{]} \AgdaFunction{substT} \AgdaBound{p} \AgdaBound{a} \AgdaFunction{≈} \AgdaBound{b}\<%
\\
\>[0]\AgdaIndent{2}{}\<[2]%
\>[2]\AgdaFunction{subst-mir1} \AgdaBound{eq} \AgdaSymbol{=} \AgdaFunction{[} \AgdaFunction{fm} \AgdaSymbol{\_} \AgdaFunction{]trans} \AgdaSymbol{(}\AgdaFunction{subst*} \AgdaSymbol{\_} \AgdaBound{eq}\AgdaSymbol{)} \AgdaFunction{trans-refl}\<%
\\
%
\\
\>[0]\AgdaIndent{2}{}\<[2]%
\>[2]\AgdaFunction{subst-mir2} \AgdaSymbol{:} \AgdaSymbol{∀\{}\AgdaBound{x} \AgdaBound{y} \AgdaSymbol{:} \AgdaFunction{∣} \AgdaBound{Γ} \AgdaFunction{∣}\AgdaSymbol{\}}\<%
\\
\>[2]\AgdaIndent{14}{}\<[14]%
\>[14]\AgdaSymbol{\{}\AgdaBound{p} \AgdaSymbol{:} \AgdaFunction{[} \AgdaBound{Γ} \AgdaFunction{]} \AgdaBound{x} \AgdaFunction{≈} \AgdaBound{y}\AgdaSymbol{\}\{}\AgdaBound{q} \AgdaSymbol{:} \AgdaFunction{[} \AgdaBound{Γ} \AgdaFunction{]} \AgdaBound{y} \AgdaFunction{≈} \AgdaBound{x}\AgdaSymbol{\}}\<%
\\
\>[2]\AgdaIndent{14}{}\<[14]%
\>[14]\AgdaSymbol{\{}\AgdaBound{a} \AgdaSymbol{:} \AgdaFunction{∣} \AgdaFunction{fm} \AgdaBound{x} \AgdaFunction{∣}\AgdaSymbol{\}\{}\AgdaBound{b} \AgdaSymbol{:} \AgdaFunction{∣} \AgdaFunction{fm} \AgdaBound{y} \AgdaFunction{∣}\AgdaSymbol{\}} \AgdaSymbol{→} \<[45]%
\>[45]\<%
\\
\>[2]\AgdaIndent{14}{}\<[14]%
\>[14]\AgdaFunction{[} \AgdaFunction{fm} \AgdaBound{y} \AgdaFunction{]} \AgdaFunction{substT} \AgdaBound{p} \AgdaBound{a} \AgdaFunction{≈} \AgdaBound{b} \AgdaSymbol{→} \AgdaFunction{[} \AgdaFunction{fm} \AgdaBound{x} \AgdaFunction{]} \AgdaBound{a} \AgdaFunction{≈} \AgdaFunction{substT} \AgdaBound{q} \AgdaBound{b}\<%
\\
\>[0]\AgdaIndent{2}{}\<[2]%
\>[2]\AgdaFunction{subst-mir2} \AgdaBound{eq} \AgdaSymbol{=} \AgdaFunction{[} \AgdaFunction{fm} \AgdaSymbol{\_} \AgdaFunction{]sym} \AgdaSymbol{(}\AgdaFunction{subst-mir1} \AgdaSymbol{(}\AgdaFunction{[} \AgdaFunction{fm} \AgdaSymbol{\_} \AgdaFunction{]sym} \AgdaBound{eq}\AgdaSymbol{))}\<%
\\
%
\\
\>\AgdaKeyword{open} \AgdaModule{Ty} \AgdaKeyword{public} \<[15]%
\>[15]\<%
\\
\>[0]\AgdaIndent{2}{}\<[2]%
\>[2]\AgdaKeyword{renaming} \AgdaSymbol{(}substT \AgdaSymbol{to} [\_]subst\AgdaSymbol{;} subst* \AgdaSymbol{to} [\_]subst*\AgdaSymbol{;} fm \AgdaSymbol{to} [\_]fm \AgdaSymbol{;}
            refl* \AgdaSymbol{to} [\_]refl* \AgdaSymbol{;} trans* \AgdaSymbol{to} [\_]trans*\AgdaSymbol{;} 
            subst-pi \AgdaSymbol{to} [\_]subst-pi \AgdaSymbol{;}
            subst-pi' \AgdaSymbol{to} [\_]subst-pi' \AgdaSymbol{;} subst-pi* \AgdaSymbol{to} [\_]subst-pi* \AgdaSymbol{;}
            trans-refl \AgdaSymbol{to} [\_]trans-refl \AgdaSymbol{;} subst-mir1 \AgdaSymbol{to} [\_]subst-mir1 \AgdaSymbol{;}
            subst-mir2 \AgdaSymbol{to} [\_]subst-mir2\AgdaSymbol{)}\<%
\\
\>\<\end{code}

Type substitution

\begin{code}\>\<%
\\
%
\\
\>\AgdaFunction{\_[\_]T} \AgdaSymbol{:} \AgdaSymbol{∀} \AgdaSymbol{\{}\AgdaBound{Γ} \AgdaBound{Δ} \AgdaSymbol{:} \AgdaFunction{Con}\AgdaSymbol{\}} \AgdaSymbol{→} \AgdaRecord{Ty} \AgdaBound{Δ} \AgdaSymbol{→} \AgdaBound{Γ} \AgdaRecord{⇉} \AgdaBound{Δ} \AgdaSymbol{→} \AgdaRecord{Ty} \AgdaBound{Γ}\<%
\\
\>\AgdaFunction{\_[\_]T} \AgdaSymbol{\{}\AgdaBound{Γ}\AgdaSymbol{\}} \AgdaSymbol{\{}\AgdaBound{Δ}\AgdaSymbol{\}} \AgdaBound{A} \AgdaBound{f}\<%
\\
\>[2]\AgdaIndent{5}{}\<[5]%
\>[5]\AgdaSymbol{=} \AgdaKeyword{record}\<%
\\
\>[2]\AgdaIndent{5}{}\<[5]%
\>[5]\AgdaSymbol{\{} \AgdaField{fm} \<[14]%
\>[14]\AgdaSymbol{=} \AgdaFunction{fm} \AgdaFunction{∘} \AgdaFunction{fn}\<%
\\
\>[2]\AgdaIndent{5}{}\<[5]%
\>[5]\AgdaSymbol{;} \AgdaField{substT} \AgdaSymbol{=} \AgdaFunction{substT} \AgdaFunction{∘} \AgdaFunction{resp}\<%
\\
\>[2]\AgdaIndent{5}{}\<[5]%
\>[5]\AgdaSymbol{;} \AgdaField{subst*} \AgdaSymbol{=} \AgdaFunction{subst*} \AgdaFunction{∘} \AgdaFunction{resp}\<%
\\
\>[2]\AgdaIndent{5}{}\<[5]%
\>[5]\AgdaSymbol{;} \AgdaField{refl*} \<[14]%
\>[14]\AgdaSymbol{=} \AgdaSymbol{λ} \AgdaBound{\_} \AgdaBound{\_} \AgdaSymbol{→} \AgdaFunction{subst-pi'}\<%
\\
\>[2]\AgdaIndent{5}{}\<[5]%
\>[5]\AgdaSymbol{;} \AgdaField{trans*} \AgdaSymbol{=} \AgdaSymbol{λ} \AgdaBound{\_} \AgdaSymbol{→} \AgdaFunction{[} \AgdaFunction{fm} \AgdaSymbol{(}\AgdaFunction{fn} \AgdaSymbol{\_)} \AgdaFunction{]trans} \AgdaSymbol{(}\AgdaFunction{trans*} \AgdaSymbol{\_)} \AgdaFunction{subst-pi}\<%
\\
\>[2]\AgdaIndent{5}{}\<[5]%
\>[5]\AgdaSymbol{\}}\<%
\\
\>[2]\AgdaIndent{5}{}\<[5]%
\>[5]\AgdaKeyword{where} \<[11]%
\>[11]\<%
\\
\>[5]\AgdaIndent{7}{}\<[7]%
\>[7]\AgdaKeyword{open} \AgdaModule{Ty} \AgdaBound{A}\<%
\\
\>[5]\AgdaIndent{7}{}\<[7]%
\>[7]\AgdaKeyword{open} \AgdaModule{\_⇉\_} \AgdaBound{f}\<%
\\
%
\\
\>\<\end{code}

Terms

\begin{code}\>\<%
\\
%
\\
\>\AgdaKeyword{record} \AgdaRecord{Tm} \AgdaSymbol{\{}\AgdaBound{Γ} \AgdaSymbol{:} \AgdaFunction{Con}\AgdaSymbol{\}(}\AgdaBound{A} \AgdaSymbol{:} \AgdaRecord{Ty} \AgdaBound{Γ}\AgdaSymbol{)} \AgdaSymbol{:} \AgdaPrimitiveType{Set} \AgdaKeyword{where}\<%
\\
\>[0]\AgdaIndent{2}{}\<[2]%
\>[2]\AgdaKeyword{constructor} \AgdaInductiveConstructor{tm:\_resp:\_}\<%
\\
\>[0]\AgdaIndent{2}{}\<[2]%
\>[2]\AgdaKeyword{field}\<%
\\
\>[2]\AgdaIndent{4}{}\<[4]%
\>[4]\AgdaField{tm} \<[10]%
\>[10]\AgdaSymbol{:} \AgdaSymbol{(}\AgdaBound{x} \AgdaSymbol{:} \AgdaFunction{∣} \AgdaBound{Γ} \AgdaFunction{∣}\AgdaSymbol{)} \AgdaSymbol{→} \AgdaFunction{∣} \AgdaFunction{[} \AgdaBound{A} \AgdaFunction{]fm} \AgdaBound{x} \AgdaFunction{∣}\<%
\\
\>[2]\AgdaIndent{4}{}\<[4]%
\>[4]\AgdaField{respt} \AgdaSymbol{:} \AgdaSymbol{∀} \AgdaSymbol{\{}\AgdaBound{x} \AgdaBound{y} \AgdaSymbol{:} \AgdaFunction{∣} \AgdaBound{Γ} \AgdaFunction{∣}\AgdaSymbol{\}} \AgdaSymbol{→} \<[30]%
\>[30]\<%
\\
\>[4]\AgdaIndent{10}{}\<[10]%
\>[10]\AgdaSymbol{(}\AgdaBound{p} \AgdaSymbol{:} \AgdaFunction{[} \AgdaBound{Γ} \AgdaFunction{]} \AgdaBound{x} \AgdaFunction{≈} \AgdaBound{y}\AgdaSymbol{)} \AgdaSymbol{→} \<[30]%
\>[30]\<%
\\
\>[4]\AgdaIndent{10}{}\<[10]%
\>[10]\AgdaFunction{[} \AgdaFunction{[} \AgdaBound{A} \AgdaFunction{]fm} \AgdaBound{y} \AgdaFunction{]} \AgdaFunction{[} \AgdaBound{A} \AgdaFunction{]subst} \AgdaBound{p} \AgdaSymbol{(}\AgdaBound{tm} \AgdaBound{x}\AgdaSymbol{)} \AgdaFunction{≈} \AgdaBound{tm} \AgdaBound{y}\<%
\\
%
\\
\>\AgdaKeyword{open} \AgdaModule{Tm} \AgdaKeyword{public} \AgdaKeyword{renaming} \AgdaSymbol{(}tm \AgdaSymbol{to} [\_]tm \AgdaSymbol{;} respt \AgdaSymbol{to} [\_]respt\AgdaSymbol{)}\<%
\\
%
\\
\>\<\end{code}

Term substitution

\begin{code}\>\<%
\\
%
\\
\>\AgdaFunction{\_[\_]m} \AgdaSymbol{:} \AgdaSymbol{∀} \AgdaSymbol{\{}\AgdaBound{Γ} \AgdaBound{Δ} \AgdaSymbol{:} \AgdaFunction{Con}\AgdaSymbol{\}\{}\AgdaBound{A} \AgdaSymbol{:} \AgdaRecord{Ty} \AgdaBound{Δ}\AgdaSymbol{\}} \AgdaSymbol{→} \AgdaRecord{Tm} \AgdaBound{A} \<[39]%
\>[39]\<%
\\
\>[0]\AgdaIndent{6}{}\<[6]%
\>[6]\AgdaSymbol{→} \AgdaSymbol{(}\AgdaBound{f} \AgdaSymbol{:} \AgdaBound{Γ} \AgdaRecord{⇉} \AgdaBound{Δ}\AgdaSymbol{)} \AgdaSymbol{→} \AgdaRecord{Tm} \AgdaSymbol{(}\AgdaBound{A} \AgdaFunction{[} \AgdaBound{f} \AgdaFunction{]T}\AgdaSymbol{)}\<%
\\
\>\AgdaFunction{\_[\_]m} \AgdaBound{t} \AgdaBound{f} \AgdaSymbol{=} \AgdaKeyword{record} \<[19]%
\>[19]\<%
\\
\>[0]\AgdaIndent{10}{}\<[10]%
\>[10]\AgdaSymbol{\{} \AgdaField{tm} \AgdaSymbol{=} \AgdaFunction{[} \AgdaBound{t} \AgdaFunction{]tm} \AgdaFunction{∘} \AgdaFunction{[} \AgdaBound{f} \AgdaFunction{]fn}\<%
\\
\>[0]\AgdaIndent{10}{}\<[10]%
\>[10]\AgdaSymbol{;} \AgdaField{respt} \AgdaSymbol{=} \AgdaFunction{[} \AgdaBound{t} \AgdaFunction{]respt} \AgdaFunction{∘} \AgdaFunction{[} \AgdaBound{f} \AgdaFunction{]resp} \<[43]%
\>[43]\<%
\\
\>[0]\AgdaIndent{10}{}\<[10]%
\>[10]\AgdaSymbol{\}}\<%
\\
\>\<\end{code}

Context comprehension

\begin{code}\>\<%
\\
\>[0]\AgdaIndent{1}{}\<[1]%
\>[1]\<%
\\
\>\AgdaFunction{\_\&\_} \AgdaSymbol{:} \AgdaSymbol{(}\AgdaBound{Γ} \AgdaSymbol{:} \AgdaFunction{Con}\AgdaSymbol{)} \AgdaSymbol{→} \AgdaRecord{Ty} \AgdaBound{Γ} \AgdaSymbol{→} \AgdaRecord{HSetoid}\<%
\\
\>\AgdaBound{Γ} \AgdaFunction{\&} \AgdaBound{A} \AgdaSymbol{=} \AgdaKeyword{record} \<[15]%
\>[15]\<%
\\
\>[0]\AgdaIndent{7}{}\<[7]%
\>[7]\AgdaSymbol{\{} \AgdaField{Carrier} \AgdaSymbol{=} \AgdaRecord{Σ[} \AgdaBound{x} \AgdaRecord{∶} \AgdaFunction{∣} \AgdaBound{Γ} \AgdaFunction{∣} \AgdaRecord{]} \AgdaFunction{∣} \AgdaFunction{fm} \AgdaBound{x} \AgdaFunction{∣}\<%
\\
\>[0]\AgdaIndent{7}{}\<[7]%
\>[7]\AgdaSymbol{;} \AgdaField{\_≈h\_} \<[15]%
\>[15]\AgdaSymbol{=} \AgdaSymbol{λ\{(}\AgdaBound{x} \AgdaInductiveConstructor{,} \AgdaBound{a}\AgdaSymbol{)} \AgdaSymbol{(}\AgdaBound{y} \AgdaInductiveConstructor{,} \AgdaBound{b}\AgdaSymbol{)} \AgdaSymbol{→} \<[37]%
\>[37]\<%
\\
\>[7]\AgdaIndent{17}{}\<[17]%
\>[17]\AgdaFunction{Σ'[} \AgdaBound{p} \AgdaFunction{∶} \AgdaBound{x} \AgdaFunction{≈h} \AgdaBound{y} \AgdaFunction{]} \AgdaFunction{[} \AgdaFunction{fm} \AgdaBound{y} \AgdaFunction{]} \AgdaFunction{substT} \AgdaBound{p} \AgdaBound{a} \AgdaFunction{≈h} \AgdaBound{b}\AgdaSymbol{\}}\<%
\\
\>[0]\AgdaIndent{7}{}\<[7]%
\>[7]\AgdaSymbol{;} \AgdaField{refl} \<[15]%
\>[15]\AgdaSymbol{=} \AgdaFunction{refl} \AgdaInductiveConstructor{,} \AgdaSymbol{(}\AgdaFunction{refl*} \AgdaSymbol{\_} \AgdaSymbol{\_)}\<%
\\
\>[0]\AgdaIndent{7}{}\<[7]%
\>[7]\AgdaSymbol{;} \AgdaField{sym} \<[15]%
\>[15]\AgdaSymbol{=} \AgdaSymbol{λ} \AgdaSymbol{\{(}\AgdaBound{p} \AgdaInductiveConstructor{,} \AgdaBound{q}\AgdaSymbol{)} \AgdaSymbol{→} \AgdaSymbol{(}\AgdaFunction{sym} \AgdaBound{p}\AgdaSymbol{)} \AgdaInductiveConstructor{,} \<[40]%
\>[40]\<%
\\
\>[7]\AgdaIndent{17}{}\<[17]%
\>[17]\AgdaFunction{[} \AgdaFunction{fm} \AgdaSymbol{\_} \AgdaFunction{]trans} \AgdaSymbol{(}\AgdaFunction{subst*} \AgdaSymbol{\_} \AgdaSymbol{(}\AgdaFunction{[} \AgdaFunction{fm} \AgdaSymbol{\_} \AgdaFunction{]sym} \AgdaBound{q}\AgdaSymbol{))} \AgdaFunction{trans-refl} \AgdaSymbol{\}}\<%
\\
\>[0]\AgdaIndent{7}{}\<[7]%
\>[7]\AgdaSymbol{;} \AgdaField{trans} \AgdaSymbol{=} \AgdaSymbol{λ} \AgdaSymbol{\{(}\AgdaBound{p} \AgdaInductiveConstructor{,} \AgdaBound{q}\AgdaSymbol{)} \AgdaSymbol{(}\AgdaBound{m} \AgdaInductiveConstructor{,} \AgdaBound{n}\AgdaSymbol{)} \AgdaSymbol{→}\<%
\\
\>[0]\AgdaIndent{17}{}\<[17]%
\>[17]\AgdaFunction{trans} \AgdaBound{p} \AgdaBound{m} \AgdaInductiveConstructor{,} \<[29]%
\>[29]\<%
\\
\>[0]\AgdaIndent{17}{}\<[17]%
\>[17]\AgdaFunction{[} \AgdaFunction{fm} \AgdaSymbol{\_} \AgdaFunction{]trans} \AgdaSymbol{(}\AgdaFunction{[} \AgdaFunction{fm} \AgdaSymbol{\_} \AgdaFunction{]trans} \AgdaSymbol{(}\AgdaFunction{[} \AgdaFunction{fm} \AgdaSymbol{\_} \AgdaFunction{]sym} \AgdaSymbol{(}\AgdaFunction{trans*} \AgdaSymbol{\_))}\<%
\\
\>[0]\AgdaIndent{17}{}\<[17]%
\>[17]\AgdaSymbol{(}\AgdaFunction{subst*} \AgdaSymbol{\_} \AgdaBound{q}\AgdaSymbol{))} \AgdaBound{n} \AgdaSymbol{\}}\<%
\\
\>[0]\AgdaIndent{7}{}\<[7]%
\>[7]\AgdaSymbol{\}}\<%
\\
\>[0]\AgdaIndent{7}{}\<[7]%
\>[7]\AgdaKeyword{where} \<[13]%
\>[13]\<%
\\
\>[7]\AgdaIndent{9}{}\<[9]%
\>[9]\AgdaKeyword{open} \AgdaModule{HSetoid} \AgdaBound{Γ}\<%
\\
\>[7]\AgdaIndent{9}{}\<[9]%
\>[9]\AgdaKeyword{open} \AgdaModule{Ty} \AgdaBound{A} \<[23]%
\>[23]\<%
\\
%
\\
\>\AgdaKeyword{infixl} \AgdaNumber{5} \_\&\_\<%
\\
%
\\
\>\AgdaFunction{fst\&} \AgdaSymbol{:} \AgdaSymbol{\{}\AgdaBound{Γ} \AgdaSymbol{:} \AgdaFunction{Con}\AgdaSymbol{\}\{}\AgdaBound{A} \AgdaSymbol{:} \AgdaRecord{Ty} \AgdaBound{Γ}\AgdaSymbol{\}} \AgdaSymbol{→} \AgdaBound{Γ} \AgdaFunction{\&} \AgdaBound{A} \AgdaRecord{⇉} \AgdaBound{Γ}\<%
\\
\>\AgdaFunction{fst\&} \AgdaSymbol{=} \AgdaKeyword{record} \<[14]%
\>[14]\<%
\\
\>[7]\AgdaIndent{9}{}\<[9]%
\>[9]\AgdaSymbol{\{} \AgdaField{fn} \AgdaSymbol{=} \AgdaFunction{proj₁}\<%
\\
\>[7]\AgdaIndent{9}{}\<[9]%
\>[9]\AgdaSymbol{;} \AgdaField{resp} \AgdaSymbol{=} \AgdaFunction{proj₁}\<%
\\
\>[7]\AgdaIndent{9}{}\<[9]%
\>[9]\AgdaSymbol{\}}\<%
\\
%
\\
\>\<\end{code}

Pairing operation

\begin{code}\>\<%
\\
%
\\
\>\AgdaFunction{\_,,\_} \AgdaSymbol{:} \AgdaSymbol{\{}\AgdaBound{Γ} \AgdaBound{Δ} \AgdaSymbol{:} \AgdaFunction{Con}\AgdaSymbol{\}\{}\AgdaBound{A} \AgdaSymbol{:} \AgdaRecord{Ty} \AgdaBound{Δ}\AgdaSymbol{\}(}\AgdaBound{f} \AgdaSymbol{:} \AgdaBound{Γ} \AgdaRecord{⇉} \AgdaBound{Δ}\AgdaSymbol{)} \AgdaSymbol{→} \AgdaSymbol{(}\AgdaRecord{Tm} \AgdaSymbol{(}\AgdaBound{A} \AgdaFunction{[} \AgdaBound{f} \AgdaFunction{]T}\AgdaSymbol{))} \AgdaSymbol{→} \AgdaBound{Γ} \AgdaRecord{⇉} \AgdaSymbol{(}\AgdaBound{Δ} \AgdaFunction{\&} \AgdaBound{A}\AgdaSymbol{)}\<%
\\
\>\AgdaBound{f} \AgdaFunction{,,} \AgdaBound{t} \AgdaSymbol{=} \AgdaKeyword{record} \<[16]%
\>[16]\<%
\\
\>[7]\AgdaIndent{9}{}\<[9]%
\>[9]\AgdaSymbol{\{} \AgdaField{fn} \AgdaSymbol{=} \AgdaFunction{⟨} \AgdaFunction{[} \AgdaBound{f} \AgdaFunction{]fn} \AgdaFunction{,} \AgdaFunction{[} \AgdaBound{t} \AgdaFunction{]tm} \AgdaFunction{⟩}\<%
\\
\>[7]\AgdaIndent{9}{}\<[9]%
\>[9]\AgdaSymbol{;} \AgdaField{resp} \AgdaSymbol{=} \AgdaFunction{⟨} \AgdaFunction{[} \AgdaBound{f} \AgdaFunction{]resp} \AgdaFunction{,} \AgdaFunction{[} \AgdaBound{t} \AgdaFunction{]respt} \AgdaFunction{⟩}\<%
\\
\>[7]\AgdaIndent{9}{}\<[9]%
\>[9]\AgdaSymbol{\}}\<%
\\
%
\\
\>\<\end{code}

Projections

\begin{code}\>\<%
\\
%
\\
\>\AgdaFunction{fst} \AgdaSymbol{:} \AgdaSymbol{\{}\AgdaBound{Γ} \AgdaBound{Δ} \AgdaSymbol{:} \AgdaFunction{Con}\AgdaSymbol{\}\{}\AgdaBound{A} \AgdaSymbol{:} \AgdaRecord{Ty} \AgdaBound{Δ}\AgdaSymbol{\}} \AgdaSymbol{→} \AgdaBound{Γ} \AgdaRecord{⇉} \AgdaSymbol{(}\AgdaBound{Δ} \AgdaFunction{\&} \AgdaBound{A}\AgdaSymbol{)} \AgdaSymbol{→} \AgdaBound{Γ} \AgdaRecord{⇉} \AgdaBound{Δ}\<%
\\
\>\AgdaFunction{fst} \AgdaBound{f} \AgdaSymbol{=} \AgdaKeyword{record} \<[15]%
\>[15]\<%
\\
\>[-6]\AgdaIndent{8}{}\<[8]%
\>[8]\AgdaSymbol{\{} \AgdaField{fn} \AgdaSymbol{=} \AgdaFunction{proj₁} \AgdaFunction{∘} \AgdaFunction{[} \AgdaBound{f} \AgdaFunction{]fn}\<%
\\
\>[0]\AgdaIndent{8}{}\<[8]%
\>[8]\AgdaSymbol{;} \AgdaField{resp} \AgdaSymbol{=} \AgdaFunction{proj₁} \AgdaFunction{∘} \AgdaFunction{[} \AgdaBound{f} \AgdaFunction{]resp} \<[35]%
\>[35]\<%
\\
\>[0]\AgdaIndent{8}{}\<[8]%
\>[8]\AgdaSymbol{\}}\<%
\\
%
\\
%
\\
\>\AgdaFunction{snd} \AgdaSymbol{:} \AgdaSymbol{\{}\AgdaBound{Γ} \AgdaBound{Δ} \AgdaSymbol{:} \AgdaFunction{Con}\AgdaSymbol{\}\{}\AgdaBound{A} \AgdaSymbol{:} \AgdaRecord{Ty} \AgdaBound{Δ}\AgdaSymbol{\}} \AgdaSymbol{→} \AgdaSymbol{(}\AgdaBound{f} \AgdaSymbol{:} \AgdaBound{Γ} \AgdaRecord{⇉} \AgdaSymbol{(}\AgdaBound{Δ} \AgdaFunction{\&} \AgdaBound{A}\AgdaSymbol{))} \AgdaSymbol{→} \AgdaRecord{Tm} \AgdaSymbol{(}\AgdaBound{A} \AgdaFunction{[} \AgdaFunction{fst} \AgdaSymbol{\{}A \AgdaSymbol{=} \AgdaBound{A}\AgdaSymbol{\}} \AgdaBound{f} \AgdaFunction{]T}\AgdaSymbol{)}\<%
\\
\>\AgdaFunction{snd} \AgdaBound{f} \AgdaSymbol{=} \AgdaKeyword{record} \<[15]%
\>[15]\<%
\\
\>[0]\AgdaIndent{8}{}\<[8]%
\>[8]\AgdaSymbol{\{} \AgdaField{tm} \AgdaSymbol{=} \AgdaFunction{proj₂} \AgdaFunction{∘} \AgdaFunction{[} \AgdaBound{f} \AgdaFunction{]fn}\<%
\\
\>[0]\AgdaIndent{8}{}\<[8]%
\>[8]\AgdaSymbol{;} \AgdaField{respt} \AgdaSymbol{=} \AgdaFunction{proj₂} \AgdaFunction{∘} \AgdaFunction{[} \AgdaBound{f} \AgdaFunction{]resp} \<[36]%
\>[36]\<%
\\
\>[0]\AgdaIndent{8}{}\<[8]%
\>[8]\AgdaSymbol{\}}\<%
\\
%
\\
%
\\
\>\AgdaFunction{\_\textasciicircum\_} \AgdaSymbol{:} \AgdaSymbol{\{}\AgdaBound{Γ} \AgdaBound{Δ} \AgdaSymbol{:} \AgdaFunction{Con}\AgdaSymbol{\}(}\AgdaBound{f} \AgdaSymbol{:} \AgdaBound{Γ} \AgdaRecord{⇉} \AgdaBound{Δ}\AgdaSymbol{)(}\AgdaBound{A} \AgdaSymbol{:} \AgdaRecord{Ty} \AgdaBound{Δ}\AgdaSymbol{)} \AgdaSymbol{→} \AgdaBound{Γ} \AgdaFunction{\&} \AgdaBound{A} \AgdaFunction{[} \AgdaBound{f} \AgdaFunction{]T} \AgdaRecord{⇉} \AgdaBound{Δ} \AgdaFunction{\&} \AgdaBound{A}\<%
\\
\>\AgdaBound{f} \AgdaFunction{\textasciicircum} \AgdaBound{A} \AgdaSymbol{=} \AgdaKeyword{record} \<[15]%
\>[15]\<%
\\
\>[0]\AgdaIndent{8}{}\<[8]%
\>[8]\AgdaSymbol{\{} \AgdaField{fn} \AgdaSymbol{=} \AgdaFunction{⟨} \AgdaFunction{[} \AgdaBound{f} \AgdaFunction{]fn} \AgdaFunction{∘} \AgdaFunction{proj₁} \AgdaFunction{,} \AgdaFunction{proj₂} \AgdaFunction{⟩}\<%
\\
\>[0]\AgdaIndent{8}{}\<[8]%
\>[8]\AgdaSymbol{;} \AgdaField{resp} \AgdaSymbol{=} \AgdaFunction{⟨} \AgdaFunction{[} \AgdaBound{f} \AgdaFunction{]resp} \AgdaFunction{∘} \AgdaFunction{proj₁} \AgdaFunction{,} \AgdaFunction{proj₂} \AgdaFunction{⟩}\<%
\\
\>[0]\AgdaIndent{8}{}\<[8]%
\>[8]\AgdaSymbol{\}}\<%
\\
%
\\
\>\<\end{code}

$\Pi$-types (object level)

\begin{code}\>\<%
\\
%
\\
%
\\
\>\AgdaFunction{Π} \AgdaSymbol{:} \AgdaSymbol{\{}\AgdaBound{Γ} \AgdaSymbol{:} \AgdaFunction{Con}\AgdaSymbol{\}(}\AgdaBound{A} \AgdaSymbol{:} \AgdaRecord{Ty} \AgdaBound{Γ}\AgdaSymbol{)(}\AgdaBound{B} \AgdaSymbol{:} \AgdaRecord{Ty} \AgdaSymbol{(}\AgdaBound{Γ} \AgdaFunction{\&} \AgdaBound{A}\AgdaSymbol{))} \AgdaSymbol{→} \AgdaRecord{Ty} \AgdaBound{Γ}\<%
\\
\>\AgdaFunction{Π} \AgdaSymbol{\{}\AgdaBound{Γ}\AgdaSymbol{\}} \AgdaBound{A} \AgdaBound{B} \AgdaSymbol{=} \AgdaKeyword{record} \<[19]%
\>[19]\<%
\\
\>[0]\AgdaIndent{2}{}\<[2]%
\>[2]\AgdaSymbol{\{} \AgdaField{fm} \AgdaSymbol{=} \AgdaSymbol{λ} \AgdaBound{x} \AgdaSymbol{→} \AgdaKeyword{let} \AgdaBound{Ax} \AgdaSymbol{=} \AgdaFunction{[} \AgdaBound{A} \AgdaFunction{]fm} \AgdaBound{x} \AgdaKeyword{in}\<%
\\
\>[0]\AgdaIndent{15}{}\<[15]%
\>[15]\AgdaKeyword{let} \AgdaBound{Bx} \AgdaSymbol{=} \AgdaSymbol{λ} \AgdaBound{a} \AgdaSymbol{→} \AgdaFunction{[} \AgdaBound{B} \AgdaFunction{]fm} \AgdaSymbol{(}\AgdaBound{x} \AgdaInductiveConstructor{,} \AgdaBound{a}\AgdaSymbol{)} \AgdaKeyword{in}\<%
\\
\>[0]\AgdaIndent{9}{}\<[9]%
\>[9]\AgdaKeyword{record}\<%
\\
\>[0]\AgdaIndent{9}{}\<[9]%
\>[9]\AgdaSymbol{\{} \AgdaField{Carrier} \AgdaSymbol{=} \AgdaRecord{Σ[} \AgdaBound{fn} \AgdaRecord{∶} \AgdaSymbol{((}\AgdaBound{a} \AgdaSymbol{:} \AgdaFunction{∣} \AgdaBound{Ax} \AgdaFunction{∣}\AgdaSymbol{)} \AgdaSymbol{→} \AgdaFunction{∣} \AgdaBound{Bx} \AgdaBound{a} \AgdaFunction{∣}\AgdaSymbol{)} \AgdaRecord{]}\<%
\\
\>[9]\AgdaIndent{21}{}\<[21]%
\>[21]\AgdaSymbol{((}\AgdaBound{a} \AgdaBound{b} \AgdaSymbol{:} \AgdaFunction{∣} \AgdaBound{Ax} \AgdaFunction{∣}\AgdaSymbol{)}\<%
\\
\>[9]\AgdaIndent{21}{}\<[21]%
\>[21]\AgdaSymbol{(}\AgdaBound{p} \AgdaSymbol{:} \AgdaFunction{[} \AgdaBound{Ax} \AgdaFunction{]} \AgdaBound{a} \AgdaFunction{≈} \AgdaBound{b}\AgdaSymbol{)} \AgdaSymbol{→}\<%
\\
\>[9]\AgdaIndent{21}{}\<[21]%
\>[21]\AgdaFunction{[} \AgdaBound{Bx} \AgdaBound{b} \AgdaFunction{]} \AgdaFunction{[} \AgdaBound{B} \AgdaFunction{]subst} \AgdaSymbol{(}\AgdaFunction{[} \AgdaBound{Γ} \AgdaFunction{]refl} \AgdaInductiveConstructor{,} \<[54]%
\>[54]\<%
\\
\>[9]\AgdaIndent{21}{}\<[21]%
\>[21]\AgdaFunction{[} \AgdaBound{Ax} \AgdaFunction{]trans} \AgdaSymbol{(}\AgdaFunction{[} \AgdaBound{A} \AgdaFunction{]refl*} \AgdaBound{x} \AgdaBound{a}\AgdaSymbol{)} \AgdaBound{p}\AgdaSymbol{)} \AgdaSymbol{(}\AgdaBound{fn} \AgdaBound{a}\AgdaSymbol{)} \AgdaFunction{≈} \AgdaBound{fn} \AgdaBound{b}\AgdaSymbol{)} \<[68]%
\>[68]\<%
\\
\>[0]\AgdaIndent{9}{}\<[9]%
\>[9]\AgdaSymbol{;} \AgdaField{\_≈h\_} \<[19]%
\>[19]\AgdaSymbol{=} \AgdaSymbol{λ\{(}\AgdaBound{f} \AgdaInductiveConstructor{,} \AgdaSymbol{\_)} \AgdaSymbol{(}\AgdaBound{g} \AgdaInductiveConstructor{,} \AgdaSymbol{\_)} \AgdaSymbol{→}\<%
\\
\>[0]\AgdaIndent{21}{}\<[21]%
\>[21]\AgdaFunction{∀'[} \AgdaBound{a} \AgdaFunction{∶} \AgdaSymbol{\_} \AgdaFunction{]} \AgdaFunction{[} \AgdaBound{Bx} \AgdaBound{a} \AgdaFunction{]} \AgdaBound{f} \AgdaBound{a} \AgdaFunction{≈h} \AgdaBound{g} \AgdaBound{a} \AgdaSymbol{\}}\<%
\\
\>[0]\AgdaIndent{9}{}\<[9]%
\>[9]\AgdaSymbol{;} \AgdaField{refl} \<[19]%
\>[19]\AgdaSymbol{=} \AgdaSymbol{λ} \AgdaBound{a} \AgdaSymbol{→} \AgdaFunction{[} \AgdaBound{Bx} \AgdaBound{a} \AgdaFunction{]refl} \<[40]%
\>[40]\<%
\\
\>[0]\AgdaIndent{9}{}\<[9]%
\>[9]\AgdaSymbol{;} \AgdaField{sym} \<[19]%
\>[19]\AgdaSymbol{=} \AgdaSymbol{λ} \AgdaBound{f} \AgdaBound{a} \AgdaSymbol{→} \AgdaFunction{[} \AgdaBound{Bx} \AgdaBound{a} \AgdaFunction{]sym} \AgdaSymbol{(}\AgdaBound{f} \AgdaBound{a}\AgdaSymbol{)}\<%
\\
\>[0]\AgdaIndent{9}{}\<[9]%
\>[9]\AgdaSymbol{;} \AgdaField{trans} \<[19]%
\>[19]\AgdaSymbol{=} \AgdaSymbol{λ} \AgdaBound{f} \AgdaBound{g} \AgdaBound{a} \AgdaSymbol{→} \AgdaFunction{[} \AgdaBound{Bx} \AgdaBound{a} \AgdaFunction{]trans} \AgdaSymbol{(}\AgdaBound{f} \AgdaBound{a}\AgdaSymbol{)} \AgdaSymbol{(}\AgdaBound{g} \AgdaBound{a}\AgdaSymbol{)}\<%
\\
\>[0]\AgdaIndent{9}{}\<[9]%
\>[9]\AgdaSymbol{\}}\<%
\\
%
\\
\>[0]\AgdaIndent{2}{}\<[2]%
\>[2]\AgdaSymbol{;} \AgdaField{substT} \AgdaSymbol{=} \AgdaSymbol{λ} \AgdaSymbol{\{}\AgdaBound{x}\AgdaSymbol{\}} \AgdaSymbol{\{}\AgdaBound{y}\AgdaSymbol{\}} \AgdaBound{p} \<[25]%
\>[25]\<%
\\
\>[0]\AgdaIndent{15}{}\<[15]%
\>[15]\AgdaSymbol{→} \AgdaKeyword{let} \AgdaBound{y2x} \AgdaSymbol{=} \AgdaSymbol{λ} \AgdaBound{a} \AgdaSymbol{→} \AgdaFunction{[} \AgdaBound{A} \AgdaFunction{]subst} \AgdaSymbol{(}\AgdaFunction{[} \AgdaBound{Γ} \AgdaFunction{]sym} \AgdaBound{p}\AgdaSymbol{)} \AgdaBound{a} \AgdaKeyword{in}\<%
\\
\>[15]\AgdaIndent{17}{}\<[17]%
\>[17]\AgdaKeyword{let} \AgdaBound{x2y} \AgdaSymbol{=} \AgdaSymbol{λ} \AgdaBound{a} \AgdaSymbol{→} \AgdaFunction{[} \AgdaBound{A} \AgdaFunction{]subst} \AgdaBound{p} \AgdaBound{a} \AgdaKeyword{in}\<%
\\
\>[15]\AgdaIndent{17}{}\<[17]%
\>[17]\AgdaKeyword{let} \AgdaBound{p'} \AgdaSymbol{=} \AgdaSymbol{λ} \AgdaBound{a} \AgdaSymbol{→} \AgdaFunction{[} \AgdaBound{A} \AgdaFunction{]trans-refl} \AgdaKeyword{in}\<%
\\
\>[-11]\AgdaIndent{13}{}\<[13]%
\>[13]\AgdaSymbol{λ\{(}\AgdaBound{f} \AgdaInductiveConstructor{,} \AgdaBound{rsp}\AgdaSymbol{)} \AgdaSymbol{→} \<[28]%
\>[28]\<%
\\
\>[0]\AgdaIndent{15}{}\<[15]%
\>[15]\AgdaSymbol{(λ} \AgdaBound{a} \AgdaSymbol{→} \AgdaFunction{[} \AgdaBound{B} \AgdaFunction{]subst} \AgdaSymbol{(}\AgdaBound{p} \AgdaInductiveConstructor{,} \AgdaBound{p'} \AgdaBound{a}\AgdaSymbol{)} \AgdaSymbol{(}\AgdaBound{f} \AgdaSymbol{(}\AgdaBound{y2x} \AgdaBound{a}\AgdaSymbol{)))}\<%
\\
\>[0]\AgdaIndent{15}{}\<[15]%
\>[15]\AgdaInductiveConstructor{,} \<[49]%
\>[49]\<%
\\
\>[0]\AgdaIndent{15}{}\<[15]%
\>[15]\AgdaSymbol{(λ} \AgdaBound{a} \AgdaBound{b} \AgdaBound{q} \AgdaSymbol{→} \<[26]%
\>[26]\<%
\\
\>[15]\AgdaIndent{16}{}\<[16]%
\>[16]\AgdaKeyword{let} \AgdaBound{a'} \AgdaSymbol{=} \AgdaBound{y2x} \AgdaBound{a} \AgdaKeyword{in} \<[34]%
\>[34]\<%
\\
\>[15]\AgdaIndent{16}{}\<[16]%
\>[16]\AgdaKeyword{let} \AgdaBound{b'} \AgdaSymbol{=} \AgdaBound{y2x} \AgdaBound{b} \AgdaKeyword{in}\<%
\\
\>[15]\AgdaIndent{16}{}\<[16]%
\>[16]\AgdaKeyword{let} \AgdaBound{q'} \AgdaSymbol{=} \AgdaFunction{[} \AgdaBound{A} \AgdaFunction{]subst*} \AgdaSymbol{(}\AgdaFunction{[} \AgdaBound{Γ} \AgdaFunction{]sym} \AgdaBound{p}\AgdaSymbol{)} \AgdaBound{q} \AgdaKeyword{in}\<%
\\
\>[15]\AgdaIndent{16}{}\<[16]%
\>[16]\AgdaKeyword{let} \AgdaBound{H} \AgdaSymbol{=} \AgdaBound{rsp} \AgdaBound{a'} \AgdaBound{b'} \AgdaBound{q'} \AgdaKeyword{in}\<%
\\
\>[15]\AgdaIndent{16}{}\<[16]%
\>[16]\AgdaKeyword{let} \AgdaBound{r} \AgdaSymbol{:} \AgdaFunction{[} \AgdaBound{Γ} \AgdaFunction{\&} \AgdaBound{A} \AgdaFunction{]} \AgdaSymbol{(}\AgdaBound{x} \AgdaInductiveConstructor{,} \AgdaBound{b'}\AgdaSymbol{)} \AgdaFunction{≈} \AgdaSymbol{(}\AgdaBound{y} \AgdaInductiveConstructor{,} \AgdaBound{b}\AgdaSymbol{)}
                    r \AgdaSymbol{=} \AgdaSymbol{(}\AgdaBound{p} \AgdaInductiveConstructor{,} \AgdaBound{p'} \AgdaBound{b}\AgdaSymbol{)} \AgdaKeyword{in}\<%
\\
\>[15]\AgdaIndent{16}{}\<[16]%
\>[16]\AgdaKeyword{let} \AgdaBound{pre} \AgdaSymbol{=} \AgdaFunction{[} \AgdaBound{B} \AgdaFunction{]subst*} \AgdaBound{r} \AgdaBound{H} \AgdaKeyword{in}\<%
\\
\>[15]\AgdaIndent{16}{}\<[16]%
\>[16]\<%
\\
\>[15]\AgdaIndent{16}{}\<[16]%
\>[16]\AgdaFunction{[} \AgdaFunction{[} \AgdaBound{B} \AgdaFunction{]fm} \AgdaSymbol{(}\AgdaBound{y} \AgdaInductiveConstructor{,} \AgdaBound{b}\AgdaSymbol{)} \AgdaFunction{]trans} \<[41]%
\>[41]\<%
\\
\>[15]\AgdaIndent{16}{}\<[16]%
\>[16]\AgdaSymbol{(}\AgdaFunction{[} \AgdaBound{B} \AgdaFunction{]trans*} \AgdaSymbol{\_)} \<[48]%
\>[48]\<%
\\
\>[15]\AgdaIndent{16}{}\<[16]%
\>[16]\AgdaSymbol{(}\AgdaFunction{[} \AgdaFunction{[} \AgdaBound{B} \AgdaFunction{]fm} \AgdaSymbol{(}\AgdaBound{y} \AgdaInductiveConstructor{,} \AgdaBound{b}\AgdaSymbol{)} \AgdaFunction{]trans} \<[42]%
\>[42]\<%
\\
\>[15]\AgdaIndent{16}{}\<[16]%
\>[16]\AgdaFunction{[} \AgdaBound{B} \AgdaFunction{]subst-pi} \<[30]%
\>[30]\<%
\\
\>[15]\AgdaIndent{16}{}\<[16]%
\>[16]\AgdaSymbol{(}\AgdaFunction{[} \AgdaFunction{[} \AgdaBound{B} \AgdaFunction{]fm} \AgdaSymbol{(}\AgdaBound{y} \AgdaInductiveConstructor{,} \AgdaBound{b}\AgdaSymbol{)} \AgdaFunction{]trans} \<[42]%
\>[42]\<%
\\
\>[15]\AgdaIndent{16}{}\<[16]%
\>[16]\AgdaSymbol{(}\AgdaFunction{[} \AgdaFunction{[} \AgdaBound{B} \AgdaFunction{]fm} \AgdaSymbol{(}\AgdaBound{y} \AgdaInductiveConstructor{,} \AgdaBound{b}\AgdaSymbol{)} \AgdaFunction{]sym} \AgdaSymbol{(}\AgdaFunction{[} \AgdaBound{B} \AgdaFunction{]trans*} \AgdaSymbol{\_))} \<[57]%
\>[57]\<%
\\
\>[15]\AgdaIndent{16}{}\<[16]%
\>[16]\AgdaBound{pre}\AgdaSymbol{))} \<[23]%
\>[23]\<%
\\
\>[15]\AgdaIndent{16}{}\<[16]%
\>[16]\AgdaSymbol{)} \<[22]%
\>[22]\<%
\\
\>[-12]\AgdaIndent{13}{}\<[13]%
\>[13]\AgdaSymbol{\}}\<%
\\
\>[0]\AgdaIndent{2}{}\<[2]%
\>[2]\AgdaSymbol{;} \AgdaField{subst*} \AgdaSymbol{=} \AgdaSymbol{λ} \AgdaBound{\_} \AgdaBound{q} \AgdaBound{\_} \AgdaSymbol{→} \AgdaFunction{[} \AgdaBound{B} \AgdaFunction{]subst*} \AgdaSymbol{\_} \AgdaSymbol{(}\AgdaBound{q} \AgdaSymbol{\_)}\<%
\\
\>[0]\AgdaIndent{2}{}\<[2]%
\>[2]\AgdaSymbol{;} \AgdaField{refl*} \AgdaSymbol{=} \AgdaSymbol{λ} \AgdaSymbol{\{}\AgdaBound{x} \AgdaSymbol{(}\AgdaBound{f} \AgdaInductiveConstructor{,} \AgdaBound{rsp}\AgdaSymbol{)} \AgdaBound{a} \<[29]%
\>[29]\<%
\\
\>[2]\AgdaIndent{12}{}\<[12]%
\>[12]\AgdaSymbol{→} \AgdaFunction{[} \AgdaFunction{[} \AgdaBound{B} \AgdaFunction{]fm} \AgdaSymbol{\_} \AgdaFunction{]trans} \AgdaFunction{[} \AgdaBound{B} \AgdaFunction{]subst-pi} \<[47]%
\>[47]\<%
\\
\>[2]\AgdaIndent{12}{}\<[12]%
\>[12]\AgdaSymbol{(}\AgdaBound{rsp} \AgdaSymbol{(}\AgdaFunction{[} \AgdaBound{A} \AgdaFunction{]subst} \AgdaSymbol{(}\AgdaFunction{[} \AgdaBound{Γ} \AgdaFunction{]sym} \AgdaFunction{[} \AgdaBound{Γ} \AgdaFunction{]refl}\AgdaSymbol{)} \AgdaBound{a}\AgdaSymbol{)} \AgdaBound{a} \AgdaFunction{[} \AgdaBound{A} \AgdaFunction{]subst-pi'}\AgdaSymbol{)} \<[72]%
\>[72]\AgdaSymbol{\}}\<%
\\
\>[0]\AgdaIndent{2}{}\<[2]%
\>[2]\AgdaSymbol{;} \AgdaField{trans*} \AgdaSymbol{=} \AgdaSymbol{λ} \AgdaSymbol{\{(}\AgdaBound{f} \AgdaInductiveConstructor{,} \AgdaBound{rsp}\AgdaSymbol{)} \AgdaBound{a} \AgdaSymbol{→}\<%
\\
\>[0]\AgdaIndent{13}{}\<[13]%
\>[13]\AgdaFunction{[} \AgdaFunction{[} \AgdaBound{B} \AgdaFunction{]fm} \AgdaSymbol{\_} \AgdaFunction{]trans} \<[32]%
\>[32]\<%
\\
\>[0]\AgdaIndent{13}{}\<[13]%
\>[13]\AgdaSymbol{(}\AgdaFunction{[} \AgdaFunction{[} \AgdaBound{B} \AgdaFunction{]fm} \AgdaSymbol{\_} \AgdaFunction{]trans} \<[33]%
\>[33]\<%
\\
\>[0]\AgdaIndent{13}{}\<[13]%
\>[13]\AgdaSymbol{(}\AgdaFunction{[} \AgdaBound{B} \AgdaFunction{]trans*} \AgdaSymbol{\_)} \<[29]%
\>[29]\<%
\\
\>[0]\AgdaIndent{13}{}\<[13]%
\>[13]\AgdaSymbol{(}\AgdaFunction{[} \AgdaFunction{[} \AgdaBound{B} \AgdaFunction{]fm} \AgdaSymbol{\_} \AgdaFunction{]sym} \AgdaSymbol{(}\AgdaFunction{[} \AgdaFunction{[} \AgdaBound{B} \AgdaFunction{]fm} \AgdaSymbol{\_} \AgdaFunction{]trans} \<[51]%
\>[51]\<%
\\
\>[0]\AgdaIndent{13}{}\<[13]%
\>[13]\AgdaSymbol{(}\AgdaFunction{[} \AgdaBound{B} \AgdaFunction{]trans*} \AgdaSymbol{\_)} \AgdaFunction{[} \AgdaBound{B} \AgdaFunction{]subst-pi}\AgdaSymbol{)))} \<[46]%
\>[46]\<%
\\
\>[0]\AgdaIndent{13}{}\<[13]%
\>[13]\AgdaSymbol{(}\AgdaFunction{[} \AgdaBound{B} \AgdaFunction{]subst*} \AgdaSymbol{\_} \AgdaSymbol{(}\AgdaBound{rsp} \AgdaSymbol{\_} \AgdaSymbol{\_} \AgdaSymbol{(}\AgdaFunction{[} \AgdaFunction{[} \AgdaBound{A} \AgdaFunction{]fm} \AgdaSymbol{\_} \AgdaFunction{]trans} \<[57]%
\>[57]\<%
\\
\>[0]\AgdaIndent{13}{}\<[13]%
\>[13]\AgdaSymbol{(}\AgdaFunction{[} \AgdaBound{A} \AgdaFunction{]trans*} \AgdaSymbol{\_)} \AgdaFunction{[} \AgdaBound{A} \AgdaFunction{]subst-pi}\AgdaSymbol{)))} \AgdaSymbol{\}} \<[48]%
\>[48]\<%
\\
\>[0]\AgdaIndent{2}{}\<[2]%
\>[2]\AgdaSymbol{\}}\<%
\\
%
\\
\>\AgdaFunction{lam} \AgdaSymbol{:} \AgdaSymbol{\{}\AgdaBound{Γ} \AgdaSymbol{:} \AgdaFunction{Con}\AgdaSymbol{\}\{}\AgdaBound{A} \AgdaSymbol{:} \AgdaRecord{Ty} \AgdaBound{Γ}\AgdaSymbol{\}\{}\AgdaBound{B} \AgdaSymbol{:} \AgdaRecord{Ty} \AgdaSymbol{(}\AgdaBound{Γ} \AgdaFunction{\&} \AgdaBound{A}\AgdaSymbol{)\}} \AgdaSymbol{→} \AgdaRecord{Tm} \AgdaBound{B} \AgdaSymbol{→} \AgdaRecord{Tm} \AgdaSymbol{(}\AgdaFunction{Π} \AgdaBound{A} \AgdaBound{B}\AgdaSymbol{)}\<%
\\
\>\AgdaFunction{lam} \AgdaSymbol{\{}\AgdaBound{Γ}\AgdaSymbol{\}} \AgdaSymbol{\{}\AgdaBound{A}\AgdaSymbol{\}} \AgdaSymbol{(}\AgdaInductiveConstructor{tm:} \AgdaBound{tm} \AgdaInductiveConstructor{resp:} \AgdaBound{respt}\AgdaSymbol{)} \AgdaSymbol{=} \<[35]%
\>[35]\<%
\\
\>[0]\AgdaIndent{2}{}\<[2]%
\>[2]\AgdaKeyword{record} \AgdaSymbol{\{} \AgdaField{tm} \AgdaSymbol{=} \AgdaSymbol{λ} \AgdaBound{x} \AgdaSymbol{→} \AgdaSymbol{(λ} \AgdaBound{a} \AgdaSymbol{→} \AgdaBound{tm} \AgdaSymbol{(}\AgdaBound{x} \AgdaInductiveConstructor{,} \AgdaBound{a}\AgdaSymbol{))} \<[41]%
\>[41]\<%
\\
\>[2]\AgdaIndent{16}{}\<[16]%
\>[16]\AgdaInductiveConstructor{,} \AgdaSymbol{(λ} \AgdaBound{a} \AgdaBound{b} \AgdaBound{p} \AgdaSymbol{→} \AgdaBound{respt} \AgdaSymbol{(}\AgdaFunction{[} \AgdaBound{Γ} \AgdaFunction{]refl} \AgdaInductiveConstructor{,} \<[48]%
\>[48]\<%
\\
\>[2]\AgdaIndent{16}{}\<[16]%
\>[16]\AgdaFunction{[} \AgdaFunction{[} \AgdaBound{A} \AgdaFunction{]fm} \AgdaBound{x} \AgdaFunction{]trans} \AgdaSymbol{(}\AgdaFunction{[} \AgdaBound{A} \AgdaFunction{]refl*} \AgdaSymbol{\_} \AgdaSymbol{\_)} \AgdaBound{p}\AgdaSymbol{))}\<%
\\
\>[5]\AgdaIndent{9}{}\<[9]%
\>[9]\AgdaSymbol{;} \AgdaField{respt} \AgdaSymbol{=} \AgdaSymbol{λ} \AgdaBound{p} \AgdaBound{\_} \AgdaSymbol{→} \AgdaBound{respt} \AgdaSymbol{(}\AgdaBound{p} \AgdaInductiveConstructor{,} \AgdaFunction{[} \AgdaBound{A} \AgdaFunction{]trans-refl}\AgdaSymbol{)} \<[55]%
\>[55]\<%
\\
\>[0]\AgdaIndent{9}{}\<[9]%
\>[9]\AgdaSymbol{\}}\<%
\\
%
\\
\>\AgdaFunction{app} \AgdaSymbol{:} \AgdaSymbol{\{}\AgdaBound{Γ} \AgdaSymbol{:} \AgdaFunction{Con}\AgdaSymbol{\}\{}\AgdaBound{A} \AgdaSymbol{:} \AgdaRecord{Ty} \AgdaBound{Γ}\AgdaSymbol{\}\{}\AgdaBound{B} \AgdaSymbol{:} \AgdaRecord{Ty} \AgdaSymbol{(}\AgdaBound{Γ} \AgdaFunction{\&} \AgdaBound{A}\AgdaSymbol{)\}} \AgdaSymbol{→} \AgdaRecord{Tm} \AgdaSymbol{(}\AgdaFunction{Π} \AgdaBound{A} \AgdaBound{B}\AgdaSymbol{)} \AgdaSymbol{→} \AgdaRecord{Tm} \AgdaBound{B}\<%
\\
\>\AgdaFunction{app} \AgdaSymbol{\{}\AgdaBound{Γ}\AgdaSymbol{\}} \AgdaSymbol{\{}\AgdaBound{A}\AgdaSymbol{\}} \AgdaSymbol{\{}\AgdaBound{B}\AgdaSymbol{\}} \AgdaSymbol{(}\AgdaInductiveConstructor{tm:} \AgdaBound{tm} \AgdaInductiveConstructor{resp:} \AgdaBound{respt}\AgdaSymbol{)} \AgdaSymbol{=} \<[39]%
\>[39]\<%
\\
\>[0]\AgdaIndent{2}{}\<[2]%
\>[2]\AgdaKeyword{record} \AgdaSymbol{\{} \AgdaField{tm} \AgdaSymbol{=} \AgdaSymbol{λ} \AgdaSymbol{\{(}\AgdaBound{x} \AgdaInductiveConstructor{,} \AgdaBound{a}\AgdaSymbol{)} \AgdaSymbol{→} \AgdaFunction{proj₁} \AgdaSymbol{(}\AgdaBound{tm} \AgdaBound{x}\AgdaSymbol{)} \AgdaBound{a}\AgdaSymbol{\}}\<%
\\
\>[0]\AgdaIndent{9}{}\<[9]%
\>[9]\AgdaSymbol{;} \AgdaField{respt} \AgdaSymbol{=} \AgdaSymbol{λ} \AgdaSymbol{\{}\AgdaBound{x}\AgdaSymbol{\}} \AgdaSymbol{\{}\AgdaBound{y}\AgdaSymbol{\}} \AgdaSymbol{→} \AgdaSymbol{λ} \AgdaSymbol{\{(}\AgdaBound{p} \AgdaInductiveConstructor{,} \AgdaBound{tr}\AgdaSymbol{)} \AgdaSymbol{→} \<[45]%
\>[45]\<%
\\
\>[9]\AgdaIndent{17}{}\<[17]%
\>[17]\AgdaKeyword{let} \AgdaBound{fresp} \AgdaSymbol{=} \AgdaFunction{proj₂} \AgdaSymbol{(}\AgdaBound{tm} \AgdaSymbol{(}\AgdaFunction{proj₁} \AgdaBound{x}\AgdaSymbol{))} \AgdaKeyword{in}\<%
\\
\>[9]\AgdaIndent{17}{}\<[17]%
\>[17]\AgdaFunction{[} \AgdaFunction{[} \AgdaBound{B} \AgdaFunction{]fm} \AgdaSymbol{\_} \AgdaFunction{]trans} \<[36]%
\>[36]\<%
\\
\>[9]\AgdaIndent{17}{}\<[17]%
\>[17]\AgdaSymbol{(}\AgdaFunction{[} \AgdaBound{B} \AgdaFunction{]subst*} \AgdaSymbol{(}\AgdaBound{p} \AgdaInductiveConstructor{,} \AgdaBound{tr}\AgdaSymbol{)} \AgdaSymbol{(}\AgdaFunction{[} \AgdaFunction{[} \AgdaBound{B} \AgdaFunction{]fm} \AgdaSymbol{\_} \AgdaFunction{]sym} \AgdaFunction{[} \AgdaBound{B} \AgdaFunction{]subst-pi'}\AgdaSymbol{))} \<[74]%
\>[74]\<%
\\
\>[9]\AgdaIndent{17}{}\<[17]%
\>[17]\AgdaSymbol{(}\AgdaFunction{[} \AgdaFunction{[} \AgdaBound{B} \AgdaFunction{]fm} \AgdaSymbol{\_} \AgdaFunction{]trans}\<%
\\
\>[9]\AgdaIndent{17}{}\<[17]%
\>[17]\AgdaSymbol{(}\AgdaFunction{[} \AgdaBound{B} \AgdaFunction{]trans*} \AgdaSymbol{\{}p \AgdaSymbol{=} \AgdaSymbol{(}\AgdaFunction{[} \AgdaBound{Γ} \AgdaFunction{]refl} \AgdaInductiveConstructor{,} \AgdaFunction{[} \AgdaBound{A} \AgdaFunction{]refl*} \AgdaSymbol{\_} \AgdaSymbol{\_)\}} \AgdaSymbol{\_)}\<%
\\
\>[9]\AgdaIndent{17}{}\<[17]%
\>[17]\AgdaSymbol{(}\AgdaFunction{[} \AgdaFunction{[} \AgdaBound{B} \AgdaFunction{]fm} \AgdaSymbol{\_} \AgdaFunction{]trans} \<[37]%
\>[37]\<%
\\
\>[9]\AgdaIndent{17}{}\<[17]%
\>[17]\AgdaFunction{[} \AgdaBound{B} \AgdaFunction{]subst-pi} \<[31]%
\>[31]\<%
\\
\>[9]\AgdaIndent{17}{}\<[17]%
\>[17]\AgdaSymbol{(}\AgdaFunction{[} \AgdaFunction{[} \AgdaBound{B} \AgdaFunction{]fm} \AgdaSymbol{\_} \AgdaFunction{]trans} \<[37]%
\>[37]\<%
\\
\>[9]\AgdaIndent{17}{}\<[17]%
\>[17]\AgdaSymbol{(}\AgdaFunction{[} \AgdaFunction{[} \AgdaBound{B} \AgdaFunction{]fm} \AgdaSymbol{\_} \AgdaFunction{]sym} \AgdaSymbol{(}\AgdaFunction{[} \AgdaBound{B} \AgdaFunction{]trans*} \AgdaSymbol{\{}q \AgdaSymbol{=} \AgdaSymbol{(}\AgdaBound{p} \AgdaInductiveConstructor{,} \AgdaFunction{[} \AgdaBound{A} \AgdaFunction{]trans-refl}\AgdaSymbol{)\}} \AgdaSymbol{\_))}\<%
\\
\>[9]\AgdaIndent{17}{}\<[17]%
\>[17]\AgdaSymbol{(}\AgdaFunction{[} \AgdaFunction{[} \AgdaBound{B} \AgdaFunction{]fm} \AgdaSymbol{\_} \AgdaFunction{]trans} \<[37]%
\>[37]\<%
\\
\>[9]\AgdaIndent{17}{}\<[17]%
\>[17]\AgdaSymbol{(}\AgdaFunction{[} \AgdaBound{B} \AgdaFunction{]subst-pi*} \AgdaSymbol{(}\AgdaBound{fresp} \AgdaSymbol{\_} \AgdaSymbol{\_} \AgdaSymbol{(}\AgdaFunction{[} \AgdaBound{A} \AgdaFunction{]subst-mir2} \AgdaBound{tr}\AgdaSymbol{)))} \<[67]%
\>[67]\<%
\\
\>[9]\AgdaIndent{17}{}\<[17]%
\>[17]\AgdaSymbol{(}\AgdaBound{respt} \AgdaBound{p} \AgdaSymbol{\_)))))} \AgdaSymbol{\}}\<%
\\
\>[0]\AgdaIndent{9}{}\<[9]%
\>[9]\AgdaSymbol{\}}\<%
\\
%
\\
\>\<\end{code}

Verification of β causes stack overflow

\begin{code}\>\<%
\\
%
\\
\>\AgdaFunction{\_⇒\_} \AgdaSymbol{:} \AgdaSymbol{\{}\AgdaBound{Γ} \AgdaSymbol{:} \AgdaFunction{Con}\AgdaSymbol{\}(}\AgdaBound{A} \AgdaBound{B} \AgdaSymbol{:} \AgdaRecord{Ty} \AgdaBound{Γ}\AgdaSymbol{)} \AgdaSymbol{→} \AgdaRecord{Ty} \AgdaBound{Γ}\<%
\\
\>\AgdaBound{A} \AgdaFunction{⇒} \AgdaBound{B} \AgdaSymbol{=} \AgdaFunction{Π} \AgdaBound{A} \AgdaSymbol{(}\AgdaBound{B} \AgdaFunction{[} \AgdaFunction{fst\&} \AgdaSymbol{\{}A \AgdaSymbol{=} \AgdaBound{A}\AgdaSymbol{\}} \AgdaFunction{]T}\AgdaSymbol{)}\<%
\\
%
\\
\>\AgdaKeyword{infixr} \AgdaNumber{6} \_⇒\_\<%
\\
\>\<\end{code}

Simpler definition for functions

\begin{code}\>\<%
\\
%
\\
\>\AgdaFunction{[\_,\_]\_⇒fm\_} \AgdaSymbol{:} \AgdaSymbol{(}\AgdaBound{Γ} \AgdaSymbol{:} \AgdaFunction{Con}\AgdaSymbol{)(}\AgdaBound{x} \AgdaSymbol{:} \AgdaFunction{∣} \AgdaBound{Γ} \AgdaFunction{∣}\AgdaSymbol{)} \AgdaSymbol{→} \AgdaRecord{HSetoid} \AgdaSymbol{→} \AgdaRecord{HSetoid} \AgdaSymbol{→} \AgdaRecord{HSetoid}\<%
\\
\>\AgdaFunction{[} \AgdaBound{Γ} \AgdaFunction{,} \AgdaBound{x} \AgdaFunction{]} \AgdaBound{Ax} \AgdaFunction{⇒fm} \AgdaBound{Bx} \<[20]%
\>[20]\<%
\\
\>[0]\AgdaIndent{2}{}\<[2]%
\>[2]\AgdaSymbol{=} \AgdaKeyword{record}\<%
\\
\>[2]\AgdaIndent{6}{}\<[6]%
\>[6]\AgdaSymbol{\{} \AgdaField{Carrier} \AgdaSymbol{=} \AgdaRecord{Σ[} \AgdaBound{fn} \AgdaRecord{∶} \AgdaSymbol{(}\AgdaFunction{∣} \AgdaBound{Ax} \AgdaFunction{∣} \AgdaSymbol{→} \AgdaFunction{∣} \AgdaBound{Bx} \AgdaFunction{∣}\AgdaSymbol{)} \AgdaRecord{]} \AgdaSymbol{((}\AgdaBound{a} \AgdaBound{b} \AgdaSymbol{:} \AgdaFunction{∣} \AgdaBound{Ax} \AgdaFunction{∣}\AgdaSymbol{)}\<%
\\
\>[6]\AgdaIndent{18}{}\<[18]%
\>[18]\AgdaSymbol{(}\AgdaBound{p} \AgdaSymbol{:} \AgdaFunction{[} \AgdaBound{Ax} \AgdaFunction{]} \AgdaBound{a} \AgdaFunction{≈} \AgdaBound{b}\AgdaSymbol{)} \AgdaSymbol{→} \AgdaFunction{[} \AgdaBound{Bx} \AgdaFunction{]} \AgdaBound{fn} \AgdaBound{a} \AgdaFunction{≈} \AgdaBound{fn} \AgdaBound{b}\AgdaSymbol{)}\<%
\\
\>[0]\AgdaIndent{6}{}\<[6]%
\>[6]\AgdaSymbol{;} \AgdaField{\_≈h\_} \<[16]%
\>[16]\AgdaSymbol{=} \AgdaSymbol{λ\{(}\AgdaBound{f} \AgdaInductiveConstructor{,} \AgdaSymbol{\_)} \AgdaSymbol{(}\AgdaBound{g} \AgdaInductiveConstructor{,} \AgdaSymbol{\_)} \AgdaSymbol{→} \AgdaFunction{∀'[} \AgdaBound{a} \AgdaFunction{∶} \AgdaSymbol{\_} \AgdaFunction{]} \AgdaFunction{[} \AgdaBound{Bx} \AgdaFunction{]} \AgdaBound{f} \AgdaBound{a} \AgdaFunction{≈h} \AgdaBound{g} \AgdaBound{a} \AgdaSymbol{\}}\<%
\\
\>[0]\AgdaIndent{6}{}\<[6]%
\>[6]\AgdaSymbol{;} \AgdaField{refl} \<[16]%
\>[16]\AgdaSymbol{=} \AgdaSymbol{λ} \AgdaBound{\_} \AgdaSymbol{→} \AgdaFunction{[} \AgdaBound{Bx} \AgdaFunction{]refl} \<[35]%
\>[35]\<%
\\
\>[0]\AgdaIndent{6}{}\<[6]%
\>[6]\AgdaSymbol{;} \AgdaField{sym} \<[16]%
\>[16]\AgdaSymbol{=} \AgdaSymbol{λ} \AgdaBound{f} \AgdaBound{a} \AgdaSymbol{→} \AgdaFunction{[} \AgdaBound{Bx} \AgdaFunction{]sym} \AgdaSymbol{(}\AgdaBound{f} \AgdaBound{a}\AgdaSymbol{)}\<%
\\
\>[0]\AgdaIndent{6}{}\<[6]%
\>[6]\AgdaSymbol{;} \AgdaField{trans} \<[16]%
\>[16]\AgdaSymbol{=} \AgdaSymbol{λ} \AgdaBound{f} \AgdaBound{g} \AgdaBound{a} \AgdaSymbol{→} \AgdaFunction{[} \AgdaBound{Bx} \AgdaFunction{]trans} \AgdaSymbol{(}\AgdaBound{f} \AgdaBound{a}\AgdaSymbol{)} \AgdaSymbol{(}\AgdaBound{g} \AgdaBound{a}\AgdaSymbol{)}\<%
\\
\>[0]\AgdaIndent{6}{}\<[6]%
\>[6]\AgdaSymbol{\}}\<%
\\
%
\\
\>\<\end{code}

$\Sigma$-types (object level)

\begin{code}\>\<%
\\
%
\\
\>\AgdaFunction{Σ''} \AgdaSymbol{:} \AgdaSymbol{\{}\AgdaBound{Γ} \AgdaSymbol{:} \AgdaFunction{Con}\AgdaSymbol{\}(}\AgdaBound{A} \AgdaSymbol{:} \AgdaRecord{Ty} \AgdaBound{Γ}\AgdaSymbol{)(}\AgdaBound{B} \AgdaSymbol{:} \AgdaRecord{Ty} \AgdaSymbol{(}\AgdaBound{Γ} \AgdaFunction{\&} \AgdaBound{A}\AgdaSymbol{))} \AgdaSymbol{→} \AgdaRecord{Ty} \AgdaBound{Γ}\<%
\\
\>\AgdaFunction{Σ''} \AgdaSymbol{\{}\AgdaBound{Γ}\AgdaSymbol{\}} \AgdaBound{A} \AgdaBound{B} \AgdaSymbol{=} \AgdaKeyword{record} \<[21]%
\>[21]\<%
\\
\>[6]\AgdaIndent{8}{}\<[8]%
\>[8]\AgdaSymbol{\{} \AgdaField{fm} \AgdaSymbol{=} \AgdaSymbol{λ} \AgdaBound{x} \AgdaSymbol{→} \AgdaKeyword{let} \AgdaBound{Ax} \AgdaSymbol{=} \AgdaFunction{[} \AgdaBound{A} \AgdaFunction{]fm} \AgdaBound{x} \AgdaKeyword{in}\<%
\\
\>[8]\AgdaIndent{15}{}\<[15]%
\>[15]\AgdaKeyword{let} \AgdaBound{Bx} \AgdaSymbol{=} \AgdaSymbol{λ} \AgdaBound{a} \AgdaSymbol{→} \AgdaFunction{[} \AgdaBound{B} \AgdaFunction{]fm} \AgdaSymbol{(}\AgdaBound{x} \AgdaInductiveConstructor{,} \AgdaBound{a}\AgdaSymbol{)} \AgdaKeyword{in}\<%
\\
\>[-2]\AgdaIndent{9}{}\<[9]%
\>[9]\AgdaKeyword{record}\<%
\\
\>[0]\AgdaIndent{11}{}\<[11]%
\>[11]\AgdaSymbol{\{} \AgdaField{Carrier} \AgdaSymbol{=} \AgdaRecord{Σ[} \AgdaBound{a} \AgdaRecord{∶} \AgdaFunction{∣} \AgdaBound{Ax} \AgdaFunction{∣} \AgdaRecord{]} \AgdaFunction{∣} \AgdaBound{Bx} \AgdaBound{a} \AgdaFunction{∣}\<%
\\
%
\\
\>[0]\AgdaIndent{11}{}\<[11]%
\>[11]\AgdaSymbol{;} \AgdaField{\_≈h\_} \<[21]%
\>[21]\AgdaSymbol{=} \AgdaSymbol{λ\{(}\AgdaBound{a₁} \AgdaInductiveConstructor{,} \AgdaBound{b₁}\AgdaSymbol{)} \AgdaSymbol{(}\AgdaBound{a₂} \AgdaInductiveConstructor{,} \AgdaBound{b₂}\AgdaSymbol{)} \AgdaSymbol{→} \<[47]%
\>[47]\<%
\\
\>[11]\AgdaIndent{23}{}\<[23]%
\>[23]\AgdaFunction{Σ'[} \AgdaBound{eq₁} \AgdaFunction{∶} \AgdaFunction{[} \AgdaBound{Ax} \AgdaFunction{]} \AgdaBound{a₁} \AgdaFunction{≈h} \AgdaBound{a₂} \AgdaFunction{]} \<[51]%
\>[51]\<%
\\
\>[11]\AgdaIndent{23}{}\<[23]%
\>[23]\AgdaFunction{[} \AgdaBound{Bx} \AgdaBound{a₂} \AgdaFunction{]} \AgdaFunction{[} \AgdaBound{B} \AgdaFunction{]subst} \<[44]%
\>[44]\<%
\\
\>[11]\AgdaIndent{23}{}\<[23]%
\>[23]\AgdaSymbol{(}\AgdaFunction{[} \AgdaBound{Γ} \AgdaFunction{]refl} \AgdaInductiveConstructor{,} \AgdaFunction{[} \AgdaFunction{[} \AgdaBound{A} \AgdaFunction{]fm} \AgdaBound{x} \AgdaFunction{]trans} \<[55]%
\>[55]\<%
\\
\>[11]\AgdaIndent{23}{}\<[23]%
\>[23]\AgdaSymbol{(}\AgdaFunction{[} \AgdaBound{A} \AgdaFunction{]refl*} \AgdaBound{x} \AgdaBound{a₁}\AgdaSymbol{)} \AgdaBound{eq₁}\AgdaSymbol{)} \AgdaBound{b₁} \AgdaFunction{≈h} \AgdaBound{b₂} \AgdaSymbol{\}}\<%
\\
%
\\
\>[0]\AgdaIndent{11}{}\<[11]%
\>[11]\AgdaSymbol{;} \AgdaField{refl} \<[21]%
\>[21]\AgdaSymbol{=} \AgdaSymbol{λ} \AgdaSymbol{\{}\AgdaBound{t}\AgdaSymbol{\}} \AgdaSymbol{→} \AgdaFunction{[} \AgdaFunction{[} \AgdaBound{A} \AgdaFunction{]fm} \AgdaBound{x} \AgdaFunction{]refl} \AgdaInductiveConstructor{,} \AgdaFunction{[} \AgdaBound{B} \AgdaFunction{]subst-pi'}\<%
\\
%
\\
\>[0]\AgdaIndent{11}{}\<[11]%
\>[11]\AgdaSymbol{;} \AgdaField{sym} \<[21]%
\>[21]\AgdaSymbol{=} \AgdaSymbol{λ} \AgdaSymbol{\{(}\AgdaBound{p} \AgdaInductiveConstructor{,} \AgdaBound{q}\AgdaSymbol{)} \AgdaSymbol{→} \AgdaSymbol{(}\AgdaFunction{[} \AgdaFunction{[} \AgdaBound{A} \AgdaFunction{]fm} \AgdaBound{x} \AgdaFunction{]sym} \AgdaBound{p}\AgdaSymbol{)} \AgdaInductiveConstructor{,} \<[59]%
\>[59]\<%
\\
\>[11]\AgdaIndent{23}{}\<[23]%
\>[23]\AgdaFunction{[} \AgdaBound{B} \AgdaFunction{]subst-mir1} \AgdaSymbol{(}\AgdaFunction{[} \AgdaFunction{[} \AgdaBound{B} \AgdaFunction{]fm} \AgdaSymbol{(}\AgdaBound{x} \AgdaInductiveConstructor{,} \AgdaSymbol{\_)} \AgdaFunction{]sym} \AgdaBound{q}\AgdaSymbol{)} \AgdaSymbol{\}}\<%
\\
%
\\
\>[0]\AgdaIndent{11}{}\<[11]%
\>[11]\AgdaSymbol{;} \AgdaField{trans} \<[21]%
\>[21]\AgdaSymbol{=} \AgdaSymbol{λ} \AgdaSymbol{\{(}\AgdaBound{p} \AgdaInductiveConstructor{,} \AgdaBound{q}\AgdaSymbol{)} \AgdaSymbol{(}\AgdaBound{r} \AgdaInductiveConstructor{,} \AgdaBound{s}\AgdaSymbol{)} \AgdaSymbol{→} \AgdaSymbol{(}\AgdaFunction{[} \AgdaFunction{[} \AgdaBound{A} \AgdaFunction{]fm} \AgdaBound{x} \AgdaFunction{]trans} \AgdaBound{p} \AgdaBound{r}\AgdaSymbol{)} \AgdaInductiveConstructor{,}\<%
\\
\>[0]\AgdaIndent{23}{}\<[23]%
\>[23]\AgdaSymbol{(}\AgdaFunction{[} \AgdaFunction{[} \AgdaBound{B} \AgdaFunction{]fm} \AgdaSymbol{(}\AgdaBound{x} \AgdaInductiveConstructor{,} \AgdaSymbol{\_)} \AgdaFunction{]trans} \<[49]%
\>[49]\<%
\\
\>[0]\AgdaIndent{23}{}\<[23]%
\>[23]\AgdaSymbol{(}\AgdaFunction{[} \AgdaFunction{[} \AgdaBound{B} \AgdaFunction{]fm} \AgdaSymbol{(}\AgdaBound{x} \AgdaInductiveConstructor{,} \AgdaSymbol{\_)} \AgdaFunction{]trans}\<%
\\
\>[0]\AgdaIndent{23}{}\<[23]%
\>[23]\AgdaSymbol{(}\AgdaFunction{[} \AgdaFunction{[} \AgdaBound{B} \AgdaFunction{]fm} \AgdaSymbol{(}\AgdaBound{x} \AgdaInductiveConstructor{,} \AgdaSymbol{\_)} \AgdaFunction{]trans} \AgdaFunction{[} \AgdaBound{B} \AgdaFunction{]subst-pi}\<%
\\
\>[0]\AgdaIndent{23}{}\<[23]%
\>[23]\AgdaSymbol{(}\AgdaFunction{[} \AgdaFunction{[} \AgdaBound{B} \AgdaFunction{]fm} \AgdaSymbol{(}\AgdaBound{x} \AgdaInductiveConstructor{,} \AgdaSymbol{\_)} \AgdaFunction{]sym} \AgdaSymbol{(}\AgdaFunction{[} \AgdaBound{B} \AgdaFunction{]trans*} \<[60]%
\>[60]\<%
\\
\>[0]\AgdaIndent{23}{}\<[23]%
\>[23]\AgdaSymbol{\{}q \AgdaSymbol{=} \AgdaFunction{[} \AgdaBound{Γ} \AgdaFunction{]refl} \AgdaInductiveConstructor{,} \AgdaFunction{[} \AgdaFunction{[} \AgdaBound{A} \AgdaFunction{]fm} \AgdaBound{x} \AgdaFunction{]trans} \<[59]%
\>[59]\<%
\\
\>[0]\AgdaIndent{23}{}\<[23]%
\>[23]\AgdaSymbol{(}\AgdaFunction{[} \AgdaBound{A} \AgdaFunction{]refl*} \AgdaBound{x} \AgdaSymbol{\_)} \AgdaBound{r}\AgdaSymbol{\}} \AgdaSymbol{\_)))} \AgdaSymbol{(}\AgdaFunction{[} \AgdaBound{B} \AgdaFunction{]subst-pi*} \AgdaBound{q}\AgdaSymbol{))} \AgdaBound{s}\AgdaSymbol{)\}}\<%
\\
%
\\
\>[0]\AgdaIndent{11}{}\<[11]%
\>[11]\AgdaSymbol{\}}\<%
\\
%
\\
\>[0]\AgdaIndent{8}{}\<[8]%
\>[8]\AgdaSymbol{;} \AgdaField{substT} \AgdaSymbol{=} \AgdaSymbol{λ} \AgdaBound{x≈y} \AgdaSymbol{→} \AgdaSymbol{λ} \AgdaSymbol{\{(}\AgdaBound{p} \AgdaInductiveConstructor{,} \AgdaBound{q}\AgdaSymbol{)} \AgdaSymbol{→} \<[40]%
\>[40]\<%
\\
\>[8]\AgdaIndent{19}{}\<[19]%
\>[19]\AgdaSymbol{(}\AgdaFunction{[} \AgdaBound{A} \AgdaFunction{]subst} \AgdaBound{x≈y} \AgdaBound{p}\AgdaSymbol{)} \AgdaInductiveConstructor{,} \AgdaFunction{[} \AgdaBound{B} \AgdaFunction{]subst} \AgdaSymbol{(}\AgdaBound{x≈y} \AgdaInductiveConstructor{,} \<[58]%
\>[58]\<%
\\
\>[8]\AgdaIndent{19}{}\<[19]%
\>[19]\AgdaFunction{[} \AgdaFunction{[} \AgdaBound{A} \AgdaFunction{]fm} \AgdaSymbol{\_} \AgdaFunction{]refl}\AgdaSymbol{)} \AgdaBound{q}\AgdaSymbol{\}}\<%
\\
%
\\
\>[0]\AgdaIndent{8}{}\<[8]%
\>[8]\AgdaSymbol{;} \AgdaField{subst*} \AgdaSymbol{=} \AgdaSymbol{λ} \AgdaBound{x≈y} \AgdaSymbol{→} \<[28]%
\>[28]\AgdaSymbol{λ} \AgdaSymbol{\{(}\AgdaBound{p} \AgdaInductiveConstructor{,} \AgdaBound{q}\AgdaSymbol{)} \AgdaSymbol{→} \AgdaFunction{[} \AgdaBound{A} \AgdaFunction{]subst*} \AgdaBound{x≈y} \AgdaBound{p} \AgdaInductiveConstructor{,} \<[61]%
\>[61]\<%
\\
\>[0]\AgdaIndent{19}{}\<[19]%
\>[19]\AgdaFunction{[} \AgdaFunction{[} \AgdaBound{B} \AgdaFunction{]fm} \AgdaSymbol{\_} \AgdaFunction{]trans} \AgdaSymbol{(}\AgdaFunction{[} \AgdaFunction{[} \AgdaBound{B} \AgdaFunction{]fm} \AgdaSymbol{\_} \AgdaFunction{]trans} \<[58]%
\>[58]\<%
\\
\>[0]\AgdaIndent{19}{}\<[19]%
\>[19]\AgdaSymbol{(}\AgdaFunction{[} \AgdaBound{B} \AgdaFunction{]trans*} \AgdaSymbol{\_)} \AgdaSymbol{(}\AgdaFunction{[} \AgdaFunction{[} \AgdaBound{B} \AgdaFunction{]fm} \AgdaSymbol{\_} \AgdaFunction{]trans} \AgdaFunction{[} \AgdaBound{B} \AgdaFunction{]subst-pi} \<[69]%
\>[69]\<%
\\
\>[0]\AgdaIndent{19}{}\<[19]%
\>[19]\AgdaSymbol{(}\AgdaFunction{[} \AgdaFunction{[} \AgdaBound{B} \AgdaFunction{]fm} \AgdaSymbol{\_} \AgdaFunction{]sym} \AgdaSymbol{(}\AgdaFunction{[} \AgdaBound{B} \AgdaFunction{]trans*} \AgdaSymbol{\_))))} \AgdaSymbol{(}\AgdaFunction{[} \AgdaBound{B} \AgdaFunction{]subst*} \<[69]%
\>[69]\<%
\\
\>[0]\AgdaIndent{19}{}\<[19]%
\>[19]\AgdaSymbol{(}\AgdaBound{x≈y} \AgdaInductiveConstructor{,} \AgdaFunction{[} \AgdaFunction{[} \AgdaBound{A} \AgdaFunction{]fm} \AgdaSymbol{\_} \AgdaFunction{]refl}\AgdaSymbol{)} \AgdaBound{q}\AgdaSymbol{)} \AgdaSymbol{\}}\<%
\\
\>[0]\AgdaIndent{8}{}\<[8]%
\>[8]\AgdaSymbol{;} \AgdaField{refl*} \AgdaSymbol{=} \AgdaSymbol{λ} \AgdaBound{x} \AgdaSymbol{→} \<[25]%
\>[25]\AgdaSymbol{λ} \AgdaSymbol{\{(}\AgdaBound{p} \AgdaInductiveConstructor{,} \AgdaBound{q}\AgdaSymbol{)} \AgdaSymbol{→} \AgdaSymbol{(}\AgdaFunction{[} \AgdaBound{A} \AgdaFunction{]refl*} \AgdaSymbol{\_} \AgdaSymbol{\_)} \AgdaInductiveConstructor{,} \<[57]%
\>[57]\<%
\\
\>[0]\AgdaIndent{18}{}\<[18]%
\>[18]\AgdaSymbol{(}\AgdaFunction{[} \AgdaFunction{[} \AgdaBound{B} \AgdaFunction{]fm} \AgdaSymbol{\_} \AgdaFunction{]trans} \AgdaSymbol{(}\AgdaFunction{[} \AgdaBound{B} \AgdaFunction{]trans*} \AgdaSymbol{\_)} \AgdaFunction{[} \AgdaBound{B} \AgdaFunction{]subst-pi'}\AgdaSymbol{)\}}\<%
\\
%
\\
\>[0]\AgdaIndent{8}{}\<[8]%
\>[8]\AgdaSymbol{;} \AgdaField{trans*} \AgdaSymbol{=} \<[20]%
\>[20]\AgdaSymbol{λ} \AgdaSymbol{\{(}\AgdaBound{p} \AgdaInductiveConstructor{,} \AgdaBound{q}\AgdaSymbol{)} \<[32]%
\>[32]\AgdaSymbol{→} \AgdaSymbol{(}\AgdaFunction{[} \AgdaBound{A} \AgdaFunction{]trans*} \AgdaSymbol{\_)} \AgdaInductiveConstructor{,} \AgdaSymbol{(}\AgdaFunction{[} \AgdaFunction{[} \AgdaBound{B} \AgdaFunction{]fm} \AgdaSymbol{\_} \AgdaFunction{]trans}\<%
\\
\>[0]\AgdaIndent{20}{}\<[20]%
\>[20]\AgdaSymbol{(}\AgdaFunction{[} \AgdaBound{B} \AgdaFunction{]trans*} \AgdaSymbol{\_)} \AgdaSymbol{(}\AgdaFunction{[} \AgdaFunction{[} \AgdaBound{B} \AgdaFunction{]fm} \AgdaSymbol{\_} \AgdaFunction{]trans} \AgdaSymbol{(}\AgdaFunction{[} \AgdaBound{B} \AgdaFunction{]trans*} \AgdaSymbol{\_)}\<%
\\
\>[0]\AgdaIndent{20}{}\<[20]%
\>[20]\AgdaFunction{[} \AgdaBound{B} \AgdaFunction{]subst-pi}\AgdaSymbol{))} \AgdaSymbol{\}}\<%
\\
\>[0]\AgdaIndent{8}{}\<[8]%
\>[8]\AgdaSymbol{\}}\<%
\\
%
\\
\>\<\end{code}


\AgdaHide{
\begin{code}\>\<%
\\
\>\AgdaComment{\{-
module TypeTerm-Equality
  (Ty-ExtEq : \{Γ : Con\}\{A B : Ty Γ\} →
              (p : [ A ]fm ≡ [ B ]fm) →
              (∀\{x\}\{y\}(xy : [ Γ ] x ≈ y)(fx : ∣ [ A ]fm x ∣) → subst (λ fm → ∣ fm y ∣) p ([ A ]subst xy fx) ≡ [ B ]subst xy (subst (λ fm → ∣ fm x ∣) p fx)) →
           A ≡ B)
  (Tm-ExtEq : \{Γ : Con\}\{A : Ty Γ\}\{t r : Tm A\} →
              ([ t ]tm ≡ [ r ]tm) →
              t ≡ r) where

  Fid : \{Γ : Con\}\{A : Ty Γ\} → A [ idCH ] ≡ A
  Fid = Ty-ExtEq PE.refl (λ xy fx → PE.refl)

  Fcomp : \{Γ Δ Φ : Con\}\{A : Ty Γ\}(f : Δ ⇉ Γ)(g : Φ ⇉ Δ) → A [ f ∘c g ] ≡ A [ f ] [ g ]
  Fcomp f g = Ty-ExtEq PE.refl (λ xy fx → PE.refl) 


  Tm-ExtEq-TyDif : \{Γ : Con\}\{A B : Ty Γ\}\{t : Tm A\}\{r : Tm B\} → (eq : A ≡ B) → 
                   (∀ x → subst (λ ty → ∣ [ ty ]fm x ∣) eq ([ t ]tm x) ≡ [ r ]tm x) → 
                   (subst Tm eq t ≡ r)
  Tm-ExtEq-TyDif PE.refl f = Tm-ExtEq (ext f)


  Fidm : \{Γ : Con\}\{A : Ty Γ\}(t : Tm A) → (subst Tm Fid (t [ idCH ]m)) ≡ t
  Fidm t = Tm-ExtEq-TyDif Fid (λ x → \{!!\}) -- Tm-ExtEq-TyDif Fid (λ x → \{!trans!\})

  Fcompm : \{Γ Δ Φ : Con\}\{A : Ty Γ\}\{t : Tm A\}(f : Δ ⇉ Γ)(g : Φ ⇉ Δ) → subst Tm (Fcomp f g) (t [ f ∘c g ]m) ≡ t [ f ]m [ g ]m
  Fcompm f g = \{!!\} 
 
-- verify Π types are consistent with substitution

  Pi-Sub : \{Γ Δ : Con\}(A : Ty Γ)(B : Ty (Γ \& A))(f : Δ ⇉ Γ) →
          (Π A B) [ f ] ≡ Π (A [ f ]) (B [ f \textasciicircum A ])
  Pi-Sub A B f = Ty-ExtEq \{!ext!\} \{!!\}

-\}}\<%
\\
%
\\
\>\<\end{code}
}

%\AgdaHide{
\begin{code}\>\<%
\\
%
\\
\>\AgdaSymbol{\{-\#} \AgdaKeyword{OPTIONS} --type-in-type \AgdaSymbol{\#-\}}\<%
\\
%
\\
\>\AgdaKeyword{import} \AgdaModule{Level}\<%
\\
\>\AgdaKeyword{open} \AgdaKeyword{import} \AgdaModule{Relation.Binary.PropositionalEquality} \AgdaSymbol{as} \AgdaModule{PE} \AgdaKeyword{hiding} \AgdaSymbol{(}refl \AgdaSymbol{;} sym \AgdaSymbol{;} trans\AgdaSymbol{;} isEquivalence\AgdaSymbol{;} [\_]\AgdaSymbol{)}\<%
\\
%
\\
\>\AgdaKeyword{module} \AgdaModule{CwF-ctd} \AgdaSymbol{(}\AgdaBound{ext} \AgdaSymbol{:} \AgdaFunction{Extensionality} \AgdaPrimitive{Level.zero} \AgdaPrimitive{Level.zero}\AgdaSymbol{)} \AgdaKeyword{where}\<%
\\
%
\\
\>\AgdaKeyword{open} \AgdaKeyword{import} \AgdaModule{Data.Unit}\<%
\\
\>\AgdaKeyword{open} \AgdaKeyword{import} \AgdaModule{Function}\<%
\\
\>\AgdaKeyword{open} \AgdaKeyword{import} \AgdaModule{Data.Product}\<%
\\
%
\\
\>\AgdaKeyword{open} \AgdaKeyword{import} \AgdaModule{CwF-setoidwo} \AgdaBound{ext} \AgdaKeyword{public}\<%
\\
%
\\
\>\AgdaKeyword{open} \AgdaKeyword{import} \AgdaModule{Data.Nat}\<%
\\
%
\\
\>\<\end{code}
}

Binary relation

\begin{code}\>\<%
\\
%
\\
\>\AgdaFunction{Rel} \AgdaSymbol{:} \AgdaSymbol{\{}\AgdaBound{Γ} \AgdaSymbol{:} \AgdaFunction{Con}\AgdaSymbol{\}} \AgdaSymbol{→} \AgdaRecord{Ty} \AgdaBound{Γ} \AgdaSymbol{→} \AgdaPrimitiveType{Set₁}\<%
\\
\>\AgdaFunction{Rel} \AgdaSymbol{\{}\AgdaBound{Γ}\AgdaSymbol{\}} \AgdaBound{A} \AgdaSymbol{=} \AgdaRecord{Ty} \AgdaSymbol{(}\AgdaBound{Γ} \AgdaFunction{\&} \AgdaBound{A} \AgdaFunction{\&} \AgdaBound{A} \AgdaFunction{[} \AgdaFunction{fst\&} \AgdaSymbol{\{}A \AgdaSymbol{=} \AgdaBound{A}\AgdaSymbol{\}} \AgdaFunction{]T}\AgdaSymbol{)}\<%
\\
%
\\
\>\<\end{code}

Natural numbers

\begin{code}\>\<%
\\
%
\\
\>\AgdaKeyword{module} \AgdaModule{Natural} \AgdaSymbol{(}\AgdaBound{Γ} \AgdaSymbol{:} \AgdaFunction{Con}\AgdaSymbol{)} \AgdaKeyword{where}\<%
\\
%
\\
\>[0]\AgdaIndent{2}{}\<[2]%
\>[2]\AgdaFunction{\_≈nat\_} \AgdaSymbol{:} \AgdaDatatype{ℕ} \AgdaSymbol{→} \AgdaDatatype{ℕ} \AgdaSymbol{→} \AgdaRecord{HProp}\<%
\\
\>[0]\AgdaIndent{2}{}\<[2]%
\>[2]\AgdaInductiveConstructor{zero} \AgdaFunction{≈nat} \AgdaInductiveConstructor{zero} \AgdaSymbol{=} \AgdaFunction{⊤'}\<%
\\
\>[0]\AgdaIndent{2}{}\<[2]%
\>[2]\AgdaInductiveConstructor{zero} \AgdaFunction{≈nat} \AgdaInductiveConstructor{suc} \AgdaBound{n} \AgdaSymbol{=} \AgdaFunction{⊥'}\<%
\\
\>[0]\AgdaIndent{2}{}\<[2]%
\>[2]\AgdaInductiveConstructor{suc} \AgdaBound{m} \AgdaFunction{≈nat} \AgdaInductiveConstructor{zero} \AgdaSymbol{=} \AgdaFunction{⊥'}\<%
\\
\>[0]\AgdaIndent{2}{}\<[2]%
\>[2]\AgdaInductiveConstructor{suc} \AgdaBound{m} \AgdaFunction{≈nat} \AgdaInductiveConstructor{suc} \AgdaBound{n} \AgdaSymbol{=} \AgdaBound{m} \AgdaFunction{≈nat} \AgdaBound{n}\<%
\\
\>[0]\AgdaIndent{2}{}\<[2]%
\>[2]\<%
\\
\>[0]\AgdaIndent{2}{}\<[2]%
\>[2]\AgdaFunction{reflNat} \AgdaSymbol{:} \AgdaSymbol{\{}\AgdaBound{x} \AgdaSymbol{:} \AgdaDatatype{ℕ}\AgdaSymbol{\}} \AgdaSymbol{→} \AgdaFunction{<} \AgdaBound{x} \AgdaFunction{≈nat} \AgdaBound{x} \AgdaFunction{>} \<[35]%
\>[35]\<%
\\
\>[0]\AgdaIndent{2}{}\<[2]%
\>[2]\AgdaFunction{reflNat} \AgdaSymbol{\{}\AgdaInductiveConstructor{zero}\AgdaSymbol{\}} \AgdaSymbol{=} \AgdaInductiveConstructor{tt}\<%
\\
\>[0]\AgdaIndent{2}{}\<[2]%
\>[2]\AgdaFunction{reflNat} \AgdaSymbol{\{}\AgdaInductiveConstructor{suc} \AgdaBound{n}\AgdaSymbol{\}} \AgdaSymbol{=} \AgdaFunction{reflNat} \AgdaSymbol{\{}\AgdaBound{n}\AgdaSymbol{\}}\<%
\\
%
\\
\>[0]\AgdaIndent{2}{}\<[2]%
\>[2]\AgdaFunction{symNat} \AgdaSymbol{:} \AgdaSymbol{\{}\AgdaBound{x} \AgdaBound{y} \AgdaSymbol{:} \AgdaDatatype{ℕ}\AgdaSymbol{\}} \AgdaSymbol{→} \AgdaFunction{<} \AgdaBound{x} \AgdaFunction{≈nat} \AgdaBound{y} \AgdaFunction{>} \AgdaSymbol{→} \AgdaFunction{<} \AgdaBound{y} \AgdaFunction{≈nat} \AgdaBound{x} \AgdaFunction{>}\<%
\\
\>[0]\AgdaIndent{2}{}\<[2]%
\>[2]\AgdaFunction{symNat} \AgdaSymbol{\{}\AgdaInductiveConstructor{zero}\AgdaSymbol{\}} \AgdaSymbol{\{}\AgdaInductiveConstructor{zero}\AgdaSymbol{\}} \AgdaBound{eq} \AgdaSymbol{=} \AgdaInductiveConstructor{tt}\<%
\\
\>[0]\AgdaIndent{2}{}\<[2]%
\>[2]\AgdaFunction{symNat} \AgdaSymbol{\{}\AgdaInductiveConstructor{zero}\AgdaSymbol{\}} \AgdaSymbol{\{}\AgdaInductiveConstructor{suc} \AgdaSymbol{\_\}} \AgdaBound{eq} \AgdaSymbol{=} \AgdaBound{eq}\<%
\\
\>[0]\AgdaIndent{2}{}\<[2]%
\>[2]\AgdaFunction{symNat} \AgdaSymbol{\{}\AgdaInductiveConstructor{suc} \AgdaSymbol{\_\}} \AgdaSymbol{\{}\AgdaInductiveConstructor{zero}\AgdaSymbol{\}} \AgdaBound{eq} \AgdaSymbol{=} \AgdaBound{eq}\<%
\\
\>[0]\AgdaIndent{2}{}\<[2]%
\>[2]\AgdaFunction{symNat} \AgdaSymbol{\{}\AgdaInductiveConstructor{suc} \AgdaBound{x}\AgdaSymbol{\}} \AgdaSymbol{\{}\AgdaInductiveConstructor{suc} \AgdaBound{y}\AgdaSymbol{\}} \AgdaBound{eq} \AgdaSymbol{=} \AgdaFunction{symNat} \AgdaSymbol{\{}\AgdaBound{x}\AgdaSymbol{\}} \AgdaSymbol{\{}\AgdaBound{y}\AgdaSymbol{\}} \AgdaBound{eq}\<%
\\
%
\\
\>[0]\AgdaIndent{2}{}\<[2]%
\>[2]\AgdaFunction{transNat} \AgdaSymbol{:} \AgdaSymbol{\{}\AgdaBound{x} \AgdaBound{y} \AgdaBound{z} \AgdaSymbol{:} \AgdaDatatype{ℕ}\AgdaSymbol{\}} \AgdaSymbol{→} \AgdaFunction{<} \AgdaBound{x} \AgdaFunction{≈nat} \AgdaBound{y} \AgdaFunction{>} \AgdaSymbol{→} \AgdaFunction{<} \AgdaBound{y} \AgdaFunction{≈nat} \AgdaBound{z} \AgdaFunction{>} \AgdaSymbol{→} \AgdaFunction{<} \AgdaBound{x} \AgdaFunction{≈nat} \AgdaBound{z} \AgdaFunction{>}\<%
\\
\>[0]\AgdaIndent{2}{}\<[2]%
\>[2]\AgdaFunction{transNat} \AgdaSymbol{\{}\AgdaInductiveConstructor{zero}\AgdaSymbol{\}} \AgdaSymbol{\{}\AgdaInductiveConstructor{zero}\AgdaSymbol{\}} \AgdaBound{xy} \AgdaBound{yz} \AgdaSymbol{=} \AgdaBound{yz}\<%
\\
\>[0]\AgdaIndent{2}{}\<[2]%
\>[2]\AgdaFunction{transNat} \AgdaSymbol{\{}\AgdaInductiveConstructor{zero}\AgdaSymbol{\}} \AgdaSymbol{\{}\AgdaInductiveConstructor{suc} \AgdaSymbol{\_\}} \AgdaSymbol{()} \AgdaBound{yz}\<%
\\
\>[0]\AgdaIndent{2}{}\<[2]%
\>[2]\AgdaFunction{transNat} \AgdaSymbol{\{}\AgdaInductiveConstructor{suc} \AgdaSymbol{\_\}} \AgdaSymbol{\{}\AgdaInductiveConstructor{zero}\AgdaSymbol{\}} \AgdaSymbol{()} \AgdaBound{yz}\<%
\\
\>[0]\AgdaIndent{2}{}\<[2]%
\>[2]\AgdaFunction{transNat} \AgdaSymbol{\{}\AgdaInductiveConstructor{suc} \AgdaSymbol{\_\}} \AgdaSymbol{\{}\AgdaInductiveConstructor{suc} \AgdaSymbol{\_\}} \AgdaSymbol{\{}\AgdaInductiveConstructor{zero}\AgdaSymbol{\}} \AgdaBound{xy} \AgdaBound{yz} \AgdaSymbol{=} \AgdaBound{yz}\<%
\\
\>[0]\AgdaIndent{2}{}\<[2]%
\>[2]\AgdaFunction{transNat} \AgdaSymbol{\{}\AgdaInductiveConstructor{suc} \AgdaBound{x}\AgdaSymbol{\}} \AgdaSymbol{\{}\AgdaInductiveConstructor{suc} \AgdaBound{y}\AgdaSymbol{\}} \AgdaSymbol{\{}\AgdaInductiveConstructor{suc} \AgdaBound{z}\AgdaSymbol{\}} \AgdaBound{xy} \AgdaBound{yz} \AgdaSymbol{=} \AgdaFunction{transNat} \AgdaSymbol{\{}\AgdaBound{x}\AgdaSymbol{\}} \AgdaSymbol{\{}\AgdaBound{y}\AgdaSymbol{\}} \AgdaSymbol{\{}\AgdaBound{z}\AgdaSymbol{\}} \AgdaBound{xy} \AgdaBound{yz}\<%
\\
%
\\
\>[0]\AgdaIndent{2}{}\<[2]%
\>[2]\AgdaFunction{⟦Nat⟧} \AgdaSymbol{:} \AgdaRecord{Ty} \AgdaBound{Γ}\<%
\\
\>[0]\AgdaIndent{2}{}\<[2]%
\>[2]\AgdaFunction{⟦Nat⟧} \AgdaSymbol{=} \AgdaKeyword{record} \<[17]%
\>[17]\<%
\\
\>[2]\AgdaIndent{4}{}\<[4]%
\>[4]\AgdaSymbol{\{} \AgdaField{fm} \AgdaSymbol{=} \AgdaSymbol{λ} \AgdaBound{γ} \AgdaSymbol{→} \AgdaKeyword{record}\<%
\\
\>[4]\AgdaIndent{9}{}\<[9]%
\>[9]\AgdaSymbol{\{} \AgdaField{Carrier} \AgdaSymbol{=} \AgdaDatatype{ℕ}\<%
\\
\>[4]\AgdaIndent{9}{}\<[9]%
\>[9]\AgdaSymbol{;} \AgdaField{\_≈h\_} \AgdaSymbol{=} \AgdaFunction{\_≈nat\_}\<%
\\
\>[4]\AgdaIndent{9}{}\<[9]%
\>[9]\AgdaSymbol{;} \AgdaField{refl} \AgdaSymbol{=} \AgdaSymbol{λ} \AgdaSymbol{\{}\AgdaBound{n}\AgdaSymbol{\}} \AgdaSymbol{→} \AgdaFunction{reflNat} \AgdaSymbol{\{}\AgdaBound{n}\AgdaSymbol{\}}\<%
\\
\>[4]\AgdaIndent{9}{}\<[9]%
\>[9]\AgdaSymbol{;} \AgdaField{sym} \AgdaSymbol{=} \AgdaSymbol{λ} \AgdaSymbol{\{}\AgdaBound{x}\AgdaSymbol{\}} \AgdaSymbol{\{}\AgdaBound{y}\AgdaSymbol{\}} \AgdaSymbol{→} \AgdaFunction{symNat} \AgdaSymbol{\{}\AgdaBound{x}\AgdaSymbol{\}} \AgdaSymbol{\{}\AgdaBound{y}\AgdaSymbol{\}}\<%
\\
\>[4]\AgdaIndent{9}{}\<[9]%
\>[9]\AgdaSymbol{;} \AgdaField{trans} \AgdaSymbol{=} \AgdaSymbol{λ} \AgdaSymbol{\{}\AgdaBound{x}\AgdaSymbol{\}} \AgdaSymbol{\{}\AgdaBound{y}\AgdaSymbol{\}} \AgdaSymbol{\{}\AgdaBound{z}\AgdaSymbol{\}} \AgdaSymbol{→} \AgdaFunction{transNat} \AgdaSymbol{\{}\AgdaBound{x}\AgdaSymbol{\}} \AgdaSymbol{\{}\AgdaBound{y}\AgdaSymbol{\}} \AgdaSymbol{\{}\AgdaBound{z}\AgdaSymbol{\}}\<%
\\
\>[4]\AgdaIndent{9}{}\<[9]%
\>[9]\AgdaSymbol{\}}\<%
\\
\>[0]\AgdaIndent{4}{}\<[4]%
\>[4]\AgdaSymbol{;} \AgdaField{substT} \AgdaSymbol{=} \AgdaSymbol{λ} \AgdaBound{\_} \AgdaSymbol{→} \AgdaFunction{id}\<%
\\
\>[0]\AgdaIndent{4}{}\<[4]%
\>[4]\AgdaSymbol{;} \AgdaField{subst*} \AgdaSymbol{=} \AgdaSymbol{λ} \AgdaBound{\_} \AgdaSymbol{→} \AgdaFunction{id}\<%
\\
\>[0]\AgdaIndent{4}{}\<[4]%
\>[4]\AgdaSymbol{;} \AgdaField{refl*} \AgdaSymbol{=} \AgdaSymbol{λ} \AgdaBound{x} \AgdaBound{a} \AgdaSymbol{→} \AgdaFunction{reflNat} \AgdaSymbol{\{}\AgdaBound{a}\AgdaSymbol{\}}\<%
\\
\>[0]\AgdaIndent{4}{}\<[4]%
\>[4]\AgdaSymbol{;} \AgdaField{trans*} \AgdaSymbol{=} \AgdaSymbol{λ} \AgdaBound{a} \AgdaSymbol{→} \AgdaFunction{reflNat} \AgdaSymbol{\{}\AgdaBound{a}\AgdaSymbol{\}} \<[33]%
\>[33]\<%
\\
\>[0]\AgdaIndent{4}{}\<[4]%
\>[4]\AgdaSymbol{\}}\<%
\\
%
\\
\>[0]\AgdaIndent{2}{}\<[2]%
\>[2]\AgdaFunction{⟦0⟧} \AgdaSymbol{:} \AgdaRecord{Tm} \AgdaFunction{⟦Nat⟧}\<%
\\
\>[0]\AgdaIndent{2}{}\<[2]%
\>[2]\AgdaFunction{⟦0⟧} \AgdaSymbol{=} \AgdaKeyword{record}\<%
\\
\>[2]\AgdaIndent{6}{}\<[6]%
\>[6]\AgdaSymbol{\{} \AgdaField{tm} \AgdaSymbol{=} \AgdaSymbol{λ} \AgdaBound{\_} \AgdaSymbol{→} \AgdaNumber{0}\<%
\\
\>[2]\AgdaIndent{6}{}\<[6]%
\>[6]\AgdaSymbol{;} \AgdaField{respt} \AgdaSymbol{=} \AgdaSymbol{λ} \AgdaBound{p} \AgdaSymbol{→} \AgdaInductiveConstructor{tt}\<%
\\
\>[2]\AgdaIndent{6}{}\<[6]%
\>[6]\AgdaSymbol{\}}\<%
\\
%
\\
\>[0]\AgdaIndent{2}{}\<[2]%
\>[2]\AgdaFunction{⟦s⟧} \AgdaSymbol{:} \AgdaRecord{Tm} \AgdaFunction{⟦Nat⟧} \AgdaSymbol{→} \AgdaRecord{Tm} \AgdaFunction{⟦Nat⟧}\<%
\\
\>[0]\AgdaIndent{2}{}\<[2]%
\>[2]\AgdaFunction{⟦s⟧} \AgdaSymbol{(}\AgdaInductiveConstructor{tm:} \AgdaBound{t} \AgdaInductiveConstructor{resp:} \AgdaBound{respt}\AgdaSymbol{)} \<[26]%
\>[26]\<%
\\
\>[2]\AgdaIndent{6}{}\<[6]%
\>[6]\AgdaSymbol{=} \AgdaKeyword{record}\<%
\\
\>[2]\AgdaIndent{6}{}\<[6]%
\>[6]\AgdaSymbol{\{} \AgdaField{tm} \AgdaSymbol{=} \AgdaInductiveConstructor{suc} \AgdaFunction{∘} \AgdaBound{t}\<%
\\
\>[2]\AgdaIndent{6}{}\<[6]%
\>[6]\AgdaSymbol{;} \AgdaField{respt} \AgdaSymbol{=} \AgdaBound{respt}\<%
\\
\>[2]\AgdaIndent{6}{}\<[6]%
\>[6]\AgdaSymbol{\}}\<%
\\
%
\\
\>\<\end{code}

Simply typed universe

\AgdaHide{
\begin{code}\>\<%
\\
%
\\
\>\AgdaComment{\{-
  data  ⟦U⟧⁰ : Set where
    nat : ⟦U⟧⁰
    arr<\_,\_> : (a b : ⟦U⟧⁰) → ⟦U⟧⁰

  \_\textasciitilde⟦U⟧\_ : ⟦U⟧⁰ → ⟦U⟧⁰ → HProp
  nat \textasciitilde⟦U⟧ nat = ⊤'
  nat \textasciitilde⟦U⟧ arr< a , b > = ⊥'
  arr< a , b > \textasciitilde⟦U⟧ nat = ⊥'
  arr< a , b > \textasciitilde⟦U⟧ arr< a' , b' > = a \textasciitilde⟦U⟧ a' ∧ b \textasciitilde⟦U⟧ b'

  reflU :  \{x : ⟦U⟧⁰\} → < x \textasciitilde⟦U⟧ x >
  reflU \{nat\} = tt
  reflU \{arr< a , b >\} = reflU \{a\} , reflU \{b\}

  symU : \{x y : ⟦U⟧⁰\} → < x \textasciitilde⟦U⟧ y > → < y \textasciitilde⟦U⟧ x >
  symU \{nat\} \{nat\} eq = tt
  symU \{nat\} \{arr< a , b >\} eq = eq
  symU \{arr< a , b >\} \{nat\} eq = eq
  symU \{arr< a , b >\} \{arr< a' , b' >\} (p , q) = (symU \{a\} \{a'\} p) 
                                               , (symU \{b\} \{b'\} q)

  transU : \{x y z : ⟦U⟧⁰\} → < x \textasciitilde⟦U⟧ y > → < y \textasciitilde⟦U⟧ z > → < x \textasciitilde⟦U⟧ z >
  transU \{nat\} \{nat\} eq1 eq2 = eq2
  transU \{nat\} \{arr< a , b >\} () eq2
  transU \{arr< a , b >\} \{nat\} () eq2
  transU \{arr< a , b >\} \{arr< a' , b' >\} \{nat\} eq1 eq2 = eq2
  transU \{arr< a , b >\} \{arr< a' , b' >\} \{arr< a0 , b0 >\} (p1 , q1) 
         (p2 , q2) = (transU \{a\} \{a'\} \{a0\} p1 p2) 
         , transU \{b\} \{b'\} \{b0\} q1 q2

  ⟦U⟧ : Ty Γ
  ⟦U⟧ = record 
    \{ fm = λ γ → record
         \{ Carrier = ⟦U⟧⁰
         ; \_≈h\_ = \_\textasciitilde⟦U⟧\_
         ; refl = λ \{x\} → reflU \{x\}
         ; sym = λ \{x\} \{y\} → symU \{x\} \{y\}
         ; trans = λ \{x\} \{y\} \{z\} → transU \{x\} \{y\} \{z\}
         \}
    ; substT = λ \_ → id
    ; subst* = λ \_ → id
    ; refl* = λ x a → reflU \{a\}
    ; trans* = λ a → reflU \{a\}
    \}

  elfm : Σ ∣ Γ ∣ (λ x → ⟦U⟧⁰) → HSetoid
  elfm (γ , nat) = [ ⟦Nat⟧ ]fm γ
  elfm (γ , arr< a , b >) = [ Γ , γ ] elfm (γ , a) ⇒fm elfm (γ , b)
-\}}\<%
\\
%
\\
\>\<\end{code}
}

\AgdaHide{
\begin{code}\>\<%
\\
%
\\
\>\AgdaComment{\{- To do : To find the way to extract the substT from ->

  elsubstT : \{x y : Σ ∣ Γ ∣ (λ x' → ⟦U⟧⁰)\} →
      Σ < [ Γ ] proj₁ x ≈h proj₁ y > (λ x' → < proj₂ x \textasciitilde⟦U⟧ proj₂ y >) →
      ∣ elfm x ∣ → ∣ elfm y ∣
  elsubstT \{\_ , nat\} \{\_ , nat\} \_ x' = x'
  elsubstT \{\_ , nat\} \{\_ , arr< a , b >\} (p , ()) x'
  elsubstT \{\_ , arr< a , b >\} \{\_ , nat\} (p , ()) x'
  elsubstT \{γ , arr< a , b >\} \{γ' , arr< a' , b' >\} (p , qa , qb) (s1 , s2) = 
   \{!!\}

  ⟦El⟧ : Ty (Γ \& ⟦U⟧)
  ⟦El⟧ = record 
       \{ fm = elfm
       ; substT = elsubstT
       ; subst* = \{!!\}
       ; refl* = \{!!\}
       ; trans* = \{!!\} 
       \}

-\}}\<%
\\
\>\<\end{code}
}

The equality type

\begin{code}\>\<%
\\
%
\\
\>\AgdaKeyword{module} \AgdaModule{Equality-Type} \AgdaSymbol{(}\AgdaBound{Γ} \AgdaSymbol{:} \AgdaFunction{Con}\AgdaSymbol{)(}\AgdaBound{A} \AgdaSymbol{:} \AgdaRecord{Ty} \AgdaBound{Γ}\AgdaSymbol{)} \AgdaKeyword{where}\<%
\\
%
\\
\>[0]\AgdaIndent{2}{}\<[2]%
\>[2]\AgdaFunction{⟦Id⟧} \AgdaSymbol{:} \AgdaFunction{Rel} \AgdaBound{A}\<%
\\
\>[0]\AgdaIndent{2}{}\<[2]%
\>[2]\AgdaFunction{⟦Id⟧} \AgdaSymbol{=} \AgdaKeyword{record} \<[16]%
\>[16]\<%
\\
\>[2]\AgdaIndent{4}{}\<[4]%
\>[4]\AgdaSymbol{\{} \AgdaField{fm} \AgdaSymbol{=} \AgdaSymbol{λ} \AgdaSymbol{\{((}\AgdaBound{x} \AgdaInductiveConstructor{,} \AgdaBound{a}\AgdaSymbol{)} \AgdaInductiveConstructor{,} \AgdaBound{b}\AgdaSymbol{)} \AgdaSymbol{→} \<[30]%
\>[30]\<%
\\
\>[4]\AgdaIndent{13}{}\<[13]%
\>[13]\AgdaKeyword{record} \<[20]%
\>[20]\<%
\\
\>[4]\AgdaIndent{13}{}\<[13]%
\>[13]\AgdaSymbol{\{} \AgdaField{Carrier} \AgdaSymbol{=} \AgdaFunction{[} \AgdaFunction{[} \AgdaBound{A} \AgdaFunction{]fm} \AgdaBound{x} \AgdaFunction{]} \AgdaBound{a} \AgdaFunction{≈} \AgdaBound{b}\<%
\\
\>[4]\AgdaIndent{13}{}\<[13]%
\>[13]\AgdaSymbol{;} \AgdaField{\_≈h\_} \AgdaSymbol{=} \AgdaSymbol{λ} \AgdaBound{\_} \AgdaBound{\_} \AgdaSymbol{→} \AgdaKeyword{record} \AgdaSymbol{\{} \AgdaField{prf} \AgdaSymbol{=} \AgdaRecord{⊤} \AgdaSymbol{;} \AgdaField{Uni} \AgdaSymbol{=} \AgdaInductiveConstructor{PE.refl} \AgdaSymbol{\}}\<%
\\
\>[4]\AgdaIndent{13}{}\<[13]%
\>[13]\AgdaSymbol{;} \AgdaField{refl} \AgdaSymbol{=} \AgdaInductiveConstructor{tt} \<[25]%
\>[25]\<%
\\
\>[4]\AgdaIndent{13}{}\<[13]%
\>[13]\AgdaSymbol{;} \AgdaField{sym} \AgdaSymbol{=} \AgdaFunction{id}\<%
\\
\>[4]\AgdaIndent{13}{}\<[13]%
\>[13]\AgdaSymbol{;} \AgdaField{trans} \AgdaSymbol{=} \AgdaSymbol{λ} \AgdaBound{\_} \AgdaBound{\_} \AgdaSymbol{→} \AgdaInductiveConstructor{tt}\<%
\\
\>[4]\AgdaIndent{13}{}\<[13]%
\>[13]\AgdaSymbol{\}}\<%
\\
\>[4]\AgdaIndent{13}{}\<[13]%
\>[13]\AgdaSymbol{\}}\<%
\\
\>[0]\AgdaIndent{4}{}\<[4]%
\>[4]\AgdaSymbol{;} \AgdaField{substT} \AgdaSymbol{=} \AgdaSymbol{λ} \AgdaSymbol{\{((}\AgdaBound{x} \AgdaInductiveConstructor{,} \AgdaBound{a}\AgdaSymbol{)} \AgdaInductiveConstructor{,} \AgdaBound{b}\AgdaSymbol{)} \AgdaBound{x0} \AgdaSymbol{→} \<[37]%
\>[37]\<%
\\
\>[0]\AgdaIndent{15}{}\<[15]%
\>[15]\AgdaFunction{[} \AgdaFunction{[} \AgdaBound{A} \AgdaFunction{]fm} \AgdaSymbol{\_} \AgdaFunction{]trans} \<[34]%
\>[34]\<%
\\
\>[0]\AgdaIndent{15}{}\<[15]%
\>[15]\AgdaSymbol{(}\AgdaFunction{[} \AgdaFunction{[} \AgdaBound{A} \AgdaFunction{]fm} \AgdaSymbol{\_} \AgdaFunction{]sym} \AgdaBound{a}\AgdaSymbol{)} \<[36]%
\>[36]\<%
\\
\>[0]\AgdaIndent{15}{}\<[15]%
\>[15]\AgdaSymbol{(}\AgdaFunction{[} \AgdaFunction{[} \AgdaBound{A} \AgdaFunction{]fm} \AgdaSymbol{\_} \AgdaFunction{]trans} \<[35]%
\>[35]\<%
\\
\>[0]\AgdaIndent{15}{}\<[15]%
\>[15]\AgdaSymbol{(}\AgdaFunction{[} \AgdaBound{A} \AgdaFunction{]subst*} \AgdaSymbol{\_} \AgdaBound{x0}\AgdaSymbol{)} \AgdaBound{b}\AgdaSymbol{)} \<[37]%
\>[37]\<%
\\
\>[0]\AgdaIndent{15}{}\<[15]%
\>[15]\AgdaSymbol{\}}\<%
\\
\>[0]\AgdaIndent{4}{}\<[4]%
\>[4]\AgdaSymbol{;} \AgdaField{subst*} \AgdaSymbol{=} \AgdaSymbol{λ} \AgdaBound{\_} \AgdaBound{\_} \AgdaSymbol{→} \AgdaInductiveConstructor{tt}\<%
\\
\>[0]\AgdaIndent{4}{}\<[4]%
\>[4]\AgdaSymbol{;} \AgdaField{refl*} \AgdaSymbol{=} \AgdaSymbol{λ} \AgdaBound{\_} \AgdaBound{\_} \AgdaSymbol{→} \AgdaInductiveConstructor{tt}\<%
\\
\>[0]\AgdaIndent{4}{}\<[4]%
\>[4]\AgdaSymbol{;} \AgdaField{trans*} \AgdaSymbol{=} \AgdaSymbol{λ} \AgdaBound{\_} \AgdaSymbol{→} \AgdaInductiveConstructor{tt} \<[24]%
\>[24]\<%
\\
\>[0]\AgdaIndent{4}{}\<[4]%
\>[4]\AgdaSymbol{\}}\<%
\\
%
\\
%
\\
\>[0]\AgdaIndent{2}{}\<[2]%
\>[2]\AgdaFunction{⟦refl⟧⁰} \AgdaSymbol{:} \AgdaRecord{Tm} \AgdaSymbol{\{}\AgdaBound{Γ} \AgdaFunction{\&} \AgdaBound{A}\AgdaSymbol{\}} \AgdaSymbol{(}\AgdaFunction{⟦Id⟧} \AgdaFunction{[} \AgdaKeyword{record} \AgdaSymbol{\{} \AgdaField{fn} \AgdaSymbol{=} \AgdaSymbol{λ} \AgdaBound{x'} \AgdaSymbol{→} \AgdaBound{x'} \AgdaInductiveConstructor{,} \AgdaFunction{proj₂} \AgdaBound{x'} \<[66]%
\>[66]\<%
\\
\>[0]\AgdaIndent{23}{}\<[23]%
\>[23]\AgdaSymbol{;} \AgdaField{resp} \AgdaSymbol{=} \AgdaSymbol{λ} \AgdaBound{x'} \AgdaSymbol{→} \AgdaBound{x'} \AgdaInductiveConstructor{,} \AgdaFunction{proj₂} \AgdaBound{x'} \AgdaSymbol{\}} \AgdaFunction{]T}\AgdaSymbol{)} \<[59]%
\>[59]\<%
\\
\>[0]\AgdaIndent{2}{}\<[2]%
\>[2]\AgdaFunction{⟦refl⟧⁰} \AgdaSymbol{=} \AgdaKeyword{record}\<%
\\
\>[0]\AgdaIndent{11}{}\<[11]%
\>[11]\AgdaSymbol{\{} \AgdaField{tm} \AgdaSymbol{=} \AgdaSymbol{λ} \AgdaSymbol{\{(}\AgdaBound{x} \AgdaInductiveConstructor{,} \AgdaBound{a}\AgdaSymbol{)} \AgdaSymbol{→} \AgdaFunction{[} \AgdaFunction{[} \AgdaBound{A} \AgdaFunction{]fm} \AgdaBound{x} \AgdaFunction{]refl} \AgdaSymbol{\{}\AgdaBound{a}\AgdaSymbol{\}} \AgdaSymbol{\}}\<%
\\
\>[0]\AgdaIndent{11}{}\<[11]%
\>[11]\AgdaSymbol{;} \AgdaField{respt} \AgdaSymbol{=} \AgdaSymbol{λ} \AgdaBound{p} \AgdaSymbol{→} \AgdaInductiveConstructor{tt}\<%
\\
\>[0]\AgdaIndent{11}{}\<[11]%
\>[11]\AgdaSymbol{\}}\<%
\\
%
\\
\>[0]\AgdaIndent{2}{}\<[2]%
\>[2]\AgdaFunction{⟦refl⟧} \AgdaSymbol{=} \<[12]%
\>[12]\AgdaFunction{lam} \AgdaSymbol{\{}\AgdaBound{Γ}\AgdaSymbol{\}} \AgdaSymbol{\{}\AgdaBound{A}\AgdaSymbol{\}} \AgdaFunction{⟦refl⟧⁰}\<%
\\
%
\\
\>\<\end{code}

Subst using equality types

\begin{code}\>\<%
\\
%
\\
\>[0]\AgdaIndent{2}{}\<[2]%
\>[2]\AgdaKeyword{module} \AgdaModule{substIn} \AgdaSymbol{(}\AgdaBound{B} \AgdaSymbol{:} \AgdaRecord{Ty} \AgdaSymbol{(}\AgdaBound{Γ} \AgdaFunction{\&} \AgdaBound{A}\AgdaSymbol{))} \AgdaKeyword{where}\<%
\\
\>[0]\AgdaIndent{2}{}\<[2]%
\>[2]\<%
\\
\>[2]\AgdaIndent{4}{}\<[4]%
\>[4]\AgdaFunction{⟦subst⟧⁰} \AgdaSymbol{:} \AgdaRecord{Tm} \AgdaSymbol{\{}\AgdaBound{Γ} \AgdaFunction{\&} \AgdaBound{A} \AgdaFunction{\&} \AgdaSymbol{(}\AgdaBound{A} \AgdaFunction{[} \AgdaFunction{fst\&} \AgdaSymbol{\{}A \AgdaSymbol{=} \AgdaBound{A}\AgdaSymbol{\}} \AgdaFunction{]T}\AgdaSymbol{)} \<[49]%
\>[49]\<%
\\
\>[4]\AgdaIndent{15}{}\<[15]%
\>[15]\AgdaFunction{\&} \AgdaFunction{⟦Id⟧} \AgdaFunction{\&} \AgdaBound{B} \AgdaFunction{[} \AgdaFunction{fst\&} \AgdaSymbol{\{}A \AgdaSymbol{=} \AgdaBound{A} \AgdaFunction{[} \AgdaFunction{fst\&} \AgdaSymbol{\{}A \AgdaSymbol{=} \AgdaBound{A}\AgdaSymbol{\}} \AgdaFunction{]T}\AgdaSymbol{\}} \<[60]%
\>[60]\AgdaFunction{]T} \<[63]%
\>[63]\<%
\\
\>[4]\AgdaIndent{15}{}\<[15]%
\>[15]\AgdaFunction{[} \AgdaFunction{fst\&} \AgdaSymbol{\{}A \AgdaSymbol{=} \AgdaFunction{⟦Id⟧}\AgdaSymbol{\}} \AgdaFunction{]T}\AgdaSymbol{\}} \<[37]%
\>[37]\<%
\\
\>[-2]\AgdaIndent{13}{}\<[13]%
\>[13]\AgdaSymbol{(}\AgdaBound{B} \AgdaFunction{[} \AgdaKeyword{record} \AgdaSymbol{\{} \AgdaField{fn} \AgdaSymbol{=} \AgdaSymbol{λ} \AgdaBound{x} \AgdaSymbol{→} \AgdaSymbol{(}\AgdaFunction{proj₁} \AgdaSymbol{(}\AgdaFunction{proj₁} \AgdaSymbol{(}\AgdaFunction{proj₁} \AgdaSymbol{(}\AgdaFunction{proj₁} \AgdaBound{x}\AgdaSymbol{))))} \<[72]%
\>[72]\<%
\\
\>[0]\AgdaIndent{13}{}\<[13]%
\>[13]\AgdaInductiveConstructor{,} \AgdaSymbol{(}\AgdaFunction{proj₂} \AgdaSymbol{(}\AgdaFunction{proj₁} \AgdaSymbol{(}\AgdaFunction{proj₁} \AgdaBound{x}\AgdaSymbol{)))} \<[41]%
\>[41]\<%
\\
\>[0]\AgdaIndent{13}{}\<[13]%
\>[13]\AgdaSymbol{;} \AgdaField{resp} \AgdaSymbol{=} \AgdaSymbol{λ} \AgdaBound{x} \AgdaSymbol{→} \AgdaFunction{proj₁} \AgdaSymbol{(}\AgdaFunction{proj₁} \AgdaSymbol{(}\AgdaFunction{proj₁} \AgdaSymbol{(}\AgdaFunction{proj₁} \AgdaBound{x}\AgdaSymbol{)))} \<[60]%
\>[60]\<%
\\
\>[0]\AgdaIndent{13}{}\<[13]%
\>[13]\AgdaInductiveConstructor{,} \AgdaFunction{proj₂} \AgdaSymbol{(}\AgdaFunction{proj₁} \AgdaSymbol{(}\AgdaFunction{proj₁} \AgdaBound{x}\AgdaSymbol{))} \AgdaSymbol{\}} \AgdaFunction{]T}\AgdaSymbol{)}\<%
\\
%
\\
\>[0]\AgdaIndent{4}{}\<[4]%
\>[4]\AgdaFunction{⟦subst⟧⁰} \AgdaSymbol{=} \AgdaKeyword{record}\<%
\\
\>[0]\AgdaIndent{11}{}\<[11]%
\>[11]\AgdaSymbol{\{} \AgdaField{tm} \AgdaSymbol{=} \AgdaSymbol{λ} \AgdaSymbol{\{((((}\AgdaBound{x} \AgdaInductiveConstructor{,} \AgdaBound{a}\AgdaSymbol{)} \AgdaInductiveConstructor{,} \AgdaBound{b}\AgdaSymbol{)} \AgdaInductiveConstructor{,} \AgdaBound{p}\AgdaSymbol{)} \AgdaInductiveConstructor{,} \AgdaBound{PA}\AgdaSymbol{)} \AgdaSymbol{→} \AgdaFunction{[} \AgdaBound{B} \AgdaFunction{]subst} \<[61]%
\>[61]\<%
\\
\>[11]\AgdaIndent{18}{}\<[18]%
\>[18]\AgdaSymbol{(}\AgdaFunction{[} \AgdaBound{Γ} \AgdaFunction{]refl} \AgdaInductiveConstructor{,} \AgdaFunction{[} \AgdaFunction{[} \AgdaBound{A} \AgdaFunction{]fm} \AgdaSymbol{\_} \AgdaFunction{]trans} \<[50]%
\>[50]\<%
\\
\>[11]\AgdaIndent{18}{}\<[18]%
\>[18]\AgdaSymbol{(}\AgdaFunction{[} \AgdaBound{A} \AgdaFunction{]refl*} \AgdaSymbol{\_} \AgdaSymbol{\_)} \AgdaBound{p}\AgdaSymbol{)} \AgdaBound{PA} \AgdaSymbol{\}}\<%
\\
\>[0]\AgdaIndent{11}{}\<[11]%
\>[11]\AgdaSymbol{;} \AgdaField{respt} \AgdaSymbol{=} \AgdaSymbol{λ} \AgdaSymbol{\{((((}\AgdaBound{m} \AgdaInductiveConstructor{,} \AgdaBound{a}\AgdaSymbol{)} \AgdaInductiveConstructor{,} \AgdaBound{b}\AgdaSymbol{)} \AgdaInductiveConstructor{,} \AgdaBound{p}\AgdaSymbol{)} \AgdaInductiveConstructor{,} \AgdaBound{PA}\AgdaSymbol{)} \AgdaSymbol{→} \<[53]%
\>[53]\<%
\\
\>[0]\AgdaIndent{13}{}\<[13]%
\>[13]\AgdaFunction{[} \AgdaFunction{[} \AgdaBound{B} \AgdaFunction{]fm} \AgdaSymbol{\_} \AgdaFunction{]trans} \<[32]%
\>[32]\<%
\\
\>[0]\AgdaIndent{13}{}\<[13]%
\>[13]\AgdaSymbol{(}\AgdaFunction{[} \AgdaBound{B} \AgdaFunction{]trans*} \AgdaSymbol{\_)} \<[29]%
\>[29]\<%
\\
\>[13]\AgdaIndent{14}{}\<[14]%
\>[14]\AgdaSymbol{(}\AgdaFunction{[} \AgdaFunction{[} \AgdaBound{B} \AgdaFunction{]fm} \AgdaSymbol{\_} \AgdaFunction{]trans} \<[34]%
\>[34]\<%
\\
\>[0]\AgdaIndent{13}{}\<[13]%
\>[13]\AgdaFunction{[} \AgdaBound{B} \AgdaFunction{]subst-pi} \<[27]%
\>[27]\<%
\\
\>[0]\AgdaIndent{13}{}\<[13]%
\>[13]\AgdaSymbol{(}\AgdaFunction{[} \AgdaFunction{[} \AgdaBound{B} \AgdaFunction{]fm} \AgdaSymbol{\_} \AgdaFunction{]trans} \<[33]%
\>[33]\<%
\\
\>[0]\AgdaIndent{13}{}\<[13]%
\>[13]\AgdaSymbol{(}\AgdaFunction{[} \AgdaFunction{[} \AgdaBound{B} \AgdaFunction{]fm} \AgdaSymbol{\_} \AgdaFunction{]sym} \AgdaSymbol{(}\AgdaFunction{[} \AgdaBound{B} \AgdaFunction{]trans*} \AgdaSymbol{\_))}\<%
\\
\>[0]\AgdaIndent{13}{}\<[13]%
\>[13]\AgdaSymbol{(}\AgdaFunction{[} \AgdaBound{B} \AgdaFunction{]subst*} \AgdaSymbol{\_} \AgdaBound{PA}\AgdaSymbol{)} \AgdaSymbol{))} \AgdaSymbol{\}}\<%
\\
\>[0]\AgdaIndent{11}{}\<[11]%
\>[11]\AgdaSymbol{\}}\<%
\\
\>\<\end{code}


\AgdaHide{
\begin{code}\>\<%
\\
%
\\
\>\AgdaComment{--    ⟦subst⟧ = lam (lam (lam ⟦subst⟧⁰))}\<%
\\
\>[0]\AgdaIndent{4}{}\<[4]%
\>[4]\<%
\\
\>\<\end{code}
}

%
\AgdaHide{
\begin{code}\>\<%
\\
%
\\
\>\AgdaSymbol{\{-\#} \AgdaKeyword{OPTIONS} --type-in-type \AgdaSymbol{\#-\}}\<%
\\
%
\\
\>\AgdaKeyword{import} \AgdaModule{Level}\<%
\\
\>\AgdaKeyword{open} \AgdaKeyword{import} \AgdaModule{Relation.Binary.PropositionalEquality} \AgdaSymbol{as} \AgdaModule{PE} \AgdaKeyword{hiding} \AgdaSymbol{(}refl \AgdaSymbol{;} sym \AgdaSymbol{;} trans\AgdaSymbol{;} isEquivalence\AgdaSymbol{;} [\_]\AgdaSymbol{)}\<%
\\
%
\\
\>\AgdaKeyword{module} \AgdaModule{CwF-quotient} \AgdaSymbol{(}\AgdaBound{ext} \AgdaSymbol{:} \AgdaFunction{Extensionality} \AgdaPrimitive{Level.zero} \AgdaPrimitive{Level.zero}\AgdaSymbol{)} \AgdaKeyword{where}\<%
\\
%
\\
\>\AgdaKeyword{open} \AgdaKeyword{import} \AgdaModule{Data.Unit}\<%
\\
\>\AgdaKeyword{open} \AgdaKeyword{import} \AgdaModule{Function}\<%
\\
\>\AgdaKeyword{open} \AgdaKeyword{import} \AgdaModule{Data.Product}\<%
\\
%
\\
%
\\
\>\AgdaComment{-- importing other CWF files}\<%
\\
%
\\
\>\AgdaKeyword{import} \AgdaModule{CwF-setoid}\<%
\\
%
\\
\>\AgdaKeyword{open} \AgdaModule{CwF-setoid} \AgdaBound{ext}\<%
\\
%
\\
\>\AgdaKeyword{import} \AgdaModule{CategoryOfSetoid}\<%
\\
\>\AgdaKeyword{module} \AgdaModule{cos'} \AgdaSymbol{=} \AgdaModule{CategoryOfSetoid} \AgdaBound{ext}\<%
\\
\>\AgdaKeyword{open} \AgdaModule{cos'}\<%
\\
%
\\
\>\AgdaKeyword{import} \AgdaModule{hProp}\<%
\\
\>\AgdaKeyword{module} \AgdaModule{hp'} \AgdaSymbol{=} \AgdaModule{hProp} \AgdaBound{ext}\<%
\\
\>\AgdaKeyword{open} \AgdaModule{hp'}\<%
\\
%
\\
\>\AgdaKeyword{import} \AgdaModule{CwF-ctd}\<%
\\
\>\AgdaKeyword{module} \AgdaModule{cc} \AgdaSymbol{=} \AgdaModule{CwF-ctd} \AgdaBound{ext}\<%
\\
\>\AgdaKeyword{open} \AgdaModule{cc}\<%
\\
%
\\
%
\\
\>\<\end{code}
}


The equality type is an essential part of a type theory. We could define it by using the equivalence relation from the setoid representation of type A. The equivalence relation is trivial since it is proof-irrelevant.

\begin{code}\>\<%
\\
%
\\
\>\AgdaFunction{Rel} \AgdaSymbol{:} \AgdaSymbol{\{}\AgdaBound{Γ} \AgdaSymbol{:} \AgdaFunction{Con}\AgdaSymbol{\}} \AgdaSymbol{→} \AgdaRecord{Ty} \AgdaBound{Γ} \AgdaSymbol{→} \AgdaPrimitiveType{Set₁}\<%
\\
\>\AgdaFunction{Rel} \AgdaSymbol{\{}\AgdaBound{Γ}\AgdaSymbol{\}} \AgdaBound{A} \AgdaSymbol{=} \AgdaRecord{Ty} \AgdaSymbol{(}\AgdaBound{Γ} \AgdaFunction{\&} \AgdaBound{A} \AgdaFunction{\&} \AgdaBound{A} \AgdaFunction{+T} \AgdaBound{A}\AgdaSymbol{)}\<%
\\
%
\\
\>\AgdaFunction{⟦Id⟧} \AgdaSymbol{:} \AgdaSymbol{\{}\AgdaBound{Γ} \AgdaSymbol{:} \AgdaFunction{Con}\AgdaSymbol{\}(}\AgdaBound{A} \AgdaSymbol{:} \AgdaRecord{Ty} \AgdaBound{Γ}\AgdaSymbol{)} \AgdaSymbol{→} \AgdaFunction{Rel} \AgdaBound{A}\<%
\\
\>\AgdaFunction{⟦Id⟧} \AgdaBound{A}\<%
\\
\>[0]\AgdaIndent{3}{}\<[3]%
\>[3]\AgdaSymbol{=} \AgdaKeyword{record} \<[12]%
\>[12]\<%
\\
\>[3]\AgdaIndent{7}{}\<[7]%
\>[7]\AgdaSymbol{\{} \AgdaField{fm} \AgdaSymbol{=} \AgdaSymbol{λ} \AgdaSymbol{\{((}\AgdaBound{x} \AgdaInductiveConstructor{,} \AgdaBound{a}\AgdaSymbol{)} \AgdaInductiveConstructor{,} \AgdaBound{b}\AgdaSymbol{)} \AgdaSymbol{→} \<[34]%
\>[34]\AgdaKeyword{record}\<%
\\
\>[7]\AgdaIndent{9}{}\<[9]%
\>[9]\AgdaSymbol{\{} \AgdaField{Carrier} \AgdaSymbol{=} \AgdaFunction{[} \AgdaFunction{[} \AgdaBound{A} \AgdaFunction{]fm} \AgdaBound{x} \AgdaFunction{]} \AgdaBound{a} \AgdaFunction{≈} \AgdaBound{b}\<%
\\
\>[7]\AgdaIndent{9}{}\<[9]%
\>[9]\AgdaSymbol{;} \AgdaField{\_≈h\_} \AgdaSymbol{=} \AgdaSymbol{λ} \AgdaBound{x₁} \AgdaBound{x₂} \AgdaSymbol{→} \AgdaFunction{⊤'}\<%
\\
\>[7]\AgdaIndent{9}{}\<[9]%
\>[9]\AgdaSymbol{;} \AgdaField{isEquiv} \AgdaSymbol{=} \AgdaKeyword{record}\<%
\\
\>[9]\AgdaIndent{13}{}\<[13]%
\>[13]\AgdaSymbol{\{} \AgdaField{refl} \AgdaSymbol{=} \AgdaSymbol{λ} \AgdaSymbol{\{}\AgdaBound{x₁}\AgdaSymbol{\}} \AgdaSymbol{→} \AgdaInductiveConstructor{tt}\<%
\\
\>[9]\AgdaIndent{13}{}\<[13]%
\>[13]\AgdaSymbol{;} \AgdaField{sym} \AgdaSymbol{=} \AgdaSymbol{λ} \AgdaBound{x₂} \AgdaSymbol{→} \AgdaInductiveConstructor{tt}\<%
\\
\>[9]\AgdaIndent{13}{}\<[13]%
\>[13]\AgdaSymbol{;} \AgdaField{trans} \AgdaSymbol{=} \AgdaSymbol{λ} \AgdaBound{x₂} \AgdaBound{x₃} \AgdaSymbol{→} \AgdaInductiveConstructor{tt}\<%
\\
\>[9]\AgdaIndent{13}{}\<[13]%
\>[13]\AgdaSymbol{\}}\<%
\\
\>[0]\AgdaIndent{9}{}\<[9]%
\>[9]\AgdaSymbol{\}} \AgdaSymbol{\}}\<%
\\
\>[0]\AgdaIndent{7}{}\<[7]%
\>[7]\AgdaSymbol{;} \AgdaField{substT} \AgdaSymbol{=} \AgdaSymbol{λ} \AgdaSymbol{\{((}\AgdaBound{x} \AgdaInductiveConstructor{,} \AgdaBound{a}\AgdaSymbol{)} \AgdaInductiveConstructor{,} \AgdaBound{b}\AgdaSymbol{)} \AgdaBound{x0} \AgdaSymbol{→} \<[40]%
\>[40]\<%
\\
\>[7]\AgdaIndent{15}{}\<[15]%
\>[15]\AgdaFunction{[} \AgdaFunction{[} \AgdaBound{A} \AgdaFunction{]fm} \AgdaSymbol{\_} \AgdaFunction{]trans} \<[34]%
\>[34]\<%
\\
\>[7]\AgdaIndent{15}{}\<[15]%
\>[15]\AgdaSymbol{(}\AgdaFunction{[} \AgdaFunction{[} \AgdaBound{A} \AgdaFunction{]fm} \AgdaSymbol{\_} \AgdaFunction{]sym} \AgdaBound{a}\AgdaSymbol{)} \<[36]%
\>[36]\<%
\\
\>[7]\AgdaIndent{15}{}\<[15]%
\>[15]\AgdaSymbol{(}\AgdaFunction{[} \AgdaFunction{[} \AgdaBound{A} \AgdaFunction{]fm} \AgdaSymbol{\_} \AgdaFunction{]trans} \<[35]%
\>[35]\<%
\\
\>[7]\AgdaIndent{15}{}\<[15]%
\>[15]\AgdaSymbol{(}\AgdaFunction{[} \AgdaBound{A} \AgdaFunction{]subst*} \AgdaSymbol{\_} \AgdaBound{x0}\AgdaSymbol{)} \AgdaBound{b}\AgdaSymbol{)} \<[37]%
\>[37]\<%
\\
\>[7]\AgdaIndent{15}{}\<[15]%
\>[15]\AgdaSymbol{\}}\<%
\\
\>[0]\AgdaIndent{7}{}\<[7]%
\>[7]\AgdaSymbol{;} \AgdaField{subst*} \AgdaSymbol{=} \AgdaSymbol{λ} \AgdaBound{p} \AgdaBound{x₁} \AgdaSymbol{→} \AgdaInductiveConstructor{tt}\<%
\\
\>[0]\AgdaIndent{7}{}\<[7]%
\>[7]\AgdaSymbol{;} \AgdaField{refl*} \AgdaSymbol{=} \AgdaSymbol{λ} \AgdaBound{x} \AgdaBound{a} \AgdaSymbol{→} \AgdaInductiveConstructor{tt}\<%
\\
\>[0]\AgdaIndent{7}{}\<[7]%
\>[7]\AgdaSymbol{;} \AgdaField{trans*} \AgdaSymbol{=} \AgdaSymbol{λ} \AgdaBound{p} \AgdaBound{q} \AgdaBound{a} \AgdaSymbol{→} \AgdaInductiveConstructor{tt} \AgdaSymbol{\}}\<%
\\
%
\\
\>\<\end{code}

The unique inhabitant $refl$ is defined as

\begin{code}\>\<%
\\
%
\\
%
\\
\>\AgdaFunction{cm-refl} \AgdaSymbol{:} \AgdaSymbol{\{}\AgdaBound{Γ} \AgdaSymbol{:} \AgdaFunction{Con}\AgdaSymbol{\}(}\AgdaBound{A} \AgdaSymbol{:} \AgdaRecord{Ty} \AgdaBound{Γ}\AgdaSymbol{)} \AgdaSymbol{→} \AgdaBound{Γ} \AgdaFunction{\&} \AgdaBound{A} \AgdaRecord{⇉} \AgdaSymbol{(}\AgdaBound{Γ} \AgdaFunction{\&} \AgdaBound{A} \AgdaFunction{\&} \AgdaBound{A} \AgdaFunction{+T} \AgdaBound{A}\AgdaSymbol{)}\<%
\\
\>\AgdaFunction{cm-refl} \AgdaBound{A} \AgdaSymbol{=} \AgdaKeyword{record} \AgdaSymbol{\{} \AgdaField{fn} \AgdaSymbol{=} \AgdaSymbol{λ} \AgdaBound{x'} \AgdaSymbol{→} \AgdaBound{x'} \AgdaInductiveConstructor{,} \AgdaFunction{proj₂} \AgdaBound{x'} \<[47]%
\>[47]\<%
\\
\>[7]\AgdaIndent{19}{}\<[19]%
\>[19]\AgdaSymbol{;} \AgdaField{resp} \AgdaSymbol{=} \AgdaSymbol{λ} \AgdaBound{x'} \AgdaSymbol{→} \AgdaBound{x'} \AgdaInductiveConstructor{,} \AgdaFunction{proj₂} \AgdaBound{x'} \AgdaSymbol{\}}\<%
\\
%
\\
\>\AgdaFunction{⟦refl⟧⁰} \AgdaSymbol{:} \AgdaSymbol{\{}\AgdaBound{Γ} \AgdaSymbol{:} \AgdaFunction{Con}\AgdaSymbol{\}(}\AgdaBound{A} \AgdaSymbol{:} \AgdaRecord{Ty} \AgdaBound{Γ}\AgdaSymbol{)} \<[30]%
\>[30]\<%
\\
\>[0]\AgdaIndent{7}{}\<[7]%
\>[7]\AgdaSymbol{→} \AgdaRecord{Tm} \AgdaSymbol{\{}\AgdaBound{Γ} \AgdaFunction{\&} \AgdaBound{A}\AgdaSymbol{\}} \AgdaSymbol{(}\AgdaFunction{⟦Id⟧} \AgdaBound{A}\<%
\\
\>[0]\AgdaIndent{10}{}\<[10]%
\>[10]\AgdaFunction{[} \AgdaFunction{cm-refl} \AgdaBound{A} \AgdaFunction{]T}\AgdaSymbol{)} \<[26]%
\>[26]\<%
\\
\>\AgdaFunction{⟦refl⟧⁰} \AgdaBound{A} \AgdaSymbol{=} \AgdaKeyword{record}\<%
\\
\>[10]\AgdaIndent{11}{}\<[11]%
\>[11]\AgdaSymbol{\{} \AgdaField{tm} \AgdaSymbol{=} \AgdaSymbol{λ} \AgdaSymbol{\{(}\AgdaBound{x} \AgdaInductiveConstructor{,} \AgdaBound{a}\AgdaSymbol{)} \AgdaSymbol{→} \AgdaFunction{[} \AgdaFunction{[} \AgdaBound{A} \AgdaFunction{]fm} \AgdaBound{x} \AgdaFunction{]refl} \AgdaSymbol{\{}\AgdaBound{a}\AgdaSymbol{\}} \AgdaSymbol{\}}\<%
\\
\>[10]\AgdaIndent{11}{}\<[11]%
\>[11]\AgdaSymbol{;} \AgdaField{respt} \AgdaSymbol{=} \AgdaSymbol{λ} \AgdaBound{p} \AgdaSymbol{→} \AgdaInductiveConstructor{tt}\<%
\\
\>[10]\AgdaIndent{11}{}\<[11]%
\>[11]\AgdaSymbol{\}}\<%
\\
%
\\
\>\AgdaFunction{⟦refl⟧} \AgdaSymbol{:} \AgdaSymbol{\{}\AgdaBound{Γ} \AgdaSymbol{:} \AgdaFunction{Con}\AgdaSymbol{\}(}\AgdaBound{A} \AgdaSymbol{:} \AgdaRecord{Ty} \AgdaBound{Γ}\AgdaSymbol{)} \<[29]%
\>[29]\<%
\\
\>[-6]\AgdaIndent{7}{}\<[7]%
\>[7]\AgdaSymbol{→} \AgdaRecord{Tm} \AgdaSymbol{\{}\AgdaBound{Γ}\AgdaSymbol{\}} \AgdaSymbol{(}\AgdaFunction{Π} \AgdaBound{A} \AgdaSymbol{(}\AgdaFunction{⟦Id⟧} \AgdaBound{A} \<[29]%
\>[29]\<%
\\
\>[0]\AgdaIndent{10}{}\<[10]%
\>[10]\AgdaFunction{[} \AgdaFunction{cm-refl} \AgdaBound{A} \AgdaFunction{]T}\AgdaSymbol{)} \AgdaSymbol{)}\<%
\\
\>\AgdaFunction{⟦refl⟧} \AgdaSymbol{\{}\AgdaBound{Γ}\AgdaSymbol{\}} \AgdaBound{A} \AgdaSymbol{=} \<[16]%
\>[16]\AgdaFunction{lam} \AgdaSymbol{\{}\AgdaBound{Γ}\AgdaSymbol{\}} \AgdaSymbol{\{}\AgdaBound{A}\AgdaSymbol{\}} \AgdaSymbol{(}\AgdaFunction{⟦refl⟧⁰} \AgdaBound{A}\AgdaSymbol{)}\<%
\\
%
\\
%
\\
\>\<\end{code}

We have an abstracted $refl$ term as well. Using $\Pi$-types we could define the eliminator for $Id$, but it is more involved.

We have done the basics for category of families of setoids. There are more types can be interpreted in this model so that we could show that it is a valid model for Type Theory. We would like to interpret quotient types in this model by following Hofmann's method in \cite{hof:95:sm} or by ourselves.


\AgdaHide{
\begin{code}\>\<%
\\
\>\AgdaComment{\{-

-- substIn (B : Ty (Γ \& A))

⟦subst⟧⁰ : \{Γ : Con\}(A : Ty Γ)(B : Ty (Γ \& A)) → 
           Tm \{Γ \& A \& (A [ fst\& A ]T) 
           \& (⟦Id⟧ A) \& B [ fst\& (A [ fst\& A ]T) ]T [ fst\& (⟦Id⟧ A) ]T\} 
         (B [ record \{ fn = λ x → (proj₁ (proj₁ (proj₁ (proj₁ x)))) , (proj₂ (proj₁ (proj₁ x))) ; resp = λ x → proj₁ (proj₁ (proj₁ (proj₁ x))) , proj₂ (proj₁ (proj₁ x)) \} ]T)

⟦subst⟧⁰ \{Γ\} A B = record
       \{ tm = λ \{((((x , a) , b) , p) , PA) → [ B ]subst ([ Γ ]refl , [ [ A ]fm \_ ]trans ([ A ]refl* \_ \_) p) PA \}
       ; respt = λ \{((((m , a) , b) , p) , PA) → 
         [ [ B ]fm \_ ]trans 
         ([ B ]trans* \_ \_ \_) 
          ([ [ B ]fm \_ ]trans 
         [ B ]subst-pi 
         ([ [ B ]fm \_ ]trans 
         ([ [ B ]fm \_ ]sym ([ B ]trans* \_ \_ \_))
         ([ B ]subst* \_ PA) )) \}
       \}


-\}}\<%
\\
%
\\
\>\AgdaComment{-- The mechanism used in Martin Hofmann's Paper}\<%
\\
%
\\
\>\AgdaKeyword{record} \AgdaRecord{Prop-Uni} \AgdaSymbol{(}\AgdaBound{Γ} \AgdaSymbol{:} \AgdaFunction{Con}\AgdaSymbol{)} \AgdaSymbol{:} \AgdaPrimitiveType{Set} \AgdaKeyword{where}\<%
\\
\>[0]\AgdaIndent{2}{}\<[2]%
\>[2]\AgdaKeyword{field}\<%
\\
%
\\
\>[0]\AgdaIndent{4}{}\<[4]%
\>[4]\AgdaField{prf} \AgdaSymbol{:} \AgdaRecord{Ty} \AgdaBound{Γ}\<%
\\
\>[0]\AgdaIndent{4}{}\<[4]%
\>[4]\AgdaField{uni} \AgdaSymbol{:} \AgdaSymbol{∀} \AgdaBound{γ} \AgdaBound{x} \AgdaBound{y} \AgdaSymbol{→} \AgdaFunction{[} \AgdaFunction{[} \AgdaBound{prf} \AgdaFunction{]fm} \AgdaBound{γ} \AgdaFunction{]} \AgdaBound{x} \AgdaFunction{≈h} \AgdaBound{y} \AgdaDatatype{≡} \AgdaFunction{⊤'}\<%
\\
\>\AgdaKeyword{open} \AgdaModule{Prop-Uni}\<%
\\
%
\\
\>\AgdaComment{-- Is it correct to write  Tm A → Tm B for dependent types?}\<%
\\
%
\\
%
\\
%
\\
\>\AgdaFunction{Id-is-prop} \AgdaSymbol{:} \AgdaSymbol{\{}\AgdaBound{Γ} \AgdaSymbol{:} \AgdaFunction{Con}\AgdaSymbol{\}(}\AgdaBound{A} \AgdaSymbol{:} \AgdaRecord{Ty} \AgdaBound{Γ}\AgdaSymbol{)} \AgdaSymbol{→} \AgdaRecord{Prop-Uni} \AgdaSymbol{(}\AgdaBound{Γ} \AgdaFunction{\&} \AgdaBound{A} \AgdaFunction{\&} \AgdaSymbol{(}\AgdaBound{A} \AgdaFunction{[} \AgdaFunction{fst\&} \AgdaBound{A} \AgdaFunction{]T}\AgdaSymbol{))}\<%
\\
\>\AgdaFunction{Id-is-prop} \AgdaBound{A} \AgdaSymbol{=} \AgdaKeyword{record} \AgdaSymbol{\{} \AgdaField{prf} \AgdaSymbol{=} \AgdaFunction{⟦Id⟧} \AgdaBound{A} \AgdaSymbol{;} \AgdaField{uni} \AgdaSymbol{=} \AgdaSymbol{λ} \AgdaBound{γ} \AgdaBound{x} \AgdaBound{y} \AgdaSymbol{→} \AgdaInductiveConstructor{PE.refl} \AgdaSymbol{\}}\<%
\\
%
\\
\>\AgdaComment{\{-
record Quo \{Γ : Con\}(A : Ty Γ)(R : Prop-Uni (Γ \& A \& (A [ fst\& \{Γ\} \{A\} ]T))) : Set where
  field
    Q : Ty Γ
    [\_] : Tm A → Tm Q
    Q-elim : ∀ (B : Ty Γ)
                 (M : Tm \{Γ \& A\} (B [ fst\& \{Γ\} \{A\} ]T))
                 (N : Tm Q)
                 (H : Tm \{Γ \& A \& A [ fst\& \{Γ\} \{A\} ]T \& prf R\} (prf (Id-is-prop B) [ fst\& \{Γ \& A \& A [ fst\& \{Γ\} \{A\} ]T\} \{prf R\} ]T)) -- (prf (Id-is-prop (B [ fst\& \{Γ\} \{A\} ]T)))
               → Tm B

-\}}\<%
\\
%
\\
%
\\
%
\\
\>\<\end{code}
}

\chapter{syntactic \wog}\label{app:wog}


\AgdaHide{
\begin{code}\>\<%
\\
\>\AgdaSymbol{\{-\#} \AgdaKeyword{OPTIONS} --type-in-type --no-positivity-check --no-termination-check \AgdaSymbol{\#-\}}\<%
\\
%
\\
%
\\
\>\AgdaKeyword{module} \AgdaModule{BasicSyntax} \AgdaKeyword{where} \<[25]%
\>[25]\<%
\\
%
\\
%
\\
\>\AgdaKeyword{open} \AgdaKeyword{import} \AgdaModule{Relation.Binary.PropositionalEquality}\<%
\\
\>\AgdaKeyword{open} \AgdaKeyword{import} \AgdaModule{Function}\<%
\\
\>\AgdaKeyword{open} \AgdaKeyword{import} \AgdaModule{Data.Product} \AgdaKeyword{renaming} \AgdaSymbol{(}\_,\_ \AgdaSymbol{to} \_,,\_\AgdaSymbol{)}\<%
\\
%
\\
%
\\
\>\AgdaKeyword{infix} \AgdaNumber{4} \_≅\_\<%
\\
\>\AgdaKeyword{infix} \AgdaNumber{4} \_=h\_\<%
\\
\>\AgdaKeyword{infixl} \AgdaNumber{6} \_+T\_ \_+S\_ \_+tm\_\<%
\\
\>\AgdaKeyword{infixl} \AgdaNumber{5} \_,\_\<%
\\
\>\AgdaKeyword{infixl} \AgdaNumber{7} \_⊚\_\<%
\\
%
\\
\>\<\end{code}
}

\section{Syntax of \tig}

\begin{code}\>\<%
\\
\>\AgdaKeyword{data} \AgdaDatatype{Con} \<[19]%
\>[19]\AgdaSymbol{:} \AgdaPrimitiveType{Set}\<%
\\
\>\AgdaKeyword{data} \AgdaDatatype{Ty} \AgdaSymbol{(}\AgdaBound{Γ} \AgdaSymbol{:} \AgdaDatatype{Con}\AgdaSymbol{)} \<[19]%
\>[19]\AgdaSymbol{:} \AgdaPrimitiveType{Set}\<%
\\
\>\AgdaKeyword{data} \AgdaDatatype{Tm} \<[19]%
\>[19]\AgdaSymbol{:} \AgdaSymbol{\{}\AgdaBound{Γ} \AgdaSymbol{:} \AgdaDatatype{Con}\AgdaSymbol{\}(}\AgdaBound{A} \AgdaSymbol{:} \AgdaDatatype{Ty} \AgdaBound{Γ}\AgdaSymbol{)} \AgdaSymbol{→} \AgdaPrimitiveType{Set}\<%
\\
\>\AgdaKeyword{data} \AgdaDatatype{Var} \<[19]%
\>[19]\AgdaSymbol{:} \AgdaSymbol{\{}\AgdaBound{Γ} \AgdaSymbol{:} \AgdaDatatype{Con}\AgdaSymbol{\}(}\AgdaBound{A} \AgdaSymbol{:} \AgdaDatatype{Ty} \AgdaBound{Γ}\AgdaSymbol{)} \AgdaSymbol{→} \AgdaPrimitiveType{Set}\<%
\\
\>\AgdaKeyword{data} \AgdaDatatype{\_⇒\_} \<[19]%
\>[19]\AgdaSymbol{:} \AgdaDatatype{Con} \AgdaSymbol{→} \AgdaDatatype{Con} \AgdaSymbol{→} \AgdaPrimitiveType{Set}\<%
\\
\>\AgdaKeyword{data} \AgdaDatatype{isContr} \<[19]%
\>[19]\AgdaSymbol{:} \AgdaDatatype{Con} \AgdaSymbol{→} \AgdaPrimitiveType{Set}\<%
\\
\>\<\end{code}

\textbf{Contexts}

\begin{code}\>\<%
\\
\>\AgdaKeyword{data} \AgdaDatatype{Con} \AgdaKeyword{where}\<%
\\
\>[0]\AgdaIndent{2}{}\<[2]%
\>[2]\AgdaInductiveConstructor{ε} \<[8]%
\>[8]\AgdaSymbol{:} \AgdaDatatype{Con}\<%
\\
\>[0]\AgdaIndent{2}{}\<[2]%
\>[2]\AgdaInductiveConstructor{\_,\_} \<[8]%
\>[8]\AgdaSymbol{:} \AgdaSymbol{(}\AgdaBound{Γ} \AgdaSymbol{:} \AgdaDatatype{Con}\AgdaSymbol{)(}\AgdaBound{A} \AgdaSymbol{:} \AgdaDatatype{Ty} \AgdaBound{Γ}\AgdaSymbol{)} \AgdaSymbol{→} \AgdaDatatype{Con}\<%
\\
\>\<\end{code}


\textbf{Types}

\begin{code}\>\<%
\\
\>\AgdaKeyword{data} \AgdaDatatype{Ty} \AgdaBound{Γ} \AgdaKeyword{where}\<%
\\
\>[0]\AgdaIndent{2}{}\<[2]%
\>[2]\AgdaInductiveConstructor{*} \<[8]%
\>[8]\AgdaSymbol{:} \AgdaDatatype{Ty} \AgdaBound{Γ}\<%
\\
\>[0]\AgdaIndent{2}{}\<[2]%
\>[2]\AgdaInductiveConstructor{\_=h\_} \<[8]%
\>[8]\AgdaSymbol{:} \AgdaSymbol{\{}\AgdaBound{A} \AgdaSymbol{:} \AgdaDatatype{Ty} \AgdaBound{Γ}\AgdaSymbol{\}(}\AgdaBound{a} \AgdaBound{b} \AgdaSymbol{:} \AgdaDatatype{Tm} \AgdaBound{A}\AgdaSymbol{)} \AgdaSymbol{→} \AgdaDatatype{Ty} \AgdaBound{Γ}\<%
\\
\>\<\end{code}

\textbf{Heterogeneous Equality for Terms}

\begin{code}\>\<%
\\
\>\AgdaKeyword{data} \AgdaDatatype{\_≅\_} \AgdaSymbol{\{}\AgdaBound{Γ} \AgdaSymbol{:} \AgdaDatatype{Con}\AgdaSymbol{\}\{}\AgdaBound{A} \AgdaSymbol{:} \AgdaDatatype{Ty} \AgdaBound{Γ}\AgdaSymbol{\}} \AgdaSymbol{:}\<%
\\
\>[2]\AgdaIndent{9}{}\<[9]%
\>[9]\AgdaSymbol{\{}\AgdaBound{B} \AgdaSymbol{:} \AgdaDatatype{Ty} \AgdaBound{Γ}\AgdaSymbol{\}} \AgdaSymbol{→} \AgdaDatatype{Tm} \AgdaBound{A} \AgdaSymbol{→} \AgdaDatatype{Tm} \AgdaBound{B} \AgdaSymbol{→} \AgdaPrimitiveType{Set} \AgdaKeyword{where}\<%
\\
\>[0]\AgdaIndent{2}{}\<[2]%
\>[2]\AgdaInductiveConstructor{refl} \AgdaSymbol{:} \AgdaSymbol{(}\AgdaBound{b} \AgdaSymbol{:} \AgdaDatatype{Tm} \AgdaBound{A}\AgdaSymbol{)} \AgdaSymbol{→} \AgdaBound{b} \AgdaDatatype{≅} \AgdaBound{b}\<%
\\
%
\\
\>\AgdaFunction{\_-¹} \<[13]%
\>[13]\AgdaSymbol{:} \AgdaSymbol{∀\{}\AgdaBound{Γ} \AgdaSymbol{:} \AgdaDatatype{Con}\AgdaSymbol{\}\{}\AgdaBound{A} \AgdaBound{B} \AgdaSymbol{:} \AgdaDatatype{Ty} \AgdaBound{Γ}\AgdaSymbol{\}}\<%
\\
\>[0]\AgdaIndent{15}{}\<[15]%
\>[15]\AgdaSymbol{\{}\AgdaBound{a} \AgdaSymbol{:} \AgdaDatatype{Tm} \AgdaBound{A}\AgdaSymbol{\}\{}\AgdaBound{b} \AgdaSymbol{:} \AgdaDatatype{Tm} \AgdaBound{B}\AgdaSymbol{\}} \AgdaSymbol{→} \AgdaBound{a} \AgdaDatatype{≅} \AgdaBound{b} \AgdaSymbol{→} \AgdaBound{b} \AgdaDatatype{≅} \AgdaBound{a}\<%
\\
\>\AgdaSymbol{(}\AgdaInductiveConstructor{refl} \AgdaSymbol{\_)} \AgdaFunction{-¹} \<[13]%
\>[13]\AgdaSymbol{=} \AgdaInductiveConstructor{refl} \AgdaSymbol{\_}\<%
\\
%
\\
\>\AgdaKeyword{infixr} \AgdaNumber{4} \_∾\_ \<%
\\
%
\\
\>\AgdaFunction{\_∾\_} \AgdaSymbol{:} \AgdaSymbol{\{}\AgdaBound{Γ} \AgdaSymbol{:} \AgdaDatatype{Con}\AgdaSymbol{\}}\<%
\\
\>[0]\AgdaIndent{6}{}\<[6]%
\>[6]\AgdaSymbol{\{}\AgdaBound{A} \AgdaBound{B} \AgdaBound{C} \AgdaSymbol{:} \AgdaDatatype{Ty} \AgdaBound{Γ}\AgdaSymbol{\}}\<%
\\
\>[0]\AgdaIndent{6}{}\<[6]%
\>[6]\AgdaSymbol{\{}\AgdaBound{a} \AgdaSymbol{:} \AgdaDatatype{Tm} \AgdaBound{A}\AgdaSymbol{\}\{}\AgdaBound{b} \AgdaSymbol{:} \AgdaDatatype{Tm} \AgdaBound{B}\AgdaSymbol{\}\{}\AgdaBound{c} \AgdaSymbol{:} \AgdaDatatype{Tm} \AgdaBound{C}\AgdaSymbol{\}} \AgdaSymbol{→}\<%
\\
\>[0]\AgdaIndent{6}{}\<[6]%
\>[6]\AgdaBound{a} \AgdaDatatype{≅} \AgdaBound{b} \AgdaSymbol{→} \<[14]%
\>[14]\<%
\\
\>[0]\AgdaIndent{6}{}\<[6]%
\>[6]\AgdaBound{b} \AgdaDatatype{≅} \AgdaBound{c} \<[12]%
\>[12]\<%
\\
\>[0]\AgdaIndent{4}{}\<[4]%
\>[4]\AgdaSymbol{→} \AgdaBound{a} \AgdaDatatype{≅} \AgdaBound{c}\<%
\\
\>\AgdaFunction{\_∾\_} \AgdaSymbol{\{}c \AgdaSymbol{=} \AgdaBound{c}\AgdaSymbol{\}} \AgdaSymbol{(}\AgdaInductiveConstructor{refl} \AgdaSymbol{.}\AgdaBound{c}\AgdaSymbol{)} \AgdaSymbol{(}\AgdaInductiveConstructor{refl} \AgdaSymbol{.}\AgdaBound{c}\AgdaSymbol{)} \AgdaSymbol{=} \AgdaInductiveConstructor{refl} \AgdaBound{c}\<%
\\
%
\\
\>\AgdaFunction{\_⟦\_⟫} \<[12]%
\>[12]\AgdaSymbol{:} \AgdaSymbol{\{}\AgdaBound{Γ} \AgdaSymbol{:} \AgdaDatatype{Con}\AgdaSymbol{\}\{}\AgdaBound{A} \AgdaBound{B} \AgdaSymbol{:} \AgdaDatatype{Ty} \AgdaBound{Γ}\AgdaSymbol{\}(}\AgdaBound{a} \AgdaSymbol{:} \AgdaDatatype{Tm} \AgdaBound{B}\AgdaSymbol{)} \<[46]%
\>[46]\<%
\\
\>[0]\AgdaIndent{12}{}\<[12]%
\>[12]\AgdaSymbol{→} \AgdaBound{A} \AgdaDatatype{≡} \AgdaBound{B} \AgdaSymbol{→} \AgdaDatatype{Tm} \AgdaBound{A}\<%
\\
\>\AgdaBound{a} \AgdaFunction{⟦} \AgdaInductiveConstructor{refl} \AgdaFunction{⟫} \<[12]%
\>[12]\AgdaSymbol{=} \AgdaBound{a}\<%
\\
%
\\
\>\AgdaFunction{cohOp} \<[12]%
\>[12]\AgdaSymbol{:} \AgdaSymbol{\{}\AgdaBound{Γ} \AgdaSymbol{:} \AgdaDatatype{Con}\AgdaSymbol{\}\{}\AgdaBound{A} \AgdaBound{B} \AgdaSymbol{:} \AgdaDatatype{Ty} \AgdaBound{Γ}\AgdaSymbol{\}\{}\AgdaBound{a} \AgdaSymbol{:} \AgdaDatatype{Tm} \AgdaBound{B}\AgdaSymbol{\}(}\AgdaBound{p} \AgdaSymbol{:} \AgdaBound{A} \AgdaDatatype{≡} \AgdaBound{B}\AgdaSymbol{)} \<[57]%
\>[57]\<%
\\
\>[0]\AgdaIndent{12}{}\<[12]%
\>[12]\AgdaSymbol{→} \AgdaBound{a} \AgdaFunction{⟦} \AgdaBound{p} \AgdaFunction{⟫} \AgdaDatatype{≅} \AgdaBound{a}\<%
\\
\>\AgdaFunction{cohOp} \AgdaInductiveConstructor{refl} \<[12]%
\>[12]\AgdaSymbol{=} \AgdaInductiveConstructor{refl} \AgdaSymbol{\_}\<%
\\
\>\<\end{code}

\begin{code}\>\<%
\\
\>\AgdaFunction{cohOp-eq} \AgdaSymbol{:} \AgdaSymbol{\{}\AgdaBound{Γ} \AgdaSymbol{:} \AgdaDatatype{Con}\AgdaSymbol{\}\{}\AgdaBound{A} \AgdaBound{B} \AgdaSymbol{:} \AgdaDatatype{Ty} \AgdaBound{Γ}\AgdaSymbol{\}\{}\AgdaBound{a} \AgdaBound{b} \AgdaSymbol{:} \AgdaDatatype{Tm} \AgdaBound{B}\AgdaSymbol{\}}\<%
\\
\>[0]\AgdaIndent{11}{}\<[11]%
\>[11]\AgdaSymbol{\{}\AgdaBound{p} \AgdaSymbol{:} \AgdaBound{A} \AgdaDatatype{≡} \AgdaBound{B}\AgdaSymbol{\}} \AgdaSymbol{→} \AgdaSymbol{(}\AgdaBound{a} \AgdaDatatype{≅} \AgdaBound{b}\AgdaSymbol{)} \<[33]%
\>[33]\<%
\\
\>[0]\AgdaIndent{9}{}\<[9]%
\>[9]\AgdaSymbol{→} \AgdaSymbol{(}\AgdaBound{a} \AgdaFunction{⟦} \AgdaBound{p} \AgdaFunction{⟫} \AgdaDatatype{≅} \AgdaBound{b} \AgdaFunction{⟦} \AgdaBound{p} \AgdaFunction{⟫}\AgdaSymbol{)}\<%
\\
\>\AgdaFunction{cohOp-eq} \AgdaSymbol{\{}\AgdaBound{Γ}\AgdaSymbol{\}} \AgdaSymbol{\{}\AgdaSymbol{.}\AgdaBound{B}\AgdaSymbol{\}} \AgdaSymbol{\{}\AgdaBound{B}\AgdaSymbol{\}} \AgdaSymbol{\{}\AgdaBound{a}\AgdaSymbol{\}} \AgdaSymbol{\{}\AgdaBound{b}\AgdaSymbol{\}} \AgdaSymbol{\{}\AgdaInductiveConstructor{refl}\AgdaSymbol{\}} \AgdaBound{r} \AgdaSymbol{=} \AgdaBound{r}\<%
\\
%
\\
\>\AgdaFunction{cohOp-hom} \AgdaSymbol{:} \AgdaSymbol{\{}\AgdaBound{Γ} \AgdaSymbol{:} \AgdaDatatype{Con}\AgdaSymbol{\}\{}\AgdaBound{A} \AgdaBound{B} \AgdaSymbol{:} \AgdaDatatype{Ty} \AgdaBound{Γ}\AgdaSymbol{\}\{}\AgdaBound{a} \AgdaBound{b} \AgdaSymbol{:} \AgdaDatatype{Tm} \AgdaBound{B}\AgdaSymbol{\}(}\AgdaBound{p} \AgdaSymbol{:} \AgdaBound{A} \AgdaDatatype{≡} \AgdaBound{B}\AgdaSymbol{)} \AgdaSymbol{→}\<%
\\
\>[9]\AgdaIndent{12}{}\<[12]%
\>[12]\AgdaSymbol{(}\AgdaBound{a} \AgdaFunction{⟦} \AgdaBound{p} \AgdaFunction{⟫} \AgdaInductiveConstructor{=h} \AgdaBound{b} \AgdaFunction{⟦} \AgdaBound{p} \AgdaFunction{⟫}\AgdaSymbol{)} \AgdaDatatype{≡} \AgdaSymbol{(}\AgdaBound{a} \AgdaInductiveConstructor{=h} \AgdaBound{b}\AgdaSymbol{)}\<%
\\
\>\AgdaFunction{cohOp-hom} \AgdaInductiveConstructor{refl} \AgdaSymbol{=} \AgdaInductiveConstructor{refl}\<%
\\
%
\\
\>\AgdaFunction{cong≅} \AgdaSymbol{:} \AgdaSymbol{\{}\AgdaBound{Γ} \AgdaBound{Δ} \AgdaSymbol{:} \AgdaDatatype{Con}\AgdaSymbol{\}\{}\AgdaBound{A} \AgdaBound{B} \AgdaSymbol{:} \AgdaDatatype{Ty} \AgdaBound{Γ}\AgdaSymbol{\}\{}\AgdaBound{a} \AgdaSymbol{:} \AgdaDatatype{Tm} \AgdaBound{A}\AgdaSymbol{\}\{}\AgdaBound{b} \AgdaSymbol{:} \AgdaDatatype{Tm} \AgdaBound{B}\AgdaSymbol{\}}\<%
\\
\>[-5]\AgdaIndent{8}{}\<[8]%
\>[8]\AgdaSymbol{\{}\AgdaBound{D} \AgdaSymbol{:} \AgdaDatatype{Ty} \AgdaBound{Γ} \AgdaSymbol{→} \AgdaDatatype{Ty} \AgdaBound{Δ}\AgdaSymbol{\}} \AgdaSymbol{→} \AgdaSymbol{(}\AgdaBound{f} \AgdaSymbol{:} \AgdaSymbol{\{}\AgdaBound{C} \AgdaSymbol{:} \AgdaDatatype{Ty} \AgdaBound{Γ}\AgdaSymbol{\}} \AgdaSymbol{→} \AgdaDatatype{Tm} \AgdaBound{C} \AgdaSymbol{→} \AgdaDatatype{Tm} \AgdaSymbol{(}\AgdaBound{D} \AgdaBound{C}\AgdaSymbol{))→} \<[64]%
\>[64]\<%
\\
\>[0]\AgdaIndent{8}{}\<[8]%
\>[8]\AgdaBound{a} \AgdaDatatype{≅} \AgdaBound{b} \<[15]%
\>[15]\AgdaSymbol{→} \AgdaBound{f} \AgdaBound{a} \AgdaDatatype{≅} \AgdaBound{f} \AgdaBound{b}\<%
\\
\>\AgdaFunction{cong≅} \AgdaBound{f} \AgdaSymbol{(}\AgdaInductiveConstructor{refl} \AgdaSymbol{\_)} \AgdaSymbol{=} \AgdaInductiveConstructor{refl} \AgdaSymbol{\_}\<%
\\
\>\<\end{code}

\textbf{Substitutions}

\begin{code}\>\<%
\\
\>\AgdaFunction{\_[\_]T} \<[8]%
\>[8]\AgdaSymbol{:} \AgdaSymbol{∀\{}\AgdaBound{Γ} \AgdaBound{Δ}\AgdaSymbol{\}} \AgdaSymbol{→} \AgdaDatatype{Ty} \AgdaBound{Δ} \AgdaSymbol{→} \AgdaBound{Γ} \AgdaDatatype{⇒} \AgdaBound{Δ} \AgdaSymbol{→} \AgdaDatatype{Ty} \AgdaBound{Γ}\<%
\\
\>\AgdaFunction{\_[\_]V} \<[8]%
\>[8]\AgdaSymbol{:} \AgdaSymbol{∀\{}\AgdaBound{Γ} \AgdaBound{Δ} \AgdaBound{A}\AgdaSymbol{\}} \AgdaSymbol{→} \AgdaDatatype{Var} \AgdaBound{A} \AgdaSymbol{→} \AgdaSymbol{(}\AgdaBound{δ} \AgdaSymbol{:} \AgdaBound{Γ} \AgdaDatatype{⇒} \AgdaBound{Δ}\AgdaSymbol{)} \AgdaSymbol{→} \AgdaDatatype{Tm} \AgdaSymbol{(}\AgdaBound{A} \AgdaFunction{[} \AgdaBound{δ} \AgdaFunction{]T}\AgdaSymbol{)}\<%
\\
\>\AgdaFunction{\_[\_]tm} \<[8]%
\>[8]\AgdaSymbol{:} \AgdaSymbol{∀\{}\AgdaBound{Γ} \AgdaBound{Δ} \AgdaBound{A}\AgdaSymbol{\}} \AgdaSymbol{→} \AgdaDatatype{Tm} \AgdaBound{A} \AgdaSymbol{→} \AgdaSymbol{(}\AgdaBound{δ} \AgdaSymbol{:} \AgdaBound{Γ} \AgdaDatatype{⇒} \AgdaBound{Δ}\AgdaSymbol{)} \AgdaSymbol{→} \AgdaDatatype{Tm} \AgdaSymbol{(}\AgdaBound{A} \AgdaFunction{[} \AgdaBound{δ} \AgdaFunction{]T}\AgdaSymbol{)} \<[59]%
\>[59]\<%
\\
\>\AgdaFunction{\_⊚\_} \AgdaSymbol{:} \AgdaSymbol{∀\{}\AgdaBound{Γ} \AgdaBound{Δ} \AgdaBound{Θ}\AgdaSymbol{\}} \AgdaSymbol{→} \AgdaBound{Δ} \AgdaDatatype{⇒} \AgdaBound{Θ} \AgdaSymbol{→} \AgdaSymbol{(}\AgdaBound{δ} \AgdaSymbol{:} \AgdaBound{Γ} \AgdaDatatype{⇒} \AgdaBound{Δ}\AgdaSymbol{)} \AgdaSymbol{→} \AgdaBound{Γ} \AgdaDatatype{⇒} \AgdaBound{Θ} \<[47]%
\>[47]\<%
\\
\>\<\end{code}


\textbf{Contexts morphisms}

\begin{code}\>\<%
\\
\>\AgdaKeyword{data} \AgdaDatatype{\_⇒\_} \AgdaKeyword{where}\<%
\\
\>[0]\AgdaIndent{2}{}\<[2]%
\>[2]\AgdaInductiveConstructor{•} \<[7]%
\>[7]\AgdaSymbol{:} \AgdaSymbol{∀\{}\AgdaBound{Γ}\AgdaSymbol{\}} \AgdaSymbol{→} \AgdaBound{Γ} \AgdaDatatype{⇒} \AgdaInductiveConstructor{ε}\<%
\\
\>[0]\AgdaIndent{2}{}\<[2]%
\>[2]\AgdaInductiveConstructor{\_,\_} \<[7]%
\>[7]\AgdaSymbol{:} \AgdaSymbol{∀\{}\AgdaBound{Γ} \AgdaBound{Δ}\AgdaSymbol{\}(}\AgdaBound{δ} \AgdaSymbol{:} \AgdaBound{Γ} \AgdaDatatype{⇒} \AgdaBound{Δ}\AgdaSymbol{)\{}\AgdaBound{A} \AgdaSymbol{:} \AgdaDatatype{Ty} \AgdaBound{Δ}\AgdaSymbol{\}(}\AgdaBound{a} \AgdaSymbol{:} \AgdaDatatype{Tm} \AgdaSymbol{(}\AgdaBound{A} \AgdaFunction{[} \AgdaBound{δ} \AgdaFunction{]T}\AgdaSymbol{))}\<%
\\
\>[2]\AgdaIndent{7}{}\<[7]%
\>[7]\AgdaSymbol{→} \AgdaBound{Γ} \AgdaDatatype{⇒} \AgdaSymbol{(}\AgdaBound{Δ} \AgdaInductiveConstructor{,} \AgdaBound{A}\AgdaSymbol{)}\<%
\\
\>\<\end{code}

\textbf{Weakening}

\begin{code}\>\>[2]\AgdaIndent{3}{}\<[3]%
\>[3]\<%
\\
\>\AgdaFunction{\_+T\_} \<[7]%
\>[7]\AgdaSymbol{:} \AgdaSymbol{∀\{}\AgdaBound{Γ}\AgdaSymbol{\}(}\AgdaBound{A} \AgdaSymbol{:} \AgdaDatatype{Ty} \AgdaBound{Γ}\AgdaSymbol{)(}\AgdaBound{B} \AgdaSymbol{:} \AgdaDatatype{Ty} \AgdaBound{Γ}\AgdaSymbol{)} \AgdaSymbol{→} \AgdaDatatype{Ty} \AgdaSymbol{(}\AgdaBound{Γ} \AgdaInductiveConstructor{,} \AgdaBound{B}\AgdaSymbol{)}\<%
\\
\>\AgdaFunction{\_+tm\_} \<[7]%
\>[7]\AgdaSymbol{:} \AgdaSymbol{∀\{}\AgdaBound{Γ} \AgdaBound{A}\AgdaSymbol{\}(}\AgdaBound{a} \AgdaSymbol{:} \AgdaDatatype{Tm} \AgdaBound{A}\AgdaSymbol{)(}\AgdaBound{B} \AgdaSymbol{:} \AgdaDatatype{Ty} \AgdaBound{Γ}\AgdaSymbol{)} \AgdaSymbol{→} \AgdaDatatype{Tm} \AgdaSymbol{(}\AgdaBound{A} \AgdaFunction{+T} \AgdaBound{B}\AgdaSymbol{)} \<[52]%
\>[52]\<%
\\
\>\AgdaFunction{\_+S\_} \<[7]%
\>[7]\AgdaSymbol{:} \AgdaSymbol{∀\{}\AgdaBound{Γ} \AgdaBound{Δ}\AgdaSymbol{\}(}\AgdaBound{δ} \AgdaSymbol{:} \AgdaBound{Γ} \AgdaDatatype{⇒} \AgdaBound{Δ}\AgdaSymbol{)(}\AgdaBound{B} \AgdaSymbol{:} \AgdaDatatype{Ty} \AgdaBound{Γ}\AgdaSymbol{)} \AgdaSymbol{→} \AgdaSymbol{(}\AgdaBound{Γ} \AgdaInductiveConstructor{,} \AgdaBound{B}\AgdaSymbol{)} \AgdaDatatype{⇒} \AgdaBound{Δ} \<[53]%
\>[53]\<%
\\
\>\<\end{code}

\begin{code}\>\<%
\\
%
\\
\>\AgdaInductiveConstructor{*} \<[9]%
\>[9]\AgdaFunction{+T} \AgdaBound{B} \AgdaSymbol{=} \AgdaInductiveConstructor{*}\<%
\\
\>\AgdaSymbol{(}\AgdaBound{a} \AgdaInductiveConstructor{=h} \AgdaBound{b}\AgdaSymbol{)} \AgdaFunction{+T} \AgdaBound{B} \AgdaSymbol{=} \AgdaBound{a} \AgdaFunction{+tm} \AgdaBound{B} \AgdaInductiveConstructor{=h} \AgdaBound{b} \AgdaFunction{+tm} \AgdaBound{B}\<%
\\
%
\\
%
\\
\>\AgdaInductiveConstructor{*} \<[9]%
\>[9]\AgdaFunction{[} \AgdaBound{δ} \AgdaFunction{]T} \AgdaSymbol{=} \AgdaInductiveConstructor{*} \<[20]%
\>[20]\<%
\\
\>\AgdaSymbol{(}\AgdaBound{a} \AgdaInductiveConstructor{=h} \AgdaBound{b}\AgdaSymbol{)} \AgdaFunction{[} \AgdaBound{δ} \AgdaFunction{]T} \AgdaSymbol{=} \AgdaBound{a} \AgdaFunction{[} \AgdaBound{δ} \AgdaFunction{]tm} \AgdaInductiveConstructor{=h} \AgdaBound{b} \AgdaFunction{[} \AgdaBound{δ} \AgdaFunction{]tm}\<%
\\
%
\\
\>\<\end{code}

\textbf{Variables and terms}

\begin{code}\>\<%
\\
\>\AgdaKeyword{data} \AgdaDatatype{Var} \AgdaKeyword{where}\<%
\\
\>[0]\AgdaIndent{2}{}\<[2]%
\>[2]\AgdaInductiveConstructor{v0} \AgdaSymbol{:} \AgdaSymbol{∀\{}\AgdaBound{Γ}\AgdaSymbol{\}\{}\AgdaBound{A} \AgdaSymbol{:} \AgdaDatatype{Ty} \AgdaBound{Γ}\AgdaSymbol{\}} \<[35]%
\>[35]\AgdaSymbol{→} \AgdaDatatype{Var} \AgdaSymbol{(}\AgdaBound{A} \AgdaFunction{+T} \AgdaBound{A}\AgdaSymbol{)}\<%
\\
\>[0]\AgdaIndent{2}{}\<[2]%
\>[2]\AgdaInductiveConstructor{vS} \AgdaSymbol{:} \AgdaSymbol{∀\{}\AgdaBound{Γ}\AgdaSymbol{\}\{}\AgdaBound{A} \AgdaBound{B} \AgdaSymbol{:} \AgdaDatatype{Ty} \AgdaBound{Γ}\AgdaSymbol{\}(}\AgdaBound{x} \AgdaSymbol{:} \AgdaDatatype{Var} \AgdaBound{A}\AgdaSymbol{)} \AgdaSymbol{→} \AgdaDatatype{Var} \AgdaSymbol{(}\AgdaBound{A} \AgdaFunction{+T} \AgdaBound{B}\AgdaSymbol{)}\<%
\\
%
\\
\>\AgdaKeyword{data} \AgdaDatatype{Tm} \AgdaKeyword{where}\<%
\\
\>[0]\AgdaIndent{2}{}\<[2]%
\>[2]\AgdaInductiveConstructor{var} \<[7]%
\>[7]\AgdaSymbol{:} \AgdaSymbol{∀\{}\AgdaBound{Γ}\AgdaSymbol{\}\{}\AgdaBound{A} \AgdaSymbol{:} \AgdaDatatype{Ty} \AgdaBound{Γ}\AgdaSymbol{\}} \AgdaSymbol{→} \AgdaDatatype{Var} \AgdaBound{A} \AgdaSymbol{→} \AgdaDatatype{Tm} \AgdaBound{A}\<%
\\
\>[0]\AgdaIndent{2}{}\<[2]%
\>[2]\AgdaInductiveConstructor{coh} \<[7]%
\>[7]\AgdaSymbol{:} \AgdaSymbol{∀\{}\AgdaBound{Γ} \AgdaBound{Δ}\AgdaSymbol{\}} \AgdaSymbol{→} \AgdaDatatype{isContr} \AgdaBound{Δ} \AgdaSymbol{→} \AgdaSymbol{(}\AgdaBound{δ} \AgdaSymbol{:} \AgdaBound{Γ} \AgdaDatatype{⇒} \AgdaBound{Δ}\AgdaSymbol{)} \<[42]%
\>[42]\<%
\\
\>[2]\AgdaIndent{7}{}\<[7]%
\>[7]\AgdaSymbol{→} \AgdaSymbol{(}\AgdaBound{A} \AgdaSymbol{:} \AgdaDatatype{Ty} \AgdaBound{Δ}\AgdaSymbol{)} \AgdaSymbol{→} \AgdaDatatype{Tm} \AgdaSymbol{(}\AgdaBound{A} \AgdaFunction{[} \AgdaBound{δ} \AgdaFunction{]T}\AgdaSymbol{)}\<%
\\
%
\\
\>\AgdaFunction{cohOpV} \AgdaSymbol{:} \AgdaSymbol{\{}\AgdaBound{Γ} \AgdaSymbol{:} \AgdaDatatype{Con}\AgdaSymbol{\}\{}\AgdaBound{A} \AgdaBound{B} \AgdaSymbol{:} \AgdaDatatype{Ty} \AgdaBound{Γ}\AgdaSymbol{\}\{}\AgdaBound{x} \AgdaSymbol{:} \AgdaDatatype{Var} \AgdaBound{A}\AgdaSymbol{\}(}\AgdaBound{p} \AgdaSymbol{:} \AgdaBound{A} \AgdaDatatype{≡} \AgdaBound{B}\AgdaSymbol{)} \AgdaSymbol{→}\<%
\\
\>[7]\AgdaIndent{9}{}\<[9]%
\>[9]\AgdaInductiveConstructor{var} \AgdaSymbol{(}\AgdaFunction{subst} \AgdaDatatype{Var} \AgdaBound{p} \AgdaBound{x}\AgdaSymbol{)} \AgdaDatatype{≅} \AgdaInductiveConstructor{var} \AgdaBound{x}\<%
\\
\>\AgdaFunction{cohOpV} \AgdaSymbol{\{}x \AgdaSymbol{=} \AgdaBound{x}\AgdaSymbol{\}} \AgdaInductiveConstructor{refl} \AgdaSymbol{=} \AgdaInductiveConstructor{refl} \AgdaSymbol{(}\AgdaInductiveConstructor{var} \AgdaBound{x}\AgdaSymbol{)}\<%
\\
%
\\
\>\AgdaFunction{cohOpVs} \AgdaSymbol{:} \AgdaSymbol{\{}\AgdaBound{Γ} \AgdaSymbol{:} \AgdaDatatype{Con}\AgdaSymbol{\}\{}\AgdaBound{A} \AgdaBound{B} \AgdaBound{C} \AgdaSymbol{:} \AgdaDatatype{Ty} \AgdaBound{Γ}\AgdaSymbol{\}\{}\AgdaBound{x} \AgdaSymbol{:} \AgdaDatatype{Var} \AgdaBound{A}\AgdaSymbol{\}(}\AgdaBound{p} \AgdaSymbol{:} \AgdaBound{A} \AgdaDatatype{≡} \AgdaBound{B}\AgdaSymbol{)} \AgdaSymbol{→} \<[58]%
\>[58]\<%
\\
\>[9]\AgdaIndent{10}{}\<[10]%
\>[10]\AgdaInductiveConstructor{var} \AgdaSymbol{(}\AgdaInductiveConstructor{vS} \AgdaSymbol{\{}B \AgdaSymbol{=} \AgdaBound{C}\AgdaSymbol{\}} \AgdaSymbol{(}\AgdaFunction{subst} \AgdaDatatype{Var} \AgdaBound{p} \AgdaBound{x}\AgdaSymbol{))} \AgdaDatatype{≅} \AgdaInductiveConstructor{var} \AgdaSymbol{(}\AgdaInductiveConstructor{vS} \AgdaBound{x}\AgdaSymbol{)}\<%
\\
\>\AgdaFunction{cohOpVs} \AgdaSymbol{\{}x \AgdaSymbol{=} \AgdaBound{x}\AgdaSymbol{\}} \AgdaInductiveConstructor{refl} \AgdaSymbol{=} \AgdaInductiveConstructor{refl} \AgdaSymbol{(}\AgdaInductiveConstructor{var} \AgdaSymbol{(}\AgdaInductiveConstructor{vS} \AgdaBound{x}\AgdaSymbol{))}\<%
\\
%
\\
\>\AgdaFunction{coh-eq} \AgdaSymbol{:} \AgdaSymbol{\{}\AgdaBound{Γ} \AgdaBound{Δ} \AgdaSymbol{:} \AgdaDatatype{Con}\AgdaSymbol{\}\{}\AgdaBound{isc} \AgdaSymbol{:} \AgdaDatatype{isContr} \AgdaBound{Δ}\AgdaSymbol{\}\{}\AgdaBound{γ} \AgdaBound{δ} \AgdaSymbol{:} \AgdaBound{Γ} \AgdaDatatype{⇒} \AgdaBound{Δ}\AgdaSymbol{\}}\<%
\\
\>[0]\AgdaIndent{9}{}\<[9]%
\>[9]\AgdaSymbol{\{}\AgdaBound{A} \AgdaSymbol{:} \AgdaDatatype{Ty} \AgdaBound{Δ}\AgdaSymbol{\}} \AgdaSymbol{→} \AgdaBound{γ} \AgdaDatatype{≡} \AgdaBound{δ} \AgdaSymbol{→} \AgdaInductiveConstructor{coh} \AgdaBound{isc} \AgdaBound{γ} \AgdaBound{A} \AgdaDatatype{≅} \AgdaInductiveConstructor{coh} \AgdaBound{isc} \AgdaBound{δ} \AgdaBound{A} \<[56]%
\>[56]\<%
\\
\>\AgdaFunction{coh-eq} \AgdaInductiveConstructor{refl} \AgdaSymbol{=} \AgdaInductiveConstructor{refl} \AgdaSymbol{\_}\<%
\\
\>\<\end{code}

\textbf{Contractible contexts}

\begin{code}\>\<%
\\
\>\AgdaKeyword{data} \AgdaDatatype{isContr} \AgdaKeyword{where}\<%
\\
\>[0]\AgdaIndent{2}{}\<[2]%
\>[2]\AgdaInductiveConstructor{c*} \<[7]%
\>[7]\AgdaSymbol{:} \AgdaDatatype{isContr} \AgdaSymbol{(}\AgdaInductiveConstructor{ε} \AgdaInductiveConstructor{,} \AgdaInductiveConstructor{*}\AgdaSymbol{)}\<%
\\
\>[0]\AgdaIndent{2}{}\<[2]%
\>[2]\AgdaInductiveConstructor{ext} \<[7]%
\>[7]\AgdaSymbol{:} \AgdaSymbol{∀\{}\AgdaBound{Γ}\AgdaSymbol{\}} \AgdaSymbol{→} \AgdaDatatype{isContr} \AgdaBound{Γ} \AgdaSymbol{→} \AgdaSymbol{\{}\AgdaBound{A} \AgdaSymbol{:} \AgdaDatatype{Ty} \AgdaBound{Γ}\AgdaSymbol{\}(}\AgdaBound{x} \AgdaSymbol{:} \AgdaDatatype{Var} \AgdaBound{A}\AgdaSymbol{)} \<[50]%
\>[50]\<%
\\
\>[2]\AgdaIndent{7}{}\<[7]%
\>[7]\AgdaSymbol{→} \AgdaDatatype{isContr} \AgdaSymbol{(}\AgdaBound{Γ} \AgdaInductiveConstructor{,} \AgdaBound{A} \AgdaInductiveConstructor{,} \AgdaSymbol{(}\AgdaInductiveConstructor{var} \AgdaSymbol{(}\AgdaInductiveConstructor{vS} \AgdaBound{x}\AgdaSymbol{)} \AgdaInductiveConstructor{=h} \AgdaInductiveConstructor{var} \AgdaInductiveConstructor{v0}\AgdaSymbol{))} \<[54]%
\>[54]\<%
\\
\>\<\end{code}

\begin{code}\>\<%
\\
\>\AgdaFunction{hom≡} \AgdaSymbol{:} \AgdaSymbol{\{}\AgdaBound{Γ} \AgdaSymbol{:} \AgdaDatatype{Con}\AgdaSymbol{\}\{}\AgdaBound{A} \AgdaBound{A'} \AgdaSymbol{:} \AgdaDatatype{Ty} \AgdaBound{Γ}\AgdaSymbol{\}}\<%
\\
\>[7]\AgdaIndent{16}{}\<[16]%
\>[16]\AgdaSymbol{\{}\AgdaBound{a} \AgdaSymbol{:} \AgdaDatatype{Tm} \AgdaBound{A}\AgdaSymbol{\}\{}\AgdaBound{a'} \AgdaSymbol{:} \AgdaDatatype{Tm} \AgdaBound{A'}\AgdaSymbol{\}(}\AgdaBound{q} \AgdaSymbol{:} \AgdaBound{a} \AgdaDatatype{≅} \AgdaBound{a'}\AgdaSymbol{)}\<%
\\
\>[7]\AgdaIndent{16}{}\<[16]%
\>[16]\AgdaSymbol{\{}\AgdaBound{b} \AgdaSymbol{:} \AgdaDatatype{Tm} \AgdaBound{A}\AgdaSymbol{\}\{}\AgdaBound{b'} \AgdaSymbol{:} \AgdaDatatype{Tm} \AgdaBound{A'}\AgdaSymbol{\}(}\AgdaBound{r} \AgdaSymbol{:} \AgdaBound{b} \AgdaDatatype{≅} \AgdaBound{b'}\AgdaSymbol{)}\<%
\\
\>[7]\AgdaIndent{16}{}\<[16]%
\>[16]\AgdaSymbol{→} \AgdaSymbol{(}\AgdaBound{a} \AgdaInductiveConstructor{=h} \AgdaBound{b}\AgdaSymbol{)} \AgdaDatatype{≡} \AgdaSymbol{(}\AgdaBound{a'} \AgdaInductiveConstructor{=h} \AgdaBound{b'}\AgdaSymbol{)}\<%
\\
\>\AgdaFunction{hom≡} \AgdaSymbol{\{}\AgdaBound{Γ}\AgdaSymbol{\}} \AgdaSymbol{\{}\AgdaSymbol{.}\AgdaBound{A'}\AgdaSymbol{\}} \AgdaSymbol{\{}\AgdaBound{A'}\AgdaSymbol{\}} \AgdaSymbol{\{}\AgdaSymbol{.}\AgdaBound{a'}\AgdaSymbol{\}} \AgdaSymbol{\{}\AgdaBound{a'}\AgdaSymbol{\}} \AgdaSymbol{(}\AgdaInductiveConstructor{refl} \AgdaSymbol{.}\AgdaBound{a'}\AgdaSymbol{)} \AgdaSymbol{\{}\AgdaSymbol{.}\AgdaBound{b'}\AgdaSymbol{\}} \AgdaSymbol{\{}\AgdaBound{b'}\AgdaSymbol{\}} \AgdaSymbol{(}\AgdaInductiveConstructor{refl} \AgdaSymbol{.}\AgdaBound{b'}\AgdaSymbol{)} \AgdaSymbol{=} \AgdaInductiveConstructor{refl}\<%
\\
%
\\
%
\\
\>\AgdaFunction{S-eq} \AgdaSymbol{:} \AgdaSymbol{\{}\AgdaBound{Γ} \AgdaBound{Δ} \AgdaSymbol{:} \AgdaDatatype{Con}\AgdaSymbol{\}\{}\AgdaBound{γ} \AgdaBound{δ} \AgdaSymbol{:} \AgdaBound{Γ} \AgdaDatatype{⇒} \AgdaBound{Δ}\AgdaSymbol{\}\{}\AgdaBound{A} \AgdaSymbol{:} \AgdaDatatype{Ty} \AgdaBound{Δ}\AgdaSymbol{\}}\<%
\\
\>[1]\AgdaIndent{8}{}\<[8]%
\>[8]\AgdaSymbol{\{}\AgdaBound{a} \AgdaSymbol{:} \AgdaDatatype{Tm} \AgdaSymbol{(}\AgdaBound{A} \AgdaFunction{[} \AgdaBound{γ} \AgdaFunction{]T}\AgdaSymbol{)\}\{}\AgdaBound{a'} \AgdaSymbol{:} \AgdaDatatype{Tm} \AgdaSymbol{(}\AgdaBound{A} \AgdaFunction{[} \AgdaBound{δ} \AgdaFunction{]T}\AgdaSymbol{)\}} \<[48]%
\>[48]\<%
\\
\>[0]\AgdaIndent{8}{}\<[8]%
\>[8]\AgdaSymbol{→} \AgdaBound{γ} \AgdaDatatype{≡} \AgdaBound{δ} \AgdaSymbol{→} \AgdaBound{a} \AgdaDatatype{≅} \AgdaBound{a'} \<[25]%
\>[25]\<%
\\
\>[0]\AgdaIndent{8}{}\<[8]%
\>[8]\AgdaSymbol{→} \AgdaDatatype{\_≡\_} \AgdaSymbol{\{\_\}} \AgdaSymbol{\{}\AgdaBound{Γ} \AgdaDatatype{⇒} \AgdaSymbol{(}\AgdaBound{Δ} \AgdaInductiveConstructor{,} \AgdaBound{A}\AgdaSymbol{)\}} \AgdaSymbol{(}\AgdaBound{γ} \AgdaInductiveConstructor{,} \AgdaBound{a}\AgdaSymbol{)} \AgdaSymbol{(}\AgdaBound{δ} \AgdaInductiveConstructor{,} \AgdaBound{a'}\AgdaSymbol{)}\<%
\\
\>\AgdaFunction{S-eq} \AgdaInductiveConstructor{refl} \AgdaSymbol{(}\AgdaInductiveConstructor{refl} \AgdaSymbol{\_)} \AgdaSymbol{=} \AgdaInductiveConstructor{refl}\<%
\\
\>\<\end{code}

\textbf{Some lemmas}

\begin{code}\>\<%
\\
\>\AgdaFunction{[⊚]T} \<[8]%
\>[8]\AgdaSymbol{:} \AgdaSymbol{∀\{}\AgdaBound{Γ} \AgdaBound{Δ} \AgdaBound{Θ} \AgdaBound{A}\AgdaSymbol{\}\{}\AgdaBound{θ} \AgdaSymbol{:} \AgdaBound{Δ} \AgdaDatatype{⇒} \AgdaBound{Θ}\AgdaSymbol{\}\{}\AgdaBound{δ} \AgdaSymbol{:} \AgdaBound{Γ} \AgdaDatatype{⇒} \AgdaBound{Δ}\AgdaSymbol{\}} \<[43]%
\>[43]\<%
\\
\>[0]\AgdaIndent{8}{}\<[8]%
\>[8]\AgdaSymbol{→} \AgdaBound{A} \AgdaFunction{[} \AgdaBound{θ} \AgdaFunction{⊚} \AgdaBound{δ} \AgdaFunction{]T} \AgdaDatatype{≡} \AgdaSymbol{(}\AgdaBound{A} \AgdaFunction{[} \AgdaBound{θ} \AgdaFunction{]T}\AgdaSymbol{)}\AgdaFunction{[} \AgdaBound{δ} \AgdaFunction{]T} \<[43]%
\>[43]\<%
\\
%
\\
\>\AgdaFunction{[⊚]v} \<[8]%
\>[8]\AgdaSymbol{:} \AgdaSymbol{∀\{}\AgdaBound{Γ} \AgdaBound{Δ} \AgdaBound{Θ} \AgdaBound{A}\AgdaSymbol{\}(}\AgdaBound{x} \AgdaSymbol{:} \AgdaDatatype{Var} \AgdaBound{A}\AgdaSymbol{)\{}\AgdaBound{θ} \AgdaSymbol{:} \AgdaBound{Δ} \AgdaDatatype{⇒} \AgdaBound{Θ}\AgdaSymbol{\}\{}\AgdaBound{δ} \AgdaSymbol{:} \AgdaBound{Γ} \AgdaDatatype{⇒} \AgdaBound{Δ}\AgdaSymbol{\}}\<%
\\
\>[0]\AgdaIndent{8}{}\<[8]%
\>[8]\AgdaSymbol{→} \AgdaBound{x} \AgdaFunction{[} \AgdaBound{θ} \AgdaFunction{⊚} \AgdaBound{δ} \AgdaFunction{]V} \AgdaDatatype{≅} \AgdaSymbol{(}\AgdaBound{x} \AgdaFunction{[} \AgdaBound{θ} \AgdaFunction{]V}\AgdaSymbol{)} \AgdaFunction{[} \AgdaBound{δ} \AgdaFunction{]tm}\<%
\\
%
\\
\>\AgdaFunction{[⊚]tm} \<[8]%
\>[8]\AgdaSymbol{:} \AgdaSymbol{∀\{}\AgdaBound{Γ} \AgdaBound{Δ} \AgdaBound{Θ} \AgdaBound{A}\AgdaSymbol{\}(}\AgdaBound{a} \AgdaSymbol{:} \AgdaDatatype{Tm} \AgdaBound{A}\AgdaSymbol{)\{}\AgdaBound{θ} \AgdaSymbol{:} \AgdaBound{Δ} \AgdaDatatype{⇒} \AgdaBound{Θ}\AgdaSymbol{\}\{}\AgdaBound{δ} \AgdaSymbol{:} \AgdaBound{Γ} \AgdaDatatype{⇒} \AgdaBound{Δ}\AgdaSymbol{\}}\<%
\\
\>[0]\AgdaIndent{8}{}\<[8]%
\>[8]\AgdaSymbol{→} \AgdaBound{a} \AgdaFunction{[} \AgdaBound{θ} \AgdaFunction{⊚} \AgdaBound{δ} \AgdaFunction{]tm} \AgdaDatatype{≅} \AgdaSymbol{(}\AgdaBound{a} \AgdaFunction{[} \AgdaBound{θ} \AgdaFunction{]tm}\AgdaSymbol{)} \AgdaFunction{[} \AgdaBound{δ} \AgdaFunction{]tm}\<%
\\
%
\\
\>\AgdaFunction{⊚assoc} \<[8]%
\>[8]\AgdaSymbol{:} \AgdaSymbol{∀\{}\AgdaBound{Γ} \AgdaBound{Δ} \AgdaBound{Θ} \AgdaBound{Ω}\AgdaSymbol{\}(}\AgdaBound{γ} \AgdaSymbol{:} \AgdaBound{Θ} \AgdaDatatype{⇒} \AgdaBound{Ω}\AgdaSymbol{)\{}\AgdaBound{θ} \AgdaSymbol{:} \AgdaBound{Δ} \AgdaDatatype{⇒} \AgdaBound{Θ}\AgdaSymbol{\}\{}\AgdaBound{δ} \AgdaSymbol{:} \AgdaBound{Γ} \AgdaDatatype{⇒} \AgdaBound{Δ}\AgdaSymbol{\}} \<[55]%
\>[55]\<%
\\
\>[0]\AgdaIndent{8}{}\<[8]%
\>[8]\AgdaSymbol{→} \AgdaSymbol{(}\AgdaBound{γ} \AgdaFunction{⊚} \AgdaBound{θ}\AgdaSymbol{)} \AgdaFunction{⊚} \AgdaBound{δ} \AgdaDatatype{≡} \AgdaBound{γ} \AgdaFunction{⊚} \AgdaSymbol{(}\AgdaBound{θ} \AgdaFunction{⊚} \AgdaBound{δ}\AgdaSymbol{)} \<[37]%
\>[37]\<%
\\
\>\<\end{code}

\begin{code}\>\<%
\\
\>\AgdaInductiveConstructor{•} \<[8]%
\>[8]\AgdaFunction{⊚} \AgdaBound{δ} \AgdaSymbol{=} \AgdaInductiveConstructor{•}\<%
\\
\>\AgdaSymbol{(}\AgdaBound{δ} \AgdaInductiveConstructor{,} \AgdaBound{a}\AgdaSymbol{)} \AgdaFunction{⊚} \AgdaBound{δ'} \AgdaSymbol{=} \AgdaSymbol{(}\AgdaBound{δ} \AgdaFunction{⊚} \AgdaBound{δ'}\AgdaSymbol{)} \AgdaInductiveConstructor{,} \AgdaBound{a} \AgdaFunction{[} \AgdaBound{δ'} \AgdaFunction{]tm} \AgdaFunction{⟦} \AgdaFunction{[⊚]T} \AgdaFunction{⟫}\<%
\\
%
\\
\>\AgdaFunction{[+S]T} \<[8]%
\>[8]\AgdaSymbol{:} \AgdaSymbol{∀\{}\AgdaBound{Γ} \AgdaBound{Δ} \AgdaBound{A} \AgdaBound{B}\AgdaSymbol{\}\{}\AgdaBound{δ} \AgdaSymbol{:} \AgdaBound{Γ} \AgdaDatatype{⇒} \AgdaBound{Δ}\AgdaSymbol{\}}\<%
\\
\>[0]\AgdaIndent{8}{}\<[8]%
\>[8]\AgdaSymbol{→} \AgdaBound{A} \AgdaFunction{[} \AgdaBound{δ} \AgdaFunction{+S} \AgdaBound{B} \AgdaFunction{]T} \AgdaDatatype{≡} \AgdaSymbol{(}\AgdaBound{A} \AgdaFunction{[} \AgdaBound{δ} \AgdaFunction{]T}\AgdaSymbol{)} \AgdaFunction{+T} \AgdaBound{B} \<[42]%
\>[42]\<%
\\
%
\\
\>\AgdaFunction{[+S]tm} \<[8]%
\>[8]\AgdaSymbol{:} \AgdaSymbol{∀\{}\AgdaBound{Γ} \AgdaBound{Δ} \AgdaBound{A} \AgdaBound{B}\AgdaSymbol{\}(}\AgdaBound{a} \AgdaSymbol{:} \AgdaDatatype{Tm} \AgdaBound{A}\AgdaSymbol{)\{}\AgdaBound{δ} \AgdaSymbol{:} \AgdaBound{Γ} \AgdaDatatype{⇒} \AgdaBound{Δ}\AgdaSymbol{\}}\<%
\\
\>[0]\AgdaIndent{8}{}\<[8]%
\>[8]\AgdaSymbol{→} \AgdaBound{a} \AgdaFunction{[} \AgdaBound{δ} \AgdaFunction{+S} \AgdaBound{B} \AgdaFunction{]tm} \AgdaDatatype{≅} \AgdaSymbol{(}\AgdaBound{a} \AgdaFunction{[} \AgdaBound{δ} \AgdaFunction{]tm}\AgdaSymbol{)} \AgdaFunction{+tm} \AgdaBound{B}\<%
\\
%
\\
\>\AgdaFunction{[+S]S} \<[8]%
\>[8]\AgdaSymbol{:} \AgdaSymbol{∀\{}\AgdaBound{Γ} \AgdaBound{Δ} \AgdaBound{Θ} \AgdaBound{B}\AgdaSymbol{\}\{}\AgdaBound{δ} \AgdaSymbol{:} \AgdaBound{Δ} \AgdaDatatype{⇒} \AgdaBound{Θ}\AgdaSymbol{\}\{}\AgdaBound{γ} \AgdaSymbol{:} \AgdaBound{Γ} \AgdaDatatype{⇒} \AgdaBound{Δ}\AgdaSymbol{\}}\<%
\\
\>[0]\AgdaIndent{8}{}\<[8]%
\>[8]\AgdaSymbol{→} \AgdaBound{δ} \AgdaFunction{⊚} \AgdaSymbol{(}\AgdaBound{γ} \AgdaFunction{+S} \AgdaBound{B}\AgdaSymbol{)} \AgdaDatatype{≡} \AgdaSymbol{(}\AgdaBound{δ} \AgdaFunction{⊚} \AgdaBound{γ}\AgdaSymbol{)} \AgdaFunction{+S} \AgdaBound{B}\<%
\\
%
\\
\>\AgdaFunction{wk-tm+} \<[12]%
\>[12]\AgdaSymbol{:} \AgdaSymbol{\{}\AgdaBound{Γ} \AgdaBound{Δ} \AgdaSymbol{:} \AgdaDatatype{Con}\AgdaSymbol{\}\{}\AgdaBound{A} \AgdaSymbol{:} \AgdaDatatype{Ty} \AgdaBound{Δ}\AgdaSymbol{\}\{}\AgdaBound{δ} \AgdaSymbol{:} \AgdaBound{Γ} \AgdaDatatype{⇒} \AgdaBound{Δ}\AgdaSymbol{\}(}\AgdaBound{B} \AgdaSymbol{:} \AgdaDatatype{Ty} \AgdaBound{Γ}\AgdaSymbol{)} \<[57]%
\>[57]\<%
\\
\>[8]\AgdaIndent{12}{}\<[12]%
\>[12]\AgdaSymbol{→} \AgdaDatatype{Tm} \AgdaSymbol{(}\AgdaBound{A} \AgdaFunction{[} \AgdaBound{δ} \AgdaFunction{]T} \AgdaFunction{+T} \AgdaBound{B}\AgdaSymbol{)} \AgdaSymbol{→} \AgdaDatatype{Tm} \AgdaSymbol{(}\AgdaBound{A} \AgdaFunction{[} \AgdaBound{δ} \AgdaFunction{+S} \AgdaBound{B} \AgdaFunction{]T}\AgdaSymbol{)}\<%
\\
\>\AgdaFunction{wk-tm+} \AgdaBound{B} \AgdaBound{t} \<[12]%
\>[12]\AgdaSymbol{=} \AgdaBound{t} \AgdaFunction{⟦} \AgdaFunction{[+S]T} \AgdaFunction{⟫}\<%
\\
%
\\
\>\AgdaInductiveConstructor{•} \<[8]%
\>[8]\AgdaFunction{+S} \AgdaBound{B} \AgdaSymbol{=} \AgdaInductiveConstructor{•}\<%
\\
\>\AgdaSymbol{(}\AgdaBound{δ} \AgdaInductiveConstructor{,} \AgdaBound{a}\AgdaSymbol{)} \AgdaFunction{+S} \AgdaBound{B} \AgdaSymbol{=} \AgdaSymbol{(}\AgdaBound{δ} \AgdaFunction{+S} \AgdaBound{B}\AgdaSymbol{)} \AgdaInductiveConstructor{,} \AgdaFunction{wk-tm+} \AgdaBound{B} \AgdaSymbol{(}\AgdaBound{a} \AgdaFunction{+tm} \AgdaBound{B}\AgdaSymbol{)}\<%
\\
%
\\
\>\AgdaFunction{[+S]T} \AgdaSymbol{\{}A \AgdaSymbol{=} \AgdaInductiveConstructor{*}\AgdaSymbol{\}} \<[18]%
\>[18]\AgdaSymbol{=} \AgdaInductiveConstructor{refl}\<%
\\
\>\AgdaFunction{[+S]T} \AgdaSymbol{\{}A \AgdaSymbol{=} \AgdaBound{a} \AgdaInductiveConstructor{=h} \AgdaBound{b}\AgdaSymbol{\}} \AgdaSymbol{=} \AgdaFunction{hom≡} \AgdaSymbol{(}\AgdaFunction{[+S]tm} \AgdaBound{a}\AgdaSymbol{)} \AgdaSymbol{(}\AgdaFunction{[+S]tm} \AgdaBound{b}\AgdaSymbol{)}\<%
\\
%
\\
\>\AgdaFunction{+T[,]T} \<[10]%
\>[10]\AgdaSymbol{:} \AgdaSymbol{∀\{}\AgdaBound{Γ} \AgdaBound{Δ} \AgdaBound{A} \AgdaBound{B}\AgdaSymbol{\}\{}\AgdaBound{δ} \AgdaSymbol{:} \AgdaBound{Γ} \AgdaDatatype{⇒} \AgdaBound{Δ}\AgdaSymbol{\}\{}\AgdaBound{b} \AgdaSymbol{:} \AgdaDatatype{Tm} \AgdaSymbol{(}\AgdaBound{B} \AgdaFunction{[} \AgdaBound{δ} \AgdaFunction{]T}\AgdaSymbol{)\}} \<[53]%
\>[53]\<%
\\
\>[-6]\AgdaIndent{10}{}\<[10]%
\>[10]\AgdaSymbol{→} \AgdaSymbol{(}\AgdaBound{A} \AgdaFunction{+T} \AgdaBound{B}\AgdaSymbol{)} \AgdaFunction{[} \AgdaBound{δ} \AgdaInductiveConstructor{,} \AgdaBound{b} \AgdaFunction{]T} \AgdaDatatype{≡} \AgdaBound{A} \AgdaFunction{[} \AgdaBound{δ} \AgdaFunction{]T}\<%
\\
%
\\
\>\AgdaFunction{+tm[,]tm} \<[10]%
\>[10]\AgdaSymbol{:} \AgdaSymbol{∀\{}\AgdaBound{Γ} \AgdaBound{Δ} \AgdaBound{A} \AgdaBound{B}\AgdaSymbol{\}\{}\AgdaBound{δ} \AgdaSymbol{:} \AgdaBound{Γ} \AgdaDatatype{⇒} \AgdaBound{Δ}\AgdaSymbol{\}\{}\AgdaBound{c} \AgdaSymbol{:} \AgdaDatatype{Tm} \AgdaSymbol{(}\AgdaBound{B} \AgdaFunction{[} \AgdaBound{δ} \AgdaFunction{]T}\AgdaSymbol{)\}}\<%
\\
\>[0]\AgdaIndent{10}{}\<[10]%
\>[10]\AgdaSymbol{→} \AgdaSymbol{(}\AgdaBound{a} \AgdaSymbol{:} \AgdaDatatype{Tm} \AgdaBound{A}\AgdaSymbol{)} \<[23]%
\>[23]\<%
\\
\>[0]\AgdaIndent{10}{}\<[10]%
\>[10]\AgdaSymbol{→} \AgdaSymbol{(}\AgdaBound{a} \AgdaFunction{+tm} \AgdaBound{B}\AgdaSymbol{)} \AgdaFunction{[} \AgdaBound{δ} \AgdaInductiveConstructor{,} \AgdaBound{c} \AgdaFunction{]tm} \AgdaDatatype{≅} \AgdaBound{a} \AgdaFunction{[} \AgdaBound{δ} \AgdaFunction{]tm} \<[46]%
\>[46]\<%
\\
%
\\
\>\AgdaSymbol{(}\AgdaInductiveConstructor{var} \AgdaBound{x}\AgdaSymbol{)} \<[12]%
\>[12]\AgdaFunction{+tm} \AgdaBound{B} \AgdaSymbol{=} \AgdaInductiveConstructor{var} \AgdaSymbol{(}\AgdaInductiveConstructor{vS} \AgdaBound{x}\AgdaSymbol{)}\<%
\\
\>\AgdaSymbol{(}\AgdaInductiveConstructor{coh} \AgdaBound{cΔ} \AgdaBound{δ} \AgdaBound{A}\AgdaSymbol{)} \AgdaFunction{+tm} \AgdaBound{B} \AgdaSymbol{=} \AgdaInductiveConstructor{coh} \AgdaBound{cΔ} \AgdaSymbol{(}\AgdaBound{δ} \AgdaFunction{+S} \AgdaBound{B}\AgdaSymbol{)} \AgdaBound{A} \AgdaFunction{⟦} \AgdaFunction{sym} \AgdaFunction{[+S]T} \AgdaFunction{⟫} \<[53]%
\>[53]\<%
\\
%
\\
\>\AgdaFunction{cong+tm} \AgdaSymbol{:} \AgdaSymbol{\{}\AgdaBound{Γ} \AgdaSymbol{:} \AgdaDatatype{Con}\AgdaSymbol{\}\{}\AgdaBound{A} \AgdaBound{B} \AgdaBound{C} \AgdaSymbol{:} \AgdaDatatype{Ty} \AgdaBound{Γ}\AgdaSymbol{\}\{}\AgdaBound{a} \AgdaSymbol{:} \AgdaDatatype{Tm} \AgdaBound{A}\AgdaSymbol{\}\{}\AgdaBound{b} \AgdaSymbol{:} \AgdaDatatype{Tm} \AgdaBound{B}\AgdaSymbol{\}} \AgdaSymbol{→} \<[56]%
\>[56]\<%
\\
\>[0]\AgdaIndent{10}{}\<[10]%
\>[10]\AgdaBound{a} \AgdaDatatype{≅} \AgdaBound{b}\<%
\\
\>[0]\AgdaIndent{8}{}\<[8]%
\>[8]\AgdaSymbol{→} \AgdaBound{a} \AgdaFunction{+tm} \AgdaBound{C} \AgdaDatatype{≅} \AgdaBound{b} \AgdaFunction{+tm} \AgdaBound{C}\<%
\\
\>\AgdaFunction{cong+tm} \AgdaSymbol{(}\AgdaInductiveConstructor{refl} \AgdaSymbol{\_)} \AgdaSymbol{=} \AgdaInductiveConstructor{refl} \AgdaSymbol{\_}\<%
\\
%
\\
\>\AgdaFunction{cong+tm2} \AgdaSymbol{:} \AgdaSymbol{\{}\AgdaBound{Γ} \AgdaSymbol{:} \AgdaDatatype{Con}\AgdaSymbol{\}\{}\AgdaBound{A} \AgdaBound{B} \AgdaBound{C} \AgdaSymbol{:} \AgdaDatatype{Ty} \AgdaBound{Γ}\AgdaSymbol{\}}\<%
\\
\>[0]\AgdaIndent{11}{}\<[11]%
\>[11]\AgdaSymbol{\{}\AgdaBound{a} \AgdaSymbol{:} \AgdaDatatype{Tm} \AgdaBound{B}\AgdaSymbol{\}(}\AgdaBound{p} \AgdaSymbol{:} \AgdaBound{A} \AgdaDatatype{≡} \AgdaBound{B}\AgdaSymbol{)} \<[33]%
\>[33]\<%
\\
\>[0]\AgdaIndent{9}{}\<[9]%
\>[9]\AgdaSymbol{→} \AgdaBound{a} \AgdaFunction{+tm} \AgdaBound{C} \AgdaDatatype{≅} \AgdaBound{a} \AgdaFunction{⟦} \AgdaBound{p} \AgdaFunction{⟫} \AgdaFunction{+tm} \AgdaBound{C}\<%
\\
\>\AgdaFunction{cong+tm2} \AgdaInductiveConstructor{refl} \AgdaSymbol{=} \AgdaInductiveConstructor{refl} \AgdaSymbol{\_}\<%
\\
%
\\
\>\AgdaFunction{wk-T} \AgdaSymbol{:} \AgdaSymbol{\{}\AgdaBound{Δ} \AgdaSymbol{:} \AgdaDatatype{Con}\AgdaSymbol{\}}\<%
\\
\>[0]\AgdaIndent{7}{}\<[7]%
\>[7]\AgdaSymbol{\{}\AgdaBound{A} \AgdaBound{B} \AgdaBound{C} \AgdaSymbol{:} \AgdaDatatype{Ty} \AgdaBound{Δ}\AgdaSymbol{\}}\<%
\\
\>[0]\AgdaIndent{7}{}\<[7]%
\>[7]\AgdaSymbol{→} \AgdaBound{A} \AgdaDatatype{≡} \AgdaBound{B} \AgdaSymbol{→} \AgdaBound{A} \AgdaFunction{+T} \AgdaBound{C} \AgdaDatatype{≡} \AgdaBound{B} \AgdaFunction{+T} \AgdaBound{C}\<%
\\
\>\AgdaFunction{wk-T} \AgdaInductiveConstructor{refl} \AgdaSymbol{=} \AgdaInductiveConstructor{refl}\<%
\\
%
\\
\>\AgdaFunction{wk-tm} \AgdaSymbol{:} \AgdaSymbol{\{}\AgdaBound{Γ} \AgdaBound{Δ} \AgdaSymbol{:} \AgdaDatatype{Con}\AgdaSymbol{\}}\<%
\\
\>[7]\AgdaIndent{9}{}\<[9]%
\>[9]\AgdaSymbol{\{}\AgdaBound{A} \AgdaSymbol{:} \AgdaDatatype{Ty} \AgdaBound{Δ}\AgdaSymbol{\}\{}\AgdaBound{δ} \AgdaSymbol{:} \AgdaBound{Γ} \AgdaDatatype{⇒} \AgdaBound{Δ}\AgdaSymbol{\}}\<%
\\
\>[7]\AgdaIndent{9}{}\<[9]%
\>[9]\AgdaSymbol{\{}\AgdaBound{B} \AgdaSymbol{:} \AgdaDatatype{Ty} \AgdaBound{Δ}\AgdaSymbol{\}\{}\AgdaBound{b} \AgdaSymbol{:} \AgdaDatatype{Tm} \AgdaSymbol{(}\AgdaBound{B} \AgdaFunction{[} \AgdaBound{δ} \AgdaFunction{]T}\AgdaSymbol{)\}} \<[40]%
\>[40]\<%
\\
\>[7]\AgdaIndent{9}{}\<[9]%
\>[9]\AgdaSymbol{→} \AgdaDatatype{Tm} \AgdaSymbol{(}\AgdaBound{A} \AgdaFunction{[} \AgdaBound{δ} \AgdaFunction{]T}\AgdaSymbol{)} \AgdaSymbol{→} \AgdaDatatype{Tm} \AgdaSymbol{((}\AgdaBound{A} \AgdaFunction{+T} \AgdaBound{B}\AgdaSymbol{)} \AgdaFunction{[} \AgdaBound{δ} \AgdaInductiveConstructor{,} \AgdaBound{b} \AgdaFunction{]T}\AgdaSymbol{)}\<%
\\
\>\AgdaFunction{wk-tm} \AgdaBound{t} \AgdaSymbol{=} \AgdaBound{t} \AgdaFunction{⟦} \AgdaFunction{+T[,]T} \AgdaFunction{⟫}\<%
\\
%
\\
\>\AgdaInductiveConstructor{v0} \<[5]%
\>[5]\AgdaFunction{[} \AgdaBound{δ} \AgdaInductiveConstructor{,} \AgdaBound{a} \AgdaFunction{]V} \AgdaSymbol{=} \AgdaFunction{wk-tm} \AgdaBound{a}\<%
\\
\>\AgdaInductiveConstructor{vS} \AgdaBound{x} \AgdaFunction{[} \AgdaBound{δ} \AgdaInductiveConstructor{,} \AgdaBound{a} \AgdaFunction{]V} \AgdaSymbol{=} \AgdaFunction{wk-tm} \AgdaSymbol{(}\AgdaBound{x} \AgdaFunction{[} \AgdaBound{δ} \AgdaFunction{]V}\AgdaSymbol{)}\<%
\\
%
\\
\>\AgdaFunction{wk-coh} \AgdaSymbol{:} \AgdaSymbol{\{}\AgdaBound{Γ} \AgdaBound{Δ} \AgdaSymbol{:} \AgdaDatatype{Con}\AgdaSymbol{\}}\<%
\\
\>[7]\AgdaIndent{9}{}\<[9]%
\>[9]\AgdaSymbol{\{}\AgdaBound{A} \AgdaSymbol{:} \AgdaDatatype{Ty} \AgdaBound{Δ}\AgdaSymbol{\}\{}\AgdaBound{δ} \AgdaSymbol{:} \AgdaBound{Γ} \AgdaDatatype{⇒} \AgdaBound{Δ}\AgdaSymbol{\}}\<%
\\
\>[7]\AgdaIndent{9}{}\<[9]%
\>[9]\AgdaSymbol{\{}\AgdaBound{B} \AgdaSymbol{:} \AgdaDatatype{Ty} \AgdaBound{Δ}\AgdaSymbol{\}\{}\AgdaBound{b} \AgdaSymbol{:} \AgdaDatatype{Tm} \AgdaSymbol{(}\AgdaBound{B} \AgdaFunction{[} \AgdaBound{δ} \AgdaFunction{]T}\AgdaSymbol{)\}} \<[40]%
\>[40]\<%
\\
\>[7]\AgdaIndent{9}{}\<[9]%
\>[9]\AgdaSymbol{\{}\AgdaBound{t} \AgdaSymbol{:} \AgdaDatatype{Tm} \AgdaSymbol{(}\AgdaBound{A} \AgdaFunction{[} \AgdaBound{δ} \AgdaFunction{]T}\AgdaSymbol{)\}} \<[29]%
\>[29]\<%
\\
\>[7]\AgdaIndent{9}{}\<[9]%
\>[9]\AgdaSymbol{→} \AgdaFunction{wk-tm} \AgdaSymbol{\{}B \AgdaSymbol{=} \AgdaBound{B}\AgdaSymbol{\}} \AgdaSymbol{\{}b \AgdaSymbol{=} \AgdaBound{b}\AgdaSymbol{\}} \AgdaBound{t} \AgdaDatatype{≅} \AgdaBound{t}\<%
\\
\>\AgdaFunction{wk-coh} \AgdaSymbol{=} \AgdaFunction{cohOp} \AgdaFunction{+T[,]T}\<%
\\
%
\\
\>\AgdaFunction{wk-coh+} \AgdaSymbol{:} \AgdaSymbol{\{}\AgdaBound{Γ} \AgdaBound{Δ} \AgdaSymbol{:} \AgdaDatatype{Con}\AgdaSymbol{\}}\<%
\\
\>[7]\AgdaIndent{9}{}\<[9]%
\>[9]\AgdaSymbol{\{}\AgdaBound{A} \AgdaSymbol{:} \AgdaDatatype{Ty} \AgdaBound{Δ}\AgdaSymbol{\}\{}\AgdaBound{δ} \AgdaSymbol{:} \AgdaBound{Γ} \AgdaDatatype{⇒} \AgdaBound{Δ}\AgdaSymbol{\}}\<%
\\
\>[7]\AgdaIndent{9}{}\<[9]%
\>[9]\AgdaSymbol{\{}\AgdaBound{B} \AgdaSymbol{:} \AgdaDatatype{Ty} \AgdaBound{Γ}\AgdaSymbol{\}} \<[20]%
\>[20]\<%
\\
\>[7]\AgdaIndent{9}{}\<[9]%
\>[9]\AgdaSymbol{\{}\AgdaBound{x} \AgdaSymbol{:} \AgdaDatatype{Tm} \AgdaSymbol{(}\AgdaBound{A} \AgdaFunction{[} \AgdaBound{δ} \AgdaFunction{]T} \AgdaFunction{+T} \AgdaBound{B}\AgdaSymbol{)\}}\<%
\\
\>[9]\AgdaIndent{10}{}\<[10]%
\>[10]\AgdaSymbol{→} \AgdaFunction{wk-tm+} \AgdaBound{B} \AgdaBound{x} \AgdaDatatype{≅} \AgdaBound{x}\<%
\\
\>\AgdaFunction{wk-coh+} \AgdaSymbol{=} \AgdaFunction{cohOp} \AgdaFunction{[+S]T}\<%
\\
%
\\
\>\AgdaFunction{wk-hom} \AgdaSymbol{:} \AgdaSymbol{\{}\AgdaBound{Γ} \AgdaBound{Δ} \AgdaSymbol{:} \AgdaDatatype{Con}\AgdaSymbol{\}}\<%
\\
\>[0]\AgdaIndent{9}{}\<[9]%
\>[9]\AgdaSymbol{\{}\AgdaBound{A} \AgdaSymbol{:} \AgdaDatatype{Ty} \AgdaBound{Δ}\AgdaSymbol{\}\{}\AgdaBound{δ} \AgdaSymbol{:} \AgdaBound{Γ} \AgdaDatatype{⇒} \AgdaBound{Δ}\AgdaSymbol{\}}\<%
\\
\>[0]\AgdaIndent{9}{}\<[9]%
\>[9]\AgdaSymbol{\{}\AgdaBound{B} \AgdaSymbol{:} \AgdaDatatype{Ty} \AgdaBound{Δ}\AgdaSymbol{\}\{}\AgdaBound{b} \AgdaSymbol{:} \AgdaDatatype{Tm} \AgdaSymbol{(}\AgdaBound{B} \AgdaFunction{[} \AgdaBound{δ} \AgdaFunction{]T}\AgdaSymbol{)\}} \<[40]%
\>[40]\<%
\\
\>[0]\AgdaIndent{9}{}\<[9]%
\>[9]\AgdaSymbol{\{}\AgdaBound{x} \AgdaBound{y} \AgdaSymbol{:} \AgdaDatatype{Tm} \AgdaSymbol{(}\AgdaBound{A} \AgdaFunction{[} \AgdaBound{δ} \AgdaFunction{]T}\AgdaSymbol{)\}}\<%
\\
\>[0]\AgdaIndent{9}{}\<[9]%
\>[9]\AgdaSymbol{→} \AgdaSymbol{(}\AgdaFunction{wk-tm} \AgdaSymbol{\{}B \AgdaSymbol{=} \AgdaBound{B}\AgdaSymbol{\}} \AgdaSymbol{\{}b \AgdaSymbol{=} \AgdaBound{b}\AgdaSymbol{\}} \AgdaBound{x} \AgdaInductiveConstructor{=h} \AgdaFunction{wk-tm} \<[45]%
\>[45]\<%
\\
\>[0]\AgdaIndent{9}{}\<[9]%
\>[9]\AgdaSymbol{\{}B \AgdaSymbol{=} \AgdaBound{B}\AgdaSymbol{\}} \AgdaSymbol{\{}b \AgdaSymbol{=} \AgdaBound{b}\AgdaSymbol{\}} \AgdaBound{y}\AgdaSymbol{)} \AgdaDatatype{≡} \AgdaSymbol{(}\AgdaBound{x} \AgdaInductiveConstructor{=h} \AgdaBound{y}\AgdaSymbol{)}\<%
\\
\>\AgdaFunction{wk-hom} \AgdaSymbol{=} \AgdaFunction{hom≡} \AgdaFunction{wk-coh} \AgdaFunction{wk-coh}\<%
\\
%
\\
%
\\
\>\AgdaFunction{wk-hom+} \AgdaSymbol{:} \AgdaSymbol{\{}\AgdaBound{Γ} \AgdaBound{Δ} \AgdaSymbol{:} \AgdaDatatype{Con}\AgdaSymbol{\}}\<%
\\
\>[0]\AgdaIndent{9}{}\<[9]%
\>[9]\AgdaSymbol{\{}\AgdaBound{A} \AgdaSymbol{:} \AgdaDatatype{Ty} \AgdaBound{Δ}\AgdaSymbol{\}\{}\AgdaBound{δ} \AgdaSymbol{:} \AgdaBound{Γ} \AgdaDatatype{⇒} \AgdaBound{Δ}\AgdaSymbol{\}}\<%
\\
\>[0]\AgdaIndent{9}{}\<[9]%
\>[9]\AgdaSymbol{\{}\AgdaBound{B} \AgdaSymbol{:} \AgdaDatatype{Ty} \AgdaBound{Γ}\AgdaSymbol{\}} \<[20]%
\>[20]\<%
\\
\>[0]\AgdaIndent{9}{}\<[9]%
\>[9]\AgdaSymbol{\{}\AgdaBound{x} \AgdaBound{y} \AgdaSymbol{:} \AgdaDatatype{Tm} \AgdaSymbol{(}\AgdaBound{A} \AgdaFunction{[} \AgdaBound{δ} \AgdaFunction{]T} \AgdaFunction{+T} \AgdaBound{B}\AgdaSymbol{)\}}\<%
\\
\>[0]\AgdaIndent{9}{}\<[9]%
\>[9]\AgdaSymbol{→} \AgdaSymbol{(}\AgdaFunction{wk-tm+} \AgdaBound{B} \AgdaBound{x} \AgdaInductiveConstructor{=h} \AgdaFunction{wk-tm+} \AgdaBound{B} \AgdaBound{y}\AgdaSymbol{)} \AgdaDatatype{≡} \AgdaSymbol{(}\AgdaBound{x} \AgdaInductiveConstructor{=h} \AgdaBound{y}\AgdaSymbol{)}\<%
\\
\>\AgdaFunction{wk-hom+} \AgdaSymbol{=} \AgdaFunction{hom≡} \AgdaFunction{wk-coh+} \AgdaFunction{wk-coh+}\<%
\\
%
\\
%
\\
\>\AgdaFunction{wk-⊚} \AgdaSymbol{:} \AgdaSymbol{\{}\AgdaBound{Γ} \AgdaBound{Δ} \AgdaBound{Θ} \AgdaSymbol{:} \AgdaDatatype{Con}\AgdaSymbol{\}}\<%
\\
\>[0]\AgdaIndent{7}{}\<[7]%
\>[7]\AgdaSymbol{\{}\AgdaBound{θ} \AgdaSymbol{:} \AgdaBound{Δ} \AgdaDatatype{⇒} \AgdaBound{Θ}\AgdaSymbol{\}\{}\AgdaBound{δ} \AgdaSymbol{:} \AgdaBound{Γ} \AgdaDatatype{⇒} \AgdaBound{Δ}\AgdaSymbol{\}\{}\AgdaBound{A} \AgdaSymbol{:} \AgdaDatatype{Ty} \AgdaBound{Θ}\AgdaSymbol{\}}\<%
\\
\>[0]\AgdaIndent{7}{}\<[7]%
\>[7]\AgdaSymbol{→} \AgdaDatatype{Tm} \AgdaSymbol{((}\AgdaBound{A} \AgdaFunction{[} \AgdaBound{θ} \AgdaFunction{]T}\AgdaSymbol{)}\AgdaFunction{[} \AgdaBound{δ} \AgdaFunction{]T}\AgdaSymbol{)} \AgdaSymbol{→} \AgdaDatatype{Tm} \AgdaSymbol{(}\AgdaBound{A} \AgdaFunction{[} \AgdaBound{θ} \AgdaFunction{⊚} \AgdaBound{δ} \AgdaFunction{]T}\AgdaSymbol{)}\<%
\\
\>\AgdaFunction{wk-⊚} \AgdaBound{t} \AgdaSymbol{=} \AgdaBound{t} \AgdaFunction{⟦} \AgdaFunction{[⊚]T} \AgdaFunction{⟫}\<%
\\
%
\\
\>\AgdaFunction{[+S]S} \AgdaSymbol{\{}δ \AgdaSymbol{=} \AgdaInductiveConstructor{•}\AgdaSymbol{\}} \AgdaSymbol{=} \AgdaInductiveConstructor{refl}\<%
\\
\>\AgdaFunction{[+S]S} \AgdaSymbol{\{}δ \AgdaSymbol{=} \AgdaBound{δ} \AgdaInductiveConstructor{,} \AgdaBound{a}\AgdaSymbol{\}} \AgdaSymbol{=} \AgdaFunction{S-eq} \AgdaFunction{[+S]S} \AgdaSymbol{(}\AgdaFunction{cohOp} \AgdaFunction{[⊚]T} \AgdaFunction{∾} \<[45]%
\>[45]\<%
\\
\>[0]\AgdaIndent{6}{}\<[6]%
\>[6]\AgdaSymbol{(}\AgdaFunction{[+S]tm} \AgdaBound{a} \AgdaFunction{∾} \AgdaFunction{cong+tm2} \AgdaFunction{[⊚]T}\AgdaSymbol{)} \AgdaFunction{∾} \AgdaFunction{wk-coh+} \AgdaFunction{-¹}\AgdaSymbol{)}\<%
\\
%
\\
%
\\
\>\AgdaFunction{wk+S+T} \AgdaSymbol{:} \AgdaSymbol{∀\{}\AgdaBound{Γ} \AgdaBound{Δ} \AgdaSymbol{:} \AgdaDatatype{Con}\AgdaSymbol{\}\{}\AgdaBound{A} \AgdaSymbol{:} \AgdaDatatype{Ty} \AgdaBound{Γ}\AgdaSymbol{\}\{}\AgdaBound{B} \AgdaSymbol{:} \AgdaDatatype{Ty} \AgdaBound{Δ}\AgdaSymbol{\}}\<%
\\
\>[0]\AgdaIndent{10}{}\<[10]%
\>[10]\AgdaSymbol{\{}\AgdaBound{γ}\AgdaSymbol{\}\{}\AgdaBound{C}\AgdaSymbol{\}} \AgdaSymbol{→} \<[19]%
\>[19]\<%
\\
\>[0]\AgdaIndent{10}{}\<[10]%
\>[10]\AgdaBound{A} \AgdaFunction{[} \AgdaBound{γ} \AgdaFunction{]T} \AgdaDatatype{≡} \AgdaBound{C} \<[23]%
\>[23]\<%
\\
\>[0]\AgdaIndent{7}{}\<[7]%
\>[7]\AgdaSymbol{→} \AgdaBound{A} \AgdaFunction{[} \AgdaBound{γ} \AgdaFunction{+S} \AgdaBound{B} \AgdaFunction{]T} \AgdaDatatype{≡} \AgdaBound{C} \AgdaFunction{+T} \AgdaBound{B}\<%
\\
\>\AgdaFunction{wk+S+T} \AgdaBound{eq} \AgdaSymbol{=} \AgdaFunction{trans} \AgdaFunction{[+S]T} \AgdaSymbol{(}\AgdaFunction{wk-T} \AgdaBound{eq}\AgdaSymbol{)}\<%
\\
%
\\
\>\AgdaFunction{wk+S+tm} \AgdaSymbol{:} \AgdaSymbol{\{}\AgdaBound{Γ} \AgdaBound{Δ} \AgdaSymbol{:} \AgdaDatatype{Con}\AgdaSymbol{\}\{}\AgdaBound{A} \AgdaSymbol{:} \AgdaDatatype{Ty} \AgdaBound{Γ}\AgdaSymbol{\}\{}\AgdaBound{B} \AgdaSymbol{:} \AgdaDatatype{Ty} \AgdaBound{Δ}\AgdaSymbol{\}}\<%
\\
\>[0]\AgdaIndent{10}{}\<[10]%
\>[10]\AgdaSymbol{(}\AgdaBound{a} \AgdaSymbol{:} \AgdaDatatype{Tm} \AgdaBound{A}\AgdaSymbol{)\{}\AgdaBound{C} \AgdaSymbol{:} \AgdaDatatype{Ty} \AgdaBound{Δ}\AgdaSymbol{\}\{}\AgdaBound{γ} \AgdaSymbol{:} \AgdaBound{Δ} \AgdaDatatype{⇒} \AgdaBound{Γ}\AgdaSymbol{\}\{}\AgdaBound{c} \AgdaSymbol{:} \AgdaDatatype{Tm} \AgdaBound{C}\AgdaSymbol{\}} \AgdaSymbol{→}\<%
\\
\>[0]\AgdaIndent{10}{}\<[10]%
\>[10]\AgdaBound{a} \AgdaFunction{[} \AgdaBound{γ} \AgdaFunction{]tm} \AgdaDatatype{≅} \AgdaBound{c} \<[24]%
\>[24]\<%
\\
\>[0]\AgdaIndent{8}{}\<[8]%
\>[8]\AgdaSymbol{→} \AgdaBound{a} \AgdaFunction{[} \AgdaBound{γ} \AgdaFunction{+S} \AgdaBound{B} \AgdaFunction{]tm} \AgdaDatatype{≅} \AgdaBound{c} \AgdaFunction{+tm} \AgdaBound{B}\<%
\\
\>\AgdaFunction{wk+S+tm} \AgdaSymbol{\_} \AgdaBound{eq} \AgdaSymbol{=} \AgdaFunction{[+S]tm} \AgdaSymbol{\_} \AgdaFunction{∾} \AgdaFunction{cong+tm} \AgdaBound{eq}\<%
\\
%
\\
%
\\
\>\AgdaFunction{wk+S+S} \AgdaSymbol{:} \AgdaSymbol{∀\{}\AgdaBound{Γ} \AgdaBound{Δ} \AgdaBound{Δ₁} \AgdaSymbol{:} \AgdaDatatype{Con}\AgdaSymbol{\}\{}\AgdaBound{δ} \AgdaSymbol{:} \AgdaBound{Δ} \AgdaDatatype{⇒} \AgdaBound{Δ₁}\AgdaSymbol{\}\{}\AgdaBound{γ} \AgdaSymbol{:} \AgdaBound{Γ} \AgdaDatatype{⇒} \AgdaBound{Δ}\AgdaSymbol{\}}\<%
\\
\>[0]\AgdaIndent{9}{}\<[9]%
\>[9]\AgdaSymbol{\{}\AgdaBound{ω} \AgdaSymbol{:} \AgdaBound{Γ} \AgdaDatatype{⇒} \AgdaBound{Δ₁}\AgdaSymbol{\}\{}\AgdaBound{B} \AgdaSymbol{:} \AgdaDatatype{Ty} \AgdaBound{Γ}\AgdaSymbol{\}}\<%
\\
\>[0]\AgdaIndent{7}{}\<[7]%
\>[7]\AgdaSymbol{→} \AgdaBound{δ} \AgdaFunction{⊚} \AgdaBound{γ} \AgdaDatatype{≡} \AgdaBound{ω}\<%
\\
\>[0]\AgdaIndent{7}{}\<[7]%
\>[7]\AgdaSymbol{→} \AgdaBound{δ} \AgdaFunction{⊚} \AgdaSymbol{(}\AgdaBound{γ} \AgdaFunction{+S} \AgdaBound{B}\AgdaSymbol{)} \AgdaDatatype{≡} \AgdaBound{ω} \AgdaFunction{+S} \AgdaBound{B}\<%
\\
\>\AgdaFunction{wk+S+S} \AgdaBound{eq} \AgdaSymbol{=} \AgdaFunction{trans} \AgdaFunction{[+S]S} \AgdaSymbol{(}\AgdaFunction{cong} \AgdaSymbol{(λ} \AgdaBound{x} \AgdaSymbol{→} \AgdaBound{x} \AgdaFunction{+S} \AgdaSymbol{\_)} \AgdaBound{eq}\AgdaSymbol{)}\<%
\\
%
\\
%
\\
\>\AgdaFunction{[⊚]T} \AgdaSymbol{\{}A \AgdaSymbol{=} \AgdaInductiveConstructor{*}\AgdaSymbol{\}} \AgdaSymbol{=} \AgdaInductiveConstructor{refl}\<%
\\
\>\AgdaFunction{[⊚]T} \AgdaSymbol{\{}A \AgdaSymbol{=} \AgdaInductiveConstructor{\_=h\_} \AgdaSymbol{\{}\AgdaBound{A}\AgdaSymbol{\}} \AgdaBound{a} \AgdaBound{b}\AgdaSymbol{\}} \AgdaSymbol{=} \AgdaFunction{hom≡} \AgdaSymbol{(}\AgdaFunction{[⊚]tm} \AgdaSymbol{\_)} \AgdaSymbol{(}\AgdaFunction{[⊚]tm} \AgdaSymbol{\_)} \<[51]%
\>[51]\<%
\\
%
\\
\>\AgdaFunction{+T[,]T} \AgdaSymbol{\{}A \AgdaSymbol{=} \AgdaInductiveConstructor{*}\AgdaSymbol{\}} \AgdaSymbol{=} \AgdaInductiveConstructor{refl}\<%
\\
\>\AgdaFunction{+T[,]T} \AgdaSymbol{\{}A \AgdaSymbol{=} \AgdaInductiveConstructor{\_=h\_} \AgdaSymbol{\{}\AgdaBound{A}\AgdaSymbol{\}} \AgdaBound{a} \AgdaBound{b}\AgdaSymbol{\}} \AgdaSymbol{=} \AgdaFunction{hom≡} \<[34]%
\>[34]\AgdaSymbol{(}\AgdaFunction{+tm[,]tm} \AgdaSymbol{\_)} \AgdaSymbol{(}\AgdaFunction{+tm[,]tm} \AgdaSymbol{\_)}\<%
\\
%
\\
\>\AgdaInductiveConstructor{var} \AgdaBound{x} \<[12]%
\>[12]\AgdaFunction{[} \AgdaBound{δ} \AgdaFunction{]tm} \AgdaSymbol{=} \AgdaBound{x} \AgdaFunction{[} \AgdaBound{δ} \AgdaFunction{]V}\<%
\\
\>\AgdaInductiveConstructor{coh} \AgdaBound{cΔ} \AgdaBound{γ} \AgdaBound{A} \<[12]%
\>[12]\AgdaFunction{[} \AgdaBound{δ} \AgdaFunction{]tm} \AgdaSymbol{=} \AgdaInductiveConstructor{coh} \AgdaBound{cΔ} \AgdaSymbol{(}\AgdaBound{γ} \AgdaFunction{⊚} \AgdaBound{δ}\AgdaSymbol{)} \AgdaBound{A} \AgdaFunction{⟦} \AgdaFunction{sym} \AgdaFunction{[⊚]T} \AgdaFunction{⟫}\<%
\\
\>\<\end{code}

\begin{code}\>\<%
\\
%
\\
%
\\
\>\AgdaFunction{congT} \AgdaSymbol{:} \AgdaSymbol{∀} \AgdaSymbol{\{}\AgdaBound{Γ} \AgdaBound{Δ} \AgdaSymbol{:} \AgdaDatatype{Con}\AgdaSymbol{\}\{}\AgdaBound{A} \AgdaBound{B} \AgdaSymbol{:} \AgdaDatatype{Ty} \AgdaBound{Δ}\AgdaSymbol{\}\{}\AgdaBound{γ} \AgdaSymbol{:} \AgdaBound{Γ} \AgdaDatatype{⇒} \AgdaBound{Δ}\AgdaSymbol{\}} \AgdaSymbol{→} \AgdaBound{A} \AgdaDatatype{≡} \AgdaBound{B} \AgdaSymbol{→} \AgdaBound{A} \AgdaFunction{[} \AgdaBound{γ} \AgdaFunction{]T} \AgdaDatatype{≡} \AgdaBound{B} \AgdaFunction{[} \AgdaBound{γ} \AgdaFunction{]T} \<[75]%
\>[75]\<%
\\
\>\AgdaFunction{congT} \AgdaInductiveConstructor{refl} \AgdaSymbol{=} \AgdaInductiveConstructor{refl}\<%
\\
%
\\
\>\AgdaFunction{congT2} \AgdaSymbol{:} \AgdaSymbol{∀} \AgdaSymbol{\{}\AgdaBound{Γ} \AgdaBound{Δ}\AgdaSymbol{\}} \AgdaSymbol{→} \AgdaSymbol{\{}\AgdaBound{δ} \AgdaBound{γ} \AgdaSymbol{:} \AgdaBound{Δ} \AgdaDatatype{⇒} \AgdaBound{Γ}\AgdaSymbol{\}\{}\AgdaBound{A} \AgdaSymbol{:} \AgdaDatatype{Ty} \AgdaBound{Γ}\AgdaSymbol{\}} \AgdaSymbol{→} \AgdaBound{δ} \AgdaDatatype{≡} \AgdaBound{γ} \AgdaSymbol{→} \AgdaBound{A} \AgdaFunction{[} \AgdaBound{δ} \AgdaFunction{]T} \AgdaDatatype{≡} \AgdaBound{A} \AgdaFunction{[} \AgdaBound{γ} \AgdaFunction{]T}\<%
\\
\>\AgdaFunction{congT2} \AgdaInductiveConstructor{refl} \AgdaSymbol{=} \AgdaInductiveConstructor{refl} \<[19]%
\>[19]\<%
\\
%
\\
\>\AgdaFunction{congV} \AgdaSymbol{:} \AgdaSymbol{\{}\AgdaBound{Γ} \AgdaBound{Δ} \AgdaSymbol{:} \AgdaDatatype{Con}\AgdaSymbol{\}\{}\AgdaBound{A} \AgdaBound{B} \AgdaSymbol{:} \AgdaDatatype{Ty} \AgdaBound{Δ}\AgdaSymbol{\}\{}\AgdaBound{a} \AgdaSymbol{:} \AgdaDatatype{Var} \AgdaBound{A}\AgdaSymbol{\}\{}\AgdaBound{b} \AgdaSymbol{:} \AgdaDatatype{Var} \AgdaBound{B}\AgdaSymbol{\}} \AgdaSymbol{→}\<%
\\
\>[0]\AgdaIndent{5}{}\<[5]%
\>[5]\AgdaInductiveConstructor{var} \AgdaBound{a} \AgdaDatatype{≅} \AgdaInductiveConstructor{var} \AgdaBound{b} \AgdaSymbol{→} \<[21]%
\>[21]\<%
\\
\>[0]\AgdaIndent{5}{}\<[5]%
\>[5]\AgdaSymbol{\{}\AgdaBound{δ} \AgdaSymbol{:} \AgdaBound{Γ} \AgdaDatatype{⇒} \AgdaBound{Δ}\AgdaSymbol{\}} \<[17]%
\>[17]\<%
\\
\>[0]\AgdaIndent{5}{}\<[5]%
\>[5]\AgdaSymbol{→} \AgdaBound{a} \AgdaFunction{[} \AgdaBound{δ} \AgdaFunction{]V} \AgdaDatatype{≅} \AgdaBound{b} \AgdaFunction{[} \AgdaBound{δ} \AgdaFunction{]V}\<%
\\
\>\AgdaFunction{congV} \AgdaSymbol{\{}\AgdaBound{Γ}\AgdaSymbol{\}} \AgdaSymbol{\{}\AgdaBound{Δ}\AgdaSymbol{\}} \AgdaSymbol{\{}\AgdaSymbol{.}\AgdaBound{B}\AgdaSymbol{\}} \AgdaSymbol{\{}\AgdaBound{B}\AgdaSymbol{\}} \AgdaSymbol{\{}\AgdaSymbol{.}\AgdaBound{b}\AgdaSymbol{\}} \AgdaSymbol{\{}\AgdaBound{b}\AgdaSymbol{\}} \AgdaSymbol{(}\AgdaInductiveConstructor{refl} \AgdaSymbol{.(}\AgdaInductiveConstructor{var} \AgdaBound{b}\AgdaSymbol{)}\AgdaSymbol{)} \AgdaSymbol{=} \AgdaInductiveConstructor{refl} \AgdaSymbol{\_}\<%
\\
%
\\
\>\AgdaFunction{congtm} \AgdaSymbol{:} \AgdaSymbol{\{}\AgdaBound{Γ} \AgdaBound{Δ} \AgdaSymbol{:} \AgdaDatatype{Con}\AgdaSymbol{\}\{}\AgdaBound{A} \AgdaBound{B} \AgdaSymbol{:} \AgdaDatatype{Ty} \AgdaBound{Γ}\AgdaSymbol{\}\{}\AgdaBound{a} \AgdaSymbol{:} \AgdaDatatype{Tm} \AgdaBound{A}\AgdaSymbol{\}\{}\AgdaBound{b} \AgdaSymbol{:} \AgdaDatatype{Tm} \AgdaBound{B}\AgdaSymbol{\}}\<%
\\
\>[5]\AgdaIndent{6}{}\<[6]%
\>[6]\AgdaSymbol{(}\AgdaBound{p} \AgdaSymbol{:} \AgdaBound{a} \AgdaDatatype{≅} \AgdaBound{b}\AgdaSymbol{)} \AgdaSymbol{→} \<[20]%
\>[20]\<%
\\
\>[5]\AgdaIndent{6}{}\<[6]%
\>[6]\AgdaSymbol{\{}\AgdaBound{δ} \AgdaSymbol{:} \AgdaBound{Δ} \AgdaDatatype{⇒} \AgdaBound{Γ}\AgdaSymbol{\}}\<%
\\
\>[5]\AgdaIndent{6}{}\<[6]%
\>[6]\AgdaSymbol{→} \AgdaBound{a} \AgdaFunction{[} \AgdaBound{δ} \AgdaFunction{]tm} \AgdaDatatype{≅} \AgdaBound{b} \AgdaFunction{[} \AgdaBound{δ} \AgdaFunction{]tm}\<%
\\
\>\AgdaFunction{congtm} \AgdaSymbol{(}\AgdaInductiveConstructor{refl} \AgdaSymbol{\_)} \AgdaSymbol{=} \AgdaInductiveConstructor{refl} \AgdaSymbol{\_} \<[25]%
\>[25]\<%
\\
%
\\
\>\AgdaFunction{congtm2} \AgdaSymbol{:} \AgdaSymbol{\{}\AgdaBound{Γ} \AgdaBound{Δ} \AgdaSymbol{:} \AgdaDatatype{Con}\AgdaSymbol{\}\{}\AgdaBound{A} \AgdaSymbol{:} \AgdaDatatype{Ty} \AgdaBound{Γ}\AgdaSymbol{\}\{}\AgdaBound{a} \AgdaSymbol{:} \AgdaDatatype{Tm} \AgdaBound{A}\AgdaSymbol{\}}\<%
\\
\>[6]\AgdaIndent{10}{}\<[10]%
\>[10]\AgdaSymbol{\{}\AgdaBound{δ} \AgdaBound{γ} \AgdaSymbol{:} \AgdaBound{Δ} \AgdaDatatype{⇒} \AgdaBound{Γ}\AgdaSymbol{\}} \AgdaSymbol{→}\<%
\\
\>[6]\AgdaIndent{10}{}\<[10]%
\>[10]\AgdaSymbol{(}\AgdaBound{p} \AgdaSymbol{:} \AgdaBound{δ} \AgdaDatatype{≡} \AgdaBound{γ}\AgdaSymbol{)}\<%
\\
\>[-4]\AgdaIndent{8}{}\<[8]%
\>[8]\AgdaSymbol{→} \AgdaBound{a} \AgdaFunction{[} \AgdaBound{δ} \AgdaFunction{]tm} \AgdaDatatype{≅} \AgdaBound{a} \AgdaFunction{[} \AgdaBound{γ} \AgdaFunction{]tm}\<%
\\
\>\AgdaFunction{congtm2} \AgdaInductiveConstructor{refl} \AgdaSymbol{=} \AgdaInductiveConstructor{refl} \AgdaSymbol{\_}\<%
\\
%
\\
\>\AgdaFunction{⊚assoc} \AgdaInductiveConstructor{•} \AgdaSymbol{=} \AgdaInductiveConstructor{refl}\<%
\\
\>\AgdaFunction{⊚assoc} \AgdaSymbol{(}\AgdaInductiveConstructor{\_,\_} \AgdaBound{γ} \AgdaSymbol{\{}\AgdaBound{A}\AgdaSymbol{\}} \AgdaBound{a}\AgdaSymbol{)} \AgdaSymbol{=} \AgdaFunction{S-eq} \AgdaSymbol{(}\AgdaFunction{⊚assoc} \AgdaBound{γ}\AgdaSymbol{)} \<[39]%
\>[39]\<%
\\
\>[0]\AgdaIndent{4}{}\<[4]%
\>[4]\AgdaSymbol{(}\AgdaFunction{cohOp} \AgdaFunction{[⊚]T} \<[16]%
\>[16]\<%
\\
\>[0]\AgdaIndent{4}{}\<[4]%
\>[4]\AgdaFunction{∾} \AgdaSymbol{(}\AgdaFunction{congtm} \AgdaSymbol{(}\AgdaFunction{cohOp} \AgdaFunction{[⊚]T}\AgdaSymbol{)}\<%
\\
\>[0]\AgdaIndent{4}{}\<[4]%
\>[4]\AgdaFunction{∾} \AgdaSymbol{((}\AgdaFunction{cohOp} \AgdaFunction{[⊚]T} \<[19]%
\>[19]\<%
\\
\>[0]\AgdaIndent{4}{}\<[4]%
\>[4]\AgdaFunction{∾} \AgdaFunction{[⊚]tm} \AgdaBound{a}\AgdaSymbol{)} \AgdaFunction{-¹}\AgdaSymbol{)))}\<%
\\
%
\\
%
\\
\>\AgdaFunction{[⊚]v} \AgdaSymbol{(}\AgdaInductiveConstructor{v0} \AgdaSymbol{\{}\AgdaBound{Γ₁}\AgdaSymbol{\}} \AgdaSymbol{\{}\AgdaBound{A}\AgdaSymbol{\})} \AgdaSymbol{\{}\AgdaBound{θ} \AgdaInductiveConstructor{,} \AgdaBound{a}\AgdaSymbol{\}} \<[28]%
\>[28]\AgdaSymbol{=} \AgdaFunction{wk-coh} \AgdaFunction{∾} \AgdaFunction{cohOp} \<[45]%
\>[45]\<%
\\
\>[0]\AgdaIndent{2}{}\<[2]%
\>[2]\AgdaFunction{[⊚]T} \AgdaFunction{∾} \AgdaFunction{congtm} \AgdaSymbol{(}\AgdaFunction{cohOp} \AgdaFunction{+T[,]T} \AgdaFunction{-¹}\AgdaSymbol{)} \<[34]%
\>[34]\<%
\\
\>\AgdaFunction{[⊚]v} \AgdaSymbol{(}\AgdaInductiveConstructor{vS} \AgdaSymbol{\{}\AgdaBound{Γ₁}\AgdaSymbol{\}} \AgdaSymbol{\{}\AgdaBound{A}\AgdaSymbol{\}} \AgdaSymbol{\{}\AgdaBound{B}\AgdaSymbol{\}} \AgdaBound{x}\AgdaSymbol{)} \AgdaSymbol{\{}\AgdaBound{θ} \AgdaInductiveConstructor{,} \AgdaBound{a}\AgdaSymbol{\}} \AgdaSymbol{=} \<[35]%
\>[35]\<%
\\
\>[0]\AgdaIndent{2}{}\<[2]%
\>[2]\AgdaFunction{wk-coh} \AgdaFunction{∾} \AgdaSymbol{(}\AgdaFunction{[⊚]v} \AgdaBound{x} \AgdaFunction{∾} \AgdaSymbol{(}\AgdaFunction{congtm} \AgdaSymbol{(}\AgdaFunction{cohOp} \AgdaFunction{+T[,]T}\AgdaSymbol{)} \AgdaFunction{-¹}\AgdaSymbol{))}\<%
\\
%
\\
%
\\
%
\\
\>\AgdaFunction{[⊚]tm} \AgdaSymbol{(}\AgdaInductiveConstructor{var} \AgdaBound{x}\AgdaSymbol{)} \AgdaSymbol{=} \AgdaFunction{[⊚]v} \AgdaBound{x}\<%
\\
\>\AgdaFunction{[⊚]tm} \AgdaSymbol{(}\AgdaInductiveConstructor{coh} \AgdaBound{c} \AgdaBound{γ} \AgdaBound{A}\AgdaSymbol{)} \AgdaSymbol{=} \AgdaFunction{cohOp} \AgdaSymbol{(}\AgdaFunction{sym} \AgdaFunction{[⊚]T}\AgdaSymbol{)} \AgdaFunction{∾} \AgdaSymbol{(}\AgdaFunction{coh-eq} \AgdaSymbol{(}\AgdaFunction{sym} \AgdaSymbol{(}\AgdaFunction{⊚assoc} \AgdaBound{γ}\AgdaSymbol{))}\<%
\\
\>[2]\AgdaIndent{11}{}\<[11]%
\>[11]\AgdaFunction{∾} \AgdaFunction{cohOp} \AgdaSymbol{(}\AgdaFunction{sym} \AgdaFunction{[⊚]T}\AgdaSymbol{)} \AgdaFunction{-¹}\AgdaSymbol{)} \AgdaFunction{∾} \AgdaFunction{congtm} \AgdaSymbol{(}\AgdaFunction{cohOp} \AgdaSymbol{(}\AgdaFunction{sym} \AgdaFunction{[⊚]T}\AgdaSymbol{)} \AgdaFunction{-¹}\AgdaSymbol{)}\<%
\\
%
\\
%
\\
\>\AgdaFunction{⊚wk} \AgdaSymbol{:} \AgdaSymbol{∀\{}\AgdaBound{Γ} \AgdaBound{Δ} \AgdaBound{Δ₁}\AgdaSymbol{\}\{}\AgdaBound{B} \AgdaSymbol{:} \AgdaDatatype{Ty} \AgdaBound{Δ}\AgdaSymbol{\}(}\AgdaBound{γ} \AgdaSymbol{:} \AgdaBound{Δ} \AgdaDatatype{⇒} \AgdaBound{Δ₁}\AgdaSymbol{)\{}\AgdaBound{δ} \AgdaSymbol{:} \AgdaBound{Γ} \AgdaDatatype{⇒} \AgdaBound{Δ}\AgdaSymbol{\}}\<%
\\
\>[3]\AgdaIndent{6}{}\<[6]%
\>[6]\AgdaSymbol{\{}\AgdaBound{c} \AgdaSymbol{:} \AgdaDatatype{Tm} \AgdaSymbol{(}\AgdaBound{B} \AgdaFunction{[} \AgdaBound{δ} \AgdaFunction{]T}\AgdaSymbol{)\}} \AgdaSymbol{→} \AgdaSymbol{(}\AgdaBound{γ} \AgdaFunction{+S} \AgdaBound{B}\AgdaSymbol{)} \AgdaFunction{⊚} \AgdaSymbol{(}\AgdaBound{δ} \AgdaInductiveConstructor{,} \AgdaBound{c}\AgdaSymbol{)} \AgdaDatatype{≡} \AgdaBound{γ} \AgdaFunction{⊚} \AgdaBound{δ}\<%
\\
\>\AgdaFunction{⊚wk} \AgdaInductiveConstructor{•} \AgdaSymbol{=} \AgdaInductiveConstructor{refl}\<%
\\
\>\AgdaFunction{⊚wk} \AgdaSymbol{(}\AgdaInductiveConstructor{\_,\_} \AgdaBound{γ} \AgdaSymbol{\{}\AgdaBound{A}\AgdaSymbol{\}} \AgdaBound{a}\AgdaSymbol{)} \AgdaSymbol{=} \AgdaFunction{S-eq} \AgdaSymbol{(}\AgdaFunction{⊚wk} \AgdaBound{γ}\AgdaSymbol{)} \AgdaSymbol{(}\AgdaFunction{cohOp} \AgdaFunction{[⊚]T} \AgdaFunction{∾}\<%
\\
\>[0]\AgdaIndent{4}{}\<[4]%
\>[4]\AgdaSymbol{(}\AgdaFunction{congtm} \AgdaSymbol{(}\AgdaFunction{cohOp} \AgdaFunction{[+S]T}\AgdaSymbol{)} \AgdaFunction{∾} \AgdaFunction{+tm[,]tm} \AgdaBound{a}\AgdaSymbol{)} \AgdaFunction{∾} \AgdaFunction{cohOp} \AgdaFunction{[⊚]T} \AgdaFunction{-¹}\AgdaSymbol{)}\<%
\\
%
\\
\>\AgdaFunction{+tm[,]tm} \AgdaSymbol{(}\AgdaInductiveConstructor{var} \AgdaBound{x}\AgdaSymbol{)} \AgdaSymbol{=} \AgdaFunction{cohOp} \AgdaFunction{+T[,]T}\<%
\\
\>\AgdaFunction{+tm[,]tm} \AgdaSymbol{(}\AgdaInductiveConstructor{coh} \AgdaBound{x} \AgdaBound{γ} \AgdaBound{A}\AgdaSymbol{)} \AgdaSymbol{=} \AgdaFunction{congtm} \AgdaSymbol{(}\AgdaFunction{cohOp} \AgdaSymbol{(}\AgdaFunction{sym} \AgdaFunction{[+S]T}\AgdaSymbol{))} \AgdaFunction{∾} \<[52]%
\>[52]\<%
\\
\>[0]\AgdaIndent{3}{}\<[3]%
\>[3]\AgdaFunction{cohOp} \AgdaSymbol{(}\AgdaFunction{sym} \AgdaFunction{[⊚]T}\AgdaSymbol{)} \AgdaFunction{∾} \AgdaFunction{coh-eq} \AgdaSymbol{(}\AgdaFunction{⊚wk} \AgdaBound{γ}\AgdaSymbol{)} \AgdaFunction{∾} \AgdaFunction{cohOp} \AgdaSymbol{(}\AgdaFunction{sym} \AgdaFunction{[⊚]T}\AgdaSymbol{)} \AgdaFunction{-¹}\<%
\\
%
\\
%
\\
%
\\
\>\AgdaFunction{[+S]V} \AgdaSymbol{:} \AgdaSymbol{\{}\AgdaBound{Γ} \AgdaBound{Δ} \AgdaSymbol{:} \AgdaDatatype{Con}\AgdaSymbol{\}\{}\AgdaBound{A} \AgdaSymbol{:} \AgdaDatatype{Ty} \AgdaBound{Δ}\AgdaSymbol{\}}\<%
\\
\>[0]\AgdaIndent{9}{}\<[9]%
\>[9]\AgdaSymbol{(}\AgdaBound{x} \AgdaSymbol{:} \AgdaDatatype{Var} \AgdaBound{A}\AgdaSymbol{)\{}\AgdaBound{δ} \AgdaSymbol{:} \AgdaBound{Γ} \AgdaDatatype{⇒} \AgdaBound{Δ}\AgdaSymbol{\}}\<%
\\
\>[0]\AgdaIndent{9}{}\<[9]%
\>[9]\AgdaSymbol{\{}\AgdaBound{B} \AgdaSymbol{:} \AgdaDatatype{Ty} \AgdaBound{Γ}\AgdaSymbol{\}}\<%
\\
\>[0]\AgdaIndent{9}{}\<[9]%
\>[9]\AgdaSymbol{→} \AgdaBound{x} \AgdaFunction{[} \AgdaBound{δ} \AgdaFunction{+S} \AgdaBound{B} \AgdaFunction{]V} \AgdaDatatype{≅} \AgdaSymbol{(}\AgdaBound{x} \AgdaFunction{[} \AgdaBound{δ} \AgdaFunction{]V}\AgdaSymbol{)} \AgdaFunction{+tm} \AgdaBound{B}\<%
\\
\>\AgdaFunction{[+S]V} \AgdaInductiveConstructor{v0} \AgdaSymbol{\{}\AgdaInductiveConstructor{\_,\_} \AgdaBound{δ} \AgdaSymbol{\{}\AgdaBound{A}\AgdaSymbol{\}} \AgdaBound{a}\AgdaSymbol{\}} \AgdaSymbol{=} \AgdaFunction{wk-coh} \AgdaFunction{∾} \AgdaFunction{wk-coh+} \AgdaFunction{∾} \AgdaFunction{cong+tm2} \AgdaFunction{+T[,]T}\<%
\\
\>\AgdaFunction{[+S]V} \AgdaSymbol{(}\AgdaInductiveConstructor{vS} \AgdaBound{x}\AgdaSymbol{)} \AgdaSymbol{\{}\AgdaBound{δ} \AgdaInductiveConstructor{,} \AgdaBound{a}\AgdaSymbol{\}} \AgdaSymbol{=} \AgdaFunction{wk-coh} \AgdaFunction{∾} \AgdaFunction{[+S]V} \AgdaBound{x} \AgdaFunction{∾} \AgdaFunction{cong+tm2} \AgdaFunction{+T[,]T}\<%
\\
%
\\
%
\\
\>\AgdaFunction{[+S]tm} \AgdaSymbol{(}\AgdaInductiveConstructor{var} \AgdaBound{x}\AgdaSymbol{)} \AgdaSymbol{=} \AgdaFunction{[+S]V} \AgdaBound{x}\<%
\\
\>\AgdaFunction{[+S]tm} \AgdaSymbol{(}\AgdaInductiveConstructor{coh} \AgdaBound{x} \AgdaBound{δ} \AgdaBound{A}\AgdaSymbol{)} \AgdaSymbol{=} \AgdaFunction{cohOp} \AgdaSymbol{(}\AgdaFunction{sym} \AgdaFunction{[⊚]T}\AgdaSymbol{)} \AgdaFunction{∾} \AgdaFunction{coh-eq} \AgdaFunction{[+S]S} \AgdaFunction{∾} \<[55]%
\>[55]\<%
\\
\>[0]\AgdaIndent{2}{}\<[2]%
\>[2]\AgdaFunction{cohOp} \AgdaSymbol{(}\AgdaFunction{sym} \AgdaFunction{[+S]T}\AgdaSymbol{)} \AgdaFunction{-¹} \AgdaFunction{∾} \AgdaFunction{cong+tm2} \AgdaSymbol{(}\AgdaFunction{sym} \AgdaFunction{[⊚]T}\AgdaSymbol{)}\<%
\\
\>\<\end{code}

\textbf{Some simple contexts}

\begin{code}\>\<%
\\
\>\AgdaFunction{x:*} \AgdaSymbol{:} \AgdaDatatype{Con}\<%
\\
\>\AgdaFunction{x:*} \AgdaSymbol{=} \AgdaInductiveConstructor{ε} \AgdaInductiveConstructor{,} \AgdaInductiveConstructor{*}\<%
\\
%
\\
\>\AgdaFunction{x:*,y:*,α:x=y} \AgdaSymbol{:} \AgdaDatatype{Con}\<%
\\
\>\AgdaFunction{x:*,y:*,α:x=y} \AgdaSymbol{=} \AgdaFunction{x:*} \AgdaInductiveConstructor{,} \AgdaInductiveConstructor{*} \AgdaInductiveConstructor{,} \AgdaSymbol{(}\AgdaInductiveConstructor{var} \AgdaSymbol{(}\AgdaInductiveConstructor{vS} \AgdaInductiveConstructor{v0}\AgdaSymbol{)} \AgdaInductiveConstructor{=h} \AgdaInductiveConstructor{var} \AgdaInductiveConstructor{v0}\AgdaSymbol{)}\<%
\\
%
\\
\>\AgdaFunction{vX} \AgdaSymbol{:} \AgdaDatatype{Tm} \AgdaSymbol{\{}\AgdaFunction{x:*,y:*,α:x=y}\AgdaSymbol{\}} \AgdaInductiveConstructor{*}\<%
\\
\>\AgdaFunction{vX} \AgdaSymbol{=} \AgdaInductiveConstructor{var} \AgdaSymbol{(}\AgdaInductiveConstructor{vS} \AgdaSymbol{(}\AgdaInductiveConstructor{vS} \AgdaInductiveConstructor{v0}\AgdaSymbol{))}\<%
\\
%
\\
\>\AgdaFunction{vY} \AgdaSymbol{:} \AgdaDatatype{Tm} \AgdaSymbol{\{}\AgdaFunction{x:*,y:*,α:x=y}\AgdaSymbol{\}} \AgdaInductiveConstructor{*}\<%
\\
\>\AgdaFunction{vY} \AgdaSymbol{=} \AgdaInductiveConstructor{var} \AgdaSymbol{(}\AgdaInductiveConstructor{vS} \AgdaInductiveConstructor{v0}\AgdaSymbol{)}\<%
\\
%
\\
\>\AgdaFunction{vα} \AgdaSymbol{:} \AgdaDatatype{Tm} \AgdaSymbol{\{}\AgdaFunction{x:*,y:*,α:x=y}\AgdaSymbol{\}} \AgdaSymbol{(}\AgdaFunction{vX} \AgdaInductiveConstructor{=h} \AgdaFunction{vY}\AgdaSymbol{)}\<%
\\
\>\AgdaFunction{vα} \AgdaSymbol{=} \AgdaInductiveConstructor{var} \AgdaInductiveConstructor{v0}\<%
\\
%
\\
\>\AgdaFunction{x:*,y:*,α:x=y,z:*,β:y=z} \AgdaSymbol{:} \AgdaDatatype{Con}\<%
\\
\>\AgdaFunction{x:*,y:*,α:x=y,z:*,β:y=z} \AgdaSymbol{=} \AgdaFunction{x:*,y:*,α:x=y} \AgdaInductiveConstructor{,} \AgdaInductiveConstructor{*} \AgdaInductiveConstructor{,} \<[46]%
\>[46]\<%
\\
\>[0]\AgdaIndent{2}{}\<[2]%
\>[2]\AgdaSymbol{(}\AgdaInductiveConstructor{var} \AgdaSymbol{(}\AgdaInductiveConstructor{vS} \AgdaSymbol{(}\AgdaInductiveConstructor{vS} \AgdaInductiveConstructor{v0}\AgdaSymbol{))} \AgdaInductiveConstructor{=h} \AgdaInductiveConstructor{var} \AgdaInductiveConstructor{v0}\AgdaSymbol{)}\<%
\\
%
\\
\>\AgdaFunction{vZ} \AgdaSymbol{:} \AgdaDatatype{Tm} \AgdaSymbol{\{}\AgdaFunction{x:*,y:*,α:x=y,z:*,β:y=z}\AgdaSymbol{\}} \AgdaInductiveConstructor{*}\<%
\\
\>\AgdaFunction{vZ} \AgdaSymbol{=} \AgdaInductiveConstructor{var} \AgdaSymbol{(}\AgdaInductiveConstructor{vS} \AgdaInductiveConstructor{v0}\AgdaSymbol{)}\<%
\\
%
\\
\>\AgdaFunction{vβ} \AgdaSymbol{:} \AgdaDatatype{Tm} \AgdaSymbol{\{}\AgdaFunction{x:*,y:*,α:x=y,z:*,β:y=z}\AgdaSymbol{\}} \AgdaSymbol{(}\AgdaFunction{vY} \AgdaFunction{+tm} \AgdaSymbol{\_} \AgdaFunction{+tm} \AgdaSymbol{\_} \AgdaInductiveConstructor{=h} \AgdaFunction{vZ}\AgdaSymbol{)}\<%
\\
\>\AgdaFunction{vβ} \AgdaSymbol{=} \AgdaInductiveConstructor{var} \AgdaInductiveConstructor{v0}\<%
\\
\>\<\end{code}




\section{Some Important Derivable Constructions}


\AgdaHide{
\begin{code}\>\<%
\\
\>\AgdaKeyword{module} \AgdaModule{IdentityContextMorphisms} \AgdaKeyword{where}\<%
\\
%
\\
%
\\
\>\AgdaKeyword{open} \AgdaKeyword{import} \AgdaModule{BasicSyntax}\<%
\\
\>\AgdaKeyword{open} \AgdaKeyword{import} \AgdaModule{Relation.Binary.PropositionalEquality} \<[50]%
\>[50]\<%
\\
\>\AgdaKeyword{open} \AgdaKeyword{import} \AgdaModule{Data.Product} \AgdaKeyword{renaming} \AgdaSymbol{(}\_,\_ \AgdaSymbol{to} \_,,\_\AgdaSymbol{)}\<%
\\
\>\AgdaKeyword{open} \AgdaKeyword{import} \AgdaModule{Data.Nat}\<%
\\
%
\\
\>\<\end{code}
}
\newcommand{\Tm}{\mathsf{Tm}}
\newcommand{\Ty}{\mathsf{Ty}}



In this section we show that it is possible to reconstruct the structure
of a (weak) $\omega$-groupoid from the syntactical framework presented
in Section \ref{sec:syntax} in the style of \cite{txa:csl}. To 
this end, let us call a term $a : \Tm~\AgdaBound{A}$ an $n$-cell if
$\AgdaFunction{level}~\AgdaBound{A}~ \AgdaSymbol{\equiv}~ \AgdaBound{n}$, where 

\begin{code}\>\<%
\\
\>\AgdaFunction{level} \<[22]%
\>[22]\AgdaSymbol{:} \AgdaSymbol{∀} \AgdaSymbol{\{}\AgdaBound{Γ}\AgdaSymbol{\}} \AgdaSymbol{→} \AgdaDatatype{Ty} \AgdaBound{Γ} \AgdaSymbol{→} \AgdaDatatype{ℕ}\<%
\\
\>\AgdaFunction{level} \AgdaInductiveConstructor{*} \<[22]%
\>[22]\AgdaSymbol{=} \AgdaNumber{0}\<%
\\
\>\AgdaFunction{level} \AgdaSymbol{(}\AgdaInductiveConstructor{\_=h\_} \AgdaSymbol{\{}\AgdaBound{A}\AgdaSymbol{\}} \AgdaSymbol{\_} \AgdaSymbol{\_)} \<[22]%
\>[22]\AgdaSymbol{=} \AgdaInductiveConstructor{suc} \AgdaSymbol{(}\AgdaFunction{level} \AgdaBound{A}\AgdaSymbol{)} \<[38]%
\>[38]\<%
\\
\>\<\end{code}
%
In any $\omega$-category, any $n$-cell $a$ has a  domain (source), $s^n_m\,a$, and
a codomain (target), $t^n_m\,a$, for each $m \le n$. These are, of
course, $(n \textminus m)$-cells. For each pair of $n$-cells such that for some
$m$, $s^n_m a \equiv t^n_m b$, there must exist their composition
${a \circ^n_m b}$ which is an $n$-cell. Composition is (weakly)
associative. Moreover for any $(n \textminus m)$-cell $\AgdaBound{x}$ there
exists an $n$-cell $\mathsf{id}^n_m\,\AgdaBound{x}$ which
behaves like a (weak) identity with respect to $\circ^n_m$.
For the time being we discuss only the construction of cells and omit
the question of coherence. 

For instance, in the simple case of bicategories, each $2$-cell $a$ has a
horizontal source $s^1_1\,a$ and target $t^1_1\,a$, and also a vertical source
$s^2_1\,a$ and target $t^2_1 a$,
which is also the source and target, of the horizontal source and target,
respectively, of $a$. There is horizontal composition of $1$-cells $\circ^1_1$: $x
\to^f y \to^g z$, and also horizontal composition of $2$-cells
$\circ^2_1$, and vertical composition of $2$-cells $\circ^2_2$. There
is a horizontal identity on $a$, $\mathsf{id}^1_1\,a$, and vertical
identity on $a$, $\mathsf{id}^2_1\,a =
\mathsf{id}^2_2\mathsf{id}^1_1\,a$. 

Thus each $\omega$-groupoid construction is defined with respect to a
\emph{level}, $m$, and depth $n \textminus m$ and the structure of
an $\omega$-groupoid is repeated on each level. As we are working purely syntactically we
may make use of this fact and define all groupoid structure only at level
$m=1$ and provide a so-called \emph{replacement operation} which allows us to lift
any cell to an arbitrary type $A$. It is called 'replacement' because
we are syntactically replacing the base type $*$ with an arbitrary
type, $A$.

An important general mechanism we rely on throughout the development
follows directly from the type of the only nontrivial constructor of $\Tm$,
$\mathsf{coh}$, which tells us that to construct a
new term of type $\Gamma \vdash A$, we need a contractible context,
$\Delta$, a type $\Delta\vdash T$ and a context morphism $\delta :
\Gamma \Rightarrow \Delta$ such that
%
\[
\AgdaBound{T} \,\AgdaFunction{[}\, \AgdaBound{δ}\,
\AgdaFunction{]T}~\AgdaDatatype{≡}~\AgdaBound{A}
\]
%
Because in a contractible context all types are inhabited we may in a
way work freely in $\Delta$ and then pull back all terms to $A$ using
$\delta$. 
To show this formally, we must first define identity context morphisms
which complete the definition of a \emph{category} of contexts and
context morphisms:

\begin{code}\>\<%
\\
\>\AgdaFunction{IdCm} \AgdaSymbol{:} \AgdaSymbol{∀\{}\AgdaBound{Γ}\AgdaSymbol{\}} \AgdaSymbol{→} \AgdaBound{Γ} \AgdaDatatype{⇒} \AgdaBound{Γ}\<%
\\
\>\<\end{code}
It satisfies the following property:

\begin{code}\>\<%
\\
\>\AgdaFunction{IC-T} \AgdaSymbol{:} \AgdaSymbol{∀\{}\AgdaBound{Γ}\AgdaSymbol{\}\{}\AgdaBound{A} \AgdaSymbol{:} \AgdaDatatype{Ty} \AgdaBound{Γ}\AgdaSymbol{\}} \AgdaSymbol{→} \AgdaBound{A} \AgdaFunction{[} \AgdaFunction{IdCm} \AgdaFunction{]T} \AgdaDatatype{≡} \AgdaBound{A}\<%
\\
\>\<\end{code}
The definition proceeds by structural recursion and therefore extends
to terms, variables and context morphisms with analogous properties. 
It allows us to define at once:

\begin{code}\>\<%
\\
\>\AgdaFunction{Coh-Contr} \<[15]%
\>[15]\AgdaSymbol{:} \AgdaSymbol{∀\{}\AgdaBound{Γ}\AgdaSymbol{\}\{}\AgdaBound{A} \AgdaSymbol{:} \AgdaDatatype{Ty} \AgdaBound{Γ}\AgdaSymbol{\}} \AgdaSymbol{→} \AgdaDatatype{isContr} \AgdaBound{Γ} \AgdaSymbol{→} \AgdaDatatype{Tm} \AgdaBound{A}\<%
\\
\>\AgdaFunction{Coh-Contr} \AgdaBound{isC} \<[15]%
\>[15]\AgdaSymbol{=} \AgdaInductiveConstructor{coh} \AgdaBound{isC} \AgdaFunction{IdCm} \AgdaSymbol{\_} \AgdaFunction{⟦} \AgdaFunction{sym} \AgdaFunction{IC-T} \AgdaFunction{⟫}\<%
\\
\>\<\end{code}
We use $\AgdaFunction{Coh-Contr}$ as follows: for each kind of cell we
want to define, we construct a minimal contractible context built out
of variables together with a context morphism that populates the
context with terms and a lemma that states an equality
between the substitution and the original type.

\AgdaHide{
\begin{code}\>\<%
\\
\>\AgdaFunction{IC-v} \<[6]%
\>[6]\AgdaSymbol{:} \AgdaSymbol{∀\{}\AgdaBound{Γ} \AgdaSymbol{:} \AgdaDatatype{Con}\AgdaSymbol{\}\{}\AgdaBound{A} \AgdaSymbol{:} \AgdaDatatype{Ty} \AgdaBound{Γ}\AgdaSymbol{\}(}\AgdaBound{x} \AgdaSymbol{:} \AgdaDatatype{Var} \AgdaBound{A}\AgdaSymbol{)} \AgdaSymbol{→} \AgdaBound{x} \AgdaFunction{[} \AgdaFunction{IdCm} \AgdaFunction{]V} \AgdaDatatype{≅} \AgdaInductiveConstructor{var} \AgdaBound{x}\<%
\\
\>\AgdaFunction{IC-cm} \<[7]%
\>[7]\AgdaSymbol{:} \AgdaSymbol{∀\{}\AgdaBound{Γ} \AgdaBound{Δ} \AgdaSymbol{:} \AgdaDatatype{Con}\AgdaSymbol{\}(}\AgdaBound{δ} \AgdaSymbol{:} \AgdaBound{Γ} \AgdaDatatype{⇒} \AgdaBound{Δ}\AgdaSymbol{)} \<[40]%
\>[40]\AgdaSymbol{→} \AgdaBound{δ} \AgdaFunction{⊚} \AgdaFunction{IdCm} \AgdaDatatype{≡} \AgdaBound{δ}\<%
\\
\>\AgdaFunction{IC-tm} \AgdaSymbol{:} \AgdaSymbol{∀\{}\AgdaBound{Γ} \AgdaSymbol{:} \AgdaDatatype{Con}\AgdaSymbol{\}\{}\AgdaBound{A} \AgdaSymbol{:} \AgdaDatatype{Ty} \AgdaBound{Γ}\AgdaSymbol{\}(}\AgdaBound{a} \AgdaSymbol{:} \AgdaDatatype{Tm} \AgdaBound{A}\AgdaSymbol{)} \AgdaSymbol{→} \AgdaBound{a} \AgdaFunction{[} \AgdaFunction{IdCm} \AgdaFunction{]tm} \AgdaDatatype{≅} \AgdaBound{a}\<%
\\
%
\\
\>\AgdaFunction{IdCm} \AgdaSymbol{\{}\AgdaInductiveConstructor{ε}\AgdaSymbol{\}} \<[15]%
\>[15]\AgdaSymbol{=} \AgdaInductiveConstructor{•}\<%
\\
\>\AgdaFunction{IdCm} \AgdaSymbol{\{}\AgdaBound{Γ} \AgdaInductiveConstructor{,} \AgdaBound{A}\AgdaSymbol{\}} \AgdaSymbol{=} \AgdaFunction{IdCm} \AgdaFunction{+S} \AgdaSymbol{\_} \AgdaInductiveConstructor{,} \AgdaInductiveConstructor{var} \AgdaInductiveConstructor{v0} \AgdaFunction{⟦} \AgdaFunction{wk+S+T} \AgdaFunction{IC-T} \AgdaFunction{⟫}\<%
\\
%
\\
\>\AgdaFunction{IC-T} \AgdaSymbol{\{}\AgdaBound{Γ}\AgdaSymbol{\}} \AgdaSymbol{\{}\AgdaInductiveConstructor{*}\AgdaSymbol{\}} \AgdaSymbol{=} \AgdaInductiveConstructor{refl}\<%
\\
\>\AgdaFunction{IC-T} \AgdaSymbol{\{}\AgdaBound{Γ}\AgdaSymbol{\}} \AgdaSymbol{\{}\AgdaBound{a} \AgdaInductiveConstructor{=h} \AgdaBound{b}\AgdaSymbol{\}} \AgdaSymbol{=} \AgdaFunction{hom≡} \AgdaSymbol{(}\AgdaFunction{IC-tm} \AgdaBound{a}\AgdaSymbol{)} \AgdaSymbol{(}\AgdaFunction{IC-tm} \AgdaBound{b}\AgdaSymbol{)}\<%
\\
%
\\
\>\AgdaFunction{IC-v} \AgdaSymbol{\{}\AgdaSymbol{.(}\AgdaBound{Γ} \AgdaInductiveConstructor{,} \AgdaBound{A}\AgdaSymbol{)}\AgdaSymbol{\}} \AgdaSymbol{\{}\AgdaSymbol{.(}\AgdaBound{A} \AgdaFunction{+T} \AgdaBound{A}\AgdaSymbol{)}\AgdaSymbol{\}} \AgdaSymbol{(}\AgdaInductiveConstructor{v0} \AgdaSymbol{\{}\AgdaBound{Γ}\AgdaSymbol{\}} \AgdaSymbol{\{}\AgdaBound{A}\AgdaSymbol{\})} \AgdaSymbol{=} \AgdaFunction{wk-coh} \AgdaFunction{∾} \AgdaFunction{cohOp} \AgdaSymbol{(}\AgdaFunction{wk+S+T} \AgdaFunction{IC-T}\AgdaSymbol{)}\<%
\\
\>\AgdaFunction{IC-v} \AgdaSymbol{\{}\AgdaSymbol{.(}\AgdaBound{Γ} \AgdaInductiveConstructor{,} \AgdaBound{B}\AgdaSymbol{)}\AgdaSymbol{\}} \AgdaSymbol{\{}\AgdaSymbol{.(}\AgdaBound{A} \AgdaFunction{+T} \AgdaBound{B}\AgdaSymbol{)}\AgdaSymbol{\}} \AgdaSymbol{(}\AgdaInductiveConstructor{vS} \AgdaSymbol{\{}\AgdaBound{Γ}\AgdaSymbol{\}} \AgdaSymbol{\{}\AgdaBound{A}\AgdaSymbol{\}} \AgdaSymbol{\{}\AgdaBound{B}\AgdaSymbol{\}} \AgdaBound{x}\AgdaSymbol{)} \AgdaSymbol{=} \AgdaFunction{wk-coh} \AgdaFunction{∾} \AgdaFunction{wk+S+tm} \AgdaSymbol{(}\AgdaInductiveConstructor{var} \AgdaBound{x}\AgdaSymbol{)} \AgdaSymbol{(}\AgdaFunction{IC-v} \AgdaSymbol{\_)}\<%
\\
%
\\
\>\AgdaFunction{IC-cm} \AgdaInductiveConstructor{•} \AgdaSymbol{=} \AgdaInductiveConstructor{refl}\<%
\\
\>\AgdaFunction{IC-cm} \AgdaSymbol{(}\AgdaBound{δ} \AgdaInductiveConstructor{,} \AgdaBound{a}\AgdaSymbol{)} \AgdaSymbol{=} \AgdaFunction{cm-eq} \AgdaSymbol{(}\AgdaFunction{IC-cm} \AgdaBound{δ}\AgdaSymbol{)} \AgdaSymbol{(}\AgdaFunction{cohOp} \AgdaFunction{[⊚]T} \AgdaFunction{∾} \AgdaFunction{IC-tm} \AgdaBound{a}\AgdaSymbol{)} \<[55]%
\>[55]\<%
\\
%
\\
\>\AgdaFunction{IC-tm} \AgdaSymbol{(}\AgdaInductiveConstructor{var} \AgdaBound{x}\AgdaSymbol{)} \AgdaSymbol{=} \AgdaFunction{IC-v} \AgdaBound{x}\<%
\\
\>\AgdaFunction{IC-tm} \AgdaSymbol{(}\AgdaInductiveConstructor{coh} \AgdaBound{x} \AgdaBound{δ} \AgdaBound{A}\AgdaSymbol{)} \AgdaSymbol{=} \AgdaFunction{cohOp} \AgdaSymbol{(}\AgdaFunction{sym} \AgdaFunction{[⊚]T}\AgdaSymbol{)} \AgdaFunction{∾} \AgdaFunction{coh-eq} \AgdaSymbol{(}\AgdaFunction{IC-cm} \AgdaBound{δ}\AgdaSymbol{)}\<%
\\
%
\\
\>\AgdaFunction{pr1} \AgdaSymbol{:} \AgdaSymbol{∀} \AgdaSymbol{\{}\AgdaBound{Γ} \AgdaBound{A}\AgdaSymbol{\}} \AgdaSymbol{→} \AgdaSymbol{(}\AgdaBound{Γ} \AgdaInductiveConstructor{,} \AgdaBound{A}\AgdaSymbol{)} \AgdaDatatype{⇒} \AgdaBound{Γ}\<%
\\
\>\AgdaFunction{pr2} \AgdaSymbol{:} \AgdaSymbol{∀} \AgdaSymbol{\{}\AgdaBound{Γ} \AgdaBound{A}\AgdaSymbol{\}} \AgdaSymbol{→} \AgdaDatatype{Tm} \AgdaSymbol{\{}\AgdaBound{Γ} \AgdaInductiveConstructor{,} \AgdaBound{A}\AgdaSymbol{\}} \AgdaSymbol{(}\AgdaBound{A} \AgdaFunction{[} \AgdaFunction{pr1} \AgdaFunction{]T}\AgdaSymbol{)}\<%
\\
%
\\
\>\AgdaFunction{pr1-wk-T} \<[10]%
\>[10]\AgdaSymbol{:} \AgdaSymbol{∀\{}\AgdaBound{Γ} \AgdaSymbol{:} \AgdaDatatype{Con}\AgdaSymbol{\}\{}\AgdaBound{A} \AgdaBound{B} \AgdaSymbol{:} \AgdaDatatype{Ty} \AgdaBound{Γ}\AgdaSymbol{\}} \AgdaSymbol{→} \AgdaBound{A} \AgdaFunction{[} \AgdaFunction{pr1} \AgdaFunction{]T} \AgdaDatatype{≡} \AgdaBound{A} \AgdaFunction{+T} \AgdaBound{B}\<%
\\
\>\AgdaFunction{pr1-wk-tm} \AgdaSymbol{:} \AgdaSymbol{∀\{}\AgdaBound{Γ} \AgdaSymbol{:} \AgdaDatatype{Con}\AgdaSymbol{\}\{}\AgdaBound{A} \AgdaBound{B} \AgdaSymbol{:} \AgdaDatatype{Ty} \AgdaBound{Γ}\AgdaSymbol{\}\{}\AgdaBound{a} \AgdaSymbol{:} \AgdaDatatype{Tm} \AgdaBound{A}\AgdaSymbol{\}} \<[45]%
\>[45]\<%
\\
\>[0]\AgdaIndent{10}{}\<[10]%
\>[10]\AgdaSymbol{→} \AgdaBound{a} \AgdaFunction{[} \AgdaFunction{pr1} \AgdaFunction{]tm} \AgdaDatatype{≅} \AgdaBound{a} \AgdaFunction{+tm} \AgdaBound{B}\<%
\\
\>\AgdaFunction{pr1-wk-cm} \AgdaSymbol{:} \AgdaSymbol{∀\{}\AgdaBound{Γ} \AgdaBound{Δ} \AgdaSymbol{:} \AgdaDatatype{Con}\AgdaSymbol{\}\{}\AgdaBound{A} \AgdaBound{B} \AgdaSymbol{:} \AgdaDatatype{Ty} \AgdaBound{Γ}\AgdaSymbol{\}(}\AgdaBound{δ} \AgdaSymbol{:} \AgdaBound{Γ} \AgdaDatatype{⇒} \AgdaBound{Δ}\AgdaSymbol{)} \<[48]%
\>[48]\<%
\\
\>[0]\AgdaIndent{10}{}\<[10]%
\>[10]\AgdaSymbol{→} \AgdaBound{δ} \AgdaFunction{⊚} \AgdaSymbol{(}\AgdaFunction{pr1} \AgdaSymbol{\{}\AgdaBound{Γ}\AgdaSymbol{\}} \AgdaSymbol{\{}\AgdaBound{B}\AgdaSymbol{\})} \AgdaDatatype{≡} \AgdaBound{δ} \AgdaFunction{+S} \AgdaSymbol{\_}\<%
\\
%
\\
\>\AgdaFunction{pr2-v0} \AgdaSymbol{:} \AgdaSymbol{∀} \AgdaSymbol{\{}\AgdaBound{Γ} \AgdaBound{A}\AgdaSymbol{\}} \AgdaSymbol{→} \AgdaFunction{pr2} \AgdaSymbol{\{}\AgdaBound{Γ}\AgdaSymbol{\}} \AgdaSymbol{\{}\AgdaBound{A}\AgdaSymbol{\}} \AgdaDatatype{≅} \AgdaInductiveConstructor{var} \AgdaInductiveConstructor{v0}\<%
\\
%
\\
\>\AgdaFunction{pr-beta} \AgdaSymbol{:} \AgdaSymbol{∀} \AgdaSymbol{\{}\AgdaBound{Γ} \AgdaBound{A}\AgdaSymbol{\}} \AgdaSymbol{→} \AgdaSymbol{(}\AgdaFunction{pr1} \AgdaSymbol{\{}\AgdaBound{Γ}\AgdaSymbol{\}} \AgdaSymbol{\{}\AgdaBound{A}\AgdaSymbol{\}} \AgdaInductiveConstructor{,} \AgdaFunction{pr2}\AgdaSymbol{)} \AgdaDatatype{≡} \AgdaFunction{IdCm}\<%
\\
%
\\
\>\AgdaFunction{pr1} \AgdaSymbol{\{}\AgdaBound{Γ}\AgdaSymbol{\}} \AgdaSymbol{=} \AgdaFunction{IdCm} \AgdaFunction{+S} \AgdaSymbol{\_}\<%
\\
%
\\
\>\AgdaFunction{pr1-wk-T} \AgdaSymbol{=} \AgdaFunction{wk+S+T} \AgdaFunction{IC-T}\<%
\\
%
\\
\>\AgdaFunction{pr1-wk-tm} \AgdaSymbol{\{}a \AgdaSymbol{=} \AgdaBound{a}\AgdaSymbol{\}} \AgdaSymbol{=} \AgdaFunction{wk+S+tm} \AgdaBound{a} \AgdaSymbol{(}\AgdaFunction{IC-tm} \AgdaBound{a}\AgdaSymbol{)}\<%
\\
%
\\
\>\AgdaFunction{pr1-wk-cm} \AgdaBound{δ} \AgdaSymbol{=} \AgdaFunction{wk+S+S} \AgdaSymbol{(}\AgdaFunction{IC-cm} \AgdaSymbol{\_)}\<%
\\
%
\\
\>\AgdaFunction{pr2} \AgdaSymbol{=} \AgdaInductiveConstructor{var} \AgdaInductiveConstructor{v0} \AgdaFunction{⟦} \AgdaFunction{wk+S+T} \AgdaFunction{IC-T} \AgdaFunction{⟫}\<%
\\
%
\\
\>\AgdaFunction{pr2-v0} \AgdaSymbol{\{}A \AgdaSymbol{=} \AgdaBound{A}\AgdaSymbol{\}} \AgdaSymbol{=} \AgdaFunction{cohOp} \AgdaSymbol{(}\AgdaFunction{trans} \AgdaFunction{[+S]T} \AgdaSymbol{(}\AgdaFunction{wk-T} \AgdaFunction{IC-T}\AgdaSymbol{))}\<%
\\
%
\\
\>\AgdaFunction{pr-beta} \AgdaSymbol{=} \AgdaInductiveConstructor{refl}\<%
\\
%
\\
%
\\
\>\AgdaKeyword{data} \AgdaDatatype{IsId} \AgdaSymbol{:} \AgdaSymbol{\{}\AgdaBound{Γ} \AgdaBound{Δ} \AgdaSymbol{:} \AgdaDatatype{Con}\AgdaSymbol{\}(}\AgdaBound{γ} \AgdaSymbol{:} \AgdaBound{Γ} \AgdaDatatype{⇒} \AgdaBound{Δ}\AgdaSymbol{)} \AgdaSymbol{→} \AgdaPrimitiveType{Set} \AgdaKeyword{where}\<%
\\
\>[0]\AgdaIndent{2}{}\<[2]%
\>[2]\AgdaInductiveConstructor{isId-bsc} \AgdaSymbol{:} \AgdaSymbol{\{}\AgdaBound{γ} \AgdaSymbol{:} \AgdaInductiveConstructor{ε} \AgdaDatatype{⇒} \AgdaInductiveConstructor{ε}\AgdaSymbol{\}} \AgdaSymbol{→} \AgdaDatatype{IsId} \AgdaBound{γ}\<%
\\
\>[0]\AgdaIndent{2}{}\<[2]%
\>[2]\AgdaInductiveConstructor{isId-ind} \AgdaSymbol{:} \AgdaSymbol{\{}\AgdaBound{Γ} \AgdaBound{Δ} \AgdaSymbol{:} \AgdaDatatype{Con}\AgdaSymbol{\}\{}\AgdaBound{γ} \AgdaSymbol{:} \AgdaBound{Γ} \AgdaDatatype{⇒} \AgdaBound{Δ}\AgdaSymbol{\}} \AgdaSymbol{→} \AgdaDatatype{IsId} \AgdaBound{γ} \AgdaSymbol{→} \<[47]%
\>[47]\<%
\\
\>[2]\AgdaIndent{13}{}\<[13]%
\>[13]\AgdaSymbol{\{}\AgdaBound{A} \AgdaSymbol{:} \AgdaDatatype{Ty} \AgdaBound{Γ}\AgdaSymbol{\}\{}\AgdaBound{B} \AgdaSymbol{:} \AgdaDatatype{Ty} \AgdaBound{Δ}\AgdaSymbol{\}} \AgdaSymbol{→} \<[36]%
\>[36]\<%
\\
\>[2]\AgdaIndent{13}{}\<[13]%
\>[13]\AgdaSymbol{(}\AgdaBound{eq} \AgdaSymbol{:} \AgdaBound{B} \AgdaFunction{[} \AgdaBound{γ} \AgdaFunction{]T} \AgdaDatatype{≡} \AgdaBound{A}\AgdaSymbol{)} \<[33]%
\>[33]\<%
\\
\>[0]\AgdaIndent{11}{}\<[11]%
\>[11]\AgdaSymbol{→} \AgdaDatatype{IsId} \AgdaSymbol{\{}\AgdaBound{Γ} \AgdaInductiveConstructor{,} \AgdaBound{A}\AgdaSymbol{\}} \AgdaSymbol{\{}\AgdaBound{Δ} \AgdaInductiveConstructor{,} \AgdaBound{B}\AgdaSymbol{\}} \AgdaSymbol{(}\AgdaBound{γ} \AgdaFunction{+S} \AgdaSymbol{\_} \AgdaInductiveConstructor{,} \AgdaInductiveConstructor{var} \AgdaInductiveConstructor{v0} \AgdaFunction{⟦} \AgdaFunction{wk+S+T} \AgdaBound{eq} \AgdaFunction{⟫}\AgdaSymbol{)}\<%
\\
%
\\
%
\\
\>\AgdaFunction{IC-IsId} \AgdaSymbol{:} \AgdaSymbol{\{}\AgdaBound{Γ} \AgdaSymbol{:} \AgdaDatatype{Con}\AgdaSymbol{\}} \AgdaSymbol{→} \AgdaDatatype{IsId} \AgdaSymbol{(}\AgdaFunction{IdCm} \AgdaSymbol{\{}\AgdaBound{Γ}\AgdaSymbol{\})}\<%
\\
\>\AgdaFunction{IC-IsId} \AgdaSymbol{\{}\AgdaInductiveConstructor{ε}\AgdaSymbol{\}} \AgdaSymbol{=} \AgdaInductiveConstructor{isId-bsc}\<%
\\
\>\AgdaFunction{IC-IsId} \AgdaSymbol{\{}\AgdaBound{Γ} \AgdaInductiveConstructor{,} \AgdaBound{A}\AgdaSymbol{\}} \AgdaSymbol{=} \AgdaInductiveConstructor{isId-ind} \AgdaSymbol{(}\AgdaFunction{IC-IsId} \AgdaSymbol{\{}\AgdaBound{Γ}\AgdaSymbol{\})} \AgdaFunction{IC-T}\<%
\\
%
\\
%
\\
\>\AgdaFunction{IC-tm'-v0} \AgdaSymbol{:} \AgdaSymbol{\{}\AgdaBound{Γ} \AgdaBound{Δ} \AgdaSymbol{:} \AgdaDatatype{Con}\AgdaSymbol{\}\{}\AgdaBound{A} \AgdaSymbol{:} \AgdaDatatype{Ty} \AgdaBound{Γ}\AgdaSymbol{\}\{}\AgdaBound{B} \AgdaSymbol{:} \AgdaDatatype{Ty} \AgdaBound{Δ}\AgdaSymbol{\}\{}\AgdaBound{γ} \AgdaSymbol{:} \AgdaSymbol{(}\AgdaBound{Γ} \AgdaInductiveConstructor{,} \AgdaBound{A}\AgdaSymbol{)} \AgdaDatatype{⇒} \AgdaSymbol{(}\AgdaBound{Δ} \AgdaInductiveConstructor{,} \AgdaBound{B}\AgdaSymbol{)\}} \AgdaSymbol{→} \AgdaDatatype{IsId} \AgdaBound{γ} \AgdaSymbol{→} \AgdaInductiveConstructor{var} \AgdaInductiveConstructor{v0} \AgdaFunction{[} \AgdaBound{γ} \AgdaFunction{]tm} \AgdaDatatype{≅} \AgdaInductiveConstructor{var} \AgdaInductiveConstructor{v0}\<%
\\
\>\AgdaFunction{IC-tm'-v0} \AgdaSymbol{(}\AgdaInductiveConstructor{isId-ind} \AgdaBound{isd} \AgdaInductiveConstructor{refl}\AgdaSymbol{)} \AgdaSymbol{=} \AgdaFunction{wk-coh} \AgdaFunction{∾} \AgdaFunction{cohOp} \AgdaSymbol{(}\AgdaFunction{trans} \AgdaFunction{[+S]T} \AgdaInductiveConstructor{refl}\AgdaSymbol{)}\<%
\\
%
\\
%
\\
\>\AgdaFunction{Id-with} \AgdaSymbol{:} \AgdaSymbol{\{}\AgdaBound{Γ} \AgdaSymbol{:} \AgdaDatatype{Con}\AgdaSymbol{\}\{}\AgdaBound{A} \AgdaSymbol{:} \AgdaDatatype{Ty} \AgdaBound{Γ}\AgdaSymbol{\}} \AgdaSymbol{→}\<%
\\
\>[0]\AgdaIndent{11}{}\<[11]%
\>[11]\AgdaSymbol{(}\AgdaBound{x} \AgdaSymbol{:} \AgdaDatatype{Tm} \AgdaBound{A}\AgdaSymbol{)} \<[22]%
\>[22]\<%
\\
\>[0]\AgdaIndent{9}{}\<[9]%
\>[9]\AgdaSymbol{→} \AgdaBound{Γ} \AgdaDatatype{⇒} \AgdaSymbol{(}\AgdaBound{Γ} \AgdaInductiveConstructor{,} \AgdaBound{A}\AgdaSymbol{)}\<%
\\
\>\AgdaFunction{Id-with} \AgdaBound{x} \AgdaSymbol{=} \AgdaFunction{IdCm} \AgdaInductiveConstructor{,} \AgdaSymbol{(}\AgdaBound{x} \AgdaFunction{⟦} \AgdaFunction{IC-T} \AgdaFunction{⟫}\AgdaSymbol{)}\<%
\\
%
\\
%
\\
\>\AgdaFunction{apply-cm''} \AgdaSymbol{:} \AgdaSymbol{\{}\AgdaBound{Γ} \AgdaBound{Δ} \AgdaSymbol{:} \AgdaDatatype{Con}\AgdaSymbol{\}\{}\AgdaBound{A} \AgdaSymbol{:} \AgdaDatatype{Ty} \AgdaBound{Γ}\AgdaSymbol{\}} \AgdaSymbol{→}\<%
\\
\>[0]\AgdaIndent{13}{}\<[13]%
\>[13]\AgdaSymbol{(}\AgdaBound{x} \AgdaSymbol{:} \AgdaDatatype{Tm} \AgdaBound{A}\AgdaSymbol{)(}\AgdaBound{γ} \AgdaSymbol{:} \AgdaBound{Γ} \AgdaDatatype{⇒} \AgdaBound{Δ}\AgdaSymbol{)\{}\AgdaBound{B} \AgdaSymbol{:} \AgdaDatatype{Ty} \AgdaBound{Δ}\AgdaSymbol{\}(}\AgdaBound{p} \AgdaSymbol{:} \AgdaBound{B} \AgdaFunction{[} \AgdaBound{γ} \AgdaFunction{]T} \AgdaDatatype{≡} \AgdaBound{A}\AgdaSymbol{)}\<%
\\
\>[0]\AgdaIndent{10}{}\<[10]%
\>[10]\AgdaSymbol{→} \AgdaBound{Γ} \AgdaDatatype{⇒} \AgdaSymbol{(}\AgdaBound{Δ} \AgdaInductiveConstructor{,} \AgdaBound{B}\AgdaSymbol{)}\<%
\\
\>\AgdaFunction{apply-cm''} \AgdaBound{x} \AgdaBound{γ} \AgdaBound{p} \AgdaSymbol{=} \AgdaBound{γ} \AgdaInductiveConstructor{,} \AgdaSymbol{(}\AgdaBound{x} \AgdaFunction{⟦} \AgdaBound{p} \AgdaFunction{⟫}\AgdaSymbol{)}\<%
\\
%
\\
%
\\
\>\AgdaFunction{apply''} \AgdaSymbol{:} \AgdaSymbol{\{}\AgdaBound{Γ} \AgdaBound{Δ} \AgdaSymbol{:} \AgdaDatatype{Con}\AgdaSymbol{\}\{}\AgdaBound{A} \AgdaSymbol{:} \AgdaDatatype{Ty} \AgdaBound{Γ}\AgdaSymbol{\}}\<%
\\
\>[0]\AgdaIndent{10}{}\<[10]%
\>[10]\AgdaSymbol{(}\AgdaBound{x} \AgdaSymbol{:} \AgdaDatatype{Tm} \AgdaBound{A}\AgdaSymbol{)(}\AgdaBound{γ} \AgdaSymbol{:} \AgdaBound{Γ} \AgdaDatatype{⇒} \AgdaBound{Δ}\AgdaSymbol{)\{}\AgdaBound{B} \AgdaSymbol{:} \AgdaDatatype{Ty} \AgdaBound{Δ}\AgdaSymbol{\}}\<%
\\
\>[0]\AgdaIndent{10}{}\<[10]%
\>[10]\AgdaSymbol{(}\AgdaBound{p} \AgdaSymbol{:} \AgdaBound{B} \AgdaFunction{[} \AgdaBound{γ} \AgdaFunction{]T} \AgdaDatatype{≡} \AgdaBound{A}\AgdaSymbol{)\{}\AgdaBound{C} \AgdaSymbol{:} \AgdaDatatype{Ty} \AgdaSymbol{(}\AgdaBound{Δ} \AgdaInductiveConstructor{,} \AgdaBound{B}\AgdaSymbol{)\}}\<%
\\
\>[0]\AgdaIndent{10}{}\<[10]%
\>[10]\AgdaSymbol{(}\AgdaBound{f} \AgdaSymbol{:} \AgdaDatatype{Tm} \AgdaSymbol{\{}\AgdaBound{Δ} \AgdaInductiveConstructor{,} \AgdaBound{B}\AgdaSymbol{\}} \AgdaBound{C}\AgdaSymbol{)}\<%
\\
\>[0]\AgdaIndent{8}{}\<[8]%
\>[8]\AgdaSymbol{→} \AgdaDatatype{Tm} \AgdaSymbol{(}\AgdaBound{C} \AgdaFunction{[} \AgdaFunction{apply-cm''} \AgdaBound{x} \AgdaBound{γ} \AgdaBound{p} \AgdaFunction{]T}\AgdaSymbol{)}\<%
\\
\>\AgdaFunction{apply''} \AgdaBound{x} \AgdaBound{γ} \AgdaBound{p} \AgdaBound{f} \AgdaSymbol{=} \AgdaBound{f} \AgdaFunction{[} \AgdaFunction{apply-cm''} \AgdaBound{x} \AgdaBound{γ} \AgdaBound{p} \AgdaFunction{]tm}\<%
\\
%
\\
\>\AgdaFunction{apply-x} \AgdaSymbol{:} \AgdaSymbol{\{}\AgdaBound{Γ} \AgdaSymbol{:} \AgdaDatatype{Con}\AgdaSymbol{\}\{}\AgdaBound{A} \AgdaSymbol{:} \AgdaDatatype{Ty} \AgdaBound{Γ}\AgdaSymbol{\}} \AgdaSymbol{→}\<%
\\
\>[0]\AgdaIndent{10}{}\<[10]%
\>[10]\AgdaSymbol{\{}\AgdaBound{x} \AgdaSymbol{:} \AgdaDatatype{Tm} \AgdaBound{A}\AgdaSymbol{\}} \<[21]%
\>[21]\<%
\\
\>[0]\AgdaIndent{8}{}\<[8]%
\>[8]\AgdaSymbol{→} \AgdaInductiveConstructor{var} \AgdaInductiveConstructor{v0} \AgdaFunction{[} \AgdaFunction{Id-with} \AgdaBound{x} \AgdaFunction{]tm} \AgdaDatatype{≅} \AgdaBound{x}\<%
\\
\>\AgdaFunction{apply-x} \AgdaSymbol{=} \AgdaFunction{wk-coh} \AgdaFunction{∾} \AgdaSymbol{(}\AgdaFunction{cohOp} \AgdaFunction{IC-T}\AgdaSymbol{)}\<%
\\
%
\\
\>\AgdaFunction{apply-id} \AgdaSymbol{:} \AgdaSymbol{\{}\AgdaBound{Γ} \AgdaSymbol{:} \AgdaDatatype{Con}\AgdaSymbol{\}\{}\AgdaBound{A} \AgdaBound{B} \AgdaSymbol{:} \AgdaDatatype{Ty} \AgdaBound{Γ}\AgdaSymbol{\}} \AgdaSymbol{→}\<%
\\
\>[0]\AgdaIndent{11}{}\<[11]%
\>[11]\AgdaSymbol{\{}\AgdaBound{x} \AgdaSymbol{:} \AgdaDatatype{Tm} \AgdaBound{A}\AgdaSymbol{\}(}\AgdaBound{y} \AgdaSymbol{:} \AgdaDatatype{Tm} \AgdaBound{B}\AgdaSymbol{)}\<%
\\
\>[0]\AgdaIndent{8}{}\<[8]%
\>[8]\AgdaSymbol{→} \AgdaSymbol{(}\AgdaBound{y} \AgdaFunction{+tm} \AgdaBound{A}\AgdaSymbol{)} \AgdaFunction{[} \AgdaFunction{Id-with} \AgdaBound{x} \AgdaFunction{]tm} \AgdaDatatype{≅} \AgdaBound{y}\<%
\\
\>\AgdaFunction{apply-id} \AgdaBound{y} \AgdaSymbol{=} \AgdaSymbol{(}\AgdaFunction{+tm[,]tm} \AgdaBound{y}\AgdaSymbol{)} \AgdaFunction{∾} \AgdaSymbol{(}\AgdaFunction{IC-tm} \AgdaSymbol{\_)}\<%
\\
%
\\
\>\AgdaFunction{apply-T} \AgdaSymbol{:} \AgdaSymbol{\{}\AgdaBound{Γ} \AgdaSymbol{:} \AgdaDatatype{Con}\AgdaSymbol{\}\{}\AgdaBound{A} \AgdaSymbol{:} \AgdaDatatype{Ty} \AgdaBound{Γ}\AgdaSymbol{\}(}\AgdaBound{B} \AgdaSymbol{:} \AgdaDatatype{Ty} \AgdaSymbol{(}\AgdaBound{Γ} \AgdaInductiveConstructor{,} \AgdaBound{A}\AgdaSymbol{))} \AgdaSymbol{→}\<%
\\
\>[0]\AgdaIndent{10}{}\<[10]%
\>[10]\AgdaSymbol{(}\AgdaBound{x} \AgdaSymbol{:} \AgdaDatatype{Tm} \AgdaBound{A}\AgdaSymbol{)} \<[21]%
\>[21]\<%
\\
\>[0]\AgdaIndent{8}{}\<[8]%
\>[8]\AgdaSymbol{→} \AgdaDatatype{Ty} \AgdaBound{Γ}\<%
\\
\>\AgdaFunction{apply-T} \AgdaBound{B} \AgdaBound{x} \AgdaSymbol{=} \AgdaBound{B} \AgdaFunction{[} \AgdaFunction{Id-with} \AgdaBound{x} \AgdaFunction{]T}\<%
\\
%
\\
%
\\
\>\AgdaFunction{apply} \AgdaSymbol{:} \AgdaSymbol{\{}\AgdaBound{Γ} \AgdaSymbol{:} \AgdaDatatype{Con}\AgdaSymbol{\}\{}\AgdaBound{A} \AgdaSymbol{:} \AgdaDatatype{Ty} \AgdaBound{Γ}\AgdaSymbol{\}\{}\AgdaBound{B} \AgdaSymbol{:} \AgdaDatatype{Ty} \AgdaSymbol{(}\AgdaBound{Γ} \AgdaInductiveConstructor{,} \AgdaBound{A}\AgdaSymbol{)\}} \AgdaSymbol{→}\<%
\\
\>[0]\AgdaIndent{8}{}\<[8]%
\>[8]\AgdaSymbol{(}\AgdaBound{f} \AgdaSymbol{:} \AgdaDatatype{Tm} \AgdaSymbol{\{}\AgdaBound{Γ} \AgdaInductiveConstructor{,} \AgdaBound{A}\AgdaSymbol{\}} \AgdaBound{B}\AgdaSymbol{)} \AgdaSymbol{→} \<[29]%
\>[29]\<%
\\
\>[0]\AgdaIndent{8}{}\<[8]%
\>[8]\AgdaSymbol{(}\AgdaBound{x} \AgdaSymbol{:} \AgdaDatatype{Tm} \AgdaBound{A}\AgdaSymbol{)} \<[19]%
\>[19]\<%
\\
\>[0]\AgdaIndent{6}{}\<[6]%
\>[6]\AgdaSymbol{→} \AgdaDatatype{Tm} \AgdaSymbol{(}\AgdaFunction{apply-T} \AgdaBound{B} \AgdaBound{x}\AgdaSymbol{)}\<%
\\
\>\AgdaFunction{apply} \AgdaBound{t} \AgdaBound{x} \AgdaSymbol{=} \AgdaBound{t} \AgdaFunction{[} \AgdaFunction{Id-with} \AgdaBound{x} \AgdaFunction{]tm}\<%
\\
%
\\
\>\AgdaFunction{apply-2} \AgdaSymbol{:} \AgdaSymbol{\{}\AgdaBound{Γ} \AgdaSymbol{:} \AgdaDatatype{Con}\AgdaSymbol{\}}\<%
\\
\>[0]\AgdaIndent{10}{}\<[10]%
\>[10]\AgdaSymbol{\{}\AgdaBound{A} \AgdaSymbol{:} \AgdaDatatype{Ty} \AgdaBound{Γ}\AgdaSymbol{\}}\<%
\\
\>[0]\AgdaIndent{10}{}\<[10]%
\>[10]\AgdaSymbol{\{}\AgdaBound{B} \AgdaSymbol{:} \AgdaDatatype{Ty} \AgdaSymbol{(}\AgdaBound{Γ} \AgdaInductiveConstructor{,} \AgdaBound{A}\AgdaSymbol{)\}}\<%
\\
\>[0]\AgdaIndent{10}{}\<[10]%
\>[10]\AgdaSymbol{\{}\AgdaBound{C} \AgdaSymbol{:} \AgdaDatatype{Ty} \AgdaSymbol{(}\AgdaBound{Γ} \AgdaInductiveConstructor{,} \AgdaBound{A} \AgdaInductiveConstructor{,} \AgdaBound{B}\AgdaSymbol{)\}}\<%
\\
\>[0]\AgdaIndent{10}{}\<[10]%
\>[10]\AgdaSymbol{(}\AgdaBound{f} \AgdaSymbol{:} \AgdaDatatype{Tm} \AgdaSymbol{\{}\AgdaBound{Γ} \AgdaInductiveConstructor{,} \AgdaBound{A} \AgdaInductiveConstructor{,} \AgdaBound{B}\AgdaSymbol{\}} \AgdaBound{C}\AgdaSymbol{)}\<%
\\
\>[0]\AgdaIndent{10}{}\<[10]%
\>[10]\AgdaSymbol{(}\AgdaBound{x} \AgdaSymbol{:} \AgdaDatatype{Tm} \AgdaBound{A}\AgdaSymbol{)(}\AgdaBound{y} \AgdaSymbol{:} \AgdaDatatype{Tm} \AgdaBound{B}\AgdaSymbol{)}\<%
\\
\>[0]\AgdaIndent{8}{}\<[8]%
\>[8]\AgdaSymbol{→} \AgdaDatatype{Tm} \AgdaSymbol{(}\AgdaFunction{apply-T} \AgdaSymbol{(}\AgdaFunction{apply-T} \AgdaBound{C} \AgdaBound{y}\AgdaSymbol{)} \AgdaBound{x}\AgdaSymbol{)}\<%
\\
\>\AgdaFunction{apply-2} \AgdaBound{f} \AgdaBound{x} \AgdaBound{y} \AgdaSymbol{=} \AgdaSymbol{(}\AgdaBound{f} \AgdaFunction{[} \<[22]%
\>[22]\AgdaFunction{Id-with} \AgdaBound{y} \AgdaFunction{]tm}\AgdaSymbol{)} \AgdaFunction{[} \<[40]%
\>[40]\AgdaFunction{Id-with} \AgdaBound{x} \AgdaFunction{]tm}\<%
\\
%
\\
\>\AgdaFunction{apply-3} \AgdaSymbol{:} \AgdaSymbol{\{}\AgdaBound{Γ} \AgdaSymbol{:} \AgdaDatatype{Con}\AgdaSymbol{\}}\<%
\\
\>[0]\AgdaIndent{10}{}\<[10]%
\>[10]\AgdaSymbol{\{}\AgdaBound{A} \AgdaSymbol{:} \AgdaDatatype{Ty} \AgdaBound{Γ}\AgdaSymbol{\}}\<%
\\
\>[0]\AgdaIndent{10}{}\<[10]%
\>[10]\AgdaSymbol{\{}\AgdaBound{B} \AgdaSymbol{:} \AgdaDatatype{Ty} \AgdaSymbol{(}\AgdaBound{Γ} \AgdaInductiveConstructor{,} \AgdaBound{A}\AgdaSymbol{)\}}\<%
\\
\>[0]\AgdaIndent{10}{}\<[10]%
\>[10]\AgdaSymbol{\{}\AgdaBound{C} \AgdaSymbol{:} \AgdaDatatype{Ty} \AgdaSymbol{(}\AgdaBound{Γ} \AgdaInductiveConstructor{,} \AgdaBound{A} \AgdaInductiveConstructor{,} \AgdaBound{B}\AgdaSymbol{)\}}\<%
\\
\>[0]\AgdaIndent{10}{}\<[10]%
\>[10]\AgdaSymbol{\{}\AgdaBound{D} \AgdaSymbol{:} \AgdaDatatype{Ty} \AgdaSymbol{(}\AgdaBound{Γ} \AgdaInductiveConstructor{,} \AgdaBound{A} \AgdaInductiveConstructor{,} \AgdaBound{B} \AgdaInductiveConstructor{,} \AgdaBound{C}\AgdaSymbol{)\}}\<%
\\
\>[0]\AgdaIndent{10}{}\<[10]%
\>[10]\AgdaSymbol{(}\AgdaBound{f} \AgdaSymbol{:} \AgdaDatatype{Tm} \AgdaSymbol{\{}\AgdaBound{Γ} \AgdaInductiveConstructor{,} \AgdaBound{A} \AgdaInductiveConstructor{,} \AgdaBound{B} \AgdaInductiveConstructor{,} \AgdaBound{C}\AgdaSymbol{\}} \AgdaBound{D}\AgdaSymbol{)}\<%
\\
\>[0]\AgdaIndent{10}{}\<[10]%
\>[10]\AgdaSymbol{(}\AgdaBound{x} \AgdaSymbol{:} \AgdaDatatype{Tm} \AgdaBound{A}\AgdaSymbol{)(}\AgdaBound{y} \AgdaSymbol{:} \AgdaDatatype{Tm} \AgdaBound{B}\AgdaSymbol{)(}\AgdaBound{z} \AgdaSymbol{:} \AgdaDatatype{Tm} \AgdaBound{C}\AgdaSymbol{)}\<%
\\
\>[0]\AgdaIndent{9}{}\<[9]%
\>[9]\AgdaSymbol{→} \<[12]%
\>[12]\AgdaDatatype{Tm} \AgdaSymbol{(}\AgdaFunction{apply-T} \AgdaSymbol{(}\AgdaFunction{apply-T} \AgdaSymbol{(}\AgdaFunction{apply-T} \AgdaBound{D} \AgdaBound{z}\AgdaSymbol{)} \AgdaBound{y}\AgdaSymbol{)} \AgdaBound{x}\AgdaSymbol{)}\<%
\\
\>\AgdaFunction{apply-3} \AgdaBound{f} \AgdaBound{x} \AgdaBound{y} \AgdaBound{z} \AgdaSymbol{=} \AgdaSymbol{((}\AgdaBound{f} \AgdaFunction{[} \<[25]%
\>[25]\AgdaFunction{Id-with} \AgdaBound{z} \AgdaFunction{]tm}\AgdaSymbol{)} \AgdaFunction{[} \<[43]%
\>[43]\AgdaFunction{Id-with} \AgdaBound{y} \AgdaFunction{]tm}\AgdaSymbol{)} \AgdaFunction{[} \AgdaFunction{Id-with} \AgdaBound{x} \AgdaFunction{]tm}\<%
\\
%
\\
\>\AgdaFunction{fun} \AgdaSymbol{:} \AgdaSymbol{\{}\AgdaBound{Γ} \AgdaSymbol{:} \AgdaDatatype{Con}\AgdaSymbol{\}\{}\AgdaBound{A} \AgdaBound{B} \AgdaSymbol{:} \AgdaDatatype{Ty} \AgdaBound{Γ}\AgdaSymbol{\}} \AgdaSymbol{→} \<[30]%
\>[30]\<%
\\
\>[0]\AgdaIndent{6}{}\<[6]%
\>[6]\AgdaDatatype{Tm} \AgdaSymbol{(}\AgdaBound{B} \AgdaFunction{+T} \AgdaBound{A}\AgdaSymbol{)}\<%
\\
\>[0]\AgdaIndent{4}{}\<[4]%
\>[4]\AgdaSymbol{→} \AgdaSymbol{(}\AgdaDatatype{Tm} \AgdaSymbol{\{}\AgdaBound{Γ}\AgdaSymbol{\}} \AgdaBound{A} \AgdaSymbol{→} \AgdaDatatype{Tm} \AgdaSymbol{\{}\AgdaBound{Γ}\AgdaSymbol{\}} \AgdaBound{B}\AgdaSymbol{)} \<[28]%
\>[28]\<%
\\
\>\AgdaFunction{fun} \AgdaBound{t} \AgdaBound{a} \AgdaSymbol{=} \AgdaSymbol{(}\AgdaBound{t} \AgdaFunction{[} \AgdaFunction{Id-with} \AgdaBound{a} \AgdaFunction{]tm}\AgdaSymbol{)} \AgdaFunction{⟦} \AgdaFunction{sym} \AgdaSymbol{(}\AgdaFunction{trans} \AgdaFunction{+T[,]T} \AgdaFunction{IC-T}\AgdaSymbol{)} \AgdaFunction{⟫}\<%
\\
%
\\
%
\\
\>\<\end{code}

}
	
\AgdaHide{
\begin{code}\>\<%
\\
\>\AgdaSymbol{\{-\#} \AgdaKeyword{OPTIONS} --type-in-type --no-positivity-check --no-termination-check \AgdaSymbol{\#-\}}\<%
\\
%
\\
\>\AgdaKeyword{module} \AgdaModule{Suspension} \AgdaKeyword{where}\<%
\\
%
\\
\>\AgdaKeyword{open} \AgdaKeyword{import} \AgdaModule{BasicSyntax}\<%
\\
\>\AgdaKeyword{open} \AgdaKeyword{import} \AgdaModule{IdentityContextMorphisms}\<%
\\
\>\AgdaKeyword{open} \AgdaKeyword{import} \AgdaModule{Relation.Binary.PropositionalEquality} \<[50]%
\>[50]\<%
\\
\>\AgdaKeyword{open} \AgdaKeyword{import} \AgdaModule{Data.Product} \AgdaKeyword{renaming} \AgdaSymbol{(}\_,\_ \AgdaSymbol{to} \_,,\_\AgdaSymbol{)}\<%
\\
\>\AgdaKeyword{open} \AgdaKeyword{import} \AgdaModule{Data.Empty}\<%
\\
\>\AgdaKeyword{open} \AgdaKeyword{import} \AgdaModule{Data.Nat}\<%
\\
%
\\
%
\\
\>\AgdaFunction{1-1cm-same} \AgdaSymbol{:} \AgdaSymbol{\{}\AgdaBound{Γ} \AgdaSymbol{:} \AgdaDatatype{Con}\AgdaSymbol{\}\{}\AgdaBound{A} \AgdaBound{B} \AgdaSymbol{:} \AgdaDatatype{Ty} \AgdaBound{Γ}\AgdaSymbol{\}} \AgdaSymbol{→} \<[37]%
\>[37]\<%
\\
\>[0]\AgdaIndent{12}{}\<[12]%
\>[12]\AgdaBound{B} \AgdaDatatype{≡} \AgdaBound{A} \AgdaSymbol{→} \AgdaSymbol{(}\AgdaBound{Γ} \AgdaInductiveConstructor{,} \AgdaBound{A}\AgdaSymbol{)} \AgdaDatatype{⇒} \AgdaSymbol{(}\AgdaBound{Γ} \AgdaInductiveConstructor{,} \AgdaBound{B}\AgdaSymbol{)}\<%
\\
\>\AgdaFunction{1-1cm-same} \AgdaBound{eq} \AgdaSymbol{=} \AgdaFunction{pr1} \AgdaInductiveConstructor{,} \AgdaFunction{pr2} \AgdaFunction{⟦} \AgdaFunction{congT} \AgdaBound{eq} \AgdaFunction{⟫} \<[39]%
\>[39]\<%
\\
%
\\
%
\\
\>\AgdaFunction{1-1cm-same-T} \AgdaSymbol{:} \AgdaSymbol{\{}\AgdaBound{Γ} \AgdaSymbol{:} \AgdaDatatype{Con}\AgdaSymbol{\}\{}\AgdaBound{A} \AgdaBound{B} \AgdaSymbol{:} \AgdaDatatype{Ty} \AgdaBound{Γ}\AgdaSymbol{\}} \AgdaSymbol{→} \<[39]%
\>[39]\<%
\\
\>[12]\AgdaIndent{15}{}\<[15]%
\>[15]\AgdaSymbol{(}\AgdaBound{eq} \AgdaSymbol{:} \AgdaBound{B} \AgdaDatatype{≡} \AgdaBound{A}\AgdaSymbol{)} \AgdaSymbol{→} \AgdaSymbol{(}\AgdaBound{A} \AgdaFunction{+T} \AgdaBound{B}\AgdaSymbol{)} \AgdaFunction{[} \AgdaFunction{1-1cm-same} \AgdaBound{eq} \AgdaFunction{]T} \AgdaDatatype{≡} \AgdaBound{A} \AgdaFunction{+T} \AgdaBound{A}\<%
\\
\>\AgdaFunction{1-1cm-same-T} \AgdaBound{eq} \AgdaSymbol{=} \AgdaFunction{trans} \AgdaFunction{+T[,]T} \AgdaSymbol{(}\AgdaFunction{trans} \AgdaFunction{[+S]T} \AgdaSymbol{(}\AgdaFunction{wk-T} \AgdaFunction{IC-T}\AgdaSymbol{))}\<%
\\
%
\\
%
\\
\>\AgdaFunction{1-1cm-same-tm} \AgdaSymbol{:} \AgdaSymbol{∀} \AgdaSymbol{\{}\AgdaBound{Γ} \AgdaSymbol{:} \AgdaDatatype{Con}\AgdaSymbol{\}\{}\AgdaBound{A} \AgdaSymbol{:} \AgdaDatatype{Ty} \AgdaBound{Γ}\AgdaSymbol{\}\{}\AgdaBound{B} \AgdaSymbol{:} \AgdaDatatype{Ty} \AgdaBound{Γ}\AgdaSymbol{\}} \AgdaSymbol{→} \<[50]%
\>[50]\<%
\\
\>[12]\AgdaIndent{15}{}\<[15]%
\>[15]\AgdaSymbol{(}\AgdaBound{eq} \AgdaSymbol{:} \AgdaBound{B} \AgdaDatatype{≡} \AgdaBound{A}\AgdaSymbol{)(}\AgdaBound{a} \AgdaSymbol{:} \AgdaDatatype{Tm} \AgdaBound{A}\AgdaSymbol{)} \AgdaSymbol{→} \AgdaSymbol{(}\AgdaBound{a} \AgdaFunction{+tm} \AgdaBound{B}\AgdaSymbol{)} \AgdaFunction{[} \AgdaFunction{1-1cm-same} \AgdaBound{eq} \AgdaFunction{]tm} \AgdaDatatype{≅} \AgdaSymbol{(}\AgdaBound{a} \AgdaFunction{+tm} \AgdaBound{A}\AgdaSymbol{)}\<%
\\
\>\AgdaFunction{1-1cm-same-tm} \AgdaBound{eq} \AgdaBound{a} \AgdaSymbol{=} \AgdaFunction{+tm[,]tm} \AgdaBound{a} \AgdaFunction{∾} \AgdaFunction{[+S]tm} \AgdaBound{a} \AgdaFunction{∾} \AgdaFunction{cong+tm} \AgdaSymbol{(}\AgdaFunction{IC-tm} \AgdaBound{a}\AgdaSymbol{)}\<%
\\
%
\\
\>\AgdaFunction{1-1cm-same-v0} \AgdaSymbol{:} \AgdaSymbol{∀} \AgdaSymbol{\{}\AgdaBound{Γ} \AgdaSymbol{:} \AgdaDatatype{Con}\AgdaSymbol{\}\{}\AgdaBound{A} \AgdaBound{B} \AgdaSymbol{:} \AgdaDatatype{Ty} \AgdaBound{Γ}\AgdaSymbol{\}} \AgdaSymbol{→} \<[42]%
\>[42]\<%
\\
\>[12]\AgdaIndent{15}{}\<[15]%
\>[15]\AgdaSymbol{(}\AgdaBound{eq} \AgdaSymbol{:} \AgdaBound{B} \AgdaDatatype{≡} \AgdaBound{A}\AgdaSymbol{)} \AgdaSymbol{→} \AgdaInductiveConstructor{var} \AgdaInductiveConstructor{v0} \AgdaFunction{[} \AgdaFunction{1-1cm-same} \AgdaBound{eq} \AgdaFunction{]tm} \AgdaDatatype{≅} \AgdaInductiveConstructor{var} \AgdaInductiveConstructor{v0}\<%
\\
\>\AgdaFunction{1-1cm-same-v0} \AgdaBound{eq} \AgdaSymbol{=} \AgdaFunction{wk-coh} \AgdaFunction{∾} \AgdaFunction{cohOp} \AgdaSymbol{(}\AgdaFunction{congT} \AgdaBound{eq}\AgdaSymbol{)} \AgdaFunction{∾} \AgdaFunction{pr2-v0}\<%
\\
%
\\
%
\\
\>\AgdaFunction{\_++\_} \AgdaSymbol{:} \AgdaDatatype{Con} \AgdaSymbol{→} \AgdaDatatype{Con} \AgdaSymbol{→} \AgdaDatatype{Con}\<%
\\
%
\\
\>\AgdaFunction{cor} \AgdaSymbol{:} \AgdaSymbol{\{}\AgdaBound{Γ} \AgdaSymbol{:} \AgdaDatatype{Con}\AgdaSymbol{\}(}\AgdaBound{Δ} \AgdaSymbol{:} \AgdaDatatype{Con}\AgdaSymbol{)} \AgdaSymbol{→} \AgdaSymbol{(}\AgdaBound{Γ} \AgdaFunction{++} \AgdaBound{Δ}\AgdaSymbol{)} \AgdaDatatype{⇒} \AgdaBound{Δ}\<%
\\
%
\\
\>\AgdaFunction{repeat-p1} \AgdaSymbol{:} \AgdaSymbol{\{}\AgdaBound{Γ} \AgdaSymbol{:} \AgdaDatatype{Con}\AgdaSymbol{\}(}\AgdaBound{Δ} \AgdaSymbol{:} \AgdaDatatype{Con}\AgdaSymbol{)} \AgdaSymbol{→} \AgdaSymbol{(}\AgdaBound{Γ} \AgdaFunction{++} \AgdaBound{Δ}\AgdaSymbol{)} \AgdaDatatype{⇒} \AgdaBound{Γ}\<%
\\
%
\\
\>\AgdaBound{Γ} \AgdaFunction{++} \AgdaInductiveConstructor{ε} \AgdaSymbol{=} \AgdaBound{Γ}\<%
\\
\>\AgdaBound{Γ} \AgdaFunction{++} \AgdaSymbol{(}\AgdaBound{Δ} \AgdaInductiveConstructor{,} \AgdaBound{A}\AgdaSymbol{)} \AgdaSymbol{=} \AgdaBound{Γ} \AgdaFunction{++} \AgdaBound{Δ} \AgdaInductiveConstructor{,} \AgdaBound{A} \AgdaFunction{[} \AgdaFunction{cor} \AgdaBound{Δ} \AgdaFunction{]T}\<%
\\
%
\\
%
\\
\>\AgdaFunction{repeat-p1} \AgdaInductiveConstructor{ε} \AgdaSymbol{=} \AgdaFunction{IdCm}\<%
\\
\>\AgdaFunction{repeat-p1} \AgdaSymbol{(}\AgdaBound{Δ} \AgdaInductiveConstructor{,} \AgdaBound{A}\AgdaSymbol{)} \AgdaSymbol{=} \AgdaFunction{repeat-p1} \AgdaBound{Δ} \AgdaFunction{⊚} \AgdaFunction{pr1}\<%
\\
%
\\
%
\\
\>\AgdaFunction{cor} \AgdaInductiveConstructor{ε} \AgdaSymbol{=} \AgdaInductiveConstructor{•}\<%
\\
\>\AgdaFunction{cor} \AgdaSymbol{(}\AgdaBound{Δ} \AgdaInductiveConstructor{,} \AgdaBound{A}\AgdaSymbol{)} \AgdaSymbol{=} \AgdaSymbol{(}\AgdaFunction{cor} \AgdaBound{Δ} \AgdaFunction{+S} \AgdaSymbol{\_)} \AgdaInductiveConstructor{,} \AgdaInductiveConstructor{var} \AgdaInductiveConstructor{v0} \AgdaFunction{⟦} \AgdaFunction{[+S]T} \AgdaFunction{⟫}\<%
\\
%
\\
%
\\
%
\\
\>\AgdaFunction{\_++cm\_} \AgdaSymbol{:} \AgdaSymbol{∀} \AgdaSymbol{\{}\AgdaBound{Γ} \AgdaBound{Δ} \AgdaBound{Θ}\AgdaSymbol{\}} \AgdaSymbol{→} \AgdaBound{Γ} \AgdaDatatype{⇒} \AgdaBound{Δ} \AgdaSymbol{→} \AgdaBound{Γ} \AgdaDatatype{⇒} \AgdaBound{Θ} \AgdaSymbol{→} \AgdaBound{Γ} \AgdaDatatype{⇒} \AgdaSymbol{(}\AgdaBound{Δ} \AgdaFunction{++} \AgdaBound{Θ}\AgdaSymbol{)}\<%
\\
\>\AgdaFunction{cor-inv} \AgdaSymbol{:} \AgdaSymbol{∀} \AgdaSymbol{\{}\AgdaBound{Γ} \AgdaBound{Δ} \AgdaBound{Θ}\AgdaSymbol{\}} \AgdaSymbol{→} \AgdaSymbol{\{}\AgdaBound{γ} \AgdaSymbol{:} \AgdaBound{Γ} \AgdaDatatype{⇒} \AgdaBound{Δ}\AgdaSymbol{\}(}\AgdaBound{δ} \AgdaSymbol{:} \AgdaBound{Γ} \AgdaDatatype{⇒} \AgdaBound{Θ}\AgdaSymbol{)} \AgdaSymbol{→} \AgdaFunction{cor} \AgdaBound{Θ} \AgdaFunction{⊚} \AgdaSymbol{(}\AgdaBound{γ} \AgdaFunction{++cm} \AgdaBound{δ}\AgdaSymbol{)} \AgdaDatatype{≡} \AgdaBound{δ}\<%
\\
%
\\
\>\AgdaBound{γ} \AgdaFunction{++cm} \AgdaInductiveConstructor{•} \AgdaSymbol{=} \AgdaBound{γ}\<%
\\
\>\AgdaBound{γ} \AgdaFunction{++cm} \AgdaSymbol{(}\AgdaBound{δ} \AgdaInductiveConstructor{,} \AgdaBound{a}\AgdaSymbol{)} \AgdaSymbol{=} \AgdaBound{γ} \AgdaFunction{++cm} \AgdaBound{δ} \AgdaInductiveConstructor{,} \AgdaBound{a} \AgdaFunction{⟦} \AgdaFunction{trans} \AgdaSymbol{(}\AgdaFunction{sym} \AgdaFunction{[⊚]T}\AgdaSymbol{)} \AgdaSymbol{(}\AgdaFunction{congT2} \AgdaSymbol{(}\AgdaFunction{cor-inv} \AgdaSymbol{\_))} \AgdaFunction{⟫} \<[72]%
\>[72]\<%
\\
%
\\
\>\AgdaFunction{cor-inv} \AgdaInductiveConstructor{•} \AgdaSymbol{=} \AgdaInductiveConstructor{refl}\<%
\\
\>\AgdaFunction{cor-inv} \AgdaSymbol{(}\AgdaBound{δ} \AgdaInductiveConstructor{,} \AgdaBound{a}\AgdaSymbol{)} \AgdaSymbol{=} \AgdaFunction{cm-eq} \AgdaSymbol{(}\AgdaFunction{trans} \AgdaSymbol{(}\AgdaFunction{⊚wk} \AgdaSymbol{\_)} \AgdaSymbol{(}\AgdaFunction{cor-inv} \AgdaBound{δ}\AgdaSymbol{))} \<[52]%
\>[52]\<%
\\
\>[-5]\AgdaIndent{8}{}\<[8]%
\>[8]\AgdaSymbol{(}\AgdaFunction{cohOp} \AgdaFunction{[⊚]T} \AgdaFunction{∾} \AgdaFunction{congtm} \AgdaSymbol{(}\AgdaFunction{cohOp} \AgdaFunction{[+S]T}\AgdaSymbol{)} \<[43]%
\>[43]\<%
\\
\>[0]\AgdaIndent{8}{}\<[8]%
\>[8]\AgdaFunction{∾} \AgdaFunction{cohOp} \AgdaFunction{+T[,]T} \<[23]%
\>[23]\<%
\\
\>[0]\AgdaIndent{8}{}\<[8]%
\>[8]\AgdaFunction{∾} \AgdaFunction{cohOp} \AgdaSymbol{(}\AgdaFunction{trans} \AgdaSymbol{(}\AgdaFunction{sym} \AgdaFunction{[⊚]T}\AgdaSymbol{)} \AgdaSymbol{(}\AgdaFunction{congT2} \AgdaSymbol{(}\AgdaFunction{cor-inv} \AgdaSymbol{\_))))}\<%
\\
%
\\
%
\\
\>\AgdaFunction{id-cm++} \AgdaSymbol{:} \AgdaSymbol{\{}\AgdaBound{Γ} \AgdaSymbol{:} \AgdaDatatype{Con}\AgdaSymbol{\}(}\AgdaBound{Δ} \AgdaBound{Θ} \AgdaSymbol{:} \AgdaDatatype{Con}\AgdaSymbol{)} \AgdaSymbol{→} \AgdaSymbol{(}\AgdaBound{Δ} \AgdaDatatype{⇒} \AgdaBound{Θ}\AgdaSymbol{)} \AgdaSymbol{→} \AgdaSymbol{(}\AgdaBound{Γ} \AgdaFunction{++} \AgdaBound{Δ}\AgdaSymbol{)} \AgdaDatatype{⇒} \AgdaSymbol{(}\AgdaBound{Γ} \AgdaFunction{++} \AgdaBound{Θ}\AgdaSymbol{)}\<%
\\
\>\AgdaFunction{id-cm++} \AgdaBound{Δ} \AgdaBound{Θ} \AgdaBound{γ} \AgdaSymbol{=} \AgdaFunction{repeat-p1} \AgdaBound{Δ} \AgdaFunction{++cm} \AgdaSymbol{(}\AgdaBound{γ} \AgdaFunction{⊚} \AgdaFunction{cor} \AgdaSymbol{\_)}\<%
\\
%
\\
\>\<\end{code}
}


\subsection{Suspension and Replacement}
\label{sec:susp-and-repl}
%
For an arbitrary type $A$ in $\Gamma$ of level $n$ one can
define a context with $2n$
variables, called the \emph{stalk} of $A$. Moreover one can
define a morphism from $\Gamma$ to the stalk of $A$ such that its
substitution into the maximal type in the stalk of $A$ gives back
$A$. The stalk of $A$ depends only on the level of $A$, the terms in
$A$ define the substitution. Here is an example of stalks of small
levels: $\varepsilon$ (the empty context) for $n=0$; $(x_0 : *, x_1 : *)$ for
$n=1$; $(x_0 : *, x_1 : *, x_2 : x_0\,=_\mathsf{h}\,x_1, x_3 :
x_0\,=_\mathsf{h}\,x_1)$ for $n=2$, etc. 
% \[
% \begin{array}{c@{\hspace{1.5cm}} c@{\hspace{1.5cm}} c@{\hspace{1.5cm}} c@{\hspace{1.5cm}} c@{\hspace{1.5cm}}}
% &&&&6\quad 7\\
% &&&4\quad 5&4 \quad 5\\
% &&2\quad 3&2\quad 3&2\quad 3\\
% &0\quad 1&0\quad 1&0\quad 1&0\quad 1\\
% \\
% n = 0 & n = 1 & n = 2 & n = 3 & n = 4 
% \end{array}
% \]

This is the $\Delta = \varepsilon$ case of a more general construction
where in we \emph{suspend} an arbitrary context $\Delta$ by adding $2n$
variables to the beginning of it, and weakening the rest of the
variables appropriately so that type $*$ becomes $x_{2n-2} =_\mathsf{h}
x_{2n-1}$. A crucial property of suspension is that it preserves
contractibility. 


\subsubsection{Suspension}
\label{sec:susp}

\emph{Suspension} is defined by iteration level-$A$-times the following
operation of one-level suspension. \AgdaFunction{ΣC} takes a
context and gives a context with two new variables of type $*$ added
at the beginning, and with all remaining types in the context suspended
by one level. 

\begin{code}\>\<%
\\
\>\AgdaFunction{ΣC} \AgdaSymbol{:} \AgdaDatatype{Con} \AgdaSymbol{→} \AgdaDatatype{Con}\<%
\\
\>\AgdaFunction{ΣT} \AgdaSymbol{:} \AgdaSymbol{∀\{}\AgdaBound{Γ}\AgdaSymbol{\}} \AgdaSymbol{→} \AgdaDatatype{Ty} \AgdaBound{Γ} \AgdaSymbol{→} \AgdaDatatype{Ty} \AgdaSymbol{(}\AgdaFunction{ΣC} \AgdaBound{Γ}\AgdaSymbol{)}\<%
\\
%
\\
\>\AgdaFunction{ΣC} \AgdaInductiveConstructor{ε} \<[12]%
\>[12]\AgdaSymbol{=} \AgdaInductiveConstructor{ε} \AgdaInductiveConstructor{,} \AgdaInductiveConstructor{*} \AgdaInductiveConstructor{,} \AgdaInductiveConstructor{*}\<%
\\
\>\AgdaFunction{ΣC} \AgdaSymbol{(}\AgdaBound{Γ} \AgdaInductiveConstructor{,} \AgdaBound{A}\AgdaSymbol{)} \<[12]%
\>[12]\AgdaSymbol{=} \AgdaFunction{ΣC} \AgdaBound{Γ} \AgdaInductiveConstructor{,} \AgdaFunction{ΣT} \AgdaBound{A}\<%
\\
\>\<\end{code}
\noindent The rest of the definitions is straightforward by structural
recursion. In particular we suspend variables, terms and context morphisms:

\begin{code}\>\<%
\\
\>\AgdaFunction{Σv} \<[5]%
\>[5]\AgdaSymbol{:} \AgdaSymbol{∀\{}\AgdaBound{Γ}\AgdaSymbol{\}\{}\AgdaBound{A} \AgdaSymbol{:} \AgdaDatatype{Ty} \AgdaBound{Γ}\AgdaSymbol{\}} \AgdaSymbol{→} \AgdaDatatype{Var} \AgdaBound{A} \AgdaSymbol{→} \AgdaDatatype{Var} \AgdaSymbol{(}\AgdaFunction{ΣT} \AgdaBound{A}\AgdaSymbol{)}\<%
\\
\>\AgdaFunction{Σtm} \<[5]%
\>[5]\AgdaSymbol{:} \AgdaSymbol{∀\{}\AgdaBound{Γ}\AgdaSymbol{\}\{}\AgdaBound{A} \AgdaSymbol{:} \AgdaDatatype{Ty} \AgdaBound{Γ}\AgdaSymbol{\}} \AgdaSymbol{→} \AgdaDatatype{Tm} \AgdaBound{A} \AgdaSymbol{→} \AgdaDatatype{Tm} \AgdaSymbol{(}\AgdaFunction{ΣT} \AgdaBound{A}\AgdaSymbol{)}\<%
\\
\>\AgdaFunction{Σs} \<[5]%
\>[5]\AgdaSymbol{:} \AgdaSymbol{∀\{}\AgdaBound{Γ} \AgdaBound{Δ}\AgdaSymbol{\}} \AgdaSymbol{→} \AgdaBound{Γ} \AgdaDatatype{⇒} \AgdaBound{Δ} \AgdaSymbol{→} \AgdaFunction{ΣC} \AgdaBound{Γ} \AgdaDatatype{⇒} \AgdaFunction{ΣC} \AgdaBound{Δ}\<%
\\
\>\<\end{code}
\AgdaHide{
\begin{code}\>\<%
\\
\>\AgdaFunction{*'} \AgdaSymbol{:} \AgdaSymbol{\{}\AgdaBound{Γ} \AgdaSymbol{:} \AgdaDatatype{Con}\AgdaSymbol{\}} \AgdaSymbol{→} \AgdaDatatype{Ty} \AgdaSymbol{(}\AgdaFunction{ΣC} \AgdaBound{Γ}\AgdaSymbol{)}\<%
\\
\>\AgdaFunction{*'} \AgdaSymbol{\{}\AgdaInductiveConstructor{ε}\AgdaSymbol{\}} \AgdaSymbol{=} \AgdaInductiveConstructor{var} \AgdaSymbol{(}\AgdaInductiveConstructor{vS} \AgdaInductiveConstructor{v0}\AgdaSymbol{)} \AgdaInductiveConstructor{=h} \AgdaInductiveConstructor{var} \AgdaInductiveConstructor{v0}\<%
\\
\>\AgdaFunction{*'} \AgdaSymbol{\{}\AgdaBound{Γ} \AgdaInductiveConstructor{,} \AgdaBound{A}\AgdaSymbol{\}} \AgdaSymbol{=} \AgdaFunction{*'} \AgdaSymbol{\{}\AgdaBound{Γ}\AgdaSymbol{\}} \AgdaFunction{+T} \AgdaSymbol{\_}\<%
\\
%
\\
\>\AgdaFunction{ΣT} \AgdaSymbol{\{}\AgdaBound{Γ}\AgdaSymbol{\}} \AgdaInductiveConstructor{*} \AgdaSymbol{=} \AgdaFunction{*'} \AgdaSymbol{\{}\AgdaBound{Γ}\AgdaSymbol{\}}\<%
\\
\>\AgdaFunction{ΣT} \AgdaSymbol{(}\AgdaBound{a} \AgdaInductiveConstructor{=h} \AgdaBound{b}\AgdaSymbol{)} \AgdaSymbol{=} \AgdaFunction{Σtm} \AgdaBound{a} \AgdaInductiveConstructor{=h} \AgdaFunction{Σtm} \AgdaBound{b}\<%
\\
%
\\
\>\AgdaFunction{Σs•} \AgdaSymbol{:} \AgdaSymbol{(}\AgdaBound{Γ} \AgdaSymbol{:} \AgdaDatatype{Con}\AgdaSymbol{)} \AgdaSymbol{→} \AgdaFunction{ΣC} \AgdaBound{Γ} \AgdaDatatype{⇒} \AgdaFunction{ΣC} \AgdaInductiveConstructor{ε}\<%
\\
\>\AgdaFunction{Σs•} \AgdaInductiveConstructor{ε} \AgdaSymbol{=} \AgdaFunction{IdCm}\<%
\\
\>\AgdaFunction{Σs•} \AgdaSymbol{(}\AgdaBound{Γ} \AgdaInductiveConstructor{,} \AgdaBound{A}\AgdaSymbol{)} \AgdaSymbol{=} \AgdaFunction{Σs•} \AgdaBound{Γ} \AgdaFunction{+S} \AgdaSymbol{\_}\<%
\\
%
\\
\>\<\end{code}
}
\noindent The following lemma establishes preservation of contractibility by
one-step suspension:

\begin{code}\>\<%
\\
\>\AgdaFunction{ΣC-Contr} \AgdaSymbol{:} \AgdaSymbol{∀} \AgdaBound{Δ} \AgdaSymbol{→} \AgdaDatatype{isContr} \AgdaBound{Δ} \AgdaSymbol{→} \AgdaDatatype{isContr} \AgdaSymbol{(}\AgdaFunction{ΣC} \AgdaBound{Δ}\AgdaSymbol{)}\<%
\\
\>\<\end{code}
\noindent It is also essential that suspension respects weakening and substitution:

\begin{code}\>\<%
\\
\>\AgdaFunction{ΣT[+T]} \<[9]%
\>[9]\AgdaSymbol{:} \AgdaSymbol{∀\{}\AgdaBound{Γ}\AgdaSymbol{\}(}\AgdaBound{A} \AgdaBound{B} \AgdaSymbol{:} \AgdaDatatype{Ty} \AgdaBound{Γ}\AgdaSymbol{)} \<[28]%
\>[28]\<%
\\
\>[8]\AgdaIndent{9}{}\<[9]%
\>[9]\AgdaSymbol{→} \AgdaFunction{ΣT} \AgdaSymbol{(}\AgdaBound{A} \AgdaFunction{+T} \AgdaBound{B}\AgdaSymbol{)} \AgdaDatatype{≡} \AgdaFunction{ΣT} \AgdaBound{A} \AgdaFunction{+T} \AgdaFunction{ΣT} \AgdaBound{B}\<%
\\
%
\\
\>\AgdaFunction{Σtm[+tm]} \AgdaSymbol{:} \AgdaSymbol{∀\{}\AgdaBound{Γ} \AgdaBound{A}\AgdaSymbol{\}(}\AgdaBound{a} \AgdaSymbol{:} \AgdaDatatype{Tm} \AgdaBound{A}\AgdaSymbol{)(}\AgdaBound{B} \AgdaSymbol{:} \AgdaDatatype{Ty} \AgdaBound{Γ}\AgdaSymbol{)} \<[38]%
\>[38]\<%
\\
\>[8]\AgdaIndent{9}{}\<[9]%
\>[9]\AgdaSymbol{→} \AgdaFunction{Σtm} \AgdaSymbol{(}\AgdaBound{a} \AgdaFunction{+tm} \AgdaBound{B}\AgdaSymbol{)} \AgdaDatatype{≅} \AgdaFunction{Σtm} \AgdaBound{a} \AgdaFunction{+tm} \AgdaFunction{ΣT} \AgdaBound{B}\<%
\\
%
\\
\>\AgdaFunction{ΣT[Σs]T} \<[9]%
\>[9]\AgdaSymbol{:} \AgdaSymbol{∀\{}\AgdaBound{Γ} \AgdaBound{Δ}\AgdaSymbol{\}(}\AgdaBound{A} \AgdaSymbol{:} \AgdaDatatype{Ty} \AgdaBound{Δ}\AgdaSymbol{)(}\AgdaBound{δ} \AgdaSymbol{:} \AgdaBound{Γ} \AgdaDatatype{⇒} \AgdaBound{Δ}\AgdaSymbol{)} \<[39]%
\>[39]\<%
\\
\>[8]\AgdaIndent{9}{}\<[9]%
\>[9]\AgdaSymbol{→} \AgdaSymbol{(}\AgdaFunction{ΣT} \AgdaBound{A}\AgdaSymbol{)} \AgdaFunction{[} \AgdaFunction{Σs} \AgdaBound{δ} \AgdaFunction{]T} \AgdaDatatype{≡} \AgdaFunction{ΣT} \AgdaSymbol{(}\AgdaBound{A} \AgdaFunction{[} \AgdaBound{δ} \AgdaFunction{]T}\AgdaSymbol{)}\<%
\\
\>\<\end{code}
\AgdaHide{
\begin{code}\>\<%
\\
\>\AgdaFunction{ΣT[+T]} \AgdaInductiveConstructor{*} \AgdaBound{B} \AgdaSymbol{=} \AgdaInductiveConstructor{refl}\<%
\\
\>\AgdaFunction{ΣT[+T]} \AgdaSymbol{(}\AgdaInductiveConstructor{\_=h\_} \AgdaSymbol{\{}\AgdaBound{A}\AgdaSymbol{\}} \AgdaBound{a} \AgdaBound{b}\AgdaSymbol{)} \AgdaBound{B} \AgdaSymbol{=} \AgdaFunction{hom≡} \AgdaSymbol{(}\AgdaFunction{Σtm[+tm]} \AgdaBound{a} \AgdaBound{B}\AgdaSymbol{)} \AgdaSymbol{(}\AgdaFunction{Σtm[+tm]} \AgdaBound{b} \AgdaBound{B}\AgdaSymbol{)}\<%
\\
%
\\
\>\AgdaFunction{Σv} \AgdaSymbol{\{}\AgdaSymbol{.(}\AgdaBound{Γ} \AgdaInductiveConstructor{,} \AgdaBound{A}\AgdaSymbol{)}\AgdaSymbol{\}} \AgdaSymbol{\{}\AgdaSymbol{.(}\AgdaBound{A} \AgdaFunction{+T} \AgdaBound{A}\AgdaSymbol{)}\AgdaSymbol{\}} \AgdaSymbol{(}\AgdaInductiveConstructor{v0} \AgdaSymbol{\{}\AgdaBound{Γ}\AgdaSymbol{\}} \AgdaSymbol{\{}\AgdaBound{A}\AgdaSymbol{\})} \AgdaSymbol{=} \AgdaFunction{subst} \AgdaDatatype{Var} \AgdaSymbol{(}\AgdaFunction{sym} \AgdaSymbol{(}\AgdaFunction{ΣT[+T]} \AgdaBound{A} \AgdaBound{A}\AgdaSymbol{))} \AgdaInductiveConstructor{v0}\<%
\\
\>\AgdaFunction{Σv} \AgdaSymbol{\{}\AgdaSymbol{.(}\AgdaBound{Γ} \AgdaInductiveConstructor{,} \AgdaBound{B}\AgdaSymbol{)}\AgdaSymbol{\}} \AgdaSymbol{\{}\AgdaSymbol{.(}\AgdaBound{A} \AgdaFunction{+T} \AgdaBound{B}\AgdaSymbol{)}\AgdaSymbol{\}} \AgdaSymbol{(}\AgdaInductiveConstructor{vS} \AgdaSymbol{\{}\AgdaBound{Γ}\AgdaSymbol{\}} \AgdaSymbol{\{}\AgdaBound{A}\AgdaSymbol{\}} \AgdaSymbol{\{}\AgdaBound{B}\AgdaSymbol{\}} \AgdaBound{x}\AgdaSymbol{)} \AgdaSymbol{=} \AgdaFunction{subst} \AgdaDatatype{Var} \AgdaSymbol{(}\AgdaFunction{sym} \AgdaSymbol{(}\AgdaFunction{ΣT[+T]} \AgdaSymbol{\{\_\}} \AgdaBound{A} \AgdaBound{B}\AgdaSymbol{))} \AgdaSymbol{(}\AgdaInductiveConstructor{vS} \AgdaSymbol{(}\AgdaFunction{Σv} \AgdaBound{x}\AgdaSymbol{))}\<%
\\
%
\\
%
\\
\>\AgdaFunction{Σtm} \AgdaSymbol{(}\AgdaInductiveConstructor{var} \AgdaBound{x}\AgdaSymbol{)} \AgdaSymbol{=} \AgdaInductiveConstructor{var} \AgdaSymbol{(}\AgdaFunction{Σv} \AgdaBound{x}\AgdaSymbol{)}\<%
\\
\>\AgdaFunction{Σtm} \AgdaSymbol{(}\AgdaInductiveConstructor{coh} \AgdaBound{x} \AgdaBound{δ} \AgdaBound{A}\AgdaSymbol{)} \AgdaSymbol{=} \AgdaInductiveConstructor{coh} \AgdaSymbol{(}\AgdaFunction{ΣC-Contr} \AgdaSymbol{\_} \AgdaBound{x}\AgdaSymbol{)} \AgdaSymbol{(}\AgdaFunction{Σs} \AgdaBound{δ}\AgdaSymbol{)} \AgdaSymbol{(}\AgdaFunction{ΣT} \AgdaBound{A}\AgdaSymbol{)} \AgdaFunction{⟦} \AgdaFunction{sym} \AgdaSymbol{(}\AgdaFunction{ΣT[Σs]T} \AgdaBound{A} \AgdaBound{δ}\AgdaSymbol{)} \AgdaFunction{⟫}\<%
\\
%
\\
%
\\
\>\AgdaFunction{Σtm-p1} \AgdaSymbol{:} \AgdaSymbol{\{}\AgdaBound{Γ} \AgdaSymbol{:} \AgdaDatatype{Con}\AgdaSymbol{\}(}\AgdaBound{A} \AgdaSymbol{:} \AgdaDatatype{Ty} \AgdaBound{Γ}\AgdaSymbol{)} \AgdaSymbol{→} \AgdaFunction{Σtm} \AgdaSymbol{\{}\AgdaBound{Γ} \AgdaInductiveConstructor{,} \AgdaBound{A}\AgdaSymbol{\}} \AgdaSymbol{(}\AgdaInductiveConstructor{var} \AgdaInductiveConstructor{v0}\AgdaSymbol{)} \AgdaDatatype{≅} \AgdaInductiveConstructor{var} \AgdaInductiveConstructor{v0} \<[61]%
\>[61]\<%
\\
\>\AgdaFunction{Σtm-p1} \AgdaBound{A} \AgdaSymbol{=} \AgdaFunction{cohOpV} \AgdaSymbol{(}\AgdaFunction{sym} \AgdaSymbol{(}\AgdaFunction{ΣT[+T]} \AgdaBound{A} \AgdaBound{A}\AgdaSymbol{))}\<%
\\
%
\\
\>\AgdaFunction{Σtm-p2} \AgdaSymbol{:} \AgdaSymbol{\{}\AgdaBound{Γ} \AgdaSymbol{:} \AgdaDatatype{Con}\AgdaSymbol{\}(}\AgdaBound{A} \AgdaBound{B} \AgdaSymbol{:} \AgdaDatatype{Ty} \AgdaBound{Γ}\AgdaSymbol{)(}\AgdaBound{x} \AgdaSymbol{:} \AgdaDatatype{Var} \AgdaBound{A}\AgdaSymbol{)} \AgdaSymbol{→} \AgdaInductiveConstructor{var} \AgdaSymbol{(}\AgdaFunction{Σv} \AgdaSymbol{(}\AgdaInductiveConstructor{vS} \AgdaSymbol{\{}B \AgdaSymbol{=} \AgdaBound{B}\AgdaSymbol{\}} \AgdaBound{x}\AgdaSymbol{))} \AgdaDatatype{≅} \AgdaInductiveConstructor{var} \AgdaSymbol{(}\AgdaInductiveConstructor{vS} \AgdaSymbol{(}\AgdaFunction{Σv} \AgdaBound{x}\AgdaSymbol{))}\<%
\\
\>\AgdaFunction{Σtm-p2} \AgdaSymbol{\{}\AgdaBound{Γ}\AgdaSymbol{\}} \AgdaBound{A} \AgdaBound{B} \AgdaBound{x} \AgdaSymbol{=} \AgdaFunction{cohOpV} \AgdaSymbol{(}\AgdaFunction{sym} \AgdaSymbol{(}\AgdaFunction{ΣT[+T]} \AgdaBound{A} \AgdaBound{B}\AgdaSymbol{))}\<%
\\
%
\\
\>\AgdaFunction{Σtm-p2-sp} \AgdaSymbol{:} \AgdaSymbol{\{}\AgdaBound{Γ} \AgdaSymbol{:} \AgdaDatatype{Con}\AgdaSymbol{\}(}\AgdaBound{A} \AgdaSymbol{:} \AgdaDatatype{Ty} \AgdaBound{Γ}\AgdaSymbol{)(}\AgdaBound{B} \AgdaSymbol{:} \AgdaDatatype{Ty} \AgdaSymbol{(}\AgdaBound{Γ} \AgdaInductiveConstructor{,} \AgdaBound{A}\AgdaSymbol{))} \AgdaSymbol{→} \AgdaFunction{Σtm} \AgdaSymbol{\{}\AgdaBound{Γ} \AgdaInductiveConstructor{,} \AgdaBound{A} \AgdaInductiveConstructor{,} \AgdaBound{B}\AgdaSymbol{\}} \AgdaSymbol{(}\AgdaInductiveConstructor{var} \AgdaSymbol{(}\AgdaInductiveConstructor{vS} \AgdaInductiveConstructor{v0}\AgdaSymbol{))} \AgdaDatatype{≅} \AgdaSymbol{(}\AgdaInductiveConstructor{var} \AgdaInductiveConstructor{v0}\AgdaSymbol{)} \AgdaFunction{+tm} \AgdaSymbol{\_}\<%
\\
\>\AgdaFunction{Σtm-p2-sp} \AgdaBound{A} \AgdaBound{B} \AgdaSymbol{=} \AgdaFunction{Σtm-p2} \AgdaSymbol{(}\AgdaBound{A} \AgdaFunction{+T} \AgdaBound{A}\AgdaSymbol{)} \AgdaBound{B} \AgdaInductiveConstructor{v0} \AgdaFunction{∾} \<[40]%
\>[40]\AgdaFunction{cong+tm} \AgdaSymbol{(}\AgdaFunction{Σtm-p1} \AgdaBound{A}\AgdaSymbol{)}\<%
\\
%
\\
\>\AgdaFunction{Σs} \AgdaSymbol{\{}\AgdaBound{Γ}\AgdaSymbol{\}} \AgdaSymbol{\{}\AgdaBound{Δ} \AgdaInductiveConstructor{,} \AgdaBound{A}\AgdaSymbol{\}} \AgdaSymbol{(}\AgdaBound{γ} \AgdaInductiveConstructor{,} \AgdaBound{a}\AgdaSymbol{)} \AgdaSymbol{=} \AgdaSymbol{(}\AgdaFunction{Σs} \AgdaBound{γ}\AgdaSymbol{)} \AgdaInductiveConstructor{,} \AgdaFunction{Σtm} \AgdaBound{a} \AgdaFunction{⟦} \AgdaFunction{ΣT[Σs]T} \AgdaBound{A} \AgdaBound{γ} \AgdaFunction{⟫} \<[56]%
\>[56]\<%
\\
\>\AgdaFunction{Σs} \AgdaSymbol{\{}\AgdaBound{Γ}\AgdaSymbol{\}} \AgdaInductiveConstructor{•} \AgdaSymbol{=} \AgdaFunction{Σs•} \AgdaBound{Γ}\<%
\\
%
\\
%
\\
\>\AgdaFunction{congΣtm} \AgdaSymbol{:} \AgdaSymbol{\{}\AgdaBound{Γ} \AgdaSymbol{:} \AgdaDatatype{Con}\AgdaSymbol{\}\{}\AgdaBound{A} \AgdaBound{B} \AgdaSymbol{:} \AgdaDatatype{Ty} \AgdaBound{Γ}\AgdaSymbol{\}\{}\AgdaBound{a} \AgdaSymbol{:} \AgdaDatatype{Tm} \AgdaBound{A}\AgdaSymbol{\}\{}\AgdaBound{b} \AgdaSymbol{:} \AgdaDatatype{Tm} \AgdaBound{B}\AgdaSymbol{\}} \AgdaSymbol{→} \AgdaBound{a} \AgdaDatatype{≅} \AgdaBound{b} \AgdaSymbol{→} \AgdaFunction{Σtm} \AgdaBound{a} \AgdaDatatype{≅} \AgdaFunction{Σtm} \AgdaBound{b}\<%
\\
\>\AgdaFunction{congΣtm} \AgdaSymbol{(}\AgdaInductiveConstructor{refl} \AgdaSymbol{\_)} \AgdaSymbol{=} \AgdaInductiveConstructor{refl} \AgdaSymbol{\_}\<%
\\
%
\\
\>\AgdaFunction{cohOpΣtm} \AgdaSymbol{:} \AgdaSymbol{∀} \AgdaSymbol{\{}\AgdaBound{Δ} \AgdaSymbol{:} \AgdaDatatype{Con}\AgdaSymbol{\}\{}\AgdaBound{A} \AgdaBound{B} \AgdaSymbol{:} \AgdaDatatype{Ty} \AgdaBound{Δ}\AgdaSymbol{\}(}\AgdaBound{t} \AgdaSymbol{:} \AgdaDatatype{Tm} \AgdaBound{B}\AgdaSymbol{)(}\AgdaBound{p} \AgdaSymbol{:} \AgdaBound{A} \AgdaDatatype{≡} \AgdaBound{B}\AgdaSymbol{)} \AgdaSymbol{→} \AgdaFunction{Σtm} \AgdaSymbol{(}\AgdaBound{t} \AgdaFunction{⟦} \AgdaBound{p} \AgdaFunction{⟫}\AgdaSymbol{)} \AgdaDatatype{≅} \AgdaFunction{Σtm} \AgdaBound{t}\<%
\\
\>\AgdaFunction{cohOpΣtm} \AgdaBound{t} \AgdaBound{p} \AgdaSymbol{=} \<[16]%
\>[16]\AgdaFunction{congΣtm} \AgdaSymbol{(}\AgdaFunction{cohOp} \AgdaBound{p}\AgdaSymbol{)}\<%
\\
%
\\
%
\\
\>\AgdaFunction{Σs⊚} \AgdaSymbol{:} \AgdaSymbol{∀} \AgdaSymbol{\{}\AgdaBound{Δ} \AgdaBound{Δ₁} \AgdaBound{Γ}\AgdaSymbol{\}(}\AgdaBound{δ} \AgdaSymbol{:} \AgdaBound{Δ} \AgdaDatatype{⇒} \AgdaBound{Δ₁}\AgdaSymbol{)(}\AgdaBound{γ} \AgdaSymbol{:} \AgdaBound{Γ} \AgdaDatatype{⇒} \AgdaBound{Δ}\AgdaSymbol{)} \AgdaSymbol{→} \AgdaFunction{Σs} \AgdaSymbol{(}\AgdaBound{δ} \AgdaFunction{⊚} \AgdaBound{γ}\AgdaSymbol{)} \AgdaDatatype{≡} \AgdaFunction{Σs} \AgdaBound{δ} \AgdaFunction{⊚} \AgdaFunction{Σs} \AgdaBound{γ}\<%
\\
%
\\
\>\AgdaFunction{Σv[Σs]v} \AgdaSymbol{:} \<[11]%
\>[11]\AgdaSymbol{∀} \AgdaSymbol{\{}\AgdaBound{Γ} \AgdaBound{Δ} \AgdaSymbol{:} \AgdaDatatype{Con}\AgdaSymbol{\}\{}\AgdaBound{A} \AgdaSymbol{:} \AgdaDatatype{Ty} \AgdaBound{Δ}\AgdaSymbol{\}(}\AgdaBound{x} \AgdaSymbol{:} \AgdaDatatype{Var} \AgdaBound{A}\AgdaSymbol{)(}\AgdaBound{δ} \AgdaSymbol{:} \AgdaBound{Γ} \AgdaDatatype{⇒} \AgdaBound{Δ}\AgdaSymbol{)} \AgdaSymbol{→} \AgdaFunction{Σv} \AgdaBound{x} \AgdaFunction{[} \AgdaFunction{Σs} \AgdaBound{δ} \AgdaFunction{]V} \AgdaDatatype{≅} \AgdaFunction{Σtm} \AgdaSymbol{(}\AgdaBound{x} \AgdaFunction{[} \AgdaBound{δ} \AgdaFunction{]V}\AgdaSymbol{)}\<%
\\
\>\AgdaFunction{Σv[Σs]v} \AgdaSymbol{(}\AgdaInductiveConstructor{v0} \AgdaSymbol{\{}\AgdaBound{Γ}\AgdaSymbol{\}} \AgdaSymbol{\{}\AgdaBound{A}\AgdaSymbol{\})} \AgdaSymbol{(}\AgdaBound{δ} \AgdaInductiveConstructor{,} \AgdaBound{a}\AgdaSymbol{)} \AgdaSymbol{=} \AgdaFunction{congtm} \AgdaSymbol{(}\AgdaFunction{Σtm-p1} \AgdaBound{A}\AgdaSymbol{)} \AgdaFunction{∾} \AgdaFunction{wk-coh} \AgdaFunction{∾} \AgdaFunction{cohOp} \AgdaSymbol{(}\AgdaFunction{ΣT[Σs]T} \AgdaBound{A} \AgdaBound{δ}\AgdaSymbol{)} \AgdaFunction{∾} \AgdaFunction{cohOpΣtm} \AgdaBound{a} \AgdaFunction{+T[,]T} \AgdaFunction{-¹}\<%
\\
\>\AgdaFunction{Σv[Σs]v} \AgdaSymbol{(}\AgdaInductiveConstructor{vS} \AgdaSymbol{\{}\AgdaBound{Γ}\AgdaSymbol{\}} \AgdaSymbol{\{}\AgdaBound{A}\AgdaSymbol{\}} \AgdaSymbol{\{}\AgdaBound{B}\AgdaSymbol{\}} \AgdaBound{x}\AgdaSymbol{)} \AgdaSymbol{(}\AgdaBound{δ} \AgdaInductiveConstructor{,} \AgdaBound{a}\AgdaSymbol{)} \AgdaSymbol{=} \AgdaFunction{congtm} \AgdaSymbol{(}\AgdaFunction{Σtm-p2} \AgdaBound{A} \AgdaBound{B} \AgdaBound{x}\AgdaSymbol{)} \AgdaFunction{∾}\<%
\\
\>[9]\AgdaIndent{39}{}\<[39]%
\>[39]\AgdaFunction{+tm[,]tm} \AgdaSymbol{(}\AgdaFunction{Σtm} \AgdaSymbol{(}\AgdaInductiveConstructor{var} \AgdaBound{x}\AgdaSymbol{))} \AgdaFunction{∾}\<%
\\
\>[9]\AgdaIndent{39}{}\<[39]%
\>[39]\AgdaFunction{Σv[Σs]v} \AgdaBound{x} \AgdaBound{δ} \AgdaFunction{∾} \AgdaFunction{cohOpΣtm} \AgdaSymbol{(}\AgdaBound{x} \AgdaFunction{[} \AgdaBound{δ} \AgdaFunction{]V}\AgdaSymbol{)} \AgdaFunction{+T[,]T} \AgdaFunction{-¹}\<%
\\
%
\\
\>\AgdaFunction{Σtm[Σs]tm} \AgdaSymbol{:} \AgdaSymbol{∀} \AgdaSymbol{\{}\AgdaBound{Γ} \AgdaBound{Δ} \AgdaSymbol{:} \AgdaDatatype{Con}\AgdaSymbol{\}\{}\AgdaBound{A} \AgdaSymbol{:} \AgdaDatatype{Ty} \AgdaBound{Δ}\AgdaSymbol{\}(}\AgdaBound{a} \AgdaSymbol{:} \AgdaDatatype{Tm} \AgdaBound{A}\AgdaSymbol{)(}\AgdaBound{δ} \AgdaSymbol{:} \AgdaBound{Γ} \AgdaDatatype{⇒} \AgdaBound{Δ}\AgdaSymbol{)} \AgdaSymbol{→} \<[59]%
\>[59]\<%
\\
\>[16]\AgdaIndent{14}{}\<[14]%
\>[14]\AgdaSymbol{(}\AgdaFunction{Σtm} \AgdaBound{a}\AgdaSymbol{)} \AgdaFunction{[} \AgdaFunction{Σs} \AgdaBound{δ} \AgdaFunction{]tm} \AgdaDatatype{≅} \AgdaFunction{Σtm} \AgdaSymbol{(}\AgdaBound{a} \AgdaFunction{[} \AgdaBound{δ} \AgdaFunction{]tm}\AgdaSymbol{)}\<%
\\
\>\AgdaFunction{Σtm[Σs]tm} \AgdaSymbol{(}\AgdaInductiveConstructor{var} \AgdaBound{x}\AgdaSymbol{)} \AgdaBound{δ} \AgdaSymbol{=} \AgdaFunction{Σv[Σs]v} \AgdaBound{x} \AgdaBound{δ}\<%
\\
\>\AgdaFunction{Σtm[Σs]tm} \AgdaSymbol{\{}\AgdaBound{Γ}\AgdaSymbol{\}} \AgdaSymbol{\{}\AgdaBound{Δ}\AgdaSymbol{\}} \AgdaSymbol{(}\AgdaInductiveConstructor{coh} \AgdaSymbol{\{}Δ \AgdaSymbol{=} \AgdaBound{Δ₁}\AgdaSymbol{\}} \AgdaBound{x} \AgdaBound{δ} \AgdaBound{A}\AgdaSymbol{)} \AgdaBound{δ₁} \AgdaSymbol{=} \AgdaFunction{congtm} \AgdaSymbol{(}\AgdaFunction{cohOp} \AgdaSymbol{(}\AgdaFunction{sym} \AgdaSymbol{(}\AgdaFunction{ΣT[Σs]T} \AgdaBound{A} \AgdaBound{δ}\AgdaSymbol{)))} \<[79]%
\>[79]\<%
\\
\>[0]\AgdaIndent{22}{}\<[22]%
\>[22]\AgdaFunction{∾} \AgdaFunction{cohOp} \AgdaSymbol{(}\AgdaFunction{sym} \AgdaFunction{[⊚]T}\AgdaSymbol{)} \<[41]%
\>[41]\<%
\\
\>[0]\AgdaIndent{22}{}\<[22]%
\>[22]\AgdaFunction{∾} \AgdaFunction{coh-eq} \AgdaSymbol{(}\AgdaFunction{sym} \AgdaSymbol{(}\AgdaFunction{Σs⊚} \AgdaBound{δ} \AgdaBound{δ₁}\AgdaSymbol{))} \<[48]%
\>[48]\<%
\\
\>[0]\AgdaIndent{22}{}\<[22]%
\>[22]\AgdaFunction{∾} \AgdaSymbol{(}\AgdaFunction{cohOpΣtm} \AgdaSymbol{(}\AgdaInductiveConstructor{coh} \AgdaBound{x} \AgdaSymbol{(}\AgdaBound{δ} \AgdaFunction{⊚} \AgdaBound{δ₁}\AgdaSymbol{)} \AgdaBound{A}\AgdaSymbol{)} \AgdaSymbol{(}\AgdaFunction{sym} \AgdaFunction{[⊚]T}\AgdaSymbol{)} \<[64]%
\>[64]\<%
\\
\>[0]\AgdaIndent{22}{}\<[22]%
\>[22]\AgdaFunction{∾} \AgdaFunction{cohOp} \AgdaSymbol{(}\AgdaFunction{sym} \AgdaSymbol{(}\AgdaFunction{ΣT[Σs]T} \AgdaBound{A} \AgdaSymbol{(}\AgdaBound{δ} \AgdaFunction{⊚} \AgdaBound{δ₁}\AgdaSymbol{))))} \AgdaFunction{-¹}\<%
\\
%
\\
\>\AgdaFunction{Σs•-left-id} \AgdaSymbol{:} \AgdaSymbol{∀\{}\AgdaBound{Γ} \AgdaBound{Δ} \AgdaSymbol{:} \AgdaDatatype{Con}\AgdaSymbol{\}(}\AgdaBound{γ} \AgdaSymbol{:} \AgdaBound{Γ} \AgdaDatatype{⇒} \AgdaBound{Δ}\AgdaSymbol{)} \AgdaSymbol{→} \AgdaFunction{Σs} \AgdaSymbol{\{}\AgdaBound{Γ}\AgdaSymbol{\}} \AgdaInductiveConstructor{•} \AgdaDatatype{≡} \AgdaFunction{Σs} \AgdaSymbol{\{}\AgdaBound{Δ}\AgdaSymbol{\}} \AgdaInductiveConstructor{•} \AgdaFunction{⊚} \AgdaFunction{Σs} \AgdaBound{γ}\<%
\\
\>\AgdaFunction{Σs•-left-id} \AgdaSymbol{\{}\AgdaInductiveConstructor{ε}\AgdaSymbol{\}} \AgdaSymbol{\{}\AgdaInductiveConstructor{ε}\AgdaSymbol{\}} \AgdaInductiveConstructor{•} \AgdaSymbol{=} \AgdaInductiveConstructor{refl}\<%
\\
\>\AgdaFunction{Σs•-left-id} \AgdaSymbol{\{}\AgdaInductiveConstructor{ε}\AgdaSymbol{\}} \AgdaSymbol{\{}\AgdaBound{Δ} \AgdaInductiveConstructor{,} \AgdaBound{A}\AgdaSymbol{\}} \AgdaSymbol{(}\AgdaBound{γ} \AgdaInductiveConstructor{,} \AgdaBound{a}\AgdaSymbol{)} \AgdaSymbol{=} \AgdaFunction{trans} \AgdaSymbol{(}\AgdaFunction{Σs•-left-id} \AgdaBound{γ}\AgdaSymbol{)} \AgdaSymbol{(}\AgdaFunction{sym} \AgdaSymbol{(}\AgdaFunction{⊚wk} \AgdaSymbol{(}\AgdaFunction{Σs•} \AgdaBound{Δ}\AgdaSymbol{)))}\<%
\\
\>\AgdaFunction{Σs•-left-id} \AgdaSymbol{\{}\AgdaBound{Γ} \AgdaInductiveConstructor{,} \AgdaBound{A}\AgdaSymbol{\}} \AgdaSymbol{\{}\AgdaInductiveConstructor{ε}\AgdaSymbol{\}} \AgdaInductiveConstructor{•} \AgdaSymbol{=} \AgdaFunction{trans} \AgdaSymbol{(}\AgdaFunction{cong} \AgdaSymbol{(λ} \AgdaBound{x} \AgdaSymbol{→} \AgdaBound{x} \AgdaFunction{+S} \AgdaFunction{ΣT} \AgdaBound{A}\AgdaSymbol{)} \AgdaSymbol{(}\AgdaFunction{Σs•-left-id} \AgdaSymbol{\{}\AgdaBound{Γ}\AgdaSymbol{\}} \AgdaSymbol{\{}\AgdaInductiveConstructor{ε}\AgdaSymbol{\}} \AgdaInductiveConstructor{•}\AgdaSymbol{))} \AgdaSymbol{(}\AgdaFunction{cm-eq} \AgdaSymbol{(}\AgdaFunction{cm-eq} \AgdaInductiveConstructor{refl} \AgdaSymbol{(}\AgdaFunction{[+S]V} \AgdaSymbol{(}\AgdaInductiveConstructor{vS} \AgdaInductiveConstructor{v0}\AgdaSymbol{)} \AgdaSymbol{\{}\AgdaFunction{Σs•} \AgdaBound{Γ}\AgdaSymbol{\}} \AgdaFunction{-¹}\AgdaSymbol{))} \AgdaSymbol{(}\AgdaFunction{[+S]V} \AgdaInductiveConstructor{v0} \AgdaSymbol{\{}\AgdaFunction{Σs•} \AgdaBound{Γ}\AgdaSymbol{\}} \AgdaFunction{-¹}\AgdaSymbol{))}\<%
\\
\>\AgdaFunction{Σs•-left-id} \AgdaSymbol{\{}\AgdaBound{Γ} \AgdaInductiveConstructor{,} \AgdaBound{A}\AgdaSymbol{\}} \AgdaSymbol{\{}\AgdaBound{Δ} \AgdaInductiveConstructor{,} \AgdaBound{A₁}\AgdaSymbol{\}} \AgdaSymbol{(}\AgdaBound{γ} \AgdaInductiveConstructor{,} \AgdaBound{a}\AgdaSymbol{)} \AgdaSymbol{=} \AgdaFunction{trans} \AgdaSymbol{(}\AgdaFunction{Σs•-left-id} \AgdaBound{γ}\AgdaSymbol{)} \AgdaSymbol{(}\AgdaFunction{sym} \AgdaSymbol{(}\AgdaFunction{⊚wk} \AgdaSymbol{(}\AgdaFunction{Σs•} \AgdaBound{Δ}\AgdaSymbol{)))} \<[81]%
\>[81]\<%
\\
%
\\
\>\AgdaFunction{Σs⊚} \AgdaInductiveConstructor{•} \AgdaBound{γ} \AgdaSymbol{=} \AgdaFunction{Σs•-left-id} \AgdaBound{γ}\<%
\\
\>\AgdaFunction{Σs⊚} \AgdaSymbol{\{}\AgdaBound{Δ}\AgdaSymbol{\}} \AgdaSymbol{(}\AgdaInductiveConstructor{\_,\_} \AgdaBound{δ} \AgdaSymbol{\{}\AgdaBound{A}\AgdaSymbol{\}} \AgdaBound{a}\AgdaSymbol{)} \AgdaBound{γ} \AgdaSymbol{=} \AgdaFunction{cm-eq} \AgdaSymbol{(}\AgdaFunction{Σs⊚} \AgdaBound{δ} \AgdaBound{γ}\AgdaSymbol{)} \AgdaSymbol{(}\AgdaFunction{cohOp} \AgdaSymbol{(}\AgdaFunction{ΣT[Σs]T} \AgdaBound{A} \AgdaSymbol{(}\AgdaBound{δ} \AgdaFunction{⊚} \AgdaBound{γ}\AgdaSymbol{))} \AgdaFunction{∾} \AgdaFunction{cohOpΣtm} \AgdaSymbol{(}\AgdaBound{a} \AgdaFunction{[} \AgdaBound{γ} \AgdaFunction{]tm}\AgdaSymbol{)} \AgdaFunction{[⊚]T} \AgdaFunction{∾} \AgdaSymbol{(}\AgdaFunction{cohOp} \AgdaFunction{[⊚]T} \AgdaFunction{∾} \AgdaFunction{congtm} \AgdaSymbol{(}\AgdaFunction{cohOp} \AgdaSymbol{(}\AgdaFunction{ΣT[Σs]T} \AgdaBound{A} \AgdaBound{δ}\AgdaSymbol{))} \AgdaFunction{∾} \AgdaFunction{Σtm[Σs]tm} \AgdaBound{a} \AgdaBound{γ}\AgdaSymbol{)} \AgdaFunction{-¹}\AgdaSymbol{)} \<[163]%
\>[163]\<%
\\
%
\\
%
\\
\>\AgdaFunction{ΣT[+S]T} \AgdaSymbol{:} \AgdaSymbol{∀\{}\AgdaBound{Γ} \AgdaBound{Δ} \AgdaSymbol{:} \AgdaDatatype{Con}\AgdaSymbol{\}(}\AgdaBound{A} \AgdaSymbol{:} \AgdaDatatype{Ty} \AgdaBound{Δ}\AgdaSymbol{)(}\AgdaBound{δ} \AgdaSymbol{:} \AgdaBound{Γ} \AgdaDatatype{⇒} \AgdaBound{Δ}\AgdaSymbol{)(}\AgdaBound{B} \AgdaSymbol{:} \AgdaDatatype{Ty} \AgdaBound{Γ}\AgdaSymbol{)} \AgdaSymbol{→} \AgdaFunction{ΣT} \AgdaBound{A} \AgdaFunction{[} \AgdaFunction{Σs} \AgdaBound{δ} \AgdaFunction{+S} \AgdaFunction{ΣT} \AgdaBound{B} \AgdaFunction{]T} \AgdaDatatype{≡} \AgdaFunction{ΣT} \AgdaSymbol{(}\AgdaBound{A} \AgdaFunction{[} \AgdaBound{δ} \AgdaFunction{]T}\AgdaSymbol{)} \AgdaFunction{+T} \AgdaFunction{ΣT} \AgdaBound{B}\<%
\\
\>\AgdaFunction{ΣT[+S]T} \AgdaBound{A} \AgdaBound{δ} \AgdaBound{B} \AgdaSymbol{=} \AgdaFunction{trans} \AgdaFunction{[+S]T} \AgdaSymbol{(}\AgdaFunction{wk-T} \AgdaSymbol{(}\AgdaFunction{ΣT[Σs]T} \AgdaBound{A} \AgdaBound{δ}\AgdaSymbol{))}\<%
\\
%
\\
\>\AgdaFunction{ΣsDis} \AgdaSymbol{:} \AgdaSymbol{∀\{}\AgdaBound{Γ} \AgdaBound{Δ} \AgdaSymbol{:} \AgdaDatatype{Con}\AgdaSymbol{\}\{}\AgdaBound{A} \AgdaSymbol{:} \AgdaDatatype{Ty} \AgdaBound{Δ}\AgdaSymbol{\}(}\AgdaBound{δ} \AgdaSymbol{:} \AgdaBound{Γ} \AgdaDatatype{⇒} \AgdaBound{Δ}\AgdaSymbol{)(}\AgdaBound{a} \AgdaSymbol{:} \AgdaDatatype{Tm} \AgdaSymbol{(}\AgdaBound{A} \AgdaFunction{[} \AgdaBound{δ} \AgdaFunction{]T}\AgdaSymbol{))(}\AgdaBound{B} \AgdaSymbol{:} \AgdaDatatype{Ty} \AgdaBound{Γ}\AgdaSymbol{)} \AgdaSymbol{→} \AgdaSymbol{(}\AgdaFunction{Σs} \AgdaSymbol{\{}\AgdaBound{Γ}\AgdaSymbol{\}} \AgdaSymbol{\{}\AgdaBound{Δ} \AgdaInductiveConstructor{,} \AgdaBound{A}\AgdaSymbol{\}} \AgdaSymbol{(}\AgdaBound{δ} \AgdaInductiveConstructor{,} \AgdaBound{a}\AgdaSymbol{))} \AgdaFunction{+S} \AgdaFunction{ΣT} \AgdaBound{B} \AgdaDatatype{≡} \AgdaFunction{Σs} \AgdaBound{δ} \AgdaFunction{+S} \AgdaFunction{ΣT} \AgdaBound{B} \AgdaInductiveConstructor{,} \AgdaSymbol{((}\AgdaFunction{Σtm} \AgdaBound{a}\AgdaSymbol{)} \AgdaFunction{+tm} \AgdaFunction{ΣT} \AgdaBound{B}\AgdaSymbol{)} \AgdaFunction{⟦} \AgdaFunction{ΣT[+S]T} \AgdaBound{A} \AgdaBound{δ} \AgdaBound{B} \AgdaFunction{⟫}\<%
\\
\>\AgdaFunction{ΣsDis} \AgdaSymbol{\{}\AgdaBound{Γ}\AgdaSymbol{\}} \AgdaSymbol{\{}\AgdaBound{Δ}\AgdaSymbol{\}} \AgdaSymbol{\{}\AgdaBound{A}\AgdaSymbol{\}} \AgdaBound{δ} \AgdaBound{a} \AgdaBound{B} \AgdaSymbol{=} \AgdaFunction{cm-eq} \AgdaInductiveConstructor{refl} \AgdaSymbol{(}\AgdaFunction{wk-coh+} \AgdaFunction{∾} \AgdaSymbol{(}\AgdaFunction{cohOp} \AgdaSymbol{(}\AgdaFunction{trans} \AgdaFunction{[+S]T} \AgdaSymbol{(}\AgdaFunction{wk-T} \AgdaSymbol{(}\AgdaFunction{ΣT[Σs]T} \AgdaBound{A} \AgdaBound{δ}\AgdaSymbol{)))} \AgdaFunction{∾} \AgdaFunction{cong+tm2} \AgdaSymbol{(}\AgdaFunction{ΣT[Σs]T} \AgdaBound{A} \AgdaBound{δ}\AgdaSymbol{))} \AgdaFunction{-¹}\AgdaSymbol{)}\<%
\\
%
\\
\>\AgdaFunction{ΣsΣT} \AgdaSymbol{:} \AgdaSymbol{∀} \AgdaSymbol{\{}\AgdaBound{Γ} \AgdaBound{Δ} \AgdaSymbol{:} \AgdaDatatype{Con}\AgdaSymbol{\}(}\AgdaBound{δ} \AgdaSymbol{:} \AgdaBound{Γ} \AgdaDatatype{⇒} \AgdaBound{Δ}\AgdaSymbol{)(}\AgdaBound{B} \AgdaSymbol{:} \AgdaDatatype{Ty} \AgdaBound{Γ}\AgdaSymbol{)} \AgdaSymbol{→} \AgdaFunction{Σs} \AgdaSymbol{(}\AgdaBound{δ} \AgdaFunction{+S} \AgdaBound{B}\AgdaSymbol{)} \AgdaDatatype{≡} \AgdaFunction{Σs} \AgdaBound{δ} \AgdaFunction{+S} \AgdaFunction{ΣT} \AgdaBound{B}\<%
\\
\>\AgdaFunction{ΣsΣT} \AgdaInductiveConstructor{•} \AgdaSymbol{\_} \AgdaSymbol{=} \AgdaInductiveConstructor{refl}\<%
\\
\>\AgdaFunction{ΣsΣT} \AgdaSymbol{(}\AgdaInductiveConstructor{\_,\_} \AgdaBound{δ} \AgdaSymbol{\{}\AgdaBound{A}\AgdaSymbol{\}} \AgdaBound{a}\AgdaSymbol{)} \AgdaBound{B} \AgdaSymbol{=} \AgdaFunction{cm-eq} \AgdaSymbol{(}\AgdaFunction{ΣsΣT} \AgdaBound{δ} \AgdaBound{B}\AgdaSymbol{)} \AgdaSymbol{(}\AgdaFunction{cohOp} \AgdaSymbol{(}\AgdaFunction{ΣT[Σs]T} \AgdaBound{A} \AgdaSymbol{(}\AgdaBound{δ} \AgdaFunction{+S} \AgdaBound{B}\AgdaSymbol{))} \AgdaFunction{∾} \AgdaFunction{cohOpΣtm} \AgdaSymbol{(}\AgdaBound{a} \AgdaFunction{+tm} \AgdaBound{B}\AgdaSymbol{)} \AgdaFunction{[+S]T} \AgdaFunction{∾} \AgdaFunction{Σtm[+tm]} \AgdaBound{a} \AgdaBound{B} \AgdaFunction{∾} \AgdaFunction{cong+tm2} \AgdaSymbol{(}\AgdaFunction{ΣT[Σs]T} \AgdaBound{A} \AgdaBound{δ}\AgdaSymbol{)} \AgdaFunction{∾} \AgdaFunction{wk-coh+} \AgdaFunction{-¹}\AgdaSymbol{)} \<[149]%
\>[149]\<%
\\
%
\\
\>\AgdaFunction{*'[Σs]T} \AgdaSymbol{:} \AgdaSymbol{\{}\AgdaBound{Γ} \AgdaBound{Δ} \AgdaSymbol{:} \AgdaDatatype{Con}\AgdaSymbol{\}} \AgdaSymbol{→} \AgdaSymbol{(}\AgdaBound{δ} \AgdaSymbol{:} \AgdaBound{Γ} \AgdaDatatype{⇒} \AgdaBound{Δ}\AgdaSymbol{)} \AgdaSymbol{→} \AgdaFunction{*'} \AgdaSymbol{\{}\AgdaBound{Δ}\AgdaSymbol{\}} \AgdaFunction{[} \AgdaFunction{Σs} \AgdaBound{δ} \AgdaFunction{]T} \AgdaDatatype{≡} \AgdaFunction{*'} \AgdaSymbol{\{}\AgdaBound{Γ}\AgdaSymbol{\}}\<%
\\
\>\AgdaFunction{*'[Σs]T} \AgdaSymbol{\{}\AgdaInductiveConstructor{ε}\AgdaSymbol{\}} \AgdaInductiveConstructor{•} \AgdaSymbol{=} \AgdaInductiveConstructor{refl}\<%
\\
\>\AgdaFunction{*'[Σs]T} \AgdaSymbol{\{}\AgdaBound{Γ} \AgdaInductiveConstructor{,} \AgdaBound{A}\AgdaSymbol{\}} \AgdaInductiveConstructor{•} \AgdaSymbol{=} \AgdaFunction{trans} \AgdaSymbol{(}\AgdaFunction{[+S]T} \AgdaSymbol{\{}A \AgdaSymbol{=} \AgdaFunction{*'} \AgdaSymbol{\{}\AgdaInductiveConstructor{ε}\AgdaSymbol{\}\}} \AgdaSymbol{\{}δ \AgdaSymbol{=} \AgdaFunction{Σs} \AgdaSymbol{\{}\AgdaBound{Γ}\AgdaSymbol{\}} \AgdaInductiveConstructor{•}\AgdaSymbol{\})} \AgdaSymbol{(}\AgdaFunction{wk-T} \AgdaSymbol{(}\AgdaFunction{*'[Σs]T} \AgdaSymbol{\{}\AgdaBound{Γ}\AgdaSymbol{\}} \AgdaInductiveConstructor{•}\AgdaSymbol{))}\<%
\\
\>\AgdaFunction{*'[Σs]T} \AgdaSymbol{\{}\AgdaBound{Γ}\AgdaSymbol{\}} \AgdaSymbol{\{}\AgdaBound{Δ} \AgdaInductiveConstructor{,} \AgdaBound{A}\AgdaSymbol{\}} \AgdaSymbol{(}\AgdaBound{γ} \AgdaInductiveConstructor{,} \AgdaBound{a}\AgdaSymbol{)} \AgdaSymbol{=} \AgdaFunction{trans} \AgdaFunction{+T[,]T} \AgdaSymbol{(}\AgdaFunction{*'[Σs]T} \AgdaBound{γ}\AgdaSymbol{)}\<%
\\
%
\\
\>\AgdaFunction{ΣT[Σs]T} \AgdaInductiveConstructor{*} \AgdaBound{δ} \AgdaSymbol{=} \AgdaFunction{*'[Σs]T} \AgdaBound{δ}\<%
\\
\>\AgdaFunction{ΣT[Σs]T} \AgdaSymbol{(}\AgdaInductiveConstructor{\_=h\_} \AgdaSymbol{\{}\AgdaBound{A}\AgdaSymbol{\}} \AgdaBound{a} \AgdaBound{b}\AgdaSymbol{)} \AgdaBound{δ} \AgdaSymbol{=} \AgdaFunction{hom≡} \AgdaSymbol{(}\AgdaFunction{Σtm[Σs]tm} \AgdaBound{a} \AgdaBound{δ}\AgdaSymbol{)} \AgdaSymbol{(}\AgdaFunction{Σtm[Σs]tm} \AgdaBound{b} \AgdaBound{δ}\AgdaSymbol{)}\<%
\\
%
\\
\>\AgdaFunction{Σtm[+tm]} \AgdaSymbol{\{}A \AgdaSymbol{=} \AgdaBound{A}\AgdaSymbol{\}} \AgdaSymbol{(}\AgdaInductiveConstructor{var} \AgdaBound{x}\AgdaSymbol{)} \AgdaBound{B} \AgdaSymbol{=} \AgdaFunction{cohOpV} \AgdaSymbol{(}\AgdaFunction{sym} \AgdaSymbol{(}\AgdaFunction{ΣT[+T]} \AgdaBound{A} \AgdaBound{B}\AgdaSymbol{))}\<%
\\
\>\AgdaFunction{Σtm[+tm]} \AgdaSymbol{\{}\AgdaBound{Γ}\AgdaSymbol{\}} \AgdaSymbol{(}\AgdaInductiveConstructor{coh} \AgdaSymbol{\{}Δ \AgdaSymbol{=} \AgdaBound{Δ}\AgdaSymbol{\}} \AgdaBound{x} \AgdaBound{δ} \AgdaBound{A}\AgdaSymbol{)} \AgdaBound{B} \AgdaSymbol{=} \AgdaFunction{cohOpΣtm} \AgdaSymbol{(}\AgdaInductiveConstructor{coh} \AgdaBound{x} \AgdaSymbol{(}\AgdaBound{δ} \AgdaFunction{+S} \AgdaBound{B}\AgdaSymbol{)} \AgdaBound{A}\AgdaSymbol{)} \AgdaSymbol{(}\AgdaFunction{sym} \AgdaFunction{[+S]T}\AgdaSymbol{)} \AgdaFunction{∾} \AgdaFunction{cohOp} \AgdaSymbol{(}\AgdaFunction{sym} \AgdaSymbol{(}\AgdaFunction{ΣT[Σs]T} \AgdaBound{A} \AgdaSymbol{(}\AgdaBound{δ} \AgdaFunction{+S} \AgdaBound{B}\AgdaSymbol{)))} \AgdaFunction{∾} \AgdaFunction{coh-eq} \AgdaSymbol{(}\AgdaFunction{ΣsΣT} \AgdaBound{δ} \AgdaBound{B}\AgdaSymbol{)} \AgdaFunction{∾} \AgdaFunction{cohOp} \AgdaSymbol{(}\AgdaFunction{sym} \AgdaFunction{[+S]T}\AgdaSymbol{)} \AgdaFunction{-¹} \AgdaFunction{∾} \AgdaFunction{cong+tm2} \AgdaSymbol{(}\AgdaFunction{sym} \AgdaSymbol{(}\AgdaFunction{ΣT[Σs]T} \AgdaBound{A} \AgdaBound{δ}\AgdaSymbol{))}\<%
\\
%
\\
%
\\
\>\AgdaFunction{ΣC-Contr} \AgdaSymbol{.(}\AgdaInductiveConstructor{ε} \AgdaInductiveConstructor{,} \AgdaInductiveConstructor{*}\AgdaSymbol{)} \AgdaInductiveConstructor{c*} \AgdaSymbol{=} \AgdaInductiveConstructor{ext} \AgdaInductiveConstructor{c*} \AgdaInductiveConstructor{v0}\<%
\\
\>\AgdaFunction{ΣC-Contr} \AgdaSymbol{.(}\AgdaBound{Γ} \AgdaInductiveConstructor{,} \AgdaBound{A} \AgdaInductiveConstructor{,} \AgdaSymbol{(}\AgdaInductiveConstructor{var} \AgdaSymbol{(}\AgdaInductiveConstructor{vS} \AgdaBound{x}\AgdaSymbol{)} \AgdaInductiveConstructor{=h} \AgdaInductiveConstructor{var} \AgdaInductiveConstructor{v0}\AgdaSymbol{))} \AgdaSymbol{(}\AgdaInductiveConstructor{ext} \AgdaSymbol{\{}\AgdaBound{Γ}\AgdaSymbol{\}} \AgdaBound{r} \AgdaSymbol{\{}\AgdaBound{A}\AgdaSymbol{\}} \AgdaBound{x}\AgdaSymbol{)} \AgdaSymbol{=} \AgdaFunction{subst} \AgdaSymbol{(λ} \AgdaBound{y} \AgdaSymbol{→} \AgdaDatatype{isContr} \AgdaSymbol{(}\AgdaFunction{ΣC} \AgdaBound{Γ} \AgdaInductiveConstructor{,} \AgdaFunction{ΣT} \AgdaBound{A} \AgdaInductiveConstructor{,} \AgdaBound{y}\AgdaSymbol{))}\<%
\\
\>[22]\AgdaIndent{65}{}\<[65]%
\>[65]\AgdaSymbol{(}\AgdaFunction{hom≡} \AgdaSymbol{(}\AgdaFunction{cohOpV} \AgdaSymbol{(}\AgdaFunction{sym} \AgdaSymbol{(}\AgdaFunction{ΣT[+T]} \AgdaBound{A} \AgdaBound{A}\AgdaSymbol{))} \AgdaFunction{-¹}\AgdaSymbol{)}\<%
\\
\>[65]\AgdaIndent{66}{}\<[66]%
\>[66]\AgdaSymbol{(}\AgdaFunction{cohOpV} \AgdaSymbol{(}\AgdaFunction{sym} \AgdaSymbol{(}\AgdaFunction{ΣT[+T]} \AgdaBound{A} \AgdaBound{A}\AgdaSymbol{))} \AgdaFunction{-¹}\AgdaSymbol{))}\<%
\\
\>[0]\AgdaIndent{65}{}\<[65]%
\>[65]\AgdaSymbol{(}\AgdaInductiveConstructor{ext} \AgdaSymbol{(}\AgdaFunction{ΣC-Contr} \AgdaBound{Γ} \AgdaBound{r}\AgdaSymbol{)} \AgdaSymbol{\{}\AgdaFunction{ΣT} \AgdaBound{A}\AgdaSymbol{\}} \AgdaSymbol{(}\AgdaFunction{Σv} \AgdaBound{x}\AgdaSymbol{))}\<%
\\
\>\<\end{code}}
General suspension to the level of a type $A$ is defined by iteration of
one-level suspension. For symmetry and ease of reading the following
suspension functions take as a parameter a type $A$ in $\Gamma$, while they
depend only on its level. 

\begin{code}\>\<%
\\
\>\AgdaFunction{ΣC-it} \<[8]%
\>[8]\AgdaSymbol{:} \AgdaSymbol{∀\{}\AgdaBound{Γ}\AgdaSymbol{\}(}\AgdaBound{A} \AgdaSymbol{:} \AgdaDatatype{Ty} \AgdaBound{Γ}\AgdaSymbol{)} \AgdaSymbol{→} \AgdaDatatype{Con} \AgdaSymbol{→} \AgdaDatatype{Con}\<%
\\
%
\\
\>\AgdaFunction{ΣT-it} \<[8]%
\>[8]\AgdaSymbol{:} \AgdaSymbol{∀\{}\AgdaBound{Γ} \AgdaBound{Δ}\AgdaSymbol{\}(}\AgdaBound{A} \AgdaSymbol{:} \AgdaDatatype{Ty} \AgdaBound{Γ}\AgdaSymbol{)} \AgdaSymbol{→} \AgdaDatatype{Ty} \AgdaBound{Δ} \AgdaSymbol{→} \AgdaDatatype{Ty} \AgdaSymbol{(}\AgdaFunction{ΣC-it} \AgdaBound{A} \AgdaBound{Δ}\AgdaSymbol{)}\<%
\\
%
\\
\>\AgdaFunction{Σtm-it} \<[8]%
\>[8]\AgdaSymbol{:} \AgdaSymbol{∀\{}\AgdaBound{Γ} \AgdaBound{Δ}\AgdaSymbol{\}(}\AgdaBound{A} \AgdaSymbol{:} \AgdaDatatype{Ty} \AgdaBound{Γ}\AgdaSymbol{)\{}\AgdaBound{B} \AgdaSymbol{:} \AgdaDatatype{Ty} \AgdaBound{Δ}\AgdaSymbol{\}} \AgdaSymbol{→} \AgdaDatatype{Tm} \AgdaBound{B} \<[44]%
\>[44]\<%
\\
\>[0]\AgdaIndent{8}{}\<[8]%
\>[8]\AgdaSymbol{→} \AgdaDatatype{Tm} \AgdaSymbol{(}\AgdaFunction{ΣT-it} \AgdaBound{A} \AgdaBound{B}\AgdaSymbol{)}\<%
\\
\>\<\end{code}
\AgdaHide{
\begin{code}\>\<%
\\
%
\\
\>\AgdaFunction{suspend-cm} \AgdaSymbol{:} \AgdaSymbol{\{}\AgdaBound{Γ} \AgdaBound{Δ} \AgdaBound{Θ} \AgdaSymbol{:} \AgdaDatatype{Con}\AgdaSymbol{\}(}\AgdaBound{A} \AgdaSymbol{:} \AgdaDatatype{Ty} \AgdaBound{Γ}\AgdaSymbol{)} \AgdaSymbol{→} \AgdaBound{Θ} \AgdaDatatype{⇒} \AgdaBound{Δ} \AgdaSymbol{→} \AgdaSymbol{(}\AgdaFunction{ΣC-it} \AgdaBound{A} \AgdaBound{Θ}\AgdaSymbol{)} \AgdaDatatype{⇒} \AgdaSymbol{(}\AgdaFunction{ΣC-it} \AgdaBound{A} \AgdaBound{Δ}\AgdaSymbol{)}\<%
\\
%
\\
\>\AgdaFunction{ΣC-it} \AgdaInductiveConstructor{*} \AgdaBound{Δ} \AgdaSymbol{=} \AgdaBound{Δ}\<%
\\
\>\AgdaFunction{ΣC-it} \AgdaSymbol{(}\AgdaInductiveConstructor{\_=h\_} \AgdaSymbol{\{}\AgdaBound{A}\AgdaSymbol{\}} \AgdaBound{a} \AgdaBound{b}\AgdaSymbol{)} \AgdaBound{Δ} \AgdaSymbol{=} \AgdaFunction{ΣC} \AgdaSymbol{(}\AgdaFunction{ΣC-it} \AgdaBound{A} \AgdaBound{Δ}\AgdaSymbol{)}\<%
\\
%
\\
\>\AgdaFunction{ΣT-it} \AgdaInductiveConstructor{*} \AgdaBound{B} \AgdaSymbol{=} \AgdaBound{B}\<%
\\
\>\AgdaFunction{ΣT-it} \AgdaSymbol{(}\AgdaInductiveConstructor{\_=h\_} \AgdaSymbol{\{}\AgdaBound{A}\AgdaSymbol{\}} \AgdaBound{a} \AgdaBound{b}\AgdaSymbol{)} \AgdaBound{B} \AgdaSymbol{=} \AgdaFunction{ΣT} \AgdaSymbol{(}\AgdaFunction{ΣT-it} \AgdaBound{A} \AgdaBound{B}\AgdaSymbol{)}\<%
\\
\>[0]\AgdaIndent{2}{}\<[2]%
\>[2]\<%
\\
\>\AgdaFunction{Σtm-it} \AgdaInductiveConstructor{*} \AgdaBound{t} \AgdaSymbol{=} \AgdaBound{t}\<%
\\
\>\AgdaFunction{Σtm-it} \AgdaSymbol{(}\AgdaInductiveConstructor{\_=h\_} \AgdaSymbol{\{}\AgdaBound{A}\AgdaSymbol{\}} \AgdaBound{a} \AgdaBound{b}\AgdaSymbol{)} \AgdaBound{t} \AgdaSymbol{=} \AgdaFunction{Σtm} \AgdaSymbol{(}\AgdaFunction{Σtm-it} \AgdaBound{A} \AgdaBound{t}\AgdaSymbol{)}\<%
\\
%
\\
\>\AgdaFunction{suspend-cm} \AgdaInductiveConstructor{*} \AgdaBound{γ} \AgdaSymbol{=} \AgdaBound{γ}\<%
\\
\>\AgdaFunction{suspend-cm} \AgdaSymbol{(}\AgdaInductiveConstructor{\_=h\_} \AgdaSymbol{\{}\AgdaBound{A}\AgdaSymbol{\}} \AgdaBound{a} \AgdaBound{b}\AgdaSymbol{)} \AgdaBound{γ} \AgdaSymbol{=} \AgdaFunction{Σs} \AgdaSymbol{(}\AgdaFunction{suspend-cm} \AgdaBound{A} \AgdaBound{γ}\AgdaSymbol{)}\<%
\\
%
\\
\>\AgdaFunction{minimum-cm} \AgdaSymbol{:} \AgdaSymbol{∀} \AgdaSymbol{\{}\AgdaBound{Γ} \AgdaSymbol{:} \AgdaDatatype{Con}\AgdaSymbol{\}(}\AgdaBound{A} \AgdaSymbol{:} \AgdaDatatype{Ty} \AgdaBound{Γ}\AgdaSymbol{)} \AgdaSymbol{→} \AgdaBound{Γ} \AgdaDatatype{⇒} \AgdaFunction{ΣC-it} \AgdaBound{A} \AgdaInductiveConstructor{ε}\<%
\\
%
\\
\>\AgdaFunction{ΣC-p1} \AgdaSymbol{:\{}\AgdaBound{Γ} \AgdaSymbol{:} \AgdaDatatype{Con}\AgdaSymbol{\}(}\AgdaBound{A} \AgdaSymbol{:} \AgdaDatatype{Ty} \AgdaBound{Γ}\AgdaSymbol{)} \AgdaSymbol{→} \AgdaFunction{ΣC} \AgdaSymbol{(}\AgdaBound{Γ} \AgdaInductiveConstructor{,} \AgdaBound{A}\AgdaSymbol{)} \AgdaDatatype{≡} \AgdaFunction{ΣC} \AgdaBound{Γ} \AgdaInductiveConstructor{,} \AgdaFunction{ΣT} \AgdaBound{A}\<%
\\
\>\AgdaFunction{ΣC-p1} \AgdaInductiveConstructor{*} \AgdaSymbol{=} \AgdaInductiveConstructor{refl}\<%
\\
\>\AgdaFunction{ΣC-p1} \AgdaSymbol{(}\AgdaBound{a} \AgdaInductiveConstructor{=h} \AgdaBound{b}\AgdaSymbol{)} \AgdaSymbol{=} \AgdaInductiveConstructor{refl}\<%
\\
%
\\
%
\\
\>\AgdaFunction{ΣC-it-p1} \AgdaSymbol{:} \AgdaSymbol{\{}\AgdaBound{Γ} \AgdaBound{Δ} \AgdaSymbol{:} \AgdaDatatype{Con}\AgdaSymbol{\}(}\AgdaBound{A} \AgdaSymbol{:} \AgdaDatatype{Ty} \AgdaBound{Γ}\AgdaSymbol{)(}\AgdaBound{B} \AgdaSymbol{:} \AgdaDatatype{Ty} \AgdaBound{Δ}\AgdaSymbol{)} \AgdaSymbol{→} \AgdaFunction{ΣC-it} \AgdaBound{A} \AgdaSymbol{(}\AgdaBound{Δ} \AgdaInductiveConstructor{,} \AgdaBound{B}\AgdaSymbol{)} \AgdaDatatype{≡} \AgdaSymbol{(}\AgdaFunction{ΣC-it} \AgdaBound{A} \AgdaBound{Δ} \AgdaInductiveConstructor{,} \AgdaFunction{ΣT-it} \AgdaBound{A} \AgdaBound{B}\AgdaSymbol{)}\<%
\\
\>\AgdaFunction{ΣC-it-p1} \AgdaInductiveConstructor{*} \AgdaBound{B} \AgdaSymbol{=} \AgdaInductiveConstructor{refl}\<%
\\
\>\AgdaFunction{ΣC-it-p1} \AgdaSymbol{(}\AgdaInductiveConstructor{\_=h\_} \AgdaSymbol{\{}\AgdaBound{A}\AgdaSymbol{\}} \AgdaBound{a} \AgdaBound{b}\AgdaSymbol{)} \AgdaBound{B} \AgdaSymbol{=} \AgdaFunction{cong} \AgdaFunction{ΣC} \AgdaSymbol{(}\AgdaFunction{ΣC-it-p1} \AgdaBound{A} \AgdaBound{B}\AgdaSymbol{)}\<%
\\
%
\\
\>\AgdaComment{-- to split ΣC-it}\<%
\\
%
\\
\>\AgdaFunction{ΣC-it-cm-spl'} \AgdaSymbol{:} \AgdaSymbol{\{}\AgdaBound{Γ} \AgdaBound{Δ} \AgdaSymbol{:} \AgdaDatatype{Con}\AgdaSymbol{\}(}\AgdaBound{A} \AgdaSymbol{:} \AgdaDatatype{Ty} \AgdaBound{Γ}\AgdaSymbol{)(}\AgdaBound{B} \AgdaSymbol{:} \AgdaDatatype{Ty} \AgdaBound{Δ}\AgdaSymbol{)} \AgdaSymbol{→} \<[50]%
\>[50]\<%
\\
\>[0]\AgdaIndent{15}{}\<[15]%
\>[15]\AgdaSymbol{(}\AgdaFunction{ΣC-it} \AgdaBound{A} \AgdaBound{Δ} \AgdaInductiveConstructor{,} \AgdaFunction{ΣT-it} \AgdaBound{A} \AgdaBound{B}\AgdaSymbol{)} \AgdaDatatype{≡} \AgdaFunction{ΣC-it} \AgdaBound{A} \AgdaSymbol{(}\AgdaBound{Δ} \AgdaInductiveConstructor{,} \AgdaBound{B}\AgdaSymbol{)}\<%
\\
\>\AgdaFunction{ΣC-it-cm-spl'} \AgdaInductiveConstructor{*} \AgdaBound{B} \AgdaSymbol{=} \AgdaInductiveConstructor{refl}\<%
\\
\>\AgdaFunction{ΣC-it-cm-spl'} \AgdaSymbol{(}\AgdaInductiveConstructor{\_=h\_} \AgdaSymbol{\{}\AgdaBound{A}\AgdaSymbol{\}} \AgdaBound{a} \AgdaBound{b}\AgdaSymbol{)} \AgdaBound{B} \AgdaSymbol{=} \AgdaFunction{cong} \AgdaFunction{ΣC} \AgdaSymbol{(}\AgdaFunction{ΣC-it-cm-spl'} \AgdaBound{A} \AgdaBound{B}\AgdaSymbol{)}\<%
\\
%
\\
\>\AgdaFunction{ΣC-it-cm-spl} \AgdaSymbol{:} \AgdaSymbol{\{}\AgdaBound{Γ} \AgdaBound{Δ} \AgdaSymbol{:} \AgdaDatatype{Con}\AgdaSymbol{\}(}\AgdaBound{A} \AgdaSymbol{:} \AgdaDatatype{Ty} \AgdaBound{Γ}\AgdaSymbol{)(}\AgdaBound{B} \AgdaSymbol{:} \AgdaDatatype{Ty} \AgdaBound{Δ}\AgdaSymbol{)} \AgdaSymbol{→} \<[49]%
\>[49]\<%
\\
\>[0]\AgdaIndent{15}{}\<[15]%
\>[15]\AgdaSymbol{(}\AgdaFunction{ΣC-it} \AgdaBound{A} \AgdaBound{Δ} \AgdaInductiveConstructor{,} \AgdaFunction{ΣT-it} \AgdaBound{A} \AgdaBound{B}\AgdaSymbol{)} \AgdaDatatype{⇒} \AgdaFunction{ΣC-it} \AgdaBound{A} \AgdaSymbol{(}\AgdaBound{Δ} \AgdaInductiveConstructor{,} \AgdaBound{B}\AgdaSymbol{)}\<%
\\
\>\AgdaFunction{ΣC-it-cm-spl} \AgdaInductiveConstructor{*} \AgdaBound{B} \AgdaSymbol{=} \AgdaFunction{IdCm}\<%
\\
\>\AgdaFunction{ΣC-it-cm-spl} \AgdaSymbol{(}\AgdaInductiveConstructor{\_=h\_} \AgdaSymbol{\{}\AgdaBound{A}\AgdaSymbol{\}} \AgdaBound{a} \AgdaBound{b}\AgdaSymbol{)} \AgdaBound{B} \AgdaSymbol{=} \AgdaFunction{Σs} \AgdaSymbol{(}\AgdaFunction{ΣC-it-cm-spl} \AgdaBound{A} \AgdaBound{B}\AgdaSymbol{)}\<%
\\
%
\\
%
\\
\>\AgdaFunction{ΣC-it-cm-spl-¹} \AgdaSymbol{:} \AgdaSymbol{\{}\AgdaBound{Γ} \AgdaBound{Δ} \AgdaSymbol{:} \AgdaDatatype{Con}\AgdaSymbol{\}(}\AgdaBound{A} \AgdaSymbol{:} \AgdaDatatype{Ty} \AgdaBound{Γ}\AgdaSymbol{)(}\AgdaBound{B} \AgdaSymbol{:} \AgdaDatatype{Ty} \AgdaBound{Δ}\AgdaSymbol{)} \AgdaSymbol{→} \<[51]%
\>[51]\<%
\\
\>[15]\AgdaIndent{16}{}\<[16]%
\>[16]\AgdaFunction{ΣC-it} \AgdaBound{A} \AgdaSymbol{(}\AgdaBound{Δ} \AgdaInductiveConstructor{,} \AgdaBound{B}\AgdaSymbol{)} \AgdaDatatype{⇒} \AgdaSymbol{(}\AgdaFunction{ΣC-it} \AgdaBound{A} \AgdaBound{Δ} \AgdaInductiveConstructor{,} \AgdaFunction{ΣT-it} \AgdaBound{A} \AgdaBound{B}\AgdaSymbol{)}\<%
\\
\>\AgdaFunction{ΣC-it-cm-spl-¹} \AgdaInductiveConstructor{*} \AgdaBound{B} \AgdaSymbol{=} \AgdaFunction{IdCm}\<%
\\
\>\AgdaFunction{ΣC-it-cm-spl-¹} \AgdaSymbol{(}\AgdaInductiveConstructor{\_=h\_} \AgdaSymbol{\{}\AgdaBound{A}\AgdaSymbol{\}} \AgdaBound{a} \AgdaBound{b}\AgdaSymbol{)} \AgdaBound{B} \AgdaSymbol{=} \AgdaFunction{Σs} \AgdaSymbol{(}\AgdaFunction{ΣC-it-cm-spl-¹} \AgdaBound{A} \AgdaBound{B}\AgdaSymbol{)}\<%
\\
%
\\
%
\\
\>\AgdaFunction{ΣC-it-cm-spl2} \AgdaSymbol{:} \AgdaSymbol{\{}\AgdaBound{Γ} \AgdaSymbol{:} \AgdaDatatype{Con}\AgdaSymbol{\}(}\AgdaBound{A} \AgdaSymbol{:} \AgdaDatatype{Ty} \AgdaBound{Γ}\AgdaSymbol{)}\<%
\\
\>[-13]\AgdaIndent{14}{}\<[14]%
\>[14]\AgdaSymbol{→} \AgdaSymbol{(}\AgdaFunction{ΣC-it} \AgdaBound{A} \AgdaInductiveConstructor{ε} \AgdaInductiveConstructor{,} \AgdaFunction{ΣT-it} \AgdaBound{A} \AgdaInductiveConstructor{*} \AgdaInductiveConstructor{,} \<[42]%
\>[42]\AgdaFunction{ΣT-it} \AgdaBound{A} \AgdaInductiveConstructor{*} \AgdaFunction{+T} \AgdaSymbol{\_)} \AgdaDatatype{⇒} \AgdaFunction{ΣC} \AgdaSymbol{(}\AgdaFunction{ΣC-it} \AgdaBound{A} \AgdaInductiveConstructor{ε}\AgdaSymbol{)}\<%
\\
\>\AgdaFunction{ΣC-it-cm-spl2} \AgdaInductiveConstructor{*} \AgdaSymbol{=} \AgdaFunction{IdCm}\<%
\\
\>\AgdaFunction{ΣC-it-cm-spl2} \AgdaSymbol{(}\AgdaInductiveConstructor{\_=h\_} \AgdaSymbol{\{}\AgdaBound{A}\AgdaSymbol{\}} \AgdaBound{a} \AgdaBound{b}\AgdaSymbol{)} \AgdaSymbol{=} \AgdaFunction{Σs} \AgdaSymbol{(}\AgdaFunction{ΣC-it-cm-spl2} \AgdaBound{A}\AgdaSymbol{)} \AgdaFunction{⊚} \AgdaFunction{1-1cm-same} \AgdaSymbol{(}\AgdaFunction{ΣT[+T]} \AgdaSymbol{(}\AgdaFunction{ΣT-it} \AgdaBound{A} \AgdaInductiveConstructor{*}\AgdaSymbol{)} \AgdaSymbol{(}\AgdaFunction{ΣT-it} \AgdaBound{A} \AgdaInductiveConstructor{*}\AgdaSymbol{))}\<%
\\
%
\\
%
\\
\>\AgdaFunction{ΣT-it-wk} \AgdaSymbol{:} \AgdaSymbol{\{}\AgdaBound{Γ} \AgdaBound{Δ} \AgdaSymbol{:} \AgdaDatatype{Con}\AgdaSymbol{\}(}\AgdaBound{A} \AgdaSymbol{:} \AgdaDatatype{Ty} \AgdaBound{Γ}\AgdaSymbol{)(}\AgdaBound{B} \AgdaSymbol{:} \AgdaDatatype{Ty} \AgdaBound{Δ}\AgdaSymbol{)} \AgdaSymbol{→} \AgdaSymbol{(}\AgdaFunction{ΣT-it} \AgdaBound{A} \AgdaInductiveConstructor{*}\AgdaSymbol{)} \AgdaFunction{[} \AgdaFunction{ΣC-it-cm-spl} \AgdaBound{A} \AgdaBound{B} \AgdaFunction{]T} \AgdaDatatype{≡} \AgdaFunction{ΣT-it} \AgdaBound{A} \AgdaInductiveConstructor{*} \AgdaFunction{+T} \AgdaSymbol{\_}\<%
\\
\>\AgdaFunction{ΣT-it-wk} \AgdaInductiveConstructor{*} \AgdaBound{B} \AgdaSymbol{=} \AgdaInductiveConstructor{refl}\<%
\\
\>\AgdaFunction{ΣT-it-wk} \AgdaSymbol{(}\AgdaInductiveConstructor{\_=h\_} \AgdaSymbol{\{}\AgdaBound{A}\AgdaSymbol{\}} \AgdaBound{a} \AgdaBound{b}\AgdaSymbol{)} \AgdaBound{B} \AgdaSymbol{=} \AgdaFunction{trans} \AgdaSymbol{(}\AgdaFunction{ΣT[Σs]T} \AgdaSymbol{(}\AgdaFunction{ΣT-it} \AgdaBound{A} \AgdaInductiveConstructor{*}\AgdaSymbol{)} \AgdaSymbol{(}\AgdaFunction{ΣC-it-cm-spl} \AgdaBound{A} \AgdaBound{B}\AgdaSymbol{))} \AgdaSymbol{(}\AgdaFunction{trans} \AgdaSymbol{(}\AgdaFunction{cong} \AgdaFunction{ΣT} \AgdaSymbol{(}\AgdaFunction{ΣT-it-wk} \AgdaBound{A} \AgdaBound{B}\AgdaSymbol{))} \AgdaSymbol{(}\AgdaFunction{ΣT[+T]} \AgdaSymbol{(}\AgdaFunction{ΣT-it} \AgdaBound{A} \AgdaInductiveConstructor{*}\AgdaSymbol{)} \AgdaSymbol{(}\AgdaFunction{ΣT-it} \AgdaBound{A} \AgdaBound{B}\AgdaSymbol{)))}\<%
\\
%
\\
\>\AgdaFunction{ΣT-it-p1} \AgdaSymbol{:} \AgdaSymbol{∀} \AgdaSymbol{\{}\AgdaBound{Γ} \AgdaSymbol{:} \AgdaDatatype{Con}\AgdaSymbol{\}(}\AgdaBound{A} \AgdaSymbol{:} \AgdaDatatype{Ty} \AgdaBound{Γ}\AgdaSymbol{)} \AgdaSymbol{→} \AgdaFunction{ΣT-it} \AgdaBound{A} \AgdaInductiveConstructor{*} \AgdaFunction{[} \AgdaFunction{minimum-cm} \AgdaBound{A} \AgdaFunction{]T} \AgdaDatatype{≡} \AgdaBound{A}\<%
\\
%
\\
\>\AgdaFunction{ΣT-it-p2} \AgdaSymbol{:} \AgdaSymbol{\{}\AgdaBound{Γ} \AgdaBound{Δ} \AgdaSymbol{:} \AgdaDatatype{Con}\AgdaSymbol{\}(}\AgdaBound{A} \AgdaSymbol{:} \AgdaDatatype{Ty} \AgdaBound{Γ}\AgdaSymbol{)\{}\AgdaBound{B} \AgdaSymbol{:} \AgdaDatatype{Ty} \AgdaBound{Δ}\AgdaSymbol{\}\{}\AgdaBound{a} \AgdaBound{b} \AgdaSymbol{:} \AgdaDatatype{Tm} \AgdaBound{B}\AgdaSymbol{\}} \AgdaSymbol{→} \AgdaFunction{ΣT-it} \AgdaBound{A} \AgdaSymbol{(}\AgdaBound{a} \AgdaInductiveConstructor{=h} \AgdaBound{b}\AgdaSymbol{)} \AgdaDatatype{≡} \AgdaSymbol{(}\AgdaFunction{Σtm-it} \AgdaBound{A} \AgdaBound{a} \AgdaInductiveConstructor{=h} \AgdaFunction{Σtm-it} \AgdaBound{A} \AgdaBound{b}\AgdaSymbol{)}\<%
\\
\>\AgdaFunction{ΣT-it-p2} \AgdaInductiveConstructor{*} \AgdaSymbol{=} \AgdaInductiveConstructor{refl}\<%
\\
\>\AgdaFunction{ΣT-it-p2} \AgdaSymbol{(}\AgdaInductiveConstructor{\_=h\_} \AgdaSymbol{\{}\AgdaBound{A}\AgdaSymbol{\}} \AgdaSymbol{\_} \AgdaSymbol{\_)} \AgdaSymbol{=} \AgdaFunction{cong} \AgdaFunction{ΣT} \AgdaSymbol{(}\AgdaFunction{ΣT-it-p2} \AgdaBound{A}\AgdaSymbol{)}\<%
\\
%
\\
%
\\
\>\AgdaFunction{ΣT-it-p3} \AgdaSymbol{:} \AgdaSymbol{\{}\AgdaBound{Γ} \AgdaBound{Δ} \AgdaSymbol{:} \AgdaDatatype{Con}\AgdaSymbol{\}(}\AgdaBound{A} \AgdaSymbol{:} \AgdaDatatype{Ty} \AgdaBound{Γ}\AgdaSymbol{)\{}\AgdaBound{B} \AgdaBound{C} \AgdaSymbol{:} \AgdaDatatype{Ty} \AgdaBound{Δ}\AgdaSymbol{\}} \AgdaSymbol{→} \AgdaFunction{ΣT-it} \AgdaBound{A} \AgdaSymbol{(}\AgdaBound{C} \AgdaFunction{+T} \AgdaBound{B}\AgdaSymbol{)} \AgdaFunction{[} \AgdaFunction{ΣC-it-cm-spl} \AgdaBound{A} \AgdaBound{B} \AgdaFunction{]T} \AgdaDatatype{≡} \AgdaFunction{ΣT-it} \AgdaBound{A} \AgdaBound{C} \AgdaFunction{+T} \AgdaSymbol{\_} \<[103]%
\>[103]\<%
\\
\>\AgdaFunction{ΣT-it-p3} \AgdaInductiveConstructor{*} \AgdaSymbol{=} \AgdaFunction{trans} \AgdaFunction{+T[,]T} \AgdaSymbol{(}\AgdaFunction{wk+S+T} \AgdaFunction{IC-T}\AgdaSymbol{)}\<%
\\
\>\AgdaFunction{ΣT-it-p3} \AgdaSymbol{(}\AgdaInductiveConstructor{\_=h\_} \AgdaSymbol{\{}\AgdaBound{A}\AgdaSymbol{\}} \AgdaBound{a} \AgdaBound{b}\AgdaSymbol{)} \AgdaSymbol{\{}\AgdaBound{B}\AgdaSymbol{\}} \AgdaSymbol{\{}\AgdaBound{C}\AgdaSymbol{\}} \AgdaSymbol{=} \AgdaFunction{trans} \AgdaSymbol{(}\AgdaFunction{ΣT[Σs]T} \AgdaSymbol{(}\AgdaFunction{ΣT-it} \AgdaBound{A} \AgdaSymbol{(}\AgdaBound{C} \AgdaFunction{+T} \AgdaBound{B}\AgdaSymbol{))} \AgdaSymbol{(}\AgdaFunction{ΣC-it-cm-spl} \AgdaBound{A} \AgdaBound{B}\AgdaSymbol{))} \AgdaSymbol{(}\AgdaFunction{trans} \AgdaSymbol{(}\AgdaFunction{cong} \AgdaFunction{ΣT} \AgdaSymbol{(}\AgdaFunction{ΣT-it-p3} \AgdaBound{A}\AgdaSymbol{))} \AgdaSymbol{(}\AgdaFunction{ΣT[+T]} \AgdaSymbol{(}\AgdaFunction{ΣT-it} \AgdaBound{A} \AgdaBound{C}\AgdaSymbol{)} \AgdaSymbol{(}\AgdaFunction{ΣT-it} \AgdaBound{A} \AgdaBound{B}\AgdaSymbol{)))}\<%
\\
%
\\
%
\\
\>\AgdaFunction{minimum-cm} \AgdaInductiveConstructor{*} \AgdaSymbol{=} \AgdaInductiveConstructor{•}\<%
\\
\>\AgdaFunction{minimum-cm} \AgdaSymbol{\{}\AgdaBound{Γ}\AgdaSymbol{\}} \AgdaSymbol{(}\AgdaInductiveConstructor{\_=h\_} \AgdaSymbol{\{}\AgdaBound{A}\AgdaSymbol{\}} \AgdaBound{a} \AgdaBound{b}\AgdaSymbol{)} \AgdaSymbol{=} \AgdaFunction{ΣC-it-cm-spl2} \AgdaBound{A} \AgdaFunction{⊚} \AgdaSymbol{((}\AgdaFunction{minimum-cm} \AgdaBound{A} \AgdaInductiveConstructor{,} \AgdaSymbol{(}\AgdaBound{a} \AgdaFunction{⟦} \AgdaFunction{ΣT-it-p1} \AgdaBound{A} \AgdaFunction{⟫}\AgdaSymbol{))} \AgdaInductiveConstructor{,} \AgdaSymbol{(}\AgdaFunction{wk-tm} \AgdaSymbol{(}\AgdaBound{b} \AgdaFunction{⟦} \AgdaFunction{ΣT-it-p1} \AgdaBound{A} \AgdaFunction{⟫}\AgdaSymbol{)))}\<%
\\
%
\\
%
\\
\>\AgdaFunction{ΣC-it-ε-Contr} \AgdaSymbol{:} \<[17]%
\>[17]\AgdaSymbol{∀} \AgdaSymbol{\{}\AgdaBound{Γ} \AgdaBound{Δ} \AgdaSymbol{:} \AgdaDatatype{Con}\AgdaSymbol{\}(}\AgdaBound{A} \AgdaSymbol{:} \AgdaDatatype{Ty} \AgdaBound{Γ}\AgdaSymbol{)} \AgdaSymbol{→} \AgdaDatatype{isContr} \AgdaBound{Δ} \AgdaSymbol{→} \AgdaDatatype{isContr} \AgdaSymbol{(}\AgdaFunction{ΣC-it} \AgdaBound{A} \AgdaBound{Δ}\AgdaSymbol{)}\<%
\\
\>\AgdaFunction{ΣC-it-ε-Contr} \AgdaInductiveConstructor{*} \AgdaBound{isC} \AgdaSymbol{=} \AgdaBound{isC}\<%
\\
\>\AgdaFunction{ΣC-it-ε-Contr} \AgdaSymbol{(}\AgdaInductiveConstructor{\_=h\_} \AgdaSymbol{\{}\AgdaBound{A}\AgdaSymbol{\}} \AgdaBound{a} \AgdaBound{b}\AgdaSymbol{)} \AgdaBound{isC} \AgdaSymbol{=} \AgdaFunction{ΣC-Contr} \AgdaSymbol{\_} \AgdaSymbol{(}\AgdaFunction{ΣC-it-ε-Contr} \AgdaBound{A} \AgdaBound{isC}\AgdaSymbol{)}\<%
\\
%
\\
%
\\
\>\AgdaFunction{wk-susp} \AgdaSymbol{:} \AgdaSymbol{∀} \AgdaSymbol{\{}\AgdaBound{Γ} \AgdaSymbol{:} \AgdaDatatype{Con}\AgdaSymbol{\}(}\AgdaBound{A} \AgdaSymbol{:} \AgdaDatatype{Ty} \AgdaBound{Γ}\AgdaSymbol{)(}\AgdaBound{a} \AgdaSymbol{:} \AgdaDatatype{Tm} \AgdaBound{A}\AgdaSymbol{)} \AgdaSymbol{→} \AgdaBound{a} \AgdaFunction{⟦} \AgdaFunction{ΣT-it-p1} \AgdaBound{A} \AgdaFunction{⟫} \AgdaDatatype{≅} \AgdaBound{a}\<%
\\
\>\AgdaFunction{wk-susp} \AgdaBound{A} \AgdaBound{a} \AgdaSymbol{=} \AgdaFunction{cohOp} \AgdaSymbol{(}\AgdaFunction{ΣT-it-p1} \AgdaBound{A}\AgdaSymbol{)}\<%
\\
%
\\
\>\AgdaFunction{fci-l1} \AgdaSymbol{:} \AgdaSymbol{∀} \AgdaSymbol{\{}\AgdaBound{Γ} \AgdaSymbol{:} \AgdaDatatype{Con}\AgdaSymbol{\}(}\AgdaBound{A} \AgdaSymbol{:} \AgdaDatatype{Ty} \AgdaBound{Γ}\AgdaSymbol{)} \AgdaSymbol{→} \AgdaFunction{ΣT} \AgdaSymbol{(}\AgdaFunction{ΣT-it} \AgdaBound{A} \AgdaInductiveConstructor{*}\AgdaSymbol{)} \AgdaFunction{[} \AgdaFunction{ΣC-it-cm-spl2} \AgdaBound{A} \AgdaFunction{]T} \AgdaDatatype{≡} \AgdaSymbol{(}\AgdaInductiveConstructor{var} \AgdaSymbol{(}\AgdaInductiveConstructor{vS} \AgdaInductiveConstructor{v0}\AgdaSymbol{)} \AgdaInductiveConstructor{=h} \AgdaInductiveConstructor{var} \AgdaInductiveConstructor{v0}\AgdaSymbol{)}\<%
\\
%
\\
\>\AgdaFunction{fci-l1} \AgdaInductiveConstructor{*} \AgdaSymbol{=} \AgdaInductiveConstructor{refl}\<%
\\
\>\AgdaFunction{fci-l1} \AgdaSymbol{\{}\AgdaBound{Γ}\AgdaSymbol{\}} \AgdaSymbol{(}\AgdaInductiveConstructor{\_=h\_} \AgdaSymbol{\{}\AgdaBound{A}\AgdaSymbol{\}} \AgdaBound{a} \AgdaBound{b}\AgdaSymbol{)} \AgdaSymbol{=} \AgdaFunction{trans} \AgdaFunction{[⊚]T} \AgdaSymbol{(}\AgdaFunction{trans}\<%
\\
\>[0]\AgdaIndent{38}{}\<[38]%
\>[38]\AgdaSymbol{(}\AgdaFunction{congT}\<%
\\
\>[38]\AgdaIndent{39}{}\<[39]%
\>[39]\AgdaSymbol{(}\AgdaFunction{trans} \AgdaSymbol{(}\AgdaFunction{ΣT[Σs]T} \AgdaSymbol{(}\AgdaFunction{ΣT} \AgdaSymbol{(}\AgdaFunction{ΣT-it} \AgdaBound{A} \AgdaInductiveConstructor{*}\AgdaSymbol{))} \AgdaSymbol{(}\AgdaFunction{ΣC-it-cm-spl2} \AgdaBound{A}\AgdaSymbol{))}\<%
\\
\>[39]\AgdaIndent{40}{}\<[40]%
\>[40]\AgdaSymbol{(}\AgdaFunction{cong} \AgdaFunction{ΣT} \AgdaSymbol{(}\AgdaFunction{fci-l1} \AgdaBound{A}\AgdaSymbol{))))}\<%
\\
\>[-37]\AgdaIndent{38}{}\<[38]%
\>[38]\AgdaSymbol{(}\AgdaFunction{hom≡}\<%
\\
\>[0]\AgdaIndent{41}{}\<[41]%
\>[41]\AgdaSymbol{(}\AgdaFunction{congtm} \AgdaSymbol{(}\AgdaFunction{Σtm-p2-sp} \AgdaSymbol{(}\AgdaFunction{ΣT-it} \AgdaBound{A} \AgdaInductiveConstructor{*}\AgdaSymbol{)} \AgdaSymbol{(}\AgdaFunction{ΣT-it} \AgdaBound{A} \AgdaInductiveConstructor{*} \AgdaFunction{+T} \AgdaFunction{ΣT-it} \AgdaBound{A} \AgdaInductiveConstructor{*}\AgdaSymbol{))} \AgdaFunction{∾}\<%
\\
\>[41]\AgdaIndent{42}{}\<[42]%
\>[42]\AgdaFunction{1-1cm-same-tm} \AgdaSymbol{(}\AgdaFunction{ΣT[+T]} \AgdaSymbol{(}\AgdaFunction{ΣT-it} \AgdaBound{A} \AgdaInductiveConstructor{*}\AgdaSymbol{)} \AgdaSymbol{(}\AgdaFunction{ΣT-it} \AgdaBound{A} \AgdaInductiveConstructor{*}\AgdaSymbol{))} \AgdaSymbol{(}\AgdaInductiveConstructor{var} \AgdaInductiveConstructor{v0}\AgdaSymbol{))}\<%
\\
\>[0]\AgdaIndent{41}{}\<[41]%
\>[41]\AgdaSymbol{(}\AgdaFunction{congtm} \AgdaSymbol{(}\AgdaFunction{Σtm-p1} \AgdaSymbol{(}\AgdaFunction{ΣT-it} \AgdaBound{A} \AgdaInductiveConstructor{*} \AgdaFunction{+T} \AgdaFunction{ΣT-it} \AgdaBound{A} \AgdaInductiveConstructor{*}\AgdaSymbol{))} \AgdaFunction{∾}\<%
\\
\>[0]\AgdaIndent{44}{}\<[44]%
\>[44]\AgdaFunction{1-1cm-same-v0} \AgdaSymbol{(}\AgdaFunction{ΣT[+T]} \AgdaSymbol{(}\AgdaFunction{ΣT-it} \AgdaBound{A} \AgdaInductiveConstructor{*}\AgdaSymbol{)} \AgdaSymbol{(}\AgdaFunction{ΣT-it} \AgdaBound{A} \AgdaInductiveConstructor{*}\AgdaSymbol{))))} \AgdaSymbol{)}\<%
\\
%
\\
\>\AgdaFunction{ΣT-it-p1} \<[10]%
\>[10]\AgdaInductiveConstructor{*} \AgdaSymbol{=} \AgdaInductiveConstructor{refl}\<%
\\
\>\AgdaFunction{ΣT-it-p1} \AgdaSymbol{(}\AgdaInductiveConstructor{\_=h\_} \AgdaSymbol{\{}\AgdaBound{A}\AgdaSymbol{\}} \AgdaBound{a} \AgdaBound{b}\AgdaSymbol{)} \AgdaSymbol{=} \AgdaFunction{trans} \AgdaFunction{[⊚]T} \AgdaSymbol{(}\AgdaFunction{trans} \AgdaSymbol{(}\AgdaFunction{congT} \AgdaSymbol{(}\AgdaFunction{fci-l1} \AgdaBound{A}\AgdaSymbol{))} \AgdaSymbol{(}\AgdaFunction{hom≡} \AgdaSymbol{(}\AgdaFunction{prf} \AgdaBound{a}\AgdaSymbol{)} \AgdaSymbol{(}\AgdaFunction{prf} \AgdaBound{b}\AgdaSymbol{)))}\<%
\\
\>[0]\AgdaIndent{2}{}\<[2]%
\>[2]\AgdaKeyword{where}\<%
\\
\>[0]\AgdaIndent{4}{}\<[4]%
\>[4]\AgdaFunction{prf} \AgdaSymbol{:} \AgdaSymbol{(}\AgdaBound{a} \AgdaSymbol{:} \AgdaDatatype{Tm} \AgdaBound{A}\AgdaSymbol{)} \AgdaSymbol{→} \AgdaSymbol{((}\AgdaBound{a} \AgdaFunction{⟦} \AgdaFunction{ΣT-it-p1} \AgdaBound{A} \AgdaFunction{⟫}\AgdaSymbol{)} \AgdaFunction{⟦} \AgdaFunction{+T[,]T} \AgdaFunction{⟫}\AgdaSymbol{)} \AgdaFunction{⟦} \AgdaFunction{+T[,]T} \AgdaFunction{⟫} \AgdaDatatype{≅} \AgdaBound{a}\<%
\\
\>[0]\AgdaIndent{4}{}\<[4]%
\>[4]\AgdaFunction{prf} \AgdaBound{a} \AgdaSymbol{=} \AgdaFunction{wk-coh} \AgdaFunction{∾} \AgdaFunction{wk-coh} \AgdaFunction{∾} \AgdaFunction{wk-susp} \AgdaBound{A} \AgdaBound{a}\<%
\\
\>[0]\AgdaIndent{1}{}\<[1]%
\>[1]\<%
\\
%
\\
%
\\
\>\AgdaFunction{Σtm-it-p1} \AgdaSymbol{:} \AgdaSymbol{\{}\AgdaBound{Γ} \AgdaBound{Δ} \AgdaSymbol{:} \AgdaDatatype{Con}\AgdaSymbol{\}(}\AgdaBound{A} \AgdaSymbol{:} \AgdaDatatype{Ty} \AgdaBound{Γ}\AgdaSymbol{)\{}\AgdaBound{B} \AgdaSymbol{:} \AgdaDatatype{Ty} \AgdaBound{Δ}\AgdaSymbol{\}} \AgdaSymbol{→} \AgdaFunction{Σtm-it} \AgdaBound{A} \AgdaSymbol{(}\AgdaInductiveConstructor{var} \AgdaInductiveConstructor{v0}\AgdaSymbol{)} \AgdaFunction{[} \AgdaFunction{ΣC-it-cm-spl} \AgdaBound{A} \AgdaBound{B} \AgdaFunction{]tm} \AgdaDatatype{≅} \AgdaInductiveConstructor{var} \AgdaInductiveConstructor{v0}\<%
\\
\>\AgdaFunction{Σtm-it-p1} \AgdaInductiveConstructor{*} \AgdaSymbol{\{}\AgdaBound{B}\AgdaSymbol{\}} \AgdaSymbol{=} \AgdaFunction{wk-coh} \AgdaFunction{∾} \AgdaFunction{cohOp} \AgdaSymbol{(}\AgdaFunction{wk+S+T} \AgdaFunction{IC-T}\AgdaSymbol{)}\<%
\\
\>\AgdaFunction{Σtm-it-p1} \AgdaSymbol{(}\AgdaInductiveConstructor{\_=h\_} \AgdaSymbol{\{}\AgdaBound{A}\AgdaSymbol{\}} \AgdaBound{a} \AgdaBound{b}\AgdaSymbol{)} \AgdaSymbol{\{}\AgdaBound{B}\AgdaSymbol{\}} \AgdaSymbol{=} \AgdaFunction{Σtm[Σs]tm} \AgdaSymbol{(}\AgdaFunction{Σtm-it} \AgdaBound{A} \AgdaSymbol{(}\AgdaInductiveConstructor{var} \AgdaInductiveConstructor{v0}\AgdaSymbol{))} \AgdaSymbol{(}\AgdaFunction{ΣC-it-cm-spl} \AgdaBound{A} \AgdaBound{B}\AgdaSymbol{)} \AgdaFunction{∾} \AgdaFunction{congΣtm} \AgdaSymbol{(}\AgdaFunction{Σtm-it-p1} \AgdaBound{A}\AgdaSymbol{)} \AgdaFunction{∾} \AgdaFunction{cohOpV} \AgdaSymbol{(}\AgdaFunction{sym} \AgdaSymbol{(}\AgdaFunction{ΣT[+T]} \AgdaSymbol{(}\AgdaFunction{ΣT-it} \AgdaBound{A} \AgdaBound{B}\AgdaSymbol{)} \AgdaSymbol{(}\AgdaFunction{ΣT-it} \AgdaBound{A} \AgdaBound{B}\AgdaSymbol{)))}\<%
\\
%
\\
%
\\
%
\\
\>\AgdaFunction{Σtm-it-p2} \AgdaSymbol{:} \AgdaSymbol{\{}\AgdaBound{Γ} \AgdaBound{Δ} \AgdaSymbol{:} \AgdaDatatype{Con}\AgdaSymbol{\}(}\AgdaBound{A} \AgdaSymbol{:} \AgdaDatatype{Ty} \AgdaBound{Γ}\AgdaSymbol{)\{}\AgdaBound{B} \AgdaBound{C} \AgdaSymbol{:} \AgdaDatatype{Ty} \AgdaBound{Δ}\AgdaSymbol{\}(}\AgdaBound{x} \AgdaSymbol{:} \AgdaDatatype{Var} \AgdaBound{B}\AgdaSymbol{)} \AgdaSymbol{→} \AgdaSymbol{(}\AgdaFunction{Σtm-it} \AgdaBound{A} \AgdaSymbol{(}\AgdaInductiveConstructor{var} \AgdaSymbol{(}\AgdaInductiveConstructor{vS} \AgdaBound{x}\AgdaSymbol{)))} \AgdaFunction{[} \AgdaFunction{ΣC-it-cm-spl} \AgdaBound{A} \AgdaBound{C} \AgdaFunction{]tm} \AgdaDatatype{≅} \AgdaFunction{Σtm-it} \AgdaBound{A} \AgdaSymbol{(}\AgdaInductiveConstructor{var} \AgdaBound{x}\AgdaSymbol{)} \AgdaFunction{+tm} \AgdaSymbol{\_}\<%
\\
\>\AgdaFunction{Σtm-it-p2} \AgdaInductiveConstructor{*} \AgdaBound{x} \AgdaSymbol{=} \AgdaFunction{wk-coh} \AgdaFunction{∾} \AgdaFunction{[+S]V} \AgdaBound{x} \AgdaFunction{∾} \AgdaFunction{cong+tm} \AgdaSymbol{(}\AgdaFunction{IC-v} \AgdaBound{x}\AgdaSymbol{)}\<%
\\
\>\AgdaFunction{Σtm-it-p2} \AgdaSymbol{\{}\AgdaBound{Γ}\AgdaSymbol{\}} \AgdaSymbol{\{}\AgdaBound{Δ}\AgdaSymbol{\}} \AgdaSymbol{(}\AgdaInductiveConstructor{\_=h\_} \AgdaSymbol{\{}\AgdaBound{A}\AgdaSymbol{\}} \AgdaBound{a} \AgdaBound{b}\AgdaSymbol{)} \AgdaSymbol{\{}\AgdaBound{B}\AgdaSymbol{\}} \AgdaSymbol{\{}\AgdaBound{C}\AgdaSymbol{\}} \AgdaBound{x} \AgdaSymbol{=} \AgdaFunction{Σtm[Σs]tm} \AgdaSymbol{(}\AgdaFunction{Σtm-it} \AgdaBound{A} \AgdaSymbol{(}\AgdaInductiveConstructor{var} \AgdaSymbol{(}\AgdaInductiveConstructor{vS} \AgdaBound{x}\AgdaSymbol{)))} \AgdaSymbol{(}\AgdaFunction{ΣC-it-cm-spl} \AgdaBound{A} \AgdaBound{C}\AgdaSymbol{)} \AgdaFunction{∾}\<%
\\
\>[0]\AgdaIndent{47}{}\<[47]%
\>[47]\AgdaFunction{congΣtm} \AgdaSymbol{(}\AgdaFunction{Σtm-it-p2} \AgdaSymbol{\{}\AgdaBound{Γ}\AgdaSymbol{\}} \AgdaSymbol{\{}\AgdaBound{Δ}\AgdaSymbol{\}} \AgdaBound{A} \AgdaSymbol{\{}\AgdaBound{B}\AgdaSymbol{\}} \AgdaBound{x}\AgdaSymbol{)} \AgdaFunction{∾}\<%
\\
\>[0]\AgdaIndent{47}{}\<[47]%
\>[47]\AgdaFunction{Σtm[+tm]} \AgdaSymbol{(}\AgdaFunction{Σtm-it} \AgdaBound{A} \AgdaSymbol{(}\AgdaInductiveConstructor{var} \AgdaBound{x}\AgdaSymbol{))} \AgdaSymbol{(}\AgdaFunction{ΣT-it} \AgdaBound{A} \AgdaBound{C}\AgdaSymbol{)}\<%
\\
\>\<\end{code}
}
\noindent Finally, it is clear that iterated suspension preserves contractibility. 

\begin{code}\>\<%
\\
\>\AgdaFunction{ΣC-it-Contr} \<[13]%
\>[13]\AgdaSymbol{:} \AgdaSymbol{∀} \AgdaSymbol{\{}\AgdaBound{Γ} \AgdaBound{Δ}\AgdaSymbol{\}(}\AgdaBound{A} \AgdaSymbol{:} \AgdaDatatype{Ty} \AgdaBound{Γ}\AgdaSymbol{)} \AgdaSymbol{→} \AgdaDatatype{isContr} \AgdaBound{Δ} \<[45]%
\>[45]\<%
\\
\>[0]\AgdaIndent{13}{}\<[13]%
\>[13]\AgdaSymbol{→} \AgdaDatatype{isContr} \AgdaSymbol{(}\AgdaFunction{ΣC-it} \AgdaBound{A} \AgdaBound{Δ}\AgdaSymbol{)}\<%
\\
\>\<\end{code}
\AgdaHide{
\begin{code}\>\<%
\\
\>\AgdaFunction{ΣC-it-Contr} \AgdaInductiveConstructor{*} \AgdaBound{x} \AgdaSymbol{=} \AgdaBound{x}\<%
\\
\>\AgdaFunction{ΣC-it-Contr} \AgdaSymbol{\{}\AgdaBound{Γ}\AgdaSymbol{\}\{}\AgdaBound{Δ}\AgdaSymbol{\}(}\AgdaInductiveConstructor{\_=h\_} \AgdaSymbol{\{}\AgdaBound{A}\AgdaSymbol{\}} \AgdaBound{a} \AgdaBound{b}\AgdaSymbol{)} \AgdaBound{x} \AgdaSymbol{=} \AgdaFunction{ΣC-Contr} \AgdaSymbol{(}\AgdaFunction{ΣC-it} \AgdaBound{A} \AgdaBound{Δ}\AgdaSymbol{)} \AgdaSymbol{(}\AgdaFunction{ΣC-it-Contr} \AgdaBound{A} \AgdaBound{x}\AgdaSymbol{)} \<[76]%
\>[76]\<%
\\
\>\<\end{code}
}
By suspending the minimal contractible context,
*, we obtain a so-called \emph{span}. They are stalks with a top variable added. For example $(x_0: *)$ (the one-variable
context) for $n=0$; $(x_0 : *, x_1 : *, x_2 : x_0\,=_\mathsf{h}\,x_1)$ for
$n=1$; $(x_0 : *, x_1 : *, x_2 : x_0\,=_\mathsf{h}\,x_1, x_3 :
x_0\,=_{\mathsf{h}}\,x_1, x_4 : x_2\,=_\mathsf{h}\,x_3)$ for $n=2$, etc. 
Spans play
an important role later in the definition of composition. 
% Following is a picture of the first few spans for increasing levels $n$ of \AgdaBound{A}.
% \[
% \begin{array}{c@{\hspace{1.5cm}} c@{\hspace{1.5cm}} c@{\hspace{1.5cm}} c@{\hspace{1.5cm}} c@{\hspace{1.5cm}}}
% &&&&8\\
% &&&6&6\quad 7\\
% &&4&4\quad 5&4 \quad 5\\
% &2&2\quad 3&2\quad 3&2\quad 3\\
% 0&0\quad 1&0\quad 1&0\quad 1&0\quad 1\\
% \\
% n = 0&n=1&n=2&n=3&n=4
% \end{array}
% \]

\subsubsection{Replacement}
\label{sec:replacement}

After we have suspended a context by inserting an appropriate number of
variables, we may proceed to a substitution which fills the stalk for
$A$ with $A$. The context morphism representing this substitution is
called $\AgdaFunction{filter}$. In the final step we combine it with
$\Gamma$, the context of $A$.  The new context contains two parts, the
first is the same as $\Gamma$, and the second is the suspended $\Delta$
substituted by $\AgdaFunction{filter}$. However, we also have to drop
the stalk of $A$ becuse it already exists in $\Gamma$.

This operation is called \emph{replacement} because we can interpret it as replacing $*$ in $\Delta$ by
$A$.
Geometrically speaking, the resulting context is a new context which corresponds to the pasting of
$\Delta$ to $\Gamma$ at $A$.

As always, we define replacement for contexts, types and terms:

\begin{code}\>\<%
\\
\>\AgdaFunction{rpl-C} \<[8]%
\>[8]\AgdaSymbol{:} \AgdaSymbol{∀\{}\AgdaBound{Γ}\AgdaSymbol{\}(}\AgdaBound{A} \AgdaSymbol{:} \AgdaDatatype{Ty} \AgdaBound{Γ}\AgdaSymbol{)} \AgdaSymbol{→} \AgdaDatatype{Con} \AgdaSymbol{→} \AgdaDatatype{Con}\<%
\\
\>\AgdaFunction{rpl-T} \<[8]%
\>[8]\AgdaSymbol{:} \AgdaSymbol{∀\{}\AgdaBound{Γ} \AgdaBound{Δ}\AgdaSymbol{\}(}\AgdaBound{A} \AgdaSymbol{:} \AgdaDatatype{Ty} \AgdaBound{Γ}\AgdaSymbol{)} \AgdaSymbol{→} \AgdaDatatype{Ty} \AgdaBound{Δ} \AgdaSymbol{→} \AgdaDatatype{Ty} \AgdaSymbol{(}\AgdaFunction{rpl-C} \AgdaBound{A} \AgdaBound{Δ}\AgdaSymbol{)}\<%
\\
\>\AgdaFunction{rpl-tm} \<[8]%
\>[8]\AgdaSymbol{:} \AgdaSymbol{∀\{}\AgdaBound{Γ} \AgdaBound{Δ}\AgdaSymbol{\}(}\AgdaBound{A} \AgdaSymbol{:} \AgdaDatatype{Ty} \AgdaBound{Γ}\AgdaSymbol{)\{}\AgdaBound{B} \AgdaSymbol{:} \AgdaDatatype{Ty} \AgdaBound{Δ}\AgdaSymbol{\}} \AgdaSymbol{→} \AgdaDatatype{Tm} \AgdaBound{B} \<[44]%
\>[44]\<%
\\
\>[0]\AgdaIndent{8}{}\<[8]%
\>[8]\AgdaSymbol{→} \AgdaDatatype{Tm} \AgdaSymbol{(}\AgdaFunction{rpl-T} \AgdaBound{A} \AgdaBound{B}\AgdaSymbol{)}\<%
\\
\>\<\end{code}
Replacement for contexts, $\AgdaFunction{rpl-C}$, defines for a type $A$ in $\Gamma$ and another context $\Delta$ 
a context which begins as $\Gamma$ and follows by each type of $\Delta$ with $*$ replaced with (pasted onto)  $A$. 

\begin{code}\>\<%
\\
\>\AgdaFunction{rpl-C} \AgdaSymbol{\{}\AgdaBound{Γ}\AgdaSymbol{\}} \AgdaBound{A} \AgdaInductiveConstructor{ε} \<[17]%
\>[17]\AgdaSymbol{=} \AgdaBound{Γ}\<%
\\
\>\AgdaFunction{rpl-C} \AgdaBound{A} \AgdaSymbol{(}\AgdaBound{Δ} \AgdaInductiveConstructor{,} \AgdaBound{B}\AgdaSymbol{)} \<[17]%
\>[17]\AgdaSymbol{=} \AgdaFunction{rpl-C} \AgdaBound{A} \AgdaBound{Δ} \AgdaInductiveConstructor{,} \AgdaFunction{rpl-T} \AgdaBound{A} \AgdaBound{B}\<%
\\
\>\<\end{code}
\noindent To this end we must define the substitution $\AgdaFunction{filter}$ which
pulls back each type from suspended $\Delta$ to the new context. 

\begin{code}\>\<%
\\
\>\AgdaFunction{filter} \<[8]%
\>[8]\AgdaSymbol{:} \AgdaSymbol{∀\{}\AgdaBound{Γ}\AgdaSymbol{\}(}\AgdaBound{Δ} \AgdaSymbol{:} \AgdaDatatype{Con}\AgdaSymbol{)(}\AgdaBound{A} \AgdaSymbol{:} \AgdaDatatype{Ty} \AgdaBound{Γ}\AgdaSymbol{)} \<[34]%
\>[34]\<%
\\
\>[0]\AgdaIndent{8}{}\<[8]%
\>[8]\AgdaSymbol{→} \AgdaFunction{rpl-C} \AgdaBound{A} \AgdaBound{Δ} \AgdaDatatype{⇒} \AgdaFunction{ΣC-it} \AgdaBound{A} \AgdaBound{Δ}\<%
\\
%
\\
\>\AgdaFunction{rpl-T} \AgdaBound{A} \AgdaBound{B} \AgdaSymbol{=} \AgdaFunction{ΣT-it} \AgdaBound{A} \AgdaBound{B} \AgdaFunction{[} \AgdaFunction{filter} \AgdaSymbol{\_} \AgdaBound{A} \AgdaFunction{]T}\<%
\\
\>\<\end{code}

\AgdaHide{
\begin{code}\>\<%
\\
\>\AgdaFunction{rpl-pr1} \<[9]%
\>[9]\AgdaSymbol{:} \AgdaSymbol{\{}\AgdaBound{Γ} \AgdaSymbol{:} \AgdaDatatype{Con}\AgdaSymbol{\}(}\AgdaBound{Δ} \AgdaSymbol{:} \AgdaDatatype{Con}\AgdaSymbol{)(}\AgdaBound{A} \AgdaSymbol{:} \AgdaDatatype{Ty} \AgdaBound{Γ}\AgdaSymbol{)} \AgdaSymbol{→} \AgdaFunction{rpl-C} \AgdaBound{A} \AgdaBound{Δ} \AgdaDatatype{⇒} \AgdaBound{Γ}\<%
\\
%
\\
\>\AgdaComment{\{-
filter : \{Γ Δ Θ : Con\}(A : Ty Γ) → Θ ⇒ Δ → (rpl-C A Θ) ⇒ (rpl-C A Δ)
-\}}\<%
\\
%
\\
\>\AgdaFunction{rpl-pr1} \AgdaInductiveConstructor{ε} \AgdaBound{A} \AgdaSymbol{=} \AgdaFunction{IdCm}\<%
\\
\>\AgdaFunction{rpl-pr1} \AgdaSymbol{(}\AgdaBound{Δ} \AgdaInductiveConstructor{,} \AgdaBound{A}\AgdaSymbol{)} \AgdaBound{A₁} \AgdaSymbol{=} \AgdaFunction{rpl-pr1} \AgdaBound{Δ} \AgdaBound{A₁} \AgdaFunction{+S} \AgdaSymbol{\_}\<%
\\
%
\\
%
\\
%
\\
\>\AgdaFunction{filter} \AgdaInductiveConstructor{ε} \AgdaBound{A} \AgdaSymbol{=} \AgdaFunction{minimum-cm} \AgdaBound{A}\<%
\\
\>\AgdaFunction{filter} \AgdaSymbol{(}\AgdaBound{Δ} \AgdaInductiveConstructor{,} \AgdaBound{A}\AgdaSymbol{)} \AgdaBound{A₁} \AgdaSymbol{=} \<[21]%
\>[21]\AgdaFunction{ΣC-it-cm-spl} \AgdaBound{A₁} \AgdaBound{A} \AgdaFunction{⊚} \AgdaSymbol{((}\AgdaFunction{filter} \AgdaBound{Δ} \AgdaBound{A₁} \AgdaFunction{+S} \AgdaSymbol{\_)} \AgdaInductiveConstructor{,} \AgdaInductiveConstructor{var} \AgdaInductiveConstructor{v0} \AgdaFunction{⟦} \AgdaFunction{[+S]T} \AgdaFunction{⟫}\AgdaSymbol{)}\<%
\\
%
\\
%
\\
\>\AgdaFunction{rpl-T-p1} \AgdaSymbol{:} \AgdaSymbol{\{}\AgdaBound{Γ} \AgdaSymbol{:} \AgdaDatatype{Con}\AgdaSymbol{\}(}\AgdaBound{Δ} \AgdaSymbol{:} \AgdaDatatype{Con}\AgdaSymbol{)(}\AgdaBound{A} \AgdaSymbol{:} \AgdaDatatype{Ty} \AgdaBound{Γ}\AgdaSymbol{)} \AgdaSymbol{→} \AgdaFunction{rpl-T} \AgdaBound{A} \AgdaInductiveConstructor{*} \AgdaDatatype{≡} \AgdaBound{A} \AgdaFunction{[} \AgdaFunction{rpl-pr1} \AgdaBound{Δ} \AgdaBound{A} \AgdaFunction{]T}\<%
\\
\>\AgdaFunction{rpl-T-p1} \AgdaInductiveConstructor{ε} \AgdaBound{A} \AgdaSymbol{=} \AgdaFunction{trans} \AgdaSymbol{(}\AgdaFunction{ΣT-it-p1} \AgdaBound{A}\AgdaSymbol{)} \AgdaSymbol{(}\AgdaFunction{sym} \AgdaFunction{IC-T}\AgdaSymbol{)}\<%
\\
\>\AgdaFunction{rpl-T-p1} \AgdaSymbol{(}\AgdaBound{Δ} \AgdaInductiveConstructor{,} \AgdaBound{A}\AgdaSymbol{)} \AgdaBound{A₁} \AgdaSymbol{=} \AgdaFunction{trans} \AgdaFunction{[⊚]T} \AgdaSymbol{(}\AgdaFunction{trans} \AgdaSymbol{(}\AgdaFunction{congT} \AgdaSymbol{(}\AgdaFunction{ΣT-it-wk} \AgdaBound{A₁} \AgdaBound{A}\AgdaSymbol{))} \AgdaSymbol{(}\AgdaFunction{trans} \AgdaFunction{+T[,]T} \AgdaSymbol{(}\AgdaFunction{trans} \AgdaFunction{[+S]T} \AgdaSymbol{(}\AgdaFunction{trans} \AgdaSymbol{(}\AgdaFunction{wk-T} \AgdaSymbol{(}\AgdaFunction{rpl-T-p1} \AgdaBound{Δ} \AgdaBound{A₁}\AgdaSymbol{))} \AgdaSymbol{(}\AgdaFunction{sym} \AgdaFunction{[+S]T}\AgdaSymbol{)))))}\<%
\\
%
\\
\>\AgdaFunction{rpl-tm} \AgdaBound{A} \AgdaBound{a} \AgdaSymbol{=} \AgdaFunction{Σtm-it} \AgdaBound{A} \AgdaBound{a} \AgdaFunction{[} \AgdaFunction{filter} \AgdaSymbol{\_} \AgdaBound{A} \AgdaFunction{]tm}\<%
\\
%
\\
%
\\
\>\AgdaFunction{rpl-tm-id} \AgdaSymbol{:} \AgdaSymbol{\{}\AgdaBound{Γ} \AgdaSymbol{:} \AgdaDatatype{Con}\AgdaSymbol{\}\{}\AgdaBound{A} \AgdaSymbol{:} \AgdaDatatype{Ty} \AgdaBound{Γ}\AgdaSymbol{\}} \AgdaSymbol{→} \AgdaDatatype{Tm} \AgdaBound{A} \AgdaSymbol{→} \AgdaDatatype{Tm} \AgdaSymbol{(}\AgdaFunction{rpl-T} \AgdaSymbol{\{}Δ \AgdaSymbol{=} \AgdaInductiveConstructor{ε}\AgdaSymbol{\}} \AgdaBound{A} \AgdaInductiveConstructor{*}\AgdaSymbol{)}\<%
\\
\>\AgdaFunction{rpl-tm-id} \AgdaBound{x} \AgdaSymbol{=} \<[15]%
\>[15]\AgdaBound{x} \AgdaFunction{⟦} \AgdaFunction{ΣT-it-p1} \AgdaSymbol{\_} \AgdaFunction{⟫}\<%
\\
%
\\
%
\\
\>\AgdaFunction{rpl-T-p2} \AgdaSymbol{:} \AgdaSymbol{\{}\AgdaBound{Γ} \AgdaSymbol{:} \AgdaDatatype{Con}\AgdaSymbol{\}(}\AgdaBound{Δ} \AgdaSymbol{:} \AgdaDatatype{Con}\AgdaSymbol{)(}\AgdaBound{A} \AgdaSymbol{:} \AgdaDatatype{Ty} \AgdaBound{Γ}\AgdaSymbol{)\{}\AgdaBound{B} \AgdaSymbol{:} \AgdaDatatype{Ty} \AgdaBound{Δ}\AgdaSymbol{\}\{}\AgdaBound{a} \AgdaBound{b} \AgdaSymbol{:} \AgdaDatatype{Tm} \AgdaBound{B}\AgdaSymbol{\}} \<[63]%
\>[63]\AgdaSymbol{→} \AgdaFunction{rpl-T} \AgdaBound{A} \AgdaSymbol{(}\AgdaBound{a} \AgdaInductiveConstructor{=h} \AgdaBound{b}\AgdaSymbol{)} \AgdaDatatype{≡} \AgdaSymbol{(}\AgdaFunction{rpl-tm} \AgdaBound{A} \AgdaBound{a} \AgdaInductiveConstructor{=h} \AgdaFunction{rpl-tm} \AgdaBound{A} \AgdaBound{b}\AgdaSymbol{)}\<%
\\
\>\AgdaFunction{rpl-T-p2} \AgdaBound{Δ} \AgdaBound{A} \AgdaSymbol{=} \AgdaFunction{congT} \AgdaSymbol{(}\AgdaFunction{ΣT-it-p2} \AgdaBound{A}\AgdaSymbol{)}\<%
\\
%
\\
%
\\
\>\AgdaFunction{rpl-T-p3} \AgdaSymbol{:} \AgdaSymbol{\{}\AgdaBound{Γ} \AgdaSymbol{:} \AgdaDatatype{Con}\AgdaSymbol{\}(}\AgdaBound{Δ} \AgdaSymbol{:} \AgdaDatatype{Con}\AgdaSymbol{)(}\AgdaBound{A} \AgdaSymbol{:} \AgdaDatatype{Ty} \AgdaBound{Γ}\AgdaSymbol{)\{}\AgdaBound{B} \AgdaSymbol{:} \AgdaDatatype{Ty} \AgdaBound{Δ}\AgdaSymbol{\}\{}\AgdaBound{C} \AgdaSymbol{:} \AgdaDatatype{Ty} \AgdaBound{Δ}\AgdaSymbol{\}}\<%
\\
\>[8]\AgdaIndent{10}{}\<[10]%
\>[10]\AgdaSymbol{→} \AgdaFunction{rpl-T} \AgdaBound{A} \AgdaSymbol{(}\AgdaBound{C} \AgdaFunction{+T} \AgdaBound{B}\AgdaSymbol{)} \AgdaDatatype{≡} \AgdaFunction{rpl-T} \AgdaBound{A} \AgdaBound{C} \AgdaFunction{+T} \AgdaSymbol{\_}\<%
\\
\>\AgdaFunction{rpl-T-p3} \AgdaSymbol{\_} \AgdaBound{A} \AgdaSymbol{=} \AgdaFunction{trans} \AgdaFunction{[⊚]T} \AgdaSymbol{(}\AgdaFunction{trans} \AgdaSymbol{(}\AgdaFunction{congT} \AgdaSymbol{(}\AgdaFunction{ΣT-it-p3} \AgdaBound{A}\AgdaSymbol{))} \AgdaSymbol{(}\AgdaFunction{trans} \AgdaFunction{+T[,]T} \AgdaFunction{[+S]T}\AgdaSymbol{))}\<%
\\
%
\\
\>\AgdaFunction{rpl-T-p3-wk} \AgdaSymbol{:} \AgdaSymbol{\{}\AgdaBound{Γ} \AgdaSymbol{:} \AgdaDatatype{Con}\AgdaSymbol{\}(}\AgdaBound{Δ} \AgdaSymbol{:} \AgdaDatatype{Con}\AgdaSymbol{)(}\AgdaBound{A} \AgdaSymbol{:} \AgdaDatatype{Ty} \AgdaBound{Γ}\AgdaSymbol{)\{}\AgdaBound{B} \AgdaSymbol{:} \AgdaDatatype{Ty} \AgdaBound{Δ}\AgdaSymbol{\}\{}\AgdaBound{C} \AgdaSymbol{:} \AgdaDatatype{Ty} \AgdaBound{Δ}\AgdaSymbol{\}\{}\AgdaBound{γ} \AgdaSymbol{:} \AgdaBound{Γ} \AgdaDatatype{⇒} \AgdaFunction{rpl-C} \AgdaBound{A} \AgdaBound{Δ}\AgdaSymbol{\}\{}\AgdaBound{b} \AgdaSymbol{:} \AgdaDatatype{Tm} \AgdaSymbol{((}\AgdaFunction{ΣT-it} \AgdaBound{A} \AgdaBound{B} \AgdaFunction{[} \AgdaFunction{filter} \AgdaBound{Δ} \AgdaBound{A} \AgdaFunction{]T}\AgdaSymbol{)} \AgdaFunction{[} \AgdaBound{γ} \AgdaFunction{]T}\AgdaSymbol{)\}}\<%
\\
\>[8]\AgdaIndent{10}{}\<[10]%
\>[10]\AgdaSymbol{→} \AgdaFunction{rpl-T} \AgdaBound{A} \AgdaSymbol{(}\AgdaBound{C} \AgdaFunction{+T} \AgdaBound{B}\AgdaSymbol{)} \AgdaFunction{[} \AgdaBound{γ} \AgdaInductiveConstructor{,} \AgdaBound{b} \AgdaFunction{]T} \AgdaDatatype{≡} \AgdaFunction{rpl-T} \AgdaBound{A} \AgdaBound{C} \AgdaFunction{[} \AgdaBound{γ} \AgdaFunction{]T}\<%
\\
\>\AgdaFunction{rpl-T-p3-wk} \AgdaBound{Δ} \AgdaBound{A} \AgdaSymbol{=} \AgdaFunction{trans} \AgdaSymbol{(}\AgdaFunction{congT} \AgdaSymbol{(}\AgdaFunction{rpl-T-p3} \AgdaBound{Δ} \AgdaBound{A}\AgdaSymbol{))} \AgdaFunction{+T[,]T}\<%
\\
%
\\
\>\AgdaFunction{rpl-tm-v0'} \AgdaSymbol{:} \AgdaSymbol{\{}\AgdaBound{Γ} \AgdaSymbol{:} \AgdaDatatype{Con}\AgdaSymbol{\}(}\AgdaBound{Δ} \AgdaSymbol{:} \AgdaDatatype{Con}\AgdaSymbol{)(}\AgdaBound{A} \AgdaSymbol{:} \AgdaDatatype{Ty} \AgdaBound{Γ}\AgdaSymbol{)\{}\AgdaBound{B} \AgdaSymbol{:} \AgdaDatatype{Ty} \AgdaBound{Δ}\AgdaSymbol{\}}\<%
\\
\>[10]\AgdaIndent{11}{}\<[11]%
\>[11]\AgdaSymbol{→} \AgdaFunction{rpl-tm} \AgdaSymbol{\{}Δ \AgdaSymbol{=} \AgdaBound{Δ} \AgdaInductiveConstructor{,} \AgdaBound{B}\AgdaSymbol{\}} \AgdaBound{A} \AgdaSymbol{(}\AgdaInductiveConstructor{var} \AgdaInductiveConstructor{v0}\AgdaSymbol{)} \AgdaDatatype{≅} \AgdaInductiveConstructor{var} \AgdaInductiveConstructor{v0}\<%
\\
\>\AgdaFunction{rpl-tm-v0'} \AgdaBound{Δ} \AgdaBound{A} \AgdaSymbol{=} \AgdaFunction{[⊚]tm} \AgdaSymbol{(}\AgdaFunction{Σtm-it} \AgdaBound{A} \AgdaSymbol{(}\AgdaInductiveConstructor{var} \AgdaInductiveConstructor{v0}\AgdaSymbol{))} \AgdaFunction{∾} \AgdaFunction{congtm} \AgdaSymbol{(}\AgdaFunction{Σtm-it-p1} \AgdaBound{A}\AgdaSymbol{)} \AgdaFunction{∾} \AgdaFunction{wk-coh} \AgdaFunction{∾} \AgdaFunction{wk-coh+}\<%
\\
%
\\
\>\AgdaFunction{rpl-tm-v0} \AgdaSymbol{:} \AgdaSymbol{\{}\AgdaBound{Γ} \AgdaSymbol{:} \AgdaDatatype{Con}\AgdaSymbol{\}(}\AgdaBound{Δ} \AgdaSymbol{:} \AgdaDatatype{Con}\AgdaSymbol{)(}\AgdaBound{A} \AgdaSymbol{:} \AgdaDatatype{Ty} \AgdaBound{Γ}\AgdaSymbol{)\{}\AgdaBound{B} \AgdaSymbol{:} \AgdaDatatype{Ty} \AgdaBound{Δ}\AgdaSymbol{\}\{}\AgdaBound{γ} \AgdaSymbol{:} \AgdaBound{Γ} \AgdaDatatype{⇒} \AgdaFunction{rpl-C} \AgdaBound{A} \AgdaBound{Δ}\AgdaSymbol{\}\{}\AgdaBound{b} \AgdaSymbol{:} \AgdaDatatype{Tm} \AgdaBound{A}\AgdaSymbol{\}\{}\AgdaBound{b'} \AgdaSymbol{:} \AgdaDatatype{Tm} \AgdaSymbol{((}\AgdaFunction{ΣT-it} \AgdaBound{A} \AgdaBound{B} \AgdaFunction{[} \AgdaFunction{filter} \AgdaBound{Δ} \AgdaBound{A} \AgdaFunction{]T}\AgdaSymbol{)} \AgdaFunction{[} \AgdaBound{γ} \AgdaFunction{]T}\AgdaSymbol{)\}}\<%
\\
\>[0]\AgdaIndent{10}{}\<[10]%
\>[10]\AgdaSymbol{→} \AgdaSymbol{(}\AgdaBound{prf} \AgdaSymbol{:} \AgdaBound{b'} \AgdaDatatype{≅} \AgdaBound{b}\AgdaSymbol{)}\<%
\\
\>[0]\AgdaIndent{10}{}\<[10]%
\>[10]\AgdaSymbol{→} \AgdaFunction{rpl-tm} \AgdaSymbol{\{}Δ \AgdaSymbol{=} \AgdaBound{Δ} \AgdaInductiveConstructor{,} \AgdaBound{B}\AgdaSymbol{\}} \AgdaBound{A} \AgdaSymbol{(}\AgdaInductiveConstructor{var} \AgdaInductiveConstructor{v0}\AgdaSymbol{)} \AgdaFunction{[} \AgdaBound{γ} \AgdaInductiveConstructor{,} \AgdaBound{b'} \AgdaFunction{]tm} \AgdaDatatype{≅} \AgdaBound{b}\<%
\\
\>\AgdaFunction{rpl-tm-v0} \AgdaBound{Δ} \AgdaBound{A} \AgdaBound{prf} \AgdaSymbol{=} \AgdaFunction{congtm} \AgdaSymbol{(}\AgdaFunction{rpl-tm-v0'} \AgdaBound{Δ} \AgdaBound{A}\AgdaSymbol{)} \AgdaFunction{∾} \AgdaFunction{wk-coh} \AgdaFunction{∾} \AgdaBound{prf}\<%
\\
%
\\
%
\\
\>\AgdaFunction{rpl-tm-vS} \AgdaSymbol{:} \AgdaSymbol{\{}\AgdaBound{Γ} \AgdaSymbol{:} \AgdaDatatype{Con}\AgdaSymbol{\}(}\AgdaBound{Δ} \AgdaSymbol{:} \AgdaDatatype{Con}\AgdaSymbol{)(}\AgdaBound{A} \AgdaSymbol{:} \AgdaDatatype{Ty} \AgdaBound{Γ}\AgdaSymbol{)\{}\AgdaBound{B} \AgdaBound{C} \AgdaSymbol{:} \AgdaDatatype{Ty} \AgdaBound{Δ}\AgdaSymbol{\}\{}\AgdaBound{γ} \AgdaSymbol{:} \AgdaBound{Γ} \AgdaDatatype{⇒} \AgdaFunction{rpl-C} \AgdaBound{A} \AgdaBound{Δ}\AgdaSymbol{\}}\<%
\\
\>[10]\AgdaIndent{13}{}\<[13]%
\>[13]\AgdaSymbol{\{}\AgdaBound{b} \AgdaSymbol{:} \AgdaDatatype{Tm} \AgdaSymbol{(}\AgdaFunction{rpl-T} \AgdaBound{A} \AgdaBound{B} \AgdaFunction{[} \AgdaBound{γ} \AgdaFunction{]T}\AgdaSymbol{)\}\{}\AgdaBound{x} \AgdaSymbol{:} \AgdaDatatype{Var} \AgdaBound{C}\AgdaSymbol{\}} \AgdaSymbol{→} \AgdaFunction{rpl-tm} \AgdaSymbol{\{}Δ \AgdaSymbol{=} \AgdaBound{Δ} \AgdaInductiveConstructor{,} \AgdaBound{B}\AgdaSymbol{\}} \AgdaBound{A} \AgdaSymbol{(}\AgdaInductiveConstructor{var} \AgdaSymbol{(}\AgdaInductiveConstructor{vS} \AgdaBound{x}\AgdaSymbol{))} \AgdaFunction{[} \AgdaBound{γ} \AgdaInductiveConstructor{,} \AgdaBound{b} \AgdaFunction{]tm} \AgdaDatatype{≅} \AgdaFunction{rpl-tm} \AgdaBound{A} \AgdaSymbol{(}\AgdaInductiveConstructor{var} \AgdaBound{x}\AgdaSymbol{)} \AgdaFunction{[} \AgdaBound{γ} \AgdaFunction{]tm}\<%
\\
\>\AgdaFunction{rpl-tm-vS} \AgdaBound{Δ} \AgdaBound{A} \AgdaSymbol{\{}x \AgdaSymbol{=} \AgdaBound{x}\AgdaSymbol{\}} \AgdaSymbol{=} \AgdaFunction{congtm} \AgdaSymbol{(}\AgdaFunction{[⊚]tm} \AgdaSymbol{(}\AgdaFunction{Σtm-it} \AgdaBound{A} \AgdaSymbol{(}\AgdaInductiveConstructor{var} \AgdaSymbol{(}\AgdaInductiveConstructor{vS} \AgdaBound{x}\AgdaSymbol{)))} \AgdaFunction{∾} \AgdaSymbol{(}\AgdaFunction{congtm} \AgdaSymbol{(}\AgdaFunction{Σtm-it-p2} \AgdaBound{A} \AgdaBound{x}\AgdaSymbol{))} \<[90]%
\>[90]\AgdaFunction{∾} \AgdaFunction{+tm[,]tm} \AgdaSymbol{(}\AgdaFunction{Σtm-it} \AgdaBound{A} \AgdaSymbol{(}\AgdaInductiveConstructor{var} \AgdaBound{x}\AgdaSymbol{))} \<[121]%
\>[121]\AgdaFunction{∾} \AgdaSymbol{(}\AgdaFunction{[+S]tm} \AgdaSymbol{(}\AgdaFunction{Σtm-it} \AgdaBound{A} \AgdaSymbol{(}\AgdaInductiveConstructor{var} \AgdaBound{x}\AgdaSymbol{))))} \AgdaFunction{∾} \AgdaFunction{+tm[,]tm} \AgdaSymbol{(}\AgdaFunction{Σtm-it} \AgdaBound{A} \AgdaSymbol{(}\AgdaInductiveConstructor{var} \AgdaBound{x}\AgdaSymbol{)} \AgdaFunction{[} \AgdaFunction{filter} \AgdaSymbol{\_} \AgdaBound{A} \AgdaFunction{]tm}\AgdaSymbol{)}\<%
\\
%
\\
\>\AgdaComment{-- basic example}\<%
\\
%
\\
\>\AgdaFunction{base-1} \AgdaSymbol{:} \AgdaSymbol{\{}\AgdaBound{Γ} \AgdaSymbol{:} \AgdaDatatype{Con}\AgdaSymbol{\}\{}\AgdaBound{A} \AgdaSymbol{:} \AgdaDatatype{Ty} \AgdaBound{Γ}\AgdaSymbol{\}} \AgdaSymbol{→} \AgdaFunction{rpl-C} \AgdaBound{A} \AgdaSymbol{(}\AgdaInductiveConstructor{ε} \AgdaInductiveConstructor{,} \AgdaInductiveConstructor{*}\AgdaSymbol{)} \AgdaDatatype{≡} \AgdaSymbol{(}\AgdaBound{Γ} \AgdaInductiveConstructor{,} \AgdaBound{A}\AgdaSymbol{)}\<%
\\
\>\AgdaFunction{base-1} \AgdaSymbol{=} \AgdaFunction{cong} \AgdaSymbol{(λ} \AgdaBound{x} \AgdaSymbol{→} \AgdaSymbol{\_} \AgdaInductiveConstructor{,} \AgdaBound{x}\AgdaSymbol{)} \AgdaSymbol{(}\AgdaFunction{ΣT-it-p1} \AgdaSymbol{\_)}\<%
\\
%
\\
%
\\
%
\\
\>\AgdaFunction{map-1} \AgdaSymbol{:} \AgdaSymbol{\{}\AgdaBound{Γ} \AgdaSymbol{:} \AgdaDatatype{Con}\AgdaSymbol{\}\{}\AgdaBound{A} \AgdaSymbol{:} \AgdaDatatype{Ty} \AgdaBound{Γ}\AgdaSymbol{\}} \AgdaSymbol{→} \AgdaSymbol{(}\AgdaBound{Γ} \AgdaInductiveConstructor{,} \AgdaBound{A}\AgdaSymbol{)} \AgdaDatatype{⇒} \AgdaFunction{rpl-C} \AgdaBound{A} \AgdaSymbol{(}\AgdaInductiveConstructor{ε} \AgdaInductiveConstructor{,} \AgdaInductiveConstructor{*}\AgdaSymbol{)}\<%
\\
\>\AgdaFunction{map-1} \AgdaSymbol{=} \AgdaFunction{1-1cm-same} \AgdaSymbol{(}\AgdaFunction{ΣT-it-p1} \AgdaSymbol{\_)}\<%
\\
%
\\
\>\AgdaComment{-- some useful lemmas}\<%
\\
%
\\
\>\AgdaFunction{rpl*-A} \AgdaSymbol{:} \AgdaSymbol{\{}\AgdaBound{Γ} \AgdaSymbol{:} \AgdaDatatype{Con}\AgdaSymbol{\}\{}\AgdaBound{A} \AgdaSymbol{:} \AgdaDatatype{Ty} \AgdaBound{Γ}\AgdaSymbol{\}} \AgdaSymbol{→} \AgdaFunction{rpl-T} \AgdaSymbol{\{}Δ \AgdaSymbol{=} \AgdaInductiveConstructor{ε}\AgdaSymbol{\}} \AgdaBound{A} \AgdaInductiveConstructor{*} \AgdaFunction{[} \AgdaFunction{IdCm} \AgdaFunction{]T} \AgdaDatatype{≡} \AgdaBound{A}\<%
\\
\>\AgdaFunction{rpl*-A} \AgdaSymbol{=} \AgdaFunction{trans} \AgdaFunction{IC-T} \AgdaSymbol{(}\AgdaFunction{ΣT-it-p1} \AgdaSymbol{\_)}\<%
\\
%
\\
\>\AgdaFunction{rpl*-a} \AgdaSymbol{:} \AgdaSymbol{\{}\AgdaBound{Γ} \AgdaSymbol{:} \AgdaDatatype{Con}\AgdaSymbol{\}(}\AgdaBound{A} \AgdaSymbol{:} \AgdaDatatype{Ty} \AgdaBound{Γ}\AgdaSymbol{)\{}\AgdaBound{a} \AgdaSymbol{:} \AgdaDatatype{Tm} \AgdaBound{A}\AgdaSymbol{\}} \AgdaSymbol{→} \AgdaFunction{rpl-tm} \AgdaSymbol{\{}Δ \AgdaSymbol{=} \AgdaInductiveConstructor{ε} \AgdaInductiveConstructor{,} \AgdaInductiveConstructor{*}\AgdaSymbol{\}} \AgdaBound{A} \AgdaSymbol{(}\AgdaInductiveConstructor{var} \AgdaInductiveConstructor{v0}\AgdaSymbol{)} \AgdaFunction{[} \AgdaFunction{IdCm} \AgdaInductiveConstructor{,} \AgdaBound{a} \AgdaFunction{⟦} \AgdaFunction{rpl*-A} \AgdaFunction{⟫} \AgdaFunction{]tm} \AgdaDatatype{≅} \AgdaBound{a}\<%
\\
\>\AgdaFunction{rpl*-a} \AgdaBound{A} \AgdaSymbol{=} \AgdaFunction{rpl-tm-v0} \AgdaInductiveConstructor{ε} \AgdaBound{A} \<[26]%
\>[26]\AgdaSymbol{(}\AgdaFunction{cohOp} \AgdaSymbol{(}\AgdaFunction{rpl*-A} \AgdaSymbol{\{}A \AgdaSymbol{=} \AgdaBound{A}\AgdaSymbol{\}))}\<%
\\
%
\\
\>\AgdaFunction{rpl*-A2} \AgdaSymbol{:} \AgdaSymbol{\{}\AgdaBound{Γ} \AgdaSymbol{:} \AgdaDatatype{Con}\AgdaSymbol{\}(}\AgdaBound{A} \AgdaSymbol{:} \AgdaDatatype{Ty} \AgdaBound{Γ}\AgdaSymbol{)\{}\AgdaBound{a} \AgdaSymbol{:} \AgdaDatatype{Tm} \AgdaSymbol{(}\AgdaFunction{rpl-T} \AgdaBound{A} \AgdaSymbol{(}\AgdaInductiveConstructor{*} \AgdaSymbol{\{}\AgdaInductiveConstructor{ε}\AgdaSymbol{\})} \AgdaFunction{[} \AgdaFunction{IdCm} \AgdaFunction{]T}\AgdaSymbol{)\}} \<[66]%
\>[66]\<%
\\
\>[-5]\AgdaIndent{8}{}\<[8]%
\>[8]\AgdaSymbol{→} \AgdaFunction{rpl-T} \AgdaBound{A} \AgdaSymbol{(}\AgdaInductiveConstructor{*} \AgdaSymbol{\{}\AgdaInductiveConstructor{ε} \AgdaInductiveConstructor{,} \AgdaInductiveConstructor{*}\AgdaSymbol{\})} \AgdaFunction{[} \AgdaFunction{IdCm} \AgdaInductiveConstructor{,} \AgdaBound{a} \AgdaFunction{]T} \AgdaDatatype{≡} \AgdaBound{A}\<%
\\
\>\AgdaFunction{rpl*-A2} \AgdaBound{A} \AgdaSymbol{=} \AgdaFunction{trans} \AgdaSymbol{(}\AgdaFunction{rpl-T-p3-wk} \AgdaInductiveConstructor{ε} \AgdaBound{A}\AgdaSymbol{)} \AgdaFunction{rpl*-A}\<%
\\
%
\\
\>\AgdaFunction{rpl-xy} \AgdaSymbol{:} \<[10]%
\>[10]\AgdaSymbol{\{}\AgdaBound{Γ} \AgdaSymbol{:} \AgdaDatatype{Con}\AgdaSymbol{\}(}\AgdaBound{A} \AgdaSymbol{:} \AgdaDatatype{Ty} \AgdaBound{Γ}\AgdaSymbol{)(}\AgdaBound{a} \AgdaBound{b} \AgdaSymbol{:} \AgdaDatatype{Tm} \AgdaBound{A}\AgdaSymbol{)}\<%
\\
\>[0]\AgdaIndent{7}{}\<[7]%
\>[7]\AgdaSymbol{→} \AgdaFunction{rpl-T} \AgdaSymbol{\{}Δ \AgdaSymbol{=} \AgdaInductiveConstructor{ε} \AgdaInductiveConstructor{,} \AgdaInductiveConstructor{*} \AgdaInductiveConstructor{,} \AgdaInductiveConstructor{*}\AgdaSymbol{\}} \AgdaBound{A} \AgdaSymbol{(}\AgdaInductiveConstructor{var} \AgdaSymbol{(}\AgdaInductiveConstructor{vS} \AgdaInductiveConstructor{v0}\AgdaSymbol{)} \AgdaInductiveConstructor{=h} \AgdaInductiveConstructor{var} \AgdaInductiveConstructor{v0}\AgdaSymbol{)} \AgdaFunction{[} \AgdaFunction{IdCm} \AgdaInductiveConstructor{,} \AgdaBound{a} \AgdaFunction{⟦} \AgdaFunction{rpl*-A} \AgdaFunction{⟫} \AgdaInductiveConstructor{,} \AgdaBound{b} \AgdaFunction{⟦} \AgdaFunction{rpl*-A2} \AgdaBound{A} \AgdaFunction{⟫} \AgdaFunction{]T}\<%
\\
\>[7]\AgdaIndent{14}{}\<[14]%
\>[14]\AgdaDatatype{≡} \AgdaSymbol{(}\AgdaBound{a} \AgdaInductiveConstructor{=h} \AgdaBound{b}\AgdaSymbol{)}\<%
\\
\>\AgdaFunction{rpl-xy} \AgdaBound{A} \AgdaBound{a} \AgdaBound{b} \AgdaSymbol{=} \<[16]%
\>[16]\AgdaFunction{trans} \AgdaSymbol{(}\AgdaFunction{congT} \AgdaSymbol{(}\AgdaFunction{rpl-T-p2} \AgdaSymbol{(}\AgdaInductiveConstructor{ε} \AgdaInductiveConstructor{,} \AgdaInductiveConstructor{*} \AgdaInductiveConstructor{,} \AgdaInductiveConstructor{*}\AgdaSymbol{)} \AgdaBound{A}\AgdaSymbol{))} \<[55]%
\>[55]\<%
\\
\>[-6]\AgdaIndent{13}{}\<[13]%
\>[13]\AgdaSymbol{(}\AgdaFunction{hom≡} \AgdaSymbol{((}\AgdaFunction{rpl-tm-vS} \AgdaSymbol{(}\AgdaInductiveConstructor{ε} \AgdaInductiveConstructor{,} \AgdaInductiveConstructor{*}\AgdaSymbol{)} \AgdaBound{A}\AgdaSymbol{)} \<[43]%
\>[43]\AgdaFunction{∾} \AgdaFunction{rpl*-a} \AgdaBound{A}\AgdaSymbol{)} \<[55]%
\>[55]\<%
\\
\>[0]\AgdaIndent{19}{}\<[19]%
\>[19]\AgdaSymbol{(}\AgdaFunction{rpl-tm-v0} \AgdaSymbol{(}\AgdaInductiveConstructor{ε} \AgdaInductiveConstructor{,} \AgdaInductiveConstructor{*}\AgdaSymbol{)} \AgdaBound{A} \AgdaSymbol{(}\AgdaFunction{cohOp} \AgdaSymbol{(}\AgdaFunction{rpl*-A2} \AgdaBound{A}\AgdaSymbol{))))}\<%
\\
%
\\
%
\\
%
\\
\>\AgdaFunction{rpl-sub} \AgdaSymbol{:} \AgdaSymbol{(}\AgdaBound{Γ} \AgdaSymbol{:} \AgdaDatatype{Con}\AgdaSymbol{)(}\AgdaBound{A} \AgdaSymbol{:} \AgdaDatatype{Ty} \AgdaBound{Γ}\AgdaSymbol{)(}\AgdaBound{a} \AgdaBound{b} \AgdaSymbol{:} \AgdaDatatype{Tm} \AgdaBound{A}\AgdaSymbol{)} \AgdaSymbol{→}\<%
\\
\>[0]\AgdaIndent{10}{}\<[10]%
\>[10]\AgdaDatatype{Tm} \AgdaSymbol{(}\AgdaBound{a} \AgdaInductiveConstructor{=h} \AgdaBound{b}\AgdaSymbol{)}\<%
\\
\>[0]\AgdaIndent{8}{}\<[8]%
\>[8]\AgdaSymbol{→} \AgdaBound{Γ} \AgdaDatatype{⇒} \AgdaFunction{rpl-C} \AgdaBound{A} \AgdaSymbol{(}\AgdaInductiveConstructor{ε} \AgdaInductiveConstructor{,} \AgdaInductiveConstructor{*} \AgdaInductiveConstructor{,} \AgdaInductiveConstructor{*} \AgdaInductiveConstructor{,} \AgdaSymbol{(}\AgdaInductiveConstructor{var} \AgdaSymbol{(}\AgdaInductiveConstructor{vS} \AgdaInductiveConstructor{v0}\AgdaSymbol{)} \AgdaInductiveConstructor{=h} \AgdaInductiveConstructor{var} \AgdaInductiveConstructor{v0}\AgdaSymbol{))}\<%
\\
\>\AgdaFunction{rpl-sub} \AgdaBound{Γ} \AgdaBound{A} \AgdaBound{a} \AgdaBound{b} \AgdaBound{t} \AgdaSymbol{=} \AgdaFunction{IdCm} \AgdaInductiveConstructor{,} \AgdaBound{a} \AgdaFunction{⟦} \AgdaFunction{rpl*-A} \AgdaFunction{⟫} \AgdaInductiveConstructor{,} \AgdaBound{b} \AgdaFunction{⟦} \AgdaFunction{rpl*-A2} \AgdaBound{A} \AgdaFunction{⟫} \AgdaInductiveConstructor{,} \AgdaBound{t} \AgdaFunction{⟦} \AgdaFunction{rpl-xy} \AgdaBound{A} \AgdaBound{a} \AgdaBound{b} \AgdaFunction{⟫}\<%
\\
\>[0]\AgdaIndent{2}{}\<[2]%
\>[2]\<%
\\
%
\\
\>\<\end{code}
}


\AgdaHide{
\begin{code}\>\<%
\\
%
\\
\>\AgdaSymbol{\{-\#} \AgdaKeyword{OPTIONS} --type-in-type --no-positivity-check --no-termination-check \AgdaSymbol{\#-\}}\<%
\\
%
\\
%
\\
\>\AgdaKeyword{module} \AgdaModule{GroupoidStructure} \AgdaKeyword{where}\<%
\\
%
\\
\>\AgdaKeyword{open} \AgdaKeyword{import} \AgdaModule{Relation.Binary.PropositionalEquality} \<[50]%
\>[50]\<%
\\
\>\AgdaKeyword{open} \AgdaKeyword{import} \AgdaModule{Data.Product} \AgdaKeyword{renaming} \AgdaSymbol{(}\_,\_ \AgdaSymbol{to} \_,,\_\AgdaSymbol{)}\<%
\\
\>\AgdaKeyword{open} \AgdaKeyword{import} \AgdaModule{Data.Nat}\<%
\\
%
\\
%
\\
\>\AgdaKeyword{open} \AgdaKeyword{import} \AgdaModule{BasicSyntax} \AgdaKeyword{renaming} \AgdaSymbol{(}\_∾\_ \AgdaSymbol{to} htrans\AgdaSymbol{)}\<%
\\
\>\AgdaKeyword{open} \AgdaKeyword{import} \AgdaModule{IdentityContextMorphisms}\<%
\\
\>\AgdaKeyword{open} \AgdaKeyword{import} \AgdaModule{Suspension}\<%
\\
%
\\
\>\<\end{code}
}

\subsection{First-level Groupoid Structure}
We can proceed to the definition of the groupoid structure of the syntax. We start with the base case: 1-cells. Replacement defined above allows us to lift this structure to an arbitrary level $n$ (we leave most of the routine details out). This shows that the syntax is a 1-groupoid on each level. In the next section we show how also the higher-groupoid structure can be defined. 

We start by an essential lemma which formalises the discussion at the
beginning of this section: to construct a term in a type $A$ in an
arbitrary context, we first restrict attention to a suitable
contractible context $\Delta$ and use lifting and substitution -- replacement -- to pull
 the term built by $\AgdaInductiveConstructor{coh}$ in $\Delta$
back. This relies on the fact that a lifted contractible context is
also contractible, and therefore any type lifted from a contractible
context is also inhabited.

\begin{code}\>\<%
\\
\>\AgdaFunction{Coh-rpl} \<[9]%
\>[9]\AgdaSymbol{:} \AgdaSymbol{∀\{}\AgdaBound{Γ} \AgdaBound{Δ}\AgdaSymbol{\}(}\AgdaBound{A} \AgdaSymbol{:} \AgdaDatatype{Ty} \AgdaBound{Γ}\AgdaSymbol{)(}\AgdaBound{B} \AgdaSymbol{:} \AgdaDatatype{Ty} \AgdaBound{Δ}\AgdaSymbol{)} \AgdaSymbol{→} \AgdaDatatype{isContr} \AgdaBound{Δ}\<%
\\
\>[0]\AgdaIndent{9}{}\<[9]%
\>[9]\AgdaSymbol{→} \AgdaDatatype{Tm} \AgdaSymbol{(}\AgdaFunction{rpl-T} \AgdaBound{A} \AgdaBound{B}\AgdaSymbol{)}\<%
\\
\>\AgdaFunction{Coh-rpl} \AgdaSymbol{\{\_\}} \AgdaSymbol{\{}\AgdaBound{Δ}\AgdaSymbol{\}} \AgdaBound{A} \AgdaSymbol{\_} \AgdaBound{isC} \AgdaSymbol{=} \AgdaInductiveConstructor{coh} \AgdaSymbol{(}\AgdaFunction{ΣC-it-ε-Contr} \AgdaBound{A} \AgdaBound{isC}\AgdaSymbol{)} \AgdaSymbol{\_} \AgdaSymbol{\_}\<%
\\
\>\<\end{code}
Next we define the reflexivity, symmetry and transitivity terms of any type. Let's start from some base cases. Each of the base cases is derivable in a different contractible context with \AgdaFunction{Coh-Contr} which gives you a coherence constant for any type in any contractible context.

\noindent \textbf{Reflexivity} (identity) It only requires a one-object context.

\begin{code}\>\<%
\\
\>\AgdaFunction{refl*-Tm} \AgdaSymbol{:} \AgdaDatatype{Tm} \AgdaSymbol{\{}\AgdaFunction{x:*}\AgdaSymbol{\}} \AgdaSymbol{(}\AgdaInductiveConstructor{var} \AgdaInductiveConstructor{v0} \AgdaInductiveConstructor{=h} \AgdaInductiveConstructor{var} \AgdaInductiveConstructor{v0}\AgdaSymbol{)}\<%
\\
\>\AgdaFunction{refl*-Tm} \AgdaSymbol{=} \AgdaFunction{Coh-Contr} \AgdaInductiveConstructor{c*}\<%
\\
\>\<\end{code}
\noindent  \textbf{Symmetry} (inverse) It is defined similarly. Note that the intricate names of contexts, as in \AgdaDatatype{Ty} \AgdaFunction{x:*,y:*,α:x=y} indicate their definitions which have been hidden. Recall that Agda treats all sequences of characters uninterrupted by whitespace as identifiers. For instance \AgdaFunction{x:*,y:*,α:x=y} is a name of a context for which we are assuming the definition:
\AgdaFunction{x:*,y:*,α:x=y} \AgdaSymbol{=} \AgdaInductiveConstructor{ε} \AgdaInductiveConstructor{,} \AgdaInductiveConstructor{*} \AgdaInductiveConstructor{,} \AgdaInductiveConstructor{*} \AgdaInductiveConstructor{,} \AgdaSymbol{(}\AgdaInductiveConstructor{var} \AgdaSymbol{(}\AgdaInductiveConstructor{vS} \AgdaInductiveConstructor{v0}\AgdaSymbol{)} \AgdaInductiveConstructor{=h} \AgdaInductiveConstructor{var} \AgdaInductiveConstructor{v0}\AgdaSymbol{)}.


\begin{code}\>\<%
\\
\>\AgdaFunction{sym*-Ty} \AgdaSymbol{:} \AgdaDatatype{Ty} \AgdaFunction{x:*,y:*,α:x=y}\<%
\\
\>\AgdaFunction{sym*-Ty} \AgdaSymbol{=} \AgdaFunction{vY} \AgdaInductiveConstructor{=h} \AgdaFunction{vX}\<%
\\
%
\\
\>\AgdaFunction{sym*-Tm} \AgdaSymbol{:} \AgdaDatatype{Tm} \AgdaSymbol{\{}\AgdaFunction{x:*,y:*,α:x=y}\AgdaSymbol{\}} \AgdaFunction{sym*-Ty}\<%
\\
\>\AgdaFunction{sym*-Tm} \AgdaSymbol{=} \AgdaFunction{Coh-Contr} \AgdaSymbol{(}\AgdaInductiveConstructor{ext} \AgdaInductiveConstructor{c*} \AgdaInductiveConstructor{v0}\AgdaSymbol{)}\<%
\\
\>\<\end{code}
\textbf{Transitivity} (composition)

\begin{code}\>\<%
\\
\>\AgdaFunction{trans*-Ty} \AgdaSymbol{:} \AgdaDatatype{Ty} \AgdaFunction{x:*,y:*,α:x=y,z:*,β:y=z}\<%
\\
\>\AgdaFunction{trans*-Ty} \AgdaSymbol{=} \AgdaSymbol{(}\AgdaFunction{vX} \AgdaFunction{+tm} \AgdaSymbol{\_} \AgdaFunction{+tm} \AgdaSymbol{\_)} \AgdaInductiveConstructor{=h} \AgdaFunction{vZ}\<%
\\
%
\\
\>\AgdaFunction{trans*-Tm} \AgdaSymbol{:} \AgdaDatatype{Tm} \AgdaFunction{trans*-Ty}\<%
\\
\>\AgdaFunction{trans*-Tm} \AgdaSymbol{=} \AgdaFunction{Coh-Contr} \AgdaSymbol{(}\AgdaInductiveConstructor{ext} \AgdaSymbol{(}\AgdaInductiveConstructor{ext} \AgdaInductiveConstructor{c*} \AgdaInductiveConstructor{v0}\AgdaSymbol{)} \AgdaSymbol{(}\AgdaInductiveConstructor{vS} \AgdaInductiveConstructor{v0}\AgdaSymbol{))}\<%
\\
\>\<\end{code}
\noindent To obtain these terms for any given type in any give context, we use replacement.

\begin{code}\>\<%
\\
\>\AgdaFunction{refl-Tm} \<[11]%
\>[11]\AgdaSymbol{:} \AgdaSymbol{\{}\AgdaBound{Γ} \AgdaSymbol{:} \AgdaDatatype{Con}\AgdaSymbol{\}(}\AgdaBound{A} \AgdaSymbol{:} \AgdaDatatype{Ty} \AgdaBound{Γ}\AgdaSymbol{)} \<[33]%
\>[33]\<%
\\
\>[9]\AgdaIndent{11}{}\<[11]%
\>[11]\AgdaSymbol{→} \AgdaDatatype{Tm} \AgdaSymbol{(}\AgdaFunction{rpl-T} \AgdaSymbol{\{}Δ \AgdaSymbol{=} \AgdaFunction{x:*}\AgdaSymbol{\}} \AgdaBound{A} \AgdaSymbol{(}\AgdaInductiveConstructor{var} \AgdaInductiveConstructor{v0} \AgdaInductiveConstructor{=h} \AgdaInductiveConstructor{var} \AgdaInductiveConstructor{v0}\AgdaSymbol{))}\<%
\\
\>\AgdaFunction{refl-Tm} \AgdaBound{A} \<[11]%
\>[11]\AgdaSymbol{=} \AgdaFunction{rpl-tm} \AgdaBound{A} \AgdaFunction{refl*-Tm}\<%
\\
%
\\
\>\AgdaFunction{sym-Tm} \AgdaSymbol{:} \AgdaSymbol{∀} \AgdaSymbol{\{}\AgdaBound{Γ}\AgdaSymbol{\}(}\AgdaBound{A} \AgdaSymbol{:} \AgdaDatatype{Ty} \AgdaBound{Γ}\AgdaSymbol{)} \AgdaSymbol{→} \AgdaDatatype{Tm} \AgdaSymbol{(}\AgdaFunction{rpl-T} \AgdaBound{A} \AgdaFunction{sym*-Ty}\AgdaSymbol{)}\<%
\\
\>\AgdaFunction{sym-Tm} \AgdaBound{A} \AgdaSymbol{=} \AgdaFunction{rpl-tm} \AgdaBound{A} \AgdaFunction{sym*-Tm}\<%
\\
%
\\
\>\AgdaFunction{trans-Tm} \AgdaSymbol{:} \AgdaSymbol{∀} \AgdaSymbol{\{}\AgdaBound{Γ}\AgdaSymbol{\}(}\AgdaBound{A} \AgdaSymbol{:} \AgdaDatatype{Ty} \AgdaBound{Γ}\AgdaSymbol{)} \AgdaSymbol{→} \AgdaDatatype{Tm} \AgdaSymbol{(}\AgdaFunction{rpl-T} \AgdaBound{A} \AgdaFunction{trans*-Ty}\AgdaSymbol{)}\<%
\\
\>\AgdaFunction{trans-Tm} \AgdaBound{A} \AgdaSymbol{=} \AgdaFunction{rpl-tm} \AgdaBound{A} \AgdaFunction{trans*-Tm}\<%
\\
\>\<\end{code}
\AgdaHide{
\begin{code}\>\<%
\\
%
\\
\>\AgdaComment{-- The version without lifting function}\<%
\\
%
\\
\>\AgdaFunction{refl-Tm'} \AgdaSymbol{:} \<[12]%
\>[12]\AgdaSymbol{\{}\AgdaBound{Γ} \AgdaSymbol{:} \AgdaDatatype{Con}\AgdaSymbol{\}(}\AgdaBound{A} \AgdaSymbol{:} \AgdaDatatype{Ty} \AgdaBound{Γ}\AgdaSymbol{)} \AgdaSymbol{→} \AgdaDatatype{Tm} \AgdaSymbol{\{}\AgdaBound{Γ} \AgdaInductiveConstructor{,} \AgdaBound{A}\AgdaSymbol{\}} \AgdaSymbol{(}\AgdaInductiveConstructor{var} \AgdaInductiveConstructor{v0} \AgdaInductiveConstructor{=h} \AgdaInductiveConstructor{var} \AgdaInductiveConstructor{v0}\AgdaSymbol{)}\<%
\\
\>\AgdaFunction{refl-Tm'} \AgdaBound{A} \AgdaSymbol{=} \AgdaSymbol{(}\AgdaFunction{refl-Tm} \AgdaBound{A}\AgdaSymbol{)} \<[26]%
\>[26]\AgdaFunction{[} \AgdaFunction{map-1} \AgdaFunction{]tm} \AgdaFunction{⟦} \AgdaFunction{prf1} \AgdaFunction{⟫}\<%
\\
\>[0]\AgdaIndent{2}{}\<[2]%
\>[2]\AgdaKeyword{where}\<%
\\
\>[0]\AgdaIndent{4}{}\<[4]%
\>[4]\AgdaFunction{prf} \AgdaSymbol{:} \AgdaFunction{rpl-tm} \AgdaSymbol{\{}Δ \AgdaSymbol{=} \AgdaFunction{x:*}\AgdaSymbol{\}} \AgdaBound{A} \AgdaSymbol{(}\AgdaInductiveConstructor{var} \AgdaInductiveConstructor{v0}\AgdaSymbol{)} \AgdaFunction{[} \AgdaFunction{map-1} \AgdaFunction{]tm} \AgdaDatatype{≅} \AgdaInductiveConstructor{var} \AgdaInductiveConstructor{v0}\<%
\\
\>[0]\AgdaIndent{4}{}\<[4]%
\>[4]\AgdaFunction{prf} \AgdaSymbol{=} \AgdaFunction{htrans} \AgdaSymbol{(}\AgdaFunction{congtm} \AgdaSymbol{(}\AgdaFunction{htrans} \AgdaSymbol{(}\AgdaFunction{[⊚]tm} \AgdaSymbol{(}\AgdaFunction{Σtm-it} \AgdaBound{A} \AgdaSymbol{(}\AgdaInductiveConstructor{var} \AgdaInductiveConstructor{v0}\AgdaSymbol{)))} \<[61]%
\>[61]\<%
\\
\>[4]\AgdaIndent{10}{}\<[10]%
\>[10]\AgdaSymbol{(}\AgdaFunction{htrans} \AgdaSymbol{(}\AgdaFunction{congtm} \AgdaSymbol{(}\AgdaFunction{Σtm-it-p1} \AgdaBound{A}\AgdaSymbol{))} \AgdaSymbol{(}\AgdaFunction{htrans} \AgdaFunction{wk-coh} \AgdaFunction{wk-coh+}\AgdaSymbol{))))} \<[68]%
\>[68]\<%
\\
\>[10]\AgdaIndent{11}{}\<[11]%
\>[11]\AgdaSymbol{(}\AgdaFunction{1-1cm-same-v0} \AgdaSymbol{(}\AgdaFunction{ΣT-it-p1} \AgdaBound{A}\AgdaSymbol{))}\<%
\\
%
\\
\>[-3]\AgdaIndent{4}{}\<[4]%
\>[4]\AgdaFunction{prf1} \AgdaSymbol{:} \AgdaSymbol{(}\AgdaInductiveConstructor{var} \AgdaInductiveConstructor{v0} \AgdaInductiveConstructor{=h} \AgdaInductiveConstructor{var} \AgdaInductiveConstructor{v0}\AgdaSymbol{)} \AgdaDatatype{≡} \AgdaFunction{rpl-T} \AgdaSymbol{\{}Δ \AgdaSymbol{=} \AgdaFunction{x:*}\AgdaSymbol{\}} \AgdaBound{A} \AgdaSymbol{(}\AgdaInductiveConstructor{var} \AgdaInductiveConstructor{v0} \AgdaInductiveConstructor{=h} \AgdaInductiveConstructor{var} \AgdaInductiveConstructor{v0}\AgdaSymbol{)} \AgdaFunction{[} \AgdaFunction{map-1} \AgdaFunction{]T}\<%
\\
\>[0]\AgdaIndent{4}{}\<[4]%
\>[4]\AgdaFunction{prf1} \AgdaSymbol{=} \AgdaFunction{sym} \AgdaSymbol{(}\AgdaFunction{trans} \AgdaSymbol{(}\AgdaFunction{congT} \AgdaSymbol{(}\AgdaFunction{rpl-T-p2} \AgdaFunction{x:*} \AgdaBound{A}\AgdaSymbol{))} \AgdaSymbol{(}\AgdaFunction{hom≡} \AgdaFunction{prf} \AgdaFunction{prf}\AgdaSymbol{))}\<%
\\
%
\\
\>\AgdaFunction{refl-Fun} \AgdaSymbol{:} \AgdaSymbol{(}\AgdaBound{Γ} \AgdaSymbol{:} \AgdaDatatype{Con}\AgdaSymbol{)(}\AgdaBound{A} \AgdaSymbol{:} \AgdaDatatype{Ty} \AgdaBound{Γ}\AgdaSymbol{)(}\AgdaBound{x} \AgdaSymbol{:} \AgdaDatatype{Tm} \AgdaBound{A}\AgdaSymbol{)} \AgdaSymbol{→} \AgdaDatatype{Tm} \AgdaSymbol{(}\AgdaBound{x} \AgdaInductiveConstructor{=h} \AgdaBound{x}\AgdaSymbol{)}\<%
\\
\>\AgdaFunction{refl-Fun} \AgdaBound{Γ} \AgdaBound{A} \AgdaBound{x} \AgdaSymbol{=} \<[18]%
\>[18]\AgdaSymbol{(}\AgdaFunction{refl-Tm} \AgdaBound{A}\AgdaSymbol{)} \<[30]%
\>[30]\<%
\\
\>[4]\AgdaIndent{17}{}\<[17]%
\>[17]\AgdaFunction{[} \AgdaFunction{IdCm} \AgdaInductiveConstructor{,} \AgdaBound{x} \AgdaFunction{⟦} \AgdaFunction{rpl*-A} \AgdaFunction{⟫} \AgdaFunction{]tm} \<[43]%
\>[43]\<%
\\
\>[4]\AgdaIndent{17}{}\<[17]%
\>[17]\AgdaFunction{⟦} \AgdaFunction{sym} \AgdaSymbol{(}\AgdaFunction{trans} \AgdaSymbol{(}\AgdaFunction{congT} \AgdaSymbol{(}\AgdaFunction{rpl-T-p2} \AgdaFunction{x:*} \AgdaBound{A}\AgdaSymbol{))} \AgdaSymbol{(}\AgdaFunction{hom≡} \AgdaSymbol{(}\AgdaFunction{rpl*-a} \AgdaBound{A}\AgdaSymbol{)} \AgdaSymbol{(}\AgdaFunction{rpl*-a} \AgdaBound{A}\AgdaSymbol{)))} \AgdaFunction{⟫}\<%
\\
%
\\
\>\AgdaFunction{Tm-sym-fun} \AgdaSymbol{:} \AgdaSymbol{(}\AgdaBound{Γ} \AgdaSymbol{:} \AgdaDatatype{Con}\AgdaSymbol{)(}\AgdaBound{A} \AgdaSymbol{:} \AgdaDatatype{Ty} \AgdaBound{Γ}\AgdaSymbol{)} \<[33]%
\>[33]\<%
\\
\>[6]\AgdaIndent{7}{}\<[7]%
\>[7]\AgdaSymbol{→} \AgdaDatatype{Tm} \AgdaSymbol{(}\AgdaFunction{rpl-T} \<[20]%
\>[20]\AgdaSymbol{\{}Δ \AgdaSymbol{=} \AgdaInductiveConstructor{ε} \AgdaInductiveConstructor{,} \AgdaInductiveConstructor{*} \AgdaInductiveConstructor{,} \AgdaInductiveConstructor{*}\AgdaSymbol{\}} \AgdaBound{A} \AgdaSymbol{(}\AgdaInductiveConstructor{var} \AgdaSymbol{(}\AgdaInductiveConstructor{vS} \AgdaInductiveConstructor{v0}\AgdaSymbol{)} \AgdaInductiveConstructor{=h} \AgdaInductiveConstructor{var} \AgdaInductiveConstructor{v0}\AgdaSymbol{))} \<[63]%
\>[63]\<%
\\
\>[0]\AgdaIndent{7}{}\<[7]%
\>[7]\AgdaSymbol{→} \AgdaDatatype{Tm} \AgdaSymbol{(}\AgdaFunction{rpl-T} \<[20]%
\>[20]\AgdaSymbol{\{}Δ \AgdaSymbol{=} \AgdaInductiveConstructor{ε} \AgdaInductiveConstructor{,} \AgdaInductiveConstructor{*} \AgdaInductiveConstructor{,} \AgdaInductiveConstructor{*}\AgdaSymbol{\}} \AgdaBound{A} \AgdaSymbol{(}\AgdaInductiveConstructor{var} \AgdaInductiveConstructor{v0} \AgdaInductiveConstructor{=h} \AgdaInductiveConstructor{var} \AgdaSymbol{(}\AgdaInductiveConstructor{vS} \AgdaInductiveConstructor{v0}\AgdaSymbol{)))}\<%
\\
\>\AgdaFunction{Tm-sym-fun} \AgdaBound{Γ} \AgdaBound{A} \AgdaSymbol{=} \AgdaFunction{fun} \AgdaSymbol{(}\AgdaFunction{sym-Tm} \AgdaBound{A} \AgdaFunction{⟦} \AgdaFunction{sym} \AgdaSymbol{(}\AgdaFunction{rpl-T-p3} \AgdaSymbol{(}\AgdaInductiveConstructor{ε} \AgdaInductiveConstructor{,} \AgdaInductiveConstructor{*} \AgdaInductiveConstructor{,} \AgdaInductiveConstructor{*}\AgdaSymbol{)} \AgdaBound{A}\AgdaSymbol{)} \AgdaFunction{⟫}\AgdaSymbol{)}\<%
\\
%
\\
\>\AgdaFunction{Tm-sym-fun2} \AgdaSymbol{:} \AgdaSymbol{(}\AgdaBound{Γ} \AgdaSymbol{:} \AgdaDatatype{Con}\AgdaSymbol{)(}\AgdaBound{A} \AgdaSymbol{:} \AgdaDatatype{Ty} \AgdaBound{Γ}\AgdaSymbol{)} \<[34]%
\>[34]\<%
\\
\>[0]\AgdaIndent{7}{}\<[7]%
\>[7]\AgdaSymbol{→} \AgdaDatatype{Tm} \AgdaSymbol{\{}\AgdaBound{Γ} \AgdaInductiveConstructor{,} \AgdaBound{A} \AgdaInductiveConstructor{,} \AgdaBound{A} \AgdaFunction{+T} \AgdaBound{A}\AgdaSymbol{\}} \AgdaSymbol{(}\AgdaInductiveConstructor{var} \AgdaSymbol{(}\AgdaInductiveConstructor{vS} \AgdaInductiveConstructor{v0}\AgdaSymbol{)} \AgdaInductiveConstructor{=h} \AgdaInductiveConstructor{var} \AgdaInductiveConstructor{v0}\AgdaSymbol{)} \<[53]%
\>[53]\<%
\\
\>[0]\AgdaIndent{7}{}\<[7]%
\>[7]\AgdaSymbol{→} \AgdaDatatype{Tm} \AgdaSymbol{\{}\AgdaBound{Γ} \AgdaInductiveConstructor{,} \AgdaBound{A} \AgdaInductiveConstructor{,} \AgdaBound{A} \AgdaFunction{+T} \AgdaBound{A}\AgdaSymbol{\}} \AgdaSymbol{(}\AgdaInductiveConstructor{var} \AgdaInductiveConstructor{v0} \AgdaInductiveConstructor{=h} \AgdaInductiveConstructor{var} \AgdaSymbol{(}\AgdaInductiveConstructor{vS} \AgdaInductiveConstructor{v0}\AgdaSymbol{))}\<%
\\
\>\AgdaFunction{Tm-sym-fun2} \AgdaBound{Γ} \AgdaBound{A} \AgdaBound{t} \AgdaSymbol{=}\<%
\\
\>[0]\AgdaIndent{2}{}\<[2]%
\>[2]\AgdaSymbol{(}\AgdaBound{t} \AgdaFunction{[} \AgdaSymbol{(}\AgdaFunction{wk-id} \AgdaInductiveConstructor{,} \<[16]%
\>[16]\<%
\\
\>[0]\AgdaIndent{2}{}\<[2]%
\>[2]\AgdaSymbol{(}\AgdaInductiveConstructor{var} \AgdaInductiveConstructor{v0} \AgdaFunction{⟦} \AgdaFunction{eq1} \AgdaFunction{⟫}\AgdaSymbol{))} \AgdaInductiveConstructor{,}\<%
\\
\>[0]\AgdaIndent{2}{}\<[2]%
\>[2]\AgdaSymbol{(}\AgdaInductiveConstructor{var} \AgdaSymbol{(}\AgdaInductiveConstructor{vS} \AgdaInductiveConstructor{v0}\AgdaSymbol{)} \AgdaFunction{⟦} \AgdaFunction{eq2} \AgdaFunction{⟫}\AgdaSymbol{)} \AgdaFunction{]tm}\AgdaSymbol{)}\<%
\\
\>[0]\AgdaIndent{2}{}\<[2]%
\>[2]\AgdaFunction{⟦} \AgdaFunction{sym} \AgdaSymbol{(}\AgdaFunction{trans} \AgdaFunction{wk-hom} \AgdaSymbol{(}\AgdaFunction{hom≡} \AgdaSymbol{(}\AgdaFunction{htrans} \AgdaSymbol{(}\AgdaFunction{cohOp} \AgdaFunction{+T[,]T}\AgdaSymbol{)} \<[51]%
\>[51]\<%
\\
\>[2]\AgdaIndent{4}{}\<[4]%
\>[4]\AgdaSymbol{(}\AgdaFunction{cohOp} \AgdaFunction{eq1}\AgdaSymbol{))} \<[17]%
\>[17]\<%
\\
\>[2]\AgdaIndent{4}{}\<[4]%
\>[4]\AgdaSymbol{(}\AgdaFunction{cohOp} \AgdaFunction{eq2}\AgdaSymbol{)))} \AgdaFunction{⟫}\<%
\\
%
\\
\>[0]\AgdaIndent{2}{}\<[2]%
\>[2]\AgdaKeyword{where} \<[8]%
\>[8]\<%
\\
\>[0]\AgdaIndent{4}{}\<[4]%
\>[4]\AgdaFunction{wk-id} \AgdaSymbol{:} \AgdaSymbol{(}\AgdaBound{Γ} \AgdaInductiveConstructor{,} \AgdaBound{A} \AgdaInductiveConstructor{,} \AgdaBound{A} \AgdaFunction{+T} \AgdaBound{A}\AgdaSymbol{)} \AgdaDatatype{⇒} \AgdaBound{Γ} \<[33]%
\>[33]\<%
\\
\>[0]\AgdaIndent{4}{}\<[4]%
\>[4]\AgdaFunction{wk-id} \AgdaSymbol{=} \AgdaSymbol{(}\AgdaFunction{IdCm} \AgdaFunction{+S} \AgdaBound{A}\AgdaSymbol{)} \AgdaFunction{+S} \AgdaSymbol{(}\AgdaBound{A} \AgdaFunction{+T} \AgdaBound{A}\AgdaSymbol{)}\<%
\\
\>[0]\AgdaIndent{2}{}\<[2]%
\>[2]\<%
\\
\>[0]\AgdaIndent{4}{}\<[4]%
\>[4]\AgdaFunction{eq1} \AgdaSymbol{:} \AgdaBound{A} \AgdaFunction{[} \AgdaFunction{wk-id} \AgdaFunction{]T} \AgdaDatatype{≡} \AgdaSymbol{(}\AgdaBound{A} \AgdaFunction{+T} \AgdaBound{A}\AgdaSymbol{)} \AgdaFunction{+T} \AgdaSymbol{(}\AgdaBound{A} \AgdaFunction{+T} \AgdaBound{A}\AgdaSymbol{)} \<[46]%
\>[46]\<%
\\
\>[0]\AgdaIndent{4}{}\<[4]%
\>[4]\AgdaFunction{eq1} \AgdaSymbol{=} \AgdaFunction{wk+S+T} \AgdaSymbol{(}\AgdaFunction{wk+S+T} \AgdaFunction{IC-T}\AgdaSymbol{)}\<%
\\
%
\\
\>[0]\AgdaIndent{4}{}\<[4]%
\>[4]\AgdaFunction{eq2} \AgdaSymbol{:} \AgdaSymbol{(}\AgdaBound{A} \AgdaFunction{+T} \AgdaBound{A}\AgdaSymbol{)} \AgdaFunction{[} \AgdaFunction{wk-id} \AgdaInductiveConstructor{,} \AgdaSymbol{(}\AgdaInductiveConstructor{var} \AgdaInductiveConstructor{v0} \AgdaFunction{⟦} \AgdaFunction{eq1} \AgdaFunction{⟫}\AgdaSymbol{)} \AgdaFunction{]T} \<[49]%
\>[49]\<%
\\
\>[4]\AgdaIndent{12}{}\<[12]%
\>[12]\AgdaDatatype{≡} \AgdaSymbol{(}\AgdaBound{A} \AgdaFunction{+T} \AgdaBound{A}\AgdaSymbol{)} \AgdaFunction{+T} \AgdaSymbol{(}\AgdaBound{A} \AgdaFunction{+T} \AgdaBound{A}\AgdaSymbol{)} \<[35]%
\>[35]\<%
\\
\>[0]\AgdaIndent{4}{}\<[4]%
\>[4]\AgdaFunction{eq2} \AgdaSymbol{=} \AgdaFunction{trans} \AgdaFunction{+T[,]T} \AgdaFunction{eq1}\<%
\\
%
\\
\>\AgdaFunction{Fun-sym} \AgdaSymbol{:} \AgdaSymbol{(}\AgdaBound{Γ} \AgdaSymbol{:} \AgdaDatatype{Con}\AgdaSymbol{)(}\AgdaBound{A} \AgdaSymbol{:} \AgdaDatatype{Ty} \AgdaBound{Γ}\AgdaSymbol{)(}\AgdaBound{a} \AgdaBound{b} \AgdaSymbol{:} \AgdaDatatype{Tm} \AgdaBound{A}\AgdaSymbol{)} \AgdaSymbol{→}\<%
\\
\>[0]\AgdaIndent{10}{}\<[10]%
\>[10]\AgdaDatatype{Tm} \AgdaSymbol{(}\AgdaBound{a} \AgdaInductiveConstructor{=h} \AgdaBound{b}\AgdaSymbol{)}\<%
\\
\>[0]\AgdaIndent{7}{}\<[7]%
\>[7]\AgdaSymbol{→} \AgdaDatatype{Tm} \AgdaSymbol{(}\AgdaBound{b} \AgdaInductiveConstructor{=h} \AgdaBound{a}\AgdaSymbol{)}\<%
\\
\>\AgdaFunction{Fun-sym} \AgdaBound{Γ} \AgdaBound{A} \AgdaBound{a} \AgdaBound{b} \AgdaBound{t} \AgdaSymbol{=} \AgdaSymbol{(}\AgdaFunction{sym-Tm} \AgdaBound{A}\AgdaSymbol{)} \AgdaFunction{[} \AgdaFunction{rpl-sub} \AgdaBound{Γ} \AgdaBound{A} \AgdaBound{a} \AgdaBound{b} \AgdaBound{t} \AgdaFunction{]tm} \<[55]%
\>[55]\<%
\\
\>[0]\AgdaIndent{9}{}\<[9]%
\>[9]\AgdaFunction{⟦} \AgdaFunction{sym} \AgdaSymbol{(}\AgdaFunction{trans} \AgdaSymbol{(}\AgdaFunction{rpl-T-p3-wk} \AgdaSymbol{(}\AgdaInductiveConstructor{ε} \AgdaInductiveConstructor{,} \AgdaInductiveConstructor{*} \AgdaInductiveConstructor{,} \AgdaInductiveConstructor{*}\AgdaSymbol{)} \AgdaBound{A}\AgdaSymbol{)} \AgdaSymbol{(}\AgdaFunction{trans} \AgdaSymbol{(}\AgdaFunction{congT} \AgdaSymbol{(}\AgdaFunction{rpl-T-p2} \AgdaSymbol{(}\AgdaInductiveConstructor{ε} \AgdaInductiveConstructor{,} \AgdaInductiveConstructor{*} \AgdaInductiveConstructor{,} \AgdaInductiveConstructor{*}\AgdaSymbol{)} \AgdaBound{A}\AgdaSymbol{))} \<[90]%
\>[90]\<%
\\
\>[9]\AgdaIndent{11}{}\<[11]%
\>[11]\AgdaSymbol{(}\AgdaFunction{hom≡} \AgdaSymbol{(}\AgdaFunction{rpl-tm-v0} \AgdaSymbol{(}\AgdaInductiveConstructor{ε} \AgdaInductiveConstructor{,} \AgdaInductiveConstructor{*}\AgdaSymbol{)} \AgdaBound{A} \AgdaSymbol{(}\AgdaFunction{cohOp} \AgdaSymbol{(}\AgdaFunction{rpl*-A2} \AgdaBound{A}\AgdaSymbol{)))} \AgdaSymbol{(}\AgdaFunction{htrans} \AgdaSymbol{(}\AgdaFunction{rpl-tm-vS} \AgdaSymbol{(}\AgdaInductiveConstructor{ε} \AgdaInductiveConstructor{,} \AgdaInductiveConstructor{*}\AgdaSymbol{)} \AgdaBound{A}\AgdaSymbol{)}\<%
\\
\>[11]\AgdaIndent{17}{}\<[17]%
\>[17]\AgdaSymbol{(}\AgdaFunction{rpl*-a} \AgdaBound{A}\AgdaSymbol{)))))} \AgdaFunction{⟫}\<%
\\
%
\\
%
\\
\>\<\end{code}
}
For each of reflexivity, symmetry and transitivity we can construct appropriate coherence 2-cells witnessing the groupoid laws. The base case for variable contexts is proved simply using contractibility as well. However the types of these laws are not as trivial as the proving parts. We use substitution to define the application of the three basic terms we have defined above.

\AgdaHide{
\begin{code}\>\<%
\\
%
\\
\>\AgdaFunction{reflX} \AgdaSymbol{:} \AgdaDatatype{Tm} \AgdaSymbol{(}\AgdaFunction{vX} \AgdaInductiveConstructor{=h} \AgdaFunction{vX}\AgdaSymbol{)}\<%
\\
\>\AgdaFunction{reflX} \AgdaSymbol{=} \AgdaFunction{refl-Tm} \AgdaInductiveConstructor{*} \AgdaFunction{+tm} \AgdaSymbol{\_} \AgdaFunction{+tm} \AgdaSymbol{\_}\<%
\\
%
\\
\>\AgdaFunction{reflY} \AgdaSymbol{:} \AgdaDatatype{Tm} \AgdaSymbol{(}\AgdaFunction{vY} \AgdaInductiveConstructor{=h} \AgdaFunction{vY}\AgdaSymbol{)}\<%
\\
\>\AgdaFunction{reflY} \AgdaSymbol{=} \AgdaFunction{refl-Tm} \AgdaInductiveConstructor{*} \AgdaFunction{+tm} \AgdaSymbol{\_}\<%
\\
%
\\
\>\AgdaFunction{m:*,n:*,α:m=n,p:*,β:n=p,q:*,γ:p=q} \AgdaSymbol{:} \AgdaDatatype{Con}\<%
\\
\>\AgdaFunction{m:*,n:*,α:m=n,p:*,β:n=p,q:*,γ:p=q} \AgdaSymbol{=} \AgdaFunction{x:*,y:*,α:x=y,z:*,β:y=z} \AgdaInductiveConstructor{,} \AgdaInductiveConstructor{*} \AgdaInductiveConstructor{,} \AgdaSymbol{(}\AgdaInductiveConstructor{var} \AgdaSymbol{(}\AgdaInductiveConstructor{vS} \AgdaSymbol{(}\AgdaInductiveConstructor{vS} \AgdaInductiveConstructor{v0}\AgdaSymbol{))} \AgdaInductiveConstructor{=h} \AgdaInductiveConstructor{var} \AgdaInductiveConstructor{v0}\AgdaSymbol{)}\<%
\\
%
\\
\>\AgdaFunction{vM} \AgdaSymbol{:} \AgdaDatatype{Tm} \AgdaSymbol{\{}\AgdaFunction{m:*,n:*,α:m=n,p:*,β:n=p,q:*,γ:p=q}\AgdaSymbol{\}} \AgdaInductiveConstructor{*}\<%
\\
\>\AgdaFunction{vM} \AgdaSymbol{=} \AgdaInductiveConstructor{var} \AgdaSymbol{(}\AgdaInductiveConstructor{vS} \AgdaSymbol{(}\AgdaInductiveConstructor{vS} \AgdaSymbol{(}\AgdaInductiveConstructor{vS} \AgdaSymbol{(}\AgdaInductiveConstructor{vS} \AgdaSymbol{(}\AgdaInductiveConstructor{vS} \AgdaSymbol{(}\AgdaInductiveConstructor{vS} \AgdaInductiveConstructor{v0}\AgdaSymbol{))))))}\<%
\\
%
\\
\>\AgdaFunction{vN} \AgdaSymbol{:} \AgdaDatatype{Tm} \AgdaSymbol{\{}\AgdaFunction{m:*,n:*,α:m=n,p:*,β:n=p,q:*,γ:p=q}\AgdaSymbol{\}} \AgdaInductiveConstructor{*}\<%
\\
\>\AgdaFunction{vN} \AgdaSymbol{=} \AgdaInductiveConstructor{var} \AgdaSymbol{(}\AgdaInductiveConstructor{vS} \AgdaSymbol{(}\AgdaInductiveConstructor{vS} \AgdaSymbol{(}\AgdaInductiveConstructor{vS} \AgdaSymbol{(}\AgdaInductiveConstructor{vS} \AgdaSymbol{(}\AgdaInductiveConstructor{vS} \AgdaInductiveConstructor{v0}\AgdaSymbol{)))))}\<%
\\
%
\\
\>\AgdaFunction{vMN} \AgdaSymbol{:} \AgdaDatatype{Tm} \AgdaSymbol{\{}\AgdaFunction{m:*,n:*,α:m=n,p:*,β:n=p,q:*,γ:p=q}\AgdaSymbol{\}} \AgdaSymbol{(}\AgdaFunction{vM} \AgdaInductiveConstructor{=h} \AgdaFunction{vN}\AgdaSymbol{)}\<%
\\
\>\AgdaFunction{vMN} \AgdaSymbol{=} \AgdaInductiveConstructor{var} \AgdaSymbol{(}\AgdaInductiveConstructor{vS} \AgdaSymbol{(}\AgdaInductiveConstructor{vS} \AgdaSymbol{(}\AgdaInductiveConstructor{vS} \AgdaSymbol{(}\AgdaInductiveConstructor{vS} \AgdaInductiveConstructor{v0}\AgdaSymbol{))))}\<%
\\
%
\\
\>\AgdaFunction{vP} \AgdaSymbol{:} \AgdaDatatype{Tm} \AgdaSymbol{\{}\AgdaFunction{m:*,n:*,α:m=n,p:*,β:n=p,q:*,γ:p=q}\AgdaSymbol{\}} \AgdaInductiveConstructor{*}\<%
\\
\>\AgdaFunction{vP} \AgdaSymbol{=} \AgdaInductiveConstructor{var} \AgdaSymbol{(}\AgdaInductiveConstructor{vS} \AgdaSymbol{(}\AgdaInductiveConstructor{vS} \AgdaSymbol{(}\AgdaInductiveConstructor{vS} \AgdaInductiveConstructor{v0}\AgdaSymbol{)))}\<%
\\
%
\\
\>\AgdaFunction{vNP} \AgdaSymbol{:} \AgdaDatatype{Tm} \AgdaSymbol{\{}\AgdaFunction{m:*,n:*,α:m=n,p:*,β:n=p,q:*,γ:p=q}\AgdaSymbol{\}} \AgdaSymbol{(}\AgdaFunction{vN} \AgdaInductiveConstructor{=h} \AgdaFunction{vP}\AgdaSymbol{)}\<%
\\
\>\AgdaFunction{vNP} \AgdaSymbol{=} \AgdaInductiveConstructor{var} \AgdaSymbol{(}\AgdaInductiveConstructor{vS} \AgdaSymbol{(}\AgdaInductiveConstructor{vS} \AgdaInductiveConstructor{v0}\AgdaSymbol{))}\<%
\\
%
\\
\>\AgdaFunction{vQ} \AgdaSymbol{:} \AgdaDatatype{Tm} \AgdaSymbol{\{}\AgdaFunction{m:*,n:*,α:m=n,p:*,β:n=p,q:*,γ:p=q}\AgdaSymbol{\}} \AgdaInductiveConstructor{*}\<%
\\
\>\AgdaFunction{vQ} \AgdaSymbol{=} \AgdaInductiveConstructor{var} \AgdaSymbol{(}\AgdaInductiveConstructor{vS} \AgdaInductiveConstructor{v0}\AgdaSymbol{)}\<%
\\
%
\\
\>\AgdaFunction{vPQ} \AgdaSymbol{:} \AgdaDatatype{Tm} \AgdaSymbol{\{}\AgdaFunction{m:*,n:*,α:m=n,p:*,β:n=p,q:*,γ:p=q}\AgdaSymbol{\}} \AgdaSymbol{(}\AgdaFunction{vP} \AgdaInductiveConstructor{=h} \AgdaFunction{vQ}\AgdaSymbol{)}\<%
\\
\>\AgdaFunction{vPQ} \AgdaSymbol{=} \AgdaInductiveConstructor{var} \AgdaInductiveConstructor{v0}\<%
\\
%
\\
\>\AgdaFunction{Ty-G-assoc*} \AgdaSymbol{:} \AgdaDatatype{Ty} \AgdaFunction{m:*,n:*,α:m=n,p:*,β:n=p,q:*,γ:p=q}\<%
\\
\>\AgdaFunction{Ty-G-assoc*} \AgdaSymbol{=} \AgdaSymbol{(}\AgdaFunction{trans*-Tm} \AgdaFunction{[} \AgdaSymbol{((((}\AgdaInductiveConstructor{•} \AgdaInductiveConstructor{,} \AgdaFunction{vM}\AgdaSymbol{)} \AgdaInductiveConstructor{,} \AgdaFunction{vP}\AgdaSymbol{)} \AgdaInductiveConstructor{,} \<[47]%
\>[47]\<%
\\
\>[17]\AgdaIndent{23}{}\<[23]%
\>[23]\AgdaSymbol{(}\AgdaFunction{trans*-Tm} \AgdaFunction{[} \AgdaFunction{pr1} \AgdaFunction{⊚} \AgdaFunction{pr1} \AgdaFunction{]tm}\AgdaSymbol{))} \AgdaInductiveConstructor{,} \AgdaFunction{vQ}\AgdaSymbol{)} \AgdaInductiveConstructor{,} \AgdaFunction{vPQ} \AgdaFunction{]tm} \AgdaInductiveConstructor{=h} \<[71]%
\>[71]\<%
\\
\>[-7]\AgdaIndent{13}{}\<[13]%
\>[13]\AgdaFunction{trans*-Tm} \AgdaFunction{[} \AgdaSymbol{(}\AgdaFunction{pr1} \AgdaFunction{⊚} \AgdaFunction{pr1} \AgdaFunction{⊚} \AgdaFunction{pr1} \AgdaFunction{⊚} \AgdaFunction{pr1} \AgdaInductiveConstructor{,} \AgdaFunction{vQ}\AgdaSymbol{)} \AgdaInductiveConstructor{,} \<[56]%
\>[56]\<%
\\
\>[0]\AgdaIndent{23}{}\<[23]%
\>[23]\AgdaSymbol{(}\AgdaFunction{trans*-Tm} \AgdaFunction{[} \AgdaSymbol{((((}\AgdaInductiveConstructor{•} \AgdaInductiveConstructor{,} \AgdaFunction{vN}\AgdaSymbol{)} \AgdaInductiveConstructor{,} \AgdaFunction{vP}\AgdaSymbol{)} \AgdaInductiveConstructor{,} \AgdaFunction{vNP}\AgdaSymbol{)} \AgdaInductiveConstructor{,} \AgdaFunction{vQ}\AgdaSymbol{)} \AgdaInductiveConstructor{,} \AgdaFunction{vPQ} \AgdaFunction{]tm}\AgdaSymbol{)} \AgdaFunction{]tm}\AgdaSymbol{)}\<%
\\
%
\\
\>\<\end{code}
}

\begin{code}\>\<%
\\
\>\AgdaFunction{Tm-right-identity*} \AgdaSymbol{:} \AgdaDatatype{Tm} \AgdaSymbol{\{}\AgdaFunction{x:*,y:*,α:x=y}\AgdaSymbol{\}}\<%
\\
\>[0]\AgdaIndent{9}{}\<[9]%
\>[9]\AgdaSymbol{(}\AgdaFunction{trans*-Tm} \AgdaFunction{[} \AgdaFunction{IdCm} \AgdaInductiveConstructor{,} \AgdaFunction{vY} \AgdaInductiveConstructor{,} \AgdaFunction{reflY} \AgdaFunction{]tm} \AgdaInductiveConstructor{=h} \AgdaFunction{vα}\AgdaSymbol{)}\<%
\\
\>\AgdaFunction{Tm-right-identity*} \AgdaSymbol{=} \AgdaFunction{Coh-Contr} \AgdaSymbol{(}\AgdaInductiveConstructor{ext} \AgdaInductiveConstructor{c*} \AgdaInductiveConstructor{v0}\AgdaSymbol{)}\<%
\\
%
\\
\>\AgdaFunction{Tm-left-identity*} \AgdaSymbol{:} \AgdaDatatype{Tm} \AgdaSymbol{\{}\AgdaFunction{x:*,y:*,α:x=y}\AgdaSymbol{\}}\<%
\\
\>[0]\AgdaIndent{9}{}\<[9]%
\>[9]\AgdaSymbol{(}\AgdaFunction{trans*-Tm} \AgdaFunction{[} \AgdaSymbol{((}\AgdaFunction{IdCm} \AgdaFunction{⊚} \AgdaFunction{pr1} \AgdaFunction{⊚} \AgdaFunction{pr1}\AgdaSymbol{)} \AgdaInductiveConstructor{,} \AgdaFunction{vX}\AgdaSymbol{)} \AgdaInductiveConstructor{,}\<%
\\
\>[9]\AgdaIndent{10}{}\<[10]%
\>[10]\AgdaFunction{reflX} \AgdaInductiveConstructor{,} \AgdaFunction{vY} \AgdaInductiveConstructor{,} \AgdaFunction{vα} \AgdaFunction{]tm} \AgdaInductiveConstructor{=h} \AgdaFunction{vα}\AgdaSymbol{)}\<%
\\
\>\AgdaFunction{Tm-left-identity*} \AgdaSymbol{=} \AgdaFunction{Coh-Contr} \AgdaSymbol{(}\AgdaInductiveConstructor{ext} \AgdaInductiveConstructor{c*} \AgdaInductiveConstructor{v0}\AgdaSymbol{)}\<%
\\
%
\\
\>\AgdaFunction{Tm-right-inverse*} \AgdaSymbol{:} \AgdaDatatype{Tm} \AgdaSymbol{\{}\AgdaFunction{x:*,y:*,α:x=y}\AgdaSymbol{\}}\<%
\\
\>[0]\AgdaIndent{9}{}\<[9]%
\>[9]\AgdaSymbol{(}\AgdaFunction{trans*-Tm} \AgdaFunction{[} \AgdaSymbol{(}\AgdaFunction{IdCm} \AgdaInductiveConstructor{,} \AgdaFunction{vX}\AgdaSymbol{)} \AgdaInductiveConstructor{,} \AgdaFunction{sym*-Tm} \AgdaFunction{]tm} \AgdaInductiveConstructor{=h} \AgdaFunction{reflX}\AgdaSymbol{)}\<%
\\
\>\AgdaFunction{Tm-right-inverse*} \AgdaSymbol{=} \AgdaFunction{Coh-Contr} \AgdaSymbol{(}\AgdaInductiveConstructor{ext} \AgdaInductiveConstructor{c*} \AgdaInductiveConstructor{v0}\AgdaSymbol{)}\<%
\\
%
\\
%
\\
%
\\
\>\AgdaFunction{Tm-left-inverse*} \AgdaSymbol{:} \AgdaDatatype{Tm} \AgdaSymbol{\{}\AgdaFunction{x:*,y:*,α:x=y}\AgdaSymbol{\}}\<%
\\
\>[0]\AgdaIndent{9}{}\<[9]%
\>[9]\AgdaSymbol{(}\AgdaFunction{trans*-Tm} \AgdaFunction{[} \AgdaSymbol{((}\AgdaInductiveConstructor{•} \AgdaInductiveConstructor{,} \AgdaFunction{vY}\AgdaSymbol{)} \AgdaInductiveConstructor{,} \AgdaFunction{vX} \AgdaInductiveConstructor{,} \AgdaFunction{sym*-Tm} \AgdaInductiveConstructor{,} \AgdaFunction{vY}\AgdaSymbol{)} \AgdaInductiveConstructor{,} \AgdaFunction{vα} \AgdaFunction{]tm} \AgdaInductiveConstructor{=h} \AgdaFunction{reflY}\AgdaSymbol{)}\<%
\\
\>\AgdaFunction{Tm-left-inverse*} \AgdaSymbol{=} \AgdaFunction{Coh-Contr} \AgdaSymbol{(}\AgdaInductiveConstructor{ext} \AgdaInductiveConstructor{c*} \AgdaInductiveConstructor{v0}\AgdaSymbol{)}\<%
\\
%
\\
\>\AgdaFunction{Tm-G-assoc*} \AgdaSymbol{:} \AgdaDatatype{Tm} \AgdaFunction{Ty-G-assoc*}\<%
\\
\>\AgdaFunction{Tm-G-assoc*} \AgdaSymbol{=} \AgdaFunction{Coh-Contr} \AgdaSymbol{(}\AgdaInductiveConstructor{ext} \AgdaSymbol{(}\AgdaInductiveConstructor{ext} \AgdaSymbol{(}\AgdaInductiveConstructor{ext} \AgdaInductiveConstructor{c*} \AgdaInductiveConstructor{v0}\AgdaSymbol{)} \AgdaSymbol{(}\AgdaInductiveConstructor{vS} \AgdaInductiveConstructor{v0}\AgdaSymbol{))} \AgdaSymbol{(}\AgdaInductiveConstructor{vS} \AgdaInductiveConstructor{v0}\AgdaSymbol{))}\<%
\\
\>\<\end{code}
\noindent Their general versions are defined using replacement. For instance, for associativity, we define:

\begin{code}\>\<%
\\
\>\AgdaFunction{Tm-G-assoc} \<[14]%
\>[14]\AgdaSymbol{:} \AgdaSymbol{∀\{}\AgdaBound{Γ}\AgdaSymbol{\}(}\AgdaBound{A} \AgdaSymbol{:} \AgdaDatatype{Ty} \AgdaBound{Γ}\AgdaSymbol{)} \AgdaSymbol{→} \AgdaDatatype{Tm} \AgdaSymbol{(}\AgdaFunction{rpl-T} \AgdaBound{A} \AgdaFunction{Ty-G-assoc*}\AgdaSymbol{)}\<%
\\
\>\AgdaFunction{Tm-G-assoc} \AgdaBound{A} \<[14]%
\>[14]\AgdaSymbol{=} \AgdaFunction{rpl-tm} \AgdaBound{A} \AgdaFunction{Tm-G-assoc*} \<[37]%
\>[37]\<%
\\
\>\<\end{code}
Following the same pattern, the n-level groupoid laws can be obtained as the coherence constants as well.

%\input{Appendices/Codes/latex/Telescopes2}


\input{Appendices/Codes/latex/GlobularTypes}



\AgdaHide{

\begin{code}\>\<%
\\
\>\AgdaSymbol{\{-\#} \AgdaKeyword{OPTIONS} --no-positivity-check --no-termination-check \AgdaSymbol{\#-\}}\<%
\\
%
\\
%
\\
%
\\
%
\\
\>\AgdaKeyword{module} \AgdaModule{Semantics} \AgdaKeyword{where}\<%
\\
%
\\
\>\AgdaKeyword{open} \AgdaKeyword{import} \AgdaModule{BasicSyntax}\<%
\\
\>\AgdaKeyword{open} \AgdaKeyword{import} \AgdaModule{IdentityContextMorphisms}\<%
\\
\>\AgdaKeyword{open} \AgdaKeyword{import} \AgdaModule{Data.Unit}\<%
\\
\>\AgdaKeyword{open} \AgdaKeyword{import} \AgdaModule{Data.Product}\<%
\\
\>\AgdaKeyword{open} \AgdaKeyword{import} \AgdaModule{Coinduction}\<%
\\
\>\AgdaKeyword{open} \AgdaKeyword{import} \AgdaModule{Relation.Binary.PropositionalEquality} \AgdaKeyword{hiding} \AgdaSymbol{(}[\_]\AgdaSymbol{)}\<%
\\
\>\AgdaKeyword{open} \AgdaKeyword{import} \AgdaModule{GroupoidStructure}\<%
\\
%
\\
\>\AgdaKeyword{open} \AgdaKeyword{import} \AgdaModule{GlobularTypes}\<%
\\
%
\\
%
\\
\>\AgdaFunction{coerce} \AgdaSymbol{:} \AgdaSymbol{\{}\AgdaBound{A} \AgdaBound{B} \AgdaSymbol{:} \AgdaPrimitiveType{Set}\AgdaSymbol{\}} \AgdaSymbol{→} \AgdaBound{B} \AgdaDatatype{≡} \AgdaBound{A} \AgdaSymbol{→} \AgdaBound{A} \AgdaSymbol{→} \AgdaBound{B}\<%
\\
\>\AgdaFunction{coerce} \AgdaInductiveConstructor{refl} \AgdaBound{a} \AgdaSymbol{=} \AgdaBound{a}\<%
\\
%
\\
\>\AgdaFunction{⊤-uni} \AgdaSymbol{:} \AgdaSymbol{∀} \AgdaSymbol{\{}\AgdaBound{A} \AgdaSymbol{:} \AgdaPrimitiveType{Set}\AgdaSymbol{\}\{}\AgdaBound{a} \AgdaBound{b} \AgdaSymbol{:} \AgdaBound{A}\AgdaSymbol{\}} \AgdaSymbol{→} \AgdaBound{A} \AgdaDatatype{≡} \AgdaRecord{⊤} \AgdaSymbol{→} \AgdaBound{a} \AgdaDatatype{≡} \AgdaBound{b}\<%
\\
\>\AgdaFunction{⊤-uni} \AgdaInductiveConstructor{refl} \AgdaSymbol{=} \AgdaInductiveConstructor{refl}\<%
\\
%
\\
\>\<\end{code}
}
Given a globular type $G$, we can interpret the syntactic objects.

\begin{code}\>\<%
\\
\>\AgdaKeyword{record} \AgdaRecord{Semantic} \AgdaSymbol{(}\AgdaBound{G} \AgdaSymbol{:} \AgdaRecord{Glob}\AgdaSymbol{)} \AgdaSymbol{:} \AgdaPrimitiveType{Set₁} \AgdaKeyword{where}\<%
\\
\>[0]\AgdaIndent{2}{}\<[2]%
\>[2]\AgdaKeyword{field}\<%
\\
\>[2]\AgdaIndent{4}{}\<[4]%
\>[4]\AgdaField{⟦\_⟧C} \<[10]%
\>[10]\AgdaSymbol{:} \AgdaDatatype{Con} \AgdaSymbol{→} \AgdaPrimitiveType{Set}\<%
\\
\>[2]\AgdaIndent{4}{}\<[4]%
\>[4]\AgdaField{⟦\_⟧T} \<[10]%
\>[10]\AgdaSymbol{:} \AgdaSymbol{∀\{}\AgdaBound{Γ}\AgdaSymbol{\}} \AgdaSymbol{→} \AgdaDatatype{Ty} \AgdaBound{Γ} \AgdaSymbol{→} \AgdaBound{⟦} \AgdaBound{Γ} \AgdaBound{⟧C} \AgdaSymbol{→} \AgdaRecord{Glob}\<%
\\
\>[2]\AgdaIndent{4}{}\<[4]%
\>[4]\AgdaField{⟦\_⟧tm} \AgdaSymbol{:} \AgdaSymbol{∀\{}\AgdaBound{Γ} \AgdaBound{A}\AgdaSymbol{\}} \AgdaSymbol{→} \AgdaDatatype{Tm} \AgdaBound{A} \AgdaSymbol{→} \AgdaSymbol{(}\AgdaBound{γ} \AgdaSymbol{:} \AgdaBound{⟦} \AgdaBound{Γ} \AgdaBound{⟧C}\AgdaSymbol{)} \AgdaSymbol{→} \AgdaFunction{∣} \AgdaBound{⟦} \AgdaBound{A} \AgdaBound{⟧T} \AgdaBound{γ} \AgdaFunction{∣}\<%
\\
\>[2]\AgdaIndent{4}{}\<[4]%
\>[4]\AgdaField{⟦\_⟧cm} \AgdaSymbol{:} \AgdaSymbol{∀\{}\AgdaBound{Γ} \AgdaBound{Δ}\AgdaSymbol{\}} \AgdaSymbol{→} \AgdaBound{Γ} \AgdaDatatype{⇒} \AgdaBound{Δ} \AgdaSymbol{→} \AgdaBound{⟦} \AgdaBound{Γ} \AgdaBound{⟧C} \AgdaSymbol{→} \AgdaBound{⟦} \AgdaBound{Δ} \AgdaBound{⟧C}\<%
\\
\>[2]\AgdaIndent{4}{}\<[4]%
\>[4]\AgdaField{π} \<[10]%
\>[10]\AgdaSymbol{:} \AgdaSymbol{∀\{}\AgdaBound{Γ} \AgdaBound{A}\AgdaSymbol{\}} \AgdaSymbol{→} \AgdaDatatype{Var} \AgdaBound{A} \AgdaSymbol{→} \AgdaSymbol{(}\AgdaBound{γ} \AgdaSymbol{:} \AgdaBound{⟦} \AgdaBound{Γ} \AgdaBound{⟧C}\AgdaSymbol{)} \AgdaSymbol{→} \AgdaFunction{∣} \AgdaBound{⟦} \AgdaBound{A} \AgdaBound{⟧T} \AgdaBound{γ} \AgdaFunction{∣}\<%
\\
\>\<\end{code}
$\AgdaField{π}$ provides the projection of the semantic variable out of a semantic context.

Following are the computation laws for the interpretations of contexts and types.

\begin{code}\>\<%
\\
\>[2]\AgdaIndent{4}{}\<[4]%
\>[4]\AgdaField{⟦\_⟧C-β1} \<[13]%
\>[13]\AgdaSymbol{:} \AgdaBound{⟦} \AgdaInductiveConstructor{ε} \AgdaBound{⟧C} \AgdaDatatype{≡} \AgdaRecord{⊤}\<%
\\
\>[2]\AgdaIndent{4}{}\<[4]%
\>[4]\AgdaField{⟦\_⟧C-β2} \<[13]%
\>[13]\AgdaSymbol{:} \AgdaSymbol{∀} \AgdaSymbol{\{}\AgdaBound{Γ} \AgdaBound{A}\AgdaSymbol{\}} \AgdaSymbol{→} \AgdaBound{⟦} \AgdaBound{Γ} \AgdaInductiveConstructor{,} \AgdaBound{A} \AgdaBound{⟧C} \AgdaDatatype{≡} \<[38]%
\>[38]\<%
\\
\>[4]\AgdaIndent{25}{}\<[25]%
\>[25]\AgdaRecord{Σ} \AgdaBound{⟦} \AgdaBound{Γ} \AgdaBound{⟧C} \AgdaSymbol{(λ} \AgdaBound{γ} \<[40]%
\>[40]\AgdaSymbol{→} \AgdaFunction{∣} \AgdaBound{⟦} \AgdaBound{A} \AgdaBound{⟧T} \AgdaBound{γ} \AgdaFunction{∣}\AgdaSymbol{)}\<%
\\
\>[0]\AgdaIndent{4}{}\<[4]%
\>[4]\<%
\\
\>[0]\AgdaIndent{4}{}\<[4]%
\>[4]\AgdaField{⟦\_⟧T-β1} \<[13]%
\>[13]\AgdaSymbol{:} \AgdaSymbol{∀\{}\AgdaBound{Γ}\AgdaSymbol{\}\{}\AgdaBound{γ} \AgdaSymbol{:} \AgdaBound{⟦} \AgdaBound{Γ} \AgdaBound{⟧C}\AgdaSymbol{\}} \AgdaSymbol{→} \AgdaBound{⟦} \AgdaInductiveConstructor{*} \AgdaBound{⟧T} \AgdaBound{γ} \AgdaDatatype{≡} \AgdaBound{G}\<%
\\
\>[0]\AgdaIndent{4}{}\<[4]%
\>[4]\AgdaField{⟦\_⟧T-β2} \<[13]%
\>[13]\AgdaSymbol{:} \AgdaSymbol{∀\{}\AgdaBound{Γ} \AgdaBound{A} \AgdaBound{u} \AgdaBound{v}\AgdaSymbol{\}\{}\AgdaBound{γ} \AgdaSymbol{:} \AgdaBound{⟦} \AgdaBound{Γ} \AgdaBound{⟧C}\AgdaSymbol{\}}\<%
\\
\>[4]\AgdaIndent{13}{}\<[13]%
\>[13]\AgdaSymbol{→} \AgdaBound{⟦} \AgdaBound{u} \AgdaInductiveConstructor{=h} \AgdaBound{v} \AgdaBound{⟧T} \AgdaBound{γ} \AgdaDatatype{≡}\<%
\\
\>[13]\AgdaIndent{15}{}\<[15]%
\>[15]\AgdaFunction{♭} \AgdaSymbol{(}\AgdaFunction{hom} \AgdaSymbol{(}\AgdaBound{⟦} \AgdaBound{A} \AgdaBound{⟧T} \AgdaBound{γ}\AgdaSymbol{)} \AgdaSymbol{(}\AgdaBound{⟦} \AgdaBound{u} \AgdaBound{⟧tm} \AgdaBound{γ}\AgdaSymbol{)} \AgdaSymbol{(}\AgdaBound{⟦} \AgdaBound{v} \AgdaBound{⟧tm} \AgdaBound{γ}\AgdaSymbol{))}\<%
\\
\>\<\end{code}
Semantic substitution and semantic weakening laws are also required.
The semantic substitution properties are essential for dealing with substitutions inside interpretation,

\begin{code}\>\<%
\\
\>[-2]\AgdaIndent{4}{}\<[4]%
\>[4]\AgdaField{semSb-T} \<[13]%
\>[13]\AgdaSymbol{:} \AgdaSymbol{∀} \AgdaSymbol{\{}\AgdaBound{Γ} \AgdaBound{Δ}\AgdaSymbol{\}(}\AgdaBound{A} \AgdaSymbol{:} \AgdaDatatype{Ty} \AgdaBound{Δ}\AgdaSymbol{)(}\AgdaBound{δ} \AgdaSymbol{:} \AgdaBound{Γ} \AgdaDatatype{⇒} \AgdaBound{Δ}\AgdaSymbol{)(}\AgdaBound{γ} \AgdaSymbol{:} \AgdaBound{⟦} \AgdaBound{Γ} \AgdaBound{⟧C}\AgdaSymbol{)}\<%
\\
\>[0]\AgdaIndent{13}{}\<[13]%
\>[13]\AgdaSymbol{→} \AgdaBound{⟦} \AgdaBound{A} \AgdaFunction{[} \AgdaBound{δ} \AgdaFunction{]T} \AgdaBound{⟧T} \AgdaBound{γ} \AgdaDatatype{≡} \AgdaBound{⟦} \AgdaBound{A} \AgdaBound{⟧T} \AgdaSymbol{(}\AgdaBound{⟦} \AgdaBound{δ} \AgdaBound{⟧cm} \AgdaBound{γ}\AgdaSymbol{)}\<%
\\
%
\\
\>[0]\AgdaIndent{4}{}\<[4]%
\>[4]\AgdaField{semSb-tm} \AgdaSymbol{:} \AgdaSymbol{∀\{}\AgdaBound{Γ} \AgdaBound{Δ}\AgdaSymbol{\}\{}\AgdaBound{A} \AgdaSymbol{:} \AgdaDatatype{Ty} \AgdaBound{Δ}\AgdaSymbol{\}(}\AgdaBound{a} \AgdaSymbol{:} \AgdaDatatype{Tm} \AgdaBound{A}\AgdaSymbol{)(}\AgdaBound{δ} \AgdaSymbol{:} \AgdaBound{Γ} \AgdaDatatype{⇒} \AgdaBound{Δ}\AgdaSymbol{)}\<%
\\
\>[0]\AgdaIndent{15}{}\<[15]%
\>[15]\AgdaSymbol{(}\AgdaBound{γ} \AgdaSymbol{:} \AgdaBound{⟦} \AgdaBound{Γ} \AgdaBound{⟧C}\AgdaSymbol{)}\<%
\\
\>[0]\AgdaIndent{13}{}\<[13]%
\>[13]\AgdaSymbol{→} \AgdaFunction{subst} \AgdaFunction{∣\_∣} \AgdaSymbol{(}\AgdaBound{semSb-T} \AgdaBound{A} \AgdaBound{δ} \AgdaBound{γ}\AgdaSymbol{)} \AgdaSymbol{(}\AgdaBound{⟦} \AgdaBound{a} \AgdaFunction{[} \AgdaBound{δ} \AgdaFunction{]tm} \AgdaBound{⟧tm} \AgdaBound{γ}\AgdaSymbol{)}\<%
\\
\>[0]\AgdaIndent{16}{}\<[16]%
\>[16]\AgdaDatatype{≡} \AgdaBound{⟦} \AgdaBound{a} \AgdaBound{⟧tm} \AgdaSymbol{(}\AgdaBound{⟦} \AgdaBound{δ} \AgdaBound{⟧cm} \AgdaBound{γ}\AgdaSymbol{)}\<%
\\
%
\\
\>[0]\AgdaIndent{4}{}\<[4]%
\>[4]\AgdaField{semSb-cm} \AgdaSymbol{:} \AgdaSymbol{∀} \AgdaSymbol{\{}\AgdaBound{Γ} \AgdaBound{Δ} \AgdaBound{Θ}\AgdaSymbol{\}(}\AgdaBound{γ} \AgdaSymbol{:} \AgdaBound{⟦} \AgdaBound{Γ} \AgdaBound{⟧C}\AgdaSymbol{)(}\AgdaBound{δ} \AgdaSymbol{:} \AgdaBound{Γ} \AgdaDatatype{⇒} \AgdaBound{Δ}\AgdaSymbol{)(}\AgdaBound{θ} \AgdaSymbol{:} \AgdaBound{Δ} \AgdaDatatype{⇒} \AgdaBound{Θ}\AgdaSymbol{)}\<%
\\
\>[0]\AgdaIndent{13}{}\<[13]%
\>[13]\AgdaSymbol{→} \AgdaBound{⟦} \AgdaBound{θ} \AgdaFunction{⊚} \AgdaBound{δ} \AgdaBound{⟧cm} \AgdaBound{γ} \AgdaDatatype{≡} \AgdaBound{⟦} \AgdaBound{θ} \AgdaBound{⟧cm} \AgdaSymbol{(}\AgdaBound{⟦} \AgdaBound{δ} \AgdaBound{⟧cm} \AgdaBound{γ}\AgdaSymbol{)}\<%
\\
\>\<\end{code}
Since the computation laws for the interpretations of terms and context morphisms are well typed up to these properties.

\begin{code}\>\<%
\\
\>[0]\AgdaIndent{4}{}\<[4]%
\>[4]\AgdaField{⟦\_⟧tm-β1} \<[14]%
\>[14]\AgdaSymbol{:} \AgdaSymbol{∀\{}\AgdaBound{Γ} \AgdaBound{A}\AgdaSymbol{\}\{}\AgdaBound{x} \AgdaSymbol{:} \AgdaDatatype{Var} \AgdaBound{A}\AgdaSymbol{\}\{}\AgdaBound{γ} \AgdaSymbol{:} \AgdaBound{⟦} \AgdaBound{Γ} \AgdaBound{⟧C}\AgdaSymbol{\}}\<%
\\
\>[0]\AgdaIndent{14}{}\<[14]%
\>[14]\AgdaSymbol{→} \AgdaBound{⟦} \AgdaInductiveConstructor{var} \AgdaBound{x} \AgdaBound{⟧tm} \AgdaBound{γ} \AgdaDatatype{≡} \AgdaBound{π} \AgdaBound{x} \AgdaBound{γ}\<%
\\
%
\\
\>[0]\AgdaIndent{4}{}\<[4]%
\>[4]\AgdaField{⟦\_⟧cm-β1} \<[14]%
\>[14]\AgdaSymbol{:} \AgdaSymbol{∀\{}\AgdaBound{Γ}\AgdaSymbol{\}\{}\AgdaBound{γ} \AgdaSymbol{:} \AgdaBound{⟦} \AgdaBound{Γ} \AgdaBound{⟧C}\AgdaSymbol{\}} \<[33]%
\>[33]\<%
\\
\>[0]\AgdaIndent{13}{}\<[13]%
\>[13]\AgdaSymbol{→} \AgdaBound{⟦} \AgdaInductiveConstructor{•} \AgdaBound{⟧cm} \AgdaBound{γ} \AgdaDatatype{≡} \AgdaFunction{coerce} \AgdaBound{⟦\_⟧C-β1} \AgdaInductiveConstructor{tt}\<%
\\
%
\\
\>[0]\AgdaIndent{4}{}\<[4]%
\>[4]\AgdaField{⟦\_⟧cm-β2} \<[14]%
\>[14]\AgdaSymbol{:} \AgdaSymbol{∀\{}\AgdaBound{Γ} \AgdaBound{Δ}\AgdaSymbol{\}\{}\AgdaBound{A} \AgdaSymbol{:} \AgdaDatatype{Ty} \AgdaBound{Δ}\AgdaSymbol{\}\{}\AgdaBound{δ} \AgdaSymbol{:} \AgdaBound{Γ} \AgdaDatatype{⇒} \AgdaBound{Δ}\AgdaSymbol{\}\{}\AgdaBound{γ} \AgdaSymbol{:} \AgdaBound{⟦} \AgdaBound{Γ} \AgdaBound{⟧C}\AgdaSymbol{\}}\<%
\\
\>[0]\AgdaIndent{16}{}\<[16]%
\>[16]\AgdaSymbol{\{}\AgdaBound{a} \AgdaSymbol{:} \AgdaDatatype{Tm} \AgdaSymbol{(}\AgdaBound{A} \AgdaFunction{[} \AgdaBound{δ} \AgdaFunction{]T}\AgdaSymbol{)\}} \AgdaSymbol{→} \AgdaBound{⟦} \AgdaBound{δ} \AgdaInductiveConstructor{,} \AgdaBound{a} \AgdaBound{⟧cm} \AgdaBound{γ} \<[52]%
\>[52]\<%
\\
\>[0]\AgdaIndent{14}{}\<[14]%
\>[14]\AgdaDatatype{≡} \AgdaFunction{coerce} \AgdaBound{⟦\_⟧C-β2} \AgdaSymbol{((}\AgdaBound{⟦} \AgdaBound{δ} \AgdaBound{⟧cm} \AgdaBound{γ}\AgdaSymbol{)} \AgdaInductiveConstructor{,}\<%
\\
\>[0]\AgdaIndent{16}{}\<[16]%
\>[16]\AgdaFunction{subst} \AgdaFunction{∣\_∣} \AgdaSymbol{(}\AgdaBound{semSb-T} \AgdaBound{A} \AgdaBound{δ} \AgdaBound{γ}\AgdaSymbol{)} \AgdaSymbol{(}\AgdaBound{⟦} \AgdaBound{a} \AgdaBound{⟧tm} \AgdaBound{γ}\AgdaSymbol{))}\<%
\\
\>\<\end{code}
The semantic weakening properties should actually be deriavable since weakening is equivalent to projection substitution.

\begin{code}\>\<%
\\
\>[0]\AgdaIndent{4}{}\<[4]%
\>[4]\AgdaField{semWk-T} \<[13]%
\>[13]\AgdaSymbol{:} \AgdaSymbol{∀} \AgdaSymbol{\{}\AgdaBound{Γ} \AgdaBound{A} \AgdaBound{B}\AgdaSymbol{\}(}\AgdaBound{γ} \AgdaSymbol{:} \AgdaBound{⟦} \AgdaBound{Γ} \AgdaBound{⟧C}\AgdaSymbol{)(}\AgdaBound{v} \AgdaSymbol{:} \AgdaFunction{∣} \AgdaBound{⟦} \AgdaBound{B} \AgdaBound{⟧T} \AgdaBound{γ} \AgdaFunction{∣}\AgdaSymbol{)}\<%
\\
\>[0]\AgdaIndent{13}{}\<[13]%
\>[13]\AgdaSymbol{→} \AgdaBound{⟦} \AgdaBound{A} \AgdaFunction{+T} \AgdaBound{B} \AgdaBound{⟧T} \AgdaSymbol{(}\AgdaFunction{coerce} \AgdaBound{⟦\_⟧C-β2} \AgdaSymbol{(}\AgdaBound{γ} \AgdaInductiveConstructor{,} \AgdaBound{v}\AgdaSymbol{))} \AgdaDatatype{≡} \<[54]%
\>[54]\<%
\\
\>[13]\AgdaIndent{15}{}\<[15]%
\>[15]\AgdaBound{⟦} \AgdaBound{A} \AgdaBound{⟧T} \AgdaBound{γ}\<%
\\
\>[0]\AgdaIndent{2}{}\<[2]%
\>[2]\<%
\\
\>[0]\AgdaIndent{4}{}\<[4]%
\>[4]\AgdaField{semWk-cm} \<[14]%
\>[14]\AgdaSymbol{:} \AgdaSymbol{∀} \AgdaSymbol{\{}\AgdaBound{Γ} \AgdaBound{Δ} \AgdaBound{B}\AgdaSymbol{\}\{}\AgdaBound{γ} \AgdaSymbol{:} \AgdaBound{⟦} \AgdaBound{Γ} \AgdaBound{⟧C}\AgdaSymbol{\}\{}\AgdaBound{v} \AgdaSymbol{:} \AgdaFunction{∣} \AgdaBound{⟦} \AgdaBound{B} \AgdaBound{⟧T} \AgdaBound{γ} \AgdaFunction{∣}\AgdaSymbol{\}}\<%
\\
\>[4]\AgdaIndent{14}{}\<[14]%
\>[14]\AgdaSymbol{→} \AgdaSymbol{(}\AgdaBound{δ} \AgdaSymbol{:} \AgdaBound{Γ} \AgdaDatatype{⇒} \AgdaBound{Δ}\AgdaSymbol{)} \AgdaSymbol{→} \AgdaBound{⟦} \AgdaBound{δ} \AgdaFunction{+S} \AgdaBound{B} \AgdaBound{⟧cm} \<[43]%
\>[43]\<%
\\
\>[14]\AgdaIndent{16}{}\<[16]%
\>[16]\AgdaSymbol{(}\AgdaFunction{coerce} \AgdaBound{⟦\_⟧C-β2} \AgdaSymbol{(}\AgdaBound{γ} \AgdaInductiveConstructor{,} \AgdaBound{v}\AgdaSymbol{))} \AgdaDatatype{≡} \AgdaBound{⟦} \AgdaBound{δ} \AgdaBound{⟧cm} \AgdaBound{γ}\<%
\\
%
\\
%
\\
\>[-2]\AgdaIndent{4}{}\<[4]%
\>[4]\AgdaField{semWk-tm} \AgdaSymbol{:} \AgdaSymbol{∀} \AgdaSymbol{\{}\AgdaBound{Γ} \AgdaBound{A} \AgdaBound{B}\AgdaSymbol{\}(}\AgdaBound{γ} \AgdaSymbol{:} \AgdaBound{⟦} \AgdaBound{Γ} \AgdaBound{⟧C}\AgdaSymbol{)(}\AgdaBound{v} \AgdaSymbol{:} \AgdaFunction{∣} \AgdaBound{⟦} \AgdaBound{B} \AgdaBound{⟧T} \AgdaBound{γ} \AgdaFunction{∣}\AgdaSymbol{)}\<%
\\
\>[0]\AgdaIndent{13}{}\<[13]%
\>[13]\AgdaSymbol{→} \AgdaSymbol{(}\AgdaBound{a} \AgdaSymbol{:} \AgdaDatatype{Tm} \AgdaBound{A}\AgdaSymbol{)} \AgdaSymbol{→} \AgdaFunction{subst} \AgdaFunction{∣\_∣} \AgdaSymbol{(}\AgdaBound{semWk-T} \AgdaBound{γ} \AgdaBound{v}\AgdaSymbol{)} \<[52]%
\>[52]\<%
\\
\>[13]\AgdaIndent{15}{}\<[15]%
\>[15]\AgdaSymbol{(}\AgdaBound{⟦} \AgdaBound{a} \AgdaFunction{+tm} \AgdaBound{B} \AgdaBound{⟧tm} \AgdaSymbol{(}\AgdaFunction{coerce} \AgdaBound{⟦\_⟧C-β2} \AgdaSymbol{(}\AgdaBound{γ} \AgdaInductiveConstructor{,} \AgdaBound{v}\AgdaSymbol{)))} \<[56]%
\>[56]\<%
\\
\>[15]\AgdaIndent{17}{}\<[17]%
\>[17]\AgdaDatatype{≡} \AgdaSymbol{(}\AgdaBound{⟦} \AgdaBound{a} \AgdaBound{⟧tm} \AgdaBound{γ}\AgdaSymbol{)}\<%
\\
\>\<\end{code}
Here we declare them as properties because they are essential for the computation laws of function $\AgdaField{π}$.

\begin{code}\>\<%
\\
\>[-2]\AgdaIndent{4}{}\<[4]%
\>[4]\AgdaField{π-β1} \<[10]%
\>[10]\AgdaSymbol{:} \AgdaSymbol{∀\{}\AgdaBound{Γ} \AgdaBound{A}\AgdaSymbol{\}(}\AgdaBound{γ} \AgdaSymbol{:} \AgdaBound{⟦} \AgdaBound{Γ} \AgdaBound{⟧C}\AgdaSymbol{)(}\AgdaBound{v} \AgdaSymbol{:} \AgdaFunction{∣} \AgdaBound{⟦} \AgdaBound{A} \AgdaBound{⟧T} \AgdaBound{γ} \AgdaFunction{∣}\AgdaSymbol{)} \<[49]%
\>[49]\<%
\\
\>[0]\AgdaIndent{10}{}\<[10]%
\>[10]\AgdaSymbol{→} \AgdaFunction{subst} \AgdaFunction{∣\_∣} \AgdaSymbol{(}\AgdaBound{semWk-T} \AgdaBound{γ} \AgdaBound{v}\AgdaSymbol{)} \<[36]%
\>[36]\<%
\\
\>[10]\AgdaIndent{12}{}\<[12]%
\>[12]\AgdaSymbol{(}\AgdaBound{π} \AgdaInductiveConstructor{v0} \AgdaSymbol{(}\AgdaFunction{coerce} \AgdaBound{⟦\_⟧C-β2} \AgdaSymbol{(}\AgdaBound{γ} \AgdaInductiveConstructor{,} \AgdaBound{v}\AgdaSymbol{)))} \AgdaDatatype{≡} \AgdaBound{v}\<%
\\
%
\\
\>[-2]\AgdaIndent{4}{}\<[4]%
\>[4]\AgdaField{π-β2} \<[10]%
\>[10]\AgdaSymbol{:} \AgdaSymbol{∀\{}\AgdaBound{Γ} \AgdaBound{A} \AgdaBound{B}\AgdaSymbol{\}(}\AgdaBound{x} \AgdaSymbol{:} \AgdaDatatype{Var} \AgdaBound{A}\AgdaSymbol{)(}\AgdaBound{γ} \AgdaSymbol{:} \AgdaBound{⟦} \AgdaBound{Γ} \AgdaBound{⟧C}\AgdaSymbol{)(}\AgdaBound{v} \AgdaSymbol{:} \AgdaFunction{∣} \AgdaBound{⟦} \AgdaBound{B} \AgdaBound{⟧T} \AgdaBound{γ} \AgdaFunction{∣}\AgdaSymbol{)} \<[62]%
\>[62]\<%
\\
\>[0]\AgdaIndent{10}{}\<[10]%
\>[10]\AgdaSymbol{→} \AgdaFunction{subst} \AgdaFunction{∣\_∣} \AgdaSymbol{(}\AgdaBound{semWk-T} \AgdaBound{γ} \AgdaBound{v}\AgdaSymbol{)} \AgdaSymbol{(}\AgdaBound{π} \AgdaSymbol{(}\AgdaInductiveConstructor{vS} \AgdaSymbol{\{}\AgdaBound{Γ}\AgdaSymbol{\}} \AgdaSymbol{\{}\AgdaBound{A}\AgdaSymbol{\}} \AgdaSymbol{\{}\AgdaBound{B}\AgdaSymbol{\}} \AgdaBound{x}\AgdaSymbol{)} \<[58]%
\>[58]\<%
\\
\>[10]\AgdaIndent{12}{}\<[12]%
\>[12]\AgdaSymbol{(}\AgdaFunction{coerce} \AgdaBound{⟦\_⟧C-β2} \AgdaSymbol{(}\AgdaBound{γ} \AgdaInductiveConstructor{,} \AgdaBound{v}\AgdaSymbol{)))} \AgdaDatatype{≡} \AgdaBound{π} \AgdaBound{x} \AgdaBound{γ}\<%
\\
\>\<\end{code}
The only part of the semantics where we have any freedom is the interpretation of the coherence constants:

\begin{code}\>\<%
\\
\>[-2]\AgdaIndent{4}{}\<[4]%
\>[4]\AgdaField{⟦coh⟧} \<[11]%
\>[11]\AgdaSymbol{:} \AgdaSymbol{∀\{}\AgdaBound{Θ}\AgdaSymbol{\}} \AgdaSymbol{→} \AgdaDatatype{isContr} \AgdaBound{Θ} \AgdaSymbol{→} \AgdaSymbol{(}\AgdaBound{A} \AgdaSymbol{:} \AgdaDatatype{Ty} \AgdaBound{Θ}\AgdaSymbol{)} \<[43]%
\>[43]\<%
\\
\>[0]\AgdaIndent{11}{}\<[11]%
\>[11]\AgdaSymbol{→} \AgdaSymbol{(}\AgdaBound{θ} \AgdaSymbol{:} \AgdaBound{⟦} \AgdaBound{Θ} \AgdaBound{⟧C}\AgdaSymbol{)} \AgdaSymbol{→} \AgdaFunction{∣} \AgdaBound{⟦} \AgdaBound{A} \AgdaBound{⟧T} \AgdaBound{θ} \AgdaFunction{∣}\<%
\\
\>\<\end{code}
However, we also need to require that the coherence constants are well
behaved wrt to substitution which in turn relies on the interpretation
of all terms. To address this we state the required properties in a
redundant form because the correctness for any other part of the
syntax follows from the defining equations we have already
stated. There seems to be no way to avoid this.

If the underlying globular type is not a globular set we need to add coherence laws, which is not very well understood. On the other hand, restricting ourselves to globular sets means that our prime examle $\AgdaFunction{Idω}$ is not an instance anymore. We should still be able to construct non-trivial globular sets, e.g. by encoding basic topological notions and defining higher homotopies as in a classical framework. However, we don't currently know a simple definition of a globular set which is a weak $\omega$-groupoid. One possibility would be to use the syntax of type theory with equality types. Indeed, we believe that this would be an alternative way to formalize weak $\omega$-groupoids.





%
\AgdaHide{

\begin{code}\>\<%
\\
%
\\
\>\AgdaKeyword{open} \AgdaKeyword{import} \AgdaModule{Level}\<%
\\
\>\AgdaKeyword{open} \AgdaKeyword{import} \AgdaModule{Relation.Binary.PropositionalEquality}\<%
\\
%
\\
\>\AgdaKeyword{module} \AgdaModule{hProp} \AgdaSymbol{(}\AgdaBound{ext} \AgdaSymbol{:} \AgdaFunction{Extensionality} \AgdaPrimitive{zero} \AgdaPrimitive{zero}\AgdaSymbol{)} \AgdaKeyword{where}\<%
\\
%
\\
\>\AgdaKeyword{open} \AgdaKeyword{import} \AgdaModule{Relation.Nullary}\<%
\\
\>\AgdaKeyword{open} \AgdaKeyword{import} \AgdaModule{Data.Unit}\<%
\\
\>\AgdaKeyword{open} \AgdaKeyword{import} \AgdaModule{Data.Empty}\<%
\\
\>\AgdaKeyword{open} \AgdaKeyword{import} \AgdaModule{Data.Nat}\<%
\\
\>\AgdaKeyword{open} \AgdaKeyword{import} \AgdaModule{Data.Product}\<%
\\
%
\\
\>\AgdaKeyword{infixr} \AgdaNumber{2} \_⇒\_\<%
\\
%
\\
\>\AgdaKeyword{infixr} \AgdaNumber{3} \_∧\_\<%
\\
%
\\
%
\\
\>\<\end{code}
}

A proof-irrelvant universe only contains sets with at most one inhabitant. 

\begin{code}\>\<%
\\
%
\\
\>\AgdaKeyword{record} \AgdaRecord{hProp} \AgdaSymbol{:} \AgdaPrimitiveType{Set₁} \AgdaKeyword{where}\<%
\\
\>[0]\AgdaIndent{2}{}\<[2]%
\>[2]\AgdaKeyword{constructor} \AgdaInductiveConstructor{hp}\<%
\\
\>[0]\AgdaIndent{2}{}\<[2]%
\>[2]\AgdaKeyword{field}\<%
\\
\>[2]\AgdaIndent{4}{}\<[4]%
\>[4]\AgdaField{prf} \AgdaSymbol{:} \AgdaPrimitiveType{Set}\<%
\\
\>[2]\AgdaIndent{4}{}\<[4]%
\>[4]\AgdaField{Uni} \AgdaSymbol{:} \AgdaSymbol{\{}\AgdaBound{p} \AgdaBound{q} \AgdaSymbol{:} \AgdaBound{prf}\AgdaSymbol{\}} \AgdaSymbol{→} \AgdaBound{p} \AgdaDatatype{≡} \AgdaBound{q}\<%
\\
%
\\
\>\AgdaKeyword{open} \AgdaModule{hProp} \AgdaKeyword{public} \AgdaKeyword{renaming} \AgdaSymbol{(}prf \AgdaSymbol{to} <\_>\AgdaSymbol{)}\<%
\\
%
\\
\>\<\end{code}

We can extract the proof of any propostion $A : hProp$ by using $<>$ and there is always a proof that all inhabitants of it are the same, in other words, if there is any proof of it, the proof is unique. This is not exactly the same as the $Prop$ universe in Altenkirch's approach which is judgemental. It is just a judgement whether a set behaves like a $Proposition$. The $hProp$ we define above is propositional since we can extract the proof of uniqueness.

We would like to have some basic propositions $\top$ and $\bot$. To distinguish them with the ones for non-proof irrelevant propositions which are already available in Agda library, we add a prime to all similar symbols.

\begin{code}\>\<%
\\
%
\\
\>\AgdaFunction{⊤'} \AgdaSymbol{:} \AgdaRecord{hProp}\<%
\\
\>\AgdaFunction{⊤'} \AgdaSymbol{=} \AgdaInductiveConstructor{hp} \AgdaRecord{⊤} \AgdaInductiveConstructor{refl}\<%
\\
%
\\
\>\AgdaFunction{⊥'} \AgdaSymbol{:} \AgdaRecord{hProp}\<%
\\
\>\AgdaFunction{⊥'} \AgdaSymbol{=} \AgdaInductiveConstructor{hp} \AgdaDatatype{⊥} \AgdaSymbol{(λ} \AgdaSymbol{\{}\AgdaBound{p}\AgdaSymbol{\}} \AgdaSymbol{→} \AgdaFunction{⊥-elim} \AgdaBound{p}\AgdaSymbol{)}\<%
\\
%
\\
\>\<\end{code}

We also want the universal and existential quantifier for $hProp$, namely it is closed under $\Pi$-types and $\Sigma$-types.
The universal quantifier of $hProp$ can be axiomitised but we decide to explicitly state that we 
require the functional extensionality to use this module. The reason is that functional extensionality is actually equivalent to the closure under $\Pi$-types.

\begin{code}\>\<%
\\
%
\\
\>\AgdaFunction{∀'} \AgdaSymbol{:} \AgdaSymbol{(}\AgdaBound{A} \AgdaSymbol{:} \AgdaPrimitiveType{Set}\AgdaSymbol{)(}\AgdaBound{P} \AgdaSymbol{:} \AgdaBound{A} \AgdaSymbol{→} \AgdaRecord{hProp}\AgdaSymbol{)} \AgdaSymbol{→} \AgdaRecord{hProp}\<%
\\
\>\AgdaFunction{∀'} \AgdaBound{A} \AgdaBound{P} \AgdaSymbol{=} \AgdaInductiveConstructor{hp} \AgdaSymbol{((}\AgdaBound{x} \AgdaSymbol{:} \AgdaBound{A}\AgdaSymbol{)} \AgdaSymbol{→} \AgdaFunction{<} \AgdaBound{P} \AgdaBound{x} \AgdaFunction{>}\AgdaSymbol{)} \AgdaSymbol{(}\AgdaBound{ext} \AgdaSymbol{(λ} \AgdaBound{x} \AgdaSymbol{→} \AgdaFunction{Uni} \AgdaSymbol{(}\AgdaBound{P} \AgdaBound{x}\AgdaSymbol{)))}\<%
\\
%
\\
\>\<\end{code}


\AgdaHide{
\begin{code}\>\<%
\\
%
\\
\>\AgdaFunction{sig-eq} \AgdaSymbol{:} \AgdaSymbol{\{}\AgdaBound{A} \AgdaSymbol{:} \AgdaPrimitiveType{Set}\AgdaSymbol{\}\{}\AgdaBound{B} \AgdaSymbol{:} \AgdaBound{A} \AgdaSymbol{→} \AgdaPrimitiveType{Set}\AgdaSymbol{\}\{}\AgdaBound{a} \AgdaBound{b} \AgdaSymbol{:} \AgdaBound{A}\AgdaSymbol{\}} \AgdaSymbol{→} \<[43]%
\>[43]\<%
\\
\>[4]\AgdaIndent{9}{}\<[9]%
\>[9]\AgdaSymbol{(}\AgdaBound{p} \AgdaSymbol{:} \AgdaBound{a} \AgdaDatatype{≡} \AgdaBound{b}\AgdaSymbol{)} \AgdaSymbol{→} \<[23]%
\>[23]\<%
\\
\>[4]\AgdaIndent{9}{}\<[9]%
\>[9]\AgdaSymbol{\{}\AgdaBound{c} \AgdaSymbol{:} \AgdaBound{B} \AgdaBound{a}\AgdaSymbol{\}\{}\AgdaBound{d} \AgdaSymbol{:} \AgdaBound{B} \AgdaBound{b}\AgdaSymbol{\}} \AgdaSymbol{→} \<[30]%
\>[30]\<%
\\
\>[4]\AgdaIndent{9}{}\<[9]%
\>[9]\AgdaSymbol{(}\AgdaFunction{subst} \AgdaSymbol{(λ} \AgdaBound{x} \AgdaSymbol{→} \AgdaBound{B} \AgdaBound{x}\AgdaSymbol{)} \AgdaBound{p} \AgdaBound{c} \AgdaDatatype{≡} \AgdaBound{d}\AgdaSymbol{)} \<[37]%
\>[37]\<%
\\
\>[4]\AgdaIndent{9}{}\<[9]%
\>[9]\AgdaSymbol{→} \AgdaDatatype{\_≡\_} \AgdaSymbol{\{\_\}} \AgdaSymbol{\{}\AgdaRecord{Σ} \AgdaBound{A} \AgdaBound{B}\AgdaSymbol{\}} \AgdaSymbol{(}\AgdaBound{a} \AgdaInductiveConstructor{,} \AgdaBound{c}\AgdaSymbol{)} \AgdaSymbol{(}\AgdaBound{b} \AgdaInductiveConstructor{,} \AgdaBound{d}\AgdaSymbol{)}\<%
\\
\>\AgdaFunction{sig-eq} \AgdaInductiveConstructor{refl} \AgdaInductiveConstructor{refl} \AgdaSymbol{=} \AgdaInductiveConstructor{refl}\<%
\\
%
\\
\>\<\end{code}
}

\begin{code}\>\<%
\\
%
\\
%
\\
\>\AgdaFunction{Σ'} \AgdaSymbol{:} \AgdaSymbol{(}\AgdaBound{P} \AgdaSymbol{:} \AgdaRecord{hProp}\AgdaSymbol{)(}\AgdaBound{Q} \AgdaSymbol{:} \AgdaFunction{<} \AgdaBound{P} \AgdaFunction{>} \AgdaSymbol{→} \AgdaRecord{hProp}\AgdaSymbol{)} \AgdaSymbol{→} \AgdaRecord{hProp}\<%
\\
\>\AgdaFunction{Σ'} \AgdaBound{P} \AgdaBound{Q} \AgdaSymbol{=} \AgdaInductiveConstructor{hp} \AgdaSymbol{(}\AgdaRecord{Σ} \AgdaFunction{<} \AgdaBound{P} \AgdaFunction{>} \AgdaSymbol{(λ} \AgdaBound{x} \AgdaSymbol{→} \AgdaFunction{<} \AgdaBound{Q} \AgdaBound{x} \AgdaFunction{>}\AgdaSymbol{))} \<[38]%
\>[38]\<%
\\
\>[9]\AgdaIndent{12}{}\<[12]%
\>[12]\AgdaSymbol{(λ} \AgdaSymbol{\{}\AgdaBound{p}\AgdaSymbol{\}} \AgdaSymbol{\{}\AgdaBound{q}\AgdaSymbol{\}} \AgdaSymbol{→} \<[25]%
\>[25]\<%
\\
\>[9]\AgdaIndent{12}{}\<[12]%
\>[12]\AgdaFunction{sig-eq} \AgdaSymbol{(}\AgdaFunction{Uni} \AgdaBound{P}\AgdaSymbol{)} \AgdaSymbol{(}\AgdaFunction{Uni} \AgdaSymbol{(}\AgdaBound{Q} \AgdaSymbol{(}\AgdaFunction{proj₁} \AgdaBound{q}\AgdaSymbol{))))}\<%
\\
%
\\
\>\<\end{code}

Implication and conjuction which are independent ones of them follow simply.

\begin{code}\>\<%
\\
%
\\
\>\AgdaFunction{\_⇒\_} \AgdaSymbol{:} \AgdaSymbol{(}\AgdaBound{P} \AgdaBound{Q} \AgdaSymbol{:} \AgdaRecord{hProp}\AgdaSymbol{)} \AgdaSymbol{→} \AgdaRecord{hProp}\<%
\\
\>\AgdaBound{P} \AgdaFunction{⇒} \AgdaBound{Q} \AgdaSymbol{=} \<[9]%
\>[9]\AgdaFunction{∀'} \AgdaFunction{<} \AgdaBound{P} \AgdaFunction{>} \AgdaSymbol{(λ} \AgdaBound{\_} \AgdaSymbol{→} \AgdaBound{Q}\AgdaSymbol{)}\<%
\\
%
\\
\>\AgdaFunction{\_∧\_} \AgdaSymbol{:} \AgdaSymbol{(}\AgdaBound{P} \AgdaBound{Q} \AgdaSymbol{:} \AgdaRecord{hProp}\AgdaSymbol{)} \AgdaSymbol{→} \AgdaRecord{hProp}\<%
\\
\>\AgdaBound{P} \AgdaFunction{∧} \AgdaBound{Q} \AgdaSymbol{=} \AgdaFunction{Σ'} \AgdaBound{P} \AgdaSymbol{(λ} \AgdaBound{\_} \AgdaSymbol{→} \AgdaBound{Q}\AgdaSymbol{)}\<%
\\
%
\\
\>\<\end{code}

\AgdaHide{
\begin{code}\>\<%
\\
\>\AgdaKeyword{syntax} ∀' A \AgdaSymbol{(λ} x \AgdaSymbol{→} B\AgdaSymbol{)} \AgdaSymbol{=} ∀'[ x ∶ A ] B


\AgdaKeyword{syntax} Σ' A \AgdaSymbol{(λ} x \AgdaSymbol{→} B\AgdaSymbol{)} \AgdaSymbol{=} Σ'[ x ∶ A ] B

\<\end{code}
}

As long as we have implication and conjuction, more operators on proposition can be defined, for instances negation and logical equivalence.

\begin{code}\>\<%
\\
%
\\
\>\AgdaFunction{¬} \AgdaSymbol{:} \AgdaRecord{hProp} \AgdaSymbol{→} \AgdaRecord{hProp}\<%
\\
\>\AgdaFunction{¬} \AgdaBound{P} \AgdaSymbol{=} \AgdaBound{P} \AgdaFunction{⇒} \AgdaFunction{⊥'} \<[13]%
\>[13]\<%
\\
%
\\
\>\AgdaFunction{\_↔\_} \<[6]%
\>[6]\AgdaSymbol{:} \AgdaSymbol{(}\AgdaBound{P} \AgdaBound{Q} \AgdaSymbol{:} \AgdaRecord{hProp}\AgdaSymbol{)} \AgdaSymbol{→} \AgdaRecord{hProp}\<%
\\
\>\AgdaBound{P} \AgdaFunction{↔} \AgdaBound{Q} \AgdaSymbol{=} \AgdaSymbol{(}\AgdaBound{P} \AgdaFunction{⇒} \AgdaBound{Q}\AgdaSymbol{)} \AgdaFunction{∧} \AgdaSymbol{(}\AgdaBound{Q} \AgdaFunction{⇒} \AgdaBound{P}\AgdaSymbol{)}\<%
\\
%
\\
\>\<\end{code}

%\section{Category}

%
\AgdaHide{
\begin{code}\>\<%
\\
%
\\
\>\AgdaSymbol{\{-\#} \AgdaKeyword{OPTIONS} --type-in-type \AgdaSymbol{\#-\}}\<%
\\
%
\\
\>\AgdaKeyword{module} \AgdaModule{Category} \AgdaKeyword{where}\<%
\\
%
\\
\>\AgdaKeyword{open} \AgdaKeyword{import} \AgdaModule{Data.Product}\<%
\\
\>\AgdaKeyword{open} \AgdaKeyword{import} \AgdaModule{Relation.Binary.PropositionalEquality}\<%
\\
%
\\
\>\AgdaKeyword{open} \AgdaKeyword{import} \AgdaModule{Level}\<%
\\
%
\\
\>\<\end{code}
}

\AgdaHide{
\begin{code}\>\<%
\\
%
\\
\>\AgdaKeyword{record} \AgdaRecord{IsCategory}\<%
\\
\>[0]\AgdaIndent{2}{}\<[2]%
\>[2]\AgdaSymbol{(}\AgdaBound{obj} \<[12]%
\>[12]\AgdaSymbol{:} \AgdaPrimitiveType{Set}\AgdaSymbol{)}\<%
\\
%
\\
\>[0]\AgdaIndent{2}{}\<[2]%
\>[2]\AgdaSymbol{(}\AgdaBound{hom} \<[12]%
\>[12]\AgdaSymbol{:} \AgdaBound{obj} \AgdaSymbol{→} \AgdaBound{obj} \AgdaSymbol{→} \AgdaPrimitiveType{Set}\AgdaSymbol{)}\<%
\\
%
\\
\>[0]\AgdaIndent{2}{}\<[2]%
\>[2]\AgdaSymbol{(}\AgdaBound{id} \<[12]%
\>[12]\AgdaSymbol{:} \AgdaSymbol{∀} \AgdaBound{α} \AgdaSymbol{→} \AgdaBound{hom} \AgdaBound{α} \AgdaBound{α}\AgdaSymbol{)}\<%
\\
%
\\
\>[0]\AgdaIndent{2}{}\<[2]%
\>[2]\AgdaSymbol{(}\AgdaBound{[\_⇒\_]\_∘\_} \AgdaSymbol{:} \AgdaSymbol{∀} \AgdaBound{α} \AgdaSymbol{\{}\AgdaBound{β}\AgdaSymbol{\}} \AgdaBound{γ}\<%
\\
\>[2]\AgdaIndent{12}{}\<[12]%
\>[12]\AgdaSymbol{→} \AgdaBound{hom} \AgdaBound{β} \AgdaBound{γ}\<%
\\
\>[2]\AgdaIndent{12}{}\<[12]%
\>[12]\AgdaSymbol{→} \AgdaBound{hom} \AgdaBound{α} \AgdaBound{β}\<%
\\
\>[2]\AgdaIndent{12}{}\<[12]%
\>[12]\AgdaSymbol{→} \AgdaBound{hom} \AgdaBound{α} \AgdaBound{γ}\AgdaSymbol{)}\<%
\\
\>[0]\AgdaIndent{2}{}\<[2]%
\>[2]\AgdaSymbol{:} \AgdaPrimitiveType{Set}\<%
\\
\>[0]\AgdaIndent{2}{}\<[2]%
\>[2]\AgdaKeyword{where}\<%
\\
\>[2]\AgdaIndent{4}{}\<[4]%
\>[4]\AgdaKeyword{constructor} \AgdaInductiveConstructor{IsCatC}\<%
\\
\>[2]\AgdaIndent{4}{}\<[4]%
\>[4]\AgdaKeyword{field}\<%
\\
\>[4]\AgdaIndent{6}{}\<[6]%
\>[6]\AgdaField{id₁} \<[11]%
\>[11]\AgdaSymbol{:} \AgdaSymbol{∀} \AgdaBound{α} \AgdaBound{β} \AgdaSymbol{(}\AgdaBound{f} \AgdaSymbol{:} \AgdaBound{hom} \AgdaBound{α} \AgdaBound{β}\AgdaSymbol{)}\<%
\\
\>[6]\AgdaIndent{11}{}\<[11]%
\>[11]\AgdaSymbol{→} \AgdaBound{[} \AgdaBound{α} \AgdaBound{⇒} \AgdaBound{β} \AgdaBound{]} \AgdaBound{f} \AgdaBound{∘} \AgdaSymbol{(}\AgdaBound{id} \AgdaBound{α}\AgdaSymbol{)} \AgdaDatatype{≡} \AgdaBound{f}\<%
\\
%
\\
\>[0]\AgdaIndent{6}{}\<[6]%
\>[6]\AgdaField{id₂} \<[11]%
\>[11]\AgdaSymbol{:} \AgdaSymbol{∀} \AgdaBound{α} \AgdaBound{β} \AgdaSymbol{(}\AgdaBound{f} \AgdaSymbol{:} \AgdaBound{hom} \AgdaBound{α} \AgdaBound{β}\AgdaSymbol{)}\<%
\\
\>[0]\AgdaIndent{11}{}\<[11]%
\>[11]\AgdaSymbol{→} \AgdaBound{[} \AgdaBound{α} \AgdaBound{⇒} \AgdaBound{β} \AgdaBound{]} \AgdaSymbol{(}\AgdaBound{id} \AgdaBound{β}\AgdaSymbol{)} \AgdaBound{∘} \AgdaBound{f} \AgdaDatatype{≡} \AgdaBound{f}\<%
\\
%
\\
\>[0]\AgdaIndent{6}{}\<[6]%
\>[6]\AgdaField{comp} \AgdaSymbol{:} \AgdaSymbol{∀} \AgdaBound{α} \AgdaSymbol{\{}\AgdaBound{β} \AgdaBound{γ}\AgdaSymbol{\}} \AgdaBound{δ} \AgdaSymbol{(}\AgdaBound{f} \AgdaSymbol{:} \AgdaBound{hom} \AgdaBound{α} \AgdaBound{β}\AgdaSymbol{)} \AgdaSymbol{(}\AgdaBound{g} \AgdaSymbol{:} \AgdaBound{hom} \AgdaBound{β} \AgdaBound{γ}\AgdaSymbol{)} \AgdaSymbol{(}\AgdaBound{h} \AgdaSymbol{:} \AgdaBound{hom} \AgdaBound{γ} \AgdaBound{δ}\AgdaSymbol{)}\<%
\\
\>[0]\AgdaIndent{11}{}\<[11]%
\>[11]\AgdaSymbol{→} \AgdaBound{[} \AgdaBound{α} \AgdaBound{⇒} \AgdaBound{δ} \AgdaBound{]} \AgdaBound{[} \AgdaBound{β} \AgdaBound{⇒} \AgdaBound{δ} \AgdaBound{]} \AgdaBound{h} \AgdaBound{∘} \AgdaBound{g} \AgdaBound{∘} \AgdaBound{f} \AgdaDatatype{≡} \AgdaBound{[} \AgdaBound{α} \AgdaBound{⇒} \AgdaBound{δ} \AgdaBound{]} \AgdaBound{h} \AgdaBound{∘} \AgdaSymbol{(}\AgdaBound{[} \AgdaBound{α} \AgdaBound{⇒} \AgdaBound{γ} \AgdaBound{]} \AgdaBound{g} \AgdaBound{∘} \AgdaBound{f}\AgdaSymbol{)}\<%
\\
%
\\
%
\\
\>\<\end{code}
}


\begin{code}\>\<%
\\
\>\AgdaKeyword{record} \AgdaRecord{Category} \AgdaSymbol{:} \AgdaPrimitiveType{Set} \AgdaKeyword{where}\<%
\\
\>[0]\AgdaIndent{2}{}\<[2]%
\>[2]\AgdaKeyword{constructor} \AgdaInductiveConstructor{CatC}\<%
\\
\>[0]\AgdaIndent{2}{}\<[2]%
\>[2]\AgdaKeyword{field}\<%
\\
\>[2]\AgdaIndent{4}{}\<[4]%
\>[4]\AgdaField{obj} \<[15]%
\>[15]\AgdaSymbol{:} \AgdaPrimitiveType{Set}\<%
\\
%
\\
\>[2]\AgdaIndent{4}{}\<[4]%
\>[4]\AgdaField{hom} \<[15]%
\>[15]\AgdaSymbol{:} \AgdaBound{obj} \AgdaSymbol{→} \AgdaBound{obj} \AgdaSymbol{→} \AgdaPrimitiveType{Set}\<%
\\
%
\\
\>[2]\AgdaIndent{4}{}\<[4]%
\>[4]\AgdaField{id} \<[15]%
\>[15]\AgdaSymbol{:} \AgdaSymbol{∀} \AgdaBound{α}\<%
\\
\>[4]\AgdaIndent{15}{}\<[15]%
\>[15]\AgdaSymbol{→} \AgdaBound{hom} \AgdaBound{α} \AgdaBound{α}\<%
\\
%
\\
\>[0]\AgdaIndent{4}{}\<[4]%
\>[4]\AgdaField{[\_⇒\_]\_∘\_} \<[15]%
\>[15]\AgdaSymbol{:} \AgdaSymbol{∀} \AgdaBound{α} \AgdaSymbol{\{}\AgdaBound{β}\AgdaSymbol{\}} \AgdaBound{γ}\<%
\\
\>[0]\AgdaIndent{15}{}\<[15]%
\>[15]\AgdaSymbol{→} \AgdaBound{hom} \AgdaBound{β} \AgdaBound{γ}\<%
\\
\>[0]\AgdaIndent{15}{}\<[15]%
\>[15]\AgdaSymbol{→} \AgdaBound{hom} \AgdaBound{α} \AgdaBound{β}\<%
\\
\>[0]\AgdaIndent{15}{}\<[15]%
\>[15]\AgdaSymbol{→} \AgdaBound{hom} \AgdaBound{α} \AgdaBound{γ}\<%
\\
%
\\
\>[0]\AgdaIndent{4}{}\<[4]%
\>[4]\AgdaField{isCategory} \AgdaSymbol{:} \AgdaRecord{IsCategory} \AgdaBound{obj} \AgdaBound{hom} \AgdaBound{id} \AgdaBound{[\_⇒\_]\_∘\_}\<%
\\
%
\\
%
\\
\>\<\end{code}

\AgdaHide{
\begin{code}\>\<%
\\
%
\\
\>[0]\AgdaIndent{2}{}\<[2]%
\>[2]\AgdaKeyword{open} \AgdaModule{IsCategory} \AgdaKeyword{public}\<%
\\
%
\\
\>\<\end{code}
}

%\section{Category of setoids}

%
\AgdaHide{

\begin{code}\>\<%
\\
%
\\
%
\\
\>\AgdaSymbol{\{-\#} \AgdaKeyword{OPTIONS} --type-in-type \AgdaSymbol{\#-\}}\<%
\\
%
\\
\>\AgdaKeyword{open} \AgdaKeyword{import} \AgdaModule{Level}\<%
\\
\>\AgdaKeyword{open} \AgdaKeyword{import} \AgdaModule{Relation.Binary.PropositionalEquality} \AgdaSymbol{as} \AgdaModule{PE} \AgdaKeyword{hiding} \AgdaSymbol{(}refl\AgdaSymbol{;} sym \AgdaSymbol{;} trans\AgdaSymbol{;} isEquivalence\AgdaSymbol{)}\<%
\\
%
\\
\>\AgdaKeyword{module} \AgdaModule{CategoryOfSetoid} \<[25]%
\>[25]\AgdaSymbol{(}\AgdaBound{ext} \AgdaSymbol{:} \AgdaFunction{Extensionality} \AgdaPrimitive{zero} \AgdaPrimitive{zero}\AgdaSymbol{)} \AgdaKeyword{where}\<%
\\
%
\\
\>\AgdaKeyword{open} \AgdaKeyword{import} \AgdaModule{Cats.Category}\<%
\\
\>\AgdaKeyword{open} \AgdaKeyword{import} \AgdaModule{Function}\<%
\\
\>\AgdaKeyword{open} \AgdaKeyword{import} \AgdaModule{Relation.Binary.Core} \AgdaKeyword{using} \AgdaSymbol{(}\_≡\_\AgdaSymbol{)} \AgdaKeyword{renaming} \AgdaSymbol{(}\_⇒\_ \AgdaSymbol{to} \_⇒'\_\AgdaSymbol{)}\<%
\\
\>\AgdaKeyword{open} \AgdaKeyword{import} \AgdaModule{Data.Unit}\<%
\\
\>\AgdaKeyword{open} \AgdaKeyword{import} \AgdaModule{Data.Empty}\<%
\\
\>\AgdaKeyword{import} \AgdaModule{hProp}\<%
\\
\>\AgdaKeyword{open} \AgdaKeyword{module} \AgdaModule{hpx} \AgdaSymbol{=} \AgdaModule{hProp} \AgdaBound{ext}\<%
\\
%
\\
%
\\
\>\AgdaComment{-- Arrow between HSetoid}\<%
\\
%
\\
\>\AgdaKeyword{infix} \AgdaNumber{5} \_⇉\_\<%
\\
%
\\
\>\AgdaComment{-- composition}\<%
\\
%
\\
\>\AgdaKeyword{infixl} \AgdaNumber{5} \_∘c\_\<%
\\
%
\\
\>\<\end{code}
}

Then we could define setoids using \textbf{hProp}. An equivalence relation has three properties reflexivity, symmetry and transitivity. Since we have $refl$ here, we call the reflexivity for propositional equality from the library with prefix as $PE.refl$. 

\begin{code}\>\<%
\\
%
\\
\>\AgdaKeyword{record} \AgdaRecord{ishEquivalence} \AgdaSymbol{\{}\AgdaBound{A} \AgdaSymbol{:} \AgdaPrimitiveType{Set}\AgdaSymbol{\}(}\AgdaBound{\_≈h\_} \AgdaSymbol{:} \AgdaBound{A} \AgdaSymbol{→} \AgdaBound{A} \AgdaSymbol{→} \AgdaRecord{hProp}\AgdaSymbol{)} \AgdaSymbol{:} \AgdaPrimitiveType{Set₁} \AgdaKeyword{where}\<%
\\
\>[0]\AgdaIndent{2}{}\<[2]%
\>[2]\AgdaKeyword{constructor} \AgdaInductiveConstructor{\_,\_,\_}\<%
\\
\>[0]\AgdaIndent{2}{}\<[2]%
\>[2]\AgdaKeyword{field}\<%
\\
\>[2]\AgdaIndent{4}{}\<[4]%
\>[4]\AgdaField{refl} \<[12]%
\>[12]\AgdaSymbol{:} \AgdaSymbol{\{}\AgdaBound{x} \AgdaSymbol{:} \AgdaBound{A}\AgdaSymbol{\}} \AgdaSymbol{→} \AgdaFunction{<} \AgdaBound{x} \AgdaBound{≈h} \AgdaBound{x} \AgdaFunction{>}\<%
\\
\>[2]\AgdaIndent{4}{}\<[4]%
\>[4]\AgdaField{sym} \<[12]%
\>[12]\AgdaSymbol{:} \AgdaSymbol{\{}\AgdaBound{x} \AgdaBound{y} \AgdaSymbol{:} \AgdaBound{A}\AgdaSymbol{\}} \AgdaSymbol{→} \AgdaFunction{<} \AgdaBound{x} \AgdaBound{≈h} \AgdaBound{y} \AgdaFunction{>} \AgdaSymbol{→} \AgdaFunction{<} \AgdaBound{y} \AgdaBound{≈h} \AgdaBound{x} \AgdaFunction{>}\<%
\\
\>[2]\AgdaIndent{4}{}\<[4]%
\>[4]\AgdaField{trans} \<[12]%
\>[12]\AgdaSymbol{:} \AgdaSymbol{\{}\AgdaBound{x} \AgdaBound{y} \AgdaBound{z} \AgdaSymbol{:} \AgdaBound{A}\AgdaSymbol{\}} \AgdaSymbol{→} \AgdaFunction{<} \AgdaBound{x} \AgdaBound{≈h} \AgdaBound{y} \AgdaFunction{>} \AgdaSymbol{→} \AgdaFunction{<} \AgdaBound{y} \AgdaBound{≈h} \AgdaBound{z} \AgdaFunction{>} \AgdaSymbol{→} \AgdaFunction{<} \AgdaBound{x} \AgdaBound{≈h} \AgdaBound{z} \AgdaFunction{>}\<%
\\
%
\\
\>\<\end{code}

Here we use \textbf{hSetoid} as the name because \textbf{Setoid} is
already used for non-proof-irrelvant setoids in the library.
For each setoid, we have a carrier type and an equivalence relation.

\begin{code}\>\<%
\\
\>\AgdaKeyword{record} \AgdaRecord{hSetoid} \AgdaSymbol{:} \AgdaPrimitiveType{Set₁} \AgdaKeyword{where}\<%
\\
\>[0]\AgdaIndent{2}{}\<[2]%
\>[2]\AgdaKeyword{constructor} \AgdaInductiveConstructor{\_,\_,\_}\<%
\\
\>[0]\AgdaIndent{2}{}\<[2]%
\>[2]\AgdaKeyword{infix} \AgdaNumber{4} \_≈h\_ \_≈\_\<%
\\
\>[0]\AgdaIndent{2}{}\<[2]%
\>[2]\AgdaKeyword{field}\<%
\\
\>[2]\AgdaIndent{4}{}\<[4]%
\>[4]\AgdaField{Carrier} \AgdaSymbol{:} \AgdaPrimitiveType{Set}\<%
\\
\>[2]\AgdaIndent{4}{}\<[4]%
\>[4]\AgdaField{\_≈h\_} \<[12]%
\>[12]\AgdaSymbol{:} \AgdaBound{Carrier} \AgdaSymbol{→} \AgdaBound{Carrier} \AgdaSymbol{→} \AgdaRecord{hProp}\<%
\\
\>[2]\AgdaIndent{4}{}\<[4]%
\>[4]\AgdaField{isEquiv} \AgdaSymbol{:} \AgdaRecord{ishEquivalence} \AgdaBound{\_≈h\_}\<%
\\
%
\\
%
\\
\>\<\end{code}

\AgdaHide{
\begin{code}\>\<%
\\
%
\\
\>[0]\AgdaIndent{2}{}\<[2]%
\>[2]\AgdaKeyword{open} \<[8]%
\>[8]\AgdaModule{ishEquivalence} \AgdaFunction{isEquiv} \AgdaKeyword{public}\<%
\\
%
\\
\>[0]\AgdaIndent{2}{}\<[2]%
\>[2]\AgdaFunction{\_≈\_} \AgdaSymbol{:} \AgdaFunction{Carrier} \AgdaSymbol{→} \AgdaFunction{Carrier} \AgdaSymbol{→} \AgdaPrimitiveType{Set}\<%
\\
\>[0]\AgdaIndent{2}{}\<[2]%
\>[2]\AgdaBound{a} \AgdaFunction{≈} \AgdaBound{b} \AgdaSymbol{=} \AgdaFunction{<} \AgdaBound{a} \AgdaFunction{≈h} \AgdaBound{b} \AgdaFunction{>}\<%
\\
\>[0]\AgdaIndent{1}{}\<[1]%
\>[1]\<%
\\
\>[0]\AgdaIndent{2}{}\<[2]%
\>[2]\AgdaFunction{PI} \AgdaSymbol{:} \AgdaSymbol{\{}\AgdaBound{x} \AgdaBound{y} \AgdaSymbol{:} \AgdaFunction{Carrier}\AgdaSymbol{\}\{}\AgdaBound{B} \AgdaSymbol{:} \AgdaPrimitiveType{Set}\AgdaSymbol{\}}\<%
\\
\>[2]\AgdaIndent{7}{}\<[7]%
\>[7]\AgdaSymbol{(}\AgdaBound{A} \AgdaSymbol{:} \AgdaBound{x} \AgdaFunction{≈} \AgdaBound{y} \AgdaSymbol{→} \AgdaBound{B}\AgdaSymbol{)\{}\AgdaBound{p} \AgdaBound{q} \AgdaSymbol{:} \AgdaBound{x} \AgdaFunction{≈} \AgdaBound{y}\AgdaSymbol{\}} \<[36]%
\>[36]\<%
\\
\>[0]\AgdaIndent{5}{}\<[5]%
\>[5]\AgdaSymbol{→} \AgdaBound{A} \AgdaBound{p} \AgdaDatatype{≡} \AgdaBound{A} \AgdaBound{q}\<%
\\
\>[0]\AgdaIndent{2}{}\<[2]%
\>[2]\AgdaFunction{PI} \AgdaSymbol{\{}\AgdaBound{x}\AgdaSymbol{\}} \AgdaSymbol{\{}\AgdaBound{y}\AgdaSymbol{\}} \AgdaBound{A} \AgdaSymbol{\{}\AgdaBound{p}\AgdaSymbol{\}} \AgdaSymbol{\{}\AgdaBound{q}\AgdaSymbol{\}} \AgdaKeyword{with} \AgdaFunction{Uni} \AgdaSymbol{(}\AgdaBound{x} \AgdaFunction{≈h} \AgdaBound{y}\AgdaSymbol{)} \AgdaSymbol{\{}\AgdaBound{p}\AgdaSymbol{\}} \AgdaSymbol{\{}\AgdaBound{q}\AgdaSymbol{\}}\<%
\\
\>[0]\AgdaIndent{2}{}\<[2]%
\>[2]\AgdaFunction{PI} \AgdaBound{A} \AgdaSymbol{|} \AgdaInductiveConstructor{PE.refl} \AgdaSymbol{=} \AgdaInductiveConstructor{PE.refl}\<%
\\
%
\\
\>[0]\AgdaIndent{2}{}\<[2]%
\>[2]\AgdaFunction{reflexive} \AgdaSymbol{:} \AgdaDatatype{\_≡\_} \AgdaFunction{⇒'} \AgdaFunction{\_≈\_}\<%
\\
\>[0]\AgdaIndent{2}{}\<[2]%
\>[2]\AgdaFunction{reflexive} \AgdaInductiveConstructor{PE.refl} \AgdaSymbol{=} \AgdaFunction{refl}\<%
\\
%
\\
\>\AgdaKeyword{open} \AgdaModule{hSetoid} \AgdaKeyword{public} \AgdaKeyword{renaming} \AgdaSymbol{(}refl \AgdaSymbol{to} [\_]refl\AgdaSymbol{;}
     sym \AgdaSymbol{to} [\_]sym\AgdaSymbol{;} \_≈\_ \AgdaSymbol{to} [\_]\_≈\_ \AgdaSymbol{;} \_≈h\_ \AgdaSymbol{to} [\_]\_≈h\_ \AgdaSymbol{;}
     Carrier \AgdaSymbol{to} ∣\_∣ \AgdaSymbol{;} trans \AgdaSymbol{to} [\_]trans\AgdaSymbol{)}\<%
\\
%
\\
%
\\
\>\AgdaFunction{[\_]uip} \AgdaSymbol{:} \AgdaSymbol{∀(}\AgdaBound{Γ} \AgdaSymbol{:} \AgdaRecord{hSetoid}\AgdaSymbol{)\{}\AgdaBound{a} \AgdaBound{b} \AgdaSymbol{:} \AgdaFunction{∣} \AgdaBound{Γ} \AgdaFunction{∣}\AgdaSymbol{\}\{}\AgdaBound{p} \AgdaBound{q} \AgdaSymbol{:} \AgdaFunction{[} \AgdaBound{Γ} \AgdaFunction{]} \AgdaBound{a} \AgdaFunction{≈} \AgdaBound{b}\AgdaSymbol{\}} \AgdaSymbol{→} \AgdaBound{p} \AgdaDatatype{≡} \AgdaBound{q}\<%
\\
\>\AgdaFunction{[} \AgdaBound{Γ} \AgdaFunction{]uip} \AgdaSymbol{\{}\AgdaBound{a}\AgdaSymbol{\}} \AgdaSymbol{\{}\AgdaBound{b}\AgdaSymbol{\}} \AgdaSymbol{=} \AgdaFunction{Uni} \AgdaSymbol{(}\AgdaFunction{[} \AgdaBound{Γ} \AgdaFunction{]} \AgdaBound{a} \AgdaFunction{≈h} \AgdaBound{b}\AgdaSymbol{)}\<%
\\
%
\\
%
\\
\>\<\end{code}
}

A morphism in this category is a function of the underlying sets which respects the equivalence relation. We don't identify the extensional equal functions in the homsets as in \textbf{E-setoids}.

\begin{code}\>\<%
\\
%
\\
\>\AgdaKeyword{record} \AgdaRecord{\_⇉\_} \AgdaSymbol{(}\AgdaBound{A} \AgdaBound{B} \AgdaSymbol{:} \AgdaRecord{hSetoid}\AgdaSymbol{)} \AgdaSymbol{:} \AgdaPrimitiveType{Set₁} \AgdaKeyword{where}\<%
\\
\>[0]\AgdaIndent{2}{}\<[2]%
\>[2]\AgdaKeyword{constructor} \AgdaInductiveConstructor{fn:\_resp:\_}\<%
\\
\>[0]\AgdaIndent{2}{}\<[2]%
\>[2]\AgdaKeyword{field}\<%
\\
\>[2]\AgdaIndent{4}{}\<[4]%
\>[4]\AgdaField{fn} \<[9]%
\>[9]\AgdaSymbol{:} \AgdaFunction{∣} \AgdaBound{A} \AgdaFunction{∣} \AgdaSymbol{→} \AgdaFunction{∣} \AgdaBound{B} \AgdaFunction{∣}\<%
\\
\>[2]\AgdaIndent{4}{}\<[4]%
\>[4]\AgdaField{resp} \AgdaSymbol{:} \AgdaSymbol{\{}\AgdaBound{x} \AgdaBound{y} \AgdaSymbol{:} \AgdaFunction{∣} \AgdaBound{A} \AgdaFunction{∣}\AgdaSymbol{\}} \AgdaSymbol{→} \<[27]%
\>[27]\<%
\\
\>[4]\AgdaIndent{11}{}\<[11]%
\>[11]\AgdaFunction{[} \AgdaBound{A} \AgdaFunction{]} \AgdaBound{x} \AgdaFunction{≈} \AgdaBound{y} \AgdaSymbol{→} \<[25]%
\>[25]\<%
\\
\>[4]\AgdaIndent{11}{}\<[11]%
\>[11]\AgdaFunction{[} \AgdaBound{B} \AgdaFunction{]} \AgdaBound{fn} \AgdaBound{x} \AgdaFunction{≈} \AgdaBound{fn} \AgdaBound{y}\<%
\\
%
\\
\>\<\end{code}

\AgdaHide{
\begin{code}\>\<%
\\
%
\\
\>\AgdaKeyword{open} \AgdaModule{\_⇉\_} \AgdaKeyword{public} \AgdaKeyword{renaming} \AgdaSymbol{(}fn \AgdaSymbol{to} [\_]fn \AgdaSymbol{;} resp \AgdaSymbol{to} [\_]resp\AgdaSymbol{)}\<%
\\
%
\\
\>\<\end{code}
}


The definitions of identity morphism and composition are straightforward and the categorical laws hold trivially as follows.

\begin{code}\>\<%
\\
%
\\
\>\AgdaFunction{id'} \AgdaSymbol{:} \AgdaSymbol{\{}\AgdaBound{Γ} \AgdaSymbol{:} \AgdaRecord{hSetoid}\AgdaSymbol{\}} \AgdaSymbol{→} \AgdaBound{Γ} \AgdaRecord{⇉} \AgdaBound{Γ} \<[28]%
\>[28]\<%
\\
\>\AgdaFunction{id'} \AgdaSymbol{=} \AgdaKeyword{record} \AgdaSymbol{\{} \AgdaField{fn} \AgdaSymbol{=} \AgdaFunction{id}\AgdaSymbol{;} \AgdaField{resp} \AgdaSymbol{=} \AgdaFunction{id}\AgdaSymbol{\}}\<%
\\
%
\\
\>\AgdaFunction{\_∘c\_} \AgdaSymbol{:} \AgdaSymbol{∀\{}\AgdaBound{Γ} \AgdaBound{Δ} \AgdaBound{Z}\AgdaSymbol{\}} \AgdaSymbol{→} \AgdaBound{Δ} \AgdaRecord{⇉} \AgdaBound{Z} \AgdaSymbol{→} \AgdaBound{Γ} \AgdaRecord{⇉} \AgdaBound{Δ} \AgdaSymbol{→} \AgdaBound{Γ} \AgdaRecord{⇉} \AgdaBound{Z}\<%
\\
\>\AgdaBound{yz} \AgdaFunction{∘c} \AgdaBound{xy} \AgdaSymbol{=} \AgdaKeyword{record} \<[18]%
\>[18]\<%
\\
\>[4]\AgdaIndent{11}{}\<[11]%
\>[11]\AgdaSymbol{\{} \AgdaField{fn} \AgdaSymbol{=} \AgdaFunction{[} \AgdaBound{yz} \AgdaFunction{]fn} \AgdaFunction{∘} \AgdaFunction{[} \AgdaBound{xy} \AgdaFunction{]fn}\<%
\\
\>[4]\AgdaIndent{11}{}\<[11]%
\>[11]\AgdaSymbol{;} \AgdaField{resp} \AgdaSymbol{=} \AgdaFunction{[} \AgdaBound{yz} \AgdaFunction{]resp} \AgdaFunction{∘} \AgdaFunction{[} \AgdaBound{xy} \AgdaFunction{]resp}\<%
\\
\>[4]\AgdaIndent{11}{}\<[11]%
\>[11]\AgdaSymbol{\}}\<%
\\
%
\\
\>\AgdaFunction{id₁} \AgdaSymbol{:} \AgdaSymbol{∀} \AgdaBound{Γ} \AgdaBound{Δ} \AgdaSymbol{(}\AgdaBound{ch} \AgdaSymbol{:} \AgdaBound{Γ} \AgdaRecord{⇉} \AgdaBound{Δ}\AgdaSymbol{)} \AgdaSymbol{→} \AgdaBound{ch} \AgdaFunction{∘c} \AgdaFunction{id'} \AgdaDatatype{≡} \AgdaBound{ch}\<%
\\
\>\AgdaFunction{id₁} \AgdaSymbol{\_} \AgdaSymbol{\_} \AgdaBound{ch} \AgdaSymbol{=} \AgdaInductiveConstructor{PE.refl}\<%
\\
%
\\
\>\AgdaFunction{id₂} \AgdaSymbol{:} \AgdaSymbol{∀} \AgdaBound{Γ} \AgdaBound{Δ} \AgdaSymbol{(}\AgdaBound{ch} \AgdaSymbol{:} \AgdaBound{Γ} \AgdaRecord{⇉} \AgdaBound{Δ}\AgdaSymbol{)} \AgdaSymbol{→} \AgdaFunction{id'} \AgdaFunction{∘c} \AgdaBound{ch} \AgdaDatatype{≡} \AgdaBound{ch}\<%
\\
\>\AgdaFunction{id₂} \AgdaSymbol{\_} \AgdaSymbol{\_} \AgdaBound{ch} \AgdaSymbol{=} \AgdaInductiveConstructor{PE.refl}\<%
\\
%
\\
\>\AgdaFunction{comp} \AgdaSymbol{:} \AgdaSymbol{∀} \AgdaBound{Γ} \AgdaSymbol{\{}\AgdaBound{Δ} \AgdaBound{Φ}\AgdaSymbol{\}} \AgdaBound{Ψ} \<[19]%
\>[19]\<%
\\
\>[-2]\AgdaIndent{9}{}\<[9]%
\>[9]\AgdaSymbol{(}\AgdaBound{f} \AgdaSymbol{:} \AgdaBound{Γ} \AgdaRecord{⇉} \AgdaBound{Δ}\AgdaSymbol{)}\<%
\\
\>[0]\AgdaIndent{9}{}\<[9]%
\>[9]\AgdaSymbol{(}\AgdaBound{g} \AgdaSymbol{:} \AgdaBound{Δ} \AgdaRecord{⇉} \AgdaBound{Φ}\AgdaSymbol{)}\<%
\\
\>[0]\AgdaIndent{9}{}\<[9]%
\>[9]\AgdaSymbol{(}\AgdaBound{h} \AgdaSymbol{:} \AgdaBound{Φ} \AgdaRecord{⇉} \AgdaBound{Ψ}\AgdaSymbol{)}\<%
\\
\>[0]\AgdaIndent{7}{}\<[7]%
\>[7]\AgdaSymbol{→} \AgdaBound{h} \AgdaFunction{∘c} \AgdaBound{g} \AgdaFunction{∘c} \AgdaBound{f} \AgdaDatatype{≡} \AgdaBound{h} \AgdaFunction{∘c} \AgdaSymbol{(}\AgdaBound{g} \AgdaFunction{∘c} \AgdaBound{f}\AgdaSymbol{)}\<%
\\
\>\AgdaFunction{comp} \AgdaSymbol{\_} \AgdaSymbol{\_} \AgdaBound{f} \AgdaBound{g} \AgdaBound{h} \AgdaSymbol{=} \AgdaInductiveConstructor{PE.refl}\<%
\\
%
\\
\>\<\end{code}

\AgdaHide{
\begin{code}\>\<%
\\
\>\AgdaComment{\{-
\_f≈\_ :  ∀\{Γ Δ : hSetoid\} → (f g : Γ ⇉ Δ) → hProp
\_f≈\_ \{Γ , \_≈h\_ , (refl , sym , trans)\} \{Δ , \_≈h₁\_ , (refl₁ , sym₁ , trans₁)\} (fn: fn resp: fresp) (fn: gn resp: gresp) 
  = record 
           \{ prf = (g : Γ) → < fn g ≈h₁ gn g >
           ; Uni = ext (λ g → Uni (fn g ≈h₁ gn g))
           \}
-\}}\<%
\\
%
\\
%
\\
\>\<\end{code}
}

Combined all components we obtain the category of setoids.

\begin{code}\>\<%
\\
%
\\
\>\AgdaFunction{setoid-Cat} \AgdaSymbol{:} \AgdaRecord{Category}\<%
\\
\>\AgdaFunction{setoid-Cat} \AgdaSymbol{=} \AgdaInductiveConstructor{CatC} \AgdaRecord{hSetoid} \AgdaRecord{\_⇉\_} \AgdaSymbol{(λ} \AgdaBound{\_} \AgdaSymbol{→} \AgdaFunction{id'}\AgdaSymbol{)} \AgdaSymbol{(λ} \AgdaBound{\_} \AgdaBound{\_} \AgdaSymbol{→} \AgdaFunction{\_∘c\_}\AgdaSymbol{)} \<[57]%
\>[57]\<%
\\
\>[0]\AgdaIndent{13}{}\<[13]%
\>[13]\AgdaSymbol{(}\AgdaInductiveConstructor{IsCatC} \AgdaFunction{id₁} \AgdaFunction{id₂} \AgdaFunction{comp}\AgdaSymbol{)}\<%
\\
%
\\
\>\<\end{code}

This category has a terminal object which is just the unit set with trivial equality. As a terminal object there is precisely one morphism from every object to it.

\begin{code}\>\<%
\\
%
\\
\>\AgdaFunction{⊤-setoid} \AgdaSymbol{:} \AgdaRecord{hSetoid}\<%
\\
\>\AgdaFunction{⊤-setoid} \AgdaSymbol{=} \AgdaKeyword{record} \AgdaSymbol{\{}\<%
\\
\>[0]\AgdaIndent{6}{}\<[6]%
\>[6]\AgdaField{Carrier} \AgdaSymbol{=} \AgdaRecord{⊤}\AgdaSymbol{;}\<%
\\
\>[0]\AgdaIndent{6}{}\<[6]%
\>[6]\AgdaField{\_≈h\_} \<[14]%
\>[14]\AgdaSymbol{=} \AgdaSymbol{λ} \AgdaBound{\_} \AgdaBound{\_} \AgdaSymbol{→} \AgdaFunction{⊤'}\AgdaSymbol{;}\<%
\\
\>[0]\AgdaIndent{6}{}\<[6]%
\>[6]\AgdaField{isEquiv} \AgdaSymbol{=} \AgdaKeyword{record} \AgdaSymbol{\{}\<%
\\
\>[6]\AgdaIndent{8}{}\<[8]%
\>[8]\AgdaField{refl} \<[16]%
\>[16]\AgdaSymbol{=} \AgdaInductiveConstructor{tt}\AgdaSymbol{;}\<%
\\
\>[6]\AgdaIndent{8}{}\<[8]%
\>[8]\AgdaField{sym} \<[16]%
\>[16]\AgdaSymbol{=} \AgdaSymbol{λ} \AgdaBound{\_} \AgdaSymbol{→} \AgdaInductiveConstructor{tt}\AgdaSymbol{;}\<%
\\
\>[6]\AgdaIndent{8}{}\<[8]%
\>[8]\AgdaField{trans} \<[16]%
\>[16]\AgdaSymbol{=} \AgdaSymbol{λ} \AgdaBound{\_} \AgdaBound{\_} \AgdaSymbol{→} \AgdaInductiveConstructor{tt} \AgdaSymbol{\}} \AgdaSymbol{\}}\<%
\\
%
\\
\>\AgdaFunction{⋆} \AgdaSymbol{:} \AgdaSymbol{\{}\AgdaBound{Δ} \AgdaSymbol{:} \AgdaRecord{hSetoid}\AgdaSymbol{\}} \AgdaSymbol{→} \AgdaBound{Δ} \AgdaRecord{⇉} \AgdaFunction{⊤-setoid}\<%
\\
\>\AgdaFunction{⋆} \AgdaSymbol{=} \AgdaKeyword{record} \<[11]%
\>[11]\<%
\\
\>[0]\AgdaIndent{6}{}\<[6]%
\>[6]\AgdaSymbol{\{} \AgdaField{fn} \AgdaSymbol{=} \AgdaSymbol{λ} \AgdaBound{\_} \AgdaSymbol{→} \AgdaInductiveConstructor{tt}\<%
\\
\>[0]\AgdaIndent{6}{}\<[6]%
\>[6]\AgdaSymbol{;} \AgdaField{resp} \AgdaSymbol{=} \AgdaSymbol{λ} \AgdaBound{\_} \AgdaSymbol{→} \AgdaInductiveConstructor{tt} \AgdaSymbol{\}}\<%
\\
%
\\
\>\AgdaFunction{unique⋆} \AgdaSymbol{:} \AgdaSymbol{\{}\AgdaBound{Δ} \AgdaSymbol{:} \AgdaRecord{hSetoid}\AgdaSymbol{\}} \AgdaSymbol{→} \AgdaSymbol{(}\AgdaBound{f} \AgdaSymbol{:} \AgdaBound{Δ} \AgdaRecord{⇉} \AgdaFunction{⊤-setoid}\AgdaSymbol{)} \AgdaSymbol{→} \AgdaBound{f} \AgdaDatatype{≡} \AgdaFunction{⋆}\<%
\\
\>\AgdaFunction{unique⋆} \AgdaBound{f} \AgdaSymbol{=} \AgdaInductiveConstructor{PE.refl}\<%
\\
%
\\
\>\<\end{code}


%\section{Categories with families of setoids}\label{cwf}



%
\AgdaHide{

\begin{code}\>\<%
\\
\>\AgdaSymbol{\{-\#} \AgdaKeyword{OPTIONS} --type-in-type \AgdaSymbol{\#-\}}\<%
\\
%
\\
%
\\
\>\AgdaKeyword{open} \AgdaKeyword{import} \AgdaModule{Level} \AgdaKeyword{hiding} \AgdaSymbol{(}lift\AgdaSymbol{)}\<%
\\
\>\AgdaKeyword{open} \AgdaKeyword{import} \AgdaModule{Relation.Binary.PropositionalEquality} \AgdaSymbol{as} \AgdaModule{PE} \AgdaKeyword{hiding} \AgdaSymbol{(}refl \AgdaSymbol{;} sym \AgdaSymbol{;} trans\AgdaSymbol{;} isEquivalence\AgdaSymbol{;} [\_]\AgdaSymbol{)}\<%
\\
%
\\
\>\AgdaKeyword{module} \AgdaModule{CwF-setoid} \AgdaSymbol{(}\AgdaBound{ext} \AgdaSymbol{:} \AgdaFunction{Extensionality} \AgdaPrimitive{zero} \AgdaPrimitive{zero}\AgdaSymbol{)} \AgdaKeyword{where}\<%
\\
%
\\
%
\\
\>\AgdaKeyword{open} \AgdaKeyword{import} \AgdaModule{Cats.Category}\<%
\\
\>\AgdaKeyword{open} \AgdaKeyword{import} \AgdaModule{Cats.Functor}\<%
\\
\>\AgdaKeyword{open} \AgdaKeyword{import} \AgdaModule{Cats.Duality}\<%
\\
\>\AgdaKeyword{open} \AgdaKeyword{import} \AgdaModule{Data.Product} \AgdaKeyword{renaming} \AgdaSymbol{(}<\_,\_> \AgdaSymbol{to} ⟨\_,\_⟩\AgdaSymbol{)}\<%
\\
\>\AgdaKeyword{open} \AgdaKeyword{import} \AgdaModule{Function}\<%
\\
%
\\
\>\AgdaKeyword{open} \AgdaKeyword{import} \AgdaModule{Relation.Binary.Core} \AgdaKeyword{using} \AgdaSymbol{(}\_≡\_\AgdaSymbol{;} \_≢\_\AgdaSymbol{)}\<%
\\
\>\AgdaKeyword{open} \AgdaKeyword{import} \AgdaModule{Data.Unit}\<%
\\
%
\\
\>\AgdaKeyword{import} \AgdaModule{CategoryOfSetoid}\<%
\\
\>\AgdaKeyword{module} \AgdaModule{cos} \AgdaSymbol{=} \AgdaModule{CategoryOfSetoid} \AgdaBound{ext}\<%
\\
\>\AgdaKeyword{open} \AgdaModule{cos}\<%
\\
%
\\
\>\AgdaKeyword{import} \AgdaModule{hProp}\<%
\\
\>\AgdaKeyword{module} \AgdaModule{hp} \AgdaSymbol{=} \AgdaModule{hProp} \AgdaBound{ext}\<%
\\
\>\AgdaKeyword{open} \AgdaModule{hp}\<%
\\
%
\\
\>\AgdaKeyword{infixl} \AgdaNumber{5} \_\&\_\<%
\\
%
\\
\>\<\end{code}
}

We would like to show two formalisation of category with families for setoids here. The first one is simple and short but not comprehensive. We have to extract all complicated components from the simple definition. However the second one gives these components one by one so that it more understandable and convenient.

The category with families works as a model for type theory. So we will introduce them from a type theoretical point of view.

The base category is the category for contexts. In the setoid version we interpret a context as a setoid as well.

To define the second component, namely the presheaf functor, it is necessary to construct the target category first. The objects of this category are families of setoids.The index setoids are the semantic types and the indexed families of setoids are terms. The morphisms are component-wise morphisms between setoids. All the categorical laws hold trivially.

\begin{code}\>\<%
\\
%
\\
\>\AgdaFunction{inxSetoids} \AgdaSymbol{:} \AgdaPrimitiveType{Set₁}\<%
\\
\>\AgdaFunction{inxSetoids} \AgdaSymbol{=} \AgdaRecord{Σ[} \AgdaBound{I} \AgdaRecord{∶} \AgdaRecord{hSetoid} \AgdaRecord{]} \AgdaSymbol{(}\AgdaFunction{∣} \AgdaBound{I} \AgdaFunction{∣} \AgdaSymbol{→} \AgdaRecord{hSetoid}\AgdaSymbol{)}\<%
\\
%
\\
\>\AgdaFunction{\_⇉setoid\_} \AgdaSymbol{:} \AgdaFunction{inxSetoids} \AgdaSymbol{→} \AgdaFunction{inxSetoids} \AgdaSymbol{→} \AgdaPrimitiveType{Set₁}\<%
\\
\>\AgdaSymbol{(}\AgdaBound{I} \AgdaInductiveConstructor{,} \AgdaBound{f}\AgdaSymbol{)} \AgdaFunction{⇉setoid} \AgdaSymbol{(}\AgdaBound{J} \AgdaInductiveConstructor{,} \AgdaBound{g}\AgdaSymbol{)} \AgdaSymbol{=} \<[26]%
\>[26]\<%
\\
\>[0]\AgdaIndent{2}{}\<[2]%
\>[2]\AgdaRecord{Σ[} \AgdaBound{i-map} \AgdaRecord{∶} \AgdaBound{I} \AgdaRecord{⇉} \AgdaBound{J} \AgdaRecord{]}\<%
\\
\>[2]\AgdaIndent{4}{}\<[4]%
\>[4]\AgdaSymbol{((}\AgdaBound{i} \AgdaSymbol{:} \AgdaFunction{∣} \AgdaBound{I} \AgdaFunction{∣}\AgdaSymbol{)} \AgdaSymbol{→} \AgdaBound{f} \AgdaBound{i} \AgdaRecord{⇉} \AgdaBound{g} \AgdaSymbol{(} \AgdaFunction{[} \AgdaBound{i-map} \AgdaFunction{]fn} \AgdaBound{i}\AgdaSymbol{))}\<%
\\
%
\\
\>\AgdaFunction{Fam-setoid} \AgdaSymbol{:} \AgdaRecord{Category}\<%
\\
\>\AgdaFunction{Fam-setoid} \AgdaSymbol{=} \AgdaInductiveConstructor{CatC} \<[18]%
\>[18]\<%
\\
\>[4]\AgdaIndent{15}{}\<[15]%
\>[15]\AgdaFunction{inxSetoids} \<[26]%
\>[26]\<%
\\
\>[4]\AgdaIndent{15}{}\<[15]%
\>[15]\AgdaFunction{\_⇉setoid\_} \<[25]%
\>[25]\<%
\\
\>[4]\AgdaIndent{15}{}\<[15]%
\>[15]\AgdaSymbol{(λ} \AgdaBound{\_} \AgdaSymbol{→} \AgdaFunction{id'} \AgdaInductiveConstructor{,} \AgdaSymbol{(λ} \AgdaBound{\_} \AgdaSymbol{→} \AgdaFunction{id'}\AgdaSymbol{))} \<[41]%
\>[41]\<%
\\
\>[4]\AgdaIndent{15}{}\<[15]%
\>[15]\AgdaSymbol{(λ} \AgdaSymbol{\{} \AgdaSymbol{\_} \AgdaSymbol{\_} \AgdaSymbol{(}\AgdaBound{fty} \AgdaInductiveConstructor{,} \AgdaBound{ftm}\AgdaSymbol{)} \AgdaSymbol{(}\AgdaBound{gty} \AgdaInductiveConstructor{,} \AgdaBound{gtm}\AgdaSymbol{)} \AgdaSymbol{→} \AgdaBound{fty} \AgdaFunction{∘c} \AgdaBound{gty} \AgdaInductiveConstructor{,}\<%
\\
\>[15]\AgdaIndent{17}{}\<[17]%
\>[17]\AgdaSymbol{(λ} \AgdaBound{i} \AgdaSymbol{→} \AgdaBound{ftm} \AgdaSymbol{(}\AgdaFunction{[} \AgdaBound{gty} \AgdaFunction{]fn} \AgdaBound{i}\AgdaSymbol{)} \AgdaFunction{∘c} \AgdaBound{gtm} \AgdaBound{i}\AgdaSymbol{)\})}\<%
\\
\>[0]\AgdaIndent{15}{}\<[15]%
\>[15]\AgdaSymbol{(}\AgdaInductiveConstructor{IsCatC} \<[23]%
\>[23]\<%
\\
\>[0]\AgdaIndent{17}{}\<[17]%
\>[17]\AgdaSymbol{(λ} \AgdaBound{α} \AgdaBound{β} \AgdaBound{f} \AgdaSymbol{→} \AgdaInductiveConstructor{PE.refl}\AgdaSymbol{)} \<[37]%
\>[37]\<%
\\
\>[0]\AgdaIndent{17}{}\<[17]%
\>[17]\AgdaSymbol{(λ} \AgdaBound{α} \AgdaBound{β} \AgdaBound{f} \AgdaSymbol{→} \AgdaInductiveConstructor{PE.refl}\AgdaSymbol{)} \<[37]%
\>[37]\<%
\\
\>[0]\AgdaIndent{17}{}\<[17]%
\>[17]\AgdaSymbol{(λ} \AgdaBound{α} \AgdaBound{δ} \AgdaBound{f} \AgdaBound{g} \AgdaBound{h} \AgdaSymbol{→} \AgdaInductiveConstructor{PE.refl}\AgdaSymbol{))}\<%
\\
%
\\
\>\<\end{code}

Since we already specify the category of contexts, we only need the presheaf which is a contravariant functor from the category of contexts to the category we defined above. The definition of category with families of setoids could be as simple as follows.

\begin{code}\>\<%
\\
%
\\
\>\AgdaKeyword{record} \AgdaRecord{CWF-setoid} \AgdaSymbol{:} \AgdaPrimitiveType{Set₁} \AgdaKeyword{where}\<%
\\
\>[0]\AgdaIndent{2}{}\<[2]%
\>[2]\AgdaKeyword{field}\<%
\\
\>[0]\AgdaIndent{4}{}\<[4]%
\>[4]\AgdaField{T} \AgdaSymbol{:} \AgdaRecord{Functor} \AgdaSymbol{(}\AgdaFunction{Op} \AgdaFunction{setoid-Cat}\AgdaSymbol{)} \AgdaFunction{Fam-setoid}\<%
\\
%
\\
\>\<\end{code}

All details of this definition are hidden including the functor laws. Therefore we will show the details as the second version.

The semantic contexts are setoids and the terminal object is just the empty context. 

\begin{code}\>\<%
\\
%
\\
\>\AgdaFunction{Con} \AgdaSymbol{=} \AgdaRecord{hSetoid}\<%
\\
%
\\
\>\AgdaFunction{emptyCon} \AgdaSymbol{=} \AgdaFunction{⊤-setoid}\<%
\\
%
\\
\>\AgdaFunction{emptysub} \AgdaSymbol{=} \AgdaFunction{⋆}\<%
\\
%
\\
\>\<\end{code}

A semantic type has following components. $fm$ is a setoid of all types. $substT$ is the substitution between types within the context. It should be a morphism between setoids so it has to preserve the equivalence relation. We also need to specify the computation rules for substitution.

\begin{code}\>\<%
\\
%
\\
\>\AgdaKeyword{record} \AgdaRecord{Ty} \AgdaSymbol{(}\AgdaBound{Γ} \AgdaSymbol{:} \AgdaFunction{Con}\AgdaSymbol{)} \AgdaSymbol{:} \AgdaPrimitiveType{Set₁} \AgdaKeyword{where}\<%
\\
\>[0]\AgdaIndent{2}{}\<[2]%
\>[2]\AgdaKeyword{field}\<%
\\
\>[0]\AgdaIndent{4}{}\<[4]%
\>[4]\AgdaField{fm} \<[11]%
\>[11]\AgdaSymbol{:} \AgdaFunction{∣} \AgdaBound{Γ} \AgdaFunction{∣} \AgdaSymbol{→} \AgdaRecord{hSetoid}\<%
\\
%
\\
\>[0]\AgdaIndent{4}{}\<[4]%
\>[4]\AgdaField{substT} \AgdaSymbol{:} \AgdaSymbol{\{}\AgdaBound{x} \AgdaBound{y} \AgdaSymbol{:} \AgdaFunction{∣} \AgdaBound{Γ} \AgdaFunction{∣}\AgdaSymbol{\}} \AgdaSymbol{→} \<[29]%
\>[29]\<%
\\
\>[4]\AgdaIndent{13}{}\<[13]%
\>[13]\AgdaFunction{[} \AgdaBound{Γ} \AgdaFunction{]} \AgdaBound{x} \AgdaFunction{≈} \AgdaBound{y} \AgdaSymbol{→}\<%
\\
\>[4]\AgdaIndent{13}{}\<[13]%
\>[13]\AgdaFunction{∣} \AgdaBound{fm} \AgdaBound{x} \AgdaFunction{∣} \AgdaSymbol{→}\<%
\\
\>[4]\AgdaIndent{13}{}\<[13]%
\>[13]\AgdaFunction{∣} \AgdaBound{fm} \AgdaBound{y} \AgdaFunction{∣}\<%
\\
\>[0]\AgdaIndent{4}{}\<[4]%
\>[4]\AgdaField{subst*} \AgdaSymbol{:} \AgdaSymbol{∀\{}\AgdaBound{x} \AgdaBound{y} \AgdaSymbol{:} \AgdaFunction{∣} \AgdaBound{Γ} \AgdaFunction{∣}\AgdaSymbol{\}}\<%
\\
\>[0]\AgdaIndent{13}{}\<[13]%
\>[13]\AgdaSymbol{(}\AgdaBound{p} \AgdaSymbol{:} \AgdaFunction{[} \AgdaBound{Γ} \AgdaFunction{]} \AgdaBound{x} \AgdaFunction{≈} \AgdaBound{y}\AgdaSymbol{)}\<%
\\
\>[0]\AgdaIndent{13}{}\<[13]%
\>[13]\AgdaSymbol{\{}\AgdaBound{a} \AgdaBound{b} \AgdaSymbol{:} \AgdaFunction{∣} \AgdaBound{fm} \AgdaBound{x} \AgdaFunction{∣}\AgdaSymbol{\}} \AgdaSymbol{→}\<%
\\
\>[0]\AgdaIndent{13}{}\<[13]%
\>[13]\AgdaFunction{[} \AgdaBound{fm} \AgdaBound{x} \AgdaFunction{]} \AgdaBound{a} \AgdaFunction{≈} \AgdaBound{b} \AgdaSymbol{→}\<%
\\
\>[0]\AgdaIndent{13}{}\<[13]%
\>[13]\AgdaFunction{[} \AgdaBound{fm} \AgdaBound{y} \AgdaFunction{]} \AgdaBound{substT} \AgdaBound{p} \AgdaBound{a} \AgdaFunction{≈} \AgdaBound{substT} \AgdaBound{p} \AgdaBound{b}\<%
\\
%
\\
\>[0]\AgdaIndent{4}{}\<[4]%
\>[4]\AgdaField{refl*} \<[11]%
\>[11]\AgdaSymbol{:} \AgdaSymbol{∀(}\AgdaBound{x} \AgdaSymbol{:} \AgdaFunction{∣} \AgdaBound{Γ} \AgdaFunction{∣}\AgdaSymbol{)}\<%
\\
\>[0]\AgdaIndent{13}{}\<[13]%
\>[13]\AgdaSymbol{(}\AgdaBound{a} \AgdaSymbol{:} \AgdaFunction{∣} \AgdaBound{fm} \AgdaBound{x} \AgdaFunction{∣}\AgdaSymbol{)} \AgdaSymbol{→} \<[30]%
\>[30]\<%
\\
\>[0]\AgdaIndent{13}{}\<[13]%
\>[13]\AgdaFunction{[} \AgdaBound{fm} \AgdaBound{x} \AgdaFunction{]} \AgdaBound{substT} \AgdaFunction{[} \AgdaBound{Γ} \AgdaFunction{]refl} \AgdaBound{a} \AgdaFunction{≈} \AgdaBound{a}\<%
\\
\>[0]\AgdaIndent{4}{}\<[4]%
\>[4]\AgdaField{trans*} \AgdaSymbol{:} \AgdaSymbol{∀\{}\AgdaBound{x} \AgdaBound{y} \AgdaBound{z} \AgdaSymbol{:} \AgdaFunction{∣} \AgdaBound{Γ} \AgdaFunction{∣}\AgdaSymbol{\}}\<%
\\
\>[0]\AgdaIndent{13}{}\<[13]%
\>[13]\AgdaSymbol{(}\AgdaBound{p} \AgdaSymbol{:} \AgdaFunction{[} \AgdaBound{Γ} \AgdaFunction{]} \AgdaBound{x} \AgdaFunction{≈} \AgdaBound{y}\AgdaSymbol{)}\<%
\\
\>[0]\AgdaIndent{13}{}\<[13]%
\>[13]\AgdaSymbol{(}\AgdaBound{q} \AgdaSymbol{:} \AgdaFunction{[} \AgdaBound{Γ} \AgdaFunction{]} \AgdaBound{y} \AgdaFunction{≈} \AgdaBound{z}\AgdaSymbol{)}\<%
\\
\>[0]\AgdaIndent{13}{}\<[13]%
\>[13]\AgdaSymbol{(}\AgdaBound{a} \AgdaSymbol{:} \AgdaFunction{∣} \AgdaBound{fm} \AgdaBound{x} \AgdaFunction{∣}\AgdaSymbol{)} \<[28]%
\>[28]\<%
\\
\>[0]\AgdaIndent{13}{}\<[13]%
\>[13]\AgdaSymbol{→} \AgdaFunction{[} \AgdaBound{fm} \AgdaBound{z} \AgdaFunction{]} \AgdaBound{substT} \AgdaBound{q} \AgdaSymbol{(}\AgdaBound{substT} \AgdaBound{p} \AgdaBound{a}\AgdaSymbol{)} \<[46]%
\>[46]\<%
\\
\>[13]\AgdaIndent{17}{}\<[17]%
\>[17]\AgdaFunction{≈} \AgdaBound{substT} \AgdaSymbol{(}\AgdaFunction{[} \AgdaBound{Γ} \AgdaFunction{]trans} \AgdaBound{p} \AgdaBound{q}\AgdaSymbol{)} \AgdaBound{a}\<%
\\
%
\\
%
\\
\>\<\end{code}

Some other lemmas on the proof irrelevance derived from these fields are not shown here since they are just auxiliary functions.

\AgdaHide{
\begin{code}\>\<%
\\
\>\AgdaComment{-- the proof-irrelevance lemmas for substT}\<%
\\
%
\\
\>[2]\AgdaIndent{2}{}\<[2]%
\>[2]\AgdaFunction{subst-pi} \AgdaSymbol{:} \AgdaSymbol{∀\{}\AgdaBound{x} \AgdaBound{y} \AgdaSymbol{:} \AgdaFunction{∣} \AgdaBound{Γ} \AgdaFunction{∣}\AgdaSymbol{\}}\<%
\\
\>[0]\AgdaIndent{14}{}\<[14]%
\>[14]\AgdaSymbol{\{}\AgdaBound{p} \AgdaBound{q} \AgdaSymbol{:} \AgdaFunction{[} \AgdaBound{Γ} \AgdaFunction{]} \AgdaBound{x} \AgdaFunction{≈} \AgdaBound{y}\AgdaSymbol{\}}\<%
\\
\>[0]\AgdaIndent{14}{}\<[14]%
\>[14]\AgdaSymbol{\{}\AgdaBound{a} \AgdaSymbol{:} \AgdaFunction{∣} \AgdaFunction{fm} \AgdaBound{x} \AgdaFunction{∣}\AgdaSymbol{\}} \AgdaSymbol{→} \AgdaFunction{[} \AgdaFunction{fm} \AgdaBound{y} \AgdaFunction{]} \AgdaFunction{substT} \AgdaBound{p} \AgdaBound{a} \AgdaFunction{≈} \AgdaFunction{substT} \AgdaBound{q} \AgdaBound{a}\<%
\\
\>[0]\AgdaIndent{2}{}\<[2]%
\>[2]\AgdaFunction{subst-pi} \AgdaSymbol{\{}\AgdaBound{x}\AgdaSymbol{\}} \AgdaSymbol{\{}\AgdaBound{y}\AgdaSymbol{\}} \AgdaSymbol{\{}\AgdaBound{p}\AgdaSymbol{\}} \AgdaSymbol{\{}\AgdaBound{q}\AgdaSymbol{\}} \AgdaSymbol{\{}\AgdaBound{a}\AgdaSymbol{\}} \AgdaSymbol{=} \AgdaFunction{reflexive} \AgdaSymbol{(}\AgdaFunction{fm} \AgdaBound{y}\AgdaSymbol{)} \AgdaSymbol{(}\AgdaFunction{PI} \AgdaBound{Γ} \AgdaSymbol{(λ} \AgdaBound{x} \AgdaSymbol{→} \AgdaFunction{substT} \AgdaBound{x} \AgdaBound{a}\AgdaSymbol{))}\<%
\\
%
\\
\>[0]\AgdaIndent{2}{}\<[2]%
\>[2]\AgdaFunction{subst-pi'} \AgdaSymbol{:} \AgdaSymbol{∀\{}\AgdaBound{x} \AgdaSymbol{:} \AgdaFunction{∣} \AgdaBound{Γ} \AgdaFunction{∣}\AgdaSymbol{\}}\<%
\\
\>[2]\AgdaIndent{15}{}\<[15]%
\>[15]\AgdaSymbol{\{}\AgdaBound{p} \AgdaSymbol{:} \AgdaFunction{[} \AgdaBound{Γ} \AgdaFunction{]} \AgdaBound{x} \AgdaFunction{≈} \AgdaBound{x}\AgdaSymbol{\}}\<%
\\
\>[2]\AgdaIndent{15}{}\<[15]%
\>[15]\AgdaSymbol{\{}\AgdaBound{a} \AgdaSymbol{:} \AgdaFunction{∣} \AgdaFunction{fm} \AgdaBound{x} \AgdaFunction{∣}\AgdaSymbol{\}} \AgdaSymbol{→} \AgdaFunction{[} \AgdaFunction{fm} \AgdaBound{x} \AgdaFunction{]} \AgdaFunction{substT} \AgdaBound{p} \AgdaBound{a} \AgdaFunction{≈} \AgdaBound{a}\<%
\\
\>[0]\AgdaIndent{2}{}\<[2]%
\>[2]\AgdaFunction{subst-pi'} \AgdaSymbol{=} \AgdaFunction{[} \AgdaFunction{fm} \AgdaSymbol{\_} \AgdaFunction{]trans} \AgdaFunction{subst-pi} \AgdaSymbol{(}\AgdaFunction{refl*} \AgdaSymbol{\_} \AgdaSymbol{\_)}\<%
\\
%
\\
\>[0]\AgdaIndent{2}{}\<[2]%
\>[2]\AgdaFunction{subst-pi*} \AgdaSymbol{:} \AgdaSymbol{∀\{}\AgdaBound{x} \AgdaBound{y} \AgdaSymbol{:} \AgdaFunction{∣} \AgdaBound{Γ} \AgdaFunction{∣}\AgdaSymbol{\}}\<%
\\
\>[2]\AgdaIndent{16}{}\<[16]%
\>[16]\AgdaSymbol{\{}\AgdaBound{p} \AgdaBound{q} \AgdaSymbol{:} \AgdaFunction{[} \AgdaBound{Γ} \AgdaFunction{]} \AgdaBound{x} \AgdaFunction{≈} \AgdaBound{y}\AgdaSymbol{\}}\<%
\\
\>[2]\AgdaIndent{16}{}\<[16]%
\>[16]\AgdaSymbol{\{}\AgdaBound{a} \AgdaBound{b} \AgdaSymbol{:} \AgdaFunction{∣} \AgdaFunction{fm} \AgdaBound{x} \AgdaFunction{∣}\AgdaSymbol{\}} \AgdaSymbol{→} \AgdaFunction{[} \AgdaFunction{fm} \AgdaBound{x} \AgdaFunction{]} \AgdaBound{a} \AgdaFunction{≈} \AgdaBound{b} \AgdaSymbol{→} \AgdaFunction{[} \AgdaFunction{fm} \AgdaBound{y} \AgdaFunction{]} \AgdaFunction{substT} \AgdaBound{p} \AgdaBound{a} \AgdaFunction{≈} \AgdaFunction{substT} \AgdaBound{q} \AgdaBound{b}\<%
\\
\>[0]\AgdaIndent{2}{}\<[2]%
\>[2]\AgdaFunction{subst-pi*} \AgdaBound{eq} \AgdaSymbol{=} \AgdaFunction{[} \AgdaFunction{fm} \AgdaSymbol{\_} \AgdaFunction{]trans} \AgdaSymbol{(}\AgdaFunction{subst*} \AgdaSymbol{\_} \AgdaBound{eq}\AgdaSymbol{)} \AgdaFunction{subst-pi}\<%
\\
%
\\
%
\\
\>\AgdaComment{-- simplify proofs of trans of inverse equality (including groupoid laws?)}\<%
\\
%
\\
\>[0]\AgdaIndent{2}{}\<[2]%
\>[2]\AgdaFunction{trans-refl} \AgdaSymbol{:} \AgdaSymbol{∀\{}\AgdaBound{x} \AgdaBound{y} \AgdaSymbol{:} \AgdaFunction{∣} \AgdaBound{Γ} \AgdaFunction{∣}\AgdaSymbol{\}}\<%
\\
\>[2]\AgdaIndent{14}{}\<[14]%
\>[14]\AgdaSymbol{\{}\AgdaBound{p} \AgdaSymbol{:} \AgdaFunction{[} \AgdaBound{Γ} \AgdaFunction{]} \AgdaBound{x} \AgdaFunction{≈} \AgdaBound{y}\AgdaSymbol{\}\{}\AgdaBound{q} \AgdaSymbol{:} \AgdaFunction{[} \AgdaBound{Γ} \AgdaFunction{]} \AgdaBound{y} \AgdaFunction{≈} \AgdaBound{x}\AgdaSymbol{\}}\<%
\\
\>[2]\AgdaIndent{14}{}\<[14]%
\>[14]\AgdaSymbol{\{}\AgdaBound{a} \AgdaSymbol{:} \AgdaFunction{∣} \AgdaFunction{fm} \AgdaBound{x} \AgdaFunction{∣}\AgdaSymbol{\}} \AgdaSymbol{→} \<[31]%
\>[31]\<%
\\
\>[2]\AgdaIndent{14}{}\<[14]%
\>[14]\AgdaFunction{[} \AgdaFunction{fm} \AgdaBound{x} \AgdaFunction{]} \AgdaFunction{substT} \AgdaBound{q} \AgdaSymbol{(}\AgdaFunction{substT} \AgdaBound{p} \AgdaBound{a}\AgdaSymbol{)} \AgdaFunction{≈} \AgdaBound{a}\<%
\\
\>[0]\AgdaIndent{2}{}\<[2]%
\>[2]\AgdaFunction{trans-refl} \AgdaSymbol{=} \AgdaFunction{[} \AgdaFunction{fm} \AgdaSymbol{\_} \AgdaFunction{]trans} \AgdaSymbol{(}\AgdaFunction{trans*} \AgdaSymbol{\_} \AgdaSymbol{\_} \AgdaSymbol{\_)} \AgdaFunction{subst-pi'}\<%
\\
%
\\
\>\AgdaComment{-- some more theorems}\<%
\\
\>[0]\AgdaIndent{2}{}\<[2]%
\>[2]\<%
\\
\>[0]\AgdaIndent{2}{}\<[2]%
\>[2]\AgdaFunction{subst-mir1} \AgdaSymbol{:} \AgdaSymbol{∀\{}\AgdaBound{x} \AgdaBound{y} \AgdaSymbol{:} \AgdaFunction{∣} \AgdaBound{Γ} \AgdaFunction{∣}\AgdaSymbol{\}}\<%
\\
\>[2]\AgdaIndent{14}{}\<[14]%
\>[14]\AgdaSymbol{\{}\AgdaBound{p} \AgdaSymbol{:} \AgdaFunction{[} \AgdaBound{Γ} \AgdaFunction{]} \AgdaBound{x} \AgdaFunction{≈} \AgdaBound{y}\AgdaSymbol{\}\{}\AgdaBound{q} \AgdaSymbol{:} \AgdaFunction{[} \AgdaBound{Γ} \AgdaFunction{]} \AgdaBound{y} \AgdaFunction{≈} \AgdaBound{x}\AgdaSymbol{\}}\<%
\\
\>[2]\AgdaIndent{14}{}\<[14]%
\>[14]\AgdaSymbol{\{}\AgdaBound{a} \AgdaSymbol{:} \AgdaFunction{∣} \AgdaFunction{fm} \AgdaBound{x} \AgdaFunction{∣}\AgdaSymbol{\}\{}\AgdaBound{b} \AgdaSymbol{:} \AgdaFunction{∣} \AgdaFunction{fm} \AgdaBound{y} \AgdaFunction{∣}\AgdaSymbol{\}} \AgdaSymbol{→} \<[45]%
\>[45]\<%
\\
\>[2]\AgdaIndent{14}{}\<[14]%
\>[14]\AgdaFunction{[} \AgdaFunction{fm} \AgdaBound{x} \AgdaFunction{]} \AgdaBound{a} \AgdaFunction{≈} \AgdaFunction{substT} \AgdaBound{q} \AgdaBound{b} \AgdaSymbol{→} \AgdaFunction{[} \AgdaFunction{fm} \AgdaBound{y} \AgdaFunction{]} \AgdaFunction{substT} \AgdaBound{p} \AgdaBound{a} \AgdaFunction{≈} \AgdaBound{b}\<%
\\
\>[0]\AgdaIndent{2}{}\<[2]%
\>[2]\AgdaFunction{subst-mir1} \AgdaBound{eq} \AgdaSymbol{=} \AgdaFunction{[} \AgdaFunction{fm} \AgdaSymbol{\_} \AgdaFunction{]trans} \AgdaSymbol{(}\AgdaFunction{subst*} \AgdaSymbol{\_} \AgdaBound{eq}\AgdaSymbol{)} \AgdaFunction{trans-refl}\<%
\\
%
\\
\>[0]\AgdaIndent{2}{}\<[2]%
\>[2]\AgdaFunction{subst-mir2} \AgdaSymbol{:} \AgdaSymbol{∀\{}\AgdaBound{x} \AgdaBound{y} \AgdaSymbol{:} \AgdaFunction{∣} \AgdaBound{Γ} \AgdaFunction{∣}\AgdaSymbol{\}}\<%
\\
\>[2]\AgdaIndent{14}{}\<[14]%
\>[14]\AgdaSymbol{\{}\AgdaBound{p} \AgdaSymbol{:} \AgdaFunction{[} \AgdaBound{Γ} \AgdaFunction{]} \AgdaBound{x} \AgdaFunction{≈} \AgdaBound{y}\AgdaSymbol{\}\{}\AgdaBound{q} \AgdaSymbol{:} \AgdaFunction{[} \AgdaBound{Γ} \AgdaFunction{]} \AgdaBound{y} \AgdaFunction{≈} \AgdaBound{x}\AgdaSymbol{\}}\<%
\\
\>[2]\AgdaIndent{14}{}\<[14]%
\>[14]\AgdaSymbol{\{}\AgdaBound{a} \AgdaSymbol{:} \AgdaFunction{∣} \AgdaFunction{fm} \AgdaBound{x} \AgdaFunction{∣}\AgdaSymbol{\}\{}\AgdaBound{b} \AgdaSymbol{:} \AgdaFunction{∣} \AgdaFunction{fm} \AgdaBound{y} \AgdaFunction{∣}\AgdaSymbol{\}} \AgdaSymbol{→} \<[45]%
\>[45]\<%
\\
\>[2]\AgdaIndent{14}{}\<[14]%
\>[14]\AgdaFunction{[} \AgdaFunction{fm} \AgdaBound{y} \AgdaFunction{]} \AgdaFunction{substT} \AgdaBound{p} \AgdaBound{a} \AgdaFunction{≈} \AgdaBound{b} \AgdaSymbol{→} \AgdaFunction{[} \AgdaFunction{fm} \AgdaBound{x} \AgdaFunction{]} \AgdaBound{a} \AgdaFunction{≈} \AgdaFunction{substT} \AgdaBound{q} \AgdaBound{b}\<%
\\
\>[0]\AgdaIndent{2}{}\<[2]%
\>[2]\AgdaFunction{subst-mir2} \AgdaBound{eq} \AgdaSymbol{=} \AgdaFunction{[} \AgdaFunction{fm} \AgdaSymbol{\_} \AgdaFunction{]sym} \AgdaSymbol{(}\AgdaFunction{subst-mir1} \AgdaSymbol{(}\AgdaFunction{[} \AgdaFunction{fm} \AgdaSymbol{\_} \AgdaFunction{]sym} \AgdaBound{eq}\AgdaSymbol{))}\<%
\\
%
\\
\>\AgdaKeyword{open} \AgdaModule{Ty} \AgdaKeyword{public} \<[15]%
\>[15]\<%
\\
\>[0]\AgdaIndent{2}{}\<[2]%
\>[2]\AgdaKeyword{renaming} \AgdaSymbol{(}substT \AgdaSymbol{to} [\_]subst\AgdaSymbol{;} subst* \AgdaSymbol{to} [\_]subst*\AgdaSymbol{;} fm \AgdaSymbol{to} [\_]fm \AgdaSymbol{;}
            refl* \AgdaSymbol{to} [\_]refl* \AgdaSymbol{;} trans* \AgdaSymbol{to} [\_]trans*\AgdaSymbol{;} subst-pi \AgdaSymbol{to} [\_]subst-pi \AgdaSymbol{;}
            subst-pi' \AgdaSymbol{to} [\_]subst-pi' \AgdaSymbol{;} subst-pi* \AgdaSymbol{to} [\_]subst-pi* \AgdaSymbol{;}
            trans-refl \AgdaSymbol{to} [\_]trans-refl \AgdaSymbol{;} subst-mir1 \AgdaSymbol{to} [\_]subst-mir1 \AgdaSymbol{;}
            subst-mir2 \AgdaSymbol{to} [\_]subst-mir2\AgdaSymbol{)}\<%
\\
%
\\
\>\<\end{code}
}

Then we have to define the substituting in a type given a context morphism and verify it preserves equivalence relation as well.

\begin{code}\>\<%
\\
%
\\
\>\AgdaFunction{\_[\_]T} \AgdaSymbol{:} \AgdaSymbol{∀} \AgdaSymbol{\{}\AgdaBound{Γ} \AgdaBound{Δ} \AgdaSymbol{:} \AgdaFunction{Con}\AgdaSymbol{\}} \AgdaSymbol{→} \AgdaRecord{Ty} \AgdaBound{Δ} \AgdaSymbol{→} \AgdaBound{Γ} \AgdaRecord{⇉} \AgdaBound{Δ} \AgdaSymbol{→} \AgdaRecord{Ty} \AgdaBound{Γ}\<%
\\
\>\AgdaBound{A} \AgdaFunction{[} \AgdaBound{f} \AgdaFunction{]T}\<%
\\
\>[2]\AgdaIndent{5}{}\<[5]%
\>[5]\AgdaSymbol{=} \AgdaKeyword{record}\<%
\\
\>[2]\AgdaIndent{5}{}\<[5]%
\>[5]\AgdaSymbol{\{} \AgdaField{fm} \<[14]%
\>[14]\AgdaSymbol{=} \AgdaFunction{fm} \AgdaFunction{∘} \AgdaFunction{fn}\<%
\\
\>[2]\AgdaIndent{5}{}\<[5]%
\>[5]\AgdaSymbol{;} \AgdaField{substT} \AgdaSymbol{=} \AgdaFunction{substT} \AgdaFunction{∘} \AgdaFunction{resp}\<%
\\
\>[2]\AgdaIndent{5}{}\<[5]%
\>[5]\AgdaSymbol{;} \AgdaField{subst*} \AgdaSymbol{=} \AgdaFunction{subst*} \AgdaFunction{∘} \AgdaFunction{resp}\<%
\\
\>[2]\AgdaIndent{5}{}\<[5]%
\>[5]\AgdaSymbol{;} \AgdaField{refl*} \<[14]%
\>[14]\AgdaSymbol{=} \AgdaSymbol{λ} \AgdaBound{\_} \AgdaBound{\_} \AgdaSymbol{→} \AgdaFunction{subst-pi'}\<%
\\
\>[2]\AgdaIndent{5}{}\<[5]%
\>[5]\AgdaSymbol{;} \AgdaField{trans*} \AgdaSymbol{=} \AgdaSymbol{λ} \AgdaBound{\_} \AgdaBound{\_} \AgdaBound{\_} \AgdaSymbol{→} \<[26]%
\>[26]\<%
\\
\>[5]\AgdaIndent{16}{}\<[16]%
\>[16]\AgdaFunction{[} \AgdaFunction{fm} \AgdaSymbol{(}\AgdaFunction{fn} \AgdaSymbol{\_)} \AgdaFunction{]trans} \AgdaSymbol{(}\AgdaFunction{trans*} \AgdaSymbol{\_} \AgdaSymbol{\_} \AgdaSymbol{\_)} \AgdaFunction{subst-pi}\<%
\\
\>[0]\AgdaIndent{5}{}\<[5]%
\>[5]\AgdaSymbol{\}}\<%
\\
\>[0]\AgdaIndent{5}{}\<[5]%
\>[5]\AgdaKeyword{where} \<[11]%
\>[11]\<%
\\
\>[5]\AgdaIndent{7}{}\<[7]%
\>[7]\AgdaKeyword{open} \AgdaModule{Ty} \AgdaBound{A}\<%
\\
\>[5]\AgdaIndent{7}{}\<[7]%
\>[7]\AgdaKeyword{open} \AgdaModule{\_⇉\_} \AgdaBound{f}\<%
\\
%
\\
\>\<\end{code}

The semantic terms are simpler. It should also preserve the equivalence relation on the elements of contexts.

\begin{code}\>\<%
\\
%
\\
\>\AgdaKeyword{record} \AgdaRecord{Tm} \AgdaSymbol{\{}\AgdaBound{Γ} \AgdaSymbol{:} \AgdaFunction{Con}\AgdaSymbol{\}(}\AgdaBound{A} \AgdaSymbol{:} \AgdaRecord{Ty} \AgdaBound{Γ}\AgdaSymbol{)} \AgdaSymbol{:} \AgdaPrimitiveType{Set} \AgdaKeyword{where}\<%
\\
\>[0]\AgdaIndent{2}{}\<[2]%
\>[2]\AgdaKeyword{constructor} \AgdaInductiveConstructor{tm:\_resp:\_}\<%
\\
\>[0]\AgdaIndent{2}{}\<[2]%
\>[2]\AgdaKeyword{field}\<%
\\
\>[2]\AgdaIndent{4}{}\<[4]%
\>[4]\AgdaField{tm} \<[10]%
\>[10]\AgdaSymbol{:} \AgdaSymbol{(}\AgdaBound{x} \AgdaSymbol{:} \AgdaFunction{∣} \AgdaBound{Γ} \AgdaFunction{∣}\AgdaSymbol{)} \AgdaSymbol{→} \AgdaFunction{∣} \AgdaFunction{[} \AgdaBound{A} \AgdaFunction{]fm} \AgdaBound{x} \AgdaFunction{∣}\<%
\\
\>[2]\AgdaIndent{4}{}\<[4]%
\>[4]\AgdaField{respt} \AgdaSymbol{:} \AgdaSymbol{∀} \AgdaSymbol{\{}\AgdaBound{x} \AgdaBound{y} \AgdaSymbol{:} \AgdaFunction{∣} \AgdaBound{Γ} \AgdaFunction{∣}\AgdaSymbol{\}} \AgdaSymbol{→} \<[30]%
\>[30]\<%
\\
\>[4]\AgdaIndent{14}{}\<[14]%
\>[14]\AgdaSymbol{(}\AgdaBound{p} \AgdaSymbol{:} \AgdaFunction{[} \AgdaBound{Γ} \AgdaFunction{]} \AgdaBound{x} \AgdaFunction{≈} \AgdaBound{y}\AgdaSymbol{)} \AgdaSymbol{→} \<[34]%
\>[34]\<%
\\
\>[4]\AgdaIndent{14}{}\<[14]%
\>[14]\AgdaFunction{[} \AgdaFunction{[} \AgdaBound{A} \AgdaFunction{]fm} \AgdaBound{y} \AgdaFunction{]} \AgdaFunction{[} \AgdaBound{A} \AgdaFunction{]subst} \AgdaBound{p} \AgdaSymbol{(}\AgdaBound{tm} \AgdaBound{x}\AgdaSymbol{)} \AgdaFunction{≈} \AgdaBound{tm} \AgdaBound{y}\<%
\\
%
\\
\>\<\end{code}

\AgdaHide{
\begin{code}\>\<%
\\
\>\AgdaKeyword{open} \AgdaModule{Tm} \AgdaKeyword{public} \AgdaKeyword{renaming} \AgdaSymbol{(}tm \AgdaSymbol{to} [\_]tm \AgdaSymbol{;} respt \AgdaSymbol{to} [\_]respt\AgdaSymbol{)}\<%
\\
%
\\
\>\<\end{code}
}

Substitution for terms can be defined as

\begin{code}\>\<%
\\
%
\\
\>\AgdaFunction{\_[\_]m} \AgdaSymbol{:} \AgdaSymbol{∀} \AgdaSymbol{\{}\AgdaBound{Γ} \AgdaBound{Δ} \AgdaSymbol{:} \AgdaFunction{Con}\AgdaSymbol{\}\{}\AgdaBound{A} \AgdaSymbol{:} \AgdaRecord{Ty} \AgdaBound{Δ}\AgdaSymbol{\}} \AgdaSymbol{→} \<[34]%
\>[34]\<%
\\
\>[0]\AgdaIndent{10}{}\<[10]%
\>[10]\AgdaRecord{Tm} \AgdaBound{A} \AgdaSymbol{→} \<[17]%
\>[17]\<%
\\
\>[0]\AgdaIndent{10}{}\<[10]%
\>[10]\AgdaSymbol{(}\AgdaBound{f} \AgdaSymbol{:} \AgdaBound{Γ} \AgdaRecord{⇉} \AgdaBound{Δ}\AgdaSymbol{)} \<[22]%
\>[22]\<%
\\
\>[0]\AgdaIndent{10}{}\<[10]%
\>[10]\AgdaSymbol{→} \AgdaRecord{Tm} \AgdaSymbol{(}\AgdaBound{A} \AgdaFunction{[} \AgdaBound{f} \AgdaFunction{]T}\AgdaSymbol{)}\<%
\\
\>\AgdaFunction{\_[\_]m} \AgdaBound{t} \AgdaBound{f} \AgdaSymbol{=} \AgdaKeyword{record} \<[19]%
\>[19]\<%
\\
\>[0]\AgdaIndent{10}{}\<[10]%
\>[10]\AgdaSymbol{\{} \AgdaField{tm} \AgdaSymbol{=} \AgdaFunction{[} \AgdaBound{t} \AgdaFunction{]tm} \AgdaFunction{∘} \AgdaFunction{[} \AgdaBound{f} \AgdaFunction{]fn}\<%
\\
\>[0]\AgdaIndent{10}{}\<[10]%
\>[10]\AgdaSymbol{;} \AgdaField{respt} \AgdaSymbol{=} \AgdaFunction{[} \AgdaBound{t} \AgdaFunction{]respt} \AgdaFunction{∘} \AgdaFunction{[} \AgdaBound{f} \AgdaFunction{]resp} \<[43]%
\>[43]\<%
\\
\>[0]\AgdaIndent{10}{}\<[10]%
\>[10]\AgdaSymbol{\}}\<%
\\
%
\\
\>\<\end{code}

Syntactically we can form a new context by using a context $\Gamma$ and a type $A : Ty \:\Gamma$. To introduce a term of it, we need a term of the semantic context $\Gamma$ and a term of semantic type $A$. It is called context comprehension. 

\begin{code}\>\<%
\\
%
\\
\>\AgdaFunction{\_\&\_} \AgdaSymbol{:} \AgdaSymbol{(}\AgdaBound{Γ} \AgdaSymbol{:} \AgdaFunction{Con}\AgdaSymbol{)} \AgdaSymbol{→} \AgdaRecord{Ty} \AgdaBound{Γ} \AgdaSymbol{→} \AgdaFunction{Con}\<%
\\
\>\AgdaBound{Γ} \AgdaFunction{\&} \AgdaBound{A} \AgdaSymbol{=} \AgdaKeyword{record} \<[15]%
\>[15]\<%
\\
\>[0]\AgdaIndent{7}{}\<[7]%
\>[7]\AgdaSymbol{\{} \AgdaField{Carrier} \AgdaSymbol{=} \AgdaRecord{Σ[} \AgdaBound{x} \AgdaRecord{∶} \AgdaFunction{∣} \AgdaBound{Γ} \AgdaFunction{∣} \AgdaRecord{]} \AgdaFunction{∣} \AgdaFunction{fm} \AgdaBound{x} \AgdaFunction{∣}\<%
\\
\>[0]\AgdaIndent{7}{}\<[7]%
\>[7]\AgdaSymbol{;} \AgdaField{\_≈h\_} \<[17]%
\>[17]\AgdaSymbol{=} \AgdaSymbol{λ\{(}\AgdaBound{x} \AgdaInductiveConstructor{,} \AgdaBound{a}\AgdaSymbol{)} \AgdaSymbol{(}\AgdaBound{y} \AgdaInductiveConstructor{,} \AgdaBound{b}\AgdaSymbol{)} \AgdaSymbol{→} \<[39]%
\>[39]\<%
\\
\>[7]\AgdaIndent{19}{}\<[19]%
\>[19]\AgdaFunction{Σ'[} \AgdaBound{p} \AgdaFunction{∶} \AgdaBound{x} \AgdaFunction{≈h} \AgdaBound{y} \AgdaFunction{]} \AgdaFunction{[} \AgdaFunction{fm} \AgdaBound{y} \AgdaFunction{]} \AgdaFunction{substT} \AgdaBound{p} \AgdaBound{a} \AgdaFunction{≈h} \AgdaBound{b}\AgdaSymbol{\}}\<%
\\
\>[0]\AgdaIndent{7}{}\<[7]%
\>[7]\AgdaSymbol{;} \AgdaField{isEquiv} \AgdaSymbol{=} \<[19]%
\>[19]\<%
\\
\>[0]\AgdaIndent{10}{}\<[10]%
\>[10]\AgdaKeyword{record} \<[17]%
\>[17]\<%
\\
\>[0]\AgdaIndent{10}{}\<[10]%
\>[10]\AgdaSymbol{\{} \AgdaField{refl} \<[18]%
\>[18]\AgdaSymbol{=} \AgdaFunction{refl} \AgdaInductiveConstructor{,} \AgdaSymbol{(}\AgdaFunction{refl*} \AgdaSymbol{\_} \AgdaSymbol{\_)}\<%
\\
\>[0]\AgdaIndent{10}{}\<[10]%
\>[10]\AgdaSymbol{;} \AgdaField{sym} \<[18]%
\>[18]\AgdaSymbol{=} \AgdaSymbol{λ} \AgdaSymbol{\{(}\AgdaBound{p} \AgdaInductiveConstructor{,} \AgdaBound{q}\AgdaSymbol{)} \AgdaSymbol{→} \AgdaSymbol{(}\AgdaFunction{sym} \AgdaBound{p}\AgdaSymbol{)} \AgdaInductiveConstructor{,} \<[43]%
\>[43]\<%
\\
\>[10]\AgdaIndent{20}{}\<[20]%
\>[20]\AgdaFunction{[} \AgdaFunction{fm} \AgdaSymbol{\_} \AgdaFunction{]trans} \<[34]%
\>[34]\<%
\\
\>[10]\AgdaIndent{20}{}\<[20]%
\>[20]\AgdaSymbol{(}\AgdaFunction{subst*} \AgdaSymbol{\_} \AgdaSymbol{(}\AgdaFunction{[} \AgdaFunction{fm} \AgdaSymbol{\_} \AgdaFunction{]sym} \AgdaBound{q}\AgdaSymbol{))} \<[47]%
\>[47]\<%
\\
\>[10]\AgdaIndent{20}{}\<[20]%
\>[20]\AgdaFunction{trans-refl} \AgdaSymbol{\}}\<%
\\
\>[0]\AgdaIndent{10}{}\<[10]%
\>[10]\AgdaSymbol{;} \AgdaField{trans} \AgdaSymbol{=} \AgdaSymbol{λ} \AgdaSymbol{\{(}\AgdaBound{p} \AgdaInductiveConstructor{,} \AgdaBound{q}\AgdaSymbol{)} \AgdaSymbol{(}\AgdaBound{m} \AgdaInductiveConstructor{,} \AgdaBound{n}\AgdaSymbol{)} \AgdaSymbol{→}\<%
\\
\>[0]\AgdaIndent{20}{}\<[20]%
\>[20]\AgdaFunction{trans} \AgdaBound{p} \AgdaBound{m} \AgdaInductiveConstructor{,} \<[32]%
\>[32]\<%
\\
\>[0]\AgdaIndent{20}{}\<[20]%
\>[20]\AgdaFunction{[} \AgdaFunction{fm} \AgdaSymbol{\_} \AgdaFunction{]trans} \<[34]%
\>[34]\<%
\\
\>[0]\AgdaIndent{20}{}\<[20]%
\>[20]\AgdaSymbol{(}\AgdaFunction{[} \AgdaFunction{fm} \AgdaSymbol{\_} \AgdaFunction{]trans} \<[35]%
\>[35]\<%
\\
\>[0]\AgdaIndent{20}{}\<[20]%
\>[20]\AgdaSymbol{(}\AgdaFunction{[} \AgdaFunction{fm} \AgdaSymbol{\_} \AgdaFunction{]sym} \AgdaSymbol{(}\AgdaFunction{trans*} \AgdaSymbol{\_} \AgdaSymbol{\_} \AgdaSymbol{\_))} \AgdaSymbol{(}\AgdaFunction{subst*} \AgdaSymbol{\_} \AgdaBound{q}\AgdaSymbol{))} \AgdaBound{n} \AgdaSymbol{\}}\<%
\\
\>[0]\AgdaIndent{10}{}\<[10]%
\>[10]\AgdaSymbol{\}}\<%
\\
\>[0]\AgdaIndent{7}{}\<[7]%
\>[7]\AgdaSymbol{\}}\<%
\\
%
\\
%
\\
\>\<\end{code}

\AgdaHide{
\begin{code}\>\<%
\\
%
\\
\>[0]\AgdaIndent{7}{}\<[7]%
\>[7]\AgdaKeyword{where} \<[13]%
\>[13]\<%
\\
\>[7]\AgdaIndent{9}{}\<[9]%
\>[9]\AgdaKeyword{open} \AgdaModule{hSetoid} \AgdaBound{Γ}\<%
\\
\>[7]\AgdaIndent{9}{}\<[9]%
\>[9]\AgdaKeyword{open} \AgdaModule{Ty} \AgdaBound{A} \<[23]%
\>[23]\<%
\\
%
\\
\>\<\end{code}
}

There are also some other morphisms come with it. Any morphism from a context $\Gamma$ to a context $\Delta \& A$ consists of a morphism from $\Gamma$ to $\Delta$ and a term of type $A$ substituted. In other words, There is an isomorphism between $Hom(\Gamma , \Delta \& A)$ and $\Sigma \gamma : Hom(\Gamma , \Delta) A [ \gamma ] $.

$fst$ projects the morphism and  $snd$ projects the term.
Indeed the $fst$ operation provides weakening for types, and the $snd$ projection enables us to interpret variables. $fst\&$ defines a morphism for each type $A$ which is a canonical projection of $A$.
We need to use $id'$ which are identity context morphisms to achieve these.

\begin{code}\>\<%
\\
%
\\
\>\AgdaFunction{fst} \AgdaSymbol{:} \AgdaSymbol{\{}\AgdaBound{Γ} \AgdaBound{Δ} \AgdaSymbol{:} \AgdaFunction{Con}\AgdaSymbol{\}(}\AgdaBound{A} \AgdaSymbol{:} \AgdaRecord{Ty} \AgdaBound{Δ}\AgdaSymbol{)} \AgdaSymbol{→} \AgdaBound{Γ} \AgdaRecord{⇉} \AgdaSymbol{(}\AgdaBound{Δ} \AgdaFunction{\&} \AgdaBound{A}\AgdaSymbol{)} \AgdaSymbol{→} \AgdaBound{Γ} \AgdaRecord{⇉} \AgdaBound{Δ}\<%
\\
\>\AgdaFunction{fst} \AgdaBound{A} \AgdaBound{f} \AgdaSymbol{=} \AgdaKeyword{record} \<[17]%
\>[17]\<%
\\
\>[-6]\AgdaIndent{8}{}\<[8]%
\>[8]\AgdaSymbol{\{} \AgdaField{fn} \AgdaSymbol{=} \AgdaFunction{proj₁} \AgdaFunction{∘} \AgdaFunction{[} \AgdaBound{f} \AgdaFunction{]fn}\<%
\\
\>[0]\AgdaIndent{8}{}\<[8]%
\>[8]\AgdaSymbol{;} \AgdaField{resp} \AgdaSymbol{=} \AgdaFunction{proj₁} \AgdaFunction{∘} \AgdaFunction{[} \AgdaBound{f} \AgdaFunction{]resp} \<[35]%
\>[35]\<%
\\
\>[0]\AgdaIndent{8}{}\<[8]%
\>[8]\AgdaSymbol{\}}\<%
\\
%
\\
\>\AgdaFunction{fst\&} \AgdaSymbol{:} \AgdaSymbol{\{}\AgdaBound{Γ} \AgdaSymbol{:} \AgdaFunction{Con}\AgdaSymbol{\}(}\AgdaBound{A} \AgdaSymbol{:} \AgdaRecord{Ty} \AgdaBound{Γ}\AgdaSymbol{)} \AgdaSymbol{→} \AgdaBound{Γ} \AgdaFunction{\&} \AgdaBound{A} \AgdaRecord{⇉} \AgdaBound{Γ}\<%
\\
\>\AgdaFunction{fst\&} \AgdaBound{A} \AgdaSymbol{=} \AgdaFunction{fst} \AgdaBound{A} \AgdaFunction{id'}\<%
\\
%
\\
\>\AgdaFunction{\_+T\_} \AgdaSymbol{:} \AgdaSymbol{\{}\AgdaBound{Γ} \AgdaSymbol{:} \AgdaFunction{Con}\AgdaSymbol{\}} \AgdaSymbol{→} \AgdaRecord{Ty} \AgdaBound{Γ} \AgdaSymbol{→} \AgdaSymbol{(}\AgdaBound{A} \AgdaSymbol{:} \AgdaRecord{Ty} \AgdaBound{Γ}\AgdaSymbol{)} \AgdaSymbol{→} \AgdaRecord{Ty} \AgdaSymbol{(}\AgdaBound{Γ} \AgdaFunction{\&} \AgdaBound{A}\AgdaSymbol{)}\<%
\\
\>\AgdaBound{B} \AgdaFunction{+T} \AgdaBound{A} \AgdaSymbol{=} \AgdaBound{B} \AgdaFunction{[} \AgdaFunction{fst\&} \AgdaBound{A} \AgdaFunction{]T}\<%
\\
%
\\
\>\AgdaFunction{snd} \AgdaSymbol{:} \AgdaSymbol{\{}\AgdaBound{Γ} \AgdaBound{Δ} \AgdaSymbol{:} \AgdaFunction{Con}\AgdaSymbol{\}(}\AgdaBound{A} \AgdaSymbol{:} \AgdaRecord{Ty} \AgdaBound{Δ}\AgdaSymbol{)} \AgdaSymbol{→} \<[30]%
\>[30]\<%
\\
\>[0]\AgdaIndent{6}{}\<[6]%
\>[6]\AgdaSymbol{(}\AgdaBound{f} \AgdaSymbol{:} \AgdaBound{Γ} \AgdaRecord{⇉} \AgdaSymbol{(}\AgdaBound{Δ} \AgdaFunction{\&} \AgdaBound{A}\AgdaSymbol{))} \<[24]%
\>[24]\<%
\\
\>[0]\AgdaIndent{6}{}\<[6]%
\>[6]\AgdaSymbol{→} \AgdaRecord{Tm} \AgdaSymbol{(}\AgdaBound{A} \AgdaFunction{[} \AgdaFunction{fst} \AgdaBound{A} \AgdaBound{f} \AgdaFunction{]T}\AgdaSymbol{)}\<%
\\
\>\AgdaFunction{snd} \AgdaBound{A} \AgdaBound{f} \AgdaSymbol{=} \AgdaKeyword{record} \<[17]%
\>[17]\<%
\\
\>[6]\AgdaIndent{8}{}\<[8]%
\>[8]\AgdaSymbol{\{} \AgdaField{tm} \AgdaSymbol{=} \AgdaFunction{proj₂} \AgdaFunction{∘} \AgdaFunction{[} \AgdaBound{f} \AgdaFunction{]fn}\<%
\\
\>[6]\AgdaIndent{8}{}\<[8]%
\>[8]\AgdaSymbol{;} \AgdaField{respt} \AgdaSymbol{=} \AgdaFunction{proj₂} \AgdaFunction{∘} \AgdaFunction{[} \AgdaBound{f} \AgdaFunction{]resp} \<[36]%
\>[36]\<%
\\
\>[6]\AgdaIndent{8}{}\<[8]%
\>[8]\AgdaSymbol{\}}\<%
\\
%
\\
\>\AgdaFunction{v0} \AgdaSymbol{:} \AgdaSymbol{\{}\AgdaBound{Γ} \AgdaSymbol{:} \AgdaFunction{Con}\AgdaSymbol{\}(}\AgdaBound{A} \AgdaSymbol{:} \AgdaRecord{Ty} \AgdaBound{Γ}\AgdaSymbol{)} \AgdaSymbol{→} \AgdaRecord{Tm} \AgdaSymbol{(}\AgdaBound{A} \AgdaFunction{+T} \AgdaBound{A}\AgdaSymbol{)}\<%
\\
\>\AgdaFunction{v0} \AgdaBound{A} \AgdaSymbol{=} \AgdaFunction{snd} \AgdaBound{A} \AgdaFunction{id'}\<%
\\
%
\\
\>\<\end{code}

Inversely we could define a pairing operation to combine a context morphism with a term. The $\eta$-law for the projection and pairing holds trivially.

\begin{code}\>\<%
\\
%
\\
\>\AgdaFunction{\_,,\_} \AgdaSymbol{:} \AgdaSymbol{\{}\AgdaBound{Γ} \AgdaBound{Δ} \AgdaSymbol{:} \AgdaFunction{Con}\AgdaSymbol{\}\{}\AgdaBound{A} \AgdaSymbol{:} \AgdaRecord{Ty} \AgdaBound{Δ}\AgdaSymbol{\}(}\AgdaBound{f} \AgdaSymbol{:} \AgdaBound{Γ} \AgdaRecord{⇉} \AgdaBound{Δ}\AgdaSymbol{)} \AgdaSymbol{→} \<[42]%
\>[42]\<%
\\
\>[-5]\AgdaIndent{7}{}\<[7]%
\>[7]\AgdaSymbol{(}\AgdaRecord{Tm} \AgdaSymbol{(}\AgdaBound{A} \AgdaFunction{[} \AgdaBound{f} \AgdaFunction{]T}\AgdaSymbol{))} \<[23]%
\>[23]\<%
\\
\>[0]\AgdaIndent{7}{}\<[7]%
\>[7]\AgdaSymbol{→} \AgdaBound{Γ} \AgdaRecord{⇉} \AgdaSymbol{(}\AgdaBound{Δ} \AgdaFunction{\&} \AgdaBound{A}\AgdaSymbol{)}\<%
\\
\>\AgdaBound{f} \AgdaFunction{,,} \AgdaBound{t} \AgdaSymbol{=} \AgdaKeyword{record} \<[16]%
\>[16]\<%
\\
\>[7]\AgdaIndent{9}{}\<[9]%
\>[9]\AgdaSymbol{\{} \AgdaField{fn} \AgdaSymbol{=} \AgdaFunction{⟨} \AgdaFunction{[} \AgdaBound{f} \AgdaFunction{]fn} \AgdaFunction{,} \AgdaFunction{[} \AgdaBound{t} \AgdaFunction{]tm} \AgdaFunction{⟩}\<%
\\
\>[7]\AgdaIndent{9}{}\<[9]%
\>[9]\AgdaSymbol{;} \AgdaField{resp} \AgdaSymbol{=} \AgdaFunction{⟨} \AgdaFunction{[} \AgdaBound{f} \AgdaFunction{]resp} \AgdaFunction{,} \AgdaFunction{[} \AgdaBound{t} \AgdaFunction{]respt} \AgdaFunction{⟩}\<%
\\
\>[7]\AgdaIndent{9}{}\<[9]%
\>[9]\AgdaSymbol{\}}\<%
\\
%
\\
\>\AgdaFunction{\&-eta} \AgdaSymbol{:} \AgdaSymbol{\{}\AgdaBound{Γ} \AgdaBound{Δ} \AgdaSymbol{:} \AgdaFunction{Con}\AgdaSymbol{\}\{}\AgdaBound{A} \AgdaSymbol{:} \AgdaRecord{Ty} \AgdaBound{Δ}\AgdaSymbol{\}(}\AgdaBound{f} \AgdaSymbol{:} \AgdaBound{Γ} \AgdaRecord{⇉} \AgdaSymbol{(}\AgdaBound{Δ} \AgdaFunction{\&} \AgdaBound{A}\AgdaSymbol{))} \<[47]%
\>[47]\<%
\\
\>[-4]\AgdaIndent{6}{}\<[6]%
\>[6]\AgdaSymbol{→} \AgdaFunction{\_,,\_} \AgdaSymbol{\{}A \AgdaSymbol{=} \AgdaBound{A}\AgdaSymbol{\}} \AgdaSymbol{(}\AgdaFunction{fst} \AgdaBound{A} \AgdaBound{f}\AgdaSymbol{)} \AgdaSymbol{(}\AgdaFunction{snd} \AgdaBound{A} \AgdaBound{f}\AgdaSymbol{)} \AgdaDatatype{≡} \AgdaBound{f}\<%
\\
\>\AgdaFunction{\&-eta} \AgdaBound{f} \AgdaSymbol{=} \AgdaInductiveConstructor{PE.refl}\<%
\\
%
\\
%
\\
\>\<\end{code}

Then a lifting operation could help us define $\Pi$-types.

\begin{code}\>\<%
\\
%
\\
\>\AgdaFunction{lift} \AgdaSymbol{:} \AgdaSymbol{\{}\AgdaBound{Γ} \AgdaBound{Δ} \AgdaSymbol{:} \AgdaFunction{Con}\AgdaSymbol{\}(}\AgdaBound{f} \AgdaSymbol{:} \AgdaBound{Γ} \AgdaRecord{⇉} \AgdaBound{Δ}\AgdaSymbol{)(}\AgdaBound{A} \AgdaSymbol{:} \AgdaRecord{Ty} \AgdaBound{Δ}\AgdaSymbol{)} \AgdaSymbol{→} \AgdaBound{Γ} \AgdaFunction{\&} \AgdaBound{A} \AgdaFunction{[} \AgdaBound{f} \AgdaFunction{]T} \AgdaRecord{⇉} \AgdaBound{Δ} \AgdaFunction{\&} \AgdaBound{A}\<%
\\
\>\AgdaFunction{lift} \AgdaBound{f} \AgdaBound{A} \AgdaSymbol{=} \AgdaKeyword{record} \<[18]%
\>[18]\<%
\\
\>[0]\AgdaIndent{8}{}\<[8]%
\>[8]\AgdaSymbol{\{} \AgdaField{fn} \AgdaSymbol{=} \AgdaFunction{⟨} \AgdaFunction{[} \AgdaBound{f} \AgdaFunction{]fn} \AgdaFunction{∘} \AgdaFunction{proj₁} \AgdaFunction{,} \AgdaFunction{proj₂} \AgdaFunction{⟩}\<%
\\
\>[0]\AgdaIndent{8}{}\<[8]%
\>[8]\AgdaSymbol{;} \AgdaField{resp} \AgdaSymbol{=} \AgdaFunction{⟨} \AgdaFunction{[} \AgdaBound{f} \AgdaFunction{]resp} \AgdaFunction{∘} \AgdaFunction{proj₁} \AgdaFunction{,} \AgdaFunction{proj₂} \AgdaFunction{⟩}\<%
\\
\>[0]\AgdaIndent{8}{}\<[8]%
\>[8]\AgdaSymbol{\}}\<%
\\
%
\\
\>\AgdaFunction{lift-eta} \AgdaSymbol{:} \AgdaSymbol{\{}\AgdaBound{Γ} \AgdaBound{Δ} \AgdaSymbol{:} \AgdaFunction{Con}\AgdaSymbol{\}}\<%
\\
\>[8]\AgdaIndent{9}{}\<[9]%
\>[9]\AgdaSymbol{(}\AgdaBound{f} \AgdaSymbol{:} \AgdaBound{Γ} \AgdaRecord{⇉} \AgdaBound{Δ}\AgdaSymbol{)(}\AgdaBound{A} \AgdaSymbol{:} \AgdaRecord{Ty} \AgdaBound{Δ}\AgdaSymbol{)(}\AgdaBound{x} \AgdaSymbol{:} \AgdaFunction{∣} \AgdaBound{Γ} \AgdaFunction{∣}\AgdaSymbol{)}\<%
\\
\>[8]\AgdaIndent{9}{}\<[9]%
\>[9]\AgdaSymbol{(}\AgdaBound{a} \AgdaSymbol{:} \AgdaFunction{∣} \AgdaFunction{[} \AgdaBound{A} \AgdaFunction{]fm} \AgdaSymbol{(}\AgdaFunction{[} \AgdaBound{f} \AgdaFunction{]fn} \AgdaBound{x}\AgdaSymbol{)} \AgdaFunction{∣}\AgdaSymbol{)} \<[39]%
\>[39]\<%
\\
\>[8]\AgdaIndent{9}{}\<[9]%
\>[9]\AgdaSymbol{→} \AgdaFunction{[} \AgdaFunction{lift} \AgdaBound{f} \AgdaBound{A} \AgdaFunction{]fn} \AgdaSymbol{(}\AgdaBound{x} \AgdaInductiveConstructor{,} \AgdaBound{a}\AgdaSymbol{)} \AgdaDatatype{≡} \AgdaSymbol{(} \AgdaFunction{[} \AgdaBound{f} \AgdaFunction{]fn} \AgdaBound{x} \AgdaInductiveConstructor{,} \AgdaBound{a}\AgdaSymbol{)}\<%
\\
\>\AgdaFunction{lift-eta} \AgdaBound{f} \AgdaBound{A} \AgdaBound{x} \AgdaBound{a} \AgdaSymbol{=} \AgdaInductiveConstructor{PE.refl}\<%
\\
%
\\
%
\\
\>\<\end{code}

One of the most complicated part of this definition is the $\Pi$-types.
$\Pi$-types is also called dependent function types. Semantically it is a function type on the underlying semantic types with a proof that the the functions respect the equivalence relation. 

%f-resp on the paper ignores refl*

\begin{code}\>\<%
\\
%
\\
\>\AgdaFunction{Π} \AgdaSymbol{:} \AgdaSymbol{\{}\AgdaBound{Γ} \AgdaSymbol{:} \AgdaFunction{Con}\AgdaSymbol{\}(}\AgdaBound{A} \AgdaSymbol{:} \AgdaRecord{Ty} \AgdaBound{Γ}\AgdaSymbol{)(}\AgdaBound{B} \AgdaSymbol{:} \AgdaRecord{Ty} \AgdaSymbol{(}\AgdaBound{Γ} \AgdaFunction{\&} \AgdaBound{A}\AgdaSymbol{))} \AgdaSymbol{→} \AgdaRecord{Ty} \AgdaBound{Γ}\<%
\\
%
\\
\>\<\end{code}

\AgdaHide{
\begin{code}\>\<%
\\
%
\\
\>\AgdaFunction{Π} \AgdaSymbol{\{}\AgdaBound{Γ}\AgdaSymbol{\}} \AgdaBound{A} \AgdaBound{B} \AgdaSymbol{=} \AgdaKeyword{record} \<[19]%
\>[19]\<%
\\
\>[-1]\AgdaIndent{2}{}\<[2]%
\>[2]\AgdaSymbol{\{} \AgdaField{fm} \AgdaSymbol{=} \AgdaSymbol{λ} \AgdaBound{x} \AgdaSymbol{→} \AgdaKeyword{let} \AgdaBound{Ax} \AgdaSymbol{=} \AgdaFunction{[} \AgdaBound{A} \AgdaFunction{]fm} \AgdaBound{x} \AgdaKeyword{in}\<%
\\
\>[0]\AgdaIndent{15}{}\<[15]%
\>[15]\AgdaKeyword{let} \AgdaBound{Bx} \AgdaSymbol{=} \AgdaSymbol{λ} \AgdaBound{a} \AgdaSymbol{→} \AgdaFunction{[} \AgdaBound{B} \AgdaFunction{]fm} \AgdaSymbol{(}\AgdaBound{x} \AgdaInductiveConstructor{,} \AgdaBound{a}\AgdaSymbol{)} \AgdaKeyword{in}\<%
\\
\>[0]\AgdaIndent{9}{}\<[9]%
\>[9]\AgdaKeyword{record}\<%
\\
\>[0]\AgdaIndent{9}{}\<[9]%
\>[9]\AgdaSymbol{\{} \AgdaField{Carrier} \AgdaSymbol{=} \AgdaRecord{Σ[} \AgdaBound{fn} \AgdaRecord{∶} \AgdaSymbol{((}\AgdaBound{a} \AgdaSymbol{:} \AgdaFunction{∣} \AgdaBound{Ax} \AgdaFunction{∣}\AgdaSymbol{)} \AgdaSymbol{→} \AgdaFunction{∣} \AgdaBound{Bx} \AgdaBound{a} \AgdaFunction{∣}\AgdaSymbol{)} \AgdaRecord{]}\<%
\\
\>[9]\AgdaIndent{21}{}\<[21]%
\>[21]\AgdaSymbol{((}\AgdaBound{a} \AgdaBound{b} \AgdaSymbol{:} \AgdaFunction{∣} \AgdaBound{Ax} \AgdaFunction{∣}\AgdaSymbol{)}\<%
\\
\>[9]\AgdaIndent{21}{}\<[21]%
\>[21]\AgdaSymbol{(}\AgdaBound{p} \AgdaSymbol{:} \AgdaFunction{[} \AgdaBound{Ax} \AgdaFunction{]} \AgdaBound{a} \AgdaFunction{≈} \AgdaBound{b}\AgdaSymbol{)} \AgdaSymbol{→}\<%
\\
\>[9]\AgdaIndent{21}{}\<[21]%
\>[21]\AgdaFunction{[} \AgdaBound{Bx} \AgdaBound{b} \AgdaFunction{]} \AgdaFunction{[} \AgdaBound{B} \AgdaFunction{]subst} \AgdaSymbol{(}\AgdaFunction{[} \AgdaBound{Γ} \AgdaFunction{]refl} \AgdaInductiveConstructor{,}\<%
\\
\>[-7]\AgdaIndent{19}{}\<[19]%
\>[19]\AgdaFunction{[} \AgdaBound{Ax} \AgdaFunction{]trans} \AgdaSymbol{(}\AgdaFunction{[} \AgdaBound{A} \AgdaFunction{]refl*} \AgdaBound{x} \AgdaBound{a}\AgdaSymbol{)} \AgdaBound{p}\AgdaSymbol{)} \AgdaSymbol{(}\AgdaBound{fn} \AgdaBound{a}\AgdaSymbol{)} \AgdaFunction{≈} \AgdaBound{fn} \AgdaBound{b}\AgdaSymbol{)} \<[66]%
\>[66]\<%
\\
\>[0]\AgdaIndent{1}{}\<[1]%
\>[1]\<%
\\
\>[1]\AgdaIndent{9}{}\<[9]%
\>[9]\AgdaSymbol{;} \AgdaField{\_≈h\_} \<[19]%
\>[19]\AgdaSymbol{=} \AgdaSymbol{λ\{(}\AgdaBound{f} \AgdaInductiveConstructor{,} \AgdaSymbol{\_)} \AgdaSymbol{(}\AgdaBound{g} \AgdaInductiveConstructor{,} \AgdaSymbol{\_)} \AgdaSymbol{→} \<[41]%
\>[41]\<%
\\
\>[9]\AgdaIndent{24}{}\<[24]%
\>[24]\AgdaFunction{∀'[} \AgdaBound{a} \AgdaFunction{∶} \AgdaSymbol{\_} \AgdaFunction{]} \AgdaFunction{[} \AgdaBound{Bx} \AgdaBound{a} \AgdaFunction{]} \AgdaBound{f} \AgdaBound{a} \AgdaFunction{≈h} \AgdaBound{g} \AgdaBound{a} \AgdaSymbol{\}}\<%
\\
\>[0]\AgdaIndent{9}{}\<[9]%
\>[9]\AgdaSymbol{;} \AgdaField{isEquiv} \AgdaSymbol{=} \AgdaKeyword{record} \AgdaSymbol{\{}\<%
\\
\>[0]\AgdaIndent{19}{}\<[19]%
\>[19]\AgdaField{refl} \<[25]%
\>[25]\AgdaSymbol{=} \AgdaSymbol{λ} \AgdaBound{a} \AgdaSymbol{→} \AgdaFunction{[} \AgdaBound{Bx} \AgdaBound{a} \AgdaFunction{]refl} \<[46]%
\>[46]\<%
\\
\>[0]\AgdaIndent{17}{}\<[17]%
\>[17]\AgdaSymbol{;} \AgdaField{sym} \<[25]%
\>[25]\AgdaSymbol{=} \AgdaSymbol{λ} \AgdaBound{f} \AgdaBound{a} \AgdaSymbol{→} \AgdaFunction{[} \AgdaBound{Bx} \AgdaBound{a} \AgdaFunction{]sym} \AgdaSymbol{(}\AgdaBound{f} \AgdaBound{a}\AgdaSymbol{)}\<%
\\
\>[0]\AgdaIndent{17}{}\<[17]%
\>[17]\AgdaSymbol{;} \AgdaField{trans} \AgdaSymbol{=} \AgdaSymbol{λ} \AgdaBound{f} \AgdaBound{g} \AgdaBound{a} \AgdaSymbol{→} \AgdaFunction{[} \AgdaBound{Bx} \AgdaBound{a} \AgdaFunction{]trans} \AgdaSymbol{(}\AgdaBound{f} \AgdaBound{a}\AgdaSymbol{)} \AgdaSymbol{(}\AgdaBound{g} \AgdaBound{a}\AgdaSymbol{)}\<%
\\
\>[17]\AgdaIndent{29}{}\<[29]%
\>[29]\AgdaSymbol{\}}\<%
\\
\>[3]\AgdaIndent{9}{}\<[9]%
\>[9]\AgdaSymbol{\}}\<%
\\
%
\\
\>[0]\AgdaIndent{2}{}\<[2]%
\>[2]\AgdaSymbol{;} \AgdaField{substT} \AgdaSymbol{=} \AgdaSymbol{λ} \AgdaSymbol{\{}\AgdaBound{x}\AgdaSymbol{\}} \AgdaSymbol{\{}\AgdaBound{y}\AgdaSymbol{\}} \AgdaBound{p} \AgdaSymbol{→}\<%
\\
\>[2]\AgdaIndent{19}{}\<[19]%
\>[19]\AgdaKeyword{let} \AgdaBound{y2x} \AgdaSymbol{=} \AgdaSymbol{λ} \AgdaBound{a} \AgdaSymbol{→} \AgdaFunction{[} \AgdaBound{A} \AgdaFunction{]subst} \AgdaSymbol{(}\AgdaFunction{[} \AgdaBound{Γ} \AgdaFunction{]sym} \AgdaBound{p}\AgdaSymbol{)} \AgdaBound{a} \AgdaKeyword{in}\<%
\\
\>[19]\AgdaIndent{20}{}\<[20]%
\>[20]\AgdaKeyword{let} \AgdaBound{x2y} \AgdaSymbol{=} \AgdaSymbol{λ} \AgdaBound{a} \AgdaSymbol{→} \AgdaFunction{[} \AgdaBound{A} \AgdaFunction{]subst} \AgdaBound{p} \AgdaBound{a} \AgdaKeyword{in}\<%
\\
\>[0]\AgdaIndent{19}{}\<[19]%
\>[19]\AgdaKeyword{let} \AgdaBound{p'} \AgdaSymbol{=} \AgdaSymbol{λ} \AgdaBound{a} \AgdaSymbol{→} \AgdaFunction{[} \AgdaBound{A} \AgdaFunction{]trans-refl} \AgdaKeyword{in}\<%
\\
\>[0]\AgdaIndent{13}{}\<[13]%
\>[13]\AgdaSymbol{λ\{(}\AgdaBound{f} \AgdaInductiveConstructor{,} \AgdaBound{rsp}\AgdaSymbol{)} \AgdaSymbol{→} \<[28]%
\>[28]\<%
\\
\>[13]\AgdaIndent{15}{}\<[15]%
\>[15]\AgdaSymbol{(λ} \AgdaBound{a} \AgdaSymbol{→} \AgdaFunction{[} \AgdaBound{B} \AgdaFunction{]subst} \AgdaSymbol{(}\AgdaBound{p} \AgdaInductiveConstructor{,} \AgdaBound{p'} \AgdaBound{a}\AgdaSymbol{)} \AgdaSymbol{(}\AgdaBound{f} \AgdaSymbol{(}\AgdaBound{y2x} \AgdaBound{a}\AgdaSymbol{)))}\<%
\\
\>[13]\AgdaIndent{15}{}\<[15]%
\>[15]\AgdaInductiveConstructor{,} \<[49]%
\>[49]\<%
\\
\>[13]\AgdaIndent{15}{}\<[15]%
\>[15]\AgdaSymbol{(λ} \AgdaBound{a} \AgdaBound{b} \AgdaBound{q} \AgdaSymbol{→} \<[26]%
\>[26]\<%
\\
\>[15]\AgdaIndent{16}{}\<[16]%
\>[16]\AgdaKeyword{let} \AgdaBound{a'} \AgdaSymbol{=} \AgdaBound{y2x} \AgdaBound{a} \AgdaKeyword{in} \<[34]%
\>[34]\<%
\\
\>[15]\AgdaIndent{16}{}\<[16]%
\>[16]\AgdaKeyword{let} \AgdaBound{b'} \AgdaSymbol{=} \AgdaBound{y2x} \AgdaBound{b} \AgdaKeyword{in}\<%
\\
\>[15]\AgdaIndent{16}{}\<[16]%
\>[16]\AgdaKeyword{let} \AgdaBound{q'} \AgdaSymbol{=} \AgdaFunction{[} \AgdaBound{A} \AgdaFunction{]subst*} \AgdaSymbol{(}\AgdaFunction{[} \AgdaBound{Γ} \AgdaFunction{]sym} \AgdaBound{p}\AgdaSymbol{)} \AgdaBound{q} \AgdaKeyword{in}\<%
\\
\>[15]\AgdaIndent{16}{}\<[16]%
\>[16]\AgdaKeyword{let} \AgdaBound{H} \AgdaSymbol{=} \AgdaBound{rsp} \AgdaBound{a'} \AgdaBound{b'} \AgdaBound{q'} \AgdaKeyword{in}\<%
\\
\>[15]\AgdaIndent{16}{}\<[16]%
\>[16]\AgdaKeyword{let} \AgdaBound{r} \AgdaSymbol{:} \AgdaFunction{[} \AgdaBound{Γ} \AgdaFunction{\&} \AgdaBound{A} \AgdaFunction{]} \AgdaSymbol{(}\AgdaBound{x} \AgdaInductiveConstructor{,} \AgdaBound{b'}\AgdaSymbol{)} \AgdaFunction{≈} \AgdaSymbol{(}\AgdaBound{y} \AgdaInductiveConstructor{,} \AgdaBound{b}\AgdaSymbol{)}
                    r \AgdaSymbol{=} \AgdaSymbol{(}\AgdaBound{p} \AgdaInductiveConstructor{,} \AgdaBound{p'} \AgdaBound{b}\AgdaSymbol{)} \AgdaKeyword{in}\<%
\\
\>[15]\AgdaIndent{16}{}\<[16]%
\>[16]\AgdaKeyword{let} \AgdaBound{pre} \AgdaSymbol{=} \AgdaFunction{[} \AgdaBound{B} \AgdaFunction{]subst*} \AgdaBound{r} \AgdaBound{H} \AgdaKeyword{in}\<%
\\
\>[15]\AgdaIndent{16}{}\<[16]%
\>[16]\<%
\\
\>[15]\AgdaIndent{16}{}\<[16]%
\>[16]\AgdaFunction{[} \AgdaFunction{[} \AgdaBound{B} \AgdaFunction{]fm} \AgdaSymbol{(}\AgdaBound{y} \AgdaInductiveConstructor{,} \AgdaBound{b}\AgdaSymbol{)} \AgdaFunction{]trans} \<[41]%
\>[41]\<%
\\
\>[15]\AgdaIndent{16}{}\<[16]%
\>[16]\AgdaSymbol{(}\AgdaFunction{[} \AgdaBound{B} \AgdaFunction{]trans*} \AgdaSymbol{\_} \AgdaSymbol{\_} \AgdaSymbol{\_)} \<[52]%
\>[52]\<%
\\
\>[15]\AgdaIndent{16}{}\<[16]%
\>[16]\AgdaSymbol{(}\AgdaFunction{[} \AgdaFunction{[} \AgdaBound{B} \AgdaFunction{]fm} \AgdaSymbol{(}\AgdaBound{y} \AgdaInductiveConstructor{,} \AgdaBound{b}\AgdaSymbol{)} \AgdaFunction{]trans} \<[42]%
\>[42]\<%
\\
\>[15]\AgdaIndent{16}{}\<[16]%
\>[16]\AgdaFunction{[} \AgdaBound{B} \AgdaFunction{]subst-pi} \<[30]%
\>[30]\<%
\\
\>[15]\AgdaIndent{16}{}\<[16]%
\>[16]\AgdaSymbol{(}\AgdaFunction{[} \AgdaFunction{[} \AgdaBound{B} \AgdaFunction{]fm} \AgdaSymbol{(}\AgdaBound{y} \AgdaInductiveConstructor{,} \AgdaBound{b}\AgdaSymbol{)} \AgdaFunction{]trans} \<[42]%
\>[42]\<%
\\
\>[15]\AgdaIndent{16}{}\<[16]%
\>[16]\AgdaSymbol{(}\AgdaFunction{[} \AgdaFunction{[} \AgdaBound{B} \AgdaFunction{]fm} \AgdaSymbol{(}\AgdaBound{y} \AgdaInductiveConstructor{,} \AgdaBound{b}\AgdaSymbol{)} \AgdaFunction{]sym} \<[40]%
\>[40]\<%
\\
\>[15]\AgdaIndent{16}{}\<[16]%
\>[16]\AgdaSymbol{(}\AgdaFunction{[} \AgdaBound{B} \AgdaFunction{]trans*} \AgdaSymbol{\_} \AgdaSymbol{\_} \AgdaSymbol{\_))} \<[37]%
\>[37]\<%
\\
\>[15]\AgdaIndent{16}{}\<[16]%
\>[16]\AgdaBound{pre}\AgdaSymbol{))} \<[23]%
\>[23]\<%
\\
\>[15]\AgdaIndent{16}{}\<[16]%
\>[16]\AgdaSymbol{)} \<[22]%
\>[22]\<%
\\
\>[-12]\AgdaIndent{13}{}\<[13]%
\>[13]\AgdaSymbol{\}}\<%
\\
\>[0]\AgdaIndent{2}{}\<[2]%
\>[2]\AgdaSymbol{;} \AgdaField{subst*} \AgdaSymbol{=} \AgdaSymbol{λ} \AgdaBound{\_} \AgdaBound{q} \AgdaBound{\_} \AgdaSymbol{→} \AgdaFunction{[} \AgdaBound{B} \AgdaFunction{]subst*} \AgdaSymbol{\_} \AgdaSymbol{(}\AgdaBound{q} \AgdaSymbol{\_)}\<%
\\
\>[0]\AgdaIndent{2}{}\<[2]%
\>[2]\AgdaSymbol{;} \AgdaField{refl*} \AgdaSymbol{=} \AgdaSymbol{λ} \AgdaSymbol{\{}\AgdaBound{x} \AgdaSymbol{(}\AgdaBound{f} \AgdaInductiveConstructor{,} \AgdaBound{rsp}\AgdaSymbol{)} \AgdaBound{a} \AgdaSymbol{→} \<[32]%
\>[32]\AgdaFunction{[} \AgdaFunction{[} \AgdaBound{B} \AgdaFunction{]fm} \AgdaSymbol{\_} \AgdaFunction{]trans} \<[51]%
\>[51]\<%
\\
\>[2]\AgdaIndent{17}{}\<[17]%
\>[17]\AgdaFunction{[} \AgdaBound{B} \AgdaFunction{]subst-pi} \AgdaSymbol{(}\AgdaBound{rsp} \AgdaSymbol{(}\AgdaFunction{[} \AgdaBound{A} \AgdaFunction{]subst} \<[48]%
\>[48]\<%
\\
\>[17]\AgdaIndent{21}{}\<[21]%
\>[21]\AgdaSymbol{(}\AgdaFunction{[} \AgdaBound{Γ} \AgdaFunction{]sym} \AgdaFunction{[} \AgdaBound{Γ} \AgdaFunction{]refl}\AgdaSymbol{)} \AgdaBound{a}\AgdaSymbol{)} \AgdaBound{a} \AgdaFunction{[} \AgdaBound{A} \AgdaFunction{]subst-pi'}\AgdaSymbol{)} \<[64]%
\>[64]\AgdaSymbol{\}}\<%
\\
\>[2]\AgdaIndent{2}{}\<[2]%
\>[2]\AgdaSymbol{;} \AgdaField{trans*} \AgdaSymbol{=} \AgdaSymbol{λ} \AgdaBound{p} \AgdaBound{q} \AgdaSymbol{→} \AgdaSymbol{λ} \AgdaSymbol{\{(}\AgdaBound{f} \AgdaInductiveConstructor{,} \AgdaBound{rsp}\AgdaSymbol{)} \AgdaBound{a} \AgdaSymbol{→}\<%
\\
\>[0]\AgdaIndent{13}{}\<[13]%
\>[13]\AgdaFunction{[} \AgdaFunction{[} \AgdaBound{B} \AgdaFunction{]fm} \AgdaSymbol{\_} \AgdaFunction{]trans} \<[32]%
\>[32]\<%
\\
\>[0]\AgdaIndent{13}{}\<[13]%
\>[13]\AgdaSymbol{(}\AgdaFunction{[} \AgdaFunction{[} \AgdaBound{B} \AgdaFunction{]fm} \AgdaSymbol{\_} \AgdaFunction{]trans} \<[33]%
\>[33]\<%
\\
\>[0]\AgdaIndent{13}{}\<[13]%
\>[13]\AgdaSymbol{(}\AgdaFunction{[} \AgdaBound{B} \AgdaFunction{]trans*} \AgdaSymbol{\_} \AgdaSymbol{\_} \AgdaSymbol{\_)} \<[33]%
\>[33]\<%
\\
\>[0]\AgdaIndent{13}{}\<[13]%
\>[13]\AgdaSymbol{(}\AgdaFunction{[} \AgdaFunction{[} \AgdaBound{B} \AgdaFunction{]fm} \AgdaSymbol{\_} \AgdaFunction{]sym} \<[31]%
\>[31]\<%
\\
\>[0]\AgdaIndent{13}{}\<[13]%
\>[13]\AgdaSymbol{(}\AgdaFunction{[} \AgdaFunction{[} \AgdaBound{B} \AgdaFunction{]fm} \AgdaSymbol{\_} \AgdaFunction{]trans} \<[33]%
\>[33]\<%
\\
\>[0]\AgdaIndent{13}{}\<[13]%
\>[13]\AgdaSymbol{(}\AgdaFunction{[} \AgdaBound{B} \AgdaFunction{]trans*} \AgdaSymbol{\_} \AgdaSymbol{\_} \AgdaSymbol{\_)} \AgdaFunction{[} \AgdaBound{B} \AgdaFunction{]subst-pi}\AgdaSymbol{)))} \<[50]%
\>[50]\<%
\\
\>[0]\AgdaIndent{13}{}\<[13]%
\>[13]\AgdaSymbol{(}\AgdaFunction{[} \AgdaBound{B} \AgdaFunction{]subst*} \AgdaSymbol{\_} \AgdaSymbol{(}\AgdaBound{rsp} \AgdaSymbol{\_} \AgdaSymbol{\_} \<[37]%
\>[37]\<%
\\
\>[0]\AgdaIndent{13}{}\<[13]%
\>[13]\AgdaSymbol{(}\AgdaFunction{[} \AgdaFunction{[} \AgdaBound{A} \AgdaFunction{]fm} \AgdaSymbol{\_} \AgdaFunction{]trans} \<[33]%
\>[33]\<%
\\
\>[0]\AgdaIndent{13}{}\<[13]%
\>[13]\AgdaSymbol{(}\AgdaFunction{[} \AgdaBound{A} \AgdaFunction{]trans*} \AgdaSymbol{\_} \AgdaSymbol{\_} \AgdaSymbol{\_)} \AgdaFunction{[} \AgdaBound{A} \AgdaFunction{]subst-pi}\AgdaSymbol{)))} \AgdaSymbol{\}} \<[52]%
\>[52]\<%
\\
\>[0]\AgdaIndent{2}{}\<[2]%
\>[2]\AgdaSymbol{\}}\<%
\\
%
\\
\>\<\end{code}
}

It also comes with two necessary operation on the terms of $Pi$-types, $\lambda$-abstraction and application.
There are $\beta-\eta$ laws to verfify for them so that we could form an isomorphism with these two operations. however technically it causes stack overflow. We may simplify these definition in the future so that we could verify them in Agda.

%to do : verification of β and η
%cause stack overflow

\begin{code}\>\<%
\\
%
\\
\>\AgdaFunction{lam} \AgdaSymbol{:} \AgdaSymbol{\{}\AgdaBound{Γ} \AgdaSymbol{:} \AgdaFunction{Con}\AgdaSymbol{\}\{}\AgdaBound{A} \AgdaSymbol{:} \AgdaRecord{Ty} \AgdaBound{Γ}\AgdaSymbol{\}\{}\AgdaBound{B} \AgdaSymbol{:} \AgdaRecord{Ty} \AgdaSymbol{(}\AgdaBound{Γ} \AgdaFunction{\&} \AgdaBound{A}\AgdaSymbol{)\}} \AgdaSymbol{→} \AgdaRecord{Tm} \AgdaBound{B} \AgdaSymbol{→} \AgdaRecord{Tm} \AgdaSymbol{(}\AgdaFunction{Π} \AgdaBound{A} \AgdaBound{B}\AgdaSymbol{)}\<%
\\
\>\<\end{code}

\AgdaHide{
\begin{code}\>\<%
\\
\>\AgdaFunction{lam} \AgdaSymbol{\{}\AgdaBound{Γ}\AgdaSymbol{\}} \AgdaSymbol{\{}\AgdaBound{A}\AgdaSymbol{\}} \AgdaSymbol{(}\AgdaInductiveConstructor{tm:} \AgdaBound{tm} \AgdaInductiveConstructor{resp:} \AgdaBound{respt}\AgdaSymbol{)} \AgdaSymbol{=} \<[35]%
\>[35]\<%
\\
\>[0]\AgdaIndent{2}{}\<[2]%
\>[2]\AgdaKeyword{record} \AgdaSymbol{\{} \AgdaField{tm} \AgdaSymbol{=} \AgdaSymbol{λ} \AgdaBound{x} \AgdaSymbol{→} \AgdaSymbol{(λ} \AgdaBound{a} \AgdaSymbol{→} \AgdaBound{tm} \AgdaSymbol{(}\AgdaBound{x} \AgdaInductiveConstructor{,} \AgdaBound{a}\AgdaSymbol{))} \AgdaInductiveConstructor{,} \<[43]%
\>[43]\<%
\\
\>[2]\AgdaIndent{11}{}\<[11]%
\>[11]\AgdaSymbol{(λ} \AgdaBound{a} \AgdaBound{b} \AgdaBound{p} \AgdaSymbol{→} \AgdaBound{respt} \AgdaSymbol{(}\AgdaFunction{[} \AgdaBound{Γ} \AgdaFunction{]refl} \AgdaInductiveConstructor{,}\<%
\\
\>[11]\AgdaIndent{13}{}\<[13]%
\>[13]\AgdaFunction{[} \AgdaFunction{[} \AgdaBound{A} \AgdaFunction{]fm} \AgdaBound{x} \AgdaFunction{]trans} \AgdaSymbol{(}\AgdaFunction{[} \AgdaBound{A} \AgdaFunction{]refl*} \AgdaSymbol{\_} \AgdaSymbol{\_)} \AgdaBound{p}\AgdaSymbol{))}\<%
\\
\>[-7]\AgdaIndent{9}{}\<[9]%
\>[9]\AgdaSymbol{;} \AgdaField{respt} \AgdaSymbol{=} \AgdaSymbol{λ} \AgdaBound{p} \AgdaBound{\_} \AgdaSymbol{→} \AgdaBound{respt} \AgdaSymbol{(}\AgdaBound{p} \AgdaInductiveConstructor{,} \AgdaFunction{[} \AgdaBound{A} \AgdaFunction{]trans-refl}\AgdaSymbol{)} \<[55]%
\>[55]\<%
\\
\>[0]\AgdaIndent{9}{}\<[9]%
\>[9]\AgdaSymbol{\}}\<%
\\
%
\\
\>\<\end{code}
}


\begin{code}\>\<%
\\
\>\AgdaFunction{app} \AgdaSymbol{:} \AgdaSymbol{\{}\AgdaBound{Γ} \AgdaSymbol{:} \AgdaFunction{Con}\AgdaSymbol{\}\{}\AgdaBound{A} \AgdaSymbol{:} \AgdaRecord{Ty} \AgdaBound{Γ}\AgdaSymbol{\}\{}\AgdaBound{B} \AgdaSymbol{:} \AgdaRecord{Ty} \AgdaSymbol{(}\AgdaBound{Γ} \AgdaFunction{\&} \AgdaBound{A}\AgdaSymbol{)\}} \AgdaSymbol{→} \AgdaRecord{Tm} \AgdaSymbol{(}\AgdaFunction{Π} \AgdaBound{A} \AgdaBound{B}\AgdaSymbol{)} \AgdaSymbol{→} \AgdaRecord{Tm} \AgdaBound{B}\<%
\\
%
\\
\>\<\end{code}

\AgdaHide{
\begin{code}\>\<%
\\
\>\AgdaFunction{app} \AgdaSymbol{\{}\AgdaBound{Γ}\AgdaSymbol{\}} \AgdaSymbol{\{}\AgdaBound{A}\AgdaSymbol{\}} \AgdaSymbol{\{}\AgdaBound{B}\AgdaSymbol{\}} \AgdaSymbol{(}\AgdaInductiveConstructor{tm:} \AgdaBound{tm} \AgdaInductiveConstructor{resp:} \AgdaBound{respt}\AgdaSymbol{)} \AgdaSymbol{=} \<[39]%
\>[39]\<%
\\
\>[0]\AgdaIndent{2}{}\<[2]%
\>[2]\AgdaKeyword{record} \AgdaSymbol{\{} \AgdaField{tm} \AgdaSymbol{=} \AgdaSymbol{λ} \AgdaSymbol{\{(}\AgdaBound{x} \AgdaInductiveConstructor{,} \AgdaBound{a}\AgdaSymbol{)} \AgdaSymbol{→} \AgdaFunction{proj₁} \AgdaSymbol{(}\AgdaBound{tm} \AgdaBound{x}\AgdaSymbol{)} \AgdaBound{a}\AgdaSymbol{\}}\<%
\\
\>[0]\AgdaIndent{9}{}\<[9]%
\>[9]\AgdaSymbol{;} \AgdaField{respt} \AgdaSymbol{=} \AgdaSymbol{λ} \AgdaSymbol{\{}\AgdaBound{x}\AgdaSymbol{\}} \AgdaSymbol{\{}\AgdaBound{y}\AgdaSymbol{\}} \AgdaSymbol{→} \AgdaSymbol{λ} \AgdaSymbol{\{(}\AgdaBound{p} \AgdaInductiveConstructor{,} \AgdaBound{tr}\AgdaSymbol{)} \AgdaSymbol{→} \<[45]%
\>[45]\<%
\\
\>[9]\AgdaIndent{13}{}\<[13]%
\>[13]\AgdaKeyword{let} \AgdaBound{fresp} \AgdaSymbol{=} \AgdaFunction{proj₂} \AgdaSymbol{(}\AgdaBound{tm} \AgdaSymbol{(}\AgdaFunction{proj₁} \AgdaBound{x}\AgdaSymbol{))} \AgdaKeyword{in}\<%
\\
\>[13]\AgdaIndent{16}{}\<[16]%
\>[16]\AgdaFunction{[} \AgdaFunction{[} \AgdaBound{B} \AgdaFunction{]fm} \AgdaSymbol{\_} \AgdaFunction{]trans} \<[35]%
\>[35]\<%
\\
\>[13]\AgdaIndent{16}{}\<[16]%
\>[16]\AgdaSymbol{(}\AgdaFunction{[} \AgdaBound{B} \AgdaFunction{]subst*} \AgdaSymbol{(}\AgdaBound{p} \AgdaInductiveConstructor{,} \AgdaBound{tr}\AgdaSymbol{)} \AgdaSymbol{(}\AgdaFunction{[} \AgdaFunction{[} \AgdaBound{B} \AgdaFunction{]fm} \AgdaSymbol{\_} \AgdaFunction{]sym} \AgdaFunction{[} \AgdaBound{B} \AgdaFunction{]subst-pi'}\AgdaSymbol{))} \<[73]%
\>[73]\<%
\\
\>[13]\AgdaIndent{16}{}\<[16]%
\>[16]\AgdaSymbol{(}\AgdaFunction{[} \AgdaFunction{[} \AgdaBound{B} \AgdaFunction{]fm} \AgdaSymbol{\_} \AgdaFunction{]trans}\<%
\\
\>[13]\AgdaIndent{16}{}\<[16]%
\>[16]\AgdaSymbol{(}\AgdaFunction{[} \AgdaBound{B} \AgdaFunction{]trans*} \AgdaSymbol{(}\AgdaFunction{[} \AgdaBound{Γ} \AgdaFunction{]refl} \AgdaInductiveConstructor{,} \AgdaFunction{[} \AgdaBound{A} \AgdaFunction{]refl*} \AgdaSymbol{\_} \AgdaSymbol{\_)} \AgdaSymbol{\_} \AgdaSymbol{\_)} \<[63]%
\>[63]\<%
\\
\>[13]\AgdaIndent{16}{}\<[16]%
\>[16]\AgdaSymbol{(}\AgdaFunction{[} \AgdaFunction{[} \AgdaBound{B} \AgdaFunction{]fm} \AgdaSymbol{\_} \AgdaFunction{]trans} \<[36]%
\>[36]\<%
\\
\>[13]\AgdaIndent{16}{}\<[16]%
\>[16]\AgdaFunction{[} \AgdaBound{B} \AgdaFunction{]subst-pi} \<[30]%
\>[30]\<%
\\
\>[13]\AgdaIndent{16}{}\<[16]%
\>[16]\AgdaSymbol{(}\AgdaFunction{[} \AgdaFunction{[} \AgdaBound{B} \AgdaFunction{]fm} \AgdaSymbol{\_} \AgdaFunction{]trans} \<[36]%
\>[36]\<%
\\
\>[13]\AgdaIndent{16}{}\<[16]%
\>[16]\AgdaSymbol{(}\AgdaFunction{[} \AgdaFunction{[} \AgdaBound{B} \AgdaFunction{]fm} \AgdaSymbol{\_} \AgdaFunction{]sym} \AgdaSymbol{(}\AgdaFunction{[} \AgdaBound{B} \AgdaFunction{]trans*} \AgdaSymbol{\_} \AgdaSymbol{(}\AgdaBound{p} \AgdaInductiveConstructor{,} \AgdaFunction{[} \AgdaBound{A} \AgdaFunction{]trans-refl}\AgdaSymbol{)} \AgdaSymbol{\_))}\<%
\\
\>[13]\AgdaIndent{16}{}\<[16]%
\>[16]\AgdaSymbol{(}\AgdaFunction{[} \AgdaFunction{[} \AgdaBound{B} \AgdaFunction{]fm} \AgdaSymbol{\_} \AgdaFunction{]trans} \<[36]%
\>[36]\<%
\\
\>[13]\AgdaIndent{16}{}\<[16]%
\>[16]\AgdaSymbol{(}\AgdaFunction{[} \AgdaBound{B} \AgdaFunction{]subst-pi*} \AgdaSymbol{(}\AgdaBound{fresp} \AgdaSymbol{\_} \AgdaSymbol{\_} \AgdaSymbol{(}\AgdaFunction{[} \AgdaBound{A} \AgdaFunction{]subst-mir2} \AgdaBound{tr}\AgdaSymbol{)))} \<[66]%
\>[66]\<%
\\
\>[13]\AgdaIndent{16}{}\<[16]%
\>[16]\AgdaSymbol{(}\AgdaBound{respt} \AgdaBound{p} \AgdaSymbol{\_)))))} \AgdaSymbol{\}}\<%
\\
\>[-6]\AgdaIndent{9}{}\<[9]%
\>[9]\AgdaSymbol{\}}\<%
\\
%
\\
%
\\
\>\<\end{code}
}

Non-dependent version of $\Pi$-types namely function types can be defined with type weakening. Since the dependence disappears, it is possible to define it straightforwardly without using $\Pi$-types.

\begin{code}\>\<%
\\
%
\\
\>\AgdaFunction{\_⇒'\_} \AgdaSymbol{:} \AgdaSymbol{\{}\AgdaBound{Γ} \AgdaSymbol{:} \AgdaFunction{Con}\AgdaSymbol{\}(}\AgdaBound{A} \AgdaBound{B} \AgdaSymbol{:} \AgdaRecord{Ty} \AgdaBound{Γ}\AgdaSymbol{)} \AgdaSymbol{→} \AgdaRecord{Ty} \AgdaBound{Γ}\<%
\\
\>\AgdaBound{A} \AgdaFunction{⇒'} \AgdaBound{B} \AgdaSymbol{=} \AgdaFunction{Π} \AgdaBound{A} \AgdaSymbol{(}\AgdaBound{B} \AgdaFunction{+T} \AgdaBound{A}\AgdaSymbol{)}\<%
\\
%
\\
\>\<\end{code}


\AgdaHide{
\begin{code}\>\<%
\\
%
\\
\>\AgdaFunction{[\_,\_]\_⇒fm\_} \AgdaSymbol{:} \AgdaSymbol{(}\AgdaBound{Γ} \AgdaSymbol{:} \AgdaFunction{Con}\AgdaSymbol{)(}\AgdaBound{x} \AgdaSymbol{:} \AgdaFunction{∣} \AgdaBound{Γ} \AgdaFunction{∣}\AgdaSymbol{)} \<[34]%
\>[34]\<%
\\
\>[0]\AgdaIndent{11}{}\<[11]%
\>[11]\AgdaSymbol{→} \AgdaRecord{hSetoid} \AgdaSymbol{→} \AgdaRecord{hSetoid} \AgdaSymbol{→} \AgdaRecord{hSetoid}\<%
\\
\>\AgdaFunction{[} \AgdaBound{Γ} \AgdaFunction{,} \AgdaBound{x} \AgdaFunction{]} \AgdaBound{Ax} \AgdaFunction{⇒fm} \AgdaBound{Bx} \<[20]%
\>[20]\<%
\\
\>[0]\AgdaIndent{2}{}\<[2]%
\>[2]\AgdaSymbol{=} \AgdaKeyword{record}\<%
\\
\>[0]\AgdaIndent{6}{}\<[6]%
\>[6]\AgdaSymbol{\{} \AgdaField{Carrier} \AgdaSymbol{=} \AgdaRecord{Σ[} \AgdaBound{fn} \AgdaRecord{∶} \AgdaSymbol{(}\AgdaFunction{∣} \AgdaBound{Ax} \AgdaFunction{∣} \AgdaSymbol{→} \AgdaFunction{∣} \AgdaBound{Bx} \AgdaFunction{∣}\AgdaSymbol{)} \AgdaRecord{]} \<[46]%
\>[46]\<%
\\
\>[6]\AgdaIndent{16}{}\<[16]%
\>[16]\AgdaSymbol{((}\AgdaBound{a} \AgdaBound{b} \AgdaSymbol{:} \AgdaFunction{∣} \AgdaBound{Ax} \AgdaFunction{∣}\AgdaSymbol{)(}\AgdaBound{p} \AgdaSymbol{:} \AgdaFunction{[} \AgdaBound{Ax} \AgdaFunction{]} \AgdaBound{a} \AgdaFunction{≈} \AgdaBound{b}\AgdaSymbol{)} \<[50]%
\>[50]\<%
\\
\>[16]\AgdaIndent{18}{}\<[18]%
\>[18]\AgdaSymbol{→} \AgdaFunction{[} \AgdaBound{Bx} \AgdaFunction{]} \AgdaBound{fn} \AgdaBound{a} \AgdaFunction{≈} \AgdaBound{fn} \AgdaBound{b}\AgdaSymbol{)}\<%
\\
\>[-4]\AgdaIndent{6}{}\<[6]%
\>[6]\AgdaSymbol{;} \AgdaField{\_≈h\_} \<[16]%
\>[16]\AgdaSymbol{=} \AgdaSymbol{λ\{(}\AgdaBound{f} \AgdaInductiveConstructor{,} \AgdaSymbol{\_)} \AgdaSymbol{(}\AgdaBound{g} \AgdaInductiveConstructor{,} \AgdaSymbol{\_)} \<[36]%
\>[36]\<%
\\
\>[0]\AgdaIndent{18}{}\<[18]%
\>[18]\AgdaSymbol{→} \AgdaFunction{∀'[} \AgdaBound{a} \AgdaFunction{∶} \AgdaSymbol{\_} \AgdaFunction{]} \AgdaFunction{[} \AgdaBound{Bx} \AgdaFunction{]} \AgdaBound{f} \AgdaBound{a} \AgdaFunction{≈h} \AgdaBound{g} \AgdaBound{a} \AgdaSymbol{\}}\<%
\\
\>[0]\AgdaIndent{6}{}\<[6]%
\>[6]\AgdaSymbol{;} \AgdaField{isEquiv} \AgdaSymbol{=} \AgdaKeyword{record} \AgdaSymbol{\{}\<%
\\
\>[0]\AgdaIndent{16}{}\<[16]%
\>[16]\AgdaField{refl} \<[22]%
\>[22]\AgdaSymbol{=} \AgdaSymbol{λ} \AgdaBound{\_} \AgdaSymbol{→} \AgdaFunction{[} \AgdaBound{Bx} \AgdaFunction{]refl} \<[41]%
\>[41]\<%
\\
\>[0]\AgdaIndent{14}{}\<[14]%
\>[14]\AgdaSymbol{;} \AgdaField{sym} \<[22]%
\>[22]\AgdaSymbol{=} \AgdaSymbol{λ} \AgdaBound{f} \AgdaBound{a} \AgdaSymbol{→} \AgdaFunction{[} \AgdaBound{Bx} \AgdaFunction{]sym} \AgdaSymbol{(}\AgdaBound{f} \AgdaBound{a}\AgdaSymbol{)}\<%
\\
\>[0]\AgdaIndent{14}{}\<[14]%
\>[14]\AgdaSymbol{;} \AgdaField{trans} \AgdaSymbol{=} \AgdaSymbol{λ} \AgdaBound{f} \AgdaBound{g} \AgdaBound{a} \AgdaSymbol{→} \AgdaFunction{[} \AgdaBound{Bx} \AgdaFunction{]trans} \AgdaSymbol{(}\AgdaBound{f} \AgdaBound{a}\AgdaSymbol{)} \AgdaSymbol{(}\AgdaBound{g} \AgdaBound{a}\AgdaSymbol{)}\<%
\\
\>[14]\AgdaIndent{26}{}\<[26]%
\>[26]\AgdaSymbol{\}}\<%
\\
\>[6]\AgdaIndent{6}{}\<[6]%
\>[6]\AgdaSymbol{\}}\<%
\\
%
\\
%
\\
\>\<\end{code}

}

%to do: verification

%verification of functor laws (do we have extensional equality for record types? or eta equality?)
%define equality with respect to propositions which are proof irrelevant


%\section{Examples of types}

%\AgdaHide{
\begin{code}\>\<%
\\
%
\\
\>\AgdaSymbol{\{-\#} \AgdaKeyword{OPTIONS} --type-in-type \AgdaSymbol{\#-\}}\<%
\\
%
\\
\>\AgdaKeyword{import} \AgdaModule{Level}\<%
\\
\>\AgdaKeyword{open} \AgdaKeyword{import} \AgdaModule{Relation.Binary.PropositionalEquality} \AgdaSymbol{as} \AgdaModule{PE} \AgdaKeyword{hiding} \AgdaSymbol{(}refl \AgdaSymbol{;} sym \AgdaSymbol{;} trans\AgdaSymbol{;} isEquivalence\AgdaSymbol{;} [\_]\AgdaSymbol{)}\<%
\\
%
\\
\>\AgdaKeyword{module} \AgdaModule{CwF-ctd} \AgdaSymbol{(}\AgdaBound{ext} \AgdaSymbol{:} \AgdaFunction{Extensionality} \AgdaPrimitive{Level.zero} \AgdaPrimitive{Level.zero}\AgdaSymbol{)} \AgdaKeyword{where}\<%
\\
%
\\
\>\AgdaKeyword{open} \AgdaKeyword{import} \AgdaModule{Data.Unit}\<%
\\
\>\AgdaKeyword{open} \AgdaKeyword{import} \AgdaModule{Function}\<%
\\
\>\AgdaKeyword{open} \AgdaKeyword{import} \AgdaModule{Data.Product}\<%
\\
%
\\
\>\AgdaKeyword{open} \AgdaKeyword{import} \AgdaModule{CwF-setoidwo} \AgdaBound{ext} \AgdaKeyword{public}\<%
\\
%
\\
\>\AgdaKeyword{open} \AgdaKeyword{import} \AgdaModule{Data.Nat}\<%
\\
%
\\
\>\<\end{code}
}

Binary relation

\begin{code}\>\<%
\\
%
\\
\>\AgdaFunction{Rel} \AgdaSymbol{:} \AgdaSymbol{\{}\AgdaBound{Γ} \AgdaSymbol{:} \AgdaFunction{Con}\AgdaSymbol{\}} \AgdaSymbol{→} \AgdaRecord{Ty} \AgdaBound{Γ} \AgdaSymbol{→} \AgdaPrimitiveType{Set₁}\<%
\\
\>\AgdaFunction{Rel} \AgdaSymbol{\{}\AgdaBound{Γ}\AgdaSymbol{\}} \AgdaBound{A} \AgdaSymbol{=} \AgdaRecord{Ty} \AgdaSymbol{(}\AgdaBound{Γ} \AgdaFunction{\&} \AgdaBound{A} \AgdaFunction{\&} \AgdaBound{A} \AgdaFunction{[} \AgdaFunction{fst\&} \AgdaSymbol{\{}A \AgdaSymbol{=} \AgdaBound{A}\AgdaSymbol{\}} \AgdaFunction{]T}\AgdaSymbol{)}\<%
\\
%
\\
\>\<\end{code}

Natural numbers

\begin{code}\>\<%
\\
%
\\
\>\AgdaKeyword{module} \AgdaModule{Natural} \AgdaSymbol{(}\AgdaBound{Γ} \AgdaSymbol{:} \AgdaFunction{Con}\AgdaSymbol{)} \AgdaKeyword{where}\<%
\\
%
\\
\>[0]\AgdaIndent{2}{}\<[2]%
\>[2]\AgdaFunction{\_≈nat\_} \AgdaSymbol{:} \AgdaDatatype{ℕ} \AgdaSymbol{→} \AgdaDatatype{ℕ} \AgdaSymbol{→} \AgdaRecord{HProp}\<%
\\
\>[0]\AgdaIndent{2}{}\<[2]%
\>[2]\AgdaInductiveConstructor{zero} \AgdaFunction{≈nat} \AgdaInductiveConstructor{zero} \AgdaSymbol{=} \AgdaFunction{⊤'}\<%
\\
\>[0]\AgdaIndent{2}{}\<[2]%
\>[2]\AgdaInductiveConstructor{zero} \AgdaFunction{≈nat} \AgdaInductiveConstructor{suc} \AgdaBound{n} \AgdaSymbol{=} \AgdaFunction{⊥'}\<%
\\
\>[0]\AgdaIndent{2}{}\<[2]%
\>[2]\AgdaInductiveConstructor{suc} \AgdaBound{m} \AgdaFunction{≈nat} \AgdaInductiveConstructor{zero} \AgdaSymbol{=} \AgdaFunction{⊥'}\<%
\\
\>[0]\AgdaIndent{2}{}\<[2]%
\>[2]\AgdaInductiveConstructor{suc} \AgdaBound{m} \AgdaFunction{≈nat} \AgdaInductiveConstructor{suc} \AgdaBound{n} \AgdaSymbol{=} \AgdaBound{m} \AgdaFunction{≈nat} \AgdaBound{n}\<%
\\
\>[0]\AgdaIndent{2}{}\<[2]%
\>[2]\<%
\\
\>[0]\AgdaIndent{2}{}\<[2]%
\>[2]\AgdaFunction{reflNat} \AgdaSymbol{:} \AgdaSymbol{\{}\AgdaBound{x} \AgdaSymbol{:} \AgdaDatatype{ℕ}\AgdaSymbol{\}} \AgdaSymbol{→} \AgdaFunction{<} \AgdaBound{x} \AgdaFunction{≈nat} \AgdaBound{x} \AgdaFunction{>} \<[35]%
\>[35]\<%
\\
\>[0]\AgdaIndent{2}{}\<[2]%
\>[2]\AgdaFunction{reflNat} \AgdaSymbol{\{}\AgdaInductiveConstructor{zero}\AgdaSymbol{\}} \AgdaSymbol{=} \AgdaInductiveConstructor{tt}\<%
\\
\>[0]\AgdaIndent{2}{}\<[2]%
\>[2]\AgdaFunction{reflNat} \AgdaSymbol{\{}\AgdaInductiveConstructor{suc} \AgdaBound{n}\AgdaSymbol{\}} \AgdaSymbol{=} \AgdaFunction{reflNat} \AgdaSymbol{\{}\AgdaBound{n}\AgdaSymbol{\}}\<%
\\
%
\\
\>[0]\AgdaIndent{2}{}\<[2]%
\>[2]\AgdaFunction{symNat} \AgdaSymbol{:} \AgdaSymbol{\{}\AgdaBound{x} \AgdaBound{y} \AgdaSymbol{:} \AgdaDatatype{ℕ}\AgdaSymbol{\}} \AgdaSymbol{→} \AgdaFunction{<} \AgdaBound{x} \AgdaFunction{≈nat} \AgdaBound{y} \AgdaFunction{>} \AgdaSymbol{→} \AgdaFunction{<} \AgdaBound{y} \AgdaFunction{≈nat} \AgdaBound{x} \AgdaFunction{>}\<%
\\
\>[0]\AgdaIndent{2}{}\<[2]%
\>[2]\AgdaFunction{symNat} \AgdaSymbol{\{}\AgdaInductiveConstructor{zero}\AgdaSymbol{\}} \AgdaSymbol{\{}\AgdaInductiveConstructor{zero}\AgdaSymbol{\}} \AgdaBound{eq} \AgdaSymbol{=} \AgdaInductiveConstructor{tt}\<%
\\
\>[0]\AgdaIndent{2}{}\<[2]%
\>[2]\AgdaFunction{symNat} \AgdaSymbol{\{}\AgdaInductiveConstructor{zero}\AgdaSymbol{\}} \AgdaSymbol{\{}\AgdaInductiveConstructor{suc} \AgdaSymbol{\_\}} \AgdaBound{eq} \AgdaSymbol{=} \AgdaBound{eq}\<%
\\
\>[0]\AgdaIndent{2}{}\<[2]%
\>[2]\AgdaFunction{symNat} \AgdaSymbol{\{}\AgdaInductiveConstructor{suc} \AgdaSymbol{\_\}} \AgdaSymbol{\{}\AgdaInductiveConstructor{zero}\AgdaSymbol{\}} \AgdaBound{eq} \AgdaSymbol{=} \AgdaBound{eq}\<%
\\
\>[0]\AgdaIndent{2}{}\<[2]%
\>[2]\AgdaFunction{symNat} \AgdaSymbol{\{}\AgdaInductiveConstructor{suc} \AgdaBound{x}\AgdaSymbol{\}} \AgdaSymbol{\{}\AgdaInductiveConstructor{suc} \AgdaBound{y}\AgdaSymbol{\}} \AgdaBound{eq} \AgdaSymbol{=} \AgdaFunction{symNat} \AgdaSymbol{\{}\AgdaBound{x}\AgdaSymbol{\}} \AgdaSymbol{\{}\AgdaBound{y}\AgdaSymbol{\}} \AgdaBound{eq}\<%
\\
%
\\
\>[0]\AgdaIndent{2}{}\<[2]%
\>[2]\AgdaFunction{transNat} \AgdaSymbol{:} \AgdaSymbol{\{}\AgdaBound{x} \AgdaBound{y} \AgdaBound{z} \AgdaSymbol{:} \AgdaDatatype{ℕ}\AgdaSymbol{\}} \AgdaSymbol{→} \AgdaFunction{<} \AgdaBound{x} \AgdaFunction{≈nat} \AgdaBound{y} \AgdaFunction{>} \AgdaSymbol{→} \AgdaFunction{<} \AgdaBound{y} \AgdaFunction{≈nat} \AgdaBound{z} \AgdaFunction{>} \AgdaSymbol{→} \AgdaFunction{<} \AgdaBound{x} \AgdaFunction{≈nat} \AgdaBound{z} \AgdaFunction{>}\<%
\\
\>[0]\AgdaIndent{2}{}\<[2]%
\>[2]\AgdaFunction{transNat} \AgdaSymbol{\{}\AgdaInductiveConstructor{zero}\AgdaSymbol{\}} \AgdaSymbol{\{}\AgdaInductiveConstructor{zero}\AgdaSymbol{\}} \AgdaBound{xy} \AgdaBound{yz} \AgdaSymbol{=} \AgdaBound{yz}\<%
\\
\>[0]\AgdaIndent{2}{}\<[2]%
\>[2]\AgdaFunction{transNat} \AgdaSymbol{\{}\AgdaInductiveConstructor{zero}\AgdaSymbol{\}} \AgdaSymbol{\{}\AgdaInductiveConstructor{suc} \AgdaSymbol{\_\}} \AgdaSymbol{()} \AgdaBound{yz}\<%
\\
\>[0]\AgdaIndent{2}{}\<[2]%
\>[2]\AgdaFunction{transNat} \AgdaSymbol{\{}\AgdaInductiveConstructor{suc} \AgdaSymbol{\_\}} \AgdaSymbol{\{}\AgdaInductiveConstructor{zero}\AgdaSymbol{\}} \AgdaSymbol{()} \AgdaBound{yz}\<%
\\
\>[0]\AgdaIndent{2}{}\<[2]%
\>[2]\AgdaFunction{transNat} \AgdaSymbol{\{}\AgdaInductiveConstructor{suc} \AgdaSymbol{\_\}} \AgdaSymbol{\{}\AgdaInductiveConstructor{suc} \AgdaSymbol{\_\}} \AgdaSymbol{\{}\AgdaInductiveConstructor{zero}\AgdaSymbol{\}} \AgdaBound{xy} \AgdaBound{yz} \AgdaSymbol{=} \AgdaBound{yz}\<%
\\
\>[0]\AgdaIndent{2}{}\<[2]%
\>[2]\AgdaFunction{transNat} \AgdaSymbol{\{}\AgdaInductiveConstructor{suc} \AgdaBound{x}\AgdaSymbol{\}} \AgdaSymbol{\{}\AgdaInductiveConstructor{suc} \AgdaBound{y}\AgdaSymbol{\}} \AgdaSymbol{\{}\AgdaInductiveConstructor{suc} \AgdaBound{z}\AgdaSymbol{\}} \AgdaBound{xy} \AgdaBound{yz} \AgdaSymbol{=} \AgdaFunction{transNat} \AgdaSymbol{\{}\AgdaBound{x}\AgdaSymbol{\}} \AgdaSymbol{\{}\AgdaBound{y}\AgdaSymbol{\}} \AgdaSymbol{\{}\AgdaBound{z}\AgdaSymbol{\}} \AgdaBound{xy} \AgdaBound{yz}\<%
\\
%
\\
\>[0]\AgdaIndent{2}{}\<[2]%
\>[2]\AgdaFunction{⟦Nat⟧} \AgdaSymbol{:} \AgdaRecord{Ty} \AgdaBound{Γ}\<%
\\
\>[0]\AgdaIndent{2}{}\<[2]%
\>[2]\AgdaFunction{⟦Nat⟧} \AgdaSymbol{=} \AgdaKeyword{record} \<[17]%
\>[17]\<%
\\
\>[2]\AgdaIndent{4}{}\<[4]%
\>[4]\AgdaSymbol{\{} \AgdaField{fm} \AgdaSymbol{=} \AgdaSymbol{λ} \AgdaBound{γ} \AgdaSymbol{→} \AgdaKeyword{record}\<%
\\
\>[4]\AgdaIndent{9}{}\<[9]%
\>[9]\AgdaSymbol{\{} \AgdaField{Carrier} \AgdaSymbol{=} \AgdaDatatype{ℕ}\<%
\\
\>[4]\AgdaIndent{9}{}\<[9]%
\>[9]\AgdaSymbol{;} \AgdaField{\_≈h\_} \AgdaSymbol{=} \AgdaFunction{\_≈nat\_}\<%
\\
\>[4]\AgdaIndent{9}{}\<[9]%
\>[9]\AgdaSymbol{;} \AgdaField{refl} \AgdaSymbol{=} \AgdaSymbol{λ} \AgdaSymbol{\{}\AgdaBound{n}\AgdaSymbol{\}} \AgdaSymbol{→} \AgdaFunction{reflNat} \AgdaSymbol{\{}\AgdaBound{n}\AgdaSymbol{\}}\<%
\\
\>[4]\AgdaIndent{9}{}\<[9]%
\>[9]\AgdaSymbol{;} \AgdaField{sym} \AgdaSymbol{=} \AgdaSymbol{λ} \AgdaSymbol{\{}\AgdaBound{x}\AgdaSymbol{\}} \AgdaSymbol{\{}\AgdaBound{y}\AgdaSymbol{\}} \AgdaSymbol{→} \AgdaFunction{symNat} \AgdaSymbol{\{}\AgdaBound{x}\AgdaSymbol{\}} \AgdaSymbol{\{}\AgdaBound{y}\AgdaSymbol{\}}\<%
\\
\>[4]\AgdaIndent{9}{}\<[9]%
\>[9]\AgdaSymbol{;} \AgdaField{trans} \AgdaSymbol{=} \AgdaSymbol{λ} \AgdaSymbol{\{}\AgdaBound{x}\AgdaSymbol{\}} \AgdaSymbol{\{}\AgdaBound{y}\AgdaSymbol{\}} \AgdaSymbol{\{}\AgdaBound{z}\AgdaSymbol{\}} \AgdaSymbol{→} \AgdaFunction{transNat} \AgdaSymbol{\{}\AgdaBound{x}\AgdaSymbol{\}} \AgdaSymbol{\{}\AgdaBound{y}\AgdaSymbol{\}} \AgdaSymbol{\{}\AgdaBound{z}\AgdaSymbol{\}}\<%
\\
\>[4]\AgdaIndent{9}{}\<[9]%
\>[9]\AgdaSymbol{\}}\<%
\\
\>[0]\AgdaIndent{4}{}\<[4]%
\>[4]\AgdaSymbol{;} \AgdaField{substT} \AgdaSymbol{=} \AgdaSymbol{λ} \AgdaBound{\_} \AgdaSymbol{→} \AgdaFunction{id}\<%
\\
\>[0]\AgdaIndent{4}{}\<[4]%
\>[4]\AgdaSymbol{;} \AgdaField{subst*} \AgdaSymbol{=} \AgdaSymbol{λ} \AgdaBound{\_} \AgdaSymbol{→} \AgdaFunction{id}\<%
\\
\>[0]\AgdaIndent{4}{}\<[4]%
\>[4]\AgdaSymbol{;} \AgdaField{refl*} \AgdaSymbol{=} \AgdaSymbol{λ} \AgdaBound{x} \AgdaBound{a} \AgdaSymbol{→} \AgdaFunction{reflNat} \AgdaSymbol{\{}\AgdaBound{a}\AgdaSymbol{\}}\<%
\\
\>[0]\AgdaIndent{4}{}\<[4]%
\>[4]\AgdaSymbol{;} \AgdaField{trans*} \AgdaSymbol{=} \AgdaSymbol{λ} \AgdaBound{a} \AgdaSymbol{→} \AgdaFunction{reflNat} \AgdaSymbol{\{}\AgdaBound{a}\AgdaSymbol{\}} \<[33]%
\>[33]\<%
\\
\>[0]\AgdaIndent{4}{}\<[4]%
\>[4]\AgdaSymbol{\}}\<%
\\
%
\\
\>[0]\AgdaIndent{2}{}\<[2]%
\>[2]\AgdaFunction{⟦0⟧} \AgdaSymbol{:} \AgdaRecord{Tm} \AgdaFunction{⟦Nat⟧}\<%
\\
\>[0]\AgdaIndent{2}{}\<[2]%
\>[2]\AgdaFunction{⟦0⟧} \AgdaSymbol{=} \AgdaKeyword{record}\<%
\\
\>[2]\AgdaIndent{6}{}\<[6]%
\>[6]\AgdaSymbol{\{} \AgdaField{tm} \AgdaSymbol{=} \AgdaSymbol{λ} \AgdaBound{\_} \AgdaSymbol{→} \AgdaNumber{0}\<%
\\
\>[2]\AgdaIndent{6}{}\<[6]%
\>[6]\AgdaSymbol{;} \AgdaField{respt} \AgdaSymbol{=} \AgdaSymbol{λ} \AgdaBound{p} \AgdaSymbol{→} \AgdaInductiveConstructor{tt}\<%
\\
\>[2]\AgdaIndent{6}{}\<[6]%
\>[6]\AgdaSymbol{\}}\<%
\\
%
\\
\>[0]\AgdaIndent{2}{}\<[2]%
\>[2]\AgdaFunction{⟦s⟧} \AgdaSymbol{:} \AgdaRecord{Tm} \AgdaFunction{⟦Nat⟧} \AgdaSymbol{→} \AgdaRecord{Tm} \AgdaFunction{⟦Nat⟧}\<%
\\
\>[0]\AgdaIndent{2}{}\<[2]%
\>[2]\AgdaFunction{⟦s⟧} \AgdaSymbol{(}\AgdaInductiveConstructor{tm:} \AgdaBound{t} \AgdaInductiveConstructor{resp:} \AgdaBound{respt}\AgdaSymbol{)} \<[26]%
\>[26]\<%
\\
\>[2]\AgdaIndent{6}{}\<[6]%
\>[6]\AgdaSymbol{=} \AgdaKeyword{record}\<%
\\
\>[2]\AgdaIndent{6}{}\<[6]%
\>[6]\AgdaSymbol{\{} \AgdaField{tm} \AgdaSymbol{=} \AgdaInductiveConstructor{suc} \AgdaFunction{∘} \AgdaBound{t}\<%
\\
\>[2]\AgdaIndent{6}{}\<[6]%
\>[6]\AgdaSymbol{;} \AgdaField{respt} \AgdaSymbol{=} \AgdaBound{respt}\<%
\\
\>[2]\AgdaIndent{6}{}\<[6]%
\>[6]\AgdaSymbol{\}}\<%
\\
%
\\
\>\<\end{code}

Simply typed universe

\AgdaHide{
\begin{code}\>\<%
\\
%
\\
\>\AgdaComment{\{-
  data  ⟦U⟧⁰ : Set where
    nat : ⟦U⟧⁰
    arr<\_,\_> : (a b : ⟦U⟧⁰) → ⟦U⟧⁰

  \_\textasciitilde⟦U⟧\_ : ⟦U⟧⁰ → ⟦U⟧⁰ → HProp
  nat \textasciitilde⟦U⟧ nat = ⊤'
  nat \textasciitilde⟦U⟧ arr< a , b > = ⊥'
  arr< a , b > \textasciitilde⟦U⟧ nat = ⊥'
  arr< a , b > \textasciitilde⟦U⟧ arr< a' , b' > = a \textasciitilde⟦U⟧ a' ∧ b \textasciitilde⟦U⟧ b'

  reflU :  \{x : ⟦U⟧⁰\} → < x \textasciitilde⟦U⟧ x >
  reflU \{nat\} = tt
  reflU \{arr< a , b >\} = reflU \{a\} , reflU \{b\}

  symU : \{x y : ⟦U⟧⁰\} → < x \textasciitilde⟦U⟧ y > → < y \textasciitilde⟦U⟧ x >
  symU \{nat\} \{nat\} eq = tt
  symU \{nat\} \{arr< a , b >\} eq = eq
  symU \{arr< a , b >\} \{nat\} eq = eq
  symU \{arr< a , b >\} \{arr< a' , b' >\} (p , q) = (symU \{a\} \{a'\} p) 
                                               , (symU \{b\} \{b'\} q)

  transU : \{x y z : ⟦U⟧⁰\} → < x \textasciitilde⟦U⟧ y > → < y \textasciitilde⟦U⟧ z > → < x \textasciitilde⟦U⟧ z >
  transU \{nat\} \{nat\} eq1 eq2 = eq2
  transU \{nat\} \{arr< a , b >\} () eq2
  transU \{arr< a , b >\} \{nat\} () eq2
  transU \{arr< a , b >\} \{arr< a' , b' >\} \{nat\} eq1 eq2 = eq2
  transU \{arr< a , b >\} \{arr< a' , b' >\} \{arr< a0 , b0 >\} (p1 , q1) 
         (p2 , q2) = (transU \{a\} \{a'\} \{a0\} p1 p2) 
         , transU \{b\} \{b'\} \{b0\} q1 q2

  ⟦U⟧ : Ty Γ
  ⟦U⟧ = record 
    \{ fm = λ γ → record
         \{ Carrier = ⟦U⟧⁰
         ; \_≈h\_ = \_\textasciitilde⟦U⟧\_
         ; refl = λ \{x\} → reflU \{x\}
         ; sym = λ \{x\} \{y\} → symU \{x\} \{y\}
         ; trans = λ \{x\} \{y\} \{z\} → transU \{x\} \{y\} \{z\}
         \}
    ; substT = λ \_ → id
    ; subst* = λ \_ → id
    ; refl* = λ x a → reflU \{a\}
    ; trans* = λ a → reflU \{a\}
    \}

  elfm : Σ ∣ Γ ∣ (λ x → ⟦U⟧⁰) → HSetoid
  elfm (γ , nat) = [ ⟦Nat⟧ ]fm γ
  elfm (γ , arr< a , b >) = [ Γ , γ ] elfm (γ , a) ⇒fm elfm (γ , b)
-\}}\<%
\\
%
\\
\>\<\end{code}
}

\AgdaHide{
\begin{code}\>\<%
\\
%
\\
\>\AgdaComment{\{- To do : To find the way to extract the substT from ->

  elsubstT : \{x y : Σ ∣ Γ ∣ (λ x' → ⟦U⟧⁰)\} →
      Σ < [ Γ ] proj₁ x ≈h proj₁ y > (λ x' → < proj₂ x \textasciitilde⟦U⟧ proj₂ y >) →
      ∣ elfm x ∣ → ∣ elfm y ∣
  elsubstT \{\_ , nat\} \{\_ , nat\} \_ x' = x'
  elsubstT \{\_ , nat\} \{\_ , arr< a , b >\} (p , ()) x'
  elsubstT \{\_ , arr< a , b >\} \{\_ , nat\} (p , ()) x'
  elsubstT \{γ , arr< a , b >\} \{γ' , arr< a' , b' >\} (p , qa , qb) (s1 , s2) = 
   \{!!\}

  ⟦El⟧ : Ty (Γ \& ⟦U⟧)
  ⟦El⟧ = record 
       \{ fm = elfm
       ; substT = elsubstT
       ; subst* = \{!!\}
       ; refl* = \{!!\}
       ; trans* = \{!!\} 
       \}

-\}}\<%
\\
\>\<\end{code}
}

The equality type

\begin{code}\>\<%
\\
%
\\
\>\AgdaKeyword{module} \AgdaModule{Equality-Type} \AgdaSymbol{(}\AgdaBound{Γ} \AgdaSymbol{:} \AgdaFunction{Con}\AgdaSymbol{)(}\AgdaBound{A} \AgdaSymbol{:} \AgdaRecord{Ty} \AgdaBound{Γ}\AgdaSymbol{)} \AgdaKeyword{where}\<%
\\
%
\\
\>[0]\AgdaIndent{2}{}\<[2]%
\>[2]\AgdaFunction{⟦Id⟧} \AgdaSymbol{:} \AgdaFunction{Rel} \AgdaBound{A}\<%
\\
\>[0]\AgdaIndent{2}{}\<[2]%
\>[2]\AgdaFunction{⟦Id⟧} \AgdaSymbol{=} \AgdaKeyword{record} \<[16]%
\>[16]\<%
\\
\>[2]\AgdaIndent{4}{}\<[4]%
\>[4]\AgdaSymbol{\{} \AgdaField{fm} \AgdaSymbol{=} \AgdaSymbol{λ} \AgdaSymbol{\{((}\AgdaBound{x} \AgdaInductiveConstructor{,} \AgdaBound{a}\AgdaSymbol{)} \AgdaInductiveConstructor{,} \AgdaBound{b}\AgdaSymbol{)} \AgdaSymbol{→} \<[30]%
\>[30]\<%
\\
\>[4]\AgdaIndent{13}{}\<[13]%
\>[13]\AgdaKeyword{record} \<[20]%
\>[20]\<%
\\
\>[4]\AgdaIndent{13}{}\<[13]%
\>[13]\AgdaSymbol{\{} \AgdaField{Carrier} \AgdaSymbol{=} \AgdaFunction{[} \AgdaFunction{[} \AgdaBound{A} \AgdaFunction{]fm} \AgdaBound{x} \AgdaFunction{]} \AgdaBound{a} \AgdaFunction{≈} \AgdaBound{b}\<%
\\
\>[4]\AgdaIndent{13}{}\<[13]%
\>[13]\AgdaSymbol{;} \AgdaField{\_≈h\_} \AgdaSymbol{=} \AgdaSymbol{λ} \AgdaBound{\_} \AgdaBound{\_} \AgdaSymbol{→} \AgdaKeyword{record} \AgdaSymbol{\{} \AgdaField{prf} \AgdaSymbol{=} \AgdaRecord{⊤} \AgdaSymbol{;} \AgdaField{Uni} \AgdaSymbol{=} \AgdaInductiveConstructor{PE.refl} \AgdaSymbol{\}}\<%
\\
\>[4]\AgdaIndent{13}{}\<[13]%
\>[13]\AgdaSymbol{;} \AgdaField{refl} \AgdaSymbol{=} \AgdaInductiveConstructor{tt} \<[25]%
\>[25]\<%
\\
\>[4]\AgdaIndent{13}{}\<[13]%
\>[13]\AgdaSymbol{;} \AgdaField{sym} \AgdaSymbol{=} \AgdaFunction{id}\<%
\\
\>[4]\AgdaIndent{13}{}\<[13]%
\>[13]\AgdaSymbol{;} \AgdaField{trans} \AgdaSymbol{=} \AgdaSymbol{λ} \AgdaBound{\_} \AgdaBound{\_} \AgdaSymbol{→} \AgdaInductiveConstructor{tt}\<%
\\
\>[4]\AgdaIndent{13}{}\<[13]%
\>[13]\AgdaSymbol{\}}\<%
\\
\>[4]\AgdaIndent{13}{}\<[13]%
\>[13]\AgdaSymbol{\}}\<%
\\
\>[0]\AgdaIndent{4}{}\<[4]%
\>[4]\AgdaSymbol{;} \AgdaField{substT} \AgdaSymbol{=} \AgdaSymbol{λ} \AgdaSymbol{\{((}\AgdaBound{x} \AgdaInductiveConstructor{,} \AgdaBound{a}\AgdaSymbol{)} \AgdaInductiveConstructor{,} \AgdaBound{b}\AgdaSymbol{)} \AgdaBound{x0} \AgdaSymbol{→} \<[37]%
\>[37]\<%
\\
\>[0]\AgdaIndent{15}{}\<[15]%
\>[15]\AgdaFunction{[} \AgdaFunction{[} \AgdaBound{A} \AgdaFunction{]fm} \AgdaSymbol{\_} \AgdaFunction{]trans} \<[34]%
\>[34]\<%
\\
\>[0]\AgdaIndent{15}{}\<[15]%
\>[15]\AgdaSymbol{(}\AgdaFunction{[} \AgdaFunction{[} \AgdaBound{A} \AgdaFunction{]fm} \AgdaSymbol{\_} \AgdaFunction{]sym} \AgdaBound{a}\AgdaSymbol{)} \<[36]%
\>[36]\<%
\\
\>[0]\AgdaIndent{15}{}\<[15]%
\>[15]\AgdaSymbol{(}\AgdaFunction{[} \AgdaFunction{[} \AgdaBound{A} \AgdaFunction{]fm} \AgdaSymbol{\_} \AgdaFunction{]trans} \<[35]%
\>[35]\<%
\\
\>[0]\AgdaIndent{15}{}\<[15]%
\>[15]\AgdaSymbol{(}\AgdaFunction{[} \AgdaBound{A} \AgdaFunction{]subst*} \AgdaSymbol{\_} \AgdaBound{x0}\AgdaSymbol{)} \AgdaBound{b}\AgdaSymbol{)} \<[37]%
\>[37]\<%
\\
\>[0]\AgdaIndent{15}{}\<[15]%
\>[15]\AgdaSymbol{\}}\<%
\\
\>[0]\AgdaIndent{4}{}\<[4]%
\>[4]\AgdaSymbol{;} \AgdaField{subst*} \AgdaSymbol{=} \AgdaSymbol{λ} \AgdaBound{\_} \AgdaBound{\_} \AgdaSymbol{→} \AgdaInductiveConstructor{tt}\<%
\\
\>[0]\AgdaIndent{4}{}\<[4]%
\>[4]\AgdaSymbol{;} \AgdaField{refl*} \AgdaSymbol{=} \AgdaSymbol{λ} \AgdaBound{\_} \AgdaBound{\_} \AgdaSymbol{→} \AgdaInductiveConstructor{tt}\<%
\\
\>[0]\AgdaIndent{4}{}\<[4]%
\>[4]\AgdaSymbol{;} \AgdaField{trans*} \AgdaSymbol{=} \AgdaSymbol{λ} \AgdaBound{\_} \AgdaSymbol{→} \AgdaInductiveConstructor{tt} \<[24]%
\>[24]\<%
\\
\>[0]\AgdaIndent{4}{}\<[4]%
\>[4]\AgdaSymbol{\}}\<%
\\
%
\\
%
\\
\>[0]\AgdaIndent{2}{}\<[2]%
\>[2]\AgdaFunction{⟦refl⟧⁰} \AgdaSymbol{:} \AgdaRecord{Tm} \AgdaSymbol{\{}\AgdaBound{Γ} \AgdaFunction{\&} \AgdaBound{A}\AgdaSymbol{\}} \AgdaSymbol{(}\AgdaFunction{⟦Id⟧} \AgdaFunction{[} \AgdaKeyword{record} \AgdaSymbol{\{} \AgdaField{fn} \AgdaSymbol{=} \AgdaSymbol{λ} \AgdaBound{x'} \AgdaSymbol{→} \AgdaBound{x'} \AgdaInductiveConstructor{,} \AgdaFunction{proj₂} \AgdaBound{x'} \<[66]%
\>[66]\<%
\\
\>[0]\AgdaIndent{23}{}\<[23]%
\>[23]\AgdaSymbol{;} \AgdaField{resp} \AgdaSymbol{=} \AgdaSymbol{λ} \AgdaBound{x'} \AgdaSymbol{→} \AgdaBound{x'} \AgdaInductiveConstructor{,} \AgdaFunction{proj₂} \AgdaBound{x'} \AgdaSymbol{\}} \AgdaFunction{]T}\AgdaSymbol{)} \<[59]%
\>[59]\<%
\\
\>[0]\AgdaIndent{2}{}\<[2]%
\>[2]\AgdaFunction{⟦refl⟧⁰} \AgdaSymbol{=} \AgdaKeyword{record}\<%
\\
\>[0]\AgdaIndent{11}{}\<[11]%
\>[11]\AgdaSymbol{\{} \AgdaField{tm} \AgdaSymbol{=} \AgdaSymbol{λ} \AgdaSymbol{\{(}\AgdaBound{x} \AgdaInductiveConstructor{,} \AgdaBound{a}\AgdaSymbol{)} \AgdaSymbol{→} \AgdaFunction{[} \AgdaFunction{[} \AgdaBound{A} \AgdaFunction{]fm} \AgdaBound{x} \AgdaFunction{]refl} \AgdaSymbol{\{}\AgdaBound{a}\AgdaSymbol{\}} \AgdaSymbol{\}}\<%
\\
\>[0]\AgdaIndent{11}{}\<[11]%
\>[11]\AgdaSymbol{;} \AgdaField{respt} \AgdaSymbol{=} \AgdaSymbol{λ} \AgdaBound{p} \AgdaSymbol{→} \AgdaInductiveConstructor{tt}\<%
\\
\>[0]\AgdaIndent{11}{}\<[11]%
\>[11]\AgdaSymbol{\}}\<%
\\
%
\\
\>[0]\AgdaIndent{2}{}\<[2]%
\>[2]\AgdaFunction{⟦refl⟧} \AgdaSymbol{=} \<[12]%
\>[12]\AgdaFunction{lam} \AgdaSymbol{\{}\AgdaBound{Γ}\AgdaSymbol{\}} \AgdaSymbol{\{}\AgdaBound{A}\AgdaSymbol{\}} \AgdaFunction{⟦refl⟧⁰}\<%
\\
%
\\
\>\<\end{code}

Subst using equality types

\begin{code}\>\<%
\\
%
\\
\>[0]\AgdaIndent{2}{}\<[2]%
\>[2]\AgdaKeyword{module} \AgdaModule{substIn} \AgdaSymbol{(}\AgdaBound{B} \AgdaSymbol{:} \AgdaRecord{Ty} \AgdaSymbol{(}\AgdaBound{Γ} \AgdaFunction{\&} \AgdaBound{A}\AgdaSymbol{))} \AgdaKeyword{where}\<%
\\
\>[0]\AgdaIndent{2}{}\<[2]%
\>[2]\<%
\\
\>[2]\AgdaIndent{4}{}\<[4]%
\>[4]\AgdaFunction{⟦subst⟧⁰} \AgdaSymbol{:} \AgdaRecord{Tm} \AgdaSymbol{\{}\AgdaBound{Γ} \AgdaFunction{\&} \AgdaBound{A} \AgdaFunction{\&} \AgdaSymbol{(}\AgdaBound{A} \AgdaFunction{[} \AgdaFunction{fst\&} \AgdaSymbol{\{}A \AgdaSymbol{=} \AgdaBound{A}\AgdaSymbol{\}} \AgdaFunction{]T}\AgdaSymbol{)} \<[49]%
\>[49]\<%
\\
\>[4]\AgdaIndent{15}{}\<[15]%
\>[15]\AgdaFunction{\&} \AgdaFunction{⟦Id⟧} \AgdaFunction{\&} \AgdaBound{B} \AgdaFunction{[} \AgdaFunction{fst\&} \AgdaSymbol{\{}A \AgdaSymbol{=} \AgdaBound{A} \AgdaFunction{[} \AgdaFunction{fst\&} \AgdaSymbol{\{}A \AgdaSymbol{=} \AgdaBound{A}\AgdaSymbol{\}} \AgdaFunction{]T}\AgdaSymbol{\}} \<[60]%
\>[60]\AgdaFunction{]T} \<[63]%
\>[63]\<%
\\
\>[4]\AgdaIndent{15}{}\<[15]%
\>[15]\AgdaFunction{[} \AgdaFunction{fst\&} \AgdaSymbol{\{}A \AgdaSymbol{=} \AgdaFunction{⟦Id⟧}\AgdaSymbol{\}} \AgdaFunction{]T}\AgdaSymbol{\}} \<[37]%
\>[37]\<%
\\
\>[-2]\AgdaIndent{13}{}\<[13]%
\>[13]\AgdaSymbol{(}\AgdaBound{B} \AgdaFunction{[} \AgdaKeyword{record} \AgdaSymbol{\{} \AgdaField{fn} \AgdaSymbol{=} \AgdaSymbol{λ} \AgdaBound{x} \AgdaSymbol{→} \AgdaSymbol{(}\AgdaFunction{proj₁} \AgdaSymbol{(}\AgdaFunction{proj₁} \AgdaSymbol{(}\AgdaFunction{proj₁} \AgdaSymbol{(}\AgdaFunction{proj₁} \AgdaBound{x}\AgdaSymbol{))))} \<[72]%
\>[72]\<%
\\
\>[0]\AgdaIndent{13}{}\<[13]%
\>[13]\AgdaInductiveConstructor{,} \AgdaSymbol{(}\AgdaFunction{proj₂} \AgdaSymbol{(}\AgdaFunction{proj₁} \AgdaSymbol{(}\AgdaFunction{proj₁} \AgdaBound{x}\AgdaSymbol{)))} \<[41]%
\>[41]\<%
\\
\>[0]\AgdaIndent{13}{}\<[13]%
\>[13]\AgdaSymbol{;} \AgdaField{resp} \AgdaSymbol{=} \AgdaSymbol{λ} \AgdaBound{x} \AgdaSymbol{→} \AgdaFunction{proj₁} \AgdaSymbol{(}\AgdaFunction{proj₁} \AgdaSymbol{(}\AgdaFunction{proj₁} \AgdaSymbol{(}\AgdaFunction{proj₁} \AgdaBound{x}\AgdaSymbol{)))} \<[60]%
\>[60]\<%
\\
\>[0]\AgdaIndent{13}{}\<[13]%
\>[13]\AgdaInductiveConstructor{,} \AgdaFunction{proj₂} \AgdaSymbol{(}\AgdaFunction{proj₁} \AgdaSymbol{(}\AgdaFunction{proj₁} \AgdaBound{x}\AgdaSymbol{))} \AgdaSymbol{\}} \AgdaFunction{]T}\AgdaSymbol{)}\<%
\\
%
\\
\>[0]\AgdaIndent{4}{}\<[4]%
\>[4]\AgdaFunction{⟦subst⟧⁰} \AgdaSymbol{=} \AgdaKeyword{record}\<%
\\
\>[0]\AgdaIndent{11}{}\<[11]%
\>[11]\AgdaSymbol{\{} \AgdaField{tm} \AgdaSymbol{=} \AgdaSymbol{λ} \AgdaSymbol{\{((((}\AgdaBound{x} \AgdaInductiveConstructor{,} \AgdaBound{a}\AgdaSymbol{)} \AgdaInductiveConstructor{,} \AgdaBound{b}\AgdaSymbol{)} \AgdaInductiveConstructor{,} \AgdaBound{p}\AgdaSymbol{)} \AgdaInductiveConstructor{,} \AgdaBound{PA}\AgdaSymbol{)} \AgdaSymbol{→} \AgdaFunction{[} \AgdaBound{B} \AgdaFunction{]subst} \<[61]%
\>[61]\<%
\\
\>[11]\AgdaIndent{18}{}\<[18]%
\>[18]\AgdaSymbol{(}\AgdaFunction{[} \AgdaBound{Γ} \AgdaFunction{]refl} \AgdaInductiveConstructor{,} \AgdaFunction{[} \AgdaFunction{[} \AgdaBound{A} \AgdaFunction{]fm} \AgdaSymbol{\_} \AgdaFunction{]trans} \<[50]%
\>[50]\<%
\\
\>[11]\AgdaIndent{18}{}\<[18]%
\>[18]\AgdaSymbol{(}\AgdaFunction{[} \AgdaBound{A} \AgdaFunction{]refl*} \AgdaSymbol{\_} \AgdaSymbol{\_)} \AgdaBound{p}\AgdaSymbol{)} \AgdaBound{PA} \AgdaSymbol{\}}\<%
\\
\>[0]\AgdaIndent{11}{}\<[11]%
\>[11]\AgdaSymbol{;} \AgdaField{respt} \AgdaSymbol{=} \AgdaSymbol{λ} \AgdaSymbol{\{((((}\AgdaBound{m} \AgdaInductiveConstructor{,} \AgdaBound{a}\AgdaSymbol{)} \AgdaInductiveConstructor{,} \AgdaBound{b}\AgdaSymbol{)} \AgdaInductiveConstructor{,} \AgdaBound{p}\AgdaSymbol{)} \AgdaInductiveConstructor{,} \AgdaBound{PA}\AgdaSymbol{)} \AgdaSymbol{→} \<[53]%
\>[53]\<%
\\
\>[0]\AgdaIndent{13}{}\<[13]%
\>[13]\AgdaFunction{[} \AgdaFunction{[} \AgdaBound{B} \AgdaFunction{]fm} \AgdaSymbol{\_} \AgdaFunction{]trans} \<[32]%
\>[32]\<%
\\
\>[0]\AgdaIndent{13}{}\<[13]%
\>[13]\AgdaSymbol{(}\AgdaFunction{[} \AgdaBound{B} \AgdaFunction{]trans*} \AgdaSymbol{\_)} \<[29]%
\>[29]\<%
\\
\>[13]\AgdaIndent{14}{}\<[14]%
\>[14]\AgdaSymbol{(}\AgdaFunction{[} \AgdaFunction{[} \AgdaBound{B} \AgdaFunction{]fm} \AgdaSymbol{\_} \AgdaFunction{]trans} \<[34]%
\>[34]\<%
\\
\>[0]\AgdaIndent{13}{}\<[13]%
\>[13]\AgdaFunction{[} \AgdaBound{B} \AgdaFunction{]subst-pi} \<[27]%
\>[27]\<%
\\
\>[0]\AgdaIndent{13}{}\<[13]%
\>[13]\AgdaSymbol{(}\AgdaFunction{[} \AgdaFunction{[} \AgdaBound{B} \AgdaFunction{]fm} \AgdaSymbol{\_} \AgdaFunction{]trans} \<[33]%
\>[33]\<%
\\
\>[0]\AgdaIndent{13}{}\<[13]%
\>[13]\AgdaSymbol{(}\AgdaFunction{[} \AgdaFunction{[} \AgdaBound{B} \AgdaFunction{]fm} \AgdaSymbol{\_} \AgdaFunction{]sym} \AgdaSymbol{(}\AgdaFunction{[} \AgdaBound{B} \AgdaFunction{]trans*} \AgdaSymbol{\_))}\<%
\\
\>[0]\AgdaIndent{13}{}\<[13]%
\>[13]\AgdaSymbol{(}\AgdaFunction{[} \AgdaBound{B} \AgdaFunction{]subst*} \AgdaSymbol{\_} \AgdaBound{PA}\AgdaSymbol{)} \AgdaSymbol{))} \AgdaSymbol{\}}\<%
\\
\>[0]\AgdaIndent{11}{}\<[11]%
\>[11]\AgdaSymbol{\}}\<%
\\
\>\<\end{code}


\AgdaHide{
\begin{code}\>\<%
\\
%
\\
\>\AgdaComment{--    ⟦subst⟧ = lam (lam (lam ⟦subst⟧⁰))}\<%
\\
\>[0]\AgdaIndent{4}{}\<[4]%
\>[4]\<%
\\
\>\<\end{code}
}

%
\AgdaHide{
\begin{code}\>\<%
\\
%
\\
\>\AgdaSymbol{\{-\#} \AgdaKeyword{OPTIONS} --type-in-type \AgdaSymbol{\#-\}}\<%
\\
%
\\
\>\AgdaKeyword{import} \AgdaModule{Level}\<%
\\
\>\AgdaKeyword{open} \AgdaKeyword{import} \AgdaModule{Relation.Binary.PropositionalEquality} \AgdaSymbol{as} \AgdaModule{PE} \AgdaKeyword{hiding} \AgdaSymbol{(}refl \AgdaSymbol{;} sym \AgdaSymbol{;} trans\AgdaSymbol{;} isEquivalence\AgdaSymbol{;} [\_]\AgdaSymbol{)}\<%
\\
%
\\
\>\AgdaKeyword{module} \AgdaModule{CwF-quotient} \AgdaSymbol{(}\AgdaBound{ext} \AgdaSymbol{:} \AgdaFunction{Extensionality} \AgdaPrimitive{Level.zero} \AgdaPrimitive{Level.zero}\AgdaSymbol{)} \AgdaKeyword{where}\<%
\\
%
\\
\>\AgdaKeyword{open} \AgdaKeyword{import} \AgdaModule{Data.Unit}\<%
\\
\>\AgdaKeyword{open} \AgdaKeyword{import} \AgdaModule{Function}\<%
\\
\>\AgdaKeyword{open} \AgdaKeyword{import} \AgdaModule{Data.Product}\<%
\\
%
\\
%
\\
\>\AgdaComment{-- importing other CWF files}\<%
\\
%
\\
\>\AgdaKeyword{import} \AgdaModule{CwF-setoid}\<%
\\
%
\\
\>\AgdaKeyword{open} \AgdaModule{CwF-setoid} \AgdaBound{ext}\<%
\\
%
\\
\>\AgdaKeyword{import} \AgdaModule{CategoryOfSetoid}\<%
\\
\>\AgdaKeyword{module} \AgdaModule{cos'} \AgdaSymbol{=} \AgdaModule{CategoryOfSetoid} \AgdaBound{ext}\<%
\\
\>\AgdaKeyword{open} \AgdaModule{cos'}\<%
\\
%
\\
\>\AgdaKeyword{import} \AgdaModule{hProp}\<%
\\
\>\AgdaKeyword{module} \AgdaModule{hp'} \AgdaSymbol{=} \AgdaModule{hProp} \AgdaBound{ext}\<%
\\
\>\AgdaKeyword{open} \AgdaModule{hp'}\<%
\\
%
\\
\>\AgdaKeyword{import} \AgdaModule{CwF-ctd}\<%
\\
\>\AgdaKeyword{module} \AgdaModule{cc} \AgdaSymbol{=} \AgdaModule{CwF-ctd} \AgdaBound{ext}\<%
\\
\>\AgdaKeyword{open} \AgdaModule{cc}\<%
\\
%
\\
%
\\
\>\<\end{code}
}


The equality type is an essential part of a type theory. We could define it by using the equivalence relation from the setoid representation of type A. The equivalence relation is trivial since it is proof-irrelevant.

\begin{code}\>\<%
\\
%
\\
\>\AgdaFunction{Rel} \AgdaSymbol{:} \AgdaSymbol{\{}\AgdaBound{Γ} \AgdaSymbol{:} \AgdaFunction{Con}\AgdaSymbol{\}} \AgdaSymbol{→} \AgdaRecord{Ty} \AgdaBound{Γ} \AgdaSymbol{→} \AgdaPrimitiveType{Set₁}\<%
\\
\>\AgdaFunction{Rel} \AgdaSymbol{\{}\AgdaBound{Γ}\AgdaSymbol{\}} \AgdaBound{A} \AgdaSymbol{=} \AgdaRecord{Ty} \AgdaSymbol{(}\AgdaBound{Γ} \AgdaFunction{\&} \AgdaBound{A} \AgdaFunction{\&} \AgdaBound{A} \AgdaFunction{+T} \AgdaBound{A}\AgdaSymbol{)}\<%
\\
%
\\
\>\AgdaFunction{⟦Id⟧} \AgdaSymbol{:} \AgdaSymbol{\{}\AgdaBound{Γ} \AgdaSymbol{:} \AgdaFunction{Con}\AgdaSymbol{\}(}\AgdaBound{A} \AgdaSymbol{:} \AgdaRecord{Ty} \AgdaBound{Γ}\AgdaSymbol{)} \AgdaSymbol{→} \AgdaFunction{Rel} \AgdaBound{A}\<%
\\
\>\AgdaFunction{⟦Id⟧} \AgdaBound{A}\<%
\\
\>[0]\AgdaIndent{3}{}\<[3]%
\>[3]\AgdaSymbol{=} \AgdaKeyword{record} \<[12]%
\>[12]\<%
\\
\>[3]\AgdaIndent{7}{}\<[7]%
\>[7]\AgdaSymbol{\{} \AgdaField{fm} \AgdaSymbol{=} \AgdaSymbol{λ} \AgdaSymbol{\{((}\AgdaBound{x} \AgdaInductiveConstructor{,} \AgdaBound{a}\AgdaSymbol{)} \AgdaInductiveConstructor{,} \AgdaBound{b}\AgdaSymbol{)} \AgdaSymbol{→} \<[34]%
\>[34]\AgdaKeyword{record}\<%
\\
\>[7]\AgdaIndent{9}{}\<[9]%
\>[9]\AgdaSymbol{\{} \AgdaField{Carrier} \AgdaSymbol{=} \AgdaFunction{[} \AgdaFunction{[} \AgdaBound{A} \AgdaFunction{]fm} \AgdaBound{x} \AgdaFunction{]} \AgdaBound{a} \AgdaFunction{≈} \AgdaBound{b}\<%
\\
\>[7]\AgdaIndent{9}{}\<[9]%
\>[9]\AgdaSymbol{;} \AgdaField{\_≈h\_} \AgdaSymbol{=} \AgdaSymbol{λ} \AgdaBound{x₁} \AgdaBound{x₂} \AgdaSymbol{→} \AgdaFunction{⊤'}\<%
\\
\>[7]\AgdaIndent{9}{}\<[9]%
\>[9]\AgdaSymbol{;} \AgdaField{isEquiv} \AgdaSymbol{=} \AgdaKeyword{record}\<%
\\
\>[9]\AgdaIndent{13}{}\<[13]%
\>[13]\AgdaSymbol{\{} \AgdaField{refl} \AgdaSymbol{=} \AgdaSymbol{λ} \AgdaSymbol{\{}\AgdaBound{x₁}\AgdaSymbol{\}} \AgdaSymbol{→} \AgdaInductiveConstructor{tt}\<%
\\
\>[9]\AgdaIndent{13}{}\<[13]%
\>[13]\AgdaSymbol{;} \AgdaField{sym} \AgdaSymbol{=} \AgdaSymbol{λ} \AgdaBound{x₂} \AgdaSymbol{→} \AgdaInductiveConstructor{tt}\<%
\\
\>[9]\AgdaIndent{13}{}\<[13]%
\>[13]\AgdaSymbol{;} \AgdaField{trans} \AgdaSymbol{=} \AgdaSymbol{λ} \AgdaBound{x₂} \AgdaBound{x₃} \AgdaSymbol{→} \AgdaInductiveConstructor{tt}\<%
\\
\>[9]\AgdaIndent{13}{}\<[13]%
\>[13]\AgdaSymbol{\}}\<%
\\
\>[0]\AgdaIndent{9}{}\<[9]%
\>[9]\AgdaSymbol{\}} \AgdaSymbol{\}}\<%
\\
\>[0]\AgdaIndent{7}{}\<[7]%
\>[7]\AgdaSymbol{;} \AgdaField{substT} \AgdaSymbol{=} \AgdaSymbol{λ} \AgdaSymbol{\{((}\AgdaBound{x} \AgdaInductiveConstructor{,} \AgdaBound{a}\AgdaSymbol{)} \AgdaInductiveConstructor{,} \AgdaBound{b}\AgdaSymbol{)} \AgdaBound{x0} \AgdaSymbol{→} \<[40]%
\>[40]\<%
\\
\>[7]\AgdaIndent{15}{}\<[15]%
\>[15]\AgdaFunction{[} \AgdaFunction{[} \AgdaBound{A} \AgdaFunction{]fm} \AgdaSymbol{\_} \AgdaFunction{]trans} \<[34]%
\>[34]\<%
\\
\>[7]\AgdaIndent{15}{}\<[15]%
\>[15]\AgdaSymbol{(}\AgdaFunction{[} \AgdaFunction{[} \AgdaBound{A} \AgdaFunction{]fm} \AgdaSymbol{\_} \AgdaFunction{]sym} \AgdaBound{a}\AgdaSymbol{)} \<[36]%
\>[36]\<%
\\
\>[7]\AgdaIndent{15}{}\<[15]%
\>[15]\AgdaSymbol{(}\AgdaFunction{[} \AgdaFunction{[} \AgdaBound{A} \AgdaFunction{]fm} \AgdaSymbol{\_} \AgdaFunction{]trans} \<[35]%
\>[35]\<%
\\
\>[7]\AgdaIndent{15}{}\<[15]%
\>[15]\AgdaSymbol{(}\AgdaFunction{[} \AgdaBound{A} \AgdaFunction{]subst*} \AgdaSymbol{\_} \AgdaBound{x0}\AgdaSymbol{)} \AgdaBound{b}\AgdaSymbol{)} \<[37]%
\>[37]\<%
\\
\>[7]\AgdaIndent{15}{}\<[15]%
\>[15]\AgdaSymbol{\}}\<%
\\
\>[0]\AgdaIndent{7}{}\<[7]%
\>[7]\AgdaSymbol{;} \AgdaField{subst*} \AgdaSymbol{=} \AgdaSymbol{λ} \AgdaBound{p} \AgdaBound{x₁} \AgdaSymbol{→} \AgdaInductiveConstructor{tt}\<%
\\
\>[0]\AgdaIndent{7}{}\<[7]%
\>[7]\AgdaSymbol{;} \AgdaField{refl*} \AgdaSymbol{=} \AgdaSymbol{λ} \AgdaBound{x} \AgdaBound{a} \AgdaSymbol{→} \AgdaInductiveConstructor{tt}\<%
\\
\>[0]\AgdaIndent{7}{}\<[7]%
\>[7]\AgdaSymbol{;} \AgdaField{trans*} \AgdaSymbol{=} \AgdaSymbol{λ} \AgdaBound{p} \AgdaBound{q} \AgdaBound{a} \AgdaSymbol{→} \AgdaInductiveConstructor{tt} \AgdaSymbol{\}}\<%
\\
%
\\
\>\<\end{code}

The unique inhabitant $refl$ is defined as

\begin{code}\>\<%
\\
%
\\
%
\\
\>\AgdaFunction{cm-refl} \AgdaSymbol{:} \AgdaSymbol{\{}\AgdaBound{Γ} \AgdaSymbol{:} \AgdaFunction{Con}\AgdaSymbol{\}(}\AgdaBound{A} \AgdaSymbol{:} \AgdaRecord{Ty} \AgdaBound{Γ}\AgdaSymbol{)} \AgdaSymbol{→} \AgdaBound{Γ} \AgdaFunction{\&} \AgdaBound{A} \AgdaRecord{⇉} \AgdaSymbol{(}\AgdaBound{Γ} \AgdaFunction{\&} \AgdaBound{A} \AgdaFunction{\&} \AgdaBound{A} \AgdaFunction{+T} \AgdaBound{A}\AgdaSymbol{)}\<%
\\
\>\AgdaFunction{cm-refl} \AgdaBound{A} \AgdaSymbol{=} \AgdaKeyword{record} \AgdaSymbol{\{} \AgdaField{fn} \AgdaSymbol{=} \AgdaSymbol{λ} \AgdaBound{x'} \AgdaSymbol{→} \AgdaBound{x'} \AgdaInductiveConstructor{,} \AgdaFunction{proj₂} \AgdaBound{x'} \<[47]%
\>[47]\<%
\\
\>[7]\AgdaIndent{19}{}\<[19]%
\>[19]\AgdaSymbol{;} \AgdaField{resp} \AgdaSymbol{=} \AgdaSymbol{λ} \AgdaBound{x'} \AgdaSymbol{→} \AgdaBound{x'} \AgdaInductiveConstructor{,} \AgdaFunction{proj₂} \AgdaBound{x'} \AgdaSymbol{\}}\<%
\\
%
\\
\>\AgdaFunction{⟦refl⟧⁰} \AgdaSymbol{:} \AgdaSymbol{\{}\AgdaBound{Γ} \AgdaSymbol{:} \AgdaFunction{Con}\AgdaSymbol{\}(}\AgdaBound{A} \AgdaSymbol{:} \AgdaRecord{Ty} \AgdaBound{Γ}\AgdaSymbol{)} \<[30]%
\>[30]\<%
\\
\>[0]\AgdaIndent{7}{}\<[7]%
\>[7]\AgdaSymbol{→} \AgdaRecord{Tm} \AgdaSymbol{\{}\AgdaBound{Γ} \AgdaFunction{\&} \AgdaBound{A}\AgdaSymbol{\}} \AgdaSymbol{(}\AgdaFunction{⟦Id⟧} \AgdaBound{A}\<%
\\
\>[0]\AgdaIndent{10}{}\<[10]%
\>[10]\AgdaFunction{[} \AgdaFunction{cm-refl} \AgdaBound{A} \AgdaFunction{]T}\AgdaSymbol{)} \<[26]%
\>[26]\<%
\\
\>\AgdaFunction{⟦refl⟧⁰} \AgdaBound{A} \AgdaSymbol{=} \AgdaKeyword{record}\<%
\\
\>[10]\AgdaIndent{11}{}\<[11]%
\>[11]\AgdaSymbol{\{} \AgdaField{tm} \AgdaSymbol{=} \AgdaSymbol{λ} \AgdaSymbol{\{(}\AgdaBound{x} \AgdaInductiveConstructor{,} \AgdaBound{a}\AgdaSymbol{)} \AgdaSymbol{→} \AgdaFunction{[} \AgdaFunction{[} \AgdaBound{A} \AgdaFunction{]fm} \AgdaBound{x} \AgdaFunction{]refl} \AgdaSymbol{\{}\AgdaBound{a}\AgdaSymbol{\}} \AgdaSymbol{\}}\<%
\\
\>[10]\AgdaIndent{11}{}\<[11]%
\>[11]\AgdaSymbol{;} \AgdaField{respt} \AgdaSymbol{=} \AgdaSymbol{λ} \AgdaBound{p} \AgdaSymbol{→} \AgdaInductiveConstructor{tt}\<%
\\
\>[10]\AgdaIndent{11}{}\<[11]%
\>[11]\AgdaSymbol{\}}\<%
\\
%
\\
\>\AgdaFunction{⟦refl⟧} \AgdaSymbol{:} \AgdaSymbol{\{}\AgdaBound{Γ} \AgdaSymbol{:} \AgdaFunction{Con}\AgdaSymbol{\}(}\AgdaBound{A} \AgdaSymbol{:} \AgdaRecord{Ty} \AgdaBound{Γ}\AgdaSymbol{)} \<[29]%
\>[29]\<%
\\
\>[-6]\AgdaIndent{7}{}\<[7]%
\>[7]\AgdaSymbol{→} \AgdaRecord{Tm} \AgdaSymbol{\{}\AgdaBound{Γ}\AgdaSymbol{\}} \AgdaSymbol{(}\AgdaFunction{Π} \AgdaBound{A} \AgdaSymbol{(}\AgdaFunction{⟦Id⟧} \AgdaBound{A} \<[29]%
\>[29]\<%
\\
\>[0]\AgdaIndent{10}{}\<[10]%
\>[10]\AgdaFunction{[} \AgdaFunction{cm-refl} \AgdaBound{A} \AgdaFunction{]T}\AgdaSymbol{)} \AgdaSymbol{)}\<%
\\
\>\AgdaFunction{⟦refl⟧} \AgdaSymbol{\{}\AgdaBound{Γ}\AgdaSymbol{\}} \AgdaBound{A} \AgdaSymbol{=} \<[16]%
\>[16]\AgdaFunction{lam} \AgdaSymbol{\{}\AgdaBound{Γ}\AgdaSymbol{\}} \AgdaSymbol{\{}\AgdaBound{A}\AgdaSymbol{\}} \AgdaSymbol{(}\AgdaFunction{⟦refl⟧⁰} \AgdaBound{A}\AgdaSymbol{)}\<%
\\
%
\\
%
\\
\>\<\end{code}

We have an abstracted $refl$ term as well. Using $\Pi$-types we could define the eliminator for $Id$, but it is more involved.

We have done the basics for category of families of setoids. There are more types can be interpreted in this model so that we could show that it is a valid model for Type Theory. We would like to interpret quotient types in this model by following Hofmann's method in \cite{hof:95:sm} or by ourselves.


\AgdaHide{
\begin{code}\>\<%
\\
\>\AgdaComment{\{-

-- substIn (B : Ty (Γ \& A))

⟦subst⟧⁰ : \{Γ : Con\}(A : Ty Γ)(B : Ty (Γ \& A)) → 
           Tm \{Γ \& A \& (A [ fst\& A ]T) 
           \& (⟦Id⟧ A) \& B [ fst\& (A [ fst\& A ]T) ]T [ fst\& (⟦Id⟧ A) ]T\} 
         (B [ record \{ fn = λ x → (proj₁ (proj₁ (proj₁ (proj₁ x)))) , (proj₂ (proj₁ (proj₁ x))) ; resp = λ x → proj₁ (proj₁ (proj₁ (proj₁ x))) , proj₂ (proj₁ (proj₁ x)) \} ]T)

⟦subst⟧⁰ \{Γ\} A B = record
       \{ tm = λ \{((((x , a) , b) , p) , PA) → [ B ]subst ([ Γ ]refl , [ [ A ]fm \_ ]trans ([ A ]refl* \_ \_) p) PA \}
       ; respt = λ \{((((m , a) , b) , p) , PA) → 
         [ [ B ]fm \_ ]trans 
         ([ B ]trans* \_ \_ \_) 
          ([ [ B ]fm \_ ]trans 
         [ B ]subst-pi 
         ([ [ B ]fm \_ ]trans 
         ([ [ B ]fm \_ ]sym ([ B ]trans* \_ \_ \_))
         ([ B ]subst* \_ PA) )) \}
       \}


-\}}\<%
\\
%
\\
\>\AgdaComment{-- The mechanism used in Martin Hofmann's Paper}\<%
\\
%
\\
\>\AgdaKeyword{record} \AgdaRecord{Prop-Uni} \AgdaSymbol{(}\AgdaBound{Γ} \AgdaSymbol{:} \AgdaFunction{Con}\AgdaSymbol{)} \AgdaSymbol{:} \AgdaPrimitiveType{Set} \AgdaKeyword{where}\<%
\\
\>[0]\AgdaIndent{2}{}\<[2]%
\>[2]\AgdaKeyword{field}\<%
\\
%
\\
\>[0]\AgdaIndent{4}{}\<[4]%
\>[4]\AgdaField{prf} \AgdaSymbol{:} \AgdaRecord{Ty} \AgdaBound{Γ}\<%
\\
\>[0]\AgdaIndent{4}{}\<[4]%
\>[4]\AgdaField{uni} \AgdaSymbol{:} \AgdaSymbol{∀} \AgdaBound{γ} \AgdaBound{x} \AgdaBound{y} \AgdaSymbol{→} \AgdaFunction{[} \AgdaFunction{[} \AgdaBound{prf} \AgdaFunction{]fm} \AgdaBound{γ} \AgdaFunction{]} \AgdaBound{x} \AgdaFunction{≈h} \AgdaBound{y} \AgdaDatatype{≡} \AgdaFunction{⊤'}\<%
\\
\>\AgdaKeyword{open} \AgdaModule{Prop-Uni}\<%
\\
%
\\
\>\AgdaComment{-- Is it correct to write  Tm A → Tm B for dependent types?}\<%
\\
%
\\
%
\\
%
\\
\>\AgdaFunction{Id-is-prop} \AgdaSymbol{:} \AgdaSymbol{\{}\AgdaBound{Γ} \AgdaSymbol{:} \AgdaFunction{Con}\AgdaSymbol{\}(}\AgdaBound{A} \AgdaSymbol{:} \AgdaRecord{Ty} \AgdaBound{Γ}\AgdaSymbol{)} \AgdaSymbol{→} \AgdaRecord{Prop-Uni} \AgdaSymbol{(}\AgdaBound{Γ} \AgdaFunction{\&} \AgdaBound{A} \AgdaFunction{\&} \AgdaSymbol{(}\AgdaBound{A} \AgdaFunction{[} \AgdaFunction{fst\&} \AgdaBound{A} \AgdaFunction{]T}\AgdaSymbol{))}\<%
\\
\>\AgdaFunction{Id-is-prop} \AgdaBound{A} \AgdaSymbol{=} \AgdaKeyword{record} \AgdaSymbol{\{} \AgdaField{prf} \AgdaSymbol{=} \AgdaFunction{⟦Id⟧} \AgdaBound{A} \AgdaSymbol{;} \AgdaField{uni} \AgdaSymbol{=} \AgdaSymbol{λ} \AgdaBound{γ} \AgdaBound{x} \AgdaBound{y} \AgdaSymbol{→} \AgdaInductiveConstructor{PE.refl} \AgdaSymbol{\}}\<%
\\
%
\\
\>\AgdaComment{\{-
record Quo \{Γ : Con\}(A : Ty Γ)(R : Prop-Uni (Γ \& A \& (A [ fst\& \{Γ\} \{A\} ]T))) : Set where
  field
    Q : Ty Γ
    [\_] : Tm A → Tm Q
    Q-elim : ∀ (B : Ty Γ)
                 (M : Tm \{Γ \& A\} (B [ fst\& \{Γ\} \{A\} ]T))
                 (N : Tm Q)
                 (H : Tm \{Γ \& A \& A [ fst\& \{Γ\} \{A\} ]T \& prf R\} (prf (Id-is-prop B) [ fst\& \{Γ \& A \& A [ fst\& \{Γ\} \{A\} ]T\} \{prf R\} ]T)) -- (prf (Id-is-prop (B [ fst\& \{Γ\} \{A\} ]T)))
               → Tm B

-\}}\<%
\\
%
\\
%
\\
%
\\
\>\<\end{code}
}






\addtocontents{toc}{\vspace{2em}} % Add a gap in the Contents, for aesthetics

\backmatter

%----------------------------------------------------------------------------------------
%	BIBLIOGRAPHY
%----------------------------------------------------------------------------------------

\label{Bibliography}

\lhead{\emph{Bibliography}} % Change the page header to say "Bibliography"

\bibliographystyle{plainnat}

%\bibliographystyle{unsrtnat} % Use the "unsrtnat" BibTeX style for formatting the Bibliography

\bibliography{my}


\end{document}
