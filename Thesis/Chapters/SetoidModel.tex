\chapter{the Setoid Model}
% always forgets the "the"


\todo{Before 30th-Dec-2013}


We first work on Altenkirch's Setoid model. We build a category
with families of setoids to accommodate the types theory described in
\cite{alti:lics99}  so that it is possible to define quotient types
following Martin Hofmann's Paper\cite{hof:95:sm}.  In this report we
only present the
necessary part for the Setoid model.



Quotient types are one of the extensional concepts in Type Theory \cite{hof:phd}. To introduce an extensional propositional equality in \itt{}, 
Altenkirch \cite{alti:lics99} proposes an intensional setoid model with a proof-irrelevant universe of
propositions \textbf{Prop} . It is called a setoid model since types are interpreted as setoids which contain a set and an equivalence relation for each of them.
The solution to introduce the extensional equality is an object type theory defined inside the setoid model which serves as the metatheory. He also proved that the extended type theory generated from the metatheory is decidable and adequate, functional extensionality is
inhabited and it permits large elimination (defining a dependent type by recursion). Within this type theory,
introduction of quotient types is straightforward.

This model is different to a setoid model as an E-category, for instance
the one introduced by Hofmann \cite{hofmann1995interpretation} . An E-category is a category equipped with
an equivalence relation for homsets. To distinguish them, we call this
category \textbf{E-setoids}.  All morphisms of \textbf{E-setoids}
gives rise to types and they are cartesian closed, namely it is a a locally
cartesian closed category (LCCC). Not all morphisms in our category of setoids give
rise to types and it is not an LCCC. Every LCCC can serve as a model for categories with
families but not every category with families has to be an
LCCC. 

However this Setoid model is still a model for Type
Theory just like the groupoid model which is a generalisation of it.
To develop this model of type theory in Agda, we have implemented the categories with families of setoids.


\section{Literature review}




\subsection{An implementation of categories with Families in Agda}

Following the work in \cite{alt:99}, we first define a
proof-irrelevant universe of propositions. We name it as \textbf{hProp}
since \textbf{Prop} is a  reserved word which can't be used and
\textbf{hProp} is a notion from Homotopy Type Theory which we will introduce later.

\subsection{hProp}

\AgdaHide{

\begin{code}\>\<%
\\
%
\\
\>\AgdaKeyword{open} \AgdaKeyword{import} \AgdaModule{Level}\<%
\\
\>\AgdaKeyword{open} \AgdaKeyword{import} \AgdaModule{Relation.Binary.PropositionalEquality}\<%
\\
%
\\
\>\AgdaKeyword{module} \AgdaModule{hProp} \AgdaSymbol{(}\AgdaBound{ext} \AgdaSymbol{:} \AgdaFunction{Extensionality} \AgdaPrimitive{zero} \AgdaPrimitive{zero}\AgdaSymbol{)} \AgdaKeyword{where}\<%
\\
%
\\
\>\AgdaKeyword{open} \AgdaKeyword{import} \AgdaModule{Relation.Nullary}\<%
\\
\>\AgdaKeyword{open} \AgdaKeyword{import} \AgdaModule{Data.Unit}\<%
\\
\>\AgdaKeyword{open} \AgdaKeyword{import} \AgdaModule{Data.Empty}\<%
\\
\>\AgdaKeyword{open} \AgdaKeyword{import} \AgdaModule{Data.Nat}\<%
\\
\>\AgdaKeyword{open} \AgdaKeyword{import} \AgdaModule{Data.Product}\<%
\\
%
\\
\>\AgdaKeyword{infixr} \AgdaNumber{2} \_⇒\_\<%
\\
%
\\
\>\AgdaKeyword{infixr} \AgdaNumber{3} \_∧\_\<%
\\
%
\\
%
\\
\>\<\end{code}
}

A proof-irrelvant universe only contains sets with at most one inhabitant. 

\begin{code}\>\<%
\\
%
\\
\>\AgdaKeyword{record} \AgdaRecord{hProp} \AgdaSymbol{:} \AgdaPrimitiveType{Set₁} \AgdaKeyword{where}\<%
\\
\>[0]\AgdaIndent{2}{}\<[2]%
\>[2]\AgdaKeyword{constructor} \AgdaInductiveConstructor{hp}\<%
\\
\>[0]\AgdaIndent{2}{}\<[2]%
\>[2]\AgdaKeyword{field}\<%
\\
\>[2]\AgdaIndent{4}{}\<[4]%
\>[4]\AgdaField{prf} \AgdaSymbol{:} \AgdaPrimitiveType{Set}\<%
\\
\>[2]\AgdaIndent{4}{}\<[4]%
\>[4]\AgdaField{Uni} \AgdaSymbol{:} \AgdaSymbol{\{}\AgdaBound{p} \AgdaBound{q} \AgdaSymbol{:} \AgdaBound{prf}\AgdaSymbol{\}} \AgdaSymbol{→} \AgdaBound{p} \AgdaDatatype{≡} \AgdaBound{q}\<%
\\
%
\\
\>\AgdaKeyword{open} \AgdaModule{hProp} \AgdaKeyword{public} \AgdaKeyword{renaming} \AgdaSymbol{(}prf \AgdaSymbol{to} <\_>\AgdaSymbol{)}\<%
\\
%
\\
\>\<\end{code}

We can extract the proof of any propostion $A : hProp$ by using $<>$ and there is always a proof that all inhabitants of it are the same, in other words, if there is any proof of it, the proof is unique. This is not exactly the same as the $Prop$ universe in Altenkirch's approach which is judgemental. It is just a judgement whether a set behaves like a $Proposition$. The $hProp$ we define above is propositional since we can extract the proof of uniqueness.

We would like to have some basic propositions $\top$ and $\bot$. To distinguish them with the ones for non-proof irrelevant propositions which are already available in Agda library, we add a prime to all similar symbols.

\begin{code}\>\<%
\\
%
\\
\>\AgdaFunction{⊤'} \AgdaSymbol{:} \AgdaRecord{hProp}\<%
\\
\>\AgdaFunction{⊤'} \AgdaSymbol{=} \AgdaInductiveConstructor{hp} \AgdaRecord{⊤} \AgdaInductiveConstructor{refl}\<%
\\
%
\\
\>\AgdaFunction{⊥'} \AgdaSymbol{:} \AgdaRecord{hProp}\<%
\\
\>\AgdaFunction{⊥'} \AgdaSymbol{=} \AgdaInductiveConstructor{hp} \AgdaDatatype{⊥} \AgdaSymbol{(λ} \AgdaSymbol{\{}\AgdaBound{p}\AgdaSymbol{\}} \AgdaSymbol{→} \AgdaFunction{⊥-elim} \AgdaBound{p}\AgdaSymbol{)}\<%
\\
%
\\
\>\<\end{code}

We also want the universal and existential quantifier for $hProp$, namely it is closed under $\Pi$-types and $\Sigma$-types.
The universal quantifier of $hProp$ can be axiomitised but we decide to explicitly state that we 
require the functional extensionality to use this module. The reason is that functional extensionality is actually equivalent to the closure under $\Pi$-types.

\begin{code}\>\<%
\\
%
\\
\>\AgdaFunction{∀'} \AgdaSymbol{:} \AgdaSymbol{(}\AgdaBound{A} \AgdaSymbol{:} \AgdaPrimitiveType{Set}\AgdaSymbol{)(}\AgdaBound{P} \AgdaSymbol{:} \AgdaBound{A} \AgdaSymbol{→} \AgdaRecord{hProp}\AgdaSymbol{)} \AgdaSymbol{→} \AgdaRecord{hProp}\<%
\\
\>\AgdaFunction{∀'} \AgdaBound{A} \AgdaBound{P} \AgdaSymbol{=} \AgdaInductiveConstructor{hp} \AgdaSymbol{((}\AgdaBound{x} \AgdaSymbol{:} \AgdaBound{A}\AgdaSymbol{)} \AgdaSymbol{→} \AgdaFunction{<} \AgdaBound{P} \AgdaBound{x} \AgdaFunction{>}\AgdaSymbol{)} \AgdaSymbol{(}\AgdaBound{ext} \AgdaSymbol{(λ} \AgdaBound{x} \AgdaSymbol{→} \AgdaFunction{Uni} \AgdaSymbol{(}\AgdaBound{P} \AgdaBound{x}\AgdaSymbol{)))}\<%
\\
%
\\
\>\<\end{code}


\AgdaHide{
\begin{code}\>\<%
\\
%
\\
\>\AgdaFunction{sig-eq} \AgdaSymbol{:} \AgdaSymbol{\{}\AgdaBound{A} \AgdaSymbol{:} \AgdaPrimitiveType{Set}\AgdaSymbol{\}\{}\AgdaBound{B} \AgdaSymbol{:} \AgdaBound{A} \AgdaSymbol{→} \AgdaPrimitiveType{Set}\AgdaSymbol{\}\{}\AgdaBound{a} \AgdaBound{b} \AgdaSymbol{:} \AgdaBound{A}\AgdaSymbol{\}} \AgdaSymbol{→} \<[43]%
\>[43]\<%
\\
\>[4]\AgdaIndent{9}{}\<[9]%
\>[9]\AgdaSymbol{(}\AgdaBound{p} \AgdaSymbol{:} \AgdaBound{a} \AgdaDatatype{≡} \AgdaBound{b}\AgdaSymbol{)} \AgdaSymbol{→} \<[23]%
\>[23]\<%
\\
\>[4]\AgdaIndent{9}{}\<[9]%
\>[9]\AgdaSymbol{\{}\AgdaBound{c} \AgdaSymbol{:} \AgdaBound{B} \AgdaBound{a}\AgdaSymbol{\}\{}\AgdaBound{d} \AgdaSymbol{:} \AgdaBound{B} \AgdaBound{b}\AgdaSymbol{\}} \AgdaSymbol{→} \<[30]%
\>[30]\<%
\\
\>[4]\AgdaIndent{9}{}\<[9]%
\>[9]\AgdaSymbol{(}\AgdaFunction{subst} \AgdaSymbol{(λ} \AgdaBound{x} \AgdaSymbol{→} \AgdaBound{B} \AgdaBound{x}\AgdaSymbol{)} \AgdaBound{p} \AgdaBound{c} \AgdaDatatype{≡} \AgdaBound{d}\AgdaSymbol{)} \<[37]%
\>[37]\<%
\\
\>[4]\AgdaIndent{9}{}\<[9]%
\>[9]\AgdaSymbol{→} \AgdaDatatype{\_≡\_} \AgdaSymbol{\{\_\}} \AgdaSymbol{\{}\AgdaRecord{Σ} \AgdaBound{A} \AgdaBound{B}\AgdaSymbol{\}} \AgdaSymbol{(}\AgdaBound{a} \AgdaInductiveConstructor{,} \AgdaBound{c}\AgdaSymbol{)} \AgdaSymbol{(}\AgdaBound{b} \AgdaInductiveConstructor{,} \AgdaBound{d}\AgdaSymbol{)}\<%
\\
\>\AgdaFunction{sig-eq} \AgdaInductiveConstructor{refl} \AgdaInductiveConstructor{refl} \AgdaSymbol{=} \AgdaInductiveConstructor{refl}\<%
\\
%
\\
\>\<\end{code}
}

\begin{code}\>\<%
\\
%
\\
%
\\
\>\AgdaFunction{Σ'} \AgdaSymbol{:} \AgdaSymbol{(}\AgdaBound{P} \AgdaSymbol{:} \AgdaRecord{hProp}\AgdaSymbol{)(}\AgdaBound{Q} \AgdaSymbol{:} \AgdaFunction{<} \AgdaBound{P} \AgdaFunction{>} \AgdaSymbol{→} \AgdaRecord{hProp}\AgdaSymbol{)} \AgdaSymbol{→} \AgdaRecord{hProp}\<%
\\
\>\AgdaFunction{Σ'} \AgdaBound{P} \AgdaBound{Q} \AgdaSymbol{=} \AgdaInductiveConstructor{hp} \AgdaSymbol{(}\AgdaRecord{Σ} \AgdaFunction{<} \AgdaBound{P} \AgdaFunction{>} \AgdaSymbol{(λ} \AgdaBound{x} \AgdaSymbol{→} \AgdaFunction{<} \AgdaBound{Q} \AgdaBound{x} \AgdaFunction{>}\AgdaSymbol{))} \<[38]%
\>[38]\<%
\\
\>[9]\AgdaIndent{12}{}\<[12]%
\>[12]\AgdaSymbol{(λ} \AgdaSymbol{\{}\AgdaBound{p}\AgdaSymbol{\}} \AgdaSymbol{\{}\AgdaBound{q}\AgdaSymbol{\}} \AgdaSymbol{→} \<[25]%
\>[25]\<%
\\
\>[9]\AgdaIndent{12}{}\<[12]%
\>[12]\AgdaFunction{sig-eq} \AgdaSymbol{(}\AgdaFunction{Uni} \AgdaBound{P}\AgdaSymbol{)} \AgdaSymbol{(}\AgdaFunction{Uni} \AgdaSymbol{(}\AgdaBound{Q} \AgdaSymbol{(}\AgdaFunction{proj₁} \AgdaBound{q}\AgdaSymbol{))))}\<%
\\
%
\\
\>\<\end{code}

Implication and conjuction which are independent ones of them follow simply.

\begin{code}\>\<%
\\
%
\\
\>\AgdaFunction{\_⇒\_} \AgdaSymbol{:} \AgdaSymbol{(}\AgdaBound{P} \AgdaBound{Q} \AgdaSymbol{:} \AgdaRecord{hProp}\AgdaSymbol{)} \AgdaSymbol{→} \AgdaRecord{hProp}\<%
\\
\>\AgdaBound{P} \AgdaFunction{⇒} \AgdaBound{Q} \AgdaSymbol{=} \<[9]%
\>[9]\AgdaFunction{∀'} \AgdaFunction{<} \AgdaBound{P} \AgdaFunction{>} \AgdaSymbol{(λ} \AgdaBound{\_} \AgdaSymbol{→} \AgdaBound{Q}\AgdaSymbol{)}\<%
\\
%
\\
\>\AgdaFunction{\_∧\_} \AgdaSymbol{:} \AgdaSymbol{(}\AgdaBound{P} \AgdaBound{Q} \AgdaSymbol{:} \AgdaRecord{hProp}\AgdaSymbol{)} \AgdaSymbol{→} \AgdaRecord{hProp}\<%
\\
\>\AgdaBound{P} \AgdaFunction{∧} \AgdaBound{Q} \AgdaSymbol{=} \AgdaFunction{Σ'} \AgdaBound{P} \AgdaSymbol{(λ} \AgdaBound{\_} \AgdaSymbol{→} \AgdaBound{Q}\AgdaSymbol{)}\<%
\\
%
\\
\>\<\end{code}

\AgdaHide{
\begin{code}\>\<%
\\
\>\AgdaKeyword{syntax} ∀' A \AgdaSymbol{(λ} x \AgdaSymbol{→} B\AgdaSymbol{)} \AgdaSymbol{=} ∀'[ x ∶ A ] B


\AgdaKeyword{syntax} Σ' A \AgdaSymbol{(λ} x \AgdaSymbol{→} B\AgdaSymbol{)} \AgdaSymbol{=} Σ'[ x ∶ A ] B

\<\end{code}
}

As long as we have implication and conjuction, more operators on proposition can be defined, for instances negation and logical equivalence.

\begin{code}\>\<%
\\
%
\\
\>\AgdaFunction{¬} \AgdaSymbol{:} \AgdaRecord{hProp} \AgdaSymbol{→} \AgdaRecord{hProp}\<%
\\
\>\AgdaFunction{¬} \AgdaBound{P} \AgdaSymbol{=} \AgdaBound{P} \AgdaFunction{⇒} \AgdaFunction{⊥'} \<[13]%
\>[13]\<%
\\
%
\\
\>\AgdaFunction{\_↔\_} \<[6]%
\>[6]\AgdaSymbol{:} \AgdaSymbol{(}\AgdaBound{P} \AgdaBound{Q} \AgdaSymbol{:} \AgdaRecord{hProp}\AgdaSymbol{)} \AgdaSymbol{→} \AgdaRecord{hProp}\<%
\\
\>\AgdaBound{P} \AgdaFunction{↔} \AgdaBound{Q} \AgdaSymbol{=} \AgdaSymbol{(}\AgdaBound{P} \AgdaFunction{⇒} \AgdaBound{Q}\AgdaSymbol{)} \AgdaFunction{∧} \AgdaSymbol{(}\AgdaBound{Q} \AgdaFunction{⇒} \AgdaBound{P}\AgdaSymbol{)}\<%
\\
%
\\
\>\<\end{code}

\subsection{Category}

To define category of setoids we should define category first.


\AgdaHide{
\begin{code}\>\<%
\\
%
\\
\>\AgdaSymbol{\{-\#} \AgdaKeyword{OPTIONS} --type-in-type \AgdaSymbol{\#-\}}\<%
\\
%
\\
\>\AgdaKeyword{module} \AgdaModule{Category} \AgdaKeyword{where}\<%
\\
%
\\
\>\AgdaKeyword{open} \AgdaKeyword{import} \AgdaModule{Data.Product}\<%
\\
\>\AgdaKeyword{open} \AgdaKeyword{import} \AgdaModule{Relation.Binary.PropositionalEquality}\<%
\\
%
\\
\>\AgdaKeyword{open} \AgdaKeyword{import} \AgdaModule{Level}\<%
\\
%
\\
\>\<\end{code}
}

\AgdaHide{
\begin{code}\>\<%
\\
%
\\
\>\AgdaKeyword{record} \AgdaRecord{IsCategory}\<%
\\
\>[0]\AgdaIndent{2}{}\<[2]%
\>[2]\AgdaSymbol{(}\AgdaBound{obj} \<[12]%
\>[12]\AgdaSymbol{:} \AgdaPrimitiveType{Set}\AgdaSymbol{)}\<%
\\
%
\\
\>[0]\AgdaIndent{2}{}\<[2]%
\>[2]\AgdaSymbol{(}\AgdaBound{hom} \<[12]%
\>[12]\AgdaSymbol{:} \AgdaBound{obj} \AgdaSymbol{→} \AgdaBound{obj} \AgdaSymbol{→} \AgdaPrimitiveType{Set}\AgdaSymbol{)}\<%
\\
%
\\
\>[0]\AgdaIndent{2}{}\<[2]%
\>[2]\AgdaSymbol{(}\AgdaBound{id} \<[12]%
\>[12]\AgdaSymbol{:} \AgdaSymbol{∀} \AgdaBound{α} \AgdaSymbol{→} \AgdaBound{hom} \AgdaBound{α} \AgdaBound{α}\AgdaSymbol{)}\<%
\\
%
\\
\>[0]\AgdaIndent{2}{}\<[2]%
\>[2]\AgdaSymbol{(}\AgdaBound{[\_⇒\_]\_∘\_} \AgdaSymbol{:} \AgdaSymbol{∀} \AgdaBound{α} \AgdaSymbol{\{}\AgdaBound{β}\AgdaSymbol{\}} \AgdaBound{γ}\<%
\\
\>[2]\AgdaIndent{12}{}\<[12]%
\>[12]\AgdaSymbol{→} \AgdaBound{hom} \AgdaBound{β} \AgdaBound{γ}\<%
\\
\>[2]\AgdaIndent{12}{}\<[12]%
\>[12]\AgdaSymbol{→} \AgdaBound{hom} \AgdaBound{α} \AgdaBound{β}\<%
\\
\>[2]\AgdaIndent{12}{}\<[12]%
\>[12]\AgdaSymbol{→} \AgdaBound{hom} \AgdaBound{α} \AgdaBound{γ}\AgdaSymbol{)}\<%
\\
\>[0]\AgdaIndent{2}{}\<[2]%
\>[2]\AgdaSymbol{:} \AgdaPrimitiveType{Set}\<%
\\
\>[0]\AgdaIndent{2}{}\<[2]%
\>[2]\AgdaKeyword{where}\<%
\\
\>[2]\AgdaIndent{4}{}\<[4]%
\>[4]\AgdaKeyword{constructor} \AgdaInductiveConstructor{IsCatC}\<%
\\
\>[2]\AgdaIndent{4}{}\<[4]%
\>[4]\AgdaKeyword{field}\<%
\\
\>[4]\AgdaIndent{6}{}\<[6]%
\>[6]\AgdaField{id₁} \<[11]%
\>[11]\AgdaSymbol{:} \AgdaSymbol{∀} \AgdaBound{α} \AgdaBound{β} \AgdaSymbol{(}\AgdaBound{f} \AgdaSymbol{:} \AgdaBound{hom} \AgdaBound{α} \AgdaBound{β}\AgdaSymbol{)}\<%
\\
\>[6]\AgdaIndent{11}{}\<[11]%
\>[11]\AgdaSymbol{→} \AgdaBound{[} \AgdaBound{α} \AgdaBound{⇒} \AgdaBound{β} \AgdaBound{]} \AgdaBound{f} \AgdaBound{∘} \AgdaSymbol{(}\AgdaBound{id} \AgdaBound{α}\AgdaSymbol{)} \AgdaDatatype{≡} \AgdaBound{f}\<%
\\
%
\\
\>[0]\AgdaIndent{6}{}\<[6]%
\>[6]\AgdaField{id₂} \<[11]%
\>[11]\AgdaSymbol{:} \AgdaSymbol{∀} \AgdaBound{α} \AgdaBound{β} \AgdaSymbol{(}\AgdaBound{f} \AgdaSymbol{:} \AgdaBound{hom} \AgdaBound{α} \AgdaBound{β}\AgdaSymbol{)}\<%
\\
\>[0]\AgdaIndent{11}{}\<[11]%
\>[11]\AgdaSymbol{→} \AgdaBound{[} \AgdaBound{α} \AgdaBound{⇒} \AgdaBound{β} \AgdaBound{]} \AgdaSymbol{(}\AgdaBound{id} \AgdaBound{β}\AgdaSymbol{)} \AgdaBound{∘} \AgdaBound{f} \AgdaDatatype{≡} \AgdaBound{f}\<%
\\
%
\\
\>[0]\AgdaIndent{6}{}\<[6]%
\>[6]\AgdaField{comp} \AgdaSymbol{:} \AgdaSymbol{∀} \AgdaBound{α} \AgdaSymbol{\{}\AgdaBound{β} \AgdaBound{γ}\AgdaSymbol{\}} \AgdaBound{δ} \AgdaSymbol{(}\AgdaBound{f} \AgdaSymbol{:} \AgdaBound{hom} \AgdaBound{α} \AgdaBound{β}\AgdaSymbol{)} \AgdaSymbol{(}\AgdaBound{g} \AgdaSymbol{:} \AgdaBound{hom} \AgdaBound{β} \AgdaBound{γ}\AgdaSymbol{)} \AgdaSymbol{(}\AgdaBound{h} \AgdaSymbol{:} \AgdaBound{hom} \AgdaBound{γ} \AgdaBound{δ}\AgdaSymbol{)}\<%
\\
\>[0]\AgdaIndent{11}{}\<[11]%
\>[11]\AgdaSymbol{→} \AgdaBound{[} \AgdaBound{α} \AgdaBound{⇒} \AgdaBound{δ} \AgdaBound{]} \AgdaBound{[} \AgdaBound{β} \AgdaBound{⇒} \AgdaBound{δ} \AgdaBound{]} \AgdaBound{h} \AgdaBound{∘} \AgdaBound{g} \AgdaBound{∘} \AgdaBound{f} \AgdaDatatype{≡} \AgdaBound{[} \AgdaBound{α} \AgdaBound{⇒} \AgdaBound{δ} \AgdaBound{]} \AgdaBound{h} \AgdaBound{∘} \AgdaSymbol{(}\AgdaBound{[} \AgdaBound{α} \AgdaBound{⇒} \AgdaBound{γ} \AgdaBound{]} \AgdaBound{g} \AgdaBound{∘} \AgdaBound{f}\AgdaSymbol{)}\<%
\\
%
\\
%
\\
\>\<\end{code}
}


\begin{code}\>\<%
\\
\>\AgdaKeyword{record} \AgdaRecord{Category} \AgdaSymbol{:} \AgdaPrimitiveType{Set} \AgdaKeyword{where}\<%
\\
\>[0]\AgdaIndent{2}{}\<[2]%
\>[2]\AgdaKeyword{constructor} \AgdaInductiveConstructor{CatC}\<%
\\
\>[0]\AgdaIndent{2}{}\<[2]%
\>[2]\AgdaKeyword{field}\<%
\\
\>[2]\AgdaIndent{4}{}\<[4]%
\>[4]\AgdaField{obj} \<[15]%
\>[15]\AgdaSymbol{:} \AgdaPrimitiveType{Set}\<%
\\
%
\\
\>[2]\AgdaIndent{4}{}\<[4]%
\>[4]\AgdaField{hom} \<[15]%
\>[15]\AgdaSymbol{:} \AgdaBound{obj} \AgdaSymbol{→} \AgdaBound{obj} \AgdaSymbol{→} \AgdaPrimitiveType{Set}\<%
\\
%
\\
\>[2]\AgdaIndent{4}{}\<[4]%
\>[4]\AgdaField{id} \<[15]%
\>[15]\AgdaSymbol{:} \AgdaSymbol{∀} \AgdaBound{α}\<%
\\
\>[4]\AgdaIndent{15}{}\<[15]%
\>[15]\AgdaSymbol{→} \AgdaBound{hom} \AgdaBound{α} \AgdaBound{α}\<%
\\
%
\\
\>[0]\AgdaIndent{4}{}\<[4]%
\>[4]\AgdaField{[\_⇒\_]\_∘\_} \<[15]%
\>[15]\AgdaSymbol{:} \AgdaSymbol{∀} \AgdaBound{α} \AgdaSymbol{\{}\AgdaBound{β}\AgdaSymbol{\}} \AgdaBound{γ}\<%
\\
\>[0]\AgdaIndent{15}{}\<[15]%
\>[15]\AgdaSymbol{→} \AgdaBound{hom} \AgdaBound{β} \AgdaBound{γ}\<%
\\
\>[0]\AgdaIndent{15}{}\<[15]%
\>[15]\AgdaSymbol{→} \AgdaBound{hom} \AgdaBound{α} \AgdaBound{β}\<%
\\
\>[0]\AgdaIndent{15}{}\<[15]%
\>[15]\AgdaSymbol{→} \AgdaBound{hom} \AgdaBound{α} \AgdaBound{γ}\<%
\\
%
\\
\>[0]\AgdaIndent{4}{}\<[4]%
\>[4]\AgdaField{isCategory} \AgdaSymbol{:} \AgdaRecord{IsCategory} \AgdaBound{obj} \AgdaBound{hom} \AgdaBound{id} \AgdaBound{[\_⇒\_]\_∘\_}\<%
\\
%
\\
%
\\
\>\<\end{code}

\AgdaHide{
\begin{code}\>\<%
\\
%
\\
\>[0]\AgdaIndent{2}{}\<[2]%
\>[2]\AgdaKeyword{open} \AgdaModule{IsCategory} \AgdaKeyword{public}\<%
\\
%
\\
\>\<\end{code}
}

$isCategory$ contains all the laws for this structure to be a category, for instance the
associativity laws for composition.

\subsection{Category of setoids}



\AgdaHide{

\begin{code}\>\<%
\\
%
\\
%
\\
\>\AgdaSymbol{\{-\#} \AgdaKeyword{OPTIONS} --type-in-type \AgdaSymbol{\#-\}}\<%
\\
%
\\
\>\AgdaKeyword{open} \AgdaKeyword{import} \AgdaModule{Level}\<%
\\
\>\AgdaKeyword{open} \AgdaKeyword{import} \AgdaModule{Relation.Binary.PropositionalEquality} \AgdaSymbol{as} \AgdaModule{PE} \AgdaKeyword{hiding} \AgdaSymbol{(}refl\AgdaSymbol{;} sym \AgdaSymbol{;} trans\AgdaSymbol{;} isEquivalence\AgdaSymbol{)}\<%
\\
%
\\
\>\AgdaKeyword{module} \AgdaModule{CategoryOfSetoid} \<[25]%
\>[25]\AgdaSymbol{(}\AgdaBound{ext} \AgdaSymbol{:} \AgdaFunction{Extensionality} \AgdaPrimitive{zero} \AgdaPrimitive{zero}\AgdaSymbol{)} \AgdaKeyword{where}\<%
\\
%
\\
\>\AgdaKeyword{open} \AgdaKeyword{import} \AgdaModule{Cats.Category}\<%
\\
\>\AgdaKeyword{open} \AgdaKeyword{import} \AgdaModule{Function}\<%
\\
\>\AgdaKeyword{open} \AgdaKeyword{import} \AgdaModule{Relation.Binary.Core} \AgdaKeyword{using} \AgdaSymbol{(}\_≡\_\AgdaSymbol{)} \AgdaKeyword{renaming} \AgdaSymbol{(}\_⇒\_ \AgdaSymbol{to} \_⇒'\_\AgdaSymbol{)}\<%
\\
\>\AgdaKeyword{open} \AgdaKeyword{import} \AgdaModule{Data.Unit}\<%
\\
\>\AgdaKeyword{open} \AgdaKeyword{import} \AgdaModule{Data.Empty}\<%
\\
\>\AgdaKeyword{import} \AgdaModule{hProp}\<%
\\
\>\AgdaKeyword{open} \AgdaKeyword{module} \AgdaModule{hpx} \AgdaSymbol{=} \AgdaModule{hProp} \AgdaBound{ext}\<%
\\
%
\\
%
\\
\>\AgdaComment{-- Arrow between HSetoid}\<%
\\
%
\\
\>\AgdaKeyword{infix} \AgdaNumber{5} \_⇉\_\<%
\\
%
\\
\>\AgdaComment{-- composition}\<%
\\
%
\\
\>\AgdaKeyword{infixl} \AgdaNumber{5} \_∘c\_\<%
\\
%
\\
\>\<\end{code}
}

Then we could define setoids using \textbf{hProp}. An equivalence relation has three properties reflexivity, symmetry and transitivity. Since we have $refl$ here, we call the reflexivity for propositional equality from the library with prefix as $PE.refl$. 

\begin{code}\>\<%
\\
%
\\
\>\AgdaKeyword{record} \AgdaRecord{ishEquivalence} \AgdaSymbol{\{}\AgdaBound{A} \AgdaSymbol{:} \AgdaPrimitiveType{Set}\AgdaSymbol{\}(}\AgdaBound{\_≈h\_} \AgdaSymbol{:} \AgdaBound{A} \AgdaSymbol{→} \AgdaBound{A} \AgdaSymbol{→} \AgdaRecord{hProp}\AgdaSymbol{)} \AgdaSymbol{:} \AgdaPrimitiveType{Set₁} \AgdaKeyword{where}\<%
\\
\>[0]\AgdaIndent{2}{}\<[2]%
\>[2]\AgdaKeyword{constructor} \AgdaInductiveConstructor{\_,\_,\_}\<%
\\
\>[0]\AgdaIndent{2}{}\<[2]%
\>[2]\AgdaKeyword{field}\<%
\\
\>[2]\AgdaIndent{4}{}\<[4]%
\>[4]\AgdaField{refl} \<[12]%
\>[12]\AgdaSymbol{:} \AgdaSymbol{\{}\AgdaBound{x} \AgdaSymbol{:} \AgdaBound{A}\AgdaSymbol{\}} \AgdaSymbol{→} \AgdaFunction{<} \AgdaBound{x} \AgdaBound{≈h} \AgdaBound{x} \AgdaFunction{>}\<%
\\
\>[2]\AgdaIndent{4}{}\<[4]%
\>[4]\AgdaField{sym} \<[12]%
\>[12]\AgdaSymbol{:} \AgdaSymbol{\{}\AgdaBound{x} \AgdaBound{y} \AgdaSymbol{:} \AgdaBound{A}\AgdaSymbol{\}} \AgdaSymbol{→} \AgdaFunction{<} \AgdaBound{x} \AgdaBound{≈h} \AgdaBound{y} \AgdaFunction{>} \AgdaSymbol{→} \AgdaFunction{<} \AgdaBound{y} \AgdaBound{≈h} \AgdaBound{x} \AgdaFunction{>}\<%
\\
\>[2]\AgdaIndent{4}{}\<[4]%
\>[4]\AgdaField{trans} \<[12]%
\>[12]\AgdaSymbol{:} \AgdaSymbol{\{}\AgdaBound{x} \AgdaBound{y} \AgdaBound{z} \AgdaSymbol{:} \AgdaBound{A}\AgdaSymbol{\}} \AgdaSymbol{→} \AgdaFunction{<} \AgdaBound{x} \AgdaBound{≈h} \AgdaBound{y} \AgdaFunction{>} \AgdaSymbol{→} \AgdaFunction{<} \AgdaBound{y} \AgdaBound{≈h} \AgdaBound{z} \AgdaFunction{>} \AgdaSymbol{→} \AgdaFunction{<} \AgdaBound{x} \AgdaBound{≈h} \AgdaBound{z} \AgdaFunction{>}\<%
\\
%
\\
\>\<\end{code}

Here we use \textbf{hSetoid} as the name because \textbf{Setoid} is
already used for non-proof-irrelvant setoids in the library.
For each setoid, we have a carrier type and an equivalence relation.

\begin{code}\>\<%
\\
\>\AgdaKeyword{record} \AgdaRecord{hSetoid} \AgdaSymbol{:} \AgdaPrimitiveType{Set₁} \AgdaKeyword{where}\<%
\\
\>[0]\AgdaIndent{2}{}\<[2]%
\>[2]\AgdaKeyword{constructor} \AgdaInductiveConstructor{\_,\_,\_}\<%
\\
\>[0]\AgdaIndent{2}{}\<[2]%
\>[2]\AgdaKeyword{infix} \AgdaNumber{4} \_≈h\_ \_≈\_\<%
\\
\>[0]\AgdaIndent{2}{}\<[2]%
\>[2]\AgdaKeyword{field}\<%
\\
\>[2]\AgdaIndent{4}{}\<[4]%
\>[4]\AgdaField{Carrier} \AgdaSymbol{:} \AgdaPrimitiveType{Set}\<%
\\
\>[2]\AgdaIndent{4}{}\<[4]%
\>[4]\AgdaField{\_≈h\_} \<[12]%
\>[12]\AgdaSymbol{:} \AgdaBound{Carrier} \AgdaSymbol{→} \AgdaBound{Carrier} \AgdaSymbol{→} \AgdaRecord{hProp}\<%
\\
\>[2]\AgdaIndent{4}{}\<[4]%
\>[4]\AgdaField{isEquiv} \AgdaSymbol{:} \AgdaRecord{ishEquivalence} \AgdaBound{\_≈h\_}\<%
\\
%
\\
%
\\
\>\<\end{code}

\AgdaHide{
\begin{code}\>\<%
\\
%
\\
\>[0]\AgdaIndent{2}{}\<[2]%
\>[2]\AgdaKeyword{open} \<[8]%
\>[8]\AgdaModule{ishEquivalence} \AgdaFunction{isEquiv} \AgdaKeyword{public}\<%
\\
%
\\
\>[0]\AgdaIndent{2}{}\<[2]%
\>[2]\AgdaFunction{\_≈\_} \AgdaSymbol{:} \AgdaFunction{Carrier} \AgdaSymbol{→} \AgdaFunction{Carrier} \AgdaSymbol{→} \AgdaPrimitiveType{Set}\<%
\\
\>[0]\AgdaIndent{2}{}\<[2]%
\>[2]\AgdaBound{a} \AgdaFunction{≈} \AgdaBound{b} \AgdaSymbol{=} \AgdaFunction{<} \AgdaBound{a} \AgdaFunction{≈h} \AgdaBound{b} \AgdaFunction{>}\<%
\\
\>[0]\AgdaIndent{1}{}\<[1]%
\>[1]\<%
\\
\>[0]\AgdaIndent{2}{}\<[2]%
\>[2]\AgdaFunction{PI} \AgdaSymbol{:} \AgdaSymbol{\{}\AgdaBound{x} \AgdaBound{y} \AgdaSymbol{:} \AgdaFunction{Carrier}\AgdaSymbol{\}\{}\AgdaBound{B} \AgdaSymbol{:} \AgdaPrimitiveType{Set}\AgdaSymbol{\}}\<%
\\
\>[2]\AgdaIndent{7}{}\<[7]%
\>[7]\AgdaSymbol{(}\AgdaBound{A} \AgdaSymbol{:} \AgdaBound{x} \AgdaFunction{≈} \AgdaBound{y} \AgdaSymbol{→} \AgdaBound{B}\AgdaSymbol{)\{}\AgdaBound{p} \AgdaBound{q} \AgdaSymbol{:} \AgdaBound{x} \AgdaFunction{≈} \AgdaBound{y}\AgdaSymbol{\}} \<[36]%
\>[36]\<%
\\
\>[0]\AgdaIndent{5}{}\<[5]%
\>[5]\AgdaSymbol{→} \AgdaBound{A} \AgdaBound{p} \AgdaDatatype{≡} \AgdaBound{A} \AgdaBound{q}\<%
\\
\>[0]\AgdaIndent{2}{}\<[2]%
\>[2]\AgdaFunction{PI} \AgdaSymbol{\{}\AgdaBound{x}\AgdaSymbol{\}} \AgdaSymbol{\{}\AgdaBound{y}\AgdaSymbol{\}} \AgdaBound{A} \AgdaSymbol{\{}\AgdaBound{p}\AgdaSymbol{\}} \AgdaSymbol{\{}\AgdaBound{q}\AgdaSymbol{\}} \AgdaKeyword{with} \AgdaFunction{Uni} \AgdaSymbol{(}\AgdaBound{x} \AgdaFunction{≈h} \AgdaBound{y}\AgdaSymbol{)} \AgdaSymbol{\{}\AgdaBound{p}\AgdaSymbol{\}} \AgdaSymbol{\{}\AgdaBound{q}\AgdaSymbol{\}}\<%
\\
\>[0]\AgdaIndent{2}{}\<[2]%
\>[2]\AgdaFunction{PI} \AgdaBound{A} \AgdaSymbol{|} \AgdaInductiveConstructor{PE.refl} \AgdaSymbol{=} \AgdaInductiveConstructor{PE.refl}\<%
\\
%
\\
\>[0]\AgdaIndent{2}{}\<[2]%
\>[2]\AgdaFunction{reflexive} \AgdaSymbol{:} \AgdaDatatype{\_≡\_} \AgdaFunction{⇒'} \AgdaFunction{\_≈\_}\<%
\\
\>[0]\AgdaIndent{2}{}\<[2]%
\>[2]\AgdaFunction{reflexive} \AgdaInductiveConstructor{PE.refl} \AgdaSymbol{=} \AgdaFunction{refl}\<%
\\
%
\\
\>\AgdaKeyword{open} \AgdaModule{hSetoid} \AgdaKeyword{public} \AgdaKeyword{renaming} \AgdaSymbol{(}refl \AgdaSymbol{to} [\_]refl\AgdaSymbol{;}
     sym \AgdaSymbol{to} [\_]sym\AgdaSymbol{;} \_≈\_ \AgdaSymbol{to} [\_]\_≈\_ \AgdaSymbol{;} \_≈h\_ \AgdaSymbol{to} [\_]\_≈h\_ \AgdaSymbol{;}
     Carrier \AgdaSymbol{to} ∣\_∣ \AgdaSymbol{;} trans \AgdaSymbol{to} [\_]trans\AgdaSymbol{)}\<%
\\
%
\\
%
\\
\>\AgdaFunction{[\_]uip} \AgdaSymbol{:} \AgdaSymbol{∀(}\AgdaBound{Γ} \AgdaSymbol{:} \AgdaRecord{hSetoid}\AgdaSymbol{)\{}\AgdaBound{a} \AgdaBound{b} \AgdaSymbol{:} \AgdaFunction{∣} \AgdaBound{Γ} \AgdaFunction{∣}\AgdaSymbol{\}\{}\AgdaBound{p} \AgdaBound{q} \AgdaSymbol{:} \AgdaFunction{[} \AgdaBound{Γ} \AgdaFunction{]} \AgdaBound{a} \AgdaFunction{≈} \AgdaBound{b}\AgdaSymbol{\}} \AgdaSymbol{→} \AgdaBound{p} \AgdaDatatype{≡} \AgdaBound{q}\<%
\\
\>\AgdaFunction{[} \AgdaBound{Γ} \AgdaFunction{]uip} \AgdaSymbol{\{}\AgdaBound{a}\AgdaSymbol{\}} \AgdaSymbol{\{}\AgdaBound{b}\AgdaSymbol{\}} \AgdaSymbol{=} \AgdaFunction{Uni} \AgdaSymbol{(}\AgdaFunction{[} \AgdaBound{Γ} \AgdaFunction{]} \AgdaBound{a} \AgdaFunction{≈h} \AgdaBound{b}\AgdaSymbol{)}\<%
\\
%
\\
%
\\
\>\<\end{code}
}

A morphism in this category is a function of the underlying sets which respects the equivalence relation. We don't identify the extensional equal functions in the homsets as in \textbf{E-setoids}.

\begin{code}\>\<%
\\
%
\\
\>\AgdaKeyword{record} \AgdaRecord{\_⇉\_} \AgdaSymbol{(}\AgdaBound{A} \AgdaBound{B} \AgdaSymbol{:} \AgdaRecord{hSetoid}\AgdaSymbol{)} \AgdaSymbol{:} \AgdaPrimitiveType{Set₁} \AgdaKeyword{where}\<%
\\
\>[0]\AgdaIndent{2}{}\<[2]%
\>[2]\AgdaKeyword{constructor} \AgdaInductiveConstructor{fn:\_resp:\_}\<%
\\
\>[0]\AgdaIndent{2}{}\<[2]%
\>[2]\AgdaKeyword{field}\<%
\\
\>[2]\AgdaIndent{4}{}\<[4]%
\>[4]\AgdaField{fn} \<[9]%
\>[9]\AgdaSymbol{:} \AgdaFunction{∣} \AgdaBound{A} \AgdaFunction{∣} \AgdaSymbol{→} \AgdaFunction{∣} \AgdaBound{B} \AgdaFunction{∣}\<%
\\
\>[2]\AgdaIndent{4}{}\<[4]%
\>[4]\AgdaField{resp} \AgdaSymbol{:} \AgdaSymbol{\{}\AgdaBound{x} \AgdaBound{y} \AgdaSymbol{:} \AgdaFunction{∣} \AgdaBound{A} \AgdaFunction{∣}\AgdaSymbol{\}} \AgdaSymbol{→} \<[27]%
\>[27]\<%
\\
\>[4]\AgdaIndent{11}{}\<[11]%
\>[11]\AgdaFunction{[} \AgdaBound{A} \AgdaFunction{]} \AgdaBound{x} \AgdaFunction{≈} \AgdaBound{y} \AgdaSymbol{→} \<[25]%
\>[25]\<%
\\
\>[4]\AgdaIndent{11}{}\<[11]%
\>[11]\AgdaFunction{[} \AgdaBound{B} \AgdaFunction{]} \AgdaBound{fn} \AgdaBound{x} \AgdaFunction{≈} \AgdaBound{fn} \AgdaBound{y}\<%
\\
%
\\
\>\<\end{code}

\AgdaHide{
\begin{code}\>\<%
\\
%
\\
\>\AgdaKeyword{open} \AgdaModule{\_⇉\_} \AgdaKeyword{public} \AgdaKeyword{renaming} \AgdaSymbol{(}fn \AgdaSymbol{to} [\_]fn \AgdaSymbol{;} resp \AgdaSymbol{to} [\_]resp\AgdaSymbol{)}\<%
\\
%
\\
\>\<\end{code}
}


The definitions of identity morphism and composition are straightforward and the categorical laws hold trivially as follows.

\begin{code}\>\<%
\\
%
\\
\>\AgdaFunction{id'} \AgdaSymbol{:} \AgdaSymbol{\{}\AgdaBound{Γ} \AgdaSymbol{:} \AgdaRecord{hSetoid}\AgdaSymbol{\}} \AgdaSymbol{→} \AgdaBound{Γ} \AgdaRecord{⇉} \AgdaBound{Γ} \<[28]%
\>[28]\<%
\\
\>\AgdaFunction{id'} \AgdaSymbol{=} \AgdaKeyword{record} \AgdaSymbol{\{} \AgdaField{fn} \AgdaSymbol{=} \AgdaFunction{id}\AgdaSymbol{;} \AgdaField{resp} \AgdaSymbol{=} \AgdaFunction{id}\AgdaSymbol{\}}\<%
\\
%
\\
\>\AgdaFunction{\_∘c\_} \AgdaSymbol{:} \AgdaSymbol{∀\{}\AgdaBound{Γ} \AgdaBound{Δ} \AgdaBound{Z}\AgdaSymbol{\}} \AgdaSymbol{→} \AgdaBound{Δ} \AgdaRecord{⇉} \AgdaBound{Z} \AgdaSymbol{→} \AgdaBound{Γ} \AgdaRecord{⇉} \AgdaBound{Δ} \AgdaSymbol{→} \AgdaBound{Γ} \AgdaRecord{⇉} \AgdaBound{Z}\<%
\\
\>\AgdaBound{yz} \AgdaFunction{∘c} \AgdaBound{xy} \AgdaSymbol{=} \AgdaKeyword{record} \<[18]%
\>[18]\<%
\\
\>[4]\AgdaIndent{11}{}\<[11]%
\>[11]\AgdaSymbol{\{} \AgdaField{fn} \AgdaSymbol{=} \AgdaFunction{[} \AgdaBound{yz} \AgdaFunction{]fn} \AgdaFunction{∘} \AgdaFunction{[} \AgdaBound{xy} \AgdaFunction{]fn}\<%
\\
\>[4]\AgdaIndent{11}{}\<[11]%
\>[11]\AgdaSymbol{;} \AgdaField{resp} \AgdaSymbol{=} \AgdaFunction{[} \AgdaBound{yz} \AgdaFunction{]resp} \AgdaFunction{∘} \AgdaFunction{[} \AgdaBound{xy} \AgdaFunction{]resp}\<%
\\
\>[4]\AgdaIndent{11}{}\<[11]%
\>[11]\AgdaSymbol{\}}\<%
\\
%
\\
\>\AgdaFunction{id₁} \AgdaSymbol{:} \AgdaSymbol{∀} \AgdaBound{Γ} \AgdaBound{Δ} \AgdaSymbol{(}\AgdaBound{ch} \AgdaSymbol{:} \AgdaBound{Γ} \AgdaRecord{⇉} \AgdaBound{Δ}\AgdaSymbol{)} \AgdaSymbol{→} \AgdaBound{ch} \AgdaFunction{∘c} \AgdaFunction{id'} \AgdaDatatype{≡} \AgdaBound{ch}\<%
\\
\>\AgdaFunction{id₁} \AgdaSymbol{\_} \AgdaSymbol{\_} \AgdaBound{ch} \AgdaSymbol{=} \AgdaInductiveConstructor{PE.refl}\<%
\\
%
\\
\>\AgdaFunction{id₂} \AgdaSymbol{:} \AgdaSymbol{∀} \AgdaBound{Γ} \AgdaBound{Δ} \AgdaSymbol{(}\AgdaBound{ch} \AgdaSymbol{:} \AgdaBound{Γ} \AgdaRecord{⇉} \AgdaBound{Δ}\AgdaSymbol{)} \AgdaSymbol{→} \AgdaFunction{id'} \AgdaFunction{∘c} \AgdaBound{ch} \AgdaDatatype{≡} \AgdaBound{ch}\<%
\\
\>\AgdaFunction{id₂} \AgdaSymbol{\_} \AgdaSymbol{\_} \AgdaBound{ch} \AgdaSymbol{=} \AgdaInductiveConstructor{PE.refl}\<%
\\
%
\\
\>\AgdaFunction{comp} \AgdaSymbol{:} \AgdaSymbol{∀} \AgdaBound{Γ} \AgdaSymbol{\{}\AgdaBound{Δ} \AgdaBound{Φ}\AgdaSymbol{\}} \AgdaBound{Ψ} \<[19]%
\>[19]\<%
\\
\>[-2]\AgdaIndent{9}{}\<[9]%
\>[9]\AgdaSymbol{(}\AgdaBound{f} \AgdaSymbol{:} \AgdaBound{Γ} \AgdaRecord{⇉} \AgdaBound{Δ}\AgdaSymbol{)}\<%
\\
\>[0]\AgdaIndent{9}{}\<[9]%
\>[9]\AgdaSymbol{(}\AgdaBound{g} \AgdaSymbol{:} \AgdaBound{Δ} \AgdaRecord{⇉} \AgdaBound{Φ}\AgdaSymbol{)}\<%
\\
\>[0]\AgdaIndent{9}{}\<[9]%
\>[9]\AgdaSymbol{(}\AgdaBound{h} \AgdaSymbol{:} \AgdaBound{Φ} \AgdaRecord{⇉} \AgdaBound{Ψ}\AgdaSymbol{)}\<%
\\
\>[0]\AgdaIndent{7}{}\<[7]%
\>[7]\AgdaSymbol{→} \AgdaBound{h} \AgdaFunction{∘c} \AgdaBound{g} \AgdaFunction{∘c} \AgdaBound{f} \AgdaDatatype{≡} \AgdaBound{h} \AgdaFunction{∘c} \AgdaSymbol{(}\AgdaBound{g} \AgdaFunction{∘c} \AgdaBound{f}\AgdaSymbol{)}\<%
\\
\>\AgdaFunction{comp} \AgdaSymbol{\_} \AgdaSymbol{\_} \AgdaBound{f} \AgdaBound{g} \AgdaBound{h} \AgdaSymbol{=} \AgdaInductiveConstructor{PE.refl}\<%
\\
%
\\
\>\<\end{code}

\AgdaHide{
\begin{code}\>\<%
\\
\>\AgdaComment{\{-
\_f≈\_ :  ∀\{Γ Δ : hSetoid\} → (f g : Γ ⇉ Δ) → hProp
\_f≈\_ \{Γ , \_≈h\_ , (refl , sym , trans)\} \{Δ , \_≈h₁\_ , (refl₁ , sym₁ , trans₁)\} (fn: fn resp: fresp) (fn: gn resp: gresp) 
  = record 
           \{ prf = (g : Γ) → < fn g ≈h₁ gn g >
           ; Uni = ext (λ g → Uni (fn g ≈h₁ gn g))
           \}
-\}}\<%
\\
%
\\
%
\\
\>\<\end{code}
}

Combined all components we obtain the category of setoids.

\begin{code}\>\<%
\\
%
\\
\>\AgdaFunction{setoid-Cat} \AgdaSymbol{:} \AgdaRecord{Category}\<%
\\
\>\AgdaFunction{setoid-Cat} \AgdaSymbol{=} \AgdaInductiveConstructor{CatC} \AgdaRecord{hSetoid} \AgdaRecord{\_⇉\_} \AgdaSymbol{(λ} \AgdaBound{\_} \AgdaSymbol{→} \AgdaFunction{id'}\AgdaSymbol{)} \AgdaSymbol{(λ} \AgdaBound{\_} \AgdaBound{\_} \AgdaSymbol{→} \AgdaFunction{\_∘c\_}\AgdaSymbol{)} \<[57]%
\>[57]\<%
\\
\>[0]\AgdaIndent{13}{}\<[13]%
\>[13]\AgdaSymbol{(}\AgdaInductiveConstructor{IsCatC} \AgdaFunction{id₁} \AgdaFunction{id₂} \AgdaFunction{comp}\AgdaSymbol{)}\<%
\\
%
\\
\>\<\end{code}

This category has a terminal object which is just the unit set with trivial equality. As a terminal object there is precisely one morphism from every object to it.

\begin{code}\>\<%
\\
%
\\
\>\AgdaFunction{⊤-setoid} \AgdaSymbol{:} \AgdaRecord{hSetoid}\<%
\\
\>\AgdaFunction{⊤-setoid} \AgdaSymbol{=} \AgdaKeyword{record} \AgdaSymbol{\{}\<%
\\
\>[0]\AgdaIndent{6}{}\<[6]%
\>[6]\AgdaField{Carrier} \AgdaSymbol{=} \AgdaRecord{⊤}\AgdaSymbol{;}\<%
\\
\>[0]\AgdaIndent{6}{}\<[6]%
\>[6]\AgdaField{\_≈h\_} \<[14]%
\>[14]\AgdaSymbol{=} \AgdaSymbol{λ} \AgdaBound{\_} \AgdaBound{\_} \AgdaSymbol{→} \AgdaFunction{⊤'}\AgdaSymbol{;}\<%
\\
\>[0]\AgdaIndent{6}{}\<[6]%
\>[6]\AgdaField{isEquiv} \AgdaSymbol{=} \AgdaKeyword{record} \AgdaSymbol{\{}\<%
\\
\>[6]\AgdaIndent{8}{}\<[8]%
\>[8]\AgdaField{refl} \<[16]%
\>[16]\AgdaSymbol{=} \AgdaInductiveConstructor{tt}\AgdaSymbol{;}\<%
\\
\>[6]\AgdaIndent{8}{}\<[8]%
\>[8]\AgdaField{sym} \<[16]%
\>[16]\AgdaSymbol{=} \AgdaSymbol{λ} \AgdaBound{\_} \AgdaSymbol{→} \AgdaInductiveConstructor{tt}\AgdaSymbol{;}\<%
\\
\>[6]\AgdaIndent{8}{}\<[8]%
\>[8]\AgdaField{trans} \<[16]%
\>[16]\AgdaSymbol{=} \AgdaSymbol{λ} \AgdaBound{\_} \AgdaBound{\_} \AgdaSymbol{→} \AgdaInductiveConstructor{tt} \AgdaSymbol{\}} \AgdaSymbol{\}}\<%
\\
%
\\
\>\AgdaFunction{⋆} \AgdaSymbol{:} \AgdaSymbol{\{}\AgdaBound{Δ} \AgdaSymbol{:} \AgdaRecord{hSetoid}\AgdaSymbol{\}} \AgdaSymbol{→} \AgdaBound{Δ} \AgdaRecord{⇉} \AgdaFunction{⊤-setoid}\<%
\\
\>\AgdaFunction{⋆} \AgdaSymbol{=} \AgdaKeyword{record} \<[11]%
\>[11]\<%
\\
\>[0]\AgdaIndent{6}{}\<[6]%
\>[6]\AgdaSymbol{\{} \AgdaField{fn} \AgdaSymbol{=} \AgdaSymbol{λ} \AgdaBound{\_} \AgdaSymbol{→} \AgdaInductiveConstructor{tt}\<%
\\
\>[0]\AgdaIndent{6}{}\<[6]%
\>[6]\AgdaSymbol{;} \AgdaField{resp} \AgdaSymbol{=} \AgdaSymbol{λ} \AgdaBound{\_} \AgdaSymbol{→} \AgdaInductiveConstructor{tt} \AgdaSymbol{\}}\<%
\\
%
\\
\>\AgdaFunction{unique⋆} \AgdaSymbol{:} \AgdaSymbol{\{}\AgdaBound{Δ} \AgdaSymbol{:} \AgdaRecord{hSetoid}\AgdaSymbol{\}} \AgdaSymbol{→} \AgdaSymbol{(}\AgdaBound{f} \AgdaSymbol{:} \AgdaBound{Δ} \AgdaRecord{⇉} \AgdaFunction{⊤-setoid}\AgdaSymbol{)} \AgdaSymbol{→} \AgdaBound{f} \AgdaDatatype{≡} \AgdaFunction{⋆}\<%
\\
\>\AgdaFunction{unique⋆} \AgdaBound{f} \AgdaSymbol{=} \AgdaInductiveConstructor{PE.refl}\<%
\\
%
\\
\>\<\end{code}



\subsection{categories with families of setoids}


A Category with families consists of a base category and a functor
\cite{clairambault2005categories}. We firstly define the categories with
families of sets in Agda  as a guidance for the one for setoids. We
would present the setoid one here since it is relevant.


\AgdaHide{

\begin{code}\>\<%
\\
\>\AgdaSymbol{\{-\#} \AgdaKeyword{OPTIONS} --type-in-type \AgdaSymbol{\#-\}}\<%
\\
%
\\
%
\\
\>\AgdaKeyword{open} \AgdaKeyword{import} \AgdaModule{Level} \AgdaKeyword{hiding} \AgdaSymbol{(}lift\AgdaSymbol{)}\<%
\\
\>\AgdaKeyword{open} \AgdaKeyword{import} \AgdaModule{Relation.Binary.PropositionalEquality} \AgdaSymbol{as} \AgdaModule{PE} \AgdaKeyword{hiding} \AgdaSymbol{(}refl \AgdaSymbol{;} sym \AgdaSymbol{;} trans\AgdaSymbol{;} isEquivalence\AgdaSymbol{;} [\_]\AgdaSymbol{)}\<%
\\
%
\\
\>\AgdaKeyword{module} \AgdaModule{CwF-setoid} \AgdaSymbol{(}\AgdaBound{ext} \AgdaSymbol{:} \AgdaFunction{Extensionality} \AgdaPrimitive{zero} \AgdaPrimitive{zero}\AgdaSymbol{)} \AgdaKeyword{where}\<%
\\
%
\\
%
\\
\>\AgdaKeyword{open} \AgdaKeyword{import} \AgdaModule{Cats.Category}\<%
\\
\>\AgdaKeyword{open} \AgdaKeyword{import} \AgdaModule{Cats.Functor}\<%
\\
\>\AgdaKeyword{open} \AgdaKeyword{import} \AgdaModule{Cats.Duality}\<%
\\
\>\AgdaKeyword{open} \AgdaKeyword{import} \AgdaModule{Data.Product} \AgdaKeyword{renaming} \AgdaSymbol{(}<\_,\_> \AgdaSymbol{to} ⟨\_,\_⟩\AgdaSymbol{)}\<%
\\
\>\AgdaKeyword{open} \AgdaKeyword{import} \AgdaModule{Function}\<%
\\
%
\\
\>\AgdaKeyword{open} \AgdaKeyword{import} \AgdaModule{Relation.Binary.Core} \AgdaKeyword{using} \AgdaSymbol{(}\_≡\_\AgdaSymbol{;} \_≢\_\AgdaSymbol{)}\<%
\\
\>\AgdaKeyword{open} \AgdaKeyword{import} \AgdaModule{Data.Unit}\<%
\\
%
\\
\>\AgdaKeyword{import} \AgdaModule{CategoryOfSetoid}\<%
\\
\>\AgdaKeyword{module} \AgdaModule{cos} \AgdaSymbol{=} \AgdaModule{CategoryOfSetoid} \AgdaBound{ext}\<%
\\
\>\AgdaKeyword{open} \AgdaModule{cos}\<%
\\
%
\\
\>\AgdaKeyword{import} \AgdaModule{hProp}\<%
\\
\>\AgdaKeyword{module} \AgdaModule{hp} \AgdaSymbol{=} \AgdaModule{hProp} \AgdaBound{ext}\<%
\\
\>\AgdaKeyword{open} \AgdaModule{hp}\<%
\\
%
\\
\>\AgdaKeyword{infixl} \AgdaNumber{5} \_\&\_\<%
\\
%
\\
\>\<\end{code}
}

We would like to show two formalisation of category with families for setoids here. The first one is simple and short but not comprehensive. We have to extract all complicated components from the simple definition. However the second one gives these components one by one so that it more understandable and convenient.

The category with families works as a model for type theory. So we will introduce them from a type theoretical point of view.

The base category is the category for contexts. In the setoid version we interpret a context as a setoid as well.

To define the second component, namely the presheaf functor, it is necessary to construct the target category first. The objects of this category are families of setoids.The index setoids are the semantic types and the indexed families of setoids are terms. The morphisms are component-wise morphisms between setoids. All the categorical laws hold trivially.

\begin{code}\>\<%
\\
%
\\
\>\AgdaFunction{inxSetoids} \AgdaSymbol{:} \AgdaPrimitiveType{Set₁}\<%
\\
\>\AgdaFunction{inxSetoids} \AgdaSymbol{=} \AgdaRecord{Σ[} \AgdaBound{I} \AgdaRecord{∶} \AgdaRecord{hSetoid} \AgdaRecord{]} \AgdaSymbol{(}\AgdaFunction{∣} \AgdaBound{I} \AgdaFunction{∣} \AgdaSymbol{→} \AgdaRecord{hSetoid}\AgdaSymbol{)}\<%
\\
%
\\
\>\AgdaFunction{\_⇉setoid\_} \AgdaSymbol{:} \AgdaFunction{inxSetoids} \AgdaSymbol{→} \AgdaFunction{inxSetoids} \AgdaSymbol{→} \AgdaPrimitiveType{Set₁}\<%
\\
\>\AgdaSymbol{(}\AgdaBound{I} \AgdaInductiveConstructor{,} \AgdaBound{f}\AgdaSymbol{)} \AgdaFunction{⇉setoid} \AgdaSymbol{(}\AgdaBound{J} \AgdaInductiveConstructor{,} \AgdaBound{g}\AgdaSymbol{)} \AgdaSymbol{=} \<[26]%
\>[26]\<%
\\
\>[0]\AgdaIndent{2}{}\<[2]%
\>[2]\AgdaRecord{Σ[} \AgdaBound{i-map} \AgdaRecord{∶} \AgdaBound{I} \AgdaRecord{⇉} \AgdaBound{J} \AgdaRecord{]}\<%
\\
\>[2]\AgdaIndent{4}{}\<[4]%
\>[4]\AgdaSymbol{((}\AgdaBound{i} \AgdaSymbol{:} \AgdaFunction{∣} \AgdaBound{I} \AgdaFunction{∣}\AgdaSymbol{)} \AgdaSymbol{→} \AgdaBound{f} \AgdaBound{i} \AgdaRecord{⇉} \AgdaBound{g} \AgdaSymbol{(} \AgdaFunction{[} \AgdaBound{i-map} \AgdaFunction{]fn} \AgdaBound{i}\AgdaSymbol{))}\<%
\\
%
\\
\>\AgdaFunction{Fam-setoid} \AgdaSymbol{:} \AgdaRecord{Category}\<%
\\
\>\AgdaFunction{Fam-setoid} \AgdaSymbol{=} \AgdaInductiveConstructor{CatC} \<[18]%
\>[18]\<%
\\
\>[4]\AgdaIndent{15}{}\<[15]%
\>[15]\AgdaFunction{inxSetoids} \<[26]%
\>[26]\<%
\\
\>[4]\AgdaIndent{15}{}\<[15]%
\>[15]\AgdaFunction{\_⇉setoid\_} \<[25]%
\>[25]\<%
\\
\>[4]\AgdaIndent{15}{}\<[15]%
\>[15]\AgdaSymbol{(λ} \AgdaBound{\_} \AgdaSymbol{→} \AgdaFunction{id'} \AgdaInductiveConstructor{,} \AgdaSymbol{(λ} \AgdaBound{\_} \AgdaSymbol{→} \AgdaFunction{id'}\AgdaSymbol{))} \<[41]%
\>[41]\<%
\\
\>[4]\AgdaIndent{15}{}\<[15]%
\>[15]\AgdaSymbol{(λ} \AgdaSymbol{\{} \AgdaSymbol{\_} \AgdaSymbol{\_} \AgdaSymbol{(}\AgdaBound{fty} \AgdaInductiveConstructor{,} \AgdaBound{ftm}\AgdaSymbol{)} \AgdaSymbol{(}\AgdaBound{gty} \AgdaInductiveConstructor{,} \AgdaBound{gtm}\AgdaSymbol{)} \AgdaSymbol{→} \AgdaBound{fty} \AgdaFunction{∘c} \AgdaBound{gty} \AgdaInductiveConstructor{,}\<%
\\
\>[15]\AgdaIndent{17}{}\<[17]%
\>[17]\AgdaSymbol{(λ} \AgdaBound{i} \AgdaSymbol{→} \AgdaBound{ftm} \AgdaSymbol{(}\AgdaFunction{[} \AgdaBound{gty} \AgdaFunction{]fn} \AgdaBound{i}\AgdaSymbol{)} \AgdaFunction{∘c} \AgdaBound{gtm} \AgdaBound{i}\AgdaSymbol{)\})}\<%
\\
\>[0]\AgdaIndent{15}{}\<[15]%
\>[15]\AgdaSymbol{(}\AgdaInductiveConstructor{IsCatC} \<[23]%
\>[23]\<%
\\
\>[0]\AgdaIndent{17}{}\<[17]%
\>[17]\AgdaSymbol{(λ} \AgdaBound{α} \AgdaBound{β} \AgdaBound{f} \AgdaSymbol{→} \AgdaInductiveConstructor{PE.refl}\AgdaSymbol{)} \<[37]%
\>[37]\<%
\\
\>[0]\AgdaIndent{17}{}\<[17]%
\>[17]\AgdaSymbol{(λ} \AgdaBound{α} \AgdaBound{β} \AgdaBound{f} \AgdaSymbol{→} \AgdaInductiveConstructor{PE.refl}\AgdaSymbol{)} \<[37]%
\>[37]\<%
\\
\>[0]\AgdaIndent{17}{}\<[17]%
\>[17]\AgdaSymbol{(λ} \AgdaBound{α} \AgdaBound{δ} \AgdaBound{f} \AgdaBound{g} \AgdaBound{h} \AgdaSymbol{→} \AgdaInductiveConstructor{PE.refl}\AgdaSymbol{))}\<%
\\
%
\\
\>\<\end{code}

Since we already specify the category of contexts, we only need the presheaf which is a contravariant functor from the category of contexts to the category we defined above. The definition of category with families of setoids could be as simple as follows.

\begin{code}\>\<%
\\
%
\\
\>\AgdaKeyword{record} \AgdaRecord{CWF-setoid} \AgdaSymbol{:} \AgdaPrimitiveType{Set₁} \AgdaKeyword{where}\<%
\\
\>[0]\AgdaIndent{2}{}\<[2]%
\>[2]\AgdaKeyword{field}\<%
\\
\>[0]\AgdaIndent{4}{}\<[4]%
\>[4]\AgdaField{T} \AgdaSymbol{:} \AgdaRecord{Functor} \AgdaSymbol{(}\AgdaFunction{Op} \AgdaFunction{setoid-Cat}\AgdaSymbol{)} \AgdaFunction{Fam-setoid}\<%
\\
%
\\
\>\<\end{code}

All details of this definition are hidden including the functor laws. Therefore we will show the details as the second version.

The semantic contexts are setoids and the terminal object is just the empty context. 

\begin{code}\>\<%
\\
%
\\
\>\AgdaFunction{Con} \AgdaSymbol{=} \AgdaRecord{hSetoid}\<%
\\
%
\\
\>\AgdaFunction{emptyCon} \AgdaSymbol{=} \AgdaFunction{⊤-setoid}\<%
\\
%
\\
\>\AgdaFunction{emptysub} \AgdaSymbol{=} \AgdaFunction{⋆}\<%
\\
%
\\
\>\<\end{code}

A semantic type has following components. $fm$ is a setoid of all types. $substT$ is the substitution between types within the context. It should be a morphism between setoids so it has to preserve the equivalence relation. We also need to specify the computation rules for substitution.

\begin{code}\>\<%
\\
%
\\
\>\AgdaKeyword{record} \AgdaRecord{Ty} \AgdaSymbol{(}\AgdaBound{Γ} \AgdaSymbol{:} \AgdaFunction{Con}\AgdaSymbol{)} \AgdaSymbol{:} \AgdaPrimitiveType{Set₁} \AgdaKeyword{where}\<%
\\
\>[0]\AgdaIndent{2}{}\<[2]%
\>[2]\AgdaKeyword{field}\<%
\\
\>[0]\AgdaIndent{4}{}\<[4]%
\>[4]\AgdaField{fm} \<[11]%
\>[11]\AgdaSymbol{:} \AgdaFunction{∣} \AgdaBound{Γ} \AgdaFunction{∣} \AgdaSymbol{→} \AgdaRecord{hSetoid}\<%
\\
%
\\
\>[0]\AgdaIndent{4}{}\<[4]%
\>[4]\AgdaField{substT} \AgdaSymbol{:} \AgdaSymbol{\{}\AgdaBound{x} \AgdaBound{y} \AgdaSymbol{:} \AgdaFunction{∣} \AgdaBound{Γ} \AgdaFunction{∣}\AgdaSymbol{\}} \AgdaSymbol{→} \<[29]%
\>[29]\<%
\\
\>[4]\AgdaIndent{13}{}\<[13]%
\>[13]\AgdaFunction{[} \AgdaBound{Γ} \AgdaFunction{]} \AgdaBound{x} \AgdaFunction{≈} \AgdaBound{y} \AgdaSymbol{→}\<%
\\
\>[4]\AgdaIndent{13}{}\<[13]%
\>[13]\AgdaFunction{∣} \AgdaBound{fm} \AgdaBound{x} \AgdaFunction{∣} \AgdaSymbol{→}\<%
\\
\>[4]\AgdaIndent{13}{}\<[13]%
\>[13]\AgdaFunction{∣} \AgdaBound{fm} \AgdaBound{y} \AgdaFunction{∣}\<%
\\
\>[0]\AgdaIndent{4}{}\<[4]%
\>[4]\AgdaField{subst*} \AgdaSymbol{:} \AgdaSymbol{∀\{}\AgdaBound{x} \AgdaBound{y} \AgdaSymbol{:} \AgdaFunction{∣} \AgdaBound{Γ} \AgdaFunction{∣}\AgdaSymbol{\}}\<%
\\
\>[0]\AgdaIndent{13}{}\<[13]%
\>[13]\AgdaSymbol{(}\AgdaBound{p} \AgdaSymbol{:} \AgdaFunction{[} \AgdaBound{Γ} \AgdaFunction{]} \AgdaBound{x} \AgdaFunction{≈} \AgdaBound{y}\AgdaSymbol{)}\<%
\\
\>[0]\AgdaIndent{13}{}\<[13]%
\>[13]\AgdaSymbol{\{}\AgdaBound{a} \AgdaBound{b} \AgdaSymbol{:} \AgdaFunction{∣} \AgdaBound{fm} \AgdaBound{x} \AgdaFunction{∣}\AgdaSymbol{\}} \AgdaSymbol{→}\<%
\\
\>[0]\AgdaIndent{13}{}\<[13]%
\>[13]\AgdaFunction{[} \AgdaBound{fm} \AgdaBound{x} \AgdaFunction{]} \AgdaBound{a} \AgdaFunction{≈} \AgdaBound{b} \AgdaSymbol{→}\<%
\\
\>[0]\AgdaIndent{13}{}\<[13]%
\>[13]\AgdaFunction{[} \AgdaBound{fm} \AgdaBound{y} \AgdaFunction{]} \AgdaBound{substT} \AgdaBound{p} \AgdaBound{a} \AgdaFunction{≈} \AgdaBound{substT} \AgdaBound{p} \AgdaBound{b}\<%
\\
%
\\
\>[0]\AgdaIndent{4}{}\<[4]%
\>[4]\AgdaField{refl*} \<[11]%
\>[11]\AgdaSymbol{:} \AgdaSymbol{∀(}\AgdaBound{x} \AgdaSymbol{:} \AgdaFunction{∣} \AgdaBound{Γ} \AgdaFunction{∣}\AgdaSymbol{)}\<%
\\
\>[0]\AgdaIndent{13}{}\<[13]%
\>[13]\AgdaSymbol{(}\AgdaBound{a} \AgdaSymbol{:} \AgdaFunction{∣} \AgdaBound{fm} \AgdaBound{x} \AgdaFunction{∣}\AgdaSymbol{)} \AgdaSymbol{→} \<[30]%
\>[30]\<%
\\
\>[0]\AgdaIndent{13}{}\<[13]%
\>[13]\AgdaFunction{[} \AgdaBound{fm} \AgdaBound{x} \AgdaFunction{]} \AgdaBound{substT} \AgdaFunction{[} \AgdaBound{Γ} \AgdaFunction{]refl} \AgdaBound{a} \AgdaFunction{≈} \AgdaBound{a}\<%
\\
\>[0]\AgdaIndent{4}{}\<[4]%
\>[4]\AgdaField{trans*} \AgdaSymbol{:} \AgdaSymbol{∀\{}\AgdaBound{x} \AgdaBound{y} \AgdaBound{z} \AgdaSymbol{:} \AgdaFunction{∣} \AgdaBound{Γ} \AgdaFunction{∣}\AgdaSymbol{\}}\<%
\\
\>[0]\AgdaIndent{13}{}\<[13]%
\>[13]\AgdaSymbol{(}\AgdaBound{p} \AgdaSymbol{:} \AgdaFunction{[} \AgdaBound{Γ} \AgdaFunction{]} \AgdaBound{x} \AgdaFunction{≈} \AgdaBound{y}\AgdaSymbol{)}\<%
\\
\>[0]\AgdaIndent{13}{}\<[13]%
\>[13]\AgdaSymbol{(}\AgdaBound{q} \AgdaSymbol{:} \AgdaFunction{[} \AgdaBound{Γ} \AgdaFunction{]} \AgdaBound{y} \AgdaFunction{≈} \AgdaBound{z}\AgdaSymbol{)}\<%
\\
\>[0]\AgdaIndent{13}{}\<[13]%
\>[13]\AgdaSymbol{(}\AgdaBound{a} \AgdaSymbol{:} \AgdaFunction{∣} \AgdaBound{fm} \AgdaBound{x} \AgdaFunction{∣}\AgdaSymbol{)} \<[28]%
\>[28]\<%
\\
\>[0]\AgdaIndent{13}{}\<[13]%
\>[13]\AgdaSymbol{→} \AgdaFunction{[} \AgdaBound{fm} \AgdaBound{z} \AgdaFunction{]} \AgdaBound{substT} \AgdaBound{q} \AgdaSymbol{(}\AgdaBound{substT} \AgdaBound{p} \AgdaBound{a}\AgdaSymbol{)} \<[46]%
\>[46]\<%
\\
\>[13]\AgdaIndent{17}{}\<[17]%
\>[17]\AgdaFunction{≈} \AgdaBound{substT} \AgdaSymbol{(}\AgdaFunction{[} \AgdaBound{Γ} \AgdaFunction{]trans} \AgdaBound{p} \AgdaBound{q}\AgdaSymbol{)} \AgdaBound{a}\<%
\\
%
\\
%
\\
\>\<\end{code}

Some other lemmas on the proof irrelevance derived from these fields are not shown here since they are just auxiliary functions.

\AgdaHide{
\begin{code}\>\<%
\\
\>\AgdaComment{-- the proof-irrelevance lemmas for substT}\<%
\\
%
\\
\>[2]\AgdaIndent{2}{}\<[2]%
\>[2]\AgdaFunction{subst-pi} \AgdaSymbol{:} \AgdaSymbol{∀\{}\AgdaBound{x} \AgdaBound{y} \AgdaSymbol{:} \AgdaFunction{∣} \AgdaBound{Γ} \AgdaFunction{∣}\AgdaSymbol{\}}\<%
\\
\>[0]\AgdaIndent{14}{}\<[14]%
\>[14]\AgdaSymbol{\{}\AgdaBound{p} \AgdaBound{q} \AgdaSymbol{:} \AgdaFunction{[} \AgdaBound{Γ} \AgdaFunction{]} \AgdaBound{x} \AgdaFunction{≈} \AgdaBound{y}\AgdaSymbol{\}}\<%
\\
\>[0]\AgdaIndent{14}{}\<[14]%
\>[14]\AgdaSymbol{\{}\AgdaBound{a} \AgdaSymbol{:} \AgdaFunction{∣} \AgdaFunction{fm} \AgdaBound{x} \AgdaFunction{∣}\AgdaSymbol{\}} \AgdaSymbol{→} \AgdaFunction{[} \AgdaFunction{fm} \AgdaBound{y} \AgdaFunction{]} \AgdaFunction{substT} \AgdaBound{p} \AgdaBound{a} \AgdaFunction{≈} \AgdaFunction{substT} \AgdaBound{q} \AgdaBound{a}\<%
\\
\>[0]\AgdaIndent{2}{}\<[2]%
\>[2]\AgdaFunction{subst-pi} \AgdaSymbol{\{}\AgdaBound{x}\AgdaSymbol{\}} \AgdaSymbol{\{}\AgdaBound{y}\AgdaSymbol{\}} \AgdaSymbol{\{}\AgdaBound{p}\AgdaSymbol{\}} \AgdaSymbol{\{}\AgdaBound{q}\AgdaSymbol{\}} \AgdaSymbol{\{}\AgdaBound{a}\AgdaSymbol{\}} \AgdaSymbol{=} \AgdaFunction{reflexive} \AgdaSymbol{(}\AgdaFunction{fm} \AgdaBound{y}\AgdaSymbol{)} \AgdaSymbol{(}\AgdaFunction{PI} \AgdaBound{Γ} \AgdaSymbol{(λ} \AgdaBound{x} \AgdaSymbol{→} \AgdaFunction{substT} \AgdaBound{x} \AgdaBound{a}\AgdaSymbol{))}\<%
\\
%
\\
\>[0]\AgdaIndent{2}{}\<[2]%
\>[2]\AgdaFunction{subst-pi'} \AgdaSymbol{:} \AgdaSymbol{∀\{}\AgdaBound{x} \AgdaSymbol{:} \AgdaFunction{∣} \AgdaBound{Γ} \AgdaFunction{∣}\AgdaSymbol{\}}\<%
\\
\>[2]\AgdaIndent{15}{}\<[15]%
\>[15]\AgdaSymbol{\{}\AgdaBound{p} \AgdaSymbol{:} \AgdaFunction{[} \AgdaBound{Γ} \AgdaFunction{]} \AgdaBound{x} \AgdaFunction{≈} \AgdaBound{x}\AgdaSymbol{\}}\<%
\\
\>[2]\AgdaIndent{15}{}\<[15]%
\>[15]\AgdaSymbol{\{}\AgdaBound{a} \AgdaSymbol{:} \AgdaFunction{∣} \AgdaFunction{fm} \AgdaBound{x} \AgdaFunction{∣}\AgdaSymbol{\}} \AgdaSymbol{→} \AgdaFunction{[} \AgdaFunction{fm} \AgdaBound{x} \AgdaFunction{]} \AgdaFunction{substT} \AgdaBound{p} \AgdaBound{a} \AgdaFunction{≈} \AgdaBound{a}\<%
\\
\>[0]\AgdaIndent{2}{}\<[2]%
\>[2]\AgdaFunction{subst-pi'} \AgdaSymbol{=} \AgdaFunction{[} \AgdaFunction{fm} \AgdaSymbol{\_} \AgdaFunction{]trans} \AgdaFunction{subst-pi} \AgdaSymbol{(}\AgdaFunction{refl*} \AgdaSymbol{\_} \AgdaSymbol{\_)}\<%
\\
%
\\
\>[0]\AgdaIndent{2}{}\<[2]%
\>[2]\AgdaFunction{subst-pi*} \AgdaSymbol{:} \AgdaSymbol{∀\{}\AgdaBound{x} \AgdaBound{y} \AgdaSymbol{:} \AgdaFunction{∣} \AgdaBound{Γ} \AgdaFunction{∣}\AgdaSymbol{\}}\<%
\\
\>[2]\AgdaIndent{16}{}\<[16]%
\>[16]\AgdaSymbol{\{}\AgdaBound{p} \AgdaBound{q} \AgdaSymbol{:} \AgdaFunction{[} \AgdaBound{Γ} \AgdaFunction{]} \AgdaBound{x} \AgdaFunction{≈} \AgdaBound{y}\AgdaSymbol{\}}\<%
\\
\>[2]\AgdaIndent{16}{}\<[16]%
\>[16]\AgdaSymbol{\{}\AgdaBound{a} \AgdaBound{b} \AgdaSymbol{:} \AgdaFunction{∣} \AgdaFunction{fm} \AgdaBound{x} \AgdaFunction{∣}\AgdaSymbol{\}} \AgdaSymbol{→} \AgdaFunction{[} \AgdaFunction{fm} \AgdaBound{x} \AgdaFunction{]} \AgdaBound{a} \AgdaFunction{≈} \AgdaBound{b} \AgdaSymbol{→} \AgdaFunction{[} \AgdaFunction{fm} \AgdaBound{y} \AgdaFunction{]} \AgdaFunction{substT} \AgdaBound{p} \AgdaBound{a} \AgdaFunction{≈} \AgdaFunction{substT} \AgdaBound{q} \AgdaBound{b}\<%
\\
\>[0]\AgdaIndent{2}{}\<[2]%
\>[2]\AgdaFunction{subst-pi*} \AgdaBound{eq} \AgdaSymbol{=} \AgdaFunction{[} \AgdaFunction{fm} \AgdaSymbol{\_} \AgdaFunction{]trans} \AgdaSymbol{(}\AgdaFunction{subst*} \AgdaSymbol{\_} \AgdaBound{eq}\AgdaSymbol{)} \AgdaFunction{subst-pi}\<%
\\
%
\\
%
\\
\>\AgdaComment{-- simplify proofs of trans of inverse equality (including groupoid laws?)}\<%
\\
%
\\
\>[0]\AgdaIndent{2}{}\<[2]%
\>[2]\AgdaFunction{trans-refl} \AgdaSymbol{:} \AgdaSymbol{∀\{}\AgdaBound{x} \AgdaBound{y} \AgdaSymbol{:} \AgdaFunction{∣} \AgdaBound{Γ} \AgdaFunction{∣}\AgdaSymbol{\}}\<%
\\
\>[2]\AgdaIndent{14}{}\<[14]%
\>[14]\AgdaSymbol{\{}\AgdaBound{p} \AgdaSymbol{:} \AgdaFunction{[} \AgdaBound{Γ} \AgdaFunction{]} \AgdaBound{x} \AgdaFunction{≈} \AgdaBound{y}\AgdaSymbol{\}\{}\AgdaBound{q} \AgdaSymbol{:} \AgdaFunction{[} \AgdaBound{Γ} \AgdaFunction{]} \AgdaBound{y} \AgdaFunction{≈} \AgdaBound{x}\AgdaSymbol{\}}\<%
\\
\>[2]\AgdaIndent{14}{}\<[14]%
\>[14]\AgdaSymbol{\{}\AgdaBound{a} \AgdaSymbol{:} \AgdaFunction{∣} \AgdaFunction{fm} \AgdaBound{x} \AgdaFunction{∣}\AgdaSymbol{\}} \AgdaSymbol{→} \<[31]%
\>[31]\<%
\\
\>[2]\AgdaIndent{14}{}\<[14]%
\>[14]\AgdaFunction{[} \AgdaFunction{fm} \AgdaBound{x} \AgdaFunction{]} \AgdaFunction{substT} \AgdaBound{q} \AgdaSymbol{(}\AgdaFunction{substT} \AgdaBound{p} \AgdaBound{a}\AgdaSymbol{)} \AgdaFunction{≈} \AgdaBound{a}\<%
\\
\>[0]\AgdaIndent{2}{}\<[2]%
\>[2]\AgdaFunction{trans-refl} \AgdaSymbol{=} \AgdaFunction{[} \AgdaFunction{fm} \AgdaSymbol{\_} \AgdaFunction{]trans} \AgdaSymbol{(}\AgdaFunction{trans*} \AgdaSymbol{\_} \AgdaSymbol{\_} \AgdaSymbol{\_)} \AgdaFunction{subst-pi'}\<%
\\
%
\\
\>\AgdaComment{-- some more theorems}\<%
\\
\>[0]\AgdaIndent{2}{}\<[2]%
\>[2]\<%
\\
\>[0]\AgdaIndent{2}{}\<[2]%
\>[2]\AgdaFunction{subst-mir1} \AgdaSymbol{:} \AgdaSymbol{∀\{}\AgdaBound{x} \AgdaBound{y} \AgdaSymbol{:} \AgdaFunction{∣} \AgdaBound{Γ} \AgdaFunction{∣}\AgdaSymbol{\}}\<%
\\
\>[2]\AgdaIndent{14}{}\<[14]%
\>[14]\AgdaSymbol{\{}\AgdaBound{p} \AgdaSymbol{:} \AgdaFunction{[} \AgdaBound{Γ} \AgdaFunction{]} \AgdaBound{x} \AgdaFunction{≈} \AgdaBound{y}\AgdaSymbol{\}\{}\AgdaBound{q} \AgdaSymbol{:} \AgdaFunction{[} \AgdaBound{Γ} \AgdaFunction{]} \AgdaBound{y} \AgdaFunction{≈} \AgdaBound{x}\AgdaSymbol{\}}\<%
\\
\>[2]\AgdaIndent{14}{}\<[14]%
\>[14]\AgdaSymbol{\{}\AgdaBound{a} \AgdaSymbol{:} \AgdaFunction{∣} \AgdaFunction{fm} \AgdaBound{x} \AgdaFunction{∣}\AgdaSymbol{\}\{}\AgdaBound{b} \AgdaSymbol{:} \AgdaFunction{∣} \AgdaFunction{fm} \AgdaBound{y} \AgdaFunction{∣}\AgdaSymbol{\}} \AgdaSymbol{→} \<[45]%
\>[45]\<%
\\
\>[2]\AgdaIndent{14}{}\<[14]%
\>[14]\AgdaFunction{[} \AgdaFunction{fm} \AgdaBound{x} \AgdaFunction{]} \AgdaBound{a} \AgdaFunction{≈} \AgdaFunction{substT} \AgdaBound{q} \AgdaBound{b} \AgdaSymbol{→} \AgdaFunction{[} \AgdaFunction{fm} \AgdaBound{y} \AgdaFunction{]} \AgdaFunction{substT} \AgdaBound{p} \AgdaBound{a} \AgdaFunction{≈} \AgdaBound{b}\<%
\\
\>[0]\AgdaIndent{2}{}\<[2]%
\>[2]\AgdaFunction{subst-mir1} \AgdaBound{eq} \AgdaSymbol{=} \AgdaFunction{[} \AgdaFunction{fm} \AgdaSymbol{\_} \AgdaFunction{]trans} \AgdaSymbol{(}\AgdaFunction{subst*} \AgdaSymbol{\_} \AgdaBound{eq}\AgdaSymbol{)} \AgdaFunction{trans-refl}\<%
\\
%
\\
\>[0]\AgdaIndent{2}{}\<[2]%
\>[2]\AgdaFunction{subst-mir2} \AgdaSymbol{:} \AgdaSymbol{∀\{}\AgdaBound{x} \AgdaBound{y} \AgdaSymbol{:} \AgdaFunction{∣} \AgdaBound{Γ} \AgdaFunction{∣}\AgdaSymbol{\}}\<%
\\
\>[2]\AgdaIndent{14}{}\<[14]%
\>[14]\AgdaSymbol{\{}\AgdaBound{p} \AgdaSymbol{:} \AgdaFunction{[} \AgdaBound{Γ} \AgdaFunction{]} \AgdaBound{x} \AgdaFunction{≈} \AgdaBound{y}\AgdaSymbol{\}\{}\AgdaBound{q} \AgdaSymbol{:} \AgdaFunction{[} \AgdaBound{Γ} \AgdaFunction{]} \AgdaBound{y} \AgdaFunction{≈} \AgdaBound{x}\AgdaSymbol{\}}\<%
\\
\>[2]\AgdaIndent{14}{}\<[14]%
\>[14]\AgdaSymbol{\{}\AgdaBound{a} \AgdaSymbol{:} \AgdaFunction{∣} \AgdaFunction{fm} \AgdaBound{x} \AgdaFunction{∣}\AgdaSymbol{\}\{}\AgdaBound{b} \AgdaSymbol{:} \AgdaFunction{∣} \AgdaFunction{fm} \AgdaBound{y} \AgdaFunction{∣}\AgdaSymbol{\}} \AgdaSymbol{→} \<[45]%
\>[45]\<%
\\
\>[2]\AgdaIndent{14}{}\<[14]%
\>[14]\AgdaFunction{[} \AgdaFunction{fm} \AgdaBound{y} \AgdaFunction{]} \AgdaFunction{substT} \AgdaBound{p} \AgdaBound{a} \AgdaFunction{≈} \AgdaBound{b} \AgdaSymbol{→} \AgdaFunction{[} \AgdaFunction{fm} \AgdaBound{x} \AgdaFunction{]} \AgdaBound{a} \AgdaFunction{≈} \AgdaFunction{substT} \AgdaBound{q} \AgdaBound{b}\<%
\\
\>[0]\AgdaIndent{2}{}\<[2]%
\>[2]\AgdaFunction{subst-mir2} \AgdaBound{eq} \AgdaSymbol{=} \AgdaFunction{[} \AgdaFunction{fm} \AgdaSymbol{\_} \AgdaFunction{]sym} \AgdaSymbol{(}\AgdaFunction{subst-mir1} \AgdaSymbol{(}\AgdaFunction{[} \AgdaFunction{fm} \AgdaSymbol{\_} \AgdaFunction{]sym} \AgdaBound{eq}\AgdaSymbol{))}\<%
\\
%
\\
\>\AgdaKeyword{open} \AgdaModule{Ty} \AgdaKeyword{public} \<[15]%
\>[15]\<%
\\
\>[0]\AgdaIndent{2}{}\<[2]%
\>[2]\AgdaKeyword{renaming} \AgdaSymbol{(}substT \AgdaSymbol{to} [\_]subst\AgdaSymbol{;} subst* \AgdaSymbol{to} [\_]subst*\AgdaSymbol{;} fm \AgdaSymbol{to} [\_]fm \AgdaSymbol{;}
            refl* \AgdaSymbol{to} [\_]refl* \AgdaSymbol{;} trans* \AgdaSymbol{to} [\_]trans*\AgdaSymbol{;} subst-pi \AgdaSymbol{to} [\_]subst-pi \AgdaSymbol{;}
            subst-pi' \AgdaSymbol{to} [\_]subst-pi' \AgdaSymbol{;} subst-pi* \AgdaSymbol{to} [\_]subst-pi* \AgdaSymbol{;}
            trans-refl \AgdaSymbol{to} [\_]trans-refl \AgdaSymbol{;} subst-mir1 \AgdaSymbol{to} [\_]subst-mir1 \AgdaSymbol{;}
            subst-mir2 \AgdaSymbol{to} [\_]subst-mir2\AgdaSymbol{)}\<%
\\
%
\\
\>\<\end{code}
}

Then we have to define the substituting in a type given a context morphism and verify it preserves equivalence relation as well.

\begin{code}\>\<%
\\
%
\\
\>\AgdaFunction{\_[\_]T} \AgdaSymbol{:} \AgdaSymbol{∀} \AgdaSymbol{\{}\AgdaBound{Γ} \AgdaBound{Δ} \AgdaSymbol{:} \AgdaFunction{Con}\AgdaSymbol{\}} \AgdaSymbol{→} \AgdaRecord{Ty} \AgdaBound{Δ} \AgdaSymbol{→} \AgdaBound{Γ} \AgdaRecord{⇉} \AgdaBound{Δ} \AgdaSymbol{→} \AgdaRecord{Ty} \AgdaBound{Γ}\<%
\\
\>\AgdaBound{A} \AgdaFunction{[} \AgdaBound{f} \AgdaFunction{]T}\<%
\\
\>[2]\AgdaIndent{5}{}\<[5]%
\>[5]\AgdaSymbol{=} \AgdaKeyword{record}\<%
\\
\>[2]\AgdaIndent{5}{}\<[5]%
\>[5]\AgdaSymbol{\{} \AgdaField{fm} \<[14]%
\>[14]\AgdaSymbol{=} \AgdaFunction{fm} \AgdaFunction{∘} \AgdaFunction{fn}\<%
\\
\>[2]\AgdaIndent{5}{}\<[5]%
\>[5]\AgdaSymbol{;} \AgdaField{substT} \AgdaSymbol{=} \AgdaFunction{substT} \AgdaFunction{∘} \AgdaFunction{resp}\<%
\\
\>[2]\AgdaIndent{5}{}\<[5]%
\>[5]\AgdaSymbol{;} \AgdaField{subst*} \AgdaSymbol{=} \AgdaFunction{subst*} \AgdaFunction{∘} \AgdaFunction{resp}\<%
\\
\>[2]\AgdaIndent{5}{}\<[5]%
\>[5]\AgdaSymbol{;} \AgdaField{refl*} \<[14]%
\>[14]\AgdaSymbol{=} \AgdaSymbol{λ} \AgdaBound{\_} \AgdaBound{\_} \AgdaSymbol{→} \AgdaFunction{subst-pi'}\<%
\\
\>[2]\AgdaIndent{5}{}\<[5]%
\>[5]\AgdaSymbol{;} \AgdaField{trans*} \AgdaSymbol{=} \AgdaSymbol{λ} \AgdaBound{\_} \AgdaBound{\_} \AgdaBound{\_} \AgdaSymbol{→} \<[26]%
\>[26]\<%
\\
\>[5]\AgdaIndent{16}{}\<[16]%
\>[16]\AgdaFunction{[} \AgdaFunction{fm} \AgdaSymbol{(}\AgdaFunction{fn} \AgdaSymbol{\_)} \AgdaFunction{]trans} \AgdaSymbol{(}\AgdaFunction{trans*} \AgdaSymbol{\_} \AgdaSymbol{\_} \AgdaSymbol{\_)} \AgdaFunction{subst-pi}\<%
\\
\>[0]\AgdaIndent{5}{}\<[5]%
\>[5]\AgdaSymbol{\}}\<%
\\
\>[0]\AgdaIndent{5}{}\<[5]%
\>[5]\AgdaKeyword{where} \<[11]%
\>[11]\<%
\\
\>[5]\AgdaIndent{7}{}\<[7]%
\>[7]\AgdaKeyword{open} \AgdaModule{Ty} \AgdaBound{A}\<%
\\
\>[5]\AgdaIndent{7}{}\<[7]%
\>[7]\AgdaKeyword{open} \AgdaModule{\_⇉\_} \AgdaBound{f}\<%
\\
%
\\
\>\<\end{code}

The semantic terms are simpler. It should also preserve the equivalence relation on the elements of contexts.

\begin{code}\>\<%
\\
%
\\
\>\AgdaKeyword{record} \AgdaRecord{Tm} \AgdaSymbol{\{}\AgdaBound{Γ} \AgdaSymbol{:} \AgdaFunction{Con}\AgdaSymbol{\}(}\AgdaBound{A} \AgdaSymbol{:} \AgdaRecord{Ty} \AgdaBound{Γ}\AgdaSymbol{)} \AgdaSymbol{:} \AgdaPrimitiveType{Set} \AgdaKeyword{where}\<%
\\
\>[0]\AgdaIndent{2}{}\<[2]%
\>[2]\AgdaKeyword{constructor} \AgdaInductiveConstructor{tm:\_resp:\_}\<%
\\
\>[0]\AgdaIndent{2}{}\<[2]%
\>[2]\AgdaKeyword{field}\<%
\\
\>[2]\AgdaIndent{4}{}\<[4]%
\>[4]\AgdaField{tm} \<[10]%
\>[10]\AgdaSymbol{:} \AgdaSymbol{(}\AgdaBound{x} \AgdaSymbol{:} \AgdaFunction{∣} \AgdaBound{Γ} \AgdaFunction{∣}\AgdaSymbol{)} \AgdaSymbol{→} \AgdaFunction{∣} \AgdaFunction{[} \AgdaBound{A} \AgdaFunction{]fm} \AgdaBound{x} \AgdaFunction{∣}\<%
\\
\>[2]\AgdaIndent{4}{}\<[4]%
\>[4]\AgdaField{respt} \AgdaSymbol{:} \AgdaSymbol{∀} \AgdaSymbol{\{}\AgdaBound{x} \AgdaBound{y} \AgdaSymbol{:} \AgdaFunction{∣} \AgdaBound{Γ} \AgdaFunction{∣}\AgdaSymbol{\}} \AgdaSymbol{→} \<[30]%
\>[30]\<%
\\
\>[4]\AgdaIndent{14}{}\<[14]%
\>[14]\AgdaSymbol{(}\AgdaBound{p} \AgdaSymbol{:} \AgdaFunction{[} \AgdaBound{Γ} \AgdaFunction{]} \AgdaBound{x} \AgdaFunction{≈} \AgdaBound{y}\AgdaSymbol{)} \AgdaSymbol{→} \<[34]%
\>[34]\<%
\\
\>[4]\AgdaIndent{14}{}\<[14]%
\>[14]\AgdaFunction{[} \AgdaFunction{[} \AgdaBound{A} \AgdaFunction{]fm} \AgdaBound{y} \AgdaFunction{]} \AgdaFunction{[} \AgdaBound{A} \AgdaFunction{]subst} \AgdaBound{p} \AgdaSymbol{(}\AgdaBound{tm} \AgdaBound{x}\AgdaSymbol{)} \AgdaFunction{≈} \AgdaBound{tm} \AgdaBound{y}\<%
\\
%
\\
\>\<\end{code}

\AgdaHide{
\begin{code}\>\<%
\\
\>\AgdaKeyword{open} \AgdaModule{Tm} \AgdaKeyword{public} \AgdaKeyword{renaming} \AgdaSymbol{(}tm \AgdaSymbol{to} [\_]tm \AgdaSymbol{;} respt \AgdaSymbol{to} [\_]respt\AgdaSymbol{)}\<%
\\
%
\\
\>\<\end{code}
}

Substitution for terms can be defined as

\begin{code}\>\<%
\\
%
\\
\>\AgdaFunction{\_[\_]m} \AgdaSymbol{:} \AgdaSymbol{∀} \AgdaSymbol{\{}\AgdaBound{Γ} \AgdaBound{Δ} \AgdaSymbol{:} \AgdaFunction{Con}\AgdaSymbol{\}\{}\AgdaBound{A} \AgdaSymbol{:} \AgdaRecord{Ty} \AgdaBound{Δ}\AgdaSymbol{\}} \AgdaSymbol{→} \<[34]%
\>[34]\<%
\\
\>[0]\AgdaIndent{10}{}\<[10]%
\>[10]\AgdaRecord{Tm} \AgdaBound{A} \AgdaSymbol{→} \<[17]%
\>[17]\<%
\\
\>[0]\AgdaIndent{10}{}\<[10]%
\>[10]\AgdaSymbol{(}\AgdaBound{f} \AgdaSymbol{:} \AgdaBound{Γ} \AgdaRecord{⇉} \AgdaBound{Δ}\AgdaSymbol{)} \<[22]%
\>[22]\<%
\\
\>[0]\AgdaIndent{10}{}\<[10]%
\>[10]\AgdaSymbol{→} \AgdaRecord{Tm} \AgdaSymbol{(}\AgdaBound{A} \AgdaFunction{[} \AgdaBound{f} \AgdaFunction{]T}\AgdaSymbol{)}\<%
\\
\>\AgdaFunction{\_[\_]m} \AgdaBound{t} \AgdaBound{f} \AgdaSymbol{=} \AgdaKeyword{record} \<[19]%
\>[19]\<%
\\
\>[0]\AgdaIndent{10}{}\<[10]%
\>[10]\AgdaSymbol{\{} \AgdaField{tm} \AgdaSymbol{=} \AgdaFunction{[} \AgdaBound{t} \AgdaFunction{]tm} \AgdaFunction{∘} \AgdaFunction{[} \AgdaBound{f} \AgdaFunction{]fn}\<%
\\
\>[0]\AgdaIndent{10}{}\<[10]%
\>[10]\AgdaSymbol{;} \AgdaField{respt} \AgdaSymbol{=} \AgdaFunction{[} \AgdaBound{t} \AgdaFunction{]respt} \AgdaFunction{∘} \AgdaFunction{[} \AgdaBound{f} \AgdaFunction{]resp} \<[43]%
\>[43]\<%
\\
\>[0]\AgdaIndent{10}{}\<[10]%
\>[10]\AgdaSymbol{\}}\<%
\\
%
\\
\>\<\end{code}

Syntactically we can form a new context by using a context $\Gamma$ and a type $A : Ty \:\Gamma$. To introduce a term of it, we need a term of the semantic context $\Gamma$ and a term of semantic type $A$. It is called context comprehension. 

\begin{code}\>\<%
\\
%
\\
\>\AgdaFunction{\_\&\_} \AgdaSymbol{:} \AgdaSymbol{(}\AgdaBound{Γ} \AgdaSymbol{:} \AgdaFunction{Con}\AgdaSymbol{)} \AgdaSymbol{→} \AgdaRecord{Ty} \AgdaBound{Γ} \AgdaSymbol{→} \AgdaFunction{Con}\<%
\\
\>\AgdaBound{Γ} \AgdaFunction{\&} \AgdaBound{A} \AgdaSymbol{=} \AgdaKeyword{record} \<[15]%
\>[15]\<%
\\
\>[0]\AgdaIndent{7}{}\<[7]%
\>[7]\AgdaSymbol{\{} \AgdaField{Carrier} \AgdaSymbol{=} \AgdaRecord{Σ[} \AgdaBound{x} \AgdaRecord{∶} \AgdaFunction{∣} \AgdaBound{Γ} \AgdaFunction{∣} \AgdaRecord{]} \AgdaFunction{∣} \AgdaFunction{fm} \AgdaBound{x} \AgdaFunction{∣}\<%
\\
\>[0]\AgdaIndent{7}{}\<[7]%
\>[7]\AgdaSymbol{;} \AgdaField{\_≈h\_} \<[17]%
\>[17]\AgdaSymbol{=} \AgdaSymbol{λ\{(}\AgdaBound{x} \AgdaInductiveConstructor{,} \AgdaBound{a}\AgdaSymbol{)} \AgdaSymbol{(}\AgdaBound{y} \AgdaInductiveConstructor{,} \AgdaBound{b}\AgdaSymbol{)} \AgdaSymbol{→} \<[39]%
\>[39]\<%
\\
\>[7]\AgdaIndent{19}{}\<[19]%
\>[19]\AgdaFunction{Σ'[} \AgdaBound{p} \AgdaFunction{∶} \AgdaBound{x} \AgdaFunction{≈h} \AgdaBound{y} \AgdaFunction{]} \AgdaFunction{[} \AgdaFunction{fm} \AgdaBound{y} \AgdaFunction{]} \AgdaFunction{substT} \AgdaBound{p} \AgdaBound{a} \AgdaFunction{≈h} \AgdaBound{b}\AgdaSymbol{\}}\<%
\\
\>[0]\AgdaIndent{7}{}\<[7]%
\>[7]\AgdaSymbol{;} \AgdaField{isEquiv} \AgdaSymbol{=} \<[19]%
\>[19]\<%
\\
\>[0]\AgdaIndent{10}{}\<[10]%
\>[10]\AgdaKeyword{record} \<[17]%
\>[17]\<%
\\
\>[0]\AgdaIndent{10}{}\<[10]%
\>[10]\AgdaSymbol{\{} \AgdaField{refl} \<[18]%
\>[18]\AgdaSymbol{=} \AgdaFunction{refl} \AgdaInductiveConstructor{,} \AgdaSymbol{(}\AgdaFunction{refl*} \AgdaSymbol{\_} \AgdaSymbol{\_)}\<%
\\
\>[0]\AgdaIndent{10}{}\<[10]%
\>[10]\AgdaSymbol{;} \AgdaField{sym} \<[18]%
\>[18]\AgdaSymbol{=} \AgdaSymbol{λ} \AgdaSymbol{\{(}\AgdaBound{p} \AgdaInductiveConstructor{,} \AgdaBound{q}\AgdaSymbol{)} \AgdaSymbol{→} \AgdaSymbol{(}\AgdaFunction{sym} \AgdaBound{p}\AgdaSymbol{)} \AgdaInductiveConstructor{,} \<[43]%
\>[43]\<%
\\
\>[10]\AgdaIndent{20}{}\<[20]%
\>[20]\AgdaFunction{[} \AgdaFunction{fm} \AgdaSymbol{\_} \AgdaFunction{]trans} \<[34]%
\>[34]\<%
\\
\>[10]\AgdaIndent{20}{}\<[20]%
\>[20]\AgdaSymbol{(}\AgdaFunction{subst*} \AgdaSymbol{\_} \AgdaSymbol{(}\AgdaFunction{[} \AgdaFunction{fm} \AgdaSymbol{\_} \AgdaFunction{]sym} \AgdaBound{q}\AgdaSymbol{))} \<[47]%
\>[47]\<%
\\
\>[10]\AgdaIndent{20}{}\<[20]%
\>[20]\AgdaFunction{trans-refl} \AgdaSymbol{\}}\<%
\\
\>[0]\AgdaIndent{10}{}\<[10]%
\>[10]\AgdaSymbol{;} \AgdaField{trans} \AgdaSymbol{=} \AgdaSymbol{λ} \AgdaSymbol{\{(}\AgdaBound{p} \AgdaInductiveConstructor{,} \AgdaBound{q}\AgdaSymbol{)} \AgdaSymbol{(}\AgdaBound{m} \AgdaInductiveConstructor{,} \AgdaBound{n}\AgdaSymbol{)} \AgdaSymbol{→}\<%
\\
\>[0]\AgdaIndent{20}{}\<[20]%
\>[20]\AgdaFunction{trans} \AgdaBound{p} \AgdaBound{m} \AgdaInductiveConstructor{,} \<[32]%
\>[32]\<%
\\
\>[0]\AgdaIndent{20}{}\<[20]%
\>[20]\AgdaFunction{[} \AgdaFunction{fm} \AgdaSymbol{\_} \AgdaFunction{]trans} \<[34]%
\>[34]\<%
\\
\>[0]\AgdaIndent{20}{}\<[20]%
\>[20]\AgdaSymbol{(}\AgdaFunction{[} \AgdaFunction{fm} \AgdaSymbol{\_} \AgdaFunction{]trans} \<[35]%
\>[35]\<%
\\
\>[0]\AgdaIndent{20}{}\<[20]%
\>[20]\AgdaSymbol{(}\AgdaFunction{[} \AgdaFunction{fm} \AgdaSymbol{\_} \AgdaFunction{]sym} \AgdaSymbol{(}\AgdaFunction{trans*} \AgdaSymbol{\_} \AgdaSymbol{\_} \AgdaSymbol{\_))} \AgdaSymbol{(}\AgdaFunction{subst*} \AgdaSymbol{\_} \AgdaBound{q}\AgdaSymbol{))} \AgdaBound{n} \AgdaSymbol{\}}\<%
\\
\>[0]\AgdaIndent{10}{}\<[10]%
\>[10]\AgdaSymbol{\}}\<%
\\
\>[0]\AgdaIndent{7}{}\<[7]%
\>[7]\AgdaSymbol{\}}\<%
\\
%
\\
%
\\
\>\<\end{code}

\AgdaHide{
\begin{code}\>\<%
\\
%
\\
\>[0]\AgdaIndent{7}{}\<[7]%
\>[7]\AgdaKeyword{where} \<[13]%
\>[13]\<%
\\
\>[7]\AgdaIndent{9}{}\<[9]%
\>[9]\AgdaKeyword{open} \AgdaModule{hSetoid} \AgdaBound{Γ}\<%
\\
\>[7]\AgdaIndent{9}{}\<[9]%
\>[9]\AgdaKeyword{open} \AgdaModule{Ty} \AgdaBound{A} \<[23]%
\>[23]\<%
\\
%
\\
\>\<\end{code}
}

There are also some other morphisms come with it. Any morphism from a context $\Gamma$ to a context $\Delta \& A$ consists of a morphism from $\Gamma$ to $\Delta$ and a term of type $A$ substituted. In other words, There is an isomorphism between $Hom(\Gamma , \Delta \& A)$ and $\Sigma \gamma : Hom(\Gamma , \Delta) A [ \gamma ] $.

$fst$ projects the morphism and  $snd$ projects the term.
Indeed the $fst$ operation provides weakening for types, and the $snd$ projection enables us to interpret variables. $fst\&$ defines a morphism for each type $A$ which is a canonical projection of $A$.
We need to use $id'$ which are identity context morphisms to achieve these.

\begin{code}\>\<%
\\
%
\\
\>\AgdaFunction{fst} \AgdaSymbol{:} \AgdaSymbol{\{}\AgdaBound{Γ} \AgdaBound{Δ} \AgdaSymbol{:} \AgdaFunction{Con}\AgdaSymbol{\}(}\AgdaBound{A} \AgdaSymbol{:} \AgdaRecord{Ty} \AgdaBound{Δ}\AgdaSymbol{)} \AgdaSymbol{→} \AgdaBound{Γ} \AgdaRecord{⇉} \AgdaSymbol{(}\AgdaBound{Δ} \AgdaFunction{\&} \AgdaBound{A}\AgdaSymbol{)} \AgdaSymbol{→} \AgdaBound{Γ} \AgdaRecord{⇉} \AgdaBound{Δ}\<%
\\
\>\AgdaFunction{fst} \AgdaBound{A} \AgdaBound{f} \AgdaSymbol{=} \AgdaKeyword{record} \<[17]%
\>[17]\<%
\\
\>[-6]\AgdaIndent{8}{}\<[8]%
\>[8]\AgdaSymbol{\{} \AgdaField{fn} \AgdaSymbol{=} \AgdaFunction{proj₁} \AgdaFunction{∘} \AgdaFunction{[} \AgdaBound{f} \AgdaFunction{]fn}\<%
\\
\>[0]\AgdaIndent{8}{}\<[8]%
\>[8]\AgdaSymbol{;} \AgdaField{resp} \AgdaSymbol{=} \AgdaFunction{proj₁} \AgdaFunction{∘} \AgdaFunction{[} \AgdaBound{f} \AgdaFunction{]resp} \<[35]%
\>[35]\<%
\\
\>[0]\AgdaIndent{8}{}\<[8]%
\>[8]\AgdaSymbol{\}}\<%
\\
%
\\
\>\AgdaFunction{fst\&} \AgdaSymbol{:} \AgdaSymbol{\{}\AgdaBound{Γ} \AgdaSymbol{:} \AgdaFunction{Con}\AgdaSymbol{\}(}\AgdaBound{A} \AgdaSymbol{:} \AgdaRecord{Ty} \AgdaBound{Γ}\AgdaSymbol{)} \AgdaSymbol{→} \AgdaBound{Γ} \AgdaFunction{\&} \AgdaBound{A} \AgdaRecord{⇉} \AgdaBound{Γ}\<%
\\
\>\AgdaFunction{fst\&} \AgdaBound{A} \AgdaSymbol{=} \AgdaFunction{fst} \AgdaBound{A} \AgdaFunction{id'}\<%
\\
%
\\
\>\AgdaFunction{\_+T\_} \AgdaSymbol{:} \AgdaSymbol{\{}\AgdaBound{Γ} \AgdaSymbol{:} \AgdaFunction{Con}\AgdaSymbol{\}} \AgdaSymbol{→} \AgdaRecord{Ty} \AgdaBound{Γ} \AgdaSymbol{→} \AgdaSymbol{(}\AgdaBound{A} \AgdaSymbol{:} \AgdaRecord{Ty} \AgdaBound{Γ}\AgdaSymbol{)} \AgdaSymbol{→} \AgdaRecord{Ty} \AgdaSymbol{(}\AgdaBound{Γ} \AgdaFunction{\&} \AgdaBound{A}\AgdaSymbol{)}\<%
\\
\>\AgdaBound{B} \AgdaFunction{+T} \AgdaBound{A} \AgdaSymbol{=} \AgdaBound{B} \AgdaFunction{[} \AgdaFunction{fst\&} \AgdaBound{A} \AgdaFunction{]T}\<%
\\
%
\\
\>\AgdaFunction{snd} \AgdaSymbol{:} \AgdaSymbol{\{}\AgdaBound{Γ} \AgdaBound{Δ} \AgdaSymbol{:} \AgdaFunction{Con}\AgdaSymbol{\}(}\AgdaBound{A} \AgdaSymbol{:} \AgdaRecord{Ty} \AgdaBound{Δ}\AgdaSymbol{)} \AgdaSymbol{→} \<[30]%
\>[30]\<%
\\
\>[0]\AgdaIndent{6}{}\<[6]%
\>[6]\AgdaSymbol{(}\AgdaBound{f} \AgdaSymbol{:} \AgdaBound{Γ} \AgdaRecord{⇉} \AgdaSymbol{(}\AgdaBound{Δ} \AgdaFunction{\&} \AgdaBound{A}\AgdaSymbol{))} \<[24]%
\>[24]\<%
\\
\>[0]\AgdaIndent{6}{}\<[6]%
\>[6]\AgdaSymbol{→} \AgdaRecord{Tm} \AgdaSymbol{(}\AgdaBound{A} \AgdaFunction{[} \AgdaFunction{fst} \AgdaBound{A} \AgdaBound{f} \AgdaFunction{]T}\AgdaSymbol{)}\<%
\\
\>\AgdaFunction{snd} \AgdaBound{A} \AgdaBound{f} \AgdaSymbol{=} \AgdaKeyword{record} \<[17]%
\>[17]\<%
\\
\>[6]\AgdaIndent{8}{}\<[8]%
\>[8]\AgdaSymbol{\{} \AgdaField{tm} \AgdaSymbol{=} \AgdaFunction{proj₂} \AgdaFunction{∘} \AgdaFunction{[} \AgdaBound{f} \AgdaFunction{]fn}\<%
\\
\>[6]\AgdaIndent{8}{}\<[8]%
\>[8]\AgdaSymbol{;} \AgdaField{respt} \AgdaSymbol{=} \AgdaFunction{proj₂} \AgdaFunction{∘} \AgdaFunction{[} \AgdaBound{f} \AgdaFunction{]resp} \<[36]%
\>[36]\<%
\\
\>[6]\AgdaIndent{8}{}\<[8]%
\>[8]\AgdaSymbol{\}}\<%
\\
%
\\
\>\AgdaFunction{v0} \AgdaSymbol{:} \AgdaSymbol{\{}\AgdaBound{Γ} \AgdaSymbol{:} \AgdaFunction{Con}\AgdaSymbol{\}(}\AgdaBound{A} \AgdaSymbol{:} \AgdaRecord{Ty} \AgdaBound{Γ}\AgdaSymbol{)} \AgdaSymbol{→} \AgdaRecord{Tm} \AgdaSymbol{(}\AgdaBound{A} \AgdaFunction{+T} \AgdaBound{A}\AgdaSymbol{)}\<%
\\
\>\AgdaFunction{v0} \AgdaBound{A} \AgdaSymbol{=} \AgdaFunction{snd} \AgdaBound{A} \AgdaFunction{id'}\<%
\\
%
\\
\>\<\end{code}

Inversely we could define a pairing operation to combine a context morphism with a term. The $\eta$-law for the projection and pairing holds trivially.

\begin{code}\>\<%
\\
%
\\
\>\AgdaFunction{\_,,\_} \AgdaSymbol{:} \AgdaSymbol{\{}\AgdaBound{Γ} \AgdaBound{Δ} \AgdaSymbol{:} \AgdaFunction{Con}\AgdaSymbol{\}\{}\AgdaBound{A} \AgdaSymbol{:} \AgdaRecord{Ty} \AgdaBound{Δ}\AgdaSymbol{\}(}\AgdaBound{f} \AgdaSymbol{:} \AgdaBound{Γ} \AgdaRecord{⇉} \AgdaBound{Δ}\AgdaSymbol{)} \AgdaSymbol{→} \<[42]%
\>[42]\<%
\\
\>[-5]\AgdaIndent{7}{}\<[7]%
\>[7]\AgdaSymbol{(}\AgdaRecord{Tm} \AgdaSymbol{(}\AgdaBound{A} \AgdaFunction{[} \AgdaBound{f} \AgdaFunction{]T}\AgdaSymbol{))} \<[23]%
\>[23]\<%
\\
\>[0]\AgdaIndent{7}{}\<[7]%
\>[7]\AgdaSymbol{→} \AgdaBound{Γ} \AgdaRecord{⇉} \AgdaSymbol{(}\AgdaBound{Δ} \AgdaFunction{\&} \AgdaBound{A}\AgdaSymbol{)}\<%
\\
\>\AgdaBound{f} \AgdaFunction{,,} \AgdaBound{t} \AgdaSymbol{=} \AgdaKeyword{record} \<[16]%
\>[16]\<%
\\
\>[7]\AgdaIndent{9}{}\<[9]%
\>[9]\AgdaSymbol{\{} \AgdaField{fn} \AgdaSymbol{=} \AgdaFunction{⟨} \AgdaFunction{[} \AgdaBound{f} \AgdaFunction{]fn} \AgdaFunction{,} \AgdaFunction{[} \AgdaBound{t} \AgdaFunction{]tm} \AgdaFunction{⟩}\<%
\\
\>[7]\AgdaIndent{9}{}\<[9]%
\>[9]\AgdaSymbol{;} \AgdaField{resp} \AgdaSymbol{=} \AgdaFunction{⟨} \AgdaFunction{[} \AgdaBound{f} \AgdaFunction{]resp} \AgdaFunction{,} \AgdaFunction{[} \AgdaBound{t} \AgdaFunction{]respt} \AgdaFunction{⟩}\<%
\\
\>[7]\AgdaIndent{9}{}\<[9]%
\>[9]\AgdaSymbol{\}}\<%
\\
%
\\
\>\AgdaFunction{\&-eta} \AgdaSymbol{:} \AgdaSymbol{\{}\AgdaBound{Γ} \AgdaBound{Δ} \AgdaSymbol{:} \AgdaFunction{Con}\AgdaSymbol{\}\{}\AgdaBound{A} \AgdaSymbol{:} \AgdaRecord{Ty} \AgdaBound{Δ}\AgdaSymbol{\}(}\AgdaBound{f} \AgdaSymbol{:} \AgdaBound{Γ} \AgdaRecord{⇉} \AgdaSymbol{(}\AgdaBound{Δ} \AgdaFunction{\&} \AgdaBound{A}\AgdaSymbol{))} \<[47]%
\>[47]\<%
\\
\>[-4]\AgdaIndent{6}{}\<[6]%
\>[6]\AgdaSymbol{→} \AgdaFunction{\_,,\_} \AgdaSymbol{\{}A \AgdaSymbol{=} \AgdaBound{A}\AgdaSymbol{\}} \AgdaSymbol{(}\AgdaFunction{fst} \AgdaBound{A} \AgdaBound{f}\AgdaSymbol{)} \AgdaSymbol{(}\AgdaFunction{snd} \AgdaBound{A} \AgdaBound{f}\AgdaSymbol{)} \AgdaDatatype{≡} \AgdaBound{f}\<%
\\
\>\AgdaFunction{\&-eta} \AgdaBound{f} \AgdaSymbol{=} \AgdaInductiveConstructor{PE.refl}\<%
\\
%
\\
%
\\
\>\<\end{code}

Then a lifting operation could help us define $\Pi$-types.

\begin{code}\>\<%
\\
%
\\
\>\AgdaFunction{lift} \AgdaSymbol{:} \AgdaSymbol{\{}\AgdaBound{Γ} \AgdaBound{Δ} \AgdaSymbol{:} \AgdaFunction{Con}\AgdaSymbol{\}(}\AgdaBound{f} \AgdaSymbol{:} \AgdaBound{Γ} \AgdaRecord{⇉} \AgdaBound{Δ}\AgdaSymbol{)(}\AgdaBound{A} \AgdaSymbol{:} \AgdaRecord{Ty} \AgdaBound{Δ}\AgdaSymbol{)} \AgdaSymbol{→} \AgdaBound{Γ} \AgdaFunction{\&} \AgdaBound{A} \AgdaFunction{[} \AgdaBound{f} \AgdaFunction{]T} \AgdaRecord{⇉} \AgdaBound{Δ} \AgdaFunction{\&} \AgdaBound{A}\<%
\\
\>\AgdaFunction{lift} \AgdaBound{f} \AgdaBound{A} \AgdaSymbol{=} \AgdaKeyword{record} \<[18]%
\>[18]\<%
\\
\>[0]\AgdaIndent{8}{}\<[8]%
\>[8]\AgdaSymbol{\{} \AgdaField{fn} \AgdaSymbol{=} \AgdaFunction{⟨} \AgdaFunction{[} \AgdaBound{f} \AgdaFunction{]fn} \AgdaFunction{∘} \AgdaFunction{proj₁} \AgdaFunction{,} \AgdaFunction{proj₂} \AgdaFunction{⟩}\<%
\\
\>[0]\AgdaIndent{8}{}\<[8]%
\>[8]\AgdaSymbol{;} \AgdaField{resp} \AgdaSymbol{=} \AgdaFunction{⟨} \AgdaFunction{[} \AgdaBound{f} \AgdaFunction{]resp} \AgdaFunction{∘} \AgdaFunction{proj₁} \AgdaFunction{,} \AgdaFunction{proj₂} \AgdaFunction{⟩}\<%
\\
\>[0]\AgdaIndent{8}{}\<[8]%
\>[8]\AgdaSymbol{\}}\<%
\\
%
\\
\>\AgdaFunction{lift-eta} \AgdaSymbol{:} \AgdaSymbol{\{}\AgdaBound{Γ} \AgdaBound{Δ} \AgdaSymbol{:} \AgdaFunction{Con}\AgdaSymbol{\}}\<%
\\
\>[8]\AgdaIndent{9}{}\<[9]%
\>[9]\AgdaSymbol{(}\AgdaBound{f} \AgdaSymbol{:} \AgdaBound{Γ} \AgdaRecord{⇉} \AgdaBound{Δ}\AgdaSymbol{)(}\AgdaBound{A} \AgdaSymbol{:} \AgdaRecord{Ty} \AgdaBound{Δ}\AgdaSymbol{)(}\AgdaBound{x} \AgdaSymbol{:} \AgdaFunction{∣} \AgdaBound{Γ} \AgdaFunction{∣}\AgdaSymbol{)}\<%
\\
\>[8]\AgdaIndent{9}{}\<[9]%
\>[9]\AgdaSymbol{(}\AgdaBound{a} \AgdaSymbol{:} \AgdaFunction{∣} \AgdaFunction{[} \AgdaBound{A} \AgdaFunction{]fm} \AgdaSymbol{(}\AgdaFunction{[} \AgdaBound{f} \AgdaFunction{]fn} \AgdaBound{x}\AgdaSymbol{)} \AgdaFunction{∣}\AgdaSymbol{)} \<[39]%
\>[39]\<%
\\
\>[8]\AgdaIndent{9}{}\<[9]%
\>[9]\AgdaSymbol{→} \AgdaFunction{[} \AgdaFunction{lift} \AgdaBound{f} \AgdaBound{A} \AgdaFunction{]fn} \AgdaSymbol{(}\AgdaBound{x} \AgdaInductiveConstructor{,} \AgdaBound{a}\AgdaSymbol{)} \AgdaDatatype{≡} \AgdaSymbol{(} \AgdaFunction{[} \AgdaBound{f} \AgdaFunction{]fn} \AgdaBound{x} \AgdaInductiveConstructor{,} \AgdaBound{a}\AgdaSymbol{)}\<%
\\
\>\AgdaFunction{lift-eta} \AgdaBound{f} \AgdaBound{A} \AgdaBound{x} \AgdaBound{a} \AgdaSymbol{=} \AgdaInductiveConstructor{PE.refl}\<%
\\
%
\\
%
\\
\>\<\end{code}

One of the most complicated part of this definition is the $\Pi$-types.
$\Pi$-types is also called dependent function types. Semantically it is a function type on the underlying semantic types with a proof that the the functions respect the equivalence relation. 

%f-resp on the paper ignores refl*

\begin{code}\>\<%
\\
%
\\
\>\AgdaFunction{Π} \AgdaSymbol{:} \AgdaSymbol{\{}\AgdaBound{Γ} \AgdaSymbol{:} \AgdaFunction{Con}\AgdaSymbol{\}(}\AgdaBound{A} \AgdaSymbol{:} \AgdaRecord{Ty} \AgdaBound{Γ}\AgdaSymbol{)(}\AgdaBound{B} \AgdaSymbol{:} \AgdaRecord{Ty} \AgdaSymbol{(}\AgdaBound{Γ} \AgdaFunction{\&} \AgdaBound{A}\AgdaSymbol{))} \AgdaSymbol{→} \AgdaRecord{Ty} \AgdaBound{Γ}\<%
\\
%
\\
\>\<\end{code}

\AgdaHide{
\begin{code}\>\<%
\\
%
\\
\>\AgdaFunction{Π} \AgdaSymbol{\{}\AgdaBound{Γ}\AgdaSymbol{\}} \AgdaBound{A} \AgdaBound{B} \AgdaSymbol{=} \AgdaKeyword{record} \<[19]%
\>[19]\<%
\\
\>[-1]\AgdaIndent{2}{}\<[2]%
\>[2]\AgdaSymbol{\{} \AgdaField{fm} \AgdaSymbol{=} \AgdaSymbol{λ} \AgdaBound{x} \AgdaSymbol{→} \AgdaKeyword{let} \AgdaBound{Ax} \AgdaSymbol{=} \AgdaFunction{[} \AgdaBound{A} \AgdaFunction{]fm} \AgdaBound{x} \AgdaKeyword{in}\<%
\\
\>[0]\AgdaIndent{15}{}\<[15]%
\>[15]\AgdaKeyword{let} \AgdaBound{Bx} \AgdaSymbol{=} \AgdaSymbol{λ} \AgdaBound{a} \AgdaSymbol{→} \AgdaFunction{[} \AgdaBound{B} \AgdaFunction{]fm} \AgdaSymbol{(}\AgdaBound{x} \AgdaInductiveConstructor{,} \AgdaBound{a}\AgdaSymbol{)} \AgdaKeyword{in}\<%
\\
\>[0]\AgdaIndent{9}{}\<[9]%
\>[9]\AgdaKeyword{record}\<%
\\
\>[0]\AgdaIndent{9}{}\<[9]%
\>[9]\AgdaSymbol{\{} \AgdaField{Carrier} \AgdaSymbol{=} \AgdaRecord{Σ[} \AgdaBound{fn} \AgdaRecord{∶} \AgdaSymbol{((}\AgdaBound{a} \AgdaSymbol{:} \AgdaFunction{∣} \AgdaBound{Ax} \AgdaFunction{∣}\AgdaSymbol{)} \AgdaSymbol{→} \AgdaFunction{∣} \AgdaBound{Bx} \AgdaBound{a} \AgdaFunction{∣}\AgdaSymbol{)} \AgdaRecord{]}\<%
\\
\>[9]\AgdaIndent{21}{}\<[21]%
\>[21]\AgdaSymbol{((}\AgdaBound{a} \AgdaBound{b} \AgdaSymbol{:} \AgdaFunction{∣} \AgdaBound{Ax} \AgdaFunction{∣}\AgdaSymbol{)}\<%
\\
\>[9]\AgdaIndent{21}{}\<[21]%
\>[21]\AgdaSymbol{(}\AgdaBound{p} \AgdaSymbol{:} \AgdaFunction{[} \AgdaBound{Ax} \AgdaFunction{]} \AgdaBound{a} \AgdaFunction{≈} \AgdaBound{b}\AgdaSymbol{)} \AgdaSymbol{→}\<%
\\
\>[9]\AgdaIndent{21}{}\<[21]%
\>[21]\AgdaFunction{[} \AgdaBound{Bx} \AgdaBound{b} \AgdaFunction{]} \AgdaFunction{[} \AgdaBound{B} \AgdaFunction{]subst} \AgdaSymbol{(}\AgdaFunction{[} \AgdaBound{Γ} \AgdaFunction{]refl} \AgdaInductiveConstructor{,}\<%
\\
\>[-7]\AgdaIndent{19}{}\<[19]%
\>[19]\AgdaFunction{[} \AgdaBound{Ax} \AgdaFunction{]trans} \AgdaSymbol{(}\AgdaFunction{[} \AgdaBound{A} \AgdaFunction{]refl*} \AgdaBound{x} \AgdaBound{a}\AgdaSymbol{)} \AgdaBound{p}\AgdaSymbol{)} \AgdaSymbol{(}\AgdaBound{fn} \AgdaBound{a}\AgdaSymbol{)} \AgdaFunction{≈} \AgdaBound{fn} \AgdaBound{b}\AgdaSymbol{)} \<[66]%
\>[66]\<%
\\
\>[0]\AgdaIndent{1}{}\<[1]%
\>[1]\<%
\\
\>[1]\AgdaIndent{9}{}\<[9]%
\>[9]\AgdaSymbol{;} \AgdaField{\_≈h\_} \<[19]%
\>[19]\AgdaSymbol{=} \AgdaSymbol{λ\{(}\AgdaBound{f} \AgdaInductiveConstructor{,} \AgdaSymbol{\_)} \AgdaSymbol{(}\AgdaBound{g} \AgdaInductiveConstructor{,} \AgdaSymbol{\_)} \AgdaSymbol{→} \<[41]%
\>[41]\<%
\\
\>[9]\AgdaIndent{24}{}\<[24]%
\>[24]\AgdaFunction{∀'[} \AgdaBound{a} \AgdaFunction{∶} \AgdaSymbol{\_} \AgdaFunction{]} \AgdaFunction{[} \AgdaBound{Bx} \AgdaBound{a} \AgdaFunction{]} \AgdaBound{f} \AgdaBound{a} \AgdaFunction{≈h} \AgdaBound{g} \AgdaBound{a} \AgdaSymbol{\}}\<%
\\
\>[0]\AgdaIndent{9}{}\<[9]%
\>[9]\AgdaSymbol{;} \AgdaField{isEquiv} \AgdaSymbol{=} \AgdaKeyword{record} \AgdaSymbol{\{}\<%
\\
\>[0]\AgdaIndent{19}{}\<[19]%
\>[19]\AgdaField{refl} \<[25]%
\>[25]\AgdaSymbol{=} \AgdaSymbol{λ} \AgdaBound{a} \AgdaSymbol{→} \AgdaFunction{[} \AgdaBound{Bx} \AgdaBound{a} \AgdaFunction{]refl} \<[46]%
\>[46]\<%
\\
\>[0]\AgdaIndent{17}{}\<[17]%
\>[17]\AgdaSymbol{;} \AgdaField{sym} \<[25]%
\>[25]\AgdaSymbol{=} \AgdaSymbol{λ} \AgdaBound{f} \AgdaBound{a} \AgdaSymbol{→} \AgdaFunction{[} \AgdaBound{Bx} \AgdaBound{a} \AgdaFunction{]sym} \AgdaSymbol{(}\AgdaBound{f} \AgdaBound{a}\AgdaSymbol{)}\<%
\\
\>[0]\AgdaIndent{17}{}\<[17]%
\>[17]\AgdaSymbol{;} \AgdaField{trans} \AgdaSymbol{=} \AgdaSymbol{λ} \AgdaBound{f} \AgdaBound{g} \AgdaBound{a} \AgdaSymbol{→} \AgdaFunction{[} \AgdaBound{Bx} \AgdaBound{a} \AgdaFunction{]trans} \AgdaSymbol{(}\AgdaBound{f} \AgdaBound{a}\AgdaSymbol{)} \AgdaSymbol{(}\AgdaBound{g} \AgdaBound{a}\AgdaSymbol{)}\<%
\\
\>[17]\AgdaIndent{29}{}\<[29]%
\>[29]\AgdaSymbol{\}}\<%
\\
\>[3]\AgdaIndent{9}{}\<[9]%
\>[9]\AgdaSymbol{\}}\<%
\\
%
\\
\>[0]\AgdaIndent{2}{}\<[2]%
\>[2]\AgdaSymbol{;} \AgdaField{substT} \AgdaSymbol{=} \AgdaSymbol{λ} \AgdaSymbol{\{}\AgdaBound{x}\AgdaSymbol{\}} \AgdaSymbol{\{}\AgdaBound{y}\AgdaSymbol{\}} \AgdaBound{p} \AgdaSymbol{→}\<%
\\
\>[2]\AgdaIndent{19}{}\<[19]%
\>[19]\AgdaKeyword{let} \AgdaBound{y2x} \AgdaSymbol{=} \AgdaSymbol{λ} \AgdaBound{a} \AgdaSymbol{→} \AgdaFunction{[} \AgdaBound{A} \AgdaFunction{]subst} \AgdaSymbol{(}\AgdaFunction{[} \AgdaBound{Γ} \AgdaFunction{]sym} \AgdaBound{p}\AgdaSymbol{)} \AgdaBound{a} \AgdaKeyword{in}\<%
\\
\>[19]\AgdaIndent{20}{}\<[20]%
\>[20]\AgdaKeyword{let} \AgdaBound{x2y} \AgdaSymbol{=} \AgdaSymbol{λ} \AgdaBound{a} \AgdaSymbol{→} \AgdaFunction{[} \AgdaBound{A} \AgdaFunction{]subst} \AgdaBound{p} \AgdaBound{a} \AgdaKeyword{in}\<%
\\
\>[0]\AgdaIndent{19}{}\<[19]%
\>[19]\AgdaKeyword{let} \AgdaBound{p'} \AgdaSymbol{=} \AgdaSymbol{λ} \AgdaBound{a} \AgdaSymbol{→} \AgdaFunction{[} \AgdaBound{A} \AgdaFunction{]trans-refl} \AgdaKeyword{in}\<%
\\
\>[0]\AgdaIndent{13}{}\<[13]%
\>[13]\AgdaSymbol{λ\{(}\AgdaBound{f} \AgdaInductiveConstructor{,} \AgdaBound{rsp}\AgdaSymbol{)} \AgdaSymbol{→} \<[28]%
\>[28]\<%
\\
\>[13]\AgdaIndent{15}{}\<[15]%
\>[15]\AgdaSymbol{(λ} \AgdaBound{a} \AgdaSymbol{→} \AgdaFunction{[} \AgdaBound{B} \AgdaFunction{]subst} \AgdaSymbol{(}\AgdaBound{p} \AgdaInductiveConstructor{,} \AgdaBound{p'} \AgdaBound{a}\AgdaSymbol{)} \AgdaSymbol{(}\AgdaBound{f} \AgdaSymbol{(}\AgdaBound{y2x} \AgdaBound{a}\AgdaSymbol{)))}\<%
\\
\>[13]\AgdaIndent{15}{}\<[15]%
\>[15]\AgdaInductiveConstructor{,} \<[49]%
\>[49]\<%
\\
\>[13]\AgdaIndent{15}{}\<[15]%
\>[15]\AgdaSymbol{(λ} \AgdaBound{a} \AgdaBound{b} \AgdaBound{q} \AgdaSymbol{→} \<[26]%
\>[26]\<%
\\
\>[15]\AgdaIndent{16}{}\<[16]%
\>[16]\AgdaKeyword{let} \AgdaBound{a'} \AgdaSymbol{=} \AgdaBound{y2x} \AgdaBound{a} \AgdaKeyword{in} \<[34]%
\>[34]\<%
\\
\>[15]\AgdaIndent{16}{}\<[16]%
\>[16]\AgdaKeyword{let} \AgdaBound{b'} \AgdaSymbol{=} \AgdaBound{y2x} \AgdaBound{b} \AgdaKeyword{in}\<%
\\
\>[15]\AgdaIndent{16}{}\<[16]%
\>[16]\AgdaKeyword{let} \AgdaBound{q'} \AgdaSymbol{=} \AgdaFunction{[} \AgdaBound{A} \AgdaFunction{]subst*} \AgdaSymbol{(}\AgdaFunction{[} \AgdaBound{Γ} \AgdaFunction{]sym} \AgdaBound{p}\AgdaSymbol{)} \AgdaBound{q} \AgdaKeyword{in}\<%
\\
\>[15]\AgdaIndent{16}{}\<[16]%
\>[16]\AgdaKeyword{let} \AgdaBound{H} \AgdaSymbol{=} \AgdaBound{rsp} \AgdaBound{a'} \AgdaBound{b'} \AgdaBound{q'} \AgdaKeyword{in}\<%
\\
\>[15]\AgdaIndent{16}{}\<[16]%
\>[16]\AgdaKeyword{let} \AgdaBound{r} \AgdaSymbol{:} \AgdaFunction{[} \AgdaBound{Γ} \AgdaFunction{\&} \AgdaBound{A} \AgdaFunction{]} \AgdaSymbol{(}\AgdaBound{x} \AgdaInductiveConstructor{,} \AgdaBound{b'}\AgdaSymbol{)} \AgdaFunction{≈} \AgdaSymbol{(}\AgdaBound{y} \AgdaInductiveConstructor{,} \AgdaBound{b}\AgdaSymbol{)}
                    r \AgdaSymbol{=} \AgdaSymbol{(}\AgdaBound{p} \AgdaInductiveConstructor{,} \AgdaBound{p'} \AgdaBound{b}\AgdaSymbol{)} \AgdaKeyword{in}\<%
\\
\>[15]\AgdaIndent{16}{}\<[16]%
\>[16]\AgdaKeyword{let} \AgdaBound{pre} \AgdaSymbol{=} \AgdaFunction{[} \AgdaBound{B} \AgdaFunction{]subst*} \AgdaBound{r} \AgdaBound{H} \AgdaKeyword{in}\<%
\\
\>[15]\AgdaIndent{16}{}\<[16]%
\>[16]\<%
\\
\>[15]\AgdaIndent{16}{}\<[16]%
\>[16]\AgdaFunction{[} \AgdaFunction{[} \AgdaBound{B} \AgdaFunction{]fm} \AgdaSymbol{(}\AgdaBound{y} \AgdaInductiveConstructor{,} \AgdaBound{b}\AgdaSymbol{)} \AgdaFunction{]trans} \<[41]%
\>[41]\<%
\\
\>[15]\AgdaIndent{16}{}\<[16]%
\>[16]\AgdaSymbol{(}\AgdaFunction{[} \AgdaBound{B} \AgdaFunction{]trans*} \AgdaSymbol{\_} \AgdaSymbol{\_} \AgdaSymbol{\_)} \<[52]%
\>[52]\<%
\\
\>[15]\AgdaIndent{16}{}\<[16]%
\>[16]\AgdaSymbol{(}\AgdaFunction{[} \AgdaFunction{[} \AgdaBound{B} \AgdaFunction{]fm} \AgdaSymbol{(}\AgdaBound{y} \AgdaInductiveConstructor{,} \AgdaBound{b}\AgdaSymbol{)} \AgdaFunction{]trans} \<[42]%
\>[42]\<%
\\
\>[15]\AgdaIndent{16}{}\<[16]%
\>[16]\AgdaFunction{[} \AgdaBound{B} \AgdaFunction{]subst-pi} \<[30]%
\>[30]\<%
\\
\>[15]\AgdaIndent{16}{}\<[16]%
\>[16]\AgdaSymbol{(}\AgdaFunction{[} \AgdaFunction{[} \AgdaBound{B} \AgdaFunction{]fm} \AgdaSymbol{(}\AgdaBound{y} \AgdaInductiveConstructor{,} \AgdaBound{b}\AgdaSymbol{)} \AgdaFunction{]trans} \<[42]%
\>[42]\<%
\\
\>[15]\AgdaIndent{16}{}\<[16]%
\>[16]\AgdaSymbol{(}\AgdaFunction{[} \AgdaFunction{[} \AgdaBound{B} \AgdaFunction{]fm} \AgdaSymbol{(}\AgdaBound{y} \AgdaInductiveConstructor{,} \AgdaBound{b}\AgdaSymbol{)} \AgdaFunction{]sym} \<[40]%
\>[40]\<%
\\
\>[15]\AgdaIndent{16}{}\<[16]%
\>[16]\AgdaSymbol{(}\AgdaFunction{[} \AgdaBound{B} \AgdaFunction{]trans*} \AgdaSymbol{\_} \AgdaSymbol{\_} \AgdaSymbol{\_))} \<[37]%
\>[37]\<%
\\
\>[15]\AgdaIndent{16}{}\<[16]%
\>[16]\AgdaBound{pre}\AgdaSymbol{))} \<[23]%
\>[23]\<%
\\
\>[15]\AgdaIndent{16}{}\<[16]%
\>[16]\AgdaSymbol{)} \<[22]%
\>[22]\<%
\\
\>[-12]\AgdaIndent{13}{}\<[13]%
\>[13]\AgdaSymbol{\}}\<%
\\
\>[0]\AgdaIndent{2}{}\<[2]%
\>[2]\AgdaSymbol{;} \AgdaField{subst*} \AgdaSymbol{=} \AgdaSymbol{λ} \AgdaBound{\_} \AgdaBound{q} \AgdaBound{\_} \AgdaSymbol{→} \AgdaFunction{[} \AgdaBound{B} \AgdaFunction{]subst*} \AgdaSymbol{\_} \AgdaSymbol{(}\AgdaBound{q} \AgdaSymbol{\_)}\<%
\\
\>[0]\AgdaIndent{2}{}\<[2]%
\>[2]\AgdaSymbol{;} \AgdaField{refl*} \AgdaSymbol{=} \AgdaSymbol{λ} \AgdaSymbol{\{}\AgdaBound{x} \AgdaSymbol{(}\AgdaBound{f} \AgdaInductiveConstructor{,} \AgdaBound{rsp}\AgdaSymbol{)} \AgdaBound{a} \AgdaSymbol{→} \<[32]%
\>[32]\AgdaFunction{[} \AgdaFunction{[} \AgdaBound{B} \AgdaFunction{]fm} \AgdaSymbol{\_} \AgdaFunction{]trans} \<[51]%
\>[51]\<%
\\
\>[2]\AgdaIndent{17}{}\<[17]%
\>[17]\AgdaFunction{[} \AgdaBound{B} \AgdaFunction{]subst-pi} \AgdaSymbol{(}\AgdaBound{rsp} \AgdaSymbol{(}\AgdaFunction{[} \AgdaBound{A} \AgdaFunction{]subst} \<[48]%
\>[48]\<%
\\
\>[17]\AgdaIndent{21}{}\<[21]%
\>[21]\AgdaSymbol{(}\AgdaFunction{[} \AgdaBound{Γ} \AgdaFunction{]sym} \AgdaFunction{[} \AgdaBound{Γ} \AgdaFunction{]refl}\AgdaSymbol{)} \AgdaBound{a}\AgdaSymbol{)} \AgdaBound{a} \AgdaFunction{[} \AgdaBound{A} \AgdaFunction{]subst-pi'}\AgdaSymbol{)} \<[64]%
\>[64]\AgdaSymbol{\}}\<%
\\
\>[2]\AgdaIndent{2}{}\<[2]%
\>[2]\AgdaSymbol{;} \AgdaField{trans*} \AgdaSymbol{=} \AgdaSymbol{λ} \AgdaBound{p} \AgdaBound{q} \AgdaSymbol{→} \AgdaSymbol{λ} \AgdaSymbol{\{(}\AgdaBound{f} \AgdaInductiveConstructor{,} \AgdaBound{rsp}\AgdaSymbol{)} \AgdaBound{a} \AgdaSymbol{→}\<%
\\
\>[0]\AgdaIndent{13}{}\<[13]%
\>[13]\AgdaFunction{[} \AgdaFunction{[} \AgdaBound{B} \AgdaFunction{]fm} \AgdaSymbol{\_} \AgdaFunction{]trans} \<[32]%
\>[32]\<%
\\
\>[0]\AgdaIndent{13}{}\<[13]%
\>[13]\AgdaSymbol{(}\AgdaFunction{[} \AgdaFunction{[} \AgdaBound{B} \AgdaFunction{]fm} \AgdaSymbol{\_} \AgdaFunction{]trans} \<[33]%
\>[33]\<%
\\
\>[0]\AgdaIndent{13}{}\<[13]%
\>[13]\AgdaSymbol{(}\AgdaFunction{[} \AgdaBound{B} \AgdaFunction{]trans*} \AgdaSymbol{\_} \AgdaSymbol{\_} \AgdaSymbol{\_)} \<[33]%
\>[33]\<%
\\
\>[0]\AgdaIndent{13}{}\<[13]%
\>[13]\AgdaSymbol{(}\AgdaFunction{[} \AgdaFunction{[} \AgdaBound{B} \AgdaFunction{]fm} \AgdaSymbol{\_} \AgdaFunction{]sym} \<[31]%
\>[31]\<%
\\
\>[0]\AgdaIndent{13}{}\<[13]%
\>[13]\AgdaSymbol{(}\AgdaFunction{[} \AgdaFunction{[} \AgdaBound{B} \AgdaFunction{]fm} \AgdaSymbol{\_} \AgdaFunction{]trans} \<[33]%
\>[33]\<%
\\
\>[0]\AgdaIndent{13}{}\<[13]%
\>[13]\AgdaSymbol{(}\AgdaFunction{[} \AgdaBound{B} \AgdaFunction{]trans*} \AgdaSymbol{\_} \AgdaSymbol{\_} \AgdaSymbol{\_)} \AgdaFunction{[} \AgdaBound{B} \AgdaFunction{]subst-pi}\AgdaSymbol{)))} \<[50]%
\>[50]\<%
\\
\>[0]\AgdaIndent{13}{}\<[13]%
\>[13]\AgdaSymbol{(}\AgdaFunction{[} \AgdaBound{B} \AgdaFunction{]subst*} \AgdaSymbol{\_} \AgdaSymbol{(}\AgdaBound{rsp} \AgdaSymbol{\_} \AgdaSymbol{\_} \<[37]%
\>[37]\<%
\\
\>[0]\AgdaIndent{13}{}\<[13]%
\>[13]\AgdaSymbol{(}\AgdaFunction{[} \AgdaFunction{[} \AgdaBound{A} \AgdaFunction{]fm} \AgdaSymbol{\_} \AgdaFunction{]trans} \<[33]%
\>[33]\<%
\\
\>[0]\AgdaIndent{13}{}\<[13]%
\>[13]\AgdaSymbol{(}\AgdaFunction{[} \AgdaBound{A} \AgdaFunction{]trans*} \AgdaSymbol{\_} \AgdaSymbol{\_} \AgdaSymbol{\_)} \AgdaFunction{[} \AgdaBound{A} \AgdaFunction{]subst-pi}\AgdaSymbol{)))} \AgdaSymbol{\}} \<[52]%
\>[52]\<%
\\
\>[0]\AgdaIndent{2}{}\<[2]%
\>[2]\AgdaSymbol{\}}\<%
\\
%
\\
\>\<\end{code}
}

It also comes with two necessary operation on the terms of $Pi$-types, $\lambda$-abstraction and application.
There are $\beta-\eta$ laws to verfify for them so that we could form an isomorphism with these two operations. however technically it causes stack overflow. We may simplify these definition in the future so that we could verify them in Agda.

%to do : verification of β and η
%cause stack overflow

\begin{code}\>\<%
\\
%
\\
\>\AgdaFunction{lam} \AgdaSymbol{:} \AgdaSymbol{\{}\AgdaBound{Γ} \AgdaSymbol{:} \AgdaFunction{Con}\AgdaSymbol{\}\{}\AgdaBound{A} \AgdaSymbol{:} \AgdaRecord{Ty} \AgdaBound{Γ}\AgdaSymbol{\}\{}\AgdaBound{B} \AgdaSymbol{:} \AgdaRecord{Ty} \AgdaSymbol{(}\AgdaBound{Γ} \AgdaFunction{\&} \AgdaBound{A}\AgdaSymbol{)\}} \AgdaSymbol{→} \AgdaRecord{Tm} \AgdaBound{B} \AgdaSymbol{→} \AgdaRecord{Tm} \AgdaSymbol{(}\AgdaFunction{Π} \AgdaBound{A} \AgdaBound{B}\AgdaSymbol{)}\<%
\\
\>\<\end{code}

\AgdaHide{
\begin{code}\>\<%
\\
\>\AgdaFunction{lam} \AgdaSymbol{\{}\AgdaBound{Γ}\AgdaSymbol{\}} \AgdaSymbol{\{}\AgdaBound{A}\AgdaSymbol{\}} \AgdaSymbol{(}\AgdaInductiveConstructor{tm:} \AgdaBound{tm} \AgdaInductiveConstructor{resp:} \AgdaBound{respt}\AgdaSymbol{)} \AgdaSymbol{=} \<[35]%
\>[35]\<%
\\
\>[0]\AgdaIndent{2}{}\<[2]%
\>[2]\AgdaKeyword{record} \AgdaSymbol{\{} \AgdaField{tm} \AgdaSymbol{=} \AgdaSymbol{λ} \AgdaBound{x} \AgdaSymbol{→} \AgdaSymbol{(λ} \AgdaBound{a} \AgdaSymbol{→} \AgdaBound{tm} \AgdaSymbol{(}\AgdaBound{x} \AgdaInductiveConstructor{,} \AgdaBound{a}\AgdaSymbol{))} \AgdaInductiveConstructor{,} \<[43]%
\>[43]\<%
\\
\>[2]\AgdaIndent{11}{}\<[11]%
\>[11]\AgdaSymbol{(λ} \AgdaBound{a} \AgdaBound{b} \AgdaBound{p} \AgdaSymbol{→} \AgdaBound{respt} \AgdaSymbol{(}\AgdaFunction{[} \AgdaBound{Γ} \AgdaFunction{]refl} \AgdaInductiveConstructor{,}\<%
\\
\>[11]\AgdaIndent{13}{}\<[13]%
\>[13]\AgdaFunction{[} \AgdaFunction{[} \AgdaBound{A} \AgdaFunction{]fm} \AgdaBound{x} \AgdaFunction{]trans} \AgdaSymbol{(}\AgdaFunction{[} \AgdaBound{A} \AgdaFunction{]refl*} \AgdaSymbol{\_} \AgdaSymbol{\_)} \AgdaBound{p}\AgdaSymbol{))}\<%
\\
\>[-7]\AgdaIndent{9}{}\<[9]%
\>[9]\AgdaSymbol{;} \AgdaField{respt} \AgdaSymbol{=} \AgdaSymbol{λ} \AgdaBound{p} \AgdaBound{\_} \AgdaSymbol{→} \AgdaBound{respt} \AgdaSymbol{(}\AgdaBound{p} \AgdaInductiveConstructor{,} \AgdaFunction{[} \AgdaBound{A} \AgdaFunction{]trans-refl}\AgdaSymbol{)} \<[55]%
\>[55]\<%
\\
\>[0]\AgdaIndent{9}{}\<[9]%
\>[9]\AgdaSymbol{\}}\<%
\\
%
\\
\>\<\end{code}
}


\begin{code}\>\<%
\\
\>\AgdaFunction{app} \AgdaSymbol{:} \AgdaSymbol{\{}\AgdaBound{Γ} \AgdaSymbol{:} \AgdaFunction{Con}\AgdaSymbol{\}\{}\AgdaBound{A} \AgdaSymbol{:} \AgdaRecord{Ty} \AgdaBound{Γ}\AgdaSymbol{\}\{}\AgdaBound{B} \AgdaSymbol{:} \AgdaRecord{Ty} \AgdaSymbol{(}\AgdaBound{Γ} \AgdaFunction{\&} \AgdaBound{A}\AgdaSymbol{)\}} \AgdaSymbol{→} \AgdaRecord{Tm} \AgdaSymbol{(}\AgdaFunction{Π} \AgdaBound{A} \AgdaBound{B}\AgdaSymbol{)} \AgdaSymbol{→} \AgdaRecord{Tm} \AgdaBound{B}\<%
\\
%
\\
\>\<\end{code}

\AgdaHide{
\begin{code}\>\<%
\\
\>\AgdaFunction{app} \AgdaSymbol{\{}\AgdaBound{Γ}\AgdaSymbol{\}} \AgdaSymbol{\{}\AgdaBound{A}\AgdaSymbol{\}} \AgdaSymbol{\{}\AgdaBound{B}\AgdaSymbol{\}} \AgdaSymbol{(}\AgdaInductiveConstructor{tm:} \AgdaBound{tm} \AgdaInductiveConstructor{resp:} \AgdaBound{respt}\AgdaSymbol{)} \AgdaSymbol{=} \<[39]%
\>[39]\<%
\\
\>[0]\AgdaIndent{2}{}\<[2]%
\>[2]\AgdaKeyword{record} \AgdaSymbol{\{} \AgdaField{tm} \AgdaSymbol{=} \AgdaSymbol{λ} \AgdaSymbol{\{(}\AgdaBound{x} \AgdaInductiveConstructor{,} \AgdaBound{a}\AgdaSymbol{)} \AgdaSymbol{→} \AgdaFunction{proj₁} \AgdaSymbol{(}\AgdaBound{tm} \AgdaBound{x}\AgdaSymbol{)} \AgdaBound{a}\AgdaSymbol{\}}\<%
\\
\>[0]\AgdaIndent{9}{}\<[9]%
\>[9]\AgdaSymbol{;} \AgdaField{respt} \AgdaSymbol{=} \AgdaSymbol{λ} \AgdaSymbol{\{}\AgdaBound{x}\AgdaSymbol{\}} \AgdaSymbol{\{}\AgdaBound{y}\AgdaSymbol{\}} \AgdaSymbol{→} \AgdaSymbol{λ} \AgdaSymbol{\{(}\AgdaBound{p} \AgdaInductiveConstructor{,} \AgdaBound{tr}\AgdaSymbol{)} \AgdaSymbol{→} \<[45]%
\>[45]\<%
\\
\>[9]\AgdaIndent{13}{}\<[13]%
\>[13]\AgdaKeyword{let} \AgdaBound{fresp} \AgdaSymbol{=} \AgdaFunction{proj₂} \AgdaSymbol{(}\AgdaBound{tm} \AgdaSymbol{(}\AgdaFunction{proj₁} \AgdaBound{x}\AgdaSymbol{))} \AgdaKeyword{in}\<%
\\
\>[13]\AgdaIndent{16}{}\<[16]%
\>[16]\AgdaFunction{[} \AgdaFunction{[} \AgdaBound{B} \AgdaFunction{]fm} \AgdaSymbol{\_} \AgdaFunction{]trans} \<[35]%
\>[35]\<%
\\
\>[13]\AgdaIndent{16}{}\<[16]%
\>[16]\AgdaSymbol{(}\AgdaFunction{[} \AgdaBound{B} \AgdaFunction{]subst*} \AgdaSymbol{(}\AgdaBound{p} \AgdaInductiveConstructor{,} \AgdaBound{tr}\AgdaSymbol{)} \AgdaSymbol{(}\AgdaFunction{[} \AgdaFunction{[} \AgdaBound{B} \AgdaFunction{]fm} \AgdaSymbol{\_} \AgdaFunction{]sym} \AgdaFunction{[} \AgdaBound{B} \AgdaFunction{]subst-pi'}\AgdaSymbol{))} \<[73]%
\>[73]\<%
\\
\>[13]\AgdaIndent{16}{}\<[16]%
\>[16]\AgdaSymbol{(}\AgdaFunction{[} \AgdaFunction{[} \AgdaBound{B} \AgdaFunction{]fm} \AgdaSymbol{\_} \AgdaFunction{]trans}\<%
\\
\>[13]\AgdaIndent{16}{}\<[16]%
\>[16]\AgdaSymbol{(}\AgdaFunction{[} \AgdaBound{B} \AgdaFunction{]trans*} \AgdaSymbol{(}\AgdaFunction{[} \AgdaBound{Γ} \AgdaFunction{]refl} \AgdaInductiveConstructor{,} \AgdaFunction{[} \AgdaBound{A} \AgdaFunction{]refl*} \AgdaSymbol{\_} \AgdaSymbol{\_)} \AgdaSymbol{\_} \AgdaSymbol{\_)} \<[63]%
\>[63]\<%
\\
\>[13]\AgdaIndent{16}{}\<[16]%
\>[16]\AgdaSymbol{(}\AgdaFunction{[} \AgdaFunction{[} \AgdaBound{B} \AgdaFunction{]fm} \AgdaSymbol{\_} \AgdaFunction{]trans} \<[36]%
\>[36]\<%
\\
\>[13]\AgdaIndent{16}{}\<[16]%
\>[16]\AgdaFunction{[} \AgdaBound{B} \AgdaFunction{]subst-pi} \<[30]%
\>[30]\<%
\\
\>[13]\AgdaIndent{16}{}\<[16]%
\>[16]\AgdaSymbol{(}\AgdaFunction{[} \AgdaFunction{[} \AgdaBound{B} \AgdaFunction{]fm} \AgdaSymbol{\_} \AgdaFunction{]trans} \<[36]%
\>[36]\<%
\\
\>[13]\AgdaIndent{16}{}\<[16]%
\>[16]\AgdaSymbol{(}\AgdaFunction{[} \AgdaFunction{[} \AgdaBound{B} \AgdaFunction{]fm} \AgdaSymbol{\_} \AgdaFunction{]sym} \AgdaSymbol{(}\AgdaFunction{[} \AgdaBound{B} \AgdaFunction{]trans*} \AgdaSymbol{\_} \AgdaSymbol{(}\AgdaBound{p} \AgdaInductiveConstructor{,} \AgdaFunction{[} \AgdaBound{A} \AgdaFunction{]trans-refl}\AgdaSymbol{)} \AgdaSymbol{\_))}\<%
\\
\>[13]\AgdaIndent{16}{}\<[16]%
\>[16]\AgdaSymbol{(}\AgdaFunction{[} \AgdaFunction{[} \AgdaBound{B} \AgdaFunction{]fm} \AgdaSymbol{\_} \AgdaFunction{]trans} \<[36]%
\>[36]\<%
\\
\>[13]\AgdaIndent{16}{}\<[16]%
\>[16]\AgdaSymbol{(}\AgdaFunction{[} \AgdaBound{B} \AgdaFunction{]subst-pi*} \AgdaSymbol{(}\AgdaBound{fresp} \AgdaSymbol{\_} \AgdaSymbol{\_} \AgdaSymbol{(}\AgdaFunction{[} \AgdaBound{A} \AgdaFunction{]subst-mir2} \AgdaBound{tr}\AgdaSymbol{)))} \<[66]%
\>[66]\<%
\\
\>[13]\AgdaIndent{16}{}\<[16]%
\>[16]\AgdaSymbol{(}\AgdaBound{respt} \AgdaBound{p} \AgdaSymbol{\_)))))} \AgdaSymbol{\}}\<%
\\
\>[-6]\AgdaIndent{9}{}\<[9]%
\>[9]\AgdaSymbol{\}}\<%
\\
%
\\
%
\\
\>\<\end{code}
}

Non-dependent version of $\Pi$-types namely function types can be defined with type weakening. Since the dependence disappears, it is possible to define it straightforwardly without using $\Pi$-types.

\begin{code}\>\<%
\\
%
\\
\>\AgdaFunction{\_⇒'\_} \AgdaSymbol{:} \AgdaSymbol{\{}\AgdaBound{Γ} \AgdaSymbol{:} \AgdaFunction{Con}\AgdaSymbol{\}(}\AgdaBound{A} \AgdaBound{B} \AgdaSymbol{:} \AgdaRecord{Ty} \AgdaBound{Γ}\AgdaSymbol{)} \AgdaSymbol{→} \AgdaRecord{Ty} \AgdaBound{Γ}\<%
\\
\>\AgdaBound{A} \AgdaFunction{⇒'} \AgdaBound{B} \AgdaSymbol{=} \AgdaFunction{Π} \AgdaBound{A} \AgdaSymbol{(}\AgdaBound{B} \AgdaFunction{+T} \AgdaBound{A}\AgdaSymbol{)}\<%
\\
%
\\
\>\<\end{code}


\AgdaHide{
\begin{code}\>\<%
\\
%
\\
\>\AgdaFunction{[\_,\_]\_⇒fm\_} \AgdaSymbol{:} \AgdaSymbol{(}\AgdaBound{Γ} \AgdaSymbol{:} \AgdaFunction{Con}\AgdaSymbol{)(}\AgdaBound{x} \AgdaSymbol{:} \AgdaFunction{∣} \AgdaBound{Γ} \AgdaFunction{∣}\AgdaSymbol{)} \<[34]%
\>[34]\<%
\\
\>[0]\AgdaIndent{11}{}\<[11]%
\>[11]\AgdaSymbol{→} \AgdaRecord{hSetoid} \AgdaSymbol{→} \AgdaRecord{hSetoid} \AgdaSymbol{→} \AgdaRecord{hSetoid}\<%
\\
\>\AgdaFunction{[} \AgdaBound{Γ} \AgdaFunction{,} \AgdaBound{x} \AgdaFunction{]} \AgdaBound{Ax} \AgdaFunction{⇒fm} \AgdaBound{Bx} \<[20]%
\>[20]\<%
\\
\>[0]\AgdaIndent{2}{}\<[2]%
\>[2]\AgdaSymbol{=} \AgdaKeyword{record}\<%
\\
\>[0]\AgdaIndent{6}{}\<[6]%
\>[6]\AgdaSymbol{\{} \AgdaField{Carrier} \AgdaSymbol{=} \AgdaRecord{Σ[} \AgdaBound{fn} \AgdaRecord{∶} \AgdaSymbol{(}\AgdaFunction{∣} \AgdaBound{Ax} \AgdaFunction{∣} \AgdaSymbol{→} \AgdaFunction{∣} \AgdaBound{Bx} \AgdaFunction{∣}\AgdaSymbol{)} \AgdaRecord{]} \<[46]%
\>[46]\<%
\\
\>[6]\AgdaIndent{16}{}\<[16]%
\>[16]\AgdaSymbol{((}\AgdaBound{a} \AgdaBound{b} \AgdaSymbol{:} \AgdaFunction{∣} \AgdaBound{Ax} \AgdaFunction{∣}\AgdaSymbol{)(}\AgdaBound{p} \AgdaSymbol{:} \AgdaFunction{[} \AgdaBound{Ax} \AgdaFunction{]} \AgdaBound{a} \AgdaFunction{≈} \AgdaBound{b}\AgdaSymbol{)} \<[50]%
\>[50]\<%
\\
\>[16]\AgdaIndent{18}{}\<[18]%
\>[18]\AgdaSymbol{→} \AgdaFunction{[} \AgdaBound{Bx} \AgdaFunction{]} \AgdaBound{fn} \AgdaBound{a} \AgdaFunction{≈} \AgdaBound{fn} \AgdaBound{b}\AgdaSymbol{)}\<%
\\
\>[-4]\AgdaIndent{6}{}\<[6]%
\>[6]\AgdaSymbol{;} \AgdaField{\_≈h\_} \<[16]%
\>[16]\AgdaSymbol{=} \AgdaSymbol{λ\{(}\AgdaBound{f} \AgdaInductiveConstructor{,} \AgdaSymbol{\_)} \AgdaSymbol{(}\AgdaBound{g} \AgdaInductiveConstructor{,} \AgdaSymbol{\_)} \<[36]%
\>[36]\<%
\\
\>[0]\AgdaIndent{18}{}\<[18]%
\>[18]\AgdaSymbol{→} \AgdaFunction{∀'[} \AgdaBound{a} \AgdaFunction{∶} \AgdaSymbol{\_} \AgdaFunction{]} \AgdaFunction{[} \AgdaBound{Bx} \AgdaFunction{]} \AgdaBound{f} \AgdaBound{a} \AgdaFunction{≈h} \AgdaBound{g} \AgdaBound{a} \AgdaSymbol{\}}\<%
\\
\>[0]\AgdaIndent{6}{}\<[6]%
\>[6]\AgdaSymbol{;} \AgdaField{isEquiv} \AgdaSymbol{=} \AgdaKeyword{record} \AgdaSymbol{\{}\<%
\\
\>[0]\AgdaIndent{16}{}\<[16]%
\>[16]\AgdaField{refl} \<[22]%
\>[22]\AgdaSymbol{=} \AgdaSymbol{λ} \AgdaBound{\_} \AgdaSymbol{→} \AgdaFunction{[} \AgdaBound{Bx} \AgdaFunction{]refl} \<[41]%
\>[41]\<%
\\
\>[0]\AgdaIndent{14}{}\<[14]%
\>[14]\AgdaSymbol{;} \AgdaField{sym} \<[22]%
\>[22]\AgdaSymbol{=} \AgdaSymbol{λ} \AgdaBound{f} \AgdaBound{a} \AgdaSymbol{→} \AgdaFunction{[} \AgdaBound{Bx} \AgdaFunction{]sym} \AgdaSymbol{(}\AgdaBound{f} \AgdaBound{a}\AgdaSymbol{)}\<%
\\
\>[0]\AgdaIndent{14}{}\<[14]%
\>[14]\AgdaSymbol{;} \AgdaField{trans} \AgdaSymbol{=} \AgdaSymbol{λ} \AgdaBound{f} \AgdaBound{g} \AgdaBound{a} \AgdaSymbol{→} \AgdaFunction{[} \AgdaBound{Bx} \AgdaFunction{]trans} \AgdaSymbol{(}\AgdaBound{f} \AgdaBound{a}\AgdaSymbol{)} \AgdaSymbol{(}\AgdaBound{g} \AgdaBound{a}\AgdaSymbol{)}\<%
\\
\>[14]\AgdaIndent{26}{}\<[26]%
\>[26]\AgdaSymbol{\}}\<%
\\
\>[6]\AgdaIndent{6}{}\<[6]%
\>[6]\AgdaSymbol{\}}\<%
\\
%
\\
%
\\
\>\<\end{code}

}

%to do: verification

%verification of functor laws (do we have extensional equality for record types? or eta equality?)
%define equality with respect to propositions which are proof irrelevant


\subsection{Examples of types}

\AgdaHide{
\begin{code}\>\<%
\\
%
\\
\>\AgdaSymbol{\{-\#} \AgdaKeyword{OPTIONS} --type-in-type \AgdaSymbol{\#-\}}\<%
\\
%
\\
\>\AgdaKeyword{import} \AgdaModule{Level}\<%
\\
\>\AgdaKeyword{open} \AgdaKeyword{import} \AgdaModule{Relation.Binary.PropositionalEquality} \AgdaSymbol{as} \AgdaModule{PE} \AgdaKeyword{hiding} \AgdaSymbol{(}refl \AgdaSymbol{;} sym \AgdaSymbol{;} trans\AgdaSymbol{;} isEquivalence\AgdaSymbol{;} [\_]\AgdaSymbol{)}\<%
\\
%
\\
\>\AgdaKeyword{module} \AgdaModule{CwF-ctd} \AgdaSymbol{(}\AgdaBound{ext} \AgdaSymbol{:} \AgdaFunction{Extensionality} \AgdaPrimitive{Level.zero} \AgdaPrimitive{Level.zero}\AgdaSymbol{)} \AgdaKeyword{where}\<%
\\
%
\\
\>\AgdaKeyword{open} \AgdaKeyword{import} \AgdaModule{Data.Unit}\<%
\\
\>\AgdaKeyword{open} \AgdaKeyword{import} \AgdaModule{Function}\<%
\\
\>\AgdaKeyword{open} \AgdaKeyword{import} \AgdaModule{Data.Product}\<%
\\
%
\\
\>\AgdaKeyword{open} \AgdaKeyword{import} \AgdaModule{CwF-setoidwo} \AgdaBound{ext} \AgdaKeyword{public}\<%
\\
%
\\
\>\AgdaKeyword{open} \AgdaKeyword{import} \AgdaModule{Data.Nat}\<%
\\
%
\\
\>\<\end{code}
}

Binary relation

\begin{code}\>\<%
\\
%
\\
\>\AgdaFunction{Rel} \AgdaSymbol{:} \AgdaSymbol{\{}\AgdaBound{Γ} \AgdaSymbol{:} \AgdaFunction{Con}\AgdaSymbol{\}} \AgdaSymbol{→} \AgdaRecord{Ty} \AgdaBound{Γ} \AgdaSymbol{→} \AgdaPrimitiveType{Set₁}\<%
\\
\>\AgdaFunction{Rel} \AgdaSymbol{\{}\AgdaBound{Γ}\AgdaSymbol{\}} \AgdaBound{A} \AgdaSymbol{=} \AgdaRecord{Ty} \AgdaSymbol{(}\AgdaBound{Γ} \AgdaFunction{\&} \AgdaBound{A} \AgdaFunction{\&} \AgdaBound{A} \AgdaFunction{[} \AgdaFunction{fst\&} \AgdaSymbol{\{}A \AgdaSymbol{=} \AgdaBound{A}\AgdaSymbol{\}} \AgdaFunction{]T}\AgdaSymbol{)}\<%
\\
%
\\
\>\<\end{code}

Natural numbers

\begin{code}\>\<%
\\
%
\\
\>\AgdaKeyword{module} \AgdaModule{Natural} \AgdaSymbol{(}\AgdaBound{Γ} \AgdaSymbol{:} \AgdaFunction{Con}\AgdaSymbol{)} \AgdaKeyword{where}\<%
\\
%
\\
\>[0]\AgdaIndent{2}{}\<[2]%
\>[2]\AgdaFunction{\_≈nat\_} \AgdaSymbol{:} \AgdaDatatype{ℕ} \AgdaSymbol{→} \AgdaDatatype{ℕ} \AgdaSymbol{→} \AgdaRecord{HProp}\<%
\\
\>[0]\AgdaIndent{2}{}\<[2]%
\>[2]\AgdaInductiveConstructor{zero} \AgdaFunction{≈nat} \AgdaInductiveConstructor{zero} \AgdaSymbol{=} \AgdaFunction{⊤'}\<%
\\
\>[0]\AgdaIndent{2}{}\<[2]%
\>[2]\AgdaInductiveConstructor{zero} \AgdaFunction{≈nat} \AgdaInductiveConstructor{suc} \AgdaBound{n} \AgdaSymbol{=} \AgdaFunction{⊥'}\<%
\\
\>[0]\AgdaIndent{2}{}\<[2]%
\>[2]\AgdaInductiveConstructor{suc} \AgdaBound{m} \AgdaFunction{≈nat} \AgdaInductiveConstructor{zero} \AgdaSymbol{=} \AgdaFunction{⊥'}\<%
\\
\>[0]\AgdaIndent{2}{}\<[2]%
\>[2]\AgdaInductiveConstructor{suc} \AgdaBound{m} \AgdaFunction{≈nat} \AgdaInductiveConstructor{suc} \AgdaBound{n} \AgdaSymbol{=} \AgdaBound{m} \AgdaFunction{≈nat} \AgdaBound{n}\<%
\\
\>[0]\AgdaIndent{2}{}\<[2]%
\>[2]\<%
\\
\>[0]\AgdaIndent{2}{}\<[2]%
\>[2]\AgdaFunction{reflNat} \AgdaSymbol{:} \AgdaSymbol{\{}\AgdaBound{x} \AgdaSymbol{:} \AgdaDatatype{ℕ}\AgdaSymbol{\}} \AgdaSymbol{→} \AgdaFunction{<} \AgdaBound{x} \AgdaFunction{≈nat} \AgdaBound{x} \AgdaFunction{>} \<[35]%
\>[35]\<%
\\
\>[0]\AgdaIndent{2}{}\<[2]%
\>[2]\AgdaFunction{reflNat} \AgdaSymbol{\{}\AgdaInductiveConstructor{zero}\AgdaSymbol{\}} \AgdaSymbol{=} \AgdaInductiveConstructor{tt}\<%
\\
\>[0]\AgdaIndent{2}{}\<[2]%
\>[2]\AgdaFunction{reflNat} \AgdaSymbol{\{}\AgdaInductiveConstructor{suc} \AgdaBound{n}\AgdaSymbol{\}} \AgdaSymbol{=} \AgdaFunction{reflNat} \AgdaSymbol{\{}\AgdaBound{n}\AgdaSymbol{\}}\<%
\\
%
\\
\>[0]\AgdaIndent{2}{}\<[2]%
\>[2]\AgdaFunction{symNat} \AgdaSymbol{:} \AgdaSymbol{\{}\AgdaBound{x} \AgdaBound{y} \AgdaSymbol{:} \AgdaDatatype{ℕ}\AgdaSymbol{\}} \AgdaSymbol{→} \AgdaFunction{<} \AgdaBound{x} \AgdaFunction{≈nat} \AgdaBound{y} \AgdaFunction{>} \AgdaSymbol{→} \AgdaFunction{<} \AgdaBound{y} \AgdaFunction{≈nat} \AgdaBound{x} \AgdaFunction{>}\<%
\\
\>[0]\AgdaIndent{2}{}\<[2]%
\>[2]\AgdaFunction{symNat} \AgdaSymbol{\{}\AgdaInductiveConstructor{zero}\AgdaSymbol{\}} \AgdaSymbol{\{}\AgdaInductiveConstructor{zero}\AgdaSymbol{\}} \AgdaBound{eq} \AgdaSymbol{=} \AgdaInductiveConstructor{tt}\<%
\\
\>[0]\AgdaIndent{2}{}\<[2]%
\>[2]\AgdaFunction{symNat} \AgdaSymbol{\{}\AgdaInductiveConstructor{zero}\AgdaSymbol{\}} \AgdaSymbol{\{}\AgdaInductiveConstructor{suc} \AgdaSymbol{\_\}} \AgdaBound{eq} \AgdaSymbol{=} \AgdaBound{eq}\<%
\\
\>[0]\AgdaIndent{2}{}\<[2]%
\>[2]\AgdaFunction{symNat} \AgdaSymbol{\{}\AgdaInductiveConstructor{suc} \AgdaSymbol{\_\}} \AgdaSymbol{\{}\AgdaInductiveConstructor{zero}\AgdaSymbol{\}} \AgdaBound{eq} \AgdaSymbol{=} \AgdaBound{eq}\<%
\\
\>[0]\AgdaIndent{2}{}\<[2]%
\>[2]\AgdaFunction{symNat} \AgdaSymbol{\{}\AgdaInductiveConstructor{suc} \AgdaBound{x}\AgdaSymbol{\}} \AgdaSymbol{\{}\AgdaInductiveConstructor{suc} \AgdaBound{y}\AgdaSymbol{\}} \AgdaBound{eq} \AgdaSymbol{=} \AgdaFunction{symNat} \AgdaSymbol{\{}\AgdaBound{x}\AgdaSymbol{\}} \AgdaSymbol{\{}\AgdaBound{y}\AgdaSymbol{\}} \AgdaBound{eq}\<%
\\
%
\\
\>[0]\AgdaIndent{2}{}\<[2]%
\>[2]\AgdaFunction{transNat} \AgdaSymbol{:} \AgdaSymbol{\{}\AgdaBound{x} \AgdaBound{y} \AgdaBound{z} \AgdaSymbol{:} \AgdaDatatype{ℕ}\AgdaSymbol{\}} \AgdaSymbol{→} \AgdaFunction{<} \AgdaBound{x} \AgdaFunction{≈nat} \AgdaBound{y} \AgdaFunction{>} \AgdaSymbol{→} \AgdaFunction{<} \AgdaBound{y} \AgdaFunction{≈nat} \AgdaBound{z} \AgdaFunction{>} \AgdaSymbol{→} \AgdaFunction{<} \AgdaBound{x} \AgdaFunction{≈nat} \AgdaBound{z} \AgdaFunction{>}\<%
\\
\>[0]\AgdaIndent{2}{}\<[2]%
\>[2]\AgdaFunction{transNat} \AgdaSymbol{\{}\AgdaInductiveConstructor{zero}\AgdaSymbol{\}} \AgdaSymbol{\{}\AgdaInductiveConstructor{zero}\AgdaSymbol{\}} \AgdaBound{xy} \AgdaBound{yz} \AgdaSymbol{=} \AgdaBound{yz}\<%
\\
\>[0]\AgdaIndent{2}{}\<[2]%
\>[2]\AgdaFunction{transNat} \AgdaSymbol{\{}\AgdaInductiveConstructor{zero}\AgdaSymbol{\}} \AgdaSymbol{\{}\AgdaInductiveConstructor{suc} \AgdaSymbol{\_\}} \AgdaSymbol{()} \AgdaBound{yz}\<%
\\
\>[0]\AgdaIndent{2}{}\<[2]%
\>[2]\AgdaFunction{transNat} \AgdaSymbol{\{}\AgdaInductiveConstructor{suc} \AgdaSymbol{\_\}} \AgdaSymbol{\{}\AgdaInductiveConstructor{zero}\AgdaSymbol{\}} \AgdaSymbol{()} \AgdaBound{yz}\<%
\\
\>[0]\AgdaIndent{2}{}\<[2]%
\>[2]\AgdaFunction{transNat} \AgdaSymbol{\{}\AgdaInductiveConstructor{suc} \AgdaSymbol{\_\}} \AgdaSymbol{\{}\AgdaInductiveConstructor{suc} \AgdaSymbol{\_\}} \AgdaSymbol{\{}\AgdaInductiveConstructor{zero}\AgdaSymbol{\}} \AgdaBound{xy} \AgdaBound{yz} \AgdaSymbol{=} \AgdaBound{yz}\<%
\\
\>[0]\AgdaIndent{2}{}\<[2]%
\>[2]\AgdaFunction{transNat} \AgdaSymbol{\{}\AgdaInductiveConstructor{suc} \AgdaBound{x}\AgdaSymbol{\}} \AgdaSymbol{\{}\AgdaInductiveConstructor{suc} \AgdaBound{y}\AgdaSymbol{\}} \AgdaSymbol{\{}\AgdaInductiveConstructor{suc} \AgdaBound{z}\AgdaSymbol{\}} \AgdaBound{xy} \AgdaBound{yz} \AgdaSymbol{=} \AgdaFunction{transNat} \AgdaSymbol{\{}\AgdaBound{x}\AgdaSymbol{\}} \AgdaSymbol{\{}\AgdaBound{y}\AgdaSymbol{\}} \AgdaSymbol{\{}\AgdaBound{z}\AgdaSymbol{\}} \AgdaBound{xy} \AgdaBound{yz}\<%
\\
%
\\
\>[0]\AgdaIndent{2}{}\<[2]%
\>[2]\AgdaFunction{⟦Nat⟧} \AgdaSymbol{:} \AgdaRecord{Ty} \AgdaBound{Γ}\<%
\\
\>[0]\AgdaIndent{2}{}\<[2]%
\>[2]\AgdaFunction{⟦Nat⟧} \AgdaSymbol{=} \AgdaKeyword{record} \<[17]%
\>[17]\<%
\\
\>[2]\AgdaIndent{4}{}\<[4]%
\>[4]\AgdaSymbol{\{} \AgdaField{fm} \AgdaSymbol{=} \AgdaSymbol{λ} \AgdaBound{γ} \AgdaSymbol{→} \AgdaKeyword{record}\<%
\\
\>[4]\AgdaIndent{9}{}\<[9]%
\>[9]\AgdaSymbol{\{} \AgdaField{Carrier} \AgdaSymbol{=} \AgdaDatatype{ℕ}\<%
\\
\>[4]\AgdaIndent{9}{}\<[9]%
\>[9]\AgdaSymbol{;} \AgdaField{\_≈h\_} \AgdaSymbol{=} \AgdaFunction{\_≈nat\_}\<%
\\
\>[4]\AgdaIndent{9}{}\<[9]%
\>[9]\AgdaSymbol{;} \AgdaField{refl} \AgdaSymbol{=} \AgdaSymbol{λ} \AgdaSymbol{\{}\AgdaBound{n}\AgdaSymbol{\}} \AgdaSymbol{→} \AgdaFunction{reflNat} \AgdaSymbol{\{}\AgdaBound{n}\AgdaSymbol{\}}\<%
\\
\>[4]\AgdaIndent{9}{}\<[9]%
\>[9]\AgdaSymbol{;} \AgdaField{sym} \AgdaSymbol{=} \AgdaSymbol{λ} \AgdaSymbol{\{}\AgdaBound{x}\AgdaSymbol{\}} \AgdaSymbol{\{}\AgdaBound{y}\AgdaSymbol{\}} \AgdaSymbol{→} \AgdaFunction{symNat} \AgdaSymbol{\{}\AgdaBound{x}\AgdaSymbol{\}} \AgdaSymbol{\{}\AgdaBound{y}\AgdaSymbol{\}}\<%
\\
\>[4]\AgdaIndent{9}{}\<[9]%
\>[9]\AgdaSymbol{;} \AgdaField{trans} \AgdaSymbol{=} \AgdaSymbol{λ} \AgdaSymbol{\{}\AgdaBound{x}\AgdaSymbol{\}} \AgdaSymbol{\{}\AgdaBound{y}\AgdaSymbol{\}} \AgdaSymbol{\{}\AgdaBound{z}\AgdaSymbol{\}} \AgdaSymbol{→} \AgdaFunction{transNat} \AgdaSymbol{\{}\AgdaBound{x}\AgdaSymbol{\}} \AgdaSymbol{\{}\AgdaBound{y}\AgdaSymbol{\}} \AgdaSymbol{\{}\AgdaBound{z}\AgdaSymbol{\}}\<%
\\
\>[4]\AgdaIndent{9}{}\<[9]%
\>[9]\AgdaSymbol{\}}\<%
\\
\>[0]\AgdaIndent{4}{}\<[4]%
\>[4]\AgdaSymbol{;} \AgdaField{substT} \AgdaSymbol{=} \AgdaSymbol{λ} \AgdaBound{\_} \AgdaSymbol{→} \AgdaFunction{id}\<%
\\
\>[0]\AgdaIndent{4}{}\<[4]%
\>[4]\AgdaSymbol{;} \AgdaField{subst*} \AgdaSymbol{=} \AgdaSymbol{λ} \AgdaBound{\_} \AgdaSymbol{→} \AgdaFunction{id}\<%
\\
\>[0]\AgdaIndent{4}{}\<[4]%
\>[4]\AgdaSymbol{;} \AgdaField{refl*} \AgdaSymbol{=} \AgdaSymbol{λ} \AgdaBound{x} \AgdaBound{a} \AgdaSymbol{→} \AgdaFunction{reflNat} \AgdaSymbol{\{}\AgdaBound{a}\AgdaSymbol{\}}\<%
\\
\>[0]\AgdaIndent{4}{}\<[4]%
\>[4]\AgdaSymbol{;} \AgdaField{trans*} \AgdaSymbol{=} \AgdaSymbol{λ} \AgdaBound{a} \AgdaSymbol{→} \AgdaFunction{reflNat} \AgdaSymbol{\{}\AgdaBound{a}\AgdaSymbol{\}} \<[33]%
\>[33]\<%
\\
\>[0]\AgdaIndent{4}{}\<[4]%
\>[4]\AgdaSymbol{\}}\<%
\\
%
\\
\>[0]\AgdaIndent{2}{}\<[2]%
\>[2]\AgdaFunction{⟦0⟧} \AgdaSymbol{:} \AgdaRecord{Tm} \AgdaFunction{⟦Nat⟧}\<%
\\
\>[0]\AgdaIndent{2}{}\<[2]%
\>[2]\AgdaFunction{⟦0⟧} \AgdaSymbol{=} \AgdaKeyword{record}\<%
\\
\>[2]\AgdaIndent{6}{}\<[6]%
\>[6]\AgdaSymbol{\{} \AgdaField{tm} \AgdaSymbol{=} \AgdaSymbol{λ} \AgdaBound{\_} \AgdaSymbol{→} \AgdaNumber{0}\<%
\\
\>[2]\AgdaIndent{6}{}\<[6]%
\>[6]\AgdaSymbol{;} \AgdaField{respt} \AgdaSymbol{=} \AgdaSymbol{λ} \AgdaBound{p} \AgdaSymbol{→} \AgdaInductiveConstructor{tt}\<%
\\
\>[2]\AgdaIndent{6}{}\<[6]%
\>[6]\AgdaSymbol{\}}\<%
\\
%
\\
\>[0]\AgdaIndent{2}{}\<[2]%
\>[2]\AgdaFunction{⟦s⟧} \AgdaSymbol{:} \AgdaRecord{Tm} \AgdaFunction{⟦Nat⟧} \AgdaSymbol{→} \AgdaRecord{Tm} \AgdaFunction{⟦Nat⟧}\<%
\\
\>[0]\AgdaIndent{2}{}\<[2]%
\>[2]\AgdaFunction{⟦s⟧} \AgdaSymbol{(}\AgdaInductiveConstructor{tm:} \AgdaBound{t} \AgdaInductiveConstructor{resp:} \AgdaBound{respt}\AgdaSymbol{)} \<[26]%
\>[26]\<%
\\
\>[2]\AgdaIndent{6}{}\<[6]%
\>[6]\AgdaSymbol{=} \AgdaKeyword{record}\<%
\\
\>[2]\AgdaIndent{6}{}\<[6]%
\>[6]\AgdaSymbol{\{} \AgdaField{tm} \AgdaSymbol{=} \AgdaInductiveConstructor{suc} \AgdaFunction{∘} \AgdaBound{t}\<%
\\
\>[2]\AgdaIndent{6}{}\<[6]%
\>[6]\AgdaSymbol{;} \AgdaField{respt} \AgdaSymbol{=} \AgdaBound{respt}\<%
\\
\>[2]\AgdaIndent{6}{}\<[6]%
\>[6]\AgdaSymbol{\}}\<%
\\
%
\\
\>\<\end{code}

Simply typed universe

\AgdaHide{
\begin{code}\>\<%
\\
%
\\
\>\AgdaComment{\{-
  data  ⟦U⟧⁰ : Set where
    nat : ⟦U⟧⁰
    arr<\_,\_> : (a b : ⟦U⟧⁰) → ⟦U⟧⁰

  \_\textasciitilde⟦U⟧\_ : ⟦U⟧⁰ → ⟦U⟧⁰ → HProp
  nat \textasciitilde⟦U⟧ nat = ⊤'
  nat \textasciitilde⟦U⟧ arr< a , b > = ⊥'
  arr< a , b > \textasciitilde⟦U⟧ nat = ⊥'
  arr< a , b > \textasciitilde⟦U⟧ arr< a' , b' > = a \textasciitilde⟦U⟧ a' ∧ b \textasciitilde⟦U⟧ b'

  reflU :  \{x : ⟦U⟧⁰\} → < x \textasciitilde⟦U⟧ x >
  reflU \{nat\} = tt
  reflU \{arr< a , b >\} = reflU \{a\} , reflU \{b\}

  symU : \{x y : ⟦U⟧⁰\} → < x \textasciitilde⟦U⟧ y > → < y \textasciitilde⟦U⟧ x >
  symU \{nat\} \{nat\} eq = tt
  symU \{nat\} \{arr< a , b >\} eq = eq
  symU \{arr< a , b >\} \{nat\} eq = eq
  symU \{arr< a , b >\} \{arr< a' , b' >\} (p , q) = (symU \{a\} \{a'\} p) 
                                               , (symU \{b\} \{b'\} q)

  transU : \{x y z : ⟦U⟧⁰\} → < x \textasciitilde⟦U⟧ y > → < y \textasciitilde⟦U⟧ z > → < x \textasciitilde⟦U⟧ z >
  transU \{nat\} \{nat\} eq1 eq2 = eq2
  transU \{nat\} \{arr< a , b >\} () eq2
  transU \{arr< a , b >\} \{nat\} () eq2
  transU \{arr< a , b >\} \{arr< a' , b' >\} \{nat\} eq1 eq2 = eq2
  transU \{arr< a , b >\} \{arr< a' , b' >\} \{arr< a0 , b0 >\} (p1 , q1) 
         (p2 , q2) = (transU \{a\} \{a'\} \{a0\} p1 p2) 
         , transU \{b\} \{b'\} \{b0\} q1 q2

  ⟦U⟧ : Ty Γ
  ⟦U⟧ = record 
    \{ fm = λ γ → record
         \{ Carrier = ⟦U⟧⁰
         ; \_≈h\_ = \_\textasciitilde⟦U⟧\_
         ; refl = λ \{x\} → reflU \{x\}
         ; sym = λ \{x\} \{y\} → symU \{x\} \{y\}
         ; trans = λ \{x\} \{y\} \{z\} → transU \{x\} \{y\} \{z\}
         \}
    ; substT = λ \_ → id
    ; subst* = λ \_ → id
    ; refl* = λ x a → reflU \{a\}
    ; trans* = λ a → reflU \{a\}
    \}

  elfm : Σ ∣ Γ ∣ (λ x → ⟦U⟧⁰) → HSetoid
  elfm (γ , nat) = [ ⟦Nat⟧ ]fm γ
  elfm (γ , arr< a , b >) = [ Γ , γ ] elfm (γ , a) ⇒fm elfm (γ , b)
-\}}\<%
\\
%
\\
\>\<\end{code}
}

\AgdaHide{
\begin{code}\>\<%
\\
%
\\
\>\AgdaComment{\{- To do : To find the way to extract the substT from ->

  elsubstT : \{x y : Σ ∣ Γ ∣ (λ x' → ⟦U⟧⁰)\} →
      Σ < [ Γ ] proj₁ x ≈h proj₁ y > (λ x' → < proj₂ x \textasciitilde⟦U⟧ proj₂ y >) →
      ∣ elfm x ∣ → ∣ elfm y ∣
  elsubstT \{\_ , nat\} \{\_ , nat\} \_ x' = x'
  elsubstT \{\_ , nat\} \{\_ , arr< a , b >\} (p , ()) x'
  elsubstT \{\_ , arr< a , b >\} \{\_ , nat\} (p , ()) x'
  elsubstT \{γ , arr< a , b >\} \{γ' , arr< a' , b' >\} (p , qa , qb) (s1 , s2) = 
   \{!!\}

  ⟦El⟧ : Ty (Γ \& ⟦U⟧)
  ⟦El⟧ = record 
       \{ fm = elfm
       ; substT = elsubstT
       ; subst* = \{!!\}
       ; refl* = \{!!\}
       ; trans* = \{!!\} 
       \}

-\}}\<%
\\
\>\<\end{code}
}

The equality type

\begin{code}\>\<%
\\
%
\\
\>\AgdaKeyword{module} \AgdaModule{Equality-Type} \AgdaSymbol{(}\AgdaBound{Γ} \AgdaSymbol{:} \AgdaFunction{Con}\AgdaSymbol{)(}\AgdaBound{A} \AgdaSymbol{:} \AgdaRecord{Ty} \AgdaBound{Γ}\AgdaSymbol{)} \AgdaKeyword{where}\<%
\\
%
\\
\>[0]\AgdaIndent{2}{}\<[2]%
\>[2]\AgdaFunction{⟦Id⟧} \AgdaSymbol{:} \AgdaFunction{Rel} \AgdaBound{A}\<%
\\
\>[0]\AgdaIndent{2}{}\<[2]%
\>[2]\AgdaFunction{⟦Id⟧} \AgdaSymbol{=} \AgdaKeyword{record} \<[16]%
\>[16]\<%
\\
\>[2]\AgdaIndent{4}{}\<[4]%
\>[4]\AgdaSymbol{\{} \AgdaField{fm} \AgdaSymbol{=} \AgdaSymbol{λ} \AgdaSymbol{\{((}\AgdaBound{x} \AgdaInductiveConstructor{,} \AgdaBound{a}\AgdaSymbol{)} \AgdaInductiveConstructor{,} \AgdaBound{b}\AgdaSymbol{)} \AgdaSymbol{→} \<[30]%
\>[30]\<%
\\
\>[4]\AgdaIndent{13}{}\<[13]%
\>[13]\AgdaKeyword{record} \<[20]%
\>[20]\<%
\\
\>[4]\AgdaIndent{13}{}\<[13]%
\>[13]\AgdaSymbol{\{} \AgdaField{Carrier} \AgdaSymbol{=} \AgdaFunction{[} \AgdaFunction{[} \AgdaBound{A} \AgdaFunction{]fm} \AgdaBound{x} \AgdaFunction{]} \AgdaBound{a} \AgdaFunction{≈} \AgdaBound{b}\<%
\\
\>[4]\AgdaIndent{13}{}\<[13]%
\>[13]\AgdaSymbol{;} \AgdaField{\_≈h\_} \AgdaSymbol{=} \AgdaSymbol{λ} \AgdaBound{\_} \AgdaBound{\_} \AgdaSymbol{→} \AgdaKeyword{record} \AgdaSymbol{\{} \AgdaField{prf} \AgdaSymbol{=} \AgdaRecord{⊤} \AgdaSymbol{;} \AgdaField{Uni} \AgdaSymbol{=} \AgdaInductiveConstructor{PE.refl} \AgdaSymbol{\}}\<%
\\
\>[4]\AgdaIndent{13}{}\<[13]%
\>[13]\AgdaSymbol{;} \AgdaField{refl} \AgdaSymbol{=} \AgdaInductiveConstructor{tt} \<[25]%
\>[25]\<%
\\
\>[4]\AgdaIndent{13}{}\<[13]%
\>[13]\AgdaSymbol{;} \AgdaField{sym} \AgdaSymbol{=} \AgdaFunction{id}\<%
\\
\>[4]\AgdaIndent{13}{}\<[13]%
\>[13]\AgdaSymbol{;} \AgdaField{trans} \AgdaSymbol{=} \AgdaSymbol{λ} \AgdaBound{\_} \AgdaBound{\_} \AgdaSymbol{→} \AgdaInductiveConstructor{tt}\<%
\\
\>[4]\AgdaIndent{13}{}\<[13]%
\>[13]\AgdaSymbol{\}}\<%
\\
\>[4]\AgdaIndent{13}{}\<[13]%
\>[13]\AgdaSymbol{\}}\<%
\\
\>[0]\AgdaIndent{4}{}\<[4]%
\>[4]\AgdaSymbol{;} \AgdaField{substT} \AgdaSymbol{=} \AgdaSymbol{λ} \AgdaSymbol{\{((}\AgdaBound{x} \AgdaInductiveConstructor{,} \AgdaBound{a}\AgdaSymbol{)} \AgdaInductiveConstructor{,} \AgdaBound{b}\AgdaSymbol{)} \AgdaBound{x0} \AgdaSymbol{→} \<[37]%
\>[37]\<%
\\
\>[0]\AgdaIndent{15}{}\<[15]%
\>[15]\AgdaFunction{[} \AgdaFunction{[} \AgdaBound{A} \AgdaFunction{]fm} \AgdaSymbol{\_} \AgdaFunction{]trans} \<[34]%
\>[34]\<%
\\
\>[0]\AgdaIndent{15}{}\<[15]%
\>[15]\AgdaSymbol{(}\AgdaFunction{[} \AgdaFunction{[} \AgdaBound{A} \AgdaFunction{]fm} \AgdaSymbol{\_} \AgdaFunction{]sym} \AgdaBound{a}\AgdaSymbol{)} \<[36]%
\>[36]\<%
\\
\>[0]\AgdaIndent{15}{}\<[15]%
\>[15]\AgdaSymbol{(}\AgdaFunction{[} \AgdaFunction{[} \AgdaBound{A} \AgdaFunction{]fm} \AgdaSymbol{\_} \AgdaFunction{]trans} \<[35]%
\>[35]\<%
\\
\>[0]\AgdaIndent{15}{}\<[15]%
\>[15]\AgdaSymbol{(}\AgdaFunction{[} \AgdaBound{A} \AgdaFunction{]subst*} \AgdaSymbol{\_} \AgdaBound{x0}\AgdaSymbol{)} \AgdaBound{b}\AgdaSymbol{)} \<[37]%
\>[37]\<%
\\
\>[0]\AgdaIndent{15}{}\<[15]%
\>[15]\AgdaSymbol{\}}\<%
\\
\>[0]\AgdaIndent{4}{}\<[4]%
\>[4]\AgdaSymbol{;} \AgdaField{subst*} \AgdaSymbol{=} \AgdaSymbol{λ} \AgdaBound{\_} \AgdaBound{\_} \AgdaSymbol{→} \AgdaInductiveConstructor{tt}\<%
\\
\>[0]\AgdaIndent{4}{}\<[4]%
\>[4]\AgdaSymbol{;} \AgdaField{refl*} \AgdaSymbol{=} \AgdaSymbol{λ} \AgdaBound{\_} \AgdaBound{\_} \AgdaSymbol{→} \AgdaInductiveConstructor{tt}\<%
\\
\>[0]\AgdaIndent{4}{}\<[4]%
\>[4]\AgdaSymbol{;} \AgdaField{trans*} \AgdaSymbol{=} \AgdaSymbol{λ} \AgdaBound{\_} \AgdaSymbol{→} \AgdaInductiveConstructor{tt} \<[24]%
\>[24]\<%
\\
\>[0]\AgdaIndent{4}{}\<[4]%
\>[4]\AgdaSymbol{\}}\<%
\\
%
\\
%
\\
\>[0]\AgdaIndent{2}{}\<[2]%
\>[2]\AgdaFunction{⟦refl⟧⁰} \AgdaSymbol{:} \AgdaRecord{Tm} \AgdaSymbol{\{}\AgdaBound{Γ} \AgdaFunction{\&} \AgdaBound{A}\AgdaSymbol{\}} \AgdaSymbol{(}\AgdaFunction{⟦Id⟧} \AgdaFunction{[} \AgdaKeyword{record} \AgdaSymbol{\{} \AgdaField{fn} \AgdaSymbol{=} \AgdaSymbol{λ} \AgdaBound{x'} \AgdaSymbol{→} \AgdaBound{x'} \AgdaInductiveConstructor{,} \AgdaFunction{proj₂} \AgdaBound{x'} \<[66]%
\>[66]\<%
\\
\>[0]\AgdaIndent{23}{}\<[23]%
\>[23]\AgdaSymbol{;} \AgdaField{resp} \AgdaSymbol{=} \AgdaSymbol{λ} \AgdaBound{x'} \AgdaSymbol{→} \AgdaBound{x'} \AgdaInductiveConstructor{,} \AgdaFunction{proj₂} \AgdaBound{x'} \AgdaSymbol{\}} \AgdaFunction{]T}\AgdaSymbol{)} \<[59]%
\>[59]\<%
\\
\>[0]\AgdaIndent{2}{}\<[2]%
\>[2]\AgdaFunction{⟦refl⟧⁰} \AgdaSymbol{=} \AgdaKeyword{record}\<%
\\
\>[0]\AgdaIndent{11}{}\<[11]%
\>[11]\AgdaSymbol{\{} \AgdaField{tm} \AgdaSymbol{=} \AgdaSymbol{λ} \AgdaSymbol{\{(}\AgdaBound{x} \AgdaInductiveConstructor{,} \AgdaBound{a}\AgdaSymbol{)} \AgdaSymbol{→} \AgdaFunction{[} \AgdaFunction{[} \AgdaBound{A} \AgdaFunction{]fm} \AgdaBound{x} \AgdaFunction{]refl} \AgdaSymbol{\{}\AgdaBound{a}\AgdaSymbol{\}} \AgdaSymbol{\}}\<%
\\
\>[0]\AgdaIndent{11}{}\<[11]%
\>[11]\AgdaSymbol{;} \AgdaField{respt} \AgdaSymbol{=} \AgdaSymbol{λ} \AgdaBound{p} \AgdaSymbol{→} \AgdaInductiveConstructor{tt}\<%
\\
\>[0]\AgdaIndent{11}{}\<[11]%
\>[11]\AgdaSymbol{\}}\<%
\\
%
\\
\>[0]\AgdaIndent{2}{}\<[2]%
\>[2]\AgdaFunction{⟦refl⟧} \AgdaSymbol{=} \<[12]%
\>[12]\AgdaFunction{lam} \AgdaSymbol{\{}\AgdaBound{Γ}\AgdaSymbol{\}} \AgdaSymbol{\{}\AgdaBound{A}\AgdaSymbol{\}} \AgdaFunction{⟦refl⟧⁰}\<%
\\
%
\\
\>\<\end{code}

Subst using equality types

\begin{code}\>\<%
\\
%
\\
\>[0]\AgdaIndent{2}{}\<[2]%
\>[2]\AgdaKeyword{module} \AgdaModule{substIn} \AgdaSymbol{(}\AgdaBound{B} \AgdaSymbol{:} \AgdaRecord{Ty} \AgdaSymbol{(}\AgdaBound{Γ} \AgdaFunction{\&} \AgdaBound{A}\AgdaSymbol{))} \AgdaKeyword{where}\<%
\\
\>[0]\AgdaIndent{2}{}\<[2]%
\>[2]\<%
\\
\>[2]\AgdaIndent{4}{}\<[4]%
\>[4]\AgdaFunction{⟦subst⟧⁰} \AgdaSymbol{:} \AgdaRecord{Tm} \AgdaSymbol{\{}\AgdaBound{Γ} \AgdaFunction{\&} \AgdaBound{A} \AgdaFunction{\&} \AgdaSymbol{(}\AgdaBound{A} \AgdaFunction{[} \AgdaFunction{fst\&} \AgdaSymbol{\{}A \AgdaSymbol{=} \AgdaBound{A}\AgdaSymbol{\}} \AgdaFunction{]T}\AgdaSymbol{)} \<[49]%
\>[49]\<%
\\
\>[4]\AgdaIndent{15}{}\<[15]%
\>[15]\AgdaFunction{\&} \AgdaFunction{⟦Id⟧} \AgdaFunction{\&} \AgdaBound{B} \AgdaFunction{[} \AgdaFunction{fst\&} \AgdaSymbol{\{}A \AgdaSymbol{=} \AgdaBound{A} \AgdaFunction{[} \AgdaFunction{fst\&} \AgdaSymbol{\{}A \AgdaSymbol{=} \AgdaBound{A}\AgdaSymbol{\}} \AgdaFunction{]T}\AgdaSymbol{\}} \<[60]%
\>[60]\AgdaFunction{]T} \<[63]%
\>[63]\<%
\\
\>[4]\AgdaIndent{15}{}\<[15]%
\>[15]\AgdaFunction{[} \AgdaFunction{fst\&} \AgdaSymbol{\{}A \AgdaSymbol{=} \AgdaFunction{⟦Id⟧}\AgdaSymbol{\}} \AgdaFunction{]T}\AgdaSymbol{\}} \<[37]%
\>[37]\<%
\\
\>[-2]\AgdaIndent{13}{}\<[13]%
\>[13]\AgdaSymbol{(}\AgdaBound{B} \AgdaFunction{[} \AgdaKeyword{record} \AgdaSymbol{\{} \AgdaField{fn} \AgdaSymbol{=} \AgdaSymbol{λ} \AgdaBound{x} \AgdaSymbol{→} \AgdaSymbol{(}\AgdaFunction{proj₁} \AgdaSymbol{(}\AgdaFunction{proj₁} \AgdaSymbol{(}\AgdaFunction{proj₁} \AgdaSymbol{(}\AgdaFunction{proj₁} \AgdaBound{x}\AgdaSymbol{))))} \<[72]%
\>[72]\<%
\\
\>[0]\AgdaIndent{13}{}\<[13]%
\>[13]\AgdaInductiveConstructor{,} \AgdaSymbol{(}\AgdaFunction{proj₂} \AgdaSymbol{(}\AgdaFunction{proj₁} \AgdaSymbol{(}\AgdaFunction{proj₁} \AgdaBound{x}\AgdaSymbol{)))} \<[41]%
\>[41]\<%
\\
\>[0]\AgdaIndent{13}{}\<[13]%
\>[13]\AgdaSymbol{;} \AgdaField{resp} \AgdaSymbol{=} \AgdaSymbol{λ} \AgdaBound{x} \AgdaSymbol{→} \AgdaFunction{proj₁} \AgdaSymbol{(}\AgdaFunction{proj₁} \AgdaSymbol{(}\AgdaFunction{proj₁} \AgdaSymbol{(}\AgdaFunction{proj₁} \AgdaBound{x}\AgdaSymbol{)))} \<[60]%
\>[60]\<%
\\
\>[0]\AgdaIndent{13}{}\<[13]%
\>[13]\AgdaInductiveConstructor{,} \AgdaFunction{proj₂} \AgdaSymbol{(}\AgdaFunction{proj₁} \AgdaSymbol{(}\AgdaFunction{proj₁} \AgdaBound{x}\AgdaSymbol{))} \AgdaSymbol{\}} \AgdaFunction{]T}\AgdaSymbol{)}\<%
\\
%
\\
\>[0]\AgdaIndent{4}{}\<[4]%
\>[4]\AgdaFunction{⟦subst⟧⁰} \AgdaSymbol{=} \AgdaKeyword{record}\<%
\\
\>[0]\AgdaIndent{11}{}\<[11]%
\>[11]\AgdaSymbol{\{} \AgdaField{tm} \AgdaSymbol{=} \AgdaSymbol{λ} \AgdaSymbol{\{((((}\AgdaBound{x} \AgdaInductiveConstructor{,} \AgdaBound{a}\AgdaSymbol{)} \AgdaInductiveConstructor{,} \AgdaBound{b}\AgdaSymbol{)} \AgdaInductiveConstructor{,} \AgdaBound{p}\AgdaSymbol{)} \AgdaInductiveConstructor{,} \AgdaBound{PA}\AgdaSymbol{)} \AgdaSymbol{→} \AgdaFunction{[} \AgdaBound{B} \AgdaFunction{]subst} \<[61]%
\>[61]\<%
\\
\>[11]\AgdaIndent{18}{}\<[18]%
\>[18]\AgdaSymbol{(}\AgdaFunction{[} \AgdaBound{Γ} \AgdaFunction{]refl} \AgdaInductiveConstructor{,} \AgdaFunction{[} \AgdaFunction{[} \AgdaBound{A} \AgdaFunction{]fm} \AgdaSymbol{\_} \AgdaFunction{]trans} \<[50]%
\>[50]\<%
\\
\>[11]\AgdaIndent{18}{}\<[18]%
\>[18]\AgdaSymbol{(}\AgdaFunction{[} \AgdaBound{A} \AgdaFunction{]refl*} \AgdaSymbol{\_} \AgdaSymbol{\_)} \AgdaBound{p}\AgdaSymbol{)} \AgdaBound{PA} \AgdaSymbol{\}}\<%
\\
\>[0]\AgdaIndent{11}{}\<[11]%
\>[11]\AgdaSymbol{;} \AgdaField{respt} \AgdaSymbol{=} \AgdaSymbol{λ} \AgdaSymbol{\{((((}\AgdaBound{m} \AgdaInductiveConstructor{,} \AgdaBound{a}\AgdaSymbol{)} \AgdaInductiveConstructor{,} \AgdaBound{b}\AgdaSymbol{)} \AgdaInductiveConstructor{,} \AgdaBound{p}\AgdaSymbol{)} \AgdaInductiveConstructor{,} \AgdaBound{PA}\AgdaSymbol{)} \AgdaSymbol{→} \<[53]%
\>[53]\<%
\\
\>[0]\AgdaIndent{13}{}\<[13]%
\>[13]\AgdaFunction{[} \AgdaFunction{[} \AgdaBound{B} \AgdaFunction{]fm} \AgdaSymbol{\_} \AgdaFunction{]trans} \<[32]%
\>[32]\<%
\\
\>[0]\AgdaIndent{13}{}\<[13]%
\>[13]\AgdaSymbol{(}\AgdaFunction{[} \AgdaBound{B} \AgdaFunction{]trans*} \AgdaSymbol{\_)} \<[29]%
\>[29]\<%
\\
\>[13]\AgdaIndent{14}{}\<[14]%
\>[14]\AgdaSymbol{(}\AgdaFunction{[} \AgdaFunction{[} \AgdaBound{B} \AgdaFunction{]fm} \AgdaSymbol{\_} \AgdaFunction{]trans} \<[34]%
\>[34]\<%
\\
\>[0]\AgdaIndent{13}{}\<[13]%
\>[13]\AgdaFunction{[} \AgdaBound{B} \AgdaFunction{]subst-pi} \<[27]%
\>[27]\<%
\\
\>[0]\AgdaIndent{13}{}\<[13]%
\>[13]\AgdaSymbol{(}\AgdaFunction{[} \AgdaFunction{[} \AgdaBound{B} \AgdaFunction{]fm} \AgdaSymbol{\_} \AgdaFunction{]trans} \<[33]%
\>[33]\<%
\\
\>[0]\AgdaIndent{13}{}\<[13]%
\>[13]\AgdaSymbol{(}\AgdaFunction{[} \AgdaFunction{[} \AgdaBound{B} \AgdaFunction{]fm} \AgdaSymbol{\_} \AgdaFunction{]sym} \AgdaSymbol{(}\AgdaFunction{[} \AgdaBound{B} \AgdaFunction{]trans*} \AgdaSymbol{\_))}\<%
\\
\>[0]\AgdaIndent{13}{}\<[13]%
\>[13]\AgdaSymbol{(}\AgdaFunction{[} \AgdaBound{B} \AgdaFunction{]subst*} \AgdaSymbol{\_} \AgdaBound{PA}\AgdaSymbol{)} \AgdaSymbol{))} \AgdaSymbol{\}}\<%
\\
\>[0]\AgdaIndent{11}{}\<[11]%
\>[11]\AgdaSymbol{\}}\<%
\\
\>\<\end{code}


\AgdaHide{
\begin{code}\>\<%
\\
%
\\
\>\AgdaComment{--    ⟦subst⟧ = lam (lam (lam ⟦subst⟧⁰))}\<%
\\
\>[0]\AgdaIndent{4}{}\<[4]%
\>[4]\<%
\\
\>\<\end{code}
}


\AgdaHide{
\begin{code}\>\<%
\\
%
\\
\>\AgdaSymbol{\{-\#} \AgdaKeyword{OPTIONS} --type-in-type \AgdaSymbol{\#-\}}\<%
\\
%
\\
\>\AgdaKeyword{import} \AgdaModule{Level}\<%
\\
\>\AgdaKeyword{open} \AgdaKeyword{import} \AgdaModule{Relation.Binary.PropositionalEquality} \AgdaSymbol{as} \AgdaModule{PE} \AgdaKeyword{hiding} \AgdaSymbol{(}refl \AgdaSymbol{;} sym \AgdaSymbol{;} trans\AgdaSymbol{;} isEquivalence\AgdaSymbol{;} [\_]\AgdaSymbol{)}\<%
\\
%
\\
\>\AgdaKeyword{module} \AgdaModule{CwF-quotient} \AgdaSymbol{(}\AgdaBound{ext} \AgdaSymbol{:} \AgdaFunction{Extensionality} \AgdaPrimitive{Level.zero} \AgdaPrimitive{Level.zero}\AgdaSymbol{)} \AgdaKeyword{where}\<%
\\
%
\\
\>\AgdaKeyword{open} \AgdaKeyword{import} \AgdaModule{Data.Unit}\<%
\\
\>\AgdaKeyword{open} \AgdaKeyword{import} \AgdaModule{Function}\<%
\\
\>\AgdaKeyword{open} \AgdaKeyword{import} \AgdaModule{Data.Product}\<%
\\
%
\\
%
\\
\>\AgdaComment{-- importing other CWF files}\<%
\\
%
\\
\>\AgdaKeyword{import} \AgdaModule{CwF-setoid}\<%
\\
%
\\
\>\AgdaKeyword{open} \AgdaModule{CwF-setoid} \AgdaBound{ext}\<%
\\
%
\\
\>\AgdaKeyword{import} \AgdaModule{CategoryOfSetoid}\<%
\\
\>\AgdaKeyword{module} \AgdaModule{cos'} \AgdaSymbol{=} \AgdaModule{CategoryOfSetoid} \AgdaBound{ext}\<%
\\
\>\AgdaKeyword{open} \AgdaModule{cos'}\<%
\\
%
\\
\>\AgdaKeyword{import} \AgdaModule{hProp}\<%
\\
\>\AgdaKeyword{module} \AgdaModule{hp'} \AgdaSymbol{=} \AgdaModule{hProp} \AgdaBound{ext}\<%
\\
\>\AgdaKeyword{open} \AgdaModule{hp'}\<%
\\
%
\\
\>\AgdaKeyword{import} \AgdaModule{CwF-ctd}\<%
\\
\>\AgdaKeyword{module} \AgdaModule{cc} \AgdaSymbol{=} \AgdaModule{CwF-ctd} \AgdaBound{ext}\<%
\\
\>\AgdaKeyword{open} \AgdaModule{cc}\<%
\\
%
\\
%
\\
\>\<\end{code}
}


The equality type is an essential part of a type theory. We could define it by using the equivalence relation from the setoid representation of type A. The equivalence relation is trivial since it is proof-irrelevant.

\begin{code}\>\<%
\\
%
\\
\>\AgdaFunction{Rel} \AgdaSymbol{:} \AgdaSymbol{\{}\AgdaBound{Γ} \AgdaSymbol{:} \AgdaFunction{Con}\AgdaSymbol{\}} \AgdaSymbol{→} \AgdaRecord{Ty} \AgdaBound{Γ} \AgdaSymbol{→} \AgdaPrimitiveType{Set₁}\<%
\\
\>\AgdaFunction{Rel} \AgdaSymbol{\{}\AgdaBound{Γ}\AgdaSymbol{\}} \AgdaBound{A} \AgdaSymbol{=} \AgdaRecord{Ty} \AgdaSymbol{(}\AgdaBound{Γ} \AgdaFunction{\&} \AgdaBound{A} \AgdaFunction{\&} \AgdaBound{A} \AgdaFunction{+T} \AgdaBound{A}\AgdaSymbol{)}\<%
\\
%
\\
\>\AgdaFunction{⟦Id⟧} \AgdaSymbol{:} \AgdaSymbol{\{}\AgdaBound{Γ} \AgdaSymbol{:} \AgdaFunction{Con}\AgdaSymbol{\}(}\AgdaBound{A} \AgdaSymbol{:} \AgdaRecord{Ty} \AgdaBound{Γ}\AgdaSymbol{)} \AgdaSymbol{→} \AgdaFunction{Rel} \AgdaBound{A}\<%
\\
\>\AgdaFunction{⟦Id⟧} \AgdaBound{A}\<%
\\
\>[0]\AgdaIndent{3}{}\<[3]%
\>[3]\AgdaSymbol{=} \AgdaKeyword{record} \<[12]%
\>[12]\<%
\\
\>[3]\AgdaIndent{7}{}\<[7]%
\>[7]\AgdaSymbol{\{} \AgdaField{fm} \AgdaSymbol{=} \AgdaSymbol{λ} \AgdaSymbol{\{((}\AgdaBound{x} \AgdaInductiveConstructor{,} \AgdaBound{a}\AgdaSymbol{)} \AgdaInductiveConstructor{,} \AgdaBound{b}\AgdaSymbol{)} \AgdaSymbol{→} \<[34]%
\>[34]\AgdaKeyword{record}\<%
\\
\>[7]\AgdaIndent{9}{}\<[9]%
\>[9]\AgdaSymbol{\{} \AgdaField{Carrier} \AgdaSymbol{=} \AgdaFunction{[} \AgdaFunction{[} \AgdaBound{A} \AgdaFunction{]fm} \AgdaBound{x} \AgdaFunction{]} \AgdaBound{a} \AgdaFunction{≈} \AgdaBound{b}\<%
\\
\>[7]\AgdaIndent{9}{}\<[9]%
\>[9]\AgdaSymbol{;} \AgdaField{\_≈h\_} \AgdaSymbol{=} \AgdaSymbol{λ} \AgdaBound{x₁} \AgdaBound{x₂} \AgdaSymbol{→} \AgdaFunction{⊤'}\<%
\\
\>[7]\AgdaIndent{9}{}\<[9]%
\>[9]\AgdaSymbol{;} \AgdaField{isEquiv} \AgdaSymbol{=} \AgdaKeyword{record}\<%
\\
\>[9]\AgdaIndent{13}{}\<[13]%
\>[13]\AgdaSymbol{\{} \AgdaField{refl} \AgdaSymbol{=} \AgdaSymbol{λ} \AgdaSymbol{\{}\AgdaBound{x₁}\AgdaSymbol{\}} \AgdaSymbol{→} \AgdaInductiveConstructor{tt}\<%
\\
\>[9]\AgdaIndent{13}{}\<[13]%
\>[13]\AgdaSymbol{;} \AgdaField{sym} \AgdaSymbol{=} \AgdaSymbol{λ} \AgdaBound{x₂} \AgdaSymbol{→} \AgdaInductiveConstructor{tt}\<%
\\
\>[9]\AgdaIndent{13}{}\<[13]%
\>[13]\AgdaSymbol{;} \AgdaField{trans} \AgdaSymbol{=} \AgdaSymbol{λ} \AgdaBound{x₂} \AgdaBound{x₃} \AgdaSymbol{→} \AgdaInductiveConstructor{tt}\<%
\\
\>[9]\AgdaIndent{13}{}\<[13]%
\>[13]\AgdaSymbol{\}}\<%
\\
\>[0]\AgdaIndent{9}{}\<[9]%
\>[9]\AgdaSymbol{\}} \AgdaSymbol{\}}\<%
\\
\>[0]\AgdaIndent{7}{}\<[7]%
\>[7]\AgdaSymbol{;} \AgdaField{substT} \AgdaSymbol{=} \AgdaSymbol{λ} \AgdaSymbol{\{((}\AgdaBound{x} \AgdaInductiveConstructor{,} \AgdaBound{a}\AgdaSymbol{)} \AgdaInductiveConstructor{,} \AgdaBound{b}\AgdaSymbol{)} \AgdaBound{x0} \AgdaSymbol{→} \<[40]%
\>[40]\<%
\\
\>[7]\AgdaIndent{15}{}\<[15]%
\>[15]\AgdaFunction{[} \AgdaFunction{[} \AgdaBound{A} \AgdaFunction{]fm} \AgdaSymbol{\_} \AgdaFunction{]trans} \<[34]%
\>[34]\<%
\\
\>[7]\AgdaIndent{15}{}\<[15]%
\>[15]\AgdaSymbol{(}\AgdaFunction{[} \AgdaFunction{[} \AgdaBound{A} \AgdaFunction{]fm} \AgdaSymbol{\_} \AgdaFunction{]sym} \AgdaBound{a}\AgdaSymbol{)} \<[36]%
\>[36]\<%
\\
\>[7]\AgdaIndent{15}{}\<[15]%
\>[15]\AgdaSymbol{(}\AgdaFunction{[} \AgdaFunction{[} \AgdaBound{A} \AgdaFunction{]fm} \AgdaSymbol{\_} \AgdaFunction{]trans} \<[35]%
\>[35]\<%
\\
\>[7]\AgdaIndent{15}{}\<[15]%
\>[15]\AgdaSymbol{(}\AgdaFunction{[} \AgdaBound{A} \AgdaFunction{]subst*} \AgdaSymbol{\_} \AgdaBound{x0}\AgdaSymbol{)} \AgdaBound{b}\AgdaSymbol{)} \<[37]%
\>[37]\<%
\\
\>[7]\AgdaIndent{15}{}\<[15]%
\>[15]\AgdaSymbol{\}}\<%
\\
\>[0]\AgdaIndent{7}{}\<[7]%
\>[7]\AgdaSymbol{;} \AgdaField{subst*} \AgdaSymbol{=} \AgdaSymbol{λ} \AgdaBound{p} \AgdaBound{x₁} \AgdaSymbol{→} \AgdaInductiveConstructor{tt}\<%
\\
\>[0]\AgdaIndent{7}{}\<[7]%
\>[7]\AgdaSymbol{;} \AgdaField{refl*} \AgdaSymbol{=} \AgdaSymbol{λ} \AgdaBound{x} \AgdaBound{a} \AgdaSymbol{→} \AgdaInductiveConstructor{tt}\<%
\\
\>[0]\AgdaIndent{7}{}\<[7]%
\>[7]\AgdaSymbol{;} \AgdaField{trans*} \AgdaSymbol{=} \AgdaSymbol{λ} \AgdaBound{p} \AgdaBound{q} \AgdaBound{a} \AgdaSymbol{→} \AgdaInductiveConstructor{tt} \AgdaSymbol{\}}\<%
\\
%
\\
\>\<\end{code}

The unique inhabitant $refl$ is defined as

\begin{code}\>\<%
\\
%
\\
%
\\
\>\AgdaFunction{cm-refl} \AgdaSymbol{:} \AgdaSymbol{\{}\AgdaBound{Γ} \AgdaSymbol{:} \AgdaFunction{Con}\AgdaSymbol{\}(}\AgdaBound{A} \AgdaSymbol{:} \AgdaRecord{Ty} \AgdaBound{Γ}\AgdaSymbol{)} \AgdaSymbol{→} \AgdaBound{Γ} \AgdaFunction{\&} \AgdaBound{A} \AgdaRecord{⇉} \AgdaSymbol{(}\AgdaBound{Γ} \AgdaFunction{\&} \AgdaBound{A} \AgdaFunction{\&} \AgdaBound{A} \AgdaFunction{+T} \AgdaBound{A}\AgdaSymbol{)}\<%
\\
\>\AgdaFunction{cm-refl} \AgdaBound{A} \AgdaSymbol{=} \AgdaKeyword{record} \AgdaSymbol{\{} \AgdaField{fn} \AgdaSymbol{=} \AgdaSymbol{λ} \AgdaBound{x'} \AgdaSymbol{→} \AgdaBound{x'} \AgdaInductiveConstructor{,} \AgdaFunction{proj₂} \AgdaBound{x'} \<[47]%
\>[47]\<%
\\
\>[7]\AgdaIndent{19}{}\<[19]%
\>[19]\AgdaSymbol{;} \AgdaField{resp} \AgdaSymbol{=} \AgdaSymbol{λ} \AgdaBound{x'} \AgdaSymbol{→} \AgdaBound{x'} \AgdaInductiveConstructor{,} \AgdaFunction{proj₂} \AgdaBound{x'} \AgdaSymbol{\}}\<%
\\
%
\\
\>\AgdaFunction{⟦refl⟧⁰} \AgdaSymbol{:} \AgdaSymbol{\{}\AgdaBound{Γ} \AgdaSymbol{:} \AgdaFunction{Con}\AgdaSymbol{\}(}\AgdaBound{A} \AgdaSymbol{:} \AgdaRecord{Ty} \AgdaBound{Γ}\AgdaSymbol{)} \<[30]%
\>[30]\<%
\\
\>[0]\AgdaIndent{7}{}\<[7]%
\>[7]\AgdaSymbol{→} \AgdaRecord{Tm} \AgdaSymbol{\{}\AgdaBound{Γ} \AgdaFunction{\&} \AgdaBound{A}\AgdaSymbol{\}} \AgdaSymbol{(}\AgdaFunction{⟦Id⟧} \AgdaBound{A}\<%
\\
\>[0]\AgdaIndent{10}{}\<[10]%
\>[10]\AgdaFunction{[} \AgdaFunction{cm-refl} \AgdaBound{A} \AgdaFunction{]T}\AgdaSymbol{)} \<[26]%
\>[26]\<%
\\
\>\AgdaFunction{⟦refl⟧⁰} \AgdaBound{A} \AgdaSymbol{=} \AgdaKeyword{record}\<%
\\
\>[10]\AgdaIndent{11}{}\<[11]%
\>[11]\AgdaSymbol{\{} \AgdaField{tm} \AgdaSymbol{=} \AgdaSymbol{λ} \AgdaSymbol{\{(}\AgdaBound{x} \AgdaInductiveConstructor{,} \AgdaBound{a}\AgdaSymbol{)} \AgdaSymbol{→} \AgdaFunction{[} \AgdaFunction{[} \AgdaBound{A} \AgdaFunction{]fm} \AgdaBound{x} \AgdaFunction{]refl} \AgdaSymbol{\{}\AgdaBound{a}\AgdaSymbol{\}} \AgdaSymbol{\}}\<%
\\
\>[10]\AgdaIndent{11}{}\<[11]%
\>[11]\AgdaSymbol{;} \AgdaField{respt} \AgdaSymbol{=} \AgdaSymbol{λ} \AgdaBound{p} \AgdaSymbol{→} \AgdaInductiveConstructor{tt}\<%
\\
\>[10]\AgdaIndent{11}{}\<[11]%
\>[11]\AgdaSymbol{\}}\<%
\\
%
\\
\>\AgdaFunction{⟦refl⟧} \AgdaSymbol{:} \AgdaSymbol{\{}\AgdaBound{Γ} \AgdaSymbol{:} \AgdaFunction{Con}\AgdaSymbol{\}(}\AgdaBound{A} \AgdaSymbol{:} \AgdaRecord{Ty} \AgdaBound{Γ}\AgdaSymbol{)} \<[29]%
\>[29]\<%
\\
\>[-6]\AgdaIndent{7}{}\<[7]%
\>[7]\AgdaSymbol{→} \AgdaRecord{Tm} \AgdaSymbol{\{}\AgdaBound{Γ}\AgdaSymbol{\}} \AgdaSymbol{(}\AgdaFunction{Π} \AgdaBound{A} \AgdaSymbol{(}\AgdaFunction{⟦Id⟧} \AgdaBound{A} \<[29]%
\>[29]\<%
\\
\>[0]\AgdaIndent{10}{}\<[10]%
\>[10]\AgdaFunction{[} \AgdaFunction{cm-refl} \AgdaBound{A} \AgdaFunction{]T}\AgdaSymbol{)} \AgdaSymbol{)}\<%
\\
\>\AgdaFunction{⟦refl⟧} \AgdaSymbol{\{}\AgdaBound{Γ}\AgdaSymbol{\}} \AgdaBound{A} \AgdaSymbol{=} \<[16]%
\>[16]\AgdaFunction{lam} \AgdaSymbol{\{}\AgdaBound{Γ}\AgdaSymbol{\}} \AgdaSymbol{\{}\AgdaBound{A}\AgdaSymbol{\}} \AgdaSymbol{(}\AgdaFunction{⟦refl⟧⁰} \AgdaBound{A}\AgdaSymbol{)}\<%
\\
%
\\
%
\\
\>\<\end{code}

We have an abstracted $refl$ term as well. Using $\Pi$-types we could define the eliminator for $Id$, but it is more involved.

We have done the basics for category of families of setoids. There are more types can be interpreted in this model so that we could show that it is a valid model for Type Theory. We would like to interpret quotient types in this model by following Hofmann's method in \cite{hof:95:sm} or by ourselves.


\AgdaHide{
\begin{code}\>\<%
\\
\>\AgdaComment{\{-

-- substIn (B : Ty (Γ \& A))

⟦subst⟧⁰ : \{Γ : Con\}(A : Ty Γ)(B : Ty (Γ \& A)) → 
           Tm \{Γ \& A \& (A [ fst\& A ]T) 
           \& (⟦Id⟧ A) \& B [ fst\& (A [ fst\& A ]T) ]T [ fst\& (⟦Id⟧ A) ]T\} 
         (B [ record \{ fn = λ x → (proj₁ (proj₁ (proj₁ (proj₁ x)))) , (proj₂ (proj₁ (proj₁ x))) ; resp = λ x → proj₁ (proj₁ (proj₁ (proj₁ x))) , proj₂ (proj₁ (proj₁ x)) \} ]T)

⟦subst⟧⁰ \{Γ\} A B = record
       \{ tm = λ \{((((x , a) , b) , p) , PA) → [ B ]subst ([ Γ ]refl , [ [ A ]fm \_ ]trans ([ A ]refl* \_ \_) p) PA \}
       ; respt = λ \{((((m , a) , b) , p) , PA) → 
         [ [ B ]fm \_ ]trans 
         ([ B ]trans* \_ \_ \_) 
          ([ [ B ]fm \_ ]trans 
         [ B ]subst-pi 
         ([ [ B ]fm \_ ]trans 
         ([ [ B ]fm \_ ]sym ([ B ]trans* \_ \_ \_))
         ([ B ]subst* \_ PA) )) \}
       \}


-\}}\<%
\\
%
\\
\>\AgdaComment{-- The mechanism used in Martin Hofmann's Paper}\<%
\\
%
\\
\>\AgdaKeyword{record} \AgdaRecord{Prop-Uni} \AgdaSymbol{(}\AgdaBound{Γ} \AgdaSymbol{:} \AgdaFunction{Con}\AgdaSymbol{)} \AgdaSymbol{:} \AgdaPrimitiveType{Set} \AgdaKeyword{where}\<%
\\
\>[0]\AgdaIndent{2}{}\<[2]%
\>[2]\AgdaKeyword{field}\<%
\\
%
\\
\>[0]\AgdaIndent{4}{}\<[4]%
\>[4]\AgdaField{prf} \AgdaSymbol{:} \AgdaRecord{Ty} \AgdaBound{Γ}\<%
\\
\>[0]\AgdaIndent{4}{}\<[4]%
\>[4]\AgdaField{uni} \AgdaSymbol{:} \AgdaSymbol{∀} \AgdaBound{γ} \AgdaBound{x} \AgdaBound{y} \AgdaSymbol{→} \AgdaFunction{[} \AgdaFunction{[} \AgdaBound{prf} \AgdaFunction{]fm} \AgdaBound{γ} \AgdaFunction{]} \AgdaBound{x} \AgdaFunction{≈h} \AgdaBound{y} \AgdaDatatype{≡} \AgdaFunction{⊤'}\<%
\\
\>\AgdaKeyword{open} \AgdaModule{Prop-Uni}\<%
\\
%
\\
\>\AgdaComment{-- Is it correct to write  Tm A → Tm B for dependent types?}\<%
\\
%
\\
%
\\
%
\\
\>\AgdaFunction{Id-is-prop} \AgdaSymbol{:} \AgdaSymbol{\{}\AgdaBound{Γ} \AgdaSymbol{:} \AgdaFunction{Con}\AgdaSymbol{\}(}\AgdaBound{A} \AgdaSymbol{:} \AgdaRecord{Ty} \AgdaBound{Γ}\AgdaSymbol{)} \AgdaSymbol{→} \AgdaRecord{Prop-Uni} \AgdaSymbol{(}\AgdaBound{Γ} \AgdaFunction{\&} \AgdaBound{A} \AgdaFunction{\&} \AgdaSymbol{(}\AgdaBound{A} \AgdaFunction{[} \AgdaFunction{fst\&} \AgdaBound{A} \AgdaFunction{]T}\AgdaSymbol{))}\<%
\\
\>\AgdaFunction{Id-is-prop} \AgdaBound{A} \AgdaSymbol{=} \AgdaKeyword{record} \AgdaSymbol{\{} \AgdaField{prf} \AgdaSymbol{=} \AgdaFunction{⟦Id⟧} \AgdaBound{A} \AgdaSymbol{;} \AgdaField{uni} \AgdaSymbol{=} \AgdaSymbol{λ} \AgdaBound{γ} \AgdaBound{x} \AgdaBound{y} \AgdaSymbol{→} \AgdaInductiveConstructor{PE.refl} \AgdaSymbol{\}}\<%
\\
%
\\
\>\AgdaComment{\{-
record Quo \{Γ : Con\}(A : Ty Γ)(R : Prop-Uni (Γ \& A \& (A [ fst\& \{Γ\} \{A\} ]T))) : Set where
  field
    Q : Ty Γ
    [\_] : Tm A → Tm Q
    Q-elim : ∀ (B : Ty Γ)
                 (M : Tm \{Γ \& A\} (B [ fst\& \{Γ\} \{A\} ]T))
                 (N : Tm Q)
                 (H : Tm \{Γ \& A \& A [ fst\& \{Γ\} \{A\} ]T \& prf R\} (prf (Id-is-prop B) [ fst\& \{Γ \& A \& A [ fst\& \{Γ\} \{A\} ]T\} \{prf R\} ]T)) -- (prf (Id-is-prop (B [ fst\& \{Γ\} \{A\} ]T)))
               → Tm B

-\}}\<%
\\
%
\\
%
\\
%
\\
\>\<\end{code}
}


\subsection{Observational equality}

%definitional distinct types

Later in in \cite{alti:ott-conf}, Altenkirch and McBride further
simplifies the setoid model by adopting McBride's heterogeneous
approach to equality. They identifies values up to observation rather than
  construction which is called observational equality. It is the
  propositional equality induced by the Setoid model.  In general we have a heterogeneous equality which
  compares terms of types which are different in construction. It only
  make sense when we can prove the types are the same. It helps us
  avoids the heavy use of $subst$ which makes formalisation and
  reasoning involved. We could simplify the setoid model by adapting this
  approach and the implementation could be easier.



\section{What we could do in this model}

\section{Quotient types in setoid model}

\todo{With Prop universe, how can we define quotient types?}